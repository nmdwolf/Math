\chapter{Bessel functions}
\section{Bessel's differential equation (BDE)}

	A Bessel's differential equation is an ordinary differential equation of the following form:
    \begin{equation}\index{Bessel!ordinary differential equation}
        \label{bessel:differential_equation}
		\boxed{z^2y'' + zy' + (z^2 - n^2)y = 0}
    \end{equation}
    
    \noindent The solutions of this ODE are the Bessel functions of the first and second kind (also called respectivelly Bessel and Neumann functions).
    
        
    \begin{equation}\index{Bessel!function}
		\label{bessel:bessel_function}
        J_n(z) = \sum_{m = 0}^\infty\frac{(-1)^m}{m!(m + n)!}\left(\frac{z}{2}\right)^{2m + n}
	\end{equation}
    
    \begin{equation}\index{Neumann function}
		\label{bessel:neumann_function}
        N_n(z) = \lim_{\nu\rightarrow n}\frac{cos(\nu \pi)J_N(z) - J_{-n}(z)}{sin(\nu\pi)}
	\end{equation}
    
    \sremark{Solution \ref{bessel:bessel_function} can be found by using Frobenius' method.}
    \begin{property}
		For $n\not\in\mathbb{N}$ the solutions $J_n(z)$ and $J_{-n}(z)$ are independent.
	\end{property}
    \remark{For $n\not\in\mathbb{N}$ the limit operation in function \ref{bessel:neumann_function} is not necessary as $\sin(n\pi)$ will never become 0 in this case.}
    
\section{Generating function}
	Define the following function:
    \begin{equation}\index{Bessel!generating function}
		\label{bessel:generating_function}
        g(x, t) = exp\left[\stylefrac{x}{2}\left(t - \stylefrac{1}{t}\right)\right]
	\end{equation}
    
    \noindent If we expand this function as a Laurent series, we obtain the following formula:
    \begin{equation}
		\label{bessel:generating_function_expansion}
        g(x, t) = \sum_{n=-\infty}^{+\infty}J_n(x)t^n
	\end{equation}
    
    \noindent By applying the residue theorem \ref{complexcalculus:residue_theorem}, we can express the functions $J_n(x)$ as follows:
    \begin{equation}
		\label{bessel:generating_function_integral}
        J_n(x) = \stylefrac{1}{2\pi i}\oint_C\stylefrac{g(x, t)}{t^{n+1}}dt
	\end{equation}
    The function $g(x, t)$ is called the generating function of the Bessel functions.
    
\section{Applications}
\subsection{Laplace equation}
	When solving the Laplace equation in cylindrical coordinates we obtain a BDE with integer $n$, which has the \textbf{cylindrical Bessel functions} \ref{bessel:bessel_function} and \ref{bessel:neumann_function} as solutions.\index{Laplace!equation}
    
\subsection{Helmholtz equation}
	When solving the Helmholtz equation in spherical coordinates we obtain a variant of the BDE for the radial part:\index{Helmholtz!equation}
    \begin{equation}
		z^2y'' + 2zy' + [z^2 - n(n+1)]y = 0
	\end{equation}
    where $n$ is an integer. The solutions, called \textbf{spherical Bessel functions}, are related to the cylindrical Bessel functions in the following way:
    \begin{equation}
		j_n(r) = \sqrt{\stylefrac{\pi}{2x}}J_{n + \frac{1}{2}}(r)
	\end{equation}
    and similarly for the Neumann functions.
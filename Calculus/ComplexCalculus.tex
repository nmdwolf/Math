\chapter{Complex Analysis}

\section{Complex algebra}

    The set of complex numbers $\mathbb{C}$ forms a 2-dimensional vector space over the field of real numbers. Furthermore, the operations of complex addition and complex multiplication also turn the complex numbers into a field.

    \newdef{Complex conjugate}{\index{complex conjugate}
        Complex conjugation
        \begin{gather}
            \overline{z}:a+bi\mapsto a-bi
        \end{gather}
        is an involution, i.e. $\overset{=}{z} = z$. It is sometimes denoted by $z^*$ instead of $\overline{z}$, but we will adopt the former notation (unless this would cause confusion).
    }

    \newformula{Real/imaginary part}{
        \nomenclature[O_Re]{$\text{Re}$}{Real part of a complex number.}
        \nomenclature[O_Im]{$\text{Im}$}{Imaginary part of a complex number.}
        A complex number $z$ can also be written as $\text{Re}(z) + i\ \text{Im}(z)$ where
        \begin{align}
            \text{Re}(z) &:= \stylefrac{z + \overline{z}}{2}\\
            \text{Im}(z) &:= \stylefrac{z - \overline{z}}{2i}.
        \end{align}
    }
    \newdef{Argument}{\index{argument}\index{polar!form}
        \nomenclature[O_arg]{$\text{arg}(z)$}{Argument of the complex number $z$.}
        Let $z$ be a complex number expressed in \textit{polar form} as follows: $z = re^{i\theta}$. The number $\theta$ is called the argument of $z$ and it is denoted by $\arg(z)$.
    }

    \newdef{Riemann sphere}{\index{Riemann!sphere}
        Consider the one-point compactification\footnote{See definition \ref{topology:alexandrov_compactification}.} $\overline{\mathbb{C}} = \mathbb{C}\cup\{\infty\}$. This set is called the Riemann sphere or \textbf{extended complex plane}. The standard operations on $\mathbb{C}$ can be generalized to $\overline{\mathbb{C}}$ for all nonzero $z \neq \infty$ in the following way:
        \begin{align}
            z + \infty &= \infty\nonumber\\
            z * \infty &= \infty\\
            \frac{z}{\infty} &= 0.\nonumber
        \end{align}
        Since there exists no multiplicative inverse for $\infty$ the Riemann sphere does not form a field.
    }

\section{Complex maps}
\subsection{Holomorphic maps}

    \newdef{Holomorphic function}{\index{holomorphic}
        A function $f$ is said to be holomorphic on an open set $U$ if it is complex differentiable at every point $z_0\in U$, i.e. for every point $z_0\in U$ the following limit exists:
        \begin{gather}
            f'(z_0) = \lim_{z\rightarrow z_0}\frac{f(z) - f(z_0)}{z-z_0}.
        \end{gather}
    }
    \newdef{Biholomorphic}{
        A complex function $f$ is said to be biholomorphic if both $f$ and $f^{-1}$ are holomorphic.
    }
    \newdef{Entire}{\index{entire}
        A function holomorphic at every point $z\in\mathbb{C}$.
    }

    \begin{property}[Cauchy-Riemann conditions]\index{Cauchy-Riemann conditions}\label{complexcalculus:cauchy_riemann}\index{Wirtinger}
        \nomenclature[A_CR]{CR}{Cauchy-Riemann}
        A holomorphic function $f$ satisfies the following conditions:
        \begin{gather}
            \pderiv{u}{x} = \pderiv{v}{y} \text{\qquad and\qquad} \pderiv{u}{y} = -\pderiv{v}{x}.
        \end{gather}
        These can be combined into one equation using the so-called \textbf{Wirtinger derivative}:
        \begin{gather}
            \label{complexcalculus:holomorphic_alternative_condition}
            \pderiv{f}{\overline{z}} = 0.
        \end{gather}
    \end{property}

    \begin{theorem}[Looman-Menchoff\footnotemark]\index{Looman-Menchoff}
        \footnotetext{This is the strongest (most general) theorem on the holomorphy of continuous functions. It generalizes the original results by Riemann and Cauchy-Goursat.}
        Let $f$ be a continuous complex-valued function defined on a subset $U\in\mathbb{C}$. If the partial derivatives of the real and imaginary part exist and if $f$ satisfies the Cauchy-Riemann conditions then $f$ is holomorphic on $U$.
    \end{theorem}

    \begin{property}
        The functions $u,v$ satisfying the CR-conditions are harmonic functions, i.e. they satisfy Laplace's equation.
    \end{property}
    \begin{property}
        The functions $u,v$ satisfying the CR-conditions have orthogonal level curves \ref{set:level_set}.
    \end{property}

    \begin{property}
        Consider a real-valued function $f$ defined on the complex plane. If it is holomorphic then the CR-conditions imply that $f$ is a constant.
    \end{property}

    \begin{theorem}[Identity theorem]\index{identity!theorem}
        If two holomorphic functions on a domain $D$ coincide on a set containing an accumulation point of $D$ then they coincide on all of $D$.
    \end{theorem}

\section{Contour integrals}

    In this and further sections, all contours have been chosen to be evaluated counter-clockwise (by convention). To obtain results concerning clockwise evaluation, most of the time adding a minus sign is sufficient.

    \newdef{Contour}{\index{contour}
        A complex-valued curve $z(t)$ that can be parametrized by two real-valued functions:
        \begin{gather}
            \begin{cases}
                x = x(t)\\
                y = y(t)
            \end{cases}
            \longrightarrow z(t) = x(t) + iy(t).
        \end{gather}
    }
    \newdef{Contour integral}{\index{integral!contour}\label{complexcalculus:contour_integral}
        The contour integral of a function $f(z) = u(z) + iv(z)$ is defined as the following line integral:
        \begin{gather}
            \int_{z_1}^{z_2}f(z)dz = \int_{(x_1,y_1)}^{(x_2,y_2)}\big[u(x,y) + iv(x,y)\big](dx + idy).
        \end{gather}
    }

    \begin{theorem}[Cauchy's Integral Theorem\footnotemark]\index{Cauchy!integral theorem}\index{rectifiable curve}\label{complexcalculus:cauchy_integral_theorem}
        \footnotetext{Also called the \textit{Cauchy-Goursat theorem}.}
        Let $\Omega$ be a simply-connected subset of $\mathbb{C}$ and let $f$ be a holomorphic function on $\Omega$. Then for every closed rectifiable\footnote{A contour with finite length.} contour $C$ in $\Omega$:
        \begin{gather}
            \oint_C f(z) dz = 0.
        \end{gather}
    \end{theorem}
    \begin{result}[Freedom of contour]
        The contour integral of a holomorphic function depends only on the limits of integration and not on the contour connecting them.
    \end{result}

    \begin{formula}[Cauchy's Integral Formula]\index{Cauchy!integral formula}\label{complexcalculus:cauchy_integral_formula}
        Let $\Omega$ be a connected subset of $\mathbb{C}$ and let $f$ be a holomorphic function on $\Omega$. Let $C$ be a contour in $\Omega$. For every point $z_0$ inside $C$ we can express the function $f$ as follows:
        \begin{gather}
            f(z_0) = \frac{1}{2\pi i}\oint_C \frac{f(z)}{z - z_0} dz.
        \end{gather}
    \end{formula}

    \begin{result}[Analytic function]\index{analytic}
        Let $\Omega$ be a connected subset of $\mathbb{C}$ and let $C$ be a closed contour in $\Omega$. If $f$ is holomorphic on $\Omega$ then $f$ is analytic \ref{calculus:analytic} on $\Omega$ and
        \begin{gather}
            \label{complexcalculus:cauchy_integral_formula_derivative}
            f^{(n)}(z_0) = \frac{1}{2\pi i}\oint_C f(z) \frac{n!}{(z - z_0)^{n+1}} dz.
        \end{gather}
        Furthermore, the derivatives are also holomorphic on $\Omega$.
    \end{result}

    \begin{theorem}[Morera]\index{Morera}
        If $f$ is continuous on a connected open set $\Omega$ and $\oint_C f(z) dz = 0$ for every closed contour $C$ in $\Omega$, then $f$ is holomorphic on $\Omega$.
    \end{theorem}

    \begin{theorem}[Liouville]\index{Liouville!theorem on entire functions}
        Every bounded entire function is constant.
    \end{theorem}

    \begin{theorem}[Sokhotski-Plemelj\footnotemark]\index{Sokhotski-Plemelj}
        \footnotetext{See for example \cite{greiner_qm}, page 104.}
        Let $f$ be a continuous complex-valued function defined on the real line and let $a<0<b$n then
        \begin{gather}
            \lim_{\varepsilon\rightarrow0^+}\int_a^b\frac{f(x)}{x\pm i\varepsilon}dx = \mp i\pi f(0) + \mathcal{P}\int_a^b\frac{f(x)}{x}dx
        \end{gather}
        where $\mathcal{P}$ denotes the Cauchy principal value.
    \end{theorem}

\section{Laurent series}

    \begin{definition}[Laurent series]\index{Laurent!series}\index{annulus}\label{complexcalculus:laurent_series}
        If $f$ is a function, analytic on an \textit{annulus}\footnote{A ring-shaped region.} A, then $f$ can be expanded as the following series:
        \begin{gather}
            f(z) = \sum^{\infty}_{n=-\infty} a_n (z - z_0)^n \qquad \text{with} \qquad a_n = \frac{1}{2\pi i} \oint \frac{f(z')}{(z' - z_0)^{n+1}} dz'.
        \end{gather}
    \end{definition}

    \begin{property}[Convergence of Lauren series]
        The Laurent series of an analytic function $f$ converges uniformly to $f$ in the annulus $R_1 < |z - z_0| < R_2$, with $R_1$ and $R_2$ the distances from $z_0$ to the two closest poles.
    \end{property}

    \newdef{Principal part}{\index{principal!part}
        The principal part of a Laurent series at the point $z_0$ is defined as the sum
        \begin{gather}
            \sum_{n=-\infty}^{-1}a_n(z-z_0)^n.
        \end{gather}
    }

\subsection{Analytic continuation}

    \newdef{Analytic continuation}{\index{analytic continuation}
        Consider a function $f$ analytic on an open subset $U\subset\mathbb{C}$. If $V\subset\mathbb{C}$ is an open subset containing $U$ and if there exists an analytic function $F$ on $V$ such that $F(z)=f(z)$ for all $z\in U$ then $F$ is called the analytic continuation of $f$ to $V$. Using the identity theorem for holomorphic functions one can prove that analytic continuations are unique (on connected domains).
    }

    \begin{theorem}[Schwarz' reflection principle]\index{Schwarz!reflection principle}
        Let $f$ be analytic on the upper half plane. If $f(z)$ is real when $z$ is real, then
        \begin{gather}
            f(\overline{z}) = \overline{f(z)}.
        \end{gather}
    \end{theorem}

\section{Singularities}
\subsection{Poles}

    \newdef{Pole}{\index{pole}
        A function $f$ has a pole of order $m>0$ at a point $z_0$ if its Laurent series at $z_0$ satisfies $\forall n<-m:a_n = 0$ and $a_{-m}\neq0$.
    }

    \newdef{Meromorphic}{\index{meromorphic}
        A function $f$ is called meromorphic if it is analytic on the whole complex plane with exception of isolated poles and removable singularities. Every meromorphic function can be written as a fraction of two holomorphic functions where the poles coincide with the zeros of the denominator.
    }

    \newdef{Essential singularity}{\index{essential!singularity}
        A function $f$ has an essential singularity at a point $z_0$ if its Laurent series at $z_0$ satisfies $\forall n\in\mathbb{N}:a_{-n}\neq0$, i.e. if its Laurent series has infinitely many negative degree terms.
    }

    \begin{theorem}[Picard's great theorem]\index{Picard!great theorem}
        Let $f$ be an analytic function with an essential singularity at $z_0$. On every punctured neighbourhood of $z_0$, $f(z)$ takes on all possible complex values, with at most a single exception, infinitely many times.
    \end{theorem}

    \newmethod{Frobenius transformation}{\index{Frobenius!transformation}
        To study the behaviour of a function $f(z)$ at $z\rightarrow\infty$, one can apply the Frobenius transformation $h = 1/z$ and study the limit $\lim_{h\rightarrow0}f(h)$.
    }

\subsection{Branch cuts}

    \newformula{Roots}{\index{root}
        Let $z\in\mathbb{C}$. The $n^{th}$ roots\footnote{Also see the fundamental theorem of algebra \ref{linalgebra:fundamental_theorem_of_algebra}.} of $z = re^{i\theta}$ are given by
        \begin{gather}
            z^{1/n} = \sqrt[n]{r}\exp\left(i\frac{\theta + 2\pi k}{n}\right)
        \end{gather}
        where $k\in\{0,1,\ldots,n\}$.
    }
    \newformula{Complex logarithm}{\index{logarithm}
        Let us parametrize $z$ as $z = re^{i\theta}$. The natural logarithm can be continued into the complex plane as follows:
        \begin{gather}
            \text{LN}(z) = \ln(r) + i(\theta + 2\pi k).
        \end{gather}
    }

    \newdef{Branch}{\index{branch}
        From these two formulas it is clear that the complex roots and logarithms are multi-valued functions. To get an unambiguous image it is necessary to fix a value of the parameter $k$. By doing so there will arise curves in the complex plane where the function is discontinuous. These are the branch cuts. A \textbf{branch} is then defined as a particular choice of the parameter $k$.

        For the logarithm the choice for $\arg(\text{LN})\in\ ]\alpha, \alpha + 2\pi]$ is often denoted by $\text{LN}_\alpha$ or $\log_\alpha$.
    }
    \newdef{Branch point}{
            Let $f$ be a complex valued function. A point $z_0$ such that there exists no neighbourhood $|z-z_0|<\varepsilon$ where $f$ is single-valued is called a branch point.
    }
    \newdef{Branch cut}{
            A line connecting exactly two branch points is called a branch cut. (One of the branch points can be at infinity.) In case of multiple branch cuts, they do not cross.
    }

    \begin{example}
        Consider the complex function \[f(z) = \stylefrac{1}{\sqrt{(z-z_1)\cdots(z-z_n)}}.\] This function has singularities at $z_1,\ldots,z_n$. If $n$ is even, this function will have $n$ (finite) branch points. This implies that the points can be grouped in pairs connected by non-intersecting branch cuts. If $n$ is odd, this function will have $n$ (finite) branch points and one branch point at infinity. The finite branch points will be grouped in pairs connected by non-intersecting branch cuts and the remaining branch point will be joined to infinity by a branch cut which does not intersect the others.
    \end{example}

    \newdef{Principal value}{\index{principal!value}
        The principal value of a multi-valued complex function is defined as the value associated with a choice of branch for which $\arg(f)\in]-\pi,\pi]$.
    }

\subsection{Residue theorem}

    \newdef{Residue}{\index{residue}\label{complexcalculus:residue_def}
        \nomenclature[O_Res]{$\text{Res}$}{Residue of a complex function.}
        By applying formula \ref{complexcalculus:contour_integral} to a polynomial function we find
        \begin{gather}
            \int_C(z-z_0)^ndz = 2\pi i\delta_{n,-1}
        \end{gather}
        where $C$ is a (circular) contour around the pole $z = z_0$. This means that integrating a Laurent series around a pole isolates the coefficient $a_{-1}$. This coefficient is therefore called the residue of the function at the given pole.
    }
    \begin{notation}
        The residue of a complex function $f(z)$ at a pole $z_0$ is denoted by \[\text{Res}[f(z)]_{z=z_0}.\]
    \end{notation}

    \begin{formula}
        For a pole of order $m$, the residue is calculated as follows:
        \begin{gather}
            \label{complexcalculus:residue}
            \operatorname{Res}\left[f(z)\right]_{z=z_j} = a_{-1} = \lim_{z\rightarrow z_0} \stylefrac{1}{(m - 1)!} \left(\pderiv{}{z}\right)^{m-1}\left(f(z)(z-z_0)\right).
        \end{gather}
        For essential singularities the residue can be found by writing out the Laurent series explicitly.
    \end{formula}

    \begin{theorem}[Residue theorem]\index{residue!theorem}\label{complexcalculus:residue_theorem}
        If $f$ is a meromorphic function in $\Omega$ and if $C$ is a closed contour in $\Omega$ which contains the poles $z_j$ of $f$, then
        \begin{gather}
            \oint_Cf(z)dz = 2\pi i\sum_j \operatorname{Res}\left[f(z)\right]_{z=z_j}.
        \end{gather}
        For poles on the contour $C$, only half of the residue contributes to the integral.
    \end{theorem}

    \begin{formula}[Argument principle]\index{argument!principle}
        Let $f$ be a meromorphic function and let $Z_f, P_f$ be respectively the number of zeros and poles of $f$ inside the contour $C$. From the residue theorem we can derive the following formula:
        \begin{gather}
            \frac{1}{2\pi i}\oint_C\frac{f(z)}{f'(z)}dz = Z_f - P_f.
        \end{gather}
    \end{formula}
    \begin{definition}[Winding number]\index{winding number}\index{index!of map}
        \nomenclature[O_Ind]{$\text{Ind}_f(z)$}{Index of a point $z\in\mathbb{C}$ with respect to a function $f$.}
        Let $f$ be a meromorphic function and let $C$ be a simple closed contour. For all $a\not\in f(C)$ the winding number, also called the \textbf{index}, of $a$ with respect to the function $f$ is defined as follows:
        \begin{gather}
            \text{Ind}_f(a) := \frac{1}{2\pi i}\oint_C\frac{f'(z)}{f(z) - a}dz.
        \end{gather}
        This number is always an integer.
    \end{definition}

\section{Limit theorems}

    \begin{theorem}[Small limit theorem]\index{limit!theorem}\label{complexcalculus:theorem:small_limit}
        Let $f$ be a function that is holomorphic almost everywhere on $\mathbb{C}$ and let the contour $C$ be a circular segment with radius $\varepsilon$ and central angle $\alpha$. If $z$ is parametrized as $z = \varepsilon e^{i\theta}$ then\[\int_Cf(z)dz = i\alpha A\] with \[A = \lim_{\varepsilon\rightarrow0}f(z).\]
    \end{theorem}

    \begin{theorem}[Great limit theorem]\label{complexcalculus:theorem:great_limit}
        Let $f$ be a function that is holomorphic almost every where on $\mathbb{C}$ and let the contour $C$ be a circular segment with radius $R$ and central angle $\alpha$. If $z$ is parametrized as $z = Re^{i\theta}$ then\[\int_Cf(z)dz = i\alpha B\] with \[B = \lim_{R\rightarrow+\infty}f(z).\]
    \end{theorem}

    \begin{theorem}[Jordan's lemma]\index{Jordan}\label{complexcalculus:theorem:jordan}
        Let $g$ be a continuous function that can be decomposed as $g(z) = f(z)e^{bz}$ and let the contour $C$ be a semicircle lying in the half-plane bounded by the real axis and oriented away of the point $\overline{b}i$. If $z$ is parametrized as $z=Re^{i\theta}$ and \[\lim_{R\rightarrow\infty}f(z) = 0,\] then \[\int_Cg(z)dz = 0.\]
    \end{theorem}
\chapter{Distributions}\label{chapter:distributions}

	The main references for this chapter are \cite{AMP1, AMP2, georgiev}. Although this chapter is technically part of functional analysis and, hence, uses the language of normed spaces (Chapter \ref{chapter:normed_spaces}), it is presented in the part on calculus due to its strong relation to measure and integration theory.

\section{Functionals}

	\newdef{Distribution}{\index{distribution}\index{generalized!function}
		The space of distributions or \textbf{generalized functions} on an open set $U\subset\mathbb{R}^n$ is defined as the set of continuous linear functionals on $\mathcal{D}(U):=C^\infty_c(U)$, the space of smooth functions with compact support.

        First $\mathcal{D}(U)$ has to be endowed with a topology. For every compact set $K\subset U$ and every $m\in\mathbb{N}$ a locally convex topology \ref{normed:locally_convex_seminorm} on $\mathcal{D}^m_K(U):=C^m_K(U)$ is constructed using the following family of seminorms:
        \begin{gather}
            \mathcal{P}=\left\{\sup_{x\in K}\|f^{(i)}(x)\|\,\middle\vert\,|i|\leq m\right\}.
        \end{gather}
        A topology on all of $\mathcal{D}^m(U)$ is then defined as the inductive limit over all compact subsets $K\subset U$, i.e. a subset of $\mathcal{D}^m(U)$ is open if and only if its intersection with all $\mathcal{D}^m_K(U)$ is open. All of these topologies are Fr\'echet \ref{normed:frechet_space}. A topology on $\mathcal{D}(U)$ is obtained by taking a further inductive limit of the $\mathcal{D}^m(U)$ over $m\in\mathbb{N}$.

		The dual space $\mathcal{D}'(U)$ is equipped with the weak-* topology \ref{hilbert:weak_star_topology} and, accordingly, a sequence of distributions $\seq{\phi}$ converges to a distribution $\phi$ if and only if $\langle\phi_n,f\rangle\longrightarrow\langle\phi,f\rangle$ for all $f\in\mathcal{D}(U)$. This definition immediately implies that two distributions $\phi,\psi$ are equal if and only if $\langle\phi,f\rangle = \langle\psi,f\rangle$ for all $f\in\mathcal{D}(U)$.
	}

    \begin{property}[Equivalent seminorms]
        The seminorms used in the definition of the locally convex topology on $\mathcal{D}(U)$ can be replaced by the following equivalent ones:
        \begin{align}
            p_{K,m}(f) := &\sup_{|i|\leq m}\sup_{x\in K}\|f^{(i)}(x)\|\label{distribution:D_seminorm}\\
            &\sup_{x\in K}\sum_{|i|\leq m}\|f^{(i)}(x)\|\\
            &\sum_{|i|\leq m}\sup_{x\in K}\|f^{(i)}(x)\|.
        \end{align}
    \end{property}

	\begin{property}
		A linear functional $\phi$ on $\mathcal{D}(U)$ is a distribution if and only if it satisfies one of the following equivalent statements:
		\begin{itemize}
            \item It is continuous when restricted to every $\mathcal{D}_K(U)$ for $K\subset U$ compact.
			\item If the sequence $\seq{f}$ converges to 0 in $\mathcal{D}(U)$, then $\langle\phi,f_n\rangle\longrightarrow0$.
			\item For every compact subset $K\subset U$ there exist a constant $C_K>0$ and an integer $m_K\geq0$ such that
			\begin{gather}
				|\langle\phi,f\rangle|\leq C_K\,p_{K,m_K}(f)
			\end{gather}
			for all $f\in\mathcal{D}_K(U)$.
		\end{itemize}
	\end{property}

	\newdef{Order}{\index{order}
		The order of a distribution $\phi$ is the smallest integer $m$ such that
		\begin{gather}
			|\langle\phi,f\rangle|\leq C_K\,p_{K,m}(f)
		\end{gather}
		for all $f\in\mathcal{D}_K(U)$ and all compact subsets $K\subset U$. Note that the integer $m$ is independent of the compact set $K$.
	}

	\begin{property}
		A distribution is of order $k$ if and only if it can be (uniquely) extended to a continuous linear functional on $\mathcal{D}^k(U)$.
	\end{property}

    \begin{theorem}[Riesz-Markov-Kakutani]\index{Riesz-Markov-Kakutani}\label{distributions:riesz_markov}
        The space of positive continuous functionals on $C_c(X)$, the space of continuous functions with compact support on a locally compact Hausdorff space $X$, is homeomorphic to the space of Radon measures \ref{lebesgue:radon_measure} on $X$. Every functional $\Lambda$ can be represented as
        \begin{gather}
            \Lambda(f) = \int_Xf\,d\mu
        \end{gather}
        for some Radon measure $\mu$.

        The topological dual of $C(\widehat{X})$, the continuous functions on the one-point compactification \ref{topology:alexandrov_compactification} (i.e. those functions that vanish at infinity), is isometrically isomorphic to the space of finite signed Radon measures (equipped with the total variation norm).
    \end{theorem}

    \begin{example}[Ordinary function as generalized function]\label{distributions:ordinary_function}
       	By Property \ref{lebesgue:measure_by_integral}, every locally integrable function $f\in L^1_{\text{loc}}$ gives rise to a distribution:
       	\begin{gather}
   	    	\langle f,g \rangle = \int_{-\infty}^\infty f(x)g(x)dx.
       	\end{gather}
       Distributions of this form are also said to be \textbf{regular}.
   	\end{example}

    \begin{property}
        The space $\mathcal{D}$ is dense in $\mathcal{D}'$.
    \end{property}
    \begin{property}[Product with smooth functions]
        For every smooth function $f$ and every distribution $\phi$, the product $f\phi$ is defined as
        \begin{gather}
            \langle f\phi,g \rangle := \langle\phi,fg\rangle.
        \end{gather}
        This turns $\mathcal{D}'$ into a $C^\infty$-module.
    \end{property}

\subsection{Support}

	\newdef{Support}{\index{support}
		The support of a distribution is defined as the smallest closed set on which it does not vanish.
	}

	\begin{property}
		A distribution has compact support if and only if it can be extended to a continuous linear functional on $C^\infty(U)$. This gives a nice duality. Distribution act on compactly supported functions and compactly supported distributions act on functions.
	\end{property}

	\begin{property}[Order]
		Distributions with compact support have finite order.
	\end{property}

	\begin{property}
		A distribution that is supported only at 0 can be written as a finite combination of derivatives of the Dirac measure.
	\end{property}

\subsection{Derivatives}

	\newdef{Derivative of a distribution}{\index{derivative!of distributions}\index{weak!derivative}\label{distributions:weak_derivative}
		The derivative of a distribution $\phi$ is defined by duality:
		\begin{gather}
			\left\langle\pderiv{\phi}{x},f\right\rangle := -\left\langle\phi,\pderiv{f}{x}\right\rangle.
		\end{gather}
		This formula is a reasonable definition, since if $\phi$ is regular, the above formula is the one obtained through integration by parts.

        In general, a function $g\in L^1_{\text{loc}}$ is said to be a \textbf{weak derivative} of a function $f\in L^1_{\text{loc}}$ if it satisfies the following equation for all $h\in\mathcal{D}$:
        \begin{gather}
            \langle f,h' \rangle = -\langle g,h \rangle.
        \end{gather}
	}

	\begin{property}[Smooth distributions]\index{smooth}
		Every distribution is smooth, i.e. it is infinitely differentiable. Furthermore, it satisfies the conclusion of Schwarz's theorem \ref{calculus:schwarz_theorem}.
	\end{property}
    \begin{property}[Constant distributions]
        If a distribution $T$ satisfies $T'=0$, then it is a regular distribution induced by a constant function.
    \end{property}

    \newdef{Fundamental solution}{\index{fundamental!solution}\label{distributions:fundamental_solution}
        Let $D$ be a differential operator. A fundamental solution for $D$ is a distribution $\phi$ such that
        \begin{gather}
            D\phi = \delta.
        \end{gather}
    }

\subsection{Examples}

	\newdef{Heaviside distribution}{\index{Heaviside!function}\label{distribution:heaviside_function}
    	The Heaviside function is defined as follows:\footnote{The case $x=0$ is often left undefined, but since this function will always enter formulas inside an integral this does not matter.}
    	\begin{gather}
			H(x) :=
			\begin{cases}
				0&x<0\\
				1&x>0
			\end{cases}
		\end{gather}
        From this definition it follows that for every $f\in\mathcal{D}(U)$:
       	\begin{gather}
       		\label{distribution:heaviside_function_integral}
			\langle H,f \rangle = \int_0^\infty f(x)\ dx.
		\end{gather}
	}

	\newdef{Dirac delta distribution}{\index{Dirac!delta function}\label{distribution:dirac_delta}
    	The Dirac delta distribution is defined as the weak derivative of the Heaviside function:
        \begin{align*}
			\langle \delta,f \rangle&:=\langle H',f \rangle\\
       		&=-\langle H,f' \rangle\\
			&=-\ds\int_0^\infty f'(x)dx\\
       		&=f(0).
		\end{align*}
	}

	\begin{property}[Sampling property]
    	The previous definition can be generalized in the following way (whenever $x_0\in U$):
    	\begin{gather}
			\label{distribution:sieving_dirac_delta}
			f(x_0) = \int_U f(x)\delta(x - x_0)dx,
		\end{gather}
		where the suggestive notation\footnote{See the section on \textit{kernels} further on.} $\delta(x-x_0)$ was used to denote the Dirac delta distribution with support at $x_0$.
	\end{property}

	\begin{definition}[Dirac comb]\index{Dirac!comb}\label{distribution:dirac_comb}
    	\begin{gather}
			\mathrm{III}_b(x) := \sum_{n=-\infty}^\infty\delta(x-nb)
		\end{gather}
	\end{definition}

	\begin{property}
		Let $f(x)\in C^1(\mathbb{R})$ be a function with roots at $x_1,x_2,\ldots,x_n$ such that $f'(x_i)\neq0$. The Dirac delta distribution has the following property:
		\begin{gather}
			\label{distribution:delta_of_function}
			\delta\big(f(x)\big) = \sum_{i=1}^n\frac{1}{|f'(x_i)|}\delta(x-x_i).
		\end{gather}
	\end{property}

	\newformula{Differentiation across discontinuities}{\index{derivative}
    	Let $f$ be a piecewise continuous function with discontinuities at $x_1,\ldots,x_n$ and assume that $f$ induces a distribution by integration. Define the jumps of $f$ at its discontinuities by $\sigma_i := f^+(x_i) - f^-(x_i)$. Next, define the (continuous) function \[f_c(x) := f(x) - \sum_{i=1}^n\sigma_iH(x-x_i).\] Differentiation of this formula  gives \[f'(x) = f'_c(x) + \sum_{i=1}^n\sigma_i\delta(x-x_i).\] It follows that the derivative in the generalized sense of a piecewise continuous function equals the derivative in the classical sense plus a summation of delta functions at the jump discontinuities.
	}

    \begin{example}[Principal value]\index{principal!value}
        The function $\frac{1}{x}$ is clearly not integrable on $\mathbb{R}$. However, its Cauchy principal value exists. This procedure also defines a distribution:
        \begin{gather}
            \left\langle\mathcal{P}\frac{1}{x},f\right\rangle := \lim_{\varepsilon\downarrow0}\int_\varepsilon^\infty\frac{f(x)-f(x^-)}{x}dx.
        \end{gather}
        Moreover, this is the distributional derivative of $\ln|x|$.
    \end{example}

\subsection{Tempered distributions}

	\newdef{Schwartz space}{\index{Schwartz!space}\label{distribution:schwartz_space}
		The Schwartz space of \textbf{rapidly decreasing functions} $\mathscr{S}(\mathbb{R}^n)$ is defined as follows:
		\begin{gather}
    		\mathscr{S}(\mathbb{R}^n) := \left\{f\in C^\infty(\mathbb{R}^n)\,\middle\vert\,\forall i,j\in\mathbb{N}^n,\forall x\in\mathbb{R}^n:\left|x^if^{(j)}(x)\right|<\infty\right\},
		\end{gather}
        where for every multi-index $i$ the symbol $x^i$ denotes the monomial $x_1^{i_1}x_2^{i_2}\cdots$. An equivalent condition is the following. For every $p\in\mathbb{N}$ and $j\in\mathbb{N}^n$, there exists a constant $M_{p,j}(f)$ such that
        \begin{gather}
            \sup_{x\in\mathbb{R}^n}(1+\|x\|^2)^p\|f^{(j)}\|\leq M_{p,j}(f).
        \end{gather}
	}
    \remark{These functions are said to be rapidly decreasing because every derivative $f^{(j)}(x)$ decays faster than any inverse power $x^i$ for $x\longrightarrow\infty$.}

	\newdef{Functions of slow growth}{
		The set of functions of slow growth $N(\mathbb{R}^n)$ is defined as follows:
		\begin{gather}
            N(\mathbb{R}^n) := \left\{f\in C^\infty(\mathbb{R}^n)\,\middle\vert\,\forall i\in\mathbb{N},\exists M_i>0:\left|f^{(i)}(x)\right|=O(\|x\|^i)\text{ for }\|x\|\longrightarrow\infty\right\}.
		\end{gather}
	}

	\begin{property}
		If $f\in\mathscr{S}(\mathbb{R})$ and $f\in N(\mathbb{R})$, then $fg\in\mathscr{S}(\mathbb{R})$.
	\end{property}

\section{Convolutions and kernels}

	\newdef{Direct product}{\index{direct product!of distributions}
		Consider two distributions $\phi\in\mathcal{D}'(U)$ and $\psi\in\mathcal{D}'(V)$. The direct product distribution $\phi\times\psi\in\mathcal{D}'(U\times V)$ is defined by one of the following two equivalent formulas:
		\begin{gather}
			\langle\phi\times\psi,f\rangle := \langle\phi,\langle\psi,f\rangle\rangle
		\end{gather}
		or
		\begin{gather}
            \langle\phi\times\psi,f\rangle := \langle\psi,\langle\phi,f\rangle\rangle.
		\end{gather}
	}

	\newdef{Convolution}{\index{convolution}
		The convolution of two distributions is defined as follows (if it exists):
		\begin{gather}
			\langle\phi\ast\psi,f\rangle := \langle\phi\times\psi,g\rangle
		\end{gather}
		where $g(x,y) := f(x+y)$. It should be noted that the convolution is commutative.
	}
	\begin{example}[Convolution with delta distribution]
		For every distribution $\phi$ one has the following property:
		\begin{gather}
			\delta\ast\phi = \phi.
		\end{gather}
	\end{example}

    \begin{formula}[Convolution of functions]
        The convolution of two (locally integrable) functions $f\ast g$ on $\mathbb{R}^n$ can be defined through Example \ref{distributions:ordinary_function}:
        \begin{gather}
            (f\ast g)(x) := \int_{-\infty}^\infty f(y)g(x-y)dy.
        \end{gather}
    \end{formula}
    \begin{property}[Young inequality]\index{Young!inequality}
        If $f,g\in L^1$, then $f\ast g$ exists a.e. and
        \begin{gather}
            \|f\ast g\|_1\leq \|f\|_1\,\|g\|_1.
        \end{gather}
        This also implies that $f\ast g\in L^1$. Furthermore, consider $p,q$ and $r\in\ ]0,\infty]$ such that
        \begin{gather}
            \frac{1}{p}+\frac{1}{q} = \frac{1}{r}+1.
        \end{gather}
        If $f\in L^p$ and $g\in L^q$, then
        \begin{gather}
            \|f\ast g\|_r\leq\|f\|_p\,\|g\|_q.
        \end{gather}
        This also implies that $f\ast g\in L^r$. A result similar to \ref{lebesgue:holders_inequality} holds for H\"older conjugates ($r=+\infty$), their convolution is an element of $L^\infty$. Furthermore, the convolution is uniformly continuous on all of $\mathbb{R}^n$ and if either $p>1$ or $q>1$, the convolution vanishes at $\infty$.
    \end{property}

    ?? COMPLETE (kernels, ...) ??

\section{Transformations}
\subsection{Fourier series}

	\newdef{Dirichlet kernel}{\index{Dirichlet!kernel}\label{distributions:dirichlet_kernel}
   		The Dirichlet kernel is the collection of functions of the form:
        \begin{gather}
            D_n(x) := \frac{1}{2\pi}\sum_{k=-n}^ne^{ikx}.
        \end{gather}
	}
    \newformula{Sieve property}{
    	If $f\in C^1([-\pi,\pi])$, then
        \begin{gather}
        	\lim_{n\rightarrow\infty}\int_{-\pi}^\pi f(x)D_n(x)dx = 0.
        \end{gather}
    }

	\newformula{Generalized Fourier series}{\index{Fourier!series}\label{distributions:fourier_series}
    	Let $f\in L^2[-l,l]$ be a $2l$-periodic function. This function can be approximated by the following series:
        \begin{gather}
            f(x) = \sum_{n=-\infty}^\infty \left(\frac{1}{2l}\int_{-l}^le^{-i\frac{n\pi x'}{l}}f(x')dx'\right) e^{i\frac{n\pi x}{l}}.
        \end{gather}
	}

    \begin{formula}[Fourier coefficients]
		As seen in the above formula, the Fourier coefficient $\widetilde{f}(k)$ can be calculated by taking an inner product \eqref{hilbert:inner_product_L2}:
		\begin{gather}
			\label{distributions:fourier_coefficients}
       		\widetilde{f}(k) = \int_{-l}^le_k^*(x)f(x)dx \qquad\text{where}\qquad e_k := \sqrt\frac{1}{2l}e^{i\frac{k\pi x}{l}}.
		\end{gather}
	\end{formula}

    \begin{formula}
       	For $2\pi$-periodic functions, the order-$n$ Fourier approximation is given by the following convolution:
       	\begin{gather}
       		s_n(x) = \sum_{k=-n}^n\widetilde{f}(k)e^{ikx} = (D_n \ast f)(x).
       	\end{gather}
    \end{formula}

    \begin{property}[Convergence of the Fourier series]
       	Let $f:\mathbb{R}\rightarrow\mathbb{R}$ be a $2\pi$-periodic function. If $f$ is piecewise $C^1$ on $[-\pi,\pi]$, then
        \begin{gather}
            (D_n\ast f)(x)\xrightarrow{\ n\longrightarrow\infty\ }\frac{f(x+)+f(x-)}{2}.
        \end{gather}
    \end{property}

	\newdef{Periodic extension}{\index{periodic!extension}
    	Let $f$ be piecewise $C^1$ on $[-L,L]$. The periodic extension $f^L$ is defined by gluing ``copies'' of $f$ together. The \textbf{normalized periodic extension} is defined as follows:
        \begin{gather}
        	f^{L,\nu}(x) := \frac{f^L(x+) + f^L(x-)}{2}.
        \end{gather}
    }
    \begin{property}
    	If $f$ is piecewise $C^1$ on $[-L,L]$, the Fourier series approximation of $f$ converges to $f^{L,\nu}$ on all of $\mathbb{R}$.
    \end{property}

\subsection{Fourier transform}\index{Fourier!transform}

    The Fourier series can be used to expand a $2l$-periodic function as an infinite series of exponentials. However, to expand a nonperiodic function $f\in L^1(\mathbb{R})$ one needs the integral Fourier transform:\footnote{All functions are required to be Lebesgue integrable to make the integral converge. Weaker conditions are possible (see the literature).}
    \begin{gather}
        \label{distributions:fourier}
        \mathcal{F}f(\omega) := \frac{1}{\sqrt{2\pi}}\int_{-\infty}^\infty f(t)e^{-i\omega t}dt.
    \end{gather}
    The inverse Fourier transform, if it exists, is given by
    \begin{gather}
        \label{distributions:inverse_fourier}
        f(t) = \mathcal{F}^{-1}(\mathcal{F}f)(t) = \frac{1}{\sqrt{2\pi}}\  \mathcal{P}\int_{-\infty}^\infty\mathcal{F}f(\omega)e^{i\omega t}d\omega.
    \end{gather}
    Equation \eqref{distributions:fourier} is called the (forward) Fourier transform of $f$ and equation \eqref{distributions:inverse_fourier} is called the inverse Fourier transform. The pair $(f,\mathcal{F}f)$ is called a \textbf{Fourier transform pair}.
    \begin{notation}
        The Fourier transform of a function $f$ is often denoted by $\widetilde{f}$ or $\widehat{f}$.
    \end{notation}

    \begin{property}
        From the Riemann-Lebesgue lemma \ref{lebesgue:riemann_lebesue_lemma} it follows that
        \begin{gather}
            \mathcal{F}f(\omega)\longrightarrow0\qquad\text{if}\qquad |\omega|\longrightarrow0.
        \end{gather}
    \end{property}

    \begin{theorem}[Parceval]\index{Parceval}\label{distributions:parcevals_theorem}
        Let $(f,\widetilde{f})$ and $(g,\widetilde{g})$ be two Fourier transform pairs.
        \begin{gather}
            \int_{-\infty}^\infty f(x)g(x)dx = \int_{-\infty}^\infty\widetilde{f}(k)\widetilde{g}(k)dk
        \end{gather}
    \end{theorem}
    \begin{result}[Plancherel]\index{Plancherel}\label{distributions:plancherel_theorem}
        The integral of the square (of the modulus) of a Fourier transform is equal to the integral of the square (of the modulus) of the original function:
        \begin{gather}
            \int_{-\infty}^\infty|f(x)|^2dx = \int_{-\infty}^\infty|\widetilde{f}(k)|^2dk.
        \end{gather}
        This implies that the Fourier transform defines an isometry on $L^2$. In this case it is often called the \textbf{Fourier-Plancherel transform}.
    \end{result}

    Now one can wonder why the Fourier transform is introduced in this chapter. The reason is that Fourier transforms can be generalized to distributions in a convenient way. Naively one could try to extend the definition through duality, but for an arbitrary $\phi\in\mathcal{D}'$ it is not guaranteed that $\mathcal{F}\phi\in\mathcal{D}'$. This is where the Schwartz spaces come in:
    \begin{property}
        The Fourier transform defines an isomorphism on $\mathscr{S}$.
    \end{property}

    One can also show that every Schwartz space has the structure of a Fr\'echet space under the family of seminorms
    \begin{gather}
        s_{p,N}(\phi) := \sup_{x\in\mathbb{R}^n}\sup_{|j|\leq N}\left|(1+\|x\|^2)^p\phi^{(j)}(x)\right|.
    \end{gather}
    The space of \textbf{tempered distributions} is then defined as the continuous dual of $\mathscr{S}$ equipped with the weak-* topology. These spaces have the following important property:
    \begin{property}
        $\mathcal{D}$ is dense in $\mathscr{S}$. This implies that tempered distributions are determined by their values on $\mathcal{D}$.
    \end{property}

    \begin{property}
        The Fourier transform of tempered distributions has some nice additional properties:
        \begin{itemize}
            \item The Fourier transform defines an isomorphism on $\mathscr{S}^*$.
            \item The Fourier transform of a compactly supported function is of slow growth.
            \item The Fourier transform of a convolution is equal to the product of the individual Fourier transforms. (Here, one should restrict to the case of a compactly supported and a tempered distribution such that the convolution is also tempered.)
        \end{itemize}
    \end{property}

    \begin{theorem}[Paley-Wiener]\index{Paley-Wiener}
        The Fourier transform of a compactly supported distribution can be extended to an analytic function on $\mathbb{C}^n$.
    \end{theorem}

\subsection{Laplace transform}

    \newformula{Laplace transform}{\index{Laplace!transform}\label{distributions:laplace}
        \begin{gather}
            \mathcal{L}\{f\}(s) := \int_{0}^\infty f(t)e^{-st}dt
        \end{gather}
	}

	\newformula{Bromwich integral}{\index{Bromwich!integral}\label{distributions:inverse_laplace}
        \begin{gather}
            f(t) = \frac{1}{2\pi i} \int_{\gamma-i\infty}^{\gamma+i\infty}\mathcal{L}\{f\}(s)e^{st}ds
        \end{gather}
	}

\subsection{Integral representations}

	\newformula{Mellin transform}{\index{Mellin}\label{distributions:mellin}
    	\begin{gather}
    		\mathcal{M}\{f(x)\}(s) := \int_0^\infty x^{s-1}f(x)dx
    	\end{gather}
	}
	\newformula{Inverse Mellin transform}{
		\begin{gather}
			\label{distributions:inverse_mellin}
			f(x) = \frac{1}{2\pi i} \int_{\gamma-i\infty}^{\gamma+i\infty}\mathcal{M}\{f(x)\}_{(s)}x^{-s}ds
		\end{gather}
	}

	\newformula{Heaviside step function}{\index{Heaviside!step function}
		\begin{gather}
			\theta(x) = \frac{1}{2\pi i}\int_{-\infty}^\infty\frac{e^{ikx}}{k-i\varepsilon}dk
		\end{gather}
	}
	\newformula{Dirac delta distribution}{\index{Dirac!delta function}
		\begin{gather}
			\delta(x) = \frac{1}{2\pi}\int_{-\infty}^\infty e^{ikx}dk
		\end{gather}
	}

\section{\difficult{Analysis on groups}}

    \newdef{Haar measure}{\index{Haar measure}
        A left (resp. right) Haar measure on a topological group is a regular Borel measure \ref{lebesgue:regular_measure} that is finite on compact subsets and invariant under the left (resp. right) group action. For locally compact groups this is a Radon measure \ref{lebesgue:radon_measure}.
    }
    \begin{example}[Lebesgue measure]
        Consider $\mathbb{R}^n$ as an additive group. Property \ref{lebesgue:translation_invariant} implies that the Lebesgue measure is a left (and right) Haar measure.
    \end{example}

    \begin{theorem}[Haar\footnotemark]
        \footnotetext{A similar theorem holds for right Haar measures.}
        If $G$ is locally compact, there exists a left Haar measure that is unique up to a scalar factor. Moreover, if $G$ is compact, this constant can be fixed by requiring the normalization condition $\mu(G) = 1$.
    \end{theorem}

    \newdef{Pontryagin dual}{\index{Pontryagin!dual}\index{character}
        Let $G$ be a locally compact Abelian group. Its (Pontryagin) dual is defined as the group of continuous homomorphisms from $G$ to the circle group:
        \begin{gather}
            G^\vee := \hom(G,S^1).
        \end{gather}
        In general this group is endowed with the compact-open topology. Elements of this group are called \textbf{group characters} of $G$.
    }
    \begin{theorem}[Pontryagin duality]
        There exists a natural isomorphism $G\mapsto G^{\vee\vee}$.
    \end{theorem}

    \begin{construct}[Fourier transform]
        Consider a locally compact Abelian group $G$ together with its canonical Haar measure $\mu$. For every $f\in L^1(G,\mu)$ one defines the Fourier transform as follows for all $\chi\in G^\vee$:
        \begin{gather}
            \widehat{f}(\chi) := \int_G f(g)\overline{\chi(g)}\,d\mu(g),
        \end{gather}
        where the identification $S^1\cong\mathrm{U}(1)$ is used.
    \end{construct}

    \begin{theorem}[Bochner]\index{Bochner}
        Consider a locally compact Abelian group $G$. There is a bijective correspondence between normalized, positive-definite, continuous functions on $G$ and probability measures on $G^\vee$, where
        \begin{gather}
            f(g) = \int_{G^\vee}\chi(g)\,d\nu(\chi).
        \end{gather}
    \end{theorem}

    ?? COMPLETE ??
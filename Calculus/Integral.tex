\section{Lebesgue integral}
\subsection{Simple functions}

    \newdef{Indicator function}{\index{indicator function}\label{lebesgue:indicator_function}
        \begin{gather}
    	    \mathbbm{1}_A(x) :=
            \begin{cases}
	            1&x\in A\\
                0&x\not\in A.
	        \end{cases}
	    \end{gather}
    }
    \newdef{Simple function}{\index{simple!function}\label{lebesgue:simple_function}
        A function $f:X\rightarrow\mathbb{R}$ on a measurable space $(X,\Sigma)$ that can be expressed as
        \begin{gather}
            f(x) = \sum_{i=1}^na_i\mathbbm{1}_{A_i}(x)
        \end{gather}
        for some $\{a_i\geq0\}_{i\leq n},\{A_i\}_{i\leq n}\subset\Sigma$ and $n\in\mathbb{N}$.
    }
    \begin{definition}[Step function]\index{step function}\label{lebesgue:step_function}
        If $(X,\Sigma)=(\mathbb{R},\mathcal{M})$ and the sets $A_i$ are intervals, the above function is often called a step function.
    \end{definition}

    \newdef{Lebesgue integral of simple functions}{\index{Lebesgue!integral}\label{lebesgue:integral_simple_function}
        Consider a simple function $\varphi$ on a measure space $(X,\Sigma,\mu)$. The Lebesgue integral of $\varphi$ over a measurable set $A\in\Sigma$ with respect to $\mu$ is given by
        \begin{gather}
            \int_A\varphi\,d\mu := \sum_{i=1}^na_i\mu(A\cap A_i).
        \end{gather}
        As usual, if the domain of integration is not mentioned explicitly, an integral over the whole space $X$ is implied.
    }
    \begin{example}
        Let $\mathbbm{1}_\mathbb{Q}$ be the indicator function of the rational numbers. Contrary to the case of Riemann integrals, the above definition makes it possible to integrate the rational indicator function over the real line:
        \begin{gather}
            \int_\mathbb{R}\mathbbm{1}_\mathbb{Q}\,d\lambda = 1\times\lambda(\mathbb{Q}) + 0\times\lambda(\mathbb{R}\backslash\mathbb{Q}) = 0,
        \end{gather}
        where the measure of the rational numbers is 0 because it is a countable set (Corollary \ref{lebesgue:countable_set_is_null}).
    \end{example}

\subsection{Measurable functions}

    \newdef{Integral for nonnegative functions}{\index{Lebesgue!integral}\label{lebesgue:integral}
        The definition for simple functions can be generalized to nonnegative measurable functions $f$ as follows:
        \begin{gather}
            \int_Af\,d\mu := \sup\left\{\int_A\varphi\,d\mu\,\middle\vert\,\varphi\text{ a simple function such that }\varphi\leq f\right\}.
        \end{gather}
        This integral is always nonnegative.
    }

    \begin{formula}
        The following equality allows to change the domain of integrals:
        \begin{gather}
            \label{lebesgue:domain_change}
            \int_Af\,d\mu = \int_Xf\mathbbm{1}_A\,d\mu.
        \end{gather}
    \end{formula}

    \begin{property}
        The Lebesgue integral over a null set is 0.
    \end{property}

    \begin{theorem}[Mean value theorem]\index{mean!value theorem}
        If $a\leq f(x)\leq b$, then $a\lambda(A)\leq\int_Af\,d\lambda\leq b\lambda(A)$.
    \end{theorem}

    \begin{property}
        Let $f$ be a nonnegative measurable function. There exists an increasing sequence $\seq{\varphi}$ of simple functions such that $\varphi_n\nearrow f$. Moreover, if $f$ is bounded on $A\in\Sigma$, the sequence can be chosen to be uniformly convergent on $A$.
    \end{property}

\subsection{Integrable functions}

    \newdef{Integrable function}{\index{integrable}\label{lebesgue:integrable_function}
        Let $A$ be a measurable subset of a measure space $(X,\Sigma,\mu)$. A measurable function $f$ is said to be integrable over $A$ if both $\int_Af^+\,d\mu$ and $\int_Af^-\,d\mu$ are finite. The Lebesgue integral of $f$ over $A$ is then defined as
        \begin{gather}
            \int_Af\,d\mu := \int_Af^+\,d\mu - \int_Af^-\,d\mu.
        \end{gather}
        If only one of the functions $f^+,f^-$ is finite, $f$ is said to be \textbf{quasi-integrable}.
    }

    \begin{property}[Absolute integrability]\label{lebesgue:absolute_integrability}
        $f$ is integrable if and only if $|f|$ is integrable. Furthermore, $\int_A|f|\,d\mu = \int_Af^+\,d\mu + \int_Af^-\,d\mu$.
    \end{property}
    \begin{property}
        Let $f,g$ be integrable functions on a measure space $(X,\Sigma,\mu)$. The following important properties hold:
        \begin{itemize}
            \item\textbf{Linearity}: $\int_A(f+\lambda g)d\mu = \int_Af\,d\mu/\lambda\int_Ag\,d\mu$ for all $\lambda\in\mathbb{R}$
            \item\textbf{Monotonicity}: $f\leq g$ a.e. implies $\int_Af\,d\mu\leq\int_Ag\,d\mu$ and $\forall A\in\Sigma:\int_Af\,d\mu\leq\int_Ag\,d\mu\implies f\leq g$ a.e.
            \item\textbf{Finiteness}: $f$ is finite a.e.
            \item $|\int_Af\,d\mu|\leq\int_A|f|\,d\mu$.
            \item $\int_Af\,d\mu=0,\forall A\in\Sigma\implies f=0$ a.e.
        \end{itemize}
    \end{property}

    \begin{definition}[Lebesgue integrable functions]
        The set of integrable functions over a set $A\in\mathcal{M}$ forms the vector space $\mathcal{L}^1(A)$.
    \end{definition}

    \begin{property}
        Let $f\in\mathcal{L}^1$ and $\varepsilon>0$. There exists a continuous (or step or even simple) function $g$, vanishing outside a finite (or even compact) set, such that $\int|f-g|\,d\mu<\varepsilon$.
    \end{property}

    \newdef{Locally integrable function}{\index{locally!integrable}\label{lebesgue:locally_integrable}
        A measurable function is said to be locally integrable if it is integrable on every compact subset of its domain. The space of locally integrable functions is denoted by $\mathcal{L}^1_{\mathrm{loc}}$.
    }
    \begin{example}
        All continuous functions are locally integrable.
    \end{example}

    \begin{property}[Absolute continuity]\index{continuity!absolute}\label{lebesgue:measure_by_integral}
        Let $f\geq0$ be a measurable function. The mapping $A\mapsto\int_Af\,d\mu$ defines a measure that is $\sigma$-finite if $f$ is locally integrable and finite if $f$ is integrable. Furthermore, this measure is said to be absolutely continuous (with respect to $\mu$). See Section \ref{section:Radon-Nikodym} for a generalization to arbitrary measures.
    \end{property}

\subsection{Convergence theorems}

    \begin{theorem}[Fatou's lemma]\index{Fatou}\label{lebesgue:fatous_lemma}
        Let $\seq{f}$ be a sequence of nonnegative measurable functions.
        \begin{gather}
            \int_A\left(\liminf_{n\rightarrow\infty}f_n\right)\,d\mu \leq \liminf_{n\rightarrow\infty}\int_Af_n\,d\mu
        \end{gather}
    \end{theorem}
    \begin{theorem}[Monotone convergence]\index{monotone!convergence theorem}\label{lebesgue:monotone_convergence_theorem}
        Let $A$ be measurable and let $\seq{f}$ be an increasing sequence of nonnegative measurable functions such that $f_n\nearrow f$ pointwise a.e.
        \begin{gather}
            \int_Af\,d\mu = \lim_{n\rightarrow\infty}\int_Af_n\,d\mu.
        \end{gather}
    \end{theorem}

    \begin{method}\label{lebesgue:linear_proofs}
        To prove results concerning integrable functions in spaces such as $\mathcal{L}^1$ it is often useful to proceed as follows:
        \begin{enumerate}
            \item Verify that the property holds for indicator functions. (This often follows by definition.)
            \item Use linearity to extend the property to simple functions.
            \item Apply the monotone convergence theorem to show that the property holds for all nonnegative measurable functions.
            \item Extend the property to all integrable functions by decomposing $f = f^+-f^-$ and applying linearity again.
        \end{enumerate}
    \end{method}

    \begin{theorem}[Dominated convergence]\index{dominated convergence theorem}\label{lebesgue:dominated_convergence_theorem}
        Let $A$ be measurable set and consider a sequence of measurable functions $\seq{f}$ such that $\forall n\in\mathbb{N}:|f_n|\leq g$ a.e. for some function $g\in\mathcal{L}^1(A)$. If $f_n\rightarrow f$ pointwise a.e., then $f$ is integrable over $A$ and
        \begin{gather}
            \int_Af\,d\mu = \lim_{n\rightarrow\infty}\int_Af_n\,d\mu.
        \end{gather}
    \end{theorem}

    \begin{property}
        Let $\seq{f}$ be a sequence of nonnegative measurable functions
        \begin{gather}
            \int_A\sum_{n=1}^{+\infty}f_n\,d\mu = \sum_{n=1}^{+\infty}\int_Af_n\,d\mu.
        \end{gather}
        One cannot conclude that the right-hand side is finite a.e., so the series on the left-hand side need not be integrable.
    \end{property}

    \begin{theorem}[Beppo Levi\footnotemark]\index{Beppo Levi}\label{lebesgue:beppo_levi}
        \footnotetext{Various other theorems and variants of this theorem can be found in the literature under the same name.}
        Suppose that \[\sum_{i=1}^\infty\int_A|f_n|\,d\mu\] is finite. The series $\sum_{i=1}^\infty f_n(x)$ converges a.e. Furthermore, the series is integrable and
        \begin{gather}
            \int_A\sum_{i=1}^\infty f_n\,d\mu = \sum_{i=1}^\infty\int_Af_n\,d\mu.
        \end{gather}
    \end{theorem}

    \begin{theorem}[Riemann-Lebesgue lemma]\index{Riemann!Riemann-Lebesgue lemma}\label{lebesgue:riemann_lebesue_lemma}
        Let $f\in\mathcal{L}^1(\mathbb{R})$. The sequences \[s_k = \int_{-\infty}^{+\infty}f(x)\sin(kx)dx\] and \[c_k = \int_{-\infty}^{+\infty}f(x)\cos(kx)dx\] both converge to 0.
    \end{theorem}

    \begin{theorem}[Birkhoff ergodicity]\index{ergodic}\index{Birkhoff|seealso{ergodic}}\label{lebesgue:ergodic}
        Let $(X,\Sigma,\mu)$ be a measure space and let $T$ be a $\mu$-ergodic map. For every measurable function $f$ and for $\mu$-almost every element $x\in X$ the integral of $f$ can be computed as an average over the orbit of $x$:
        \begin{gather}
            \lim_{n\rightarrow+\infty}\frac{1}{n+1}\sum_{t=0}^nf(T^n(x)) = \int f\,d\mu.
        \end{gather}
    \end{theorem}

\subsection{Relation to the Riemann integral}

    \begin{property}
        Let $f:[a,b]\rightarrow\mathbb{R}$ be a bounded function.
        \begin{itemize}
            \item $f$ is Riemann-integrable if and only if $f$ is continuous a.e. with respect to the Lebesgue measure on $[a,b]$, i.e. the set of discontinuities of $f$ has measure zero.
            \item Riemann-integrable functions on $[a,b]$ are integrable with respect to the Lebesgue measure on $[a,b]$ and the integrals coincide.
        \end{itemize}
    \end{property}

    \begin{property}
        If $f\geq0$ and the improper Riemann integral \ref{calculus:improper_integral} exists, the Lebesgue integral $\int_{\mathbb{R}}f\,d\mu$ exists and the two integrals coincide. Note that positivity of $f$ is required here. Because the Lebesgue integral is absolute \ref{lebesgue:absolute_integrability}, positive and negative parts cannot cancel (Lebesgue integrals can never be conditionally convergent).
    \end{property}

    The following definition should be compared to \ref{lebesgue:indicator_function} and \ref{distribution:dirac_delta}.
    \newdef{Dirac measure}{\index{Dirac}\label{lebesgue:dirac_measure}
        Define the Dirac measure as follows:
        \begin{gather}
            \delta_a(A) :=
            \begin{cases}
                1&a\in A\\
                0&a\not\in A.
            \end{cases}
        \end{gather}
        Integration with respect to the Dirac measure has the following important property:
        \begin{gather}
            \int f\,d\delta_a = f(a).
        \end{gather}
    }
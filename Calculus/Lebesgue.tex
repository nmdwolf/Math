\chapter{Measure Theory and Lebesgue Integration}\label{chapter:lebesgue}

    The main references for this chapter are \cite{measure, AMP1}.

\section{Measures}
\subsection{General definitions}

    \newdef{Measure}{\index{measure}\index{outer!measure}\index{$\sigma$!additivity}\label{lebesgue:measure}
        Let $X$ be a set and let $\Sigma$ be a $\sigma$-algebra over $X$. A function $\mu:\Sigma\rightarrow\overline{\mathbb{R}}$ is called a measure if it satisfies the following conditions:
        \begin{enumerate}
            \item \textbf{Non-negativity}: $\forall E\in\Sigma:\mu(E) \geq0$,
            \item \textbf{Measure zero}: $\mu(\emptyset) = 0$, and
            \item \textbf{Countable additivity}\footnote{This is also called \textbf{$\sigma$-additivity}.} : $\forall i\neq j:E_i\cap E_j=\emptyset\implies\mu\left(\bigcup_{i=1}^\infty E_i\right) = \sum_{i=1}^\infty \mu(E_i)$.
        \end{enumerate}
        When $\mu$ only satisfies countable subadditivity, i.e. the equality in the last condition becomes an inequality $\leq$, for any collection of sets (disjoint or not) it is called an \textbf{outer measure}.
    }

    \newdef{Measure space}{\label{lebesgue:measure_space}\index{measurable!set}
        The pair $(X, \Sigma)$ is called a measurable space. The elements $E\in\Sigma$ are called \textbf{measurable sets}. The triplet $(X, \Sigma, \mu)$ is called a measure space.
    }

    \begin{method}
        To show that two measures coincide on a $\sigma$-algebra, it suffices to show that they coincide on the generating sets and apply the monotone class theorem \ref{set:theorem:monotone_class}.
    \end{method}

    \newdef{Almost everywhere\footnotemark}{\label{lebesgue:almost_everywhere}\index{almost everywhere}
        \footnotetext{In probability theory this is often called \textbf{almost surely}.}
        Let $(X, \Sigma, \mu)$ be a measure space. A property $P$ is said to hold on X almost everywhere (a.e.) if it satisfies the following equation:
        \begin{gather}
            \mu\big(\{x\in X:\neg P(x)\}\big) = 0.
        \end{gather}
    }

    \newdef{Complete measure space}{\index{complete!measure space}
        The measure space $(X,\Sigma,\mu)$ is said to be complete if for every $E\in\Sigma$ with $\mu(E) = 0$ the following property holds for all $A\subset E$:
        \begin{gather}
            A\in\Sigma \qquad\text{and}\qquad \mu(A) = 0.
        \end{gather}
    }
    \newdef{Completion}{
        Let $\mathcal{F},\mathcal{G}$ be $\sigma$-algebras over a set $X$. $\mathcal{G}$ is said to be the completion of $\mathcal{F}$ if it is the smallest $\sigma$-algebra such that the measure space $(X,\mathcal{G},\mu)$ is complete.
    }

    \newdef{Borel measure}{\index{Borel!measure}
        Consider a topological space $X$ together with its Borel $\sigma$-algebra $\mathcal{B}$ (see definition \ref{topology:borel_set}). Any measure defined on the measurable space $(X, \mathcal{B})$ is called a Borel measure.
    }
    \newdef{Regular measure}{\label{lebesgue:regular_measure}
        Let $\mu$ be a measure on a measurable space $(X, \Sigma)$. $\mu$ is called a regular measure if it satisfies the following equations for every measurable set $B$:
        \begin{align}
            \mu(B) =& \inf\big\{\mu(O):O \text{ open and measurable}, O\supset B\big\}\\
            \mu(B) =& \sup\big\{\mu(F):F \text{ compact and measurable}, F\subset B\big\}.
        \end{align}
    }
    \newdef{Radon measure}{\index{Radon!measure}\index{locally!finite}\label{lebesgue:radon_measure}
        A regular Borel measure with the additional property that it is \textbf{locally finite}, i.e. every point has a neighbourhood of finite measure. If we restrict ourselves to locally compact Hausdorff spaces then this is equivalent to requiring that every compact subset has finite measure.
    }

    \newdef{\texorpdfstring{$\sigma$-}{sigma-}finite measure}{\index{$\sigma$!finite}\label{lebesgue:sigma_finite_measure}
        Let $(X,\Sigma,\mu)$ be a measure space. The measure $\mu$ is said to be $\sigma$-finite if there exists a sequence $\seq{A}$ of measurable sets such that $\bigcup_{n=1}^{+\infty}A_n = X$ with $\forall A_n:\mu(A_n) < +\infty$.
    }

    \newdef{Measure-preserving map}{
        Let $(X,\Sigma,\mu)$ be a measure space and consider a map $T:X\rightarrow X$. $T$ is said to be measure-preserving if it satisfies the following equation:
        \begin{gather}
            \mu\left(T^{-1}(A)\right) = \mu(A)
        \end{gather}
        for all $A\in\Omega$. This equation can also be written using a pushforward notation: $T_\ast\mu=\mu$. These form the morphisms in the category of measure spaces.
    }

    \newdef{Ergodic map}{\index{ergodic}
        Let $(X,\Omega)$ be a measure space. Consider a measure-preserving map $T:X\rightarrow X$. $T$ is said to be ergodic if the following conditions is satisfied:
        \begin{gather}
            T(A) = A\implies \mu(A) = 0 \lor \mu(X\backslash A) = 0.
        \end{gather}
        This is equivalent to stating that for every set $A\in\Sigma$ with positive measure the following condition holds:
        \begin{gather}
            \mu\left(\bigcup_{n=1}^\infty T^{-n}(A)\right) = 1.
        \end{gather}
    }

    \begin{property}
        Consider a topological space $X$ with Borel $\sigma$-algebra $\mathcal{B}$. Almost every $T$-orbit is dense in the support of $\mu$.
    \end{property}

    \newdef{Mixing}{\index{mixing}
        An endomorphism of a measure spaces $(X,\Sigma,\mu)$ is said to be mixing if for all measurable spaces $A,B$ the following equality holds:
        \begin{gather}
            \lim_{n\rightarrow+\infty}\mu\left(T^{-n}(A)\cap B\right) = \mu(A)\mu(B)
        \end{gather}
    }
    \begin{property}
        All mixing transformations are ergodic.
    \end{property}

\subsection{Lebesgue measure}

    \newdef{Null set}{\index{null!set}
        A set $A\subset\mathbb{R}$ is called a null set if it can be covered by a sequence of intervals of arbitrarily small length, i.e. $\forall\varepsilon>0$ there exists a sequence $\seq{I}$ such that
        \begin{gather}
            A \subseteq \bigcup_{n=1}^{+\infty}I_n
        \end{gather}
        and
        \begin{gather}
            \sum_{i=1}^{+\infty}l(I_n) < \varepsilon.
        \end{gather}
    }

    \begin{property}
        Let $(E_i)_{i\in\mathbb{N}}$ be a sequence of null sets. The union $\bigcup_{i=1}^{+\infty}E_i$ is also null.
    \end{property}
    \begin{result}\label{lebesgue:theorem:countable_set_is_null}
        Any countable set is null.
    \end{result}

    \newdef{Lebesgue outer measure}{\index{Lebesgue!outer measure}\label{lebesgue:outer_measure}
        Let $X\subseteq\mathbb{R}$ be a set. The (Lebesgue) outer measure of $X$ is defined as follows:
        \begin{gather}
            m^*(X) := \inf\left\{\sum_{i=1}^{+\infty} l(I_i)\text{ with }(I_i)_{i\in\mathbb{N}} \text{ a sequence of open intervals that covers }X\right\}.
        \end{gather}
    }

    \begin{property}
        Let $I$ be an interval. The outer measure equals the length: $m^*(I) = l(I)$.
    \end{property}
    \begin{property}
        The outer measure is translation-invariant: $m^*(A + t) = m^*(A)$ for all $A,t$.
    \end{property}
    \begin{property}
        The Lebesgue outer measure is an outer measure in the sense of definition \ref{lebesgue:measure}.
    \end{property}

    \begin{theorem}[Carath\'eodory's criterion]\index{Carath\'eodory!criterion}\index{Lebesgue!measure}\index{measurable!set}\label{lebesgue:lebesgue_measure}
        Let $X$ be a subset of $\mathbb{R}$. If $X$ satisfies the following equation, it is said to be \textbf{Lebesgue measurable}:
        \begin{gather}
            \forall E\subseteq\mathbb{R}:m^*(E) = m^*(E\cap X) + m^*(E\cap X^c).
        \end{gather}
        This is denoted by $X\in\mathcal{M}$ and the outer measure $m^*(X)$ is called the Lebesgue measure of $X$. It is denoted by $m(X)$.
    \end{theorem}
    \begin{property}
        All null sets and intervals are measurable.
    \end{property}
    \newprop{Countable additivity}{\index{countable!additivity}
        For every sequence $(E_i)_{i\in\mathbb{N}}$ with $E_i\in\mathcal{M}$ satisfying $i\neq j:E_i\cap E_j = \emptyset$ the following equation holds:
        \begin{gather}
            m\left(\bigcup_{i=1}^{+\infty}E_i\right) = \sum_{i=1}^{+\infty}m(E_i).
        \end{gather}
    }
    \begin{remark}
       Previous property, together with the properties of the outer measure, implies that the Lebesgue measure is indeed a proper measure as defined in \ref{lebesgue:measure}. Furthermore, $\mathcal{M}$ is a $\sigma$-algebra\footnote{See definition \ref{set:sigma_algebra}.} over $\mathbb{R}$.
    \end{remark}

    \begin{property}
        For every $A\subset\mathbb{R}$ there exists a sequence $(O_i)_{i\in\mathbb{N}}$ of open sets such that
        \begin{gather}
            \label{lebesgue:theorem:open_cover_existence}
            A\subset\bigcap_iO_i\qquad\text{and}\qquad m\left(\bigcap_iO_i\right) = m^*(A).
        \end{gather}
    \end{property}
    \begin{property}
        For every $E\in\mathcal{M}$ there exists a sequence $(F_i)_{i\in\mathbb{N}}$ of closed sets such that
        \begin{gather}
            \label{lebesgue:theorem:closed_cover_existence}
            \bigcup_iF_i\subset E\qquad\text{and}\qquad m\left(\bigcup_iF_i\right) = m(E).
        \end{gather}
    \end{property}
    \sremark{The previous 2 theorems imply that the Lebesgue measure is a regular Borel measure (see definition \ref{lebesgue:regular_measure}).}

    \begin{property}
        Let $E\subset\mathbb{R}$. $E\in\mathcal{M}$ if and only if for every $\varepsilon>0$ there exist an open set $O\supset E$ and a closed set $F\subset E$ such that $m^*(O\backslash E) < \varepsilon$ and $m^*(E\backslash F)<\varepsilon$.
    \end{property}

    \begin{property}
        Let $(A_i)_{i\in\mathbb{N}}$ be a sequence of sets with $\forall i:A_i\in\mathcal{M}$. The following two properties apply:
        \begin{align}
            \forall i: A_i\subseteq A_{i+1} &\implies m\left(\bigcup_{i=1}^{+\infty}A_i\right) = \lim_{i\rightarrow+\infty}m(A_i)\\
            \forall i: A_i\supseteq A_{i+1} \land m(A_1)<+\infty &\implies m\left(\bigcap_{i=1}^{+\infty}A_i\right) = \lim_{i\rightarrow+\infty}m(A_i).
        \end{align}
    \end{property}
    \remark{This property is not only valid for the Lebesgue measure but for every countably additive set function.}
    \begin{property}[Continuity]
        The Lebesgue measure $m(X)$ is continuous at $\emptyset$, i.e. if $\lim_{i\rightarrow\infty}A_i=\emptyset$ then $\lim_{i\rightarrow+\infty}m(A_i) = 0$.
    \end{property}

    \begin{property}[Relation between Lebesgue and Borel algebras]\label{lebesgue:completion_remark}
        The Lebesgue $\sigma$-algebra $\mathcal{M}$ is the completion of the Borel $\sigma$-algebra $\mathcal{B}$. (This is in fact how the Lebesgue $\sigma$-algebra was introduced historically.)
    \end{property}

    \begin{construct}[Restriction]\index{Lebesgue!restricted measure}\label{lebesgue:restricted_lebesgue_measure}
        Let $B\subset\mathbb{R}$ be a measurable set with measure $m(B)>0$. The restriction of the Lebesgue measure to the set $B$ is defined as follows:
        \begin{gather}
            \mathcal{M}_B := \left\{A\cap B:A\in\mathcal{M}\right\}\qquad\text{and}\qquad\forall E\in\mathcal{M}_B:m_B(E) := m(E).
        \end{gather}
        Furthermore, the measure space $(B,\mathcal{M}_B,m_B)$ is complete.
    \end{construct}

\subsection{Measurable functions}

    \newdef{Measurable function}{\index{measurable!function}\label{lebesgue:measurable_function}
        A function $f$ is (Lebesgue) measurable if for every interval $I\subset\mathbb{R}:f^{-1}(I)\in\mathcal{M}$.
    }
    \newdef{Borel measurable function}{\index{Borel!measurable function}\label{lebesgue:borel_measurable_function}
        A function $f$ is called Borel measurable\footnote{These functions are often simply called \textbf{Borel functions}.} if for every interval $I\subset\mathbb{R}:f^{-1}(I)\in\mathcal{B}$.
    }
    \remark{The inclusion $\mathcal{B}\subset\mathcal{M}$ implies that every Borel-measurable function is also Lebesgue-measurable.}

    \begin{property}
        The class of Borel/Lebesgue measurable functions defined on $E\in\mathcal{M}$ forms an algebra.
    \end{property}

    \begin{example}
        Following types of functions are measurable:
        \begin{itemize}
            \item monotonic functions,
            \item continuous functions, and
            \item indicator functions.
        \end{itemize}
    \end{example}
    \begin{result}
        Let $f,g$ be measurable functions and let $F:\mathbb{R}\times\mathbb{R}\rightarrow\mathbb{R}$ be a continuous function. The composition $F(f(x), g(x))$ is also measurable.
    \end{result}

    \begin{property}
        Let $f$ be a measurable function. The level set $\{x:f(x) = a\}$ is measurable for all $a\in\mathbb{R}$.
    \end{property}

    \begin{property}
        Define following functions, which are measurable if $f$ is measurable as a result of previous properties:
        \begin{gather}
            \label{lebesgue:positive_part}
            f^+(x) = \max(f,0) =
            \begin{cases}
                f(x)&f(x)>0\\
                0&f(x)\leq0
            \end{cases}
        \end{gather}
        \begin{gather}
            \label{lebesgue:negative_part}
            f^-(x) = \max(-f,0) =
            \begin{cases}
                0&f(x)>0\\
                -f(x)&f(x)\leq0.
            \end{cases}
        \end{gather}
        The function $f:E\rightarrow\mathbb{R}$ is measurable if and only if both $f^+$ and $f^-$ are measurable. Furthermore, $f$ is measurable if $|f|$ is measurable (the converse is false).
    \end{property}

\subsection{Limit operations}

    \begin{property}
        Let $\seq{f}$ be a sequence of Borel/Lebesgue measurable functions. The following functions are also measurable:
        \begin{itemize}
            \item $\ds\min_{i\leq k}(f_i)$ and $\ds\max_{i\leq k}(f_i)$,
            \item $\ds\inf_{i\in\mathbb{N}}(f_i)$ and $\ds\sup_{i\in\mathbb{N}}(f_i)$, and
            \item $\ds\liminf_{i\rightarrow+\infty}(f_i)$ and $\ds\limsup_{i\rightarrow+\infty}(f_i)$.
        \end{itemize}
    \end{property}

    \begin{property}
        Let $f$ be a measurable function and let $g$ be a function such that $f=g$ almost everywhere. Then $g$ is measurable as well.
    \end{property}
    \result{A result of the previous two properties is the following: if a sequence of measurable functions converges pointwise a.e. then the limit is also a measurable function.}

    \newdef{Essential supremum}{\index{essential!supremum}\label{lebesgue:essential_supremum}
        \begin{gather}
            \esssup(f) := \sup\{z:f\geq z\text{ a.e.}\}
        \end{gather}
    }
    \newdef{Essential infimum}{\index{essential!infimum}\label{lebesgue:essential_infimum}
        \begin{gather}
            \essinf(f) := \inf\{z:f\leq z\text{ a.e.}\}
        \end{gather}
    }
    \begin{property}
        Every measurable function $f$ satisfies the following inequalities:
        \begin{itemize}
            \item $f\leq\esssup(f)\text{ a.e.}$ and $f\geq\essinf(f)\text{ a.e.}$
            \item $\esssup(f)\leq\sup(f)$ and $\essinf(f)\geq\inf(f)$.
        \end{itemize}
        The latter pair of inequalities becomes a pair of equalities if $f$ is continuous.
    \end{property}
    \begin{property}
        If $f,g$ are measurable functions then $\esssup(f+g)\leq\esssup(f) + \esssup(g)$. An analogous inequality holds for the essential infimum.
    \end{property}
\section{Lebesgue integral}
\subsection{Simple functions}

    \newdef{Indicator function}{\index{indicator function}\label{lebesgue:indicator_function}
        \begin{gather}
    	    \mathbbm{1}_A(x) :=
            \begin{cases}
	            1&x\in A\\
                0&x\not\in A.
	        \end{cases}
	    \end{gather}
    }
    \newdef{Simple function}{\index{simple!function}\label{lebesgue:simple_function}
            A function $f:X\rightarrow\mathbb{R}$ on a measurable space $(X,\Sigma)$ that can be expressed in the following way:
            \begin{gather}
                f(x) = \sum_{i=1}^n a_i\mathbbm{1}_{A_i}(x)
            \end{gather}
            for some $\{a_i\geq0\}_{i\leq n}, \{A_i\in\Sigma\}_{i\leq n}$ and $n\in\mathbb{N}$.
    }
    \begin{definition}[Step function]\index{step function}\label{lebesgue:step_function}
        If $(X,\Sigma)\equiv(\mathbb{R},\mathcal{M})$ and the sets $A_i$ are intervals, the above function is often called a step function.
    \end{definition}

    \newdef{Lebesgue integral of simple functions}{\index{Lebesgue!integral}\label{lebesgue:integral_simple_function}
        Consider a simple function $\varphi$ on a measure space $(X,\Sigma,\mu)$. The Lebesgue integral of $\varphi$ over a measurable set $A\in\Sigma$ with respect to $\mu$ is given by
        \begin{gather}
            \int_A\varphi\,d\mu = \sum_{i=1}^na_i\mu(A\cap A_i).
        \end{gather}
        As usual, if the domain of integration is not mentioned explicitly, an integral over the whole space $X$ is implied.
    }
    \begin{example}
        Let $\mathbbm{1}_\mathbb{Q}$ be the indicator function of the set of rational numbers. This function is clearly a simple function. Previous formula makes it possible to integrate the rational indicator function over the real line (which is not possible in the sense of Riemann):
        \begin{gather}
            \int_\mathbb{R}\mathbbm{1}_\mathbb{Q}\,d\lambda = 1\times\lambda(\mathbb{Q}) + 0\times\lambda(\mathbb{R}\backslash\mathbb{Q}) = 0
        \end{gather}
        where the measure of the rational numbers is 0 because it is a countable set (Corollary \ref{lebesgue:theorem:countable_set_is_null}).
    \end{example}

\subsection{Measurable functions}

    \newdef{Integral for non-negative functions}{\index{Lebesgue!integral}\label{lebesgue:integral}
        The definition for simple functions can be generalized to non-negative measurable functions $f$ as follows:
        \begin{gather}
            \int_Af\,d\mu := \sup\left\{\int_A\varphi\,d\mu:\varphi\text{ a simple function such that }\varphi\leq f\right\}.
        \end{gather}
        This integral is always non-negative.
    }

    \begin{formula}
        The following equality allows to change the domain of integrals:
        \begin{gather}
            \label{lebesgue:interchanging_domains_with_indicator_function}
            \int_Af\,d\mu = \int_Xf\mathbbm{1}_A\,d\mu.
        \end{gather}
    \end{formula}

    \begin{property}
        The Lebesgue integral over a null set is 0.
    \end{property}

    \begin{theorem}[Mean value theorem]\index{mean!value theorem}
        If $a\leq f(x)\leq b$, then $a\lambda(A)\leq\int_Af\,d\lambda\leq b\lambda(A)$.
    \end{theorem}

    \begin{property}
        Let $f$ be a non-negative measurable function. There exists an increasing sequence $\seq{\varphi}$ of simple functions such that $\varphi_n\nearrow f$. Moreover, if $f$ is bounded on $A\in\Sigma$, the sequence can be chosen to be uniformly convergent on $A$.
    \end{property}

\subsection{Integrable functions}

    \newdef{Integrable function}{\index{integrable}\label{lebesgue:integrable_function}
        Let $A$ be a measurable subset of a measure space $(X,\Sigma,\mu)$. A measurable function $f$ is said to be integrable over $A$ if both $\int_Af^+\,d\mu$ and $\int_Af^-\,d\mu$ are finite. The Lebesgue integral of $f$ over $A$ is then defined as
        \begin{gather}
            \int_Af\,d\mu = \int_Af^+\,d\mu - \int_Af^-\,d\mu.
        \end{gather}
        If only one of the functions $f^+,f^-$ is finite, $f$ is called \textbf{quasi-integrable}.
    }

    \begin{property}[Absolute integrability]\label{lebesgue:absolute_integrability}
        $f$ is integrable if and only if $|f|$ is integrable. Furthermore, $\int_A|f|\,d\mu = \int_Af^+\,d\mu + \int_Af^-\,d\mu$.
    \end{property}
    \begin{property}
        Let $f,g$ be integrable functions on a measure space $(X,\Sigma,\mu)$. The following important properties apply:
        \begin{itemize}
            \item The Lebesgue integral is linear.
            \item $f\leq g$ a.e. implies $\int_Af\,d\mu\leq\int_Ag\,d\mu$.
            \item $\forall A\in\Sigma,\int_Af\,d\mu\leq\int_Ag\,d\mu\implies f\leq g$ a.e.
            \item $f$ is finite a.e.
            \item $|\int_Af\,d\mu|\leq\int_A|f|\,d\mu$.
            \item $f\geq0\land\int_Af\,d\mu=0\implies f=0$ a.e. and, more generally for all integrable $f$:
            \begin{gather}
                \int_Af\,d\mu=0,\forall A\in\Sigma\implies f=0\text{ a.e.}
            \end{gather}
        \end{itemize}
    \end{property}

    \begin{definition}[Lebesgue integrable functions]
        The set of functions integrable over a set $A\in\mathcal{M}$ forms the vector space $\mathcal{L}^1(A)$.
    \end{definition}

    \begin{property}
        Let $f\in\mathcal{L}^1$ and $\varepsilon>0$. There exists a continuous (or step or even simple) function $g$, vanishing outside a finite (or even compact) set, such that $\int|f-g|\,d\mu<\varepsilon$.
    \end{property}

    \newdef{Locally integrable function}{\index{locally!integrable}\label{lebesgue:locally_integrable}
        A measurable function is said to be locally integrable if it is integrable on every compact subset of its domain. The space of locally integrable functions is denoted by $\mathcal{L}^1_{\text{loc}}$.
    }
    \begin{example}
        All continuous functions are locally integrable.
    \end{example}

    \begin{property}[Absolute continuity]\index{continuity!absolute continuity}\label{lebesgue:theorem:measure_by_integral}
        Let $f\geq0$ be measurable. The mapping $A\mapsto\int_Af\,d\mu$ defines a measure (it is $\sigma$-finite if $f$ is locally integrable and finite if $f$ is integrable). Furthermore, this measure is said to be \textbf{absolutely continuous} (with respect to $\mu$). See Section \ref{lebesgue:section:Radon-Nikodym} for a generalization to arbitrary measures.
    \end{property}

\subsection{Convergence theorems}

    \begin{theorem}[Fatou's lemma]\index{Fatou}\label{lebesgue:theorem:fatous_lemma}
        Let $\seq{f}$ be a sequence of non-negative measurable functions.
        \begin{gather}
            \int_A\left(\liminf_{n\rightarrow\infty}f_n\right)\,d\mu \leq \liminf_{n\rightarrow\infty}\int_Af_n\,d\mu
        \end{gather}
    \end{theorem}
    \begin{theorem}[Monotone convergence]\index{monotone!convergence theorem}\label{lebesgue:theorem:monotone_convergence_theorem}
        Let $A$ be measurable and let $\seq{f}$ be an increasing sequence of non-negative measurable functions such that $f_n\nearrow f$ pointwise a.e.
        \begin{gather}
            \int_Af\,d\mu = \lim_{n\rightarrow\infty}\int_Af_n(x)\,d\mu.
        \end{gather}
    \end{theorem}

    \begin{method}\label{lebesgue:method:linear_proofs}
        To prove results concerning integrable functions in spaces such as $\mathcal{L}^1$ it is often useful to proceed as follows:
        \begin{enumerate}
            \item Verify that the property holds for indicator functions. (This often follows by definition.)
            \item Use the linearity to extend the property to simple functions.
            \item Apply the monotone convergence theorem to show that the property holds for all non-negative measurable functions.
            \item Extend the property to all integrable functions by expanding $f = f^+ - f^-$ and applying linearity again.
        \end{enumerate}
    \end{method}

    \begin{theorem}[Dominated convergence]\index{dominated convergence theorem}\label{lebesgue:theorem:dominated_convergence_theorem}
        Let $A$ be measurable and consider a sequence of measurable functions $\seq{f}$ such that $\forall n\in\mathbb{N}:|f_n|\leq g$ a.e. for some function $g\in\mathcal{L}^1(A)$. If $f_n\rightarrow f$ pointwise a.e. then $f$ is integrable over $A$ and
        \begin{gather}
            \int_Af\,d\mu = \lim_{n\rightarrow\infty}\int_Af_n(x)\,d\mu.
        \end{gather}
    \end{theorem}

    \begin{property}
        Let $\seq{f}$ be a sequence of nonnegative measurable functions. The following equality applies:
        \begin{gather}
            \int_A\sum_{n=1}^{+\infty}f_n(x)\,d\mu = \sum_{n=1}^{+\infty}\int_Af_n(x)\,d\mu.
        \end{gather}
        One cannot conclude that the right-hand side is finite a.e., so the series on the left-hand side need not be integrable.
    \end{property}

    \begin{theorem}[Beppo Levi\footnotemark]\index{Beppo Levi}\label{lebesgue:theorem:beppo_levi}
        \footnotetext{Note that various other theorems and variants of this theorem can be found in the literature under the same name.}
        Suppose that \[\sum_{i=1}^\infty\int_A|f_n|(x)\,d\mu\text{ is finite.}\] The series $\sum_{i=1}^\infty f_n(x)$ converges a.e. Furthermore, the series is integrable and
        \begin{gather}
            \int_A\sum_{i=1}^\infty f_n(x)\,d\mu = \sum_{i=1}^\infty\int_Af_n(x)\,d\mu.
        \end{gather}
    \end{theorem}

    \begin{theorem}[Riemann-Lebesgue lemma]\index{Riemann!Riemann-Lebesgue lemma}\label{lebesgue:riemann_lebesue_lemma}
        Let $f\in\mathcal{L}^1$. The sequences \[s_k = \int_{-\infty}^{+\infty}f(x)\sin(kx)dx\] and \[c_k = \int_{-\infty}^{+\infty}f(x)\cos(kx)dx\] both converge to 0.
    \end{theorem}

    \begin{theorem}[Birkhoff ergodicity]\index{ergodic}\index{Birkhoff|seealso{ergodic}}\label{lebesgue:ergodic}
        Let $(X,\Sigma,\mu)$ be a measure space and let $T$ be a $\mu$-ergodic map. For every measurable function $f$ and for $\mu$-almost every element $x\in X$ the integral of $f$ can be computed as an average over the orbit of $x$:
        \begin{gather}
            \lim_{n\rightarrow+\infty}\frac{1}{n+1}\sum_{t=0}^nf(T^n(x)) = \int f\,d\mu.
        \end{gather}
    \end{theorem}

\subsection{Relation to the Riemann integral}

    \begin{property}
        Let $f:[a,b]\rightarrow\mathbb{R}$ be a bounded function.
        \begin{itemize}
            \item $f$ is Riemann-integrable if and only if $f$ is continuous a.e. with respect to the Lebesgue measure on $[a,b]$, i.e. the set of discontinuities of $f$ has measure zero.
            \item Riemann-integrable functions on $[a,b]$ are integrable with respect to the Lebesgue measure on $[a,b]$ and the integrals coincide.
        \end{itemize}
    \end{property}

    \begin{property}
        If $f\geq0$ and the improper Riemann integral \ref{calculus:improper_integral} exists, the Lebesgue integral $\int_{\mathbb{R}}f\,d\mu$ exists and the two integrals coincide. Note that positivity of $f$ is required here. Because the Lbesgue integral is absolute \ref{lebesgue:absolute_integrability}, positive and negative parts cannot cancel, i.e. Lebesgue integrals can never be conditionally convergent.
    \end{property}

    The following definition should be compared to \ref{distribution:dirac_delta}.
    \newdef{Dirac measure}{\index{Dirac}\label{lebesgue:dirac_measure}
        Define the Dirac measure as follows:
        \begin{gather}
            \delta_a(A) =
            \begin{cases}
                1&a\in A\\
                0&a\not\in A.
            \end{cases}
        \end{gather}
        Integration with respect to the Dirac measure has the following nice property:
        \begin{gather}
            \int f\,d\delta_a = f(a).
        \end{gather}
    }

\section{Examples}

    \newdef{Dirac measure\footnotemark}{\index{Dirac}\label{lebesgue:dirac_measure}
        \footnotetext{Compare to definition \ref{distribution:dirac_delta}. }
        We define the Dirac measure as follows:
        \begin{gather}
            \delta_a(X) =
            \begin{cases}
                1&a\in X\\
                0&a\not\in X
            \end{cases}
        \end{gather}
        The integration with respect to the Dirac measure has the following nice property:
        \begin{gather}
            \int g(x)d\ \delta_a = g(a).
        \end{gather}
    }
    \begin{example}
        Let $\mu=\delta_2, X = [-4,1]$ and $Y = [-2,17]$. The following two integrals are easily computed: \[\int_Xd\mu = 0\qquad\qquad\qquad \int_Yd\mu = 1.\]
    \end{example}

\section{Space of integrable functions}
\subsection{Distance}\index{distance}

    To define a distance between functions, a notion of the length of a function is introduced first. Normally this would not be a problem, one could use the integral of a function to define a norm. However, the fact that two functions differing on a null set have the same integral carries problems with it: a nonzero function could have a zero length. To avoid this issue one quotients out these degenerate functions::
    \newdef{$L^1$-space}{
        \nomenclature[S_L1]{$L^1(\Omega)$}{Space of integrable functions on $\Omega$.}
        Define the set of equivalence classes $L^1 = \mathcal{L}^1_{/\equiv}$ by introducing the following equivalence relation: $f\equiv g$ if and only if $f=g$ a.e.
    }
    \begin{property}
        $L^1$ is a Banach space \ref{linalgebra:banach_space}. The norm on $L^1$ is given by
        \begin{gather}
            \label{lebesgue:L1_norm}
            \|f\|_1 := \int|f|\,d\mu.
        \end{gather}
    \end{property}

\subsection{Hilbert space \texorpdfstring{$L^2$}{L2}}\label{lebesgue:section:hilbert_space}

    \begin{property}\label{lebesgue:L2_hilbert_space}
        $L^2$ is a Hilbert space \ref{hilbert:hilbert_space}. The norm on $L^2$ is given by
        \begin{gather}
            \label{lebesgue:L2_norm}
            \|f\|_2 := \left(\int|f|^2\,d\mu\right)^{\frac{1}{2}}.
        \end{gather}
        This norm is induced by the following inner product:
        \begin{gather}
            \label{lebesgue:L2_inner_product}
            \langle f|g \rangle := \int\overline{f}g\,d\mu.
        \end{gather}
    \end{property}

    \begin{formula}[Cauchy-Schwarz inequality]\index{Cauchy-Schwarz}\label{lebesgue:schwarz_inequality}
        Let $f,g\in L^2(X,\mathbb{C})$. We have that $fg\in L^1(X,\mathbb{C})$ and
        \begin{gather}
        \left|\int\overline{f}g\,d\mu\right|\leq\|fg\|_1\leq\|f\|_2\|g\|_2.
        \end{gather}
    \end{formula}
    \sremark{This follows immediately from Formula \ref{lebesgue:holders_inequality}.}

\subsection{\texorpdfstring{$L^p$}{Lp}-spaces}

    Generalizing the previous two function classes leads to the notion of $L^p$-spaces with the following norm:
    \begin{formula}
        For all $1\leq p\leq+\infty$, $L^p(X)$ is a Banach space when equipped with the following norm:
        \begin{gather}
            \label{lebesgue:Lp_norm}
            \|f\|_p := \left(\int_X |f|^p\,d\mu\right)^{\frac{1}{p}}.
        \end{gather}
    \end{formula}
    \remark{Note that $L^2$ is the only $L^p$-space that is also a Hilbert space. The other $L^p$-spaces do not have a norm induced by an inner product.}

    \newformula{H\"{o}lder's inequality}{\index{H\"older!inequality}\index{H\"older!conjugates}\label{lebesgue:holders_inequality}
        Let $\frac{1}{p} + \frac{1}{q} = 1$ with $p\geq1$ (numbers satisfying this equality are called \textbf{H\"older conjugates}). For every $f\in L^p$ and $g\in L^q$ one has that
        \begin{gather}
            \|fg\|_1\leq\|f\|_p\|g\|_q.
        \end{gather}
        This also implies that $fg\in L^1$.
    }
    \newformula{Minkowski's inequality}{\index{Minkowski!inequality}\label{lebesgue:minkowskis_inequality}
        For every $p\geq1$ and $f,g\in L^p$ one has that
        \begin{gather}
            \|f+g\|_p\leq\|f\|_p + \|g\|_p.
        \end{gather}
        This also implies that $f+g\in L^p$.
    }

    \begin{property}[Inclusions]
        $L^1(X)\cap L^\infty(X)\subset L^2(X)$. Moreover, if $X$ has finite measure, then $L^q(X)\subset L^p(X)$ whenever $1\leq p\leq q<+\infty$.
    \end{property}

    Using the H\"older inequality one can prove the following property:
    \begin{property}\label{lebesgue:Lp_duals}
        Let $p,q$ be H\"older conjugates. The spaces $L^p$ and $L^q$ are topological duals, i.e. every function $f\in L^p$ can be identified (one-to-one) with a continuous functional on $L^q$.
    \end{property}

    \newdef{Essentially bounded function}{
        Let $f$ be a measurable function satisfying $\esssup|f|<+\infty$. The function $f$ is said to be essentially bounded and the set of all such functions is denoted by $L^\infty$ (again after quotienting out all functions that are equal a.e.).
    }

    \begin{formula}\index{supremum}
        A norm on $L^\infty$ is given by
        \begin{gather}
            \|f\|_\infty := \esssup|f|.
        \end{gather}
        This norm is called the \textbf{supremum norm} and it induces the supremum metric \ref{topology:supremum_distance}.
    \end{formula}
    \begin{property}
        Equipped with the above norm the space $L^\infty$ becomes a Banach space.
    \end{property}

\section{Product measures}
\subsection{Real hyperspace \texorpdfstring{$\mathbb{R}^n$}{Rn}}

    The notions of intervals and lengths from the one dimensional case can be generalized to higher dimensions in the following way:
    \newdef{Hypercube}{\index{hypercube}
        Let $I_1,\ldots,I_n$ be a sequence of intervals. The hypercube spanned by them is defined as the following set:
        \begin{gather}
            \mathbf{I} := I_1\times\cdots\times I_n.
        \end{gather}
    }
    \newdef{Generalized length}{\index{volume}
        Let $\mathbf{I}$ be a hypercube induced by the set of intervals $I_1,\ldots,I_n$. The generalized length (or \textbf{volume}) of $\mathbf{I}$ is defined as
        \begin{gather}
            l(\mathbf{I}) := \prod_{i=1}^{n}l(I_i).
        \end{gather}
    }

\subsection{Construction of the product measure}

    In this section we will work with the general notation $(\Omega,\mathcal{F},P)$ to denote a measure space. The general condition for multi-dimensional Lebesgue measures is given by the following equation which should hold for all $A_1\in\mathcal{F}_1$ and $A_2\in\mathcal{F}_2$:
    \begin{gather}
        \label{lebesgue:product_measure:general_condition}
        P(A_1\times A_2) = P_1(A_1)P_2(A_2).
    \end{gather}

    \newdef{Section}{\index{section}
        Let $A=A_1\times A_2$. The following two sets are called sections:
        \begin{align*}
            A_{\omega_1} &:= \{\omega_2\in\Omega_2:(\omega_1,\omega_2)\in A\}\subset\Omega_2,\\
            A_{\omega_2} &:= \{\omega_1\in\Omega_1:(\omega_1,\omega_2)\in A\}\subset\Omega_1.
        \end{align*}
    }
    \begin{property}
        Let $\mathcal{F} = \mathcal{F}_1\times\mathcal{F}_2$. If $A\in\mathcal{F}$, then $A_{\omega_1}\in\mathcal{F}_2$ for each $\omega_1$ and $A_{\omega_2}\in\mathcal{F}_1$ for each $\omega_2$. Equivalently, the sets $\mathcal{G}_1 = \{A\in\mathcal{F}\,|\,\forall\omega_1:A_{\omega_1}\in\mathcal{F}_2\}$ and $\mathcal{G}_2 = \{A\in\mathcal{F}\,|\,\forall\omega_2: A_{\omega_2}\in\mathcal{F}_1\}$ coincide with the product $\sigma$-algebra $\mathcal{F}$.
    \end{property}

    \begin{property}
        The function $A_{\omega_2}\mapsto P(A_{\omega_2})$ is a step function:
        \begin{gather*}
            P(A_{\omega_2}) =
            \begin{cases}
                P_1(A_1)&\omega_2\in A_2\\
                0&\omega_2\not\in A_2.
            \end{cases}
        \end{gather*}
    \end{property}

    \begin{formula}[Product measure]\index{product!measure}
        From the previous property it follows that the product measure $P(A)$ can be written in the following way:
        \begin{gather}
            P(A) = \int_{\Omega_2} P_1(A_{\omega_2})dP_2(\omega_2).
        \end{gather}
    \end{formula}
    \begin{property}
        Let $P_1, P_2$ be finite measures. If $A\in\mathcal{F}$, the functions
        \[\omega_1\mapsto P_2(A_{\omega_1}) \qquad\text{and}\qquad \omega_2\mapsto P_1(A_{\omega_2})\]
        are measurable with respect to $\mathcal{F}_1$ and $\mathcal{F}_2$ respectively and
        \begin{gather}
            \int_{\Omega_2} P_1(A_{\omega_2})dP_2(\omega_2) = \int_{\Omega_1} P_2(A_{\omega_1})dP_1(\omega_1).
        \end{gather}
        Furthermore, the set function $P$ is countably additive and if any other product measure coincides with $P$ on all rectangles, it coincides with $P$ on the whole product $\sigma$-algebra.
    \end{property}

\subsection{Fubini's theorem}

    \begin{property}
        Let $f:\Omega_1\times\Omega_2\rightarrow\mathbb{R}$ be a non-negative function. If $f$ is measurable with respect to $\mathcal{F}_1\times\mathcal{F}_2$, then for each $\omega_1\in\Omega_1$ the function $\omega_2\mapsto f(\omega_1,\omega_2)$ is measurable with respect to $\mathcal{F}_2$ (and vice versa). Their integrals with respect to $P_1$ and $P_2$ respectively are also measurable.
    \end{property}
    \newdef{Section}{\index{section}
        The functions $\omega_1\mapsto f(\omega_1,\omega_2)$ and $\omega_2\mapsto f(\omega_1,\omega_2)$ are called sections of $f$.
    }

    \begin{theorem}[Tonelli]\index{Tonelli}
        Let $f:\Omega_1\times\Omega_2\rightarrow\mathbb{R}$ be a non-negative function. The following equalities hold:
        \begin{gather}
            \label{lebesgue:tonelli_theorem}
            \begin{split}
                \int_{\Omega_1\times\Omega_2}f(\omega_1,\omega_2)d(P_1\times P_2)(\omega_1,\omega_2) = \int_{\Omega_1}\left(\int_{\Omega_2}f(\omega_1,\omega_2)dP_2(\omega_2)\right)dP_1(\omega_1)\\ = \int_{\Omega_2}\left(\int_{\Omega_1}f(\omega_1,\omega_2)dP_1(\omega_1)\right)dP_2(\omega_2).
            \end{split}
        \end{gather}
    \end{theorem}

    \begin{result}[Fubini]\index{Fubini}
        Let $f\in L^1(\Omega_1\times\Omega_2)$. The sections of $f$ are integrable in the appropriate spaces. Furthermore, the functions $\omega_1\mapsto\int_{\Omega_2} fdP_2$ and $\omega_2\mapsto\int_{\Omega_1}fdP_1$ are in $L^1(\Omega_1)$ and $L^1(\Omega_2)$ respectively and equality \eqref{lebesgue:tonelli_theorem} holds.
    \end{result}
    \remark{The previous construction and theorems also apply to higher dimensional product spaces. These theorems provide a way to construct higher-dimensional Lebesgue measures $m_n$ by defining them as the completion of the product of $n$ one-dimensional Lebesgue measures.}
\section{Radon-Nikodym theorem}\label{lebesgue:section:Radon-Nikodym}

    \begin{definition}\index{continuity!absolute continuity}\label{lebesgue:absolute_continuity}
        Let $(X,\Sigma)$ be a measurable space and let $\mu, \nu$ be two measures defined on this space. Then $\nu$ is said to be \textbf{absolutely continuous} with respect to $\mu$ if
        \begin{gather}
            \forall A\in\Sigma: \mu(A) = 0\implies\nu(A) = 0.
        \end{gather}
    \end{definition}
    \begin{notation}
        This relation is often denoted by $\nu\ll\mu$.
    \end{notation}

    The following property relates the notion of absolute continuity above with that of Definition \ref{calculus:absolute_continuity}:
    \begin{property}[Absolute continuity]
        Let $\mu, \nu$ be finite measures on a measurable space $(X, \Sigma)$. Then $\nu\ll\mu$ if and only if
        \begin{gather}
            \forall\varepsilon>0:\exists\delta>0:\forall A\in\Sigma:\mu(A)<\delta\implies\nu(A)<\varepsilon.
        \end{gather}
    \end{property}

    \newdef{Singular measures}{\index{measure!singular}\index{orthogonal!measure|see{measure, singular}}
        Consider two measures $\mu,\nu$. If there exists a set $A$ such that $\mu(A)=0=\nu(A^c)$, they are said to be singular (or \textbf{orthogonal}). This is denoted by $\mu\perp\nu$.
    }
    \begin{theorem}[Lebesgue's decomposition theorem]
        Let $\mu,\nu$ be two $\sigma$-finite measures. There exist two other $\sigma$-finite measures $\nu_a,\nu_s$ such that $\nu=\nu_a+\nu_s$ where $\nu_a\ll\mu$ and $\nu_s\perp\mu$.
    \end{theorem}

    \begin{definition}[Dominated measure]\index{measure!dominated}
            Let $\mu, \nu$ be two measures defined on a measurable space $(X, \Sigma)$. Then $\mu$ is said to \textbf{dominate} $\nu$ if $0\leq\nu(F)\leq\mu(F)$ for every $F\in\Sigma$.
    \end{definition}

    \begin{theorem}[Radon-Nikodym theorem for dominated measures]\index{Radon-Nikodym}~\newline
        Let $\mu$ be a finite measure on $(X, \Sigma)$ and let $\nu$ be a measure dominated by $\mu$. There exists a non-negative $\Sigma$-measurable function $f$ such that $\nu(A) = \int_Af\,d\mu$ for all $A\in\Sigma$.
    \end{theorem}

    \newdef{Radon-Nikodym derivative}{\index{derivative|seealso{Radon-Nikodym}}
        The function $f$ in the previous theorem is called the Radon-Nikodym derivative of $\nu$ with respect to $\mu$. It is generally denoted by $\deriv{\nu}{\mu}$.
    }

    \begin{theorem}[Radon-Nikodym theorem]\index{Radon-Nikodym}
        Let $(X,\Sigma)$ be a measurable space and let $\mu,\nu$ be two $\sigma$-finite measures defined on $\Sigma$ such that $\nu\ll\mu$. There exists a non-negative measurable function $f:X\rightarrow\mathbb{R}$ such that $\nu(A) = \int_Af\,d\mu$ for all $A\in\Sigma$.
    \end{theorem}
    \remark{The function $f$ in this theorem is unique up to a $\mu$-null (and thus $\nu$-null) set.}
    \begin{property}
        In general the Radon-Nikodym derivative is not integrable (unless the measures are finite). However, it is always locally integrable \ref{lebesgue:locally_integrable}. Together with Property \ref{lebesgue:theorem:measure_by_integral} this implies that (densities of) absolutely continuous measures are in bijection with locally integrable functions.
    \end{property}

    \begin{property}[Change of variables]
        Let $\mu, \nu$ be finite measures such that $\mu$ dominates $\nu$ and let $\deriv{\nu}{\mu}$ be the associated Radon-Nikodym derivative. For every $\nu$-integrable function $f$ the following equality holds
        \begin{gather}
            \int_A f\,d\nu = \int_A fh_\nu\,d\mu
        \end{gather}
        for all $A\in\Sigma$.
    \end{property}

    \begin{property}\index{chain!rule}
        Let $\lambda,\nu$ and $\mu$ be $\sigma$-finite measures. If $\lambda\ll\mu$ and $\nu\ll\mu$, then the following two properties hold:
        \begin{itemize}
            \item \textbf{Linearity}: $\ds\deriv{(\lambda+\nu)}{\mu} = \deriv{\lambda}{\mu} + \deriv{\lambda}{\mu}$.
            \item \textbf{Chain rule}: If $\lambda\ll\nu$, then $\ds\deriv{\lambda}{\mu} = \deriv{\lambda}{\nu}\deriv{\nu}{\mu}$ a.e.
        \end{itemize}
    \end{property}
\section{Generalizations}

    The previous sections on integration theory (except for the section on the Radon-Nikodym theorem) where all stated in terms of the Lebesgue measure $\mu\equiv m$. However, all that we really needed was the fact that $\mu$ defined a genuine measure on some (complete) measurable space $(\mathbb{R}, \Sigma\subset P(\mathbb{R}))$, together with the properties that followed from it. The conclusion is that almost all statements hold for any measure on any complete measurable space. These include among others Fatou's lemma, the monotone and dominated convergence theorems and Fubini's theorem.

    The general construction starts, as in the case of the Lebesgue measure, from an outer measure $\mu^*$ on a set $X$. The main point of deviation from the Lebesgue construction occurs at this point. Instead of starting from the Borel $\sigma$-algebra and going to the completion (see property \ref{lebesgue:completion_remark}), we start from Carath\'eodory's criterion \ref{lebesgue:lebesgue_measure} and define the $\sigma$-algebra of $\mu$-measurable sets as the collection of those subsets $E\subseteq X$ that satisfy the criterion.

\subsection{Lebesgue-Stieltjes integral}

    As an example of the above considerations we construct an alternative measure (and associated integral) on the Borel $\sigma$-algebra of the real line $\mathbb{R}$. (This construction will have an application in the study of density functions in probability theory.)

    We start from definition \ref{lebesgue:outer_measure}. Consider a function $F$ that is right-continuous, i.e. $F(x^+)=F(x)$, and monotonically increasing. We generalize the length of an interval in the following way:
    \newdef{$F$-length}{\index{length}
        Consider an interval of the form $]a,b]$. The $F$-length of this interval is defined as follows:
        \begin{gather}
            l_F\big(]a,b]\big) := F(b) - F(a).
        \end{gather}
        The restriction to half-open intervals assures that this function is additive when taking unions of intervals. The footnote in definition \ref{topology:borel_set} also assures that the $\sigma$-algebra generated by these intervals is the usual Borel $\sigma$-algebra on $\mathbb{R}$.
    }

    An immediate extension of definition \ref{lebesgue:outer_measure} gives us the outer measure associated to $F$:
    \newdef{$F$-outer measure}{\index{outer!measure}\label{lebesgue:lebesgue_stieltjes_measure}
        Let $X\subseteq\mathbb{R}$ be a set. The $F$-outer measure of $X$ is defined as follows:
        \begin{gather}
            \mu_F^*(X) := \inf\left\{\sum_{i=1}^{+\infty} l_F(I_i)\text{ with }(I_i)_{i\in\mathbb{N}} \text{ a sequence of half-open intervals that cover }X\right\}.
        \end{gather}
    }
    Using this outer measure we can define the $\mu_F$-measurable sets as those satisfying Carath\'eodory's criterion. The main difference with the Lebesgue measure is that $\mu_F$ is not necessarily translation-invariant and that singletons are not necessarily null:
    \begin{property}
        The $F$-measure of a singleton $\{x\}$ is equal to the jump of $F$ at $x$:
        \begin{gather}
            \mu_F\big(\{x\}\big) = F(x) - F(x^-).
        \end{gather}
    \end{property}
    It follows that the Lebesgue-Stieltjes measures having null singletons are exactly those for which $F$ is continuous.

    \begin{property}[Regularity]\index{Borel!measure}
        The Lebesgue-Stieltjes measure is a regular Borel measure. Furthermore, every (finite) regular Borel measure $\mu$ on $\mathbb{R}$ is equal to a Lebesgue-Stieltjes measure where \[F(x) = \mu\big(]-\infty,x]\big).\]
    \end{property}

    \begin{example}[Lebesgue measure]\index{Lebesgue!measure}
        The Lebesgue measure is equal to the Lebesgue-Stieltjes measure where \[F(x)=x.\]
    \end{example}
    \begin{example}[Dirac measure]\index{Dirac!measure}
        The Dirac measure at $a$ can be obtained as the Lebesgue-Stieltjes measure where \[F=\mathbbm{1}_{[a,\infty[}.\]
    \end{example}

\subsection{Signed measures}

    \newdef{Signed measure}{\index{measure!signed}
        Let $X$ be a set and let $\Sigma$ be a $\sigma$-algebra over $X$. A function $\mu:\Sigma\rightarrow]-\infty, +\infty]$ is called a signed measure if it satisfies the following conditions:
        \begin{enumerate}
            \item \textbf{Measure zero}: $\mu(\emptyset) = 0$
            \item \textbf{Countable additivity}\footnote{This is also called \textbf{$\sigma$-additivity}.} : $\forall i\neq j:E_i\cap E_j=\emptyset\implies\mu\left(\bigcup_{i=1}^\infty E_i\right) = \sum_{i=1}^\infty \mu(E_i)$.
        \end{enumerate}
        Note that these requirements are the same as for an ordinary measure (see definition \ref{lebesgue:measure}) except that we now allow the function to become negative. We do not allow it to become $-\infty$ to exclude undefined expressions such as $\infty-\infty$.
    }
    \remark{An important consequence of this generalization is that signed measures are not necessarily monotonic, i.e. $A\subseteq B\slashed{\implies}\mu(A)\leq\mu(B)$. In fact this is a strict relation: a signed measure is monotonic if and only if it is a genuine measure.}

    \newdef{Total variation}{\index{variation}
        Consider a signed measure $\mu$ on a  measurable space $(X, \Sigma)$. The total variation $|\mu|$ is the measure defined as follows:
        \begin{gather}
            |\mu|(A) := \sup\left\{\sum_{P\in\mathcal{P}}|\mu(P)|: \mathcal{P}\subset\Sigma, \mathcal{P}\text{ covers }A\right\}.
        \end{gather}
        Using this measure one can decompose the signed measure $\mu$ as a difference of two genuine measures:
        \begin{gather}
            \mu = \mu^+-\mu^-
        \end{gather}
        where
        \begin{gather}
            \mu^+ = \frac{1}{2}(|\mu|+\mu)\qquad\qquad\qquad\mu^- = \frac{1}{2}(|\mu|-\mu).
        \end{gather}
        Furthermore, this decomposition is minimal in the sense that if $\mu=\lambda_1-\lambda_2$ for any two measures then $\mu^+\leq\lambda_1$ and $\mu^-\leq\lambda_2$.
    }

    The following theorem generalizes both the Radon-Nikodym as Lebesgue decompositions theorems to the case of signed measures:
    \begin{theorem}\index{Radon-Nikodym}\label{lebesgue:signed_radon_nikodym}
        Consider a $\sigma$-finite signed measure $\mu$ and a $\sigma$-finite measure $\nu$ on a measurable space $(X, \Sigma)$. There exists a $\nu$-a.e. unique integrable function $f\in L^1(\nu)$ and a $\sigma$-finite measure $\mu_s\perp\nu$ such that for all $A\in\Sigma$:
        \begin{gather}
            \mu(A) = \int_Afd\nu + \mu_s(A).
        \end{gather}
        As before we call $f$ the Radon-Nikodym derivative of $\mu$ and we denote it by $\deriv{\mu}{\nu}$.
    \end{theorem}

    \begin{theorem}[Hahn-Jordan]\index{Hahn-Jordan}
        Consider a signed measure $\mu$ on a measurable space $(X, \Sigma)$. There exists a set $A\in\Sigma$ such that the minimal decomposition $\mu=\mu^+-\mu^-$ in terms of two measures $\mu^\pm$ is given by
        \begin{gather}
            \mu^+(B) = \mu(A\cap B)\qquad\qquad\qquad\mu^-(B)=\mu(A^c\cap B).
        \end{gather}
    \end{theorem}

    \newdef{Integral with respect to a signed measure}{\index{integral!signed measure}
        Let $\mu$ be a signed measure on a measurable space $(X, \Sigma)$ together with a measurable function $f$ on $A\in\Sigma$. The integral of $f$ with respect to $\mu$ is defined as follows:
        \begin{gather}
            \int_Af\ d\mu = \int_Af\ d\mu^+ - \int_Af\ d\mu^-.
        \end{gather}
    }

    \newdef{Lebesgue-Stieltjes signed measure}{
        Let $F$ be a function of bounded variation. According to property \ref{calculus:bounded_variation_decomposition} we can write it as $F=F_1-F-2$ where $F_1, F_2$ are monotonically increasing absolutely continuous functions. The Lebesgue-Stieltjes (signed) measure associated to $F$ is defined as $\mu_F = \mu_{F_1}-\mu_{F_2}$.
    }

    \begin{theorem}[Fundamental theorem of calculus: Lebesgue]\index{fundamental theorem!of calculus}
        Let $F$ be an absolutely continuous function on the closed interval $[a,b]$. Then $F$ is differentiable $m$-a.e. ($m$ being the Lebesgue measure) and its associated  Lebesgue-Stieltjes measure $\mu_F$ has Radon-Nikodym derivative $\deriv{\mu_F}{m}=F'$ $m$-a.e. Furthermore, for all $x\in[a,b]$ one has
        \begin{gather}
            F(x) - F(a) = \mu_F([a,x]) = \int_a^xF'(t)dt.
        \end{gather}
    \end{theorem}
    \begin{result}
        If $F$ is absolutely continuous and $F'=0$ $m$-a.e. then $F$ is constant.
    \end{result}
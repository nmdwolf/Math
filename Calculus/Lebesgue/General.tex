\section{Generalizations}

    The previous sections on integration theory (except for the section on the Radon-Nikodym theorem) where all stated in terms of the Lebesgue measure $\mu\equiv m$. However, all that we really needed was the fact that $\mu$ defined a genuine measure on some (complete) measurable space $(\mathbb{R}, \Sigma\subset P(\mathbb{R}))$, together with the properties that followed from it. The conclusion is that almost all statements hold for any measure on any complete measurable space. These include among others Fatou's lemma, the monotone and dominated convergence theorems and Fubini's theorem.

    The general construction starts, as in the case of the Lebesgue measure, from an outer measure $\mu^*$ on a set $X$. The main point of deviation from the Lebesgue construction occurs at this point. Instead of starting from the Borel $\sigma$-algebra and going to the completion (see property \ref{lebesgue:completion_remark}), we start from Carath\'eodory's criterion \ref{lebesgue:lebesgue_measure} and define the $\sigma$-algebra of $\mu$-measurable sets as the collection of those subsets $E\subseteq X$ that satisfy the criterion.

\subsection{Lebesgue-Stieltjes integral}

    As an example of the above considerations we construct an alternative measure (and associated integral) on the Borel $\sigma$-algebra of the real line $\mathbb{R}$. (This construction will have an application in the study of density functions in probability theory.)

    We start from definition \ref{lebesgue:outer_measure}. Consider a function $F$ that is right-continuous, i.e. $F(x^+)=F(x)$, and monotonically increasing. We generalize the length of an interval in the following way:
    \newdef{$F$-length}{\index{length}
        Consider an interval of the form $]a,b]$. The $F$-length of this interval is defined as follows:
        \begin{gather}
            l_F\big(]a,b]\big) := F(b) - F(a).
        \end{gather}
        The restriction to half-open intervals assures that this function is additive when taking unions of intervals. The footnote in definition \ref{topology:borel_set} also assures that the $\sigma$-algebra generated by these intervals is the usual Borel $\sigma$-algebra on $\mathbb{R}$.
    }

    An immediate extension of definition \ref{lebesgue:outer_measure} gives us the outer measure associated to $F$:
    \newdef{$F$-outer measure}{\index{outer!measure}\label{lebesgue:lebesgue_stieltjes_measure}
        Let $X\subseteq\mathbb{R}$ be a set. The $F$-outer measure of $X$ is defined as follows:
        \begin{gather}
            \mu_F^*(X) := \inf\left\{\sum_{i=1}^{+\infty} l_F(I_i)\text{ with }(I_i)_{i\in\mathbb{N}} \text{ a sequence of half-open intervals that cover }X\right\}.
        \end{gather}
    }
    Using this outer measure we can define the $\mu_F$-measurable sets as those satisfying Carath\'eodory's criterion. The main difference with the Lebesgue measure is that $\mu_F$ is not necessarily translation-invariant and that singletons are not necessarily null:
    \begin{property}
        The $F$-measure of a singleton $\{x\}$ is equal to the jump of $F$ at $x$:
        \begin{gather}
            \mu_F\big(\{x\}\big) = F(x) - F(x^-).
        \end{gather}
    \end{property}
    It follows that the Lebesgue-Stieltjes measures having null singletons are exactly those for which $F$ is continuous.

    \begin{property}[Regularity]\index{Borel!measure}
        The Lebesgue-Stieltjes measure is a regular Borel measure. Furthermore, every (finite) regular Borel measure $\mu$ on $\mathbb{R}$ is equal to a Lebesgue-Stieltjes measure where \[F(x) = \mu\big(]-\infty,x]\big).\]
    \end{property}

    \begin{example}[Lebesgue measure]\index{Lebesgue!measure}
        The Lebesgue measure is equal to the Lebesgue-Stieltjes measure where \[F(x)=x.\]
    \end{example}
    \begin{example}[Dirac measure]\index{Dirac!measure}
        The Dirac measure at $a$ can be obtained as the Lebesgue-Stieltjes measure where \[F=\mathbbm{1}_{[a,\infty[}.\]
    \end{example}

\subsection{Signed measures}

    \newdef{Signed measure}{\index{measure!signed}
        Let $X$ be a set and let $\Sigma$ be a $\sigma$-algebra over $X$. A function $\mu:\Sigma\rightarrow]-\infty, +\infty]$ is called a signed measure if it satisfies the following conditions:
        \begin{enumerate}
            \item \textbf{Measure zero}: $\mu(\emptyset) = 0$
            \item \textbf{Countable additivity}\footnote{This is also called \textbf{$\sigma$-additivity}.} : $\forall i\neq j:E_i\cap E_j=\emptyset\implies\mu\left(\bigcup_{i=1}^\infty E_i\right) = \sum_{i=1}^\infty \mu(E_i)$.
        \end{enumerate}
        Note that these requirements are the same as for an ordinary measure (see definition \ref{lebesgue:measure}) except that we now allow the function to become negative. We do not allow it to become $-\infty$ to exclude undefined expressions such as $\infty-\infty$.
    }
    \remark{An important consequence of this generalization is that signed measures are not necessarily monotonic, i.e. $A\subseteq B\slashed{\implies}\mu(A)\leq\mu(B)$. In fact this is a strict relation: a signed measure is monotonic if and only if it is a genuine measure.}

    \newdef{Total variation}{\index{variation}
        Consider a signed measure $\mu$ on a  measurable space $(X, \Sigma)$. The total variation $|\mu|$ is the measure defined as follows:
        \begin{gather}
            |\mu|(A) := \sup\left\{\sum_{P\in\mathcal{P}}|\mu(P)|: \mathcal{P}\subset\Sigma, \mathcal{P}\text{ covers }A\right\}.
        \end{gather}
        Using this measure one can decompose the signed measure $\mu$ as a difference of two genuine measures:
        \begin{gather}
            \mu = \mu^+-\mu^-
        \end{gather}
        where
        \begin{gather}
            \mu^+ = \frac{1}{2}(|\mu|+\mu)\qquad\qquad\qquad\mu^- = \frac{1}{2}(|\mu|-\mu).
        \end{gather}
        Furthermore, this decomposition is minimal in the sense that if $\mu=\lambda_1-\lambda_2$ for any two measures then $\mu^+\leq\lambda_1$ and $\mu^-\leq\lambda_2$.
    }

    The following theorem generalizes both the Radon-Nikodym as Lebesgue decompositions theorems to the case of signed measures:
    \begin{theorem}\index{Radon-Nikodym}\label{lebesgue:signed_radon_nikodym}
        Consider a $\sigma$-finite signed measure $\mu$ and a $\sigma$-finite measure $\nu$ on a measurable space $(X, \Sigma)$. There exists a $\nu$-a.e. unique integrable function $f\in L^1(\nu)$ and a $\sigma$-finite measure $\mu_s\perp\nu$ such that for all $A\in\Sigma$:
        \begin{gather}
            \mu(A) = \int_Afd\nu + \mu_s(A).
        \end{gather}
        As before we call $f$ the Radon-Nikodym derivative of $\mu$ and we denote it by $\deriv{\mu}{\nu}$.
    \end{theorem}

    \begin{theorem}[Hahn-Jordan]\index{Hahn-Jordan}
        Consider a signed measure $\mu$ on a measurable space $(X, \Sigma)$. There exists a set $A\in\Sigma$ such that the minimal decomposition $\mu=\mu^+-\mu^-$ in terms of two measures $\mu^\pm$ is given by
        \begin{gather}
            \mu^+(B) = \mu(A\cap B)\qquad\qquad\qquad\mu^-(B)=\mu(A^c\cap B).
        \end{gather}
    \end{theorem}

    \newdef{Integral with respect to a signed measure}{\index{integral!signed measure}
        Let $\mu$ be a signed measure on a measurable space $(X, \Sigma)$ together with a measurable function $f$ on $A\in\Sigma$. The integral of $f$ with respect to $\mu$ is defined as follows:
        \begin{gather}
            \int_Af\ d\mu = \int_Af\ d\mu^+ - \int_Af\ d\mu^-.
        \end{gather}
    }

    \newdef{Lebesgue-Stieltjes signed measure}{
        Let $F$ be a function of bounded variation. According to property \ref{calculus:bounded_variation_decomposition} we can write it as $F=F_1-F-2$ where $F_1, F_2$ are monotonically increasing absolutely continuous functions. The Lebesgue-Stieltjes (signed) measure associated to $F$ is defined as $\mu_F = \mu_{F_1}-\mu_{F_2}$.
    }

    \begin{theorem}[Fundamental theorem of calculus: Lebesgue]\index{fundamental theorem!of calculus}
        Let $F$ be an absolutely continuous function on the closed interval $[a,b]$. Then $F$ is differentiable $m$-a.e. ($m$ being the Lebesgue measure) and its associated  Lebesgue-Stieltjes measure $\mu_F$ has Radon-Nikodym derivative $\deriv{\mu_F}{m}=F'$ $m$-a.e. Furthermore, for all $x\in[a,b]$ one has
        \begin{gather}
            F(x) - F(a) = \mu_F([a,x]) = \int_a^xF'(t)dt.
        \end{gather}
    \end{theorem}
    \begin{result}
        If $F$ is absolutely continuous and $F'=0$ $m$-a.e. then $F$ is constant.
    \end{result}
\chapter{Measure Theory and Lebesgue Integration}\label{chapter:lebesgue}

    The main references for this chapter are \cite{measure, AMP1}.

\section{Measures}
\subsection{General definitions}

    \newdef{Measure}{\index{measure}\index{outer!measure}\index{$\sigma$!additivity}\label{lebesgue:measure}
        Let $X$ be a set and let $\Sigma$ be a $\sigma$-algebra over $X$. A function $\mu:\Sigma\rightarrow\overline{\mathbb{R}}$ is called a measure if it satisfies the following conditions:
        \begin{enumerate}
            \item \textbf{Non-negativity}: $\forall E\in\Sigma:\mu(E) \geq0$,
            \item \textbf{Measure zero}: $\mu(\emptyset) = 0$, and
            \item \textbf{$\sigma$-additivity}: $\forall i\neq j:E_i\cap E_j=\emptyset\implies\mu\left(\bigcup_{i=1}^\infty E_i\right) = \sum_{i=1}^\infty \mu(E_i)$.
        \end{enumerate}
        When $\mu$ only satisfies countable subadditivity, i.e. the equality in the last condition becomes an inequality $\leq$, for any collection of sets (disjoint or not) it is called an \textbf{outer measure}.
    }
    \begin{remark}
        To show that two measures coincide on a $\sigma$-algebra, it suffices to show that they coincide on the generating sets and apply the monotone class theorem \ref{set:theorem:monotone_class}.
    \end{remark}

    \newdef{Measure space}{\label{lebesgue:measure_space}\index{measurable!set}
        The pair $(X,\Sigma)$ is called a measurable space. The elements $E\in\Sigma$ are called \textbf{measurable sets}. The triplet $(X,\Sigma,\mu)$ is called a measure space.
    }

    \newdef{Null set}{\index{null!set}
        A set $A\subset\mathbb{R}$ is said to be null if $\mu(A)=0$.
    }

    \newdef{Almost everywhere\footnotemark}{\label{lebesgue:almost_everywhere}\index{almost everywhere}
        \footnotetext{In probability theory this is often called \textbf{almost surely}.}
        Let $(X,\Sigma,\mu)$ be a measure space. A property $P$ is said to hold on X almost everywhere (a.e.) if it satisfies the following equation:
        \begin{gather}
            \mu\big(\{x\in X:\neg P(x)\}\big) = 0,
        \end{gather}
        i.e. it holds everywhere except for a null set.
    }

    \newdef{Complete measure space}{\index{complete!measure space}
        The measure space $(X,\Sigma,\mu)$ is said to be complete if for every $E\in\Sigma$ with $\mu(E) = 0$ the implication $A\subset E\implies A\in\Sigma$ holds. Additivity then necessarily implies that $\mu(A) = 0$.
    }
    \newdef{Completion}{
        Let $\mathcal{F},\mathcal{G}$ be $\sigma$-algebras over a set $X$. $\mathcal{G}$ is said to be the completion of $\mathcal{F}$ if it is the smallest $\sigma$-algebra such that the measure space $(X,\mathcal{G},\mu)$ is complete.
    }

    \newdef{Borel measure}{\index{Borel!measure}
        Consider a topological space together with its Borel $\sigma$-algebra \ref{topology:borel_set}. Any measure defined on this measurable space is called a Borel measure.
    }
    \newdef{Regular measure}{\label{lebesgue:regular_measure}
        Let $\mu$ be a measure on a measurable space $(X,\Sigma)$. $\mu$ is called a regular measure if it satisfies the following equations for every measurable set $B$:
        \begin{align}
            \mu(B) =& \inf\big\{\mu(O):O \text{ open and measurable}, O\supset B\big\}\\
            \mu(B) =& \sup\big\{\mu(F):F \text{ compact and measurable}, F\subset B\big\}.
        \end{align}
        A Borel regular measure can also be chracterized as a Borel measure such that for every subset $A\subseteq X$ there exists a Borel set $B$ such that $A\subseteq B$ and $\inf\big\{\mu(E):A\subseteq E\in\Sigma\big\}=\mu(B)$.
    }
    \newdef{Radon measure}{\index{Radon!measure}\index{locally!finite}\label{lebesgue:radon_measure}
        A Borel measure on a Hausdorff space that is outer regular, inner regular on open sets and \textbf{locally finite}, i.e. every point has a neighbourhood of finite measure. When restricted to locally-compact Hausdorff spaces, this is equivalent to requiring that every compact subset has finite measure.
    }

    \newdef{\texorpdfstring{$\sigma$-}{sigma-}finite measure}{\index{$\sigma$!finite}\label{lebesgue:sigma_finite_measure}
        Let $(X,\Sigma,\mu)$ be a measure space. The measure $\mu$ is said to be $\sigma$-finite if there exists a sequence $\seq{A}$ of measurable sets such that $\bigcup_{n=1}^{+\infty}A_n = X$ with $\forall n\in\mathbb{N}:\mu(A_n) < +\infty$.
    }

    \newdef{Measure-preserving map}{
        Let $(X,\Sigma,\mu)$ be a measure space and consider a map $T:X\rightarrow X$. $T$ is said to be measure-preserving if it satisfies the following equation:
        \begin{gather}
            \mu\left(T^{-1}(A)\right) = \mu(A)
        \end{gather}
        for all $A\in\Omega$. This equation can also be written using a pushforward notation: $T_\ast\mu=\mu$. These form the morphisms in the category of measure spaces.
    }

    \newdef{Ergodic map}{\index{ergodic}
        Let $(X,\Omega)$ be a measure space. Consider a measure-preserving map $T:X\rightarrow X$. $T$ is said to be ergodic if the following conditions is satisfied:
        \begin{gather}
            T(A) = A\implies \mu(A) = 0 \lor \mu(X\backslash A) = 0.
        \end{gather}
        This is equivalent to stating that for every set $A\in\Sigma$ with positive measure the following condition holds:
        \begin{gather}
            \mu\left(\bigcup_{n=1}^\infty T^{-n}(A)\right) = 1.
        \end{gather}
    }

    \begin{property}
        Consider a topological space $X$ with Borel $\sigma$-algebra $\mathcal{B}$. Almost every $T$-orbit is dense in the support of $\mu$.
    \end{property}

    \newdef{Mixing}{\index{mixing}
        An endomorphism of a measure spaces $(X,\Sigma,\mu)$ is said to be mixing if for all measurable spaces $A,B$ the following equality holds:
        \begin{gather}
            \lim_{n\rightarrow+\infty}\mu\left(T^{-n}(A)\cap B\right) = \mu(A)\mu(B)
        \end{gather}
    }
    \begin{property}
        All mixing transformations are ergodic.
    \end{property}

\subsection{Lebesgue measure}

    \newdef{Lebesgue outer measure}{\index{Lebesgue!outer measure}\label{lebesgue:outer_measure}
        Let $X\subseteq\mathbb{R}$ be a set. The (Lebesgue) outer measure of $X$ is defined as follows:
        \begin{gather}
            \label{lebesgue:outer_measure_formula}
            \lambda^*(X) := \inf\left\{\sum_{i=1}^{+\infty} l(I_i)\text{ with }(I_i)_{i\in\mathbb{N}} \text{ a sequence of open intervals that covers }X\right\}.
        \end{gather}
    }

    \begin{property}
        Let $I$ be an interval. The outer measure equals the length: $\lambda^*(I) = l(I)$.
    \end{property}
    \begin{property}\label{lebesgue:translation_invariant}
        The outer measure is translation-invariant: $\lambda^*(A + t) = \lambda^*(A)$ for all $A,t$.
    \end{property}
    \begin{property}
        The Lebesgue outer measure is an outer measure in the sense of Definition \ref{lebesgue:measure}.
    \end{property}

    \begin{theorem}[Carath\'eodory's criterion]\index{Carath\'eodory!criterion}\index{Lebesgue!measure}\index{measurable!set}\label{lebesgue:lebesgue_measure}
        Let $X$ be a subset of $\mathbb{R}$. If $X$ satisfies the following equation, it is said to be \textbf{Lebesgue measurable}:
        \begin{gather}
            \label{lebesgue:caratheodory}
            \forall E\subseteq\mathbb{R}:\lambda^*(E) = \lambda^*(E\cap X) + \lambda^*(E\cap X^c).
        \end{gather}
        This is denoted by $X\in\mathcal{M}$ and the outer measure $\lambda^*(X)$ is called the Lebesgue measure of $X$. It is denoted by $\lambda(X)$.
    \end{theorem}
    \begin{construct}[Carath\'eodory's extension theorem]\index{premeasure}\label{lebesgue:extension_theorem}
        In fact Carath\'eodory's criterion holds in a general setting. Every outer measure $\mu^*$ gives rise to a $\sigma$-algebra consisting of those sets that satisfy \eqref{lebesgue:caratheodory} with respect to $\mu^*$. Furthermore, consider a \textbf{premeasure} $\mu_0$, i.e. a $\sigma$-additive function defined on an algebra of sets \ref{set:algebra_of_sets} such that $\mu_0(\emptyset) = 0$. Equation \eqref{lebesgue:outer_measure_formula} can be used to define an outer measure $\mu^*$ in terms of the premeasure $\mu_0$ (by taking covers in the algebra of sets). The $\sigma$-algebra generated by this outer measure contains the given algebra of sets and $\mu^*$ restricts to $\mu_0$. This shows that any premeasure can be extended to a genuine measure. Moreover, it can be shown that this measure is complete.
    \end{construct}
    \begin{result}\label{lebesgue:completion_remark}
        The Lebesgue $\sigma$-algebra $\mathcal{M}$ is the completion of the Borel $\sigma$-algebra $\mathcal{B}$. (This is in fact how the Lebesgue $\sigma$-algebra was introduced historically.)
    \end{result}

    \begin{property}\label{lebesgue:theorem:countable_set_is_null}
        Any countable set is null with respect to the Lebesgue outer measure.
    \end{property}

    \begin{property}
        The Lebesgue measure is a regular Borel measure \ref{lebesgue:regular_measure}. For every $A\subset\mathbb{R}$ there exists a sequence $\seq{O}$ of open sets such that
        \begin{gather}
            \label{lebesgue:theorem:open_cover_existence}
            A\subset\bigcap_iO_i\qquad\text{and}\qquad\lambda\left(\bigcap_iO_i\right) = \lambda^*(A),
        \end{gather}
        and for every $E\in\mathcal{M}$ there exists a sequence $\seq{F}$ of closed sets such that
        \begin{gather}
            \label{lebesgue:theorem:closed_cover_existence}
            \bigcup_iF_i\subset E\qquad\text{and}\qquad\lambda\left(\bigcup_iF_i\right) = \lambda(E).
        \end{gather}
    \end{property}

    \begin{property}
        Let $E\subset\mathbb{R}$. $E\in\mathcal{M}$ if and only if for every $\varepsilon>0$ there exist an open set $O\supset E$ and a closed set $F\subset E$ such that $\lambda^*(O\backslash E) < \varepsilon$ and $\lambda^*(E\backslash F)<\varepsilon$.
    \end{property}

    \begin{property}
        Let $\seq{A}$ be a sequence of sets with $\forall i:A_i\in\mathcal{M}$. The following two properties apply:
        \begin{align}
            \forall i: A_i\subseteq A_{i+1} &\implies \lambda\left(\bigcup_{i=1}^{+\infty}A_i\right) = \lim_{i\rightarrow+\infty}\lambda(A_i)\\
            \forall i: A_i\supseteq A_{i+1} \land \lambda(A_1)<+\infty &\implies \lambda\left(\bigcap_{i=1}^{+\infty}A_i\right) = \lim_{i\rightarrow+\infty}\lambda(A_i).
        \end{align}
    \end{property}
    \remark{This property is not only valid for the Lebesgue measure but for every $\sigma$-additive set function.}

    \begin{construct}[Restriction]\index{Lebesgue!restricted measure}\label{lebesgue:restricted_lebesgue_measure}
        Let $B\in\mathcal{M}$ be a Lebesgue-measurable set with measure $\lambda(B)>0$. The restriction of the Lebesgue measure to the set $B$ is defined as follows:
        \begin{gather}
            \mathcal{M}_B := \left\{A\cap B:A\in\mathcal{M}\right\}\qquad\text{and}\qquad\forall E\in\mathcal{M}_B:\lambda_B(E) := \lambda(E).
        \end{gather}
        Furthermore, the measure space $(B,\mathcal{M}_B,\lambda_B)$ is complete.
    \end{construct}

\subsection{Measurable functions}

    \newdef{Measurable function}{\index{measurable!function}
        Consider two measurable spaces $(X,\Sigma_X)$ and $(Y,\Sigma_Y)$. A function $f:X\rightarrow Y$ is said to be measurable if for every measurable set $A\in\Sigma_Y$ the inverse image $f^{-1}(A)$ is also measurable. Equivalently, the $\sigma$-algebra generated by the inverse images of measurable sets in $\Sigma_Y$ should be a sub-$\sigma$-algebra of $\Sigma_X$.
    }

    Two important examples are given below:
    \begin{example}[Borel-measurable function]\label{lebesgue:borel_measurable_function}
        A continuous function $f:X\rightarrow Y$ is called Borel(-measurable) if for every open set $O\in\mathcal{B}_Y:f^{-1}(O)\in\mathcal{B}_X$.
    \end{example}
    \begin{example}[Lebesgue-measurable function]\label{lebesgue:measurable_function}
        A function $f:\mathbb{R}\rightarrow\mathbb{R}$ such that for every interval $I\subset\mathbb{R}:f^{-1}(I)\in\mathcal{M}$.
    \end{example}
    \remark{The inclusion $\mathcal{B}\subset\mathcal{M}$ implies that every Borel-measurable function is also Lebesgue-measurable.}

    \begin{property}
        The class of Borel/Lebesgue-measurable functions defined on $E\in\mathcal{M}$ forms an algebra.
    \end{property}

    \begin{example}
        Following types of functions are Lebesgue-measurable:
        \begin{itemize}
            \item monotonic functions,
            \item continuous functions, and
            \item indicator functions.
        \end{itemize}
    \end{example}
    \begin{result}
        Let $f,g$ be Lebesgue-measurable functions and let $F:\mathbb{R}\times\mathbb{R}\rightarrow\mathbb{R}$ be a continuous function. The composition $F(f(x),g(x))$ is also measurable.
    \end{result}

    \begin{property}
        Let $f$ be a Lebesgue-measurable function. The level set $\{x:f(x) = a\}$ is measurable for all $a\in\mathbb{R}$.
    \end{property}

    \begin{property}
        Define the following functions (which are measurable if $f$ is measurable as a result of previous properties):
        \begin{align}
            \label{lebesgue:positive_part}
            f^+(x) &:= \max(f,0) =
            \begin{cases}
                f(x)&f(x)>0\\
                0&f(x)\leq0
            \end{cases}\\\nonumber\\
            \label{lebesgue:negative_part}
            f^-(x) &:= \max(-f,0) =
            \begin{cases}
                0&f(x)>0\\
                -f(x)&f(x)\leq0.
            \end{cases}
        \end{align}
        The function $f:\mathbb{R}\rightarrow\mathbb{R}$ is measurable if and only if both $f^+$ and $f^-$ are measurable. Furthermore, $f$ is measurable if $|f|$ is measurable (the converse is false).
    \end{property}

\subsection{Limit operations}

    \begin{property}
        Let $\seq{f}$ be a sequence of Borel/Lebesgue-measurable functions. The following functions are also measurable:
        \begin{itemize}
            \item $\ds\min_{i\leq k}(f_i)$ and $\ds\max_{i\leq k}(f_i)$,
            \item $\ds\inf_{i\in\mathbb{N}}(f_i)$ and $\ds\sup_{i\in\mathbb{N}}(f_i)$, and
            \item $\ds\liminf_{i\rightarrow+\infty}(f_i)$ and $\ds\limsup_{i\rightarrow+\infty}(f_i)$.
        \end{itemize}
    \end{property}

    \begin{property}
        If $f$ is a measurable function and $g$ is a function such that $f=g$ almost everywhere, then $g$ is measurable as well.
    \end{property}
    \result{As a result of the previous two properties, if a sequence of measurable functions converges pointwise a.e., the limit is also a measurable function.}

    \newdef{Essential supremum}{\index{essential!supremum}\label{lebesgue:essential_supremum}
        \begin{gather}
            \esssup(f) := \inf\{z:f\leq z\text{ a.e.}\}
        \end{gather}
    }
    \newdef{Essential infimum}{\index{essential!infimum}\label{lebesgue:essential_infimum}
        \begin{gather}
            \essinf(f) := \sup\{z:f\geq z\text{ a.e.}\}
        \end{gather}
    }
    \begin{property}
        Every measurable function $f$ satisfies the following inequalities:
        \begin{itemize}
            \item $f\leq\esssup(f)\text{ a.e.}$ and $f\geq\essinf(f)\text{ a.e.}$
            \item $\esssup(f)\leq\sup(f)$ and $\essinf(f)\geq\inf(f)$.
        \end{itemize}
        The latter pair of inequalities becomes a pair of equalities if $f$ is continuous.
    \end{property}
    \begin{property}
        If $f,g$ are measurable functions, then $\esssup(f+g)\leq\esssup(f) + \esssup(g)$. An analogous inequality holds for the essential infimum.
    \end{property}
\section{Lebesgue integral}
\subsection{Simple functions}

    \newdef{Indicator function}{\index{indicator function}
        An important function when working with sets is the following one:
        \begin{gather}
            \label{lebesgue:indicator_function}
    	    \mathbbm{1}_A(x) =
            \begin{cases}
	            1&x\in A\\
                0&x\not\in A
	        \end{cases}
	    \end{gather}
    }
    \newdef{Simple function}{\index{simple!function}\label{lebesgue:simple_function}
            Let $f$ be a function that takes on a finite number of non-negative values $\{a_i\}$ with for every $i\neq j: f^{-1}(a_i)\cap f^{-1}(a_j) = \emptyset$. $f$ is called a simple function if it can be expanded in the following way:
            \begin{gather}
                f(x) = \sum_{i=1}^n a_i\mathbbm{1}_{A_i}(x)
            \end{gather}
            with $A_i = f^{-1}(a_i)\in\mathcal{M}$.
    }
    \begin{definition}[Step function]\index{step function}\label{lebesgue:step_function}
        If the sets $A_i$ are intervals, the above function is often called a step function.
    \end{definition}

    \newformula{Lebesgue integral of simple functions}{\index{Lebesgue!integral}\label{lebesgue:integral_simple_function}
        Consider a simple function $\varphi$. Furthermore, let $\mu:\mathcal{M}\rightarrow\mathbb{R}$ be the Lebesgue measure and let $E$ be a measurable set. The Lebesgue integral of $\varphi$ over a $E$ with respect to $\mu$ is given by
        \begin{gather}
            \int_E\varphi\ d\mu = \sum_{i=1}^na_i\mu(E\cap A_i).
        \end{gather}
    }
    \begin{example}
        Let $\mathbbm{1}_\mathbb{Q}$ be the indicator function of the set of rational numbers. This function is clearly a simple function. Previous formula makes it possible to integrate the rational indicator function over the real line (which is not possible in the sense of Riemann):
        \begin{gather}
            \int_\mathbb{R}\mathbbm{1}_\mathbb{Q}\ d\mu = 1\times m(\mathbb{Q}) + 0\times m(\mathbb{R}\backslash\mathbb{Q}) = 0
        \end{gather}
        where the measure of the rational numbers is 0 because it is a countable set (see corollary \ref{lebesgue:theorem:countable_set_is_null}).
    \end{example}

\subsection{Measurable functions}

    \newformula{Lebesgue integral for non-negative functions}{\index{Lebesgue!integral}\label{lebesgue:integral}
        The definition for simple functions can be geenralized to non-negative measurable functions $f$ as follows:
        \begin{gather}
            \int_Ef\ d\mu = \sup\left\{\int_E\varphi\ d\mu:\varphi \text{ a simple function such that } \varphi\leq f\right\}.
        \end{gather}
        This integral is always non-negative.
    }

    \begin{formula}
        The following equality follows directly from $A\subseteq\mathbb{R}:A\cup A^c = \mathbb{R}$:
            \begin{gather}
                \label{lebesgue:interchanging_domains_with_indicator_function}
                \int_Af\ d\mu = \int f\mathbbm{1}_A\ d\mu.
            \end{gather}
    \end{formula}

    \begin{property}
        Let $f$ be a non-negative measurable function. Then $f=0$ a.e. if and only if $\int_\mathbb{R}f\ d\mu = 0$.
    \end{property}

    \begin{property}
        The Lebesgue integral over a null set is 0.
    \end{property}
    \begin{property}\label{lebesgue:general_properties}
        Let $f,g$ be measurable functions. The Lebesgue integral has the following properties:
        \begin{itemize}
            \item $f\leq g$ a.e. implies $\int f d\mu\leq\int g\ d\mu$.
            \item Let $A$ be a measurable set and consider a subset $B\subset A$. Then $\int_Bf\ d\mu\leq\int_Af\ d\mu$.
            \item The Lebesgue integral is linear.
            \item For every two disjoint measurable sets $A$ and $B$ we have that $\int_{A\cup B}f\ d\mu = \int_Af\ d\mu + \int_Bf\ d\mu$.
        \end{itemize}
    \end{property}

    \begin{theorem}[Mean value theorem]\index{mean!value theorem}
        If $a\leq f(x)\leq b$, then $am(A)\leq\int_Af\ d\mu\leq bm(A)$.
    \end{theorem}

    \begin{property}
        Let $f$ be a non-negative measurable function. There exists an increasing sequence $(\varphi_i)_{i\in\mathbb{N}}$ of simple functions such that $\varphi_i\nearrow f$.
    \end{property}
    \begin{property}
        Let $f$ be a bounded measurable function defined on the interval $[a,b]$. For every $\varepsilon>0$ there exists a step function\footnote{See definition \ref{lebesgue:step_function}.} $h$ such that $\int_a^b|f-h|\ d\mu<\varepsilon$.
    \end{property}

\subsection{Integrable functions}

    \newdef{Integrable function}{\index{integrable}\label{lebesgue:integrable_function}
        Let $E\in\mathcal{M}$. A measurable function $f$ is said to be integrable over $E$ if both $\int_Ef^+\ d\mu$ and $\int_Ef^-\ d\mu$ are finite. The Lebesgue integral of $f$ over $E$ is then defined as
        \begin{gather}
            \int_E f\ d\mu = \int_E f^+\ d\mu - \int_E f^-\ d\mu.
        \end{gather}
    }
    \sremark{The difference between the integral in definition \ref{lebesgue:integral} and the integral of an integrable function is that with the latter $f$ does not have to be non-negative.}

    \begin{property}
        $f$ is integrable if and only if $|f|$ is integrable. Furthermore, $\int_E|f|\ d\mu = \int_E f^+\ d\mu + \int_E f^-\ d\mu$.
    \end{property}
    \begin{property}
        Let $f,g$ be integrable functions. The following important properties apply:
        \begin{itemize}
            \item $f+g$ is also integrable.
            \item $\forall E\in\mathcal{M}, \int_Ef\ d\mu\leq\int_Eg\ d\mu\implies f\leq g$ a.e.
            \item Let $c\in\mathbb{R}$. $\int_E(cf)\ d\mu = c(\int_Ef\ d\mu)$.
            \item $f$ is finite a.e.
            \item $|\int f\ d\mu|\leq\int|f|\ d\mu$
            \item $f\geq0\land\int f\ d\mu=0\implies f=0$ a.e.
        \end{itemize}
    \end{property}

    \begin{definition}[Lebesgue integrable functions]
        The set of functions integrable over a set $E\in\mathcal{M}$ forms the vector space $\mathcal{L}^1(E)$.
    \end{definition}

    \begin{property}
        Let $f\in\mathcal{L}^1$ and $\varepsilon>0$. There exists a continuous function $g$, vanishing outside some finite interval, such that $\int|f-g|\ d\mu<\varepsilon$.
    \end{property}

    \begin{property}\index{continuity!absolute continuity}\label{lebesgue:theorem:measure_by_integral}
        Let $f\geq0$. The mapping $E\mapsto\int_Ef\ d\mu$ is a measure on $\mathcal{M}$ (if it exists, i.e. if $f$ is integrable). Furthermore, this measure is said to be \textbf{absolutely continuous}. (See section \ref{lebesgue:section:Radon-Nikodym} for further information.)
    \end{property}

    \newdef{Locally integrable function}{\index{locally!integrable}
        A measurable function is said to be locally integrable if it is integrable on every compact subset of its domain. The space of locally integrable functions is denoted by $\mathcal{L}^1_{loc}$.
    }
    \begin{example}
        All continuous functions are locally integrable.
    \end{example}

\subsection{Convergence theorems}

    \begin{theorem}[Fatou's lemma]\index{Fatou}\label{lebesgue:theorem:fatous_lemma}
        Let $(f_n)_{n\in\mathbb{N}}$ be a sequence of non-negative measurable functions.
        \begin{gather}
            \int_E\left(\liminf_{n\rightarrow\infty}f_n\right)\ d\mu \leq \liminf_{n\rightarrow\infty}\int_Ef_n\ d\mu
        \end{gather}
    \end{theorem}
    \begin{theorem}[Monotone convergence theorem]\index{monotone!convergence theorem}\label{lebesgue:theorem:monotone_convergence_theorem}
        Let $E\in\mathcal{M}$ and let $(f_n)_{n\in\mathbb{N}}$ be an increasing sequence of non-negative measurable functions such that $f_n\nearrow f$ pointwise a.e. We have the following powerful equality:
        \begin{gather}
            \int_E f\ d\mu = \lim_{n\rightarrow\infty}\int_E f_n(x)\ d\mu.
        \end{gather}
    \end{theorem}

    \begin{method}\label{lebesgue:method:linear_proofs}
        To prove results concerning integrable functions in spaces such as $\mathcal{L}^1(E)$ it is often useful to proceed as follows:
        \begin{enumerate}
            \item Verify that the property holds for indicator functions. (This often follows by definition.)
            \item Use the linearity to extend the property to simple functions.
            \item Apply the monotone convergence theorem to show that the property holds for all non-negative measurable functions.
            \item Extend the property to all integrable functions by expanding $f = f^+ - f^-$ and applying linearity again.
        \end{enumerate}
    \end{method}

    \begin{theorem}[Dominated convergence theorem]\index{dominated convergence theorem}\label{lebesgue:theorem:dominated_convergence_theorem}
        Let $E\in\mathcal{M}$ and let $(f_n)_{n\in\mathbb{N}}$ be a sequence of measurable functions with $\forall n:|f_n|\leq g$ a.e. for some function $g\in\mathcal{L}^1(E)$. If $f_n\rightarrow f$ pointwise a.e. then $f$ is integrable over $E$ and
        \begin{gather}
            \int_E f\ d\mu = \lim_{n\rightarrow\infty}\int_E f_n(x)\ d\mu.
        \end{gather}
    \end{theorem}

    \begin{property}[Interch]
        Let $(f_n)_{n\in\mathbb{N}}$ be a sequence of non-negative measurable functions. The following equality applies:
        \begin{gather}
            \int\sum_{n=1}^{+\infty}f_n(x)\ d\mu = \sum_{n=1}^{+\infty}\int f_n(x)\ d\mu.
        \end{gather}
        We cannot conclude that the right-hand side is finite a.e., so the series on the left-hand side need not be integrable.
    \end{property}

    \begin{theorem}[Beppo Levi\footnotemark]\index{Beppo Levi}\label{lebesgue:theorem:beppo_levi}
        \footnotetext{Note that various other theorems and variants can be found in the literature under the same name.}
        Suppose that \[\sum_{i=1}^\infty\int|f_n|(x)\ d\mu\text{ is finite.}\] The series $\sum_{i=1}^\infty f_n(x)$ converges a.e. Furthermore, the series is integrable and
        \begin{gather}
            \int\sum_{i=1}^\infty f_n(x)\ d\mu = \sum_{i=1}^\infty\int f_n(x)\ d\mu.
        \end{gather}
    \end{theorem}

    \begin{theorem}[Riemann-Lebesgue lemma]\index{Riemann!Riemann-Lebesgue lemma}\label{lebesgue:riemann_lebesue_lemma}
        Let $f\in\mathcal{L}^1$. The sequences \[s_k = \int_{-\infty}^{+\infty}f(x)\sin(kx)dx\] and \[c_k = \int_{-\infty}^{+\infty}f(x)\cos(kx)dx\] both converge to 0.
    \end{theorem}

\subsection{Relation to the Riemann integral}

    \begin{property}
        Let $f:[a,b]\rightarrow\mathbb{R}$ be a bounded function.
        \begin{itemize}
            \item $f$ is Riemann-integrable if and only if $f$ is continuous a.e. with respect to the Lebesgue measure on $[a,b]$, i.e. the set of discontinuities of $f$ has measure zero.
            \item Riemann-integrable functions on $[a,b]$ are integrable with respect to the Lebesgue measure on $[a,b]$ and the integrals coincide.
        \end{itemize}
    \end{property}

    \begin{property}
        If $f\geq0$ and the improper Riemann integral \ref{calculus:improper_integral} exists, then the Lebesgue integral $\int f\ d\mu$ exists and the two integrals coincide.
    \end{property}

\section{Examples}

    \newdef{Dirac measure\footnotemark}{\index{Dirac}\label{lebesgue:dirac_measure}
        \footnotetext{Compare to definition \ref{distribution:dirac_delta}. }
        We define the Dirac measure as follows:
        \begin{gather}
            \delta_a(X) =
            \begin{cases}
                1&a\in X\\
                0&a\not\in X
            \end{cases}
        \end{gather}
        The integration with respect to the Dirac measure has the following nice property:
        \begin{gather}
            \int g(x)d\ \delta_a = g(a).
        \end{gather}
    }
    \begin{example}
        Let $\mu=\delta_2, X = [-4,1]$ and $Y = [-2,17]$. The following two integrals are easily computed: \[\int_Xd\mu = 0\qquad\qquad\qquad \int_Yd\mu = 1.\]
    \end{example}


\section{Space of integrable functions}
\subsection{Distance}\index{distance}

    To define a distance between functions, we first have to define some notion of length of a function. Normally this would not be a problem, because we know how to integrate integrable functions, however the fact that two functions differing on a null set have the same integral carries problems with it, i.e. a nonzero function could have a zero length. Therefore we will define the ''length'' function on a restricted vector space:
    \newdef{$L^1$-space}{
        \nomenclature[S_L1]{$L^1(\Omega)$}{Space of integrable functions on $\Omega$.}
        Define the set of equivalence classes $L^1(E) = \mathcal{L}^1(E)_{/\equiv}$ by introducing the following equivalence relation on $\mathcal{L}^1(E)$: $f\equiv g$ if and only if $f=g$ a.e.
    }
    \begin{property}
        $L^1(E)$ is a Banach space\footnote{See definition \ref{linalgebra:banach_space}.}. The norm on $L^1(E)$ is given by
        \begin{gather}
            \label{lebesgue:L1_norm}
            ||f||_1 := \int_E |f|\ d\mu.
        \end{gather}
    \end{property}

\subsection{Hilbert space \texorpdfstring{$L^2$}{L2}}\label{lebesgue:section:hilbert_space}

    \begin{property}\label{lebesgue:L2_hilbert_space}
        $L^2$ is a Hilbert space\footnote{See definition \ref{hilbert:hilbert_space}.}. The norm on $L^2(E)$ is given by
        \begin{gather}
            \label{lebesgue:L2_norm}
            ||f||_2 := \left(\int_E |f|^2\ d\mu\right)^{\frac{1}{2}}.
        \end{gather}
        This norm is induced by the following inner product:
        \begin{gather}
            \label{lebesgue:L2_inner_product}
            \langle f|g \rangle := \int_E f\overline{g}\ d\mu.
        \end{gather}
    \end{property}

    \newdef{Orthogonality}{\index{orthogonal}
        As $L^2$ is a Hilbert space and thus has an inner product $\langle\cdot|\cdot\rangle$, it is possible to introduce the concept of orthogonality of functions in the following way:
        \begin{gather}
            \label{lebesgue:orthogonal_functions}
            \langle f|g \rangle = 0\implies\text{f and g are orthogonal}.
        \end{gather}
        Furthermore, it is also possible to introduce the angle between functions in the same way as equation \ref{linalgebra:angle}.
    }

    \begin{formula}[Cauchy-Schwarz inequality]\index{Cauchy-Schwarz}\label{lebesgue:schwarz_inequality}
        Let $f,g\in L^2(E,\mathbb{C})$. We have that $fg\in L^1(E,\mathbb{C})$ and
        \begin{gather}
        \left|\int_E f\overline{g}\ d\mu\right|\leq||fg||_1\leq||f||_2||g||_2.
        \end{gather}
    \end{formula}
    \sremark{This follows immediately from formula \ref{lebesgue:holders_inequality}.}

    \begin{property}
        If $E$ has finite Lebesgue measure then $L^2(E)\subset L^1(E)$.
    \end{property}

\subsection{\texorpdfstring{$L^p$}{Lp}-spaces}

    Generalizing the previous two Lebesgue function classes leads us to the notion of $L^p$-spaces with the following norm:
    \begin{property}
        For all $1\leq p\leq+\infty$, $L^p(E)$ is a Banach space when equipped with the following norm:
        \begin{gather}
            \label{lebesgue:Lp_norm}
            ||f||_p := \left(\int_E |f|^p\ d\mu\right)^{\frac{1}{p}}.
        \end{gather}
    \end{property}
    \remark{Note that $L^2$ is the only $L^p$ space that is also a Hilbert space. The other $L^p$-spaces do not have a norm induced by an inner product.}

    \newformula{H\"{o}lder's inequality}{\index{H\"older!inequality}\index{H\"older!conjugates}\label{lebesgue:holders_inequality}
        Let $\frac{1}{p} + \frac{1}{q} = 1$ with $p\geq1$ (numbers satisfying this equality are called \textbf{H\"older conjugates}). For every $f\in L^p(E)$ and $g\in L^q(E)$ we have that $fg\in L^1(E)$ and that
        \begin{gather}
            ||fg||_1\leq||f||_p||g||_q.
        \end{gather}
    }
    \newformula{Minkowski's inequality}{\index{Minkowski!inequality}\label{lebesgue:minkowskis_inequality}
        For every $p\geq1$ and $f,g\in L^p(E)$ we have
        \begin{gather}
            ||f+g||_p\leq||f||_p + ||g||_p.
        \end{gather}
    }

    Generalizing the property from the previous section we obtain:
    \begin{property}
        If $E$ has finite Lebesgue measure then $L^q(E)\subset L^p(E)$ whenever $1\leq p\leq q<+\infty$.
    \end{property}

    Using the H\"older inequality one can probe the following property:
    \begin{property}\label{lebesgue:Lp_duals}
        Let $p, q$ be H\"older conjugates. The spaces $L^p$ and $L^q$ are topological duals, i.e. every function $f\in L^p$ can be identified (one-to-one) with a continuous functional on $L^q$.
    \end{property}

\subsection{\texorpdfstring{$L^\infty$}{L-infinity}-space of essentially bounded measurable functions}

    \newdef{Essentially bounded function}{
        Let $f$ be a measurable function satisfying $\esssup |f| <+\infty$. The function $f$ is said to be essentially bounded and the set of all such functions is denoted by $L^\infty(E)$.
    }

    \begin{formula}\index{supremum}
        A norm on $L^\infty$ is given by
        \begin{gather}
            ||f||_\infty := \esssup|f|.
        \end{gather}
        This norm is called the \textbf{supremum norm} and it induces the supremum metric \ref{topology:supremum_distance}.
    \end{formula}
    \begin{property}
        Equipped with the above norm the space $L^\infty$ becomes a Banach space.
    \end{property}

\section{Product measures}
\subsection{Real hyperspace \texorpdfstring{$\mathbb{R}^n$}\ }

    The notions of intervals and lengths from the one dimensional case can be generalized to higher dimensions in the following way:
    \newdef{Hypercube}{\index{hypercube}
        Let $I_1,\ldots,I_n$ be a sequence of intervals. The hypercube spanned by them is defined as the following set:
        \begin{gather}
            \mathbf{I} := I_1\times\cdots\times I_n.
        \end{gather}
    }
    \newdef{Generalized length}{\index{volume}
        Let $\mathbf{I}$ be a hypercube induced by the set of intervals $I_1,\ldots,I_n$. The generalized length (or \textbf{volume}) of $\mathbf{I}$ is defined as
        \begin{gather}
            l(\mathbf{I}) := \prod_{i=1}^{n}l(I_i).
        \end{gather}
    }

\subsection{Construction of the product measure}

    In this section we will work with the general notation $(\Omega, \mathcal{F}, P)$ to denote a measure space.

    \newprop{General condition}{
        The general condition for multi-dimensional\newline Lebesgue measures is given by the following equation which should hold for all $A_1\in\mathcal{F}_1$ and $A_2\in\mathcal{F}_2$:
        \begin{gather}
            \label{lebesgue:product_measure:general_condition}
            P(A_1\times A_2) = P_1(A_1)P_2(A_2).
        \end{gather}
    }

    \newdef{Section}{\index{section}
        Let $A=A_1\times A_2$. The following two sets are called sections:
        \begin{align*}
            A_{\omega_1} &:= \{\omega_2\in\Omega_2:(\omega_1,\omega_2)\in A\}\subset\Omega_2
            A_{\omega_2} &:= \{\omega_1\in\Omega_1:(\omega_1,\omega_2)\in A\}\subset\Omega_1.
        \end{align*}
    }
    \begin{property}
        Let $\mathcal{F} = \mathcal{F}_1\times\mathcal{F}_2$. If $A\in\mathcal{F}$ then for each $\omega_1$, $A_{\omega_1}\in\mathcal{F}_2$ and for each $\omega_2$, $A_{\omega_2}\in\mathcal{F}_1$. Equivalently the sets $\mathcal{G}_1 = \{A\in\mathcal{F}:\forall \omega_1,A_{\omega_1}\in\mathcal{F}_2\}$ and $\mathcal{G}_2 = \{A\in\mathcal{F}:\forall \omega_2, A_{\omega_2}\in\mathcal{F}_1\}$ coincide with the product $\sigma$-algebra $\mathcal{F}$.
    \end{property}

    \begin{property}
        The function $A_{\omega_2}\mapsto P(A_{\omega_2})$ is a step function:
        \begin{gather*}
            P(A_{\omega_2}) =
            \begin{cases}
                P_1(A_1)&\omega_2\in A_2\\
                0&\omega_2\not\in A_2.
            \end{cases}
        \end{gather*}
    \end{property}

    \begin{formula}[Product measure]\index{product!measure}
        From the previous property it follows that we can write the product measure $P(A)$ in the following way:
        \begin{gather}
            P(A) = \int_{\Omega_2} P_1(A_{\omega_2})dP_2(\omega_2).
        \end{gather}
    \end{formula}
    \begin{property}
        Let $P_1, P_2$ be finite measures. If $A\in\mathcal{F}$ then the functions
        \[\omega_1\mapsto P_2(A_{\omega_1}) \qquad\text{and}\qquad \omega_2\mapsto P_1(A_{\omega_2})\]
        are measurable with respect to $\mathcal{F}_1$ and $\mathcal{F}_2$ respectively and
        \begin{gather}
            \int_{\Omega_2} P_1(A_{\omega_2})dP_2(\omega_2) = \int_{\Omega_1} P_2(A_{\omega_1})dP_1(\omega_1).
        \end{gather}
        Furthermore, the set function $P$ is countably additive and if any other product measure coincides with $P$ on all rectangles, it coincides with $P$ on the whole product $\sigma$-algebra.
    \end{property}

\subsection{Fubini's theorem}

    \begin{property}
        Let $f:\Omega_1\times\Omega_2\rightarrow\mathbb{R}$ be a non-negative function. If $f$ is measurable with respect to $\mathcal{F}_1\times\mathcal{F}_2$ then for each $\omega_1\in\Omega_1$ the function $\omega_2\mapsto f(\omega_1,\omega_2)$ is measurable with respect to $\mathcal{F}_2$ (and vice versa). Their integrals with respect to $P_1$ and $P_2$ respectively are also measurable.
    \end{property}
    \newdef{Section}{\index{section}
        The functions $\omega_1\mapsto f(\omega_1,\omega_2)$ and $\omega_2\mapsto f(\omega_1,\omega_2)$ are called sections of $f$.
    }

    \begin{theorem}[Tonelli]\index{Tonelli}
        Let $f:\Omega_1\times\Omega_2\rightarrow\mathbb{R}$ be a non-negative function. The following equalities hold:
        \begin{gather}
            \label{lebesgue:tonelli_theorem}
            \begin{split}
                \int_{\Omega_1\times\Omega_2}f(\omega_1,\omega_2)d(P_1\times P_2)(\omega_1,\omega_2) = \int_{\Omega_1}\left(\int_{\Omega_2}f(\omega_1,\omega_2)dP_2(\omega_2)\right)dP_1(\omega_1)\\ = \int_{\Omega_2}\left(\int_{\Omega_1}f(\omega_1,\omega_2)dP_1(\omega_1)\right)dP_2(\omega_2).
            \end{split}
        \end{gather}
    \end{theorem}

    \begin{result}[Fubini]\index{Fubini}
        Let $f\in L^1(\Omega_1\times\Omega_2)$. The sections of $f$ are integrable in the appropriate spaces. Furthermore, the functions $\omega_1\mapsto\int_{\Omega_2} fdP_2$ and $\omega_2\mapsto\int_{\Omega_1}fdP_1$ are in $L^1(\Omega_1)$ and $L^1(\Omega_2)$ respectively and equality \ref{lebesgue:tonelli_theorem} holds.
    \end{result}
    \remark{The previous construction and theorems also apply to higher dimensional product spaces. These theorems provide a way to construct higher-dimensional Lebesgue measures $m_n$ by defining them as the completion of the product of $n$ one-dimensional Lebesgue measures.}
\section{Radon-Nikodym theorem}
\label{lebesgue:section:Radon-Nikodym}
	
    \begin{definition}\index{continuity!absolute continuity}
		Let $(\Omega,\mathcal{F})$ be a measurable space. Let $\mu, \nu$ be two measures defined on this space. $\nu$ is said to be \textbf{absolutely continuous with respect to} $\mu$ if
        \begin{equation}
        	\label{lebesgue:absolute_continuity}
        	\forall A\in\mathcal{F}: \mu(A) = 0\implies\nu(A) = 0
		\end{equation}
	\end{definition}
    \begin{notation}
		This relation is denoted by $\nu \ll \mu$.
	\end{notation}
    \begin{theorem}[Absolute continuity]
    	Let $\mu, \nu$ be finite measures on a measurable space $(\Omega, \mathcal{F})$. Then $\nu\ll\mu$ if and only if
        \begin{equation}
        	\forall\varepsilon>0:\exists\delta>0:\forall A\in\mathcal{F}:\mu(A)<\delta\implies\nu(A)<\varepsilon
        \end{equation}
	\end{theorem}
    
    Property \ref{lebesgue:theorem:measure_by_integral} can be generalized to arbitrary measure spaces as follows:
    \begin{property}
		Let $(\Omega,\mathcal{F}, \mu)$ be a measure space. Let $f:\Omega\rightarrow\mathbb{R}$ be a measurable function such that $\int fd\mu$ exists. Then $\nu(f) = \int_Ffd\mu$ defines a measure $\nu\ll\mu$.
	\end{property}
    
    \begin{definition}[Dominated measure]\index{measure!dominated}
    	Let $\mu, \nu$ be two measures. $\mu$ is said to \textbf{dominate} $\nu$ if $0\leq\nu(F)\leq\mu(F)$ for every $F\in\mathcal{F}$.
    \end{definition}
    
    \begin{theorem}[Radon-Nikodym theorem for dominated measures]\index{Radon-Nikodym!theorem}~\newline
		Let $\mu$ be a measure such that $\mu(\Omega) = 1$. Let $\nu$ be a measure dominated by $\mu$. There exists a non-negative $\mathcal{F}$-measurable function $h$ such that $\nu(F) = \int_F hd\mu$ for all $F\in\mathcal{F}$.
	\end{theorem}
    \sremark{The assumption $\mu(\Omega) = 1$ is non-restrictive as every other finite measure $\phi$ can be normalized by putting $\mu = \frac{\phi}{\phi(\Omega)}$.}
    
    \newdef{Radon-Nikodym derivative}{\index{Radon-Nikodym!derivative}\index{derivative!Radon-Nikodym}
    	The function $h$ as defined in previous theorem is called the Radon-Nikodym derivative of $\nu$ with respect to $\mu$ and we denote it by $\ds\deriv{\nu}{\mu}$.
    }
    
    \begin{theorem}[Radon-Nikodym theorem]\index{Radon-Nikodym!theorem}
		Let $(\Omega,\mathcal{F})$ be a measurable space. Let $\mu,\nu$ be two $\sigma$-finite measures defined on this space such that $\nu\ll\mu$. There exists a non-negative measurable function $g:\Omega\rightarrow\mathbb{R}$ such that $\nu(F) = \int_F gd\mu$ for all $F\in\mathcal{F}$.
	\end{theorem}
    \remark{
    	The function $g$ in the previous theorem is unique up to a $\mu$-null (or $\nu$-null) set.
    }
    
    \begin{property}
		Let $\mu, \nu$ be finite measures such that $\mu$ dominates $\nu$. Let $h_\nu = \deriv{\nu}{\mu}$ be the associated Radon-Nikodym derivative. For every non-negative $\mathcal{F}$-measurable function $f$ we have
        \begin{equation}
			\int_\Omega fd\nu = \int_\Omega fh_\nu d\mu
		\end{equation}
	\end{property}
    \remark{This property also holds for all functions $f\in L^1(\mu)$.}
    
    \begin{property}
    	Let $\lambda,\nu,\mu$ be $\sigma$-finite measures. If $\lambda\ll\mu$ and $\nu\ll\mu$ then we have:
        \begin{itemize}
        	\item $\ds\deriv{(\lambda+\nu)}{\mu} = \deriv{\lambda}{\mu} + \deriv{\lambda}{\mu}$ a.e.
            \item Chain rule: if $\lambda\ll\nu$ then $\ds\deriv{\lambda}{\mu} = \deriv{\lambda}{\nu}\deriv{\nu}{\mu}$ a.e.
        \end{itemize}
    \end{property}
    
\section{Lebesgue-Stieltjes measure}
\section{Generalizations}

    The previous sections on integration theory (except for the section on the Radon-Nikodym theorem) where all stated in terms of the Lebesgue measure $\mu\equiv m$. However, all that we really needed was the fact that $\mu$ defined a genuine measure on some (complete) measurable space $(\mathbb{R}, \Sigma\subset P(\mathbb{R}))$, together with the properties that followed from it. The conclusion is that almost all statements hold for any measure on any complete measurable space. These include among others Fatou's lemma, the monotone and dominated convergence theorems and Fubini's theorem.

    The general construction starts, as in the case of the Lebesgue measure, from an outer measure $\mu^*$ on a set $X$. The main point of deviation from the Lebesgue construction occurs at this point. Instead of starting from the Borel $\sigma$-algebra and going to the completion (see property \ref{lebesgue:completion_remark}), we start from Carath\'eodory's criterion \ref{lebesgue:lebesgue_measure} and define the $\sigma$-algebra of $\mu$-measurable sets as the collection of those subsets $E\subseteq X$ that satisfy the criterion.

\subsection{Lebesgue-Stieltjes integral}

    As an example of the above considerations we construct an alternative measure (and associated integral) on the Borel $\sigma$-algebra of the real line $\mathbb{R}$. (This construction will have an application in the study of density functions in probability theory.)

    We start from definition \ref{lebesgue:outer_measure}. Consider a function $F$ that is right-continuous, i.e. $F(x^+)=F(x)$, and monotonically increasing. We generalize the length of an interval in the following way:
    \newdef{$F$-length}{\index{length}
        Consider an interval of the form $]a,b]$. The $F$-length of this interval is defined as follows:
        \begin{gather}
            l_F\big(]a,b]\big) := F(b) - F(a).
        \end{gather}
        The restriction to half-open intervals assures that this function is additive when taking unions of intervals. The footnote in definition \ref{topology:borel_set} also assures that the $\sigma$-algebra generated by these intervals is the usual Borel $\sigma$-algebra on $\mathbb{R}$.
    }

    An immediate extension of definition \ref{lebesgue:outer_measure} gives us the outer measure associated to $F$:
    \newdef{$F$-outer measure}{\index{outer!measure}\label{lebesgue:lebesgue_stieltjes_measure}
        Let $X\subseteq\mathbb{R}$ be a set. The $F$-outer measure of $X$ is defined as follows:
        \begin{gather}
            \mu_F^*(X) := \inf\left\{\sum_{i=1}^{+\infty} l_F(I_i)\text{ with }(I_i)_{i\in\mathbb{N}} \text{ a sequence of half-open intervals that cover }X\right\}.
        \end{gather}
    }
    Using this outer measure we can define the $\mu_F$-measurable sets as those satisfying Carath\'eodory's criterion. The main difference with the Lebesgue measure is that $\mu_F$ is not necessarily translation-invariant and that singletons are not necessarily null:
    \begin{property}
        The $F$-measure of a singleton $\{x\}$ is equal to the jump of $F$ at $x$:
        \begin{gather}
            \mu_F\big(\{x\}\big) = F(x) - F(x^-).
        \end{gather}
    \end{property}
    It follows that the Lebesgue-Stieltjes measures having null singletons are exactly those for which $F$ is continuous.

    \begin{property}[Regularity]\index{Borel!measure}
        The Lebesgue-Stieltjes measure is a regular Borel measure. Furthermore, every (finite) regular Borel measure $\mu$ on $\mathbb{R}$ is equal to a Lebesgue-Stieltjes measure where \[F(x) = \mu\big(]-\infty,x]\big).\]
    \end{property}

    \begin{example}[Lebesgue measure]\index{Lebesgue!measure}
        The Lebesgue measure is equal to the Lebesgue-Stieltjes measure where \[F(x)=x.\]
    \end{example}
    \begin{example}[Dirac measure]\index{Dirac!measure}
        The Dirac measure at $a$ can be obtained as the Lebesgue-Stieltjes measure where \[F=\mathbbm{1}_{[a,\infty[}.\]
    \end{example}

\subsection{Signed measures}

    \newdef{Signed measure}{\index{measure!signed}
        Let $X$ be a set and let $\Sigma$ be a $\sigma$-algebra over $X$. A function $\mu:\Sigma\rightarrow]-\infty, +\infty]$ is called a signed measure if it satisfies the following conditions:
        \begin{enumerate}
            \item \textbf{Measure zero}: $\mu(\emptyset) = 0$
            \item \textbf{Countable additivity}\footnote{This is also called \textbf{$\sigma$-additivity}.} : $\forall i\neq j:E_i\cap E_j=\emptyset\implies\mu\left(\bigcup_{i=1}^\infty E_i\right) = \sum_{i=1}^\infty \mu(E_i)$.
        \end{enumerate}
        Note that these requirements are the same as for an ordinary measure (see definition \ref{lebesgue:measure}) except that we now allow the function to become negative. We do not allow it to become $-\infty$ to exclude undefined expressions such as $\infty-\infty$.
    }
    \remark{An important consequence of this generalization is that signed measures are not necessarily monotonic, i.e. $A\subseteq B\slashed{\implies}\mu(A)\leq\mu(B)$. In fact this is a strict relation: a signed measure is monotonic if and only if it is a genuine measure.}

    \newdef{Total variation}{\index{variation}
        Consider a signed measure $\mu$ on a  measurable space $(X, \Sigma)$. The total variation $|\mu|$ is the measure defined as follows:
        \begin{gather}
            |\mu|(A) := \sup\left\{\sum_{P\in\mathcal{P}}\mu(P): \mathcal{P}\subset\Sigma, \mathcal{P}\text{ covers }A\right\}.
        \end{gather}
        Using this measure one can decompose the signed measure $\mu$ as a difference of two genuine measures:
        \begin{gather}
            \mu = \mu^+-\mu^-
        \end{gather}
        where
        \begin{gather}
            \mu^+ = \frac{1}{2}(|\mu|+\mu)\qquad\qquad\qquad\mu^- = \frac{1}{2}(|\mu|-\mu).
        \end{gather}
        Furthermore, this decomposition is minimal in the sense that if $\mu=\lambda_1-\lambda_2$ for any two measures then $\mu^+\leq\lambda_1$ and $\mu^-\leq\lambda_2$.
    }

    The following theorem generalizes both the Radon-Nikodym as Lebesgue decompositions theorems to the case of signed measures:
    \begin{theorem}\index{Radon-Nikodym}\label{lebesgue:signed_radon_nikodym}
        Consider a $\sigma$-finite signed measure $\mu$ and a $\sigma$-finite measure $\nu$ on a measurable space $(X, \Sigma)$. There exists a $\nu$-a.e. unique integrable function $f\in L^1(\nu)$ and a $\sigma$-finite measure $\mu_s\perp\nu$ such that for all $A\in\Sigma$:
        \begin{gather}
            \mu(A) = \int_Afd\nu + \mu_s(A).
        \end{gather}
        As before we call $f$ the Radon-Nikodym derivative of $\mu$ and we denote it by $\deriv{\mu}{\nu}$.
    \end{theorem}

    \begin{theorem}[Hahn-Jordan]\index{Hahn-Jordan}
        Consider a signed measure $\mu$ on a measurable space $(X, \Sigma)$. There exists a set $A\in\Sigma$ such that the minimal decomposition $\mu=\mu^+-\mu^-$ in terms of two measures $\mu^\pm$ is given by
        \begin{gather}
            \mu^+(B) = \mu(A\cap B)\qquad\qquad\qquad\mu^-(B)=\mu(A^c\cap B).
        \end{gather}
    \end{theorem}

    \newdef{Integral with respect to a signed measure}{\index{integral!signed measure}
        Let $\mu$ be a signed measure on a measurable space $(X, \Sigma)$ together with a measurable function $f$ on $A\in\Sigma$. The integral of $f$ with respect to $\mu$ is defined as follows:
        \begin{gather}
            \int_Af\ d\mu = \int_Af\ d\mu^+ - \int_Af\ d\mu^-.
        \end{gather}
    }

    \newdef{Lebesgue-Stieltjes signed measure}{
        Let $F$ be a function of bounded variation. According to property \ref{calculus:bounded_variation_decomposition} we can write it as $F=F_1-F-2$ where $F_1, F_2$ are monotonically increasing absolutely continuous functions. The Lebesgue-Stieltjes (signed) measure associated to $F$ is defined as $\mu_F = \mu_{F_1}-\mu_{F_2}$.
    }

    \begin{theorem}[Fundamental theorem of calculus: Lebesgue]\index{fundamental theorem!of calculus}
        Let $F$ be an absolutely continuous function on the closed interval $[a,b]$. Then $F$ is differentiable $m$-a.e. ($m$ being the Lebesgue measure) and its associated  Lebesgue-Stieltjes measure $\mu_F$ has Radon-Nikodym derivative $\deriv{\mu_F}{m}=F'$ $m$-a.e. Furthermore, for all $x\in[a,b]$ one has
        \begin{gather}
            F(x) - F(a) = \mu_F([a,x]) = \int_a^xF'(t)dt.
        \end{gather}
    \end{theorem}
    \begin{result}
        If $F$ is absolutely continuous and $F'=0$ $m$-a.e. then $F$ is constant.
    \end{result}
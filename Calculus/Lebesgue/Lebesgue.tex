\chapter{Measure Theory and Lebesgue Integration}
\label{chapter:lebesgue}

\section{Measures}
\subsection{General definitions}

    \newdef{Measure}{\index{measure}\index{outer!measure}\index{$\sigma$!additivity}\label{lebesgue:measure}
        Let $X$ be a set and let $\Sigma$ be a $\sigma$-algebra over $X$. A function $\mu:\Sigma\rightarrow\overline{\mathbb{R}}$ is called a measure if it satisfies the following conditions:
        \begin{enumerate}
            \item \textbf{Non-negativity}: $\forall E\in\Sigma:\mu(E) \geq0$
            \item \textbf{Measure zero}: $\mu(\emptyset) = 0$
            \item \textbf{Countable additivity}\footnote{This is also called \textbf{$\sigma$-additivity}.} : $\forall i\neq j:E_i\cap E_j=\emptyset\implies\mu\left(\bigcup_{i=1}^\infty E_i\right) = \sum_{i=1}^\infty \mu(E_i)$
        \end{enumerate}
        When $\mu$ only satisfies countable subadditivity, i.e. the equality in the last condition becomes an inequality $\leq$, for any collection of sets (disjoint or not) it is called an \textbf{outer measure}.
    }

    \newdef{Measure space}{\label{lebesgue:measure_space}\index{measurable!set}
        The pair $(X, \Sigma)$ is called a measurable space. The elements $E\in\Sigma$ are called \textbf{measurable sets}. The triplet $(X, \Sigma, \mu)$ is called a measure space.
    }

    \begin{method}
        To show that two measures coincide on a $\sigma$-algebra, it suffices to show that they coincide on the generating sets and apply the monotone class theorem \ref{set:theorem:monotone_class}.
    \end{method}

    \newdef{Almost everywhere\footnotemark}{\label{lebesgue:almost_everywhere}\index{almost everywhere}
        \footnotetext{In probability theory this is often called \textbf{almost surely}.}
        Let $(X, \Sigma, \mu)$ be a measure space. A property $P$ is said to hold on X almost everywhere (a.e.) if it satisfies the following equation:
        \begin{gather}
            \mu\big(\{x\in X:\neg P(x)\}\big) = 0.
        \end{gather}
    }

    \newdef{Complete measure space}{\index{complete!measure space}
        The measure space $(X,\Sigma,\mu)$ is said to be complete if for every $E\in\Sigma$ with $\mu(E) = 0$ the following property holds for all $A\subset E$:
        \begin{gather}
            A\in\Sigma \qquad\text{and}\qquad \mu(A) = 0.
        \end{gather}
    }
    \newdef{Completion}{
        Let $\mathcal{F},\mathcal{G}$ be $\sigma$-algebras over a set $X$. $\mathcal{G}$ is said to be the completion of $\mathcal{F}$ if it is the smallest $\sigma$-algebra such that the measure space $(X,\mathcal{G},\mu)$ is complete.
    }

    \newdef{Borel measure}{\index{Borel!measure}
        Consider a topological space $X$ together with its Borel $\sigma$-algebra $\mathcal{B}$ (see definition \ref{topology:borel_set}). Any measure defined on the measurable space $(X, \mathcal{B})$ is called a Borel measure.
    }
    \newdef{Regular measure}{\label{lebesgue:regular_measure}
        Let $\mu$ be a measure on a measurable space $(X, \Sigma)$. $\mu$ is called a regular measure if it satisifes the following equations for every measurable set $B$:
        \begin{align}
            \mu(B) =& \inf\big\{\mu(O):O \text{ open and measurable}, O\supset B\big\}\\
            \mu(B) =& \sup\big\{\mu(F):F \text{ compact and measurable}, F\subset B\big\}.
        \end{align}
    }
    \newdef{Radon measure}{\index{Radon!measure}\index{locally!finite}
        A regular Borel measure with the additional property that it is \textbf{locally finite}, i.e. every point has a neighbourhood of finite measure. If we restrict ourselves to locally compact Hausdorff spaces then this is equivalent to requiring that every compact subset has finite measure.
    }

    \newdef{\texorpdfstring{$\sigma$-}{sigma-}finite measure}{\index{$\sigma$!finite}\label{lebesgue:sigma_finite_measure}
        Let $(X,\Sigma,\mu)$ be a measure space. The measure $\mu$ is said to be $\sigma$-finite if there exists a sequence $\seq{A}$ of measurable sets such that $\bigcup_{n=1}^{+\infty}A_n = X$ with $\forall A_n:\mu(A_n) < +\infty$.
    }

    \newdef{Measure-preserving map}{
        Let $(X,\Sigma,\mu)$ be a measure space and consider a map $T:X\rightarrow X$. $T$ is said to be measure-preserving if it satisfies the following equation:
        \begin{gather}
            \mu\left(T^{-1}(A)\right) = \mu(A)
        \end{gather}
        for all $A\in\Omega$. This equation can also be written using a pushforward notation: $T_\ast\mu = \mu$.
    }

    \newdef{Ergodic map}{\index{ergodic}
        Let $(X, \Omega)$ be a measure space. Consider a measure-preserving map $T:X\rightarrow X$. $T$ is said to be ergodic if the following conditions is satisfied:
        \begin{gather}
            T(A) = A\implies \mu(A) = 0 \lor \mu(X\backslash A) = 0.
        \end{gather}
    }

\subsection{Lebesgue measure}

    \newdef{Null set}{\index{null set}
        A set $A\subset\mathbb{R}$ is called a null set if it can be covered by a sequence of intervals of arbitrarily small length, i.e. $\forall\varepsilon>0$ there exists a sequence $\seq{I}$ such that
        \begin{gather}
            A \subseteq \bigcup_{n=1}^{+\infty}I_n
        \end{gather}
        and
        \begin{gather}
            \sum_{i=1}^{+\infty}l(I_n) < \varepsilon.
        \end{gather}
    }

    \begin{property}
        Let $(E_i)_{i\in\mathbb{N}}$ be a sequence of null sets. The union $\bigcup_{i=1}^{+\infty}E_i$ is also null.
    \end{property}
    \begin{result}\label{lebesgue:theorem:countable_set_is_null}
        Any countable set is null.
    \end{result}

    \newdef{Lebesgue outer measure}{\index{Lebesgue!outer measure}\label{lebesgue:outer_measure}
        Let $X\subseteq\mathbb{R}$ be a set. The (Lebesgue) outer measure of $X$ is defined as follows:
        \begin{gather}
            m^*(X) := \inf\left\{\sum_{i=1}^{+\infty} l(I_i)\text{ with }(I_i)_{i\in\mathbb{N}} \text{ a sequence of open intervals that covers }X\right\}.
        \end{gather}
    }

    \begin{property}
        Let $I$ be an interval. The outer measure equals the length: $m^*(I) = l(I)$.
    \end{property}
    \begin{property}
        The outer measure is translation-invariant: $m^*(A + t) = m^*(A)$ for all $A,t$.
    \end{property}
    \begin{property}
        The Lebesgue outer measure is an outer measure in the sense of definition \ref{lebesgue:measure}.
    \end{property}

    \begin{theorem}[Carath\'eodory's criterion]\index{Carath\'eodory!criterion}\index{Lebesgue!measure}\index{measurable!set}\label{lebesgue:lebesgue_measure}
        Let $X$ be a subset of $\mathbb{R}$. If $X$ satisfies the following equation then it is said to be \textbf{Lebesgue measurable}:
        \begin{gather}
            \forall E\subseteq\mathbb{R}:m^*(E) = m^*(E\cap X) + m^*(E\cap X^c).
        \end{gather}
        This is denoted by $X\in\mathcal{M}$ and the outer measure $m^*(X)$ is called the Lebesgue measure of $X$. It is denoted by $m(X)$.
    \end{theorem}
    \begin{property}
        All null sets and intervals are measurable.
    \end{property}
    \newprop{Countable additivity}{\index{countable!additivity}
        For every sequence $(E_i)_{i\in\mathbb{N}}$ with $E_i\in\mathcal{M}$ satisfying $i\neq j:E_i\cap E_j = \emptyset$ the following equation holds:
        \begin{gather}
            m\left(\bigcup_{i=1}^{+\infty}E_i\right) = \sum_{i=1}^{+\infty}m(E_i).
        \end{gather}
    }
    \begin{remark}
       Previous property, together with the properties of the outer measure, implies that the Lebesgue measure is indeed a proper measure as defined in \ref{lebesgue:measure}. Furthermore, $\mathcal{M}$ is a $\sigma$-algebra\footnote{See definition \ref{set:sigma_algebra}.} over $\mathbb{R}$.
    \end{remark}

    \begin{property}
        For every $A\subset\mathbb{R}$ there exists a sequence $(O_i)_{i\in\mathbb{N}}$ of open sets such that
        \begin{gather}
            \label{lebesgue:theorem:open_cover_existence}
            A\subset\bigcap_iO_i\qquad\text{and}\qquad m\left(\bigcap_iO_i\right) = m^*(A).
        \end{gather}
    \end{property}
    \begin{property}
        For every $E\in\mathcal{M}$ there exists a sequence $(F_i)_{i\in\mathbb{N}}$ of closed sets such that
        \begin{gather}
            \label{lebesgue:theorem:closed_cover_existence}
            \bigcup_iF_i\subset E\qquad\text{and}\qquad m\left(\bigcup_iF_i\right) = m(E).
        \end{gather}
    \end{property}
    \sremark{The previous 2 theorems imply that the Lebesgue measure is a regular Borel measure (see definition \ref{lebesgue:regular_measure}).}

    \begin{property}
        Let $E\subset\mathbb{R}$. $E\in\mathcal{M}$ if and only if for every $\varepsilon>0$ there exist an open set $O\supset E$ and a closed set $F\subset E$ such that $m^*(O\backslash E) < \varepsilon$ and $m^*(E\backslash F)<\varepsilon$.
    \end{property}

    \begin{property}
        Let $(A_i)_{i\in\mathbb{N}}$ be a sequence of sets with $\forall i:A_i\in\mathcal{M}$. The following two properties apply:
        \begin{align}
            \forall i: A_i\subseteq A_{i+1} &\implies m\left(\bigcup_{i=1}^{+\infty}A_i\right) = \lim_{i\rightarrow+\infty}m(A_i)\\
            \forall i: A_i\supseteq A_{i+1} \land m(A_1)<+\infty &\implies m\left(\bigcap_{i=1}^{+\infty}A_i\right) = \lim_{i\rightarrow+\infty}m(A_i).
        \end{align}
    \end{property}
    \remark{This property is not only valid for the Lebesgue measure but for every countably additive set function.}
    \begin{property}[Continuity]
        The Lebesgue measure $m(X)$ is continuous at $\emptyset$, i.e. if $\lim_{i\rightarrow\infty}A_i=\emptyset$ then $\lim_{i\rightarrow+\infty}m(A_i) = 0$.
    \end{property}

    \begin{property}[Relation between Lebesgue and Borel algebras]\label{lebesgue:completion_remark}
        The Lebesgue $\sigma$-algebra $\mathcal{M}$ is the completion of the Borel $\sigma$-algebra $\mathcal{B}$. (This is in fact how the Lebesgue $\sigma$-algebra was introduced historically.)
    \end{property}

    \begin{construct}[Restriction]\index{Lebesgue!restricted measure}\label{lebesgue:restricted_lebesgue_measure}
        Let $B\subset\mathbb{R}$ be a measurable set with measure $m(B)>0$. The restriction of the Lebesgue measure to the set $B$ is defined as follows:
        \begin{gather}
            \mathcal{M}_B := \left\{A\cap B:A\in\mathcal{M}\right\}\qquad\text{and}\qquad\forall E\in\mathcal{M}_B:m_B(E) := m(E).
        \end{gather}
        Furthermore, the measure space $(B,\mathcal{M}_B,m_B)$ is complete.
    \end{construct}

\subsection{Measurable functions}

    \newdef{Measurable function}{\index{measurable!function}\label{lebesgue:measurable_function}
        A function $f$ is (Lebesgue) measurable if for every interval $I\subset\mathbb{R}:f^{-1}(I)\in\mathcal{M}$.
    }
    \newdef{Borel measurable function}{\index{Borel!measurable function}\label{lebesgue:borel_measurable_function}
        A function $f$ is called Borel measurable\footnote{These functions are often simply called \textbf{Borel functions}.} if for every interval $I\subset\mathbb{R}:f^{-1}(I)\in\mathcal{B}$.
    }
    \remark{The inclusion $\mathcal{B}\subset\mathcal{M}$ implies that every Borel-measurable function is also Lebesgue-measurable.}

    \begin{property}
        The class of Borel/Lebesgue measurable functions defined on $E\in\mathcal{M}$ forms an algebra.
    \end{property}

    \begin{example}
        Following types of functions are measurable:
        \begin{itemize}
            \item monotonic functions
            \item continuous functions
            \item indicator functions.
        \end{itemize}
    \end{example}
    \begin{result}
        Let $f,g$ be measurable functions and let $F:\mathbb{R}\times\mathbb{R}\rightarrow\mathbb{R}$ be a continuous function. The composition $F(f(x), g(x))$ is also measurable.
    \end{result}

    \begin{property}
        Let $f$ be a measurable function. The level set $\{x:f(x) = a\}$ is measurable for all $a\in\mathbb{R}$.
    \end{property}

    \begin{property}
        Define following functions, which are measurable if $f$ is measurable as a result of previous properties:
        \begin{gather}
            \label{lebesgue:positive_part}
            f^+(x) = \max(f,0) =
            \begin{cases}
                f(x)&f(x)>0\\
                0&f(x)\leq0
            \end{cases}
        \end{gather}
        \begin{gather}
            \label{lebesgue:negative_part}
            f^-(x) = \max(-f,0) =
            \begin{cases}
                0&f(x)>0\\
                -f(x)&f(x)\leq0.
            \end{cases}
        \end{gather}
        The function $f:E\rightarrow\mathbb{R}$ is measurable if and only if both $f^+$ and $f^-$ are measurable. Furthermore, $f$ is measurable if $|f|$ is measurable (the converse is false).
    \end{property}

\subsection{Limit operations}

    \begin{property}
        Let $\seq{f}$ be a sequence of Borel/Lebesgue measurable functions. The following functions are also measurable:
        \begin{itemize}
            \item $\ds\min_{i\leq k}(f_i)$ and $\ds\max_{i\leq k}(f_i)$
            \item $\ds\inf_{i\in\mathbb{N}}(f_i)$ and $\ds\sup_{i\in\mathbb{N}}(f_i)$
            \item $\ds\liminf_{i\rightarrow+\infty}(f_i)$ and $\ds\limsup_{i\rightarrow+\infty}(f_i)$
        \end{itemize}
    \end{property}

    \begin{property}
        Let $f$ be a measurable function and let $g$ be a function such that $f=g$ almost everywhere. Then $g$ is measurable as well.
    \end{property}
    \result{A result of the previous two properties is the following: if a sequence of measurable functions converges pointwise a.e. then the limit is also a measurable function.}

    \newdef{Essential supremum}{\index{essential!supremum}\label{lebesgue:essential_supremum}
        \begin{gather}
            \esssup(f) := \sup\{z:f\geq z\text{ a.e.}\}
        \end{gather}
    }
    \newdef{Essential infimum}{\index{essential!infimum}\label{lebesgue:essential_infimum}
        \begin{gather}
            \essinf(f) := \inf\{z:f\leq z\text{ a.e.}\}
        \end{gather}
    }
    \begin{property}
        Every measurable function $f$ satisfies the following inequalities:
        \begin{itemize}
            \item $f\leq\esssup(f)\text{ a.e.}$ and $f\geq\essinf(f)\text{ a.e.}$
            \item $\esssup(f)\leq\sup(f)$ and $\essinf(f)\geq\inf(f)$.
        \end{itemize}
        The latter pair of inequalities becomes a pair of equalities if $f$ is continuous.
    \end{property}
    \begin{property}
        If $f,g$ are measurable functions then $\esssup(f+g)\leq\esssup(f) + \esssup(g)$. An analogous inequality holds for the essential infimum.
    \end{property}
\section{Lebesgue integral}
\subsection{Simple functions}

    \newdef{Indicator function}{\index{indicator function}
        An important function when working with sets is the following one:
        \begin{gather}
            \label{lebesgue:indicator_function}
    	    \mathbbm{1}_A(x) =
            \begin{cases}
	            1&x\in A\\
                0&x\not\in A
	        \end{cases}
	    \end{gather}
    }
    \newdef{Simple function}{\index{simple!function}\label{lebesgue:simple_function}
            Let $f$ be a function that takes on a finite number of non-negative values $\{a_i\}$ with for every $i\neq j: f^{-1}(a_i)\cap f^{-1}(a_j) = \emptyset$. $f$ is called a simple function if it can be expanded in the following way:
            \begin{gather}
                f(x) = \sum_{i=1}^n a_i\mathbbm{1}_{A_i}(x)
            \end{gather}
            with $A_i = f^{-1}(a_i)\in\mathcal{M}$.
    }
    \begin{definition}[Step function]\index{step function}\label{lebesgue:step_function}
        If the sets $A_i$ are intervals, the above function is often called a step function.
    \end{definition}

    \newformula{Lebesgue integral of simple functions}{\index{Lebesgue!integral}\label{lebesgue:integral_simple_function}
        Consider a simple function $\varphi$. Furthermore, let $\mu:\mathcal{M}\rightarrow\mathbb{R}$ be the Lebesgue measure and let $E$ be a measurable set. The Lebesgue integral of $\varphi$ over a $E$ with respect to $\mu$ is given by
        \begin{gather}
            \int_E\varphi\ d\mu = \sum_{i=1}^na_i\mu(E\cap A_i).
        \end{gather}
    }
    \begin{example}
        Let $\mathbbm{1}_\mathbb{Q}$ be the indicator function of the set of rational numbers. This function is clearly a simple function. Previous formula makes it possible to integrate the rational indicator function over the real line (which is not possible in the sense of Riemann):
        \begin{gather}
            \int_\mathbb{R}\mathbbm{1}_\mathbb{Q}\ d\mu = 1\times m(\mathbb{Q}) + 0\times m(\mathbb{R}\backslash\mathbb{Q}) = 0
        \end{gather}
        where the measure of the rational numbers is 0 because it is a countable set (see corollary \ref{lebesgue:theorem:countable_set_is_null}).
    \end{example}

\subsection{Measurable functions}

    \newformula{Lebesgue integral for non-negative functions}{\index{Lebesgue!integral}\label{lebesgue:integral}
        The definition for simple functions can be geenralized to non-negative measurable functions $f$ as follows:
        \begin{gather}
            \int_Ef\ d\mu = \sup\left\{\int_E\varphi\ d\mu:\varphi \text{ a simple function such that } \varphi\leq f\right\}.
        \end{gather}
        This integral is always non-negative.
    }

    \begin{formula}
        The following equality follows directly from $A\subseteq\mathbb{R}:A\cup A^c = \mathbb{R}$:
            \begin{gather}
                \label{lebesgue:interchanging_domains_with_indicator_function}
                \int_Af\ d\mu = \int f\mathbbm{1}_A\ d\mu.
            \end{gather}
    \end{formula}

    \begin{property}
        Let $f$ be a non-negative measurable function. Then $f=0$ a.e. if and only if $\int_\mathbb{R}f\ d\mu = 0$.
    \end{property}

    \begin{property}
        The Lebesgue integral over a null set is 0.
    \end{property}
    \begin{property}\label{lebesgue:general_properties}
        Let $f,g$ be measurable functions. The Lebesgue integral has the following properties:
        \begin{itemize}
            \item $f\leq g$ a.e. implies $\int f d\mu\leq\int g\ d\mu$.
            \item Let $A$ be a measurable set and consider a subset $B\subset A$. Then $\int_Bf\ d\mu\leq\int_Af\ d\mu$.
            \item The Lebesgue integral is linear.
            \item For every two disjoint measurable sets $A$ and $B$ we have that $\int_{A\cup B}f\ d\mu = \int_Af\ d\mu + \int_Bf\ d\mu$.
        \end{itemize}
    \end{property}

    \begin{theorem}[Mean value theorem]\index{mean!value theorem}
        If $a\leq f(x)\leq b$, then $am(A)\leq\int_Af\ d\mu\leq bm(A)$.
    \end{theorem}

    \begin{property}
        Let $f$ be a non-negative measurable function. There exists an increasing sequence $(\varphi_i)_{i\in\mathbb{N}}$ of simple functions such that $\varphi_i\nearrow f$.
    \end{property}
    \begin{property}
        Let $f$ be a bounded measurable function defined on the interval $[a,b]$. For every $\varepsilon>0$ there exists a step function\footnote{See definition \ref{lebesgue:step_function}.} $h$ such that $\int_a^b|f-h|\ d\mu<\varepsilon$.
    \end{property}

\subsection{Integrable functions}

    \newdef{Integrable function}{\index{integrable}\label{lebesgue:integrable_function}
        Let $E\in\mathcal{M}$. A measurable function $f$ is said to be integrable over $E$ if both $\int_Ef^+\ d\mu$ and $\int_Ef^-\ d\mu$ are finite. The Lebesgue integral of $f$ over $E$ is then defined as
        \begin{gather}
            \int_E f\ d\mu = \int_E f^+\ d\mu - \int_E f^-\ d\mu.
        \end{gather}
    }
    \sremark{The difference between the integral in definition \ref{lebesgue:integral} and the integral of an integrable function is that with the latter $f$ does not have to be non-negative.}

    \begin{property}
        $f$ is integrable if and only if $|f|$ is integrable. Furthermore, $\int_E|f|\ d\mu = \int_E f^+\ d\mu + \int_E f^-\ d\mu$.
    \end{property}
    \begin{property}
        Let $f,g$ be integrable functions. The following important properties apply:
        \begin{itemize}
            \item $f+g$ is also integrable.
            \item $\forall E\in\mathcal{M}, \int_Ef\ d\mu\leq\int_Eg\ d\mu\implies f\leq g$ a.e.
            \item Let $c\in\mathbb{R}$. $\int_E(cf)\ d\mu = c(\int_Ef\ d\mu)$.
            \item $f$ is finite a.e.
            \item $|\int f\ d\mu|\leq\int|f|\ d\mu$
            \item $f\geq0\land\int f\ d\mu=0\implies f=0$ a.e.
        \end{itemize}
    \end{property}

    \begin{definition}[Lebesgue integrable functions]
        The set of functions integrable over a set $E\in\mathcal{M}$ forms the vector space $\mathcal{L}^1(E)$.
    \end{definition}

    \begin{property}
        Let $f\in\mathcal{L}^1$ and $\varepsilon>0$. There exists a continuous function $g$, vanishing outside some finite interval, such that $\int|f-g|\ d\mu<\varepsilon$.
    \end{property}

    \begin{property}\index{continuity!absolute continuity}\label{lebesgue:theorem:measure_by_integral}
        Let $f\geq0$. The mapping $E\mapsto\int_Ef\ d\mu$ is a measure on $\mathcal{M}$ (if it exists, i.e. if $f$ is integrable). Furthermore, this measure is said to be \textbf{absolutely continuous}. (See section \ref{lebesgue:section:Radon-Nikodym} for further information.)
    \end{property}

    \newdef{Locally integrable function}{\index{locally!integrable}
        A measurable function is said to be locally integrable if it is integrable on every compact subset of its domain. The space of locally integrable functions is denoted by $\mathcal{L}^1_{loc}$.
    }

\subsection{Convergence theorems}

    \begin{theorem}[Fatou's lemma]\index{Fatou}\label{lebesgue:theorem:fatous_lemma}
        Let $(f_n)_{n\in\mathbb{N}}$ be a sequence of non-negative measurable functions.
        \begin{gather}
            \int_E\left(\liminf_{n\rightarrow\infty}f_n\right)\ d\mu \leq \liminf_{n\rightarrow\infty}\int_Ef_n\ d\mu
        \end{gather}
    \end{theorem}
    \begin{theorem}[Monotone convergence theorem]\index{monotone!convergence theorem}\label{lebesgue:theorem:monotone_convergence_theorem}
        Let $E\in\mathcal{M}$ and let $(f_n)_{n\in\mathbb{N}}$ be an increasing sequence of non-negative measurable functions such that $f_n\nearrow f$ pointwise a.e. We have the following powerful equality:
        \begin{gather}
            \int_E f\ d\mu = \lim_{n\rightarrow\infty}\int_E f_n(x)\ d\mu.
        \end{gather}
    \end{theorem}

    \begin{method}\label{lebesgue:method:linear_proofs}
        To prove results concerning integrable functions in spaces such as $\mathcal{L}^1(E)$ it is often useful to proceed as follows:
        \begin{enumerate}
            \item Verify that the property holds for indicator functions. (This often follows by definition.)
            \item Use the linearity to extend the property to simple functions.
            \item Apply the monotone convergence theorem to show that the property holds for all non-negative measurable functions.
            \item Extend the property to all integrable functions by expanding $f = f^+ - f^-$ and applying linearity again.
        \end{enumerate}
    \end{method}

    \begin{theorem}[Dominated convergence theorem]\index{dominated convergence theorem}\label{lebesgue:theorem:dominated_convergence_theorem}
        Let $E\in\mathcal{M}$ and let $(f_n)_{n\in\mathbb{N}}$ be a sequence of measurable functions with $\forall n:|f_n|\leq g$ a.e. for some function $g\in\mathcal{L}^1(E)$. If $f_n\rightarrow f$ pointwise a.e. then $f$ is integrable over $E$ and
        \begin{gather}
            \int_E f\ d\mu = \lim_{n\rightarrow\infty}\int_E f_n(x)\ d\mu.
        \end{gather}
    \end{theorem}

    \begin{property}[Interch]
        Let $(f_n)_{n\in\mathbb{N}}$ be a sequence of non-negative measurable functions. The following equality applies:
        \begin{gather}
            \int\sum_{n=1}^{+\infty}f_n(x)\ d\mu = \sum_{n=1}^{+\infty}\int f_n(x)\ d\mu.
        \end{gather}
        We cannot conclude that the right-hand side is finite a.e., so the series on the left-hand side need not be integrable.
    \end{property}

    \begin{theorem}[Beppo Levi\footnotemark]\index{Beppo Levi}\label{lebesgue:theorem:beppo_levi}
        \footnotetext{Note that various other theorems and variants can be found in the literature under the same name.}
        Suppose that \[\sum_{i=1}^\infty\int|f_n|(x)\ d\mu\text{ is finite.}\] The series $\sum_{i=1}^\infty f_n(x)$ converges a.e. Furthermore, the series is integrable and
        \begin{gather}
            \int\sum_{i=1}^\infty f_n(x)\ d\mu = \sum_{i=1}^\infty\int f_n(x)\ d\mu.
        \end{gather}
    \end{theorem}

    \begin{theorem}[Riemann-Lebesgue lemma]\index{Riemann!Riemann-Lebesgue lemma}\label{lebesgue:riemann_lebesue_lemma}
        Let $f\in\mathcal{L}^1$. The sequences \[s_k = \int_{-\infty}^{+\infty}f(x)\sin(kx)dx\] and \[c_k = \int_{-\infty}^{+\infty}f(x)\cos(kx)dx\] both converge to 0.
    \end{theorem}

\subsection{Relation to the Riemann integral}

    \begin{property}
        Let $f:[a,b]\rightarrow\mathbb{R}$ be a bounded function.
        \begin{itemize}
            \item $f$ is Riemann-integrable if and only if $f$ is continuous a.e. with respect to the Lebesgue measure on $[a,b]$, i.e. the set of discontinuities of $f$ has measure zero.
            \item Riemann-integrable functions on $[a,b]$ are integrable with respect to the Lebesgue measure on $[a,b]$ and the integrals coincide.
        \end{itemize}
    \end{property}

    \begin{property}
        If $f\geq0$ and the improper Riemann integral \ref{calculus:improper_integral} exists, then the Lebesgue integral $\int f\ d\mu$ exists and the two integrals coincide.
    \end{property}

\section{Examples}
    	
    	\newdef{Dirac measure\footnotemark}{\index{Dirac}
        	\footnotetext{Compare to definition \ref{distribution:dirac_delta}. }
        	We define the Dirac measure as follows:
        	\begin{equation}
			\label{lebesgue:dirac_measure}
        	        \delta_a(X) = \left\{
        	        \begin{array}{cc}
                		1&\text{if } a\in X\\
        	        	0&\text{if } a\not\in X\\
                	\end{array}\right.
		\end{equation}
	        The integration with respect to the Dirac measure has the following nice property\footnote{This equality can be proved by applying formula \ref{prop:change_of_variable} with $X\equiv a$.}:
	        \begin{equation}
			\int g(x)d\delta_a = g(a)
		\end{equation}
        }
        \begin{example}
		Let $\mu=\delta_2, X = (-4;1)$ and $Y = (-2;17)$. The following two integrals are easily computed:
	        \[\int_Xd\mu = 0\qquad\qquad\qquad \int_Yd\mu = 1\]
	\end{example}

\section{Space of integrable functions}
\subsection{Distance}\index{distance}
	To define a distance between functions, we first have to define some notion of length of a function. Normally this would not be a problem, because we know how to integrate integrable functions, however the fact that two functions differing on a null set have the same integral carries problems with it, i.e. a nonzero function could have a zero length. Therefore we will define the 'length' function on a restricted vector space:\par
    
    \noindent Define the following set of equivalence classes $L^1(E) = \mathcal{L}^1(E)_{/\equiv}$ by introducing the equivalence relation: $f\equiv g$ if and only if $f=g$ a.e.
    \begin{property}
		$L^1(E)$ is a Banach space\footnotemark.
	\end{property}
    \footnotetext{See definition \ref{linalgebra:banach_space}.}
    
    \begin{formula}
		A norm on $L^1(E)$ is given by:
        \begin{equation}
			\label{lebesgue:L1_norm}
            ||f||_1 = \int_E |f|dm
		\end{equation}
	\end{formula}
    
\subsection{Hilbert space \texorpdfstring{$L^2$}{L2}}\index{Hilbert!space|see{L$^2$}}
	\label{lebesgue:section:hilbert_space}
    
    \begin{property}
    	\label{lebesgue:L2_hilbert_space}
		$L^2$ is a Hilbert space\footnotemark.
	\end{property}
    \footnotetext{See definition \ref{hilbert:hilbert_space}.}
	\begin{formula}
		A norm on $L^2(E)$ is given by:
        \begin{equation}
			\label{lebesgue:L2_norm}
            ||f||_2 = \left(\int_E |f|^2dm\right)^{\frac{1}{2}}
		\end{equation}
        This norm is induced by the following inner product:
        \begin{equation}
			\label{lebesgue:L2_inner_product}
            \boxed{\langle f|g \rangle = \int_E f\overline{g}dm}
		\end{equation}
	\end{formula}
    Now instead of deriving $L^2$ from $\mathcal{L}^2$ we do the opposite. We define $\mathcal{L}^2$ as the set of measurable functions for which equation \ref{lebesgue:L2_norm} is finite.
    
    \newdef{Orthogonality}{\index{orthogonal}
    	As $L^2$ is a Hilbert space and thus has an inner product $\langle\cdot|\cdot\rangle$, it is possible to introduce the concept of orthogonality of functions in the following way:
        \begin{equation}
			\label{lebesgue:orthogonal_functions}
            \langle f|g \rangle = 0\implies\text{f and g are orthogonal}
		\end{equation}
        Furthermore it is also possible to introduce the angle between functions in the same way as equation \ref{linalgebra:angle}.
    }
    
    \begin{formula}[Cauchy-Schwarz inequality]\index{Cauchy-Schwarz}
		Let $f,g\in L^2(E,\mathbb{C})$. We have that $fg\in L^1(E\mathbb{C})$ and:
        \begin{equation}
			\label{lebesgue:schwarz_inequality}
            \boxed{\left|\int_E f\overline{g}dm\right|\leq||fg||_1\leq||f||_2||g||_2}
		\end{equation}
	\end{formula}
    \sremark{This follows immediately from formula \ref{lebesgue:holders_inequality}.}
    
    \begin{property}
		If $E$ has finite Lebesgue measure then $L^2(E)\subset L^1(E)$.
	\end{property}
    
\subsection{\texorpdfstring{$L^p$}{Lp} spaces}
	Generalizing the previous two Lebesgue function classes leads us to the notion of $L^p$ spaces with the following norm:
    
    \begin{property}For all $1\leq p\leq+\infty$ $L^p(E)$ is a Banach space with a norm given by:
    	\begin{equation}
			\label{lebesgue:Lp_norm}
            ||f||_p = \left(\int_E |f|^p\ dm\right)^{\frac{1}{p}}
		\end{equation}
    \end{property}
    \remark{Note that $L^2$ is the only $L^p$ space that is also a Hilbert space. The other $L^p$ spaces do not have a norm induced by an inner product.}
    
    \newformula{H\"{o}lder's inequality}{\index{H\"older's inequality}
    	Let $\frac{1}{p} + \frac{1}{q} = 1$ with $p\geq1$. For every $f\in L^p(E)$ and $g\in L^q(E)$ we have that $fg\in L^1(E)$ and:
        \begin{equation}
        	\label{lebesgue:holders_inequality}
			||fg||_1\leq||f||_p||g||_q
		\end{equation}
    }
    \newformula{Minkowski's inequality}{\index{Minkowski!inequality}
    	For every $p\geq1$ and $f,g\in L^p(E)$ we have
        \begin{equation}
			\label{lebesgue:minkowskis_inequality}
            ||f+g||_p\leq||f||_p + ||g||_p
		\end{equation}
    }
    \begin{property}
		If $E$ has finite Lebesgue measure then $L^q(E)\subset L^p(E)$ when $1\leq p\leq q<+\infty$.
	\end{property}
    
\subsection{\texorpdfstring{$L^\infty$}{L-infinity} space of essentially bounded measurable functions}
	\newdef{Essentially bounded function}{
    	Let $f$ be a measurable function satisfying $\esssup |f| <+\infty$. The function $f$ is said to be essentially bounded and the set of all such functions is denoted by $L^\infty(E)$.
    }
    
    \begin{formula}\index{supremum}
		A norm on $L^\infty$ is given by:
        \begin{equation}
			||f||_\infty = \esssup|f|
		\end{equation}	
        This norm is called the \textbf{supremum norm} and it induces the supremum metric \ref{topology:supremum_distance}.
	\end{formula}
    \begin{property}
		$L^\infty$ is a Banach space.
	\end{property}
	
\section{Product measures}
\subsection{Real hyperspace \texorpdfstring{$\mathbb{R}^n$}\ }
	
    The notions of intervals and lengths from the one dimensional case can be generalized to more dimensions in the following way:
    \newdef{Hypercube}{\index{hypercube}
    	Let $I_1, ..., I_n$ be a sequence of intervals.
    	\begin{equation}
			\mathbf{I} = I_1\times...\times I_n
		\end{equation}
    }
    \newdef{Generalized length}{
    	Let $\mathbf{I}$ be a hypercube induced by the sequence of intervals $I_1,...,I_n$. The length of $\mathbf{I}$ is given by:
        \begin{equation}
			l(\mathbf{I}) = \prod_{i=1}^{n}l(I_i)
		\end{equation}
    }
    
\subsection{Construction of the product measure}

    \newprop{General condition}{
    	The general condition for multi-dimensional\newline Lebesgue measures is given by following equation which should hold for all $A_1\in\mathcal{F}_1$ and $A_2\in\mathcal{F}_2$:
    	\begin{equation}
        	\label{lebesgue:product_measure:general_condition}
			\boxed{P(A_1\times A_2) = P_1(A_1)P_2(A_2)}
		\end{equation}
    }

	\newdef{Section}{\index{section}
    	Let $A=A_1\times A_2$. The following two sets are called sections:
        \[
        	A_{\omega_1} = \{\omega_2\in\Omega_2:(\omega_1,\omega_2)\in A\}\subset\Omega_2
        \]
        \[
        	A_{\omega_2} = \{\omega_1\in\Omega_1:(\omega_1,\omega_2)\in A\}\subset\Omega_1
        \]
    }
    \begin{property}
		Let $\mathcal{F} = \mathcal{F}_1\times\mathcal{F}_2$. If $A\in\mathcal{F}$ then for each $\omega_1$, $A_{\omega_1}\in\mathcal{F}_2$ and for each $\omega_2$, $A_{\omega_2}\in\mathcal{F}_1$. Equivalently the sets $\mathcal{G}_1 = \{A\in\mathcal{F}:\forall \omega_1,A_{\omega_1}\in\mathcal{F}_2\}$ and $\mathcal{G}_2 = \{A\in\mathcal{F}:\forall \omega_2, A_{\omega_2}\in\mathcal{F}_1\}$ coincide with the product $\sigma$-algebra $\mathcal{F}$.
	\end{property}
    
    \begin{property}
		The function $A_{\omega_2}\mapsto P(A_{\omega_2})$ is a step function:
        \[
        	P(A_{\omega_2}) = \left\{
            \begin{array}{ccc}
				P_1(A_1)&\text{if}&\omega_2\in A_2\\
                0&\text{if}&\omega_2\not\in A_2
			\end{array}
            \right.
        \]
	\end{property}
    
    \begin{formula}[Product measure]\index{product!measure}
		From previous property it follows that we can write the product measure $P(A)$ in the following way:
        \begin{equation}
			\boxed{P(A) = \int_{\Omega_2} P_1(A_{\omega_2})dP_2(\omega_2)}
		\end{equation}
	\end{formula}
    \begin{property}
		Let $P_1, P_2$ be finite. If $A\in\mathcal{F}$ then the functions
        \[
        	\omega_1\mapsto P_2(A_{\omega_1}) \qquad\qquad \omega_2\mapsto P_1(A_{\omega_2})
        \]
        are measurable with respect to $\mathcal{F}_1$ and $\mathcal{F}_2$ respectively and
        \begin{equation}
			\boxed{\int_{\Omega_2} P_1(A_{\omega_2})dP_2(\omega_2) = \int_{\Omega_1} P_2(A_{\omega_1})dP_1(\omega_1)}
		\end{equation}
        Furthermore the set function $P$ is countably additive and if any other product measure coincides with $P$ on all rectangles, it is equal to $P$ on the whole product $\sigma$-algebra.
	\end{property}

    
\subsection{Fubini's theorem}
	\begin{property}
		Let $f:\Omega_1\times\Omega_2\rightarrow\mathbb{R}$ be a non-negtaive function. If $f$ is measurable with respect to $\mathcal{F}_1\times\mathcal{F}_2$ then for each $\omega_1\in\Omega_1$ the function $\omega_2\mapsto f(\omega_1,\omega_2)$ is measurable with respect to $\mathcal{F}_2$ (and vice versa). There integrals with respect to $P_1$ and $P_2$ respectively are also measurable.
	\end{property}
    \newdef{Section of a function}{\index{section}
    	The functions $\omega_1\mapsto f(\omega_1,\omega_2), \omega_2\mapsto f(\omega_1,\omega_2)$ are called sections of $f$.
	}
    
    \begin{theorem}[Tonelli's theorem]\index{Tonelli}
		Let $f:\Omega_1\times\Omega_2\rightarrow\mathbb{R}$ be a non-negative function. The following equalities apply:
        \begin{equation}
        	\label{lebesgue:tonelli_theorem}
        	\begin{split}
			\int_{\Omega_1\times\Omega_2}f(\omega_1,\omega_2)d(P_1\times P_2)(\omega_1,\omega_2) = \int_{\Omega_1}\left(\int_{\Omega_2}f(\omega_1,\omega_2)dP_2(\omega_2)\right)dP_1(\omega_1)\\ = \int_{\Omega_2}\left(\int_{\Omega_1}f(\omega_1,\omega_2)dP_1(\omega_1)\right)dP_2(\omega_2)
            \end{split}
		\end{equation}
	\end{theorem}
    
    \begin{result}[Fubini's theorem]\index{Fubini's theorem}
		Let $f\in L^1(\Omega_1\times\Omega_2)$. The sections are integrable in the appropriate spaces. Furthermore the functions $\omega_1\mapsto\int_{\Omega_2} fdP_2$ and $\omega_2\mapsto\int_{\Omega_1}fdP_1$ are in $L^1(\Omega_1)$ and $L^1(\Omega_2)$ respectively and equality \ref{lebesgue:tonelli_theorem} holds.
	\end{result}
    \remark{The previous construction and theorems also apply for higher dimensional product spaces. These thereoms provide a way to construct higher-dimensional Lebesgue measures $m_n$ by defining them as the completion of the product of $n$ one-dimensional Lebesgue measures.}

\section{Radon-Nikodym theorem}\label{lebesgue:section:Radon-Nikodym}

    \begin{definition}\index{continuity!absolute continuity}\label{lebesgue:absolute_continuity}
        Let $(X,\Sigma)$ be a measurable space and let $\mu, \nu$ be two measures defined on this space. Then $\nu$ is said to be \textbf{absolutely continuous} with respect to $\mu$ if
        \begin{gather}
            \forall A\in\Sigma: \mu(A) = 0\implies\nu(A) = 0.
        \end{gather}
    \end{definition}
    \begin{notation}
        This relation is often denoted by $\nu\ll\mu$.
    \end{notation}

    The following property relates the notion of absolute continuity above with that of definition \ref{calculus:absolute_continuity}:
    \begin{property}[Absolute continuity]
        Let $\mu, \nu$ be finite measures on a measurable space $(X, \Sigma)$. Then $\nu\ll\mu$ if and only if
        \begin{gather}
            \forall\varepsilon>0:\exists\delta>0:\forall A\in\Sigma:\mu(A)<\delta\implies\nu(A)<\varepsilon.
        \end{gather}
    \end{property}

    \newdef{Singular measures}{\index{measure!singular}\index{orthogonal!measure|see{measure, singular}}
        Consider two measures $\mu,\nu$. If there exists a set $A$ such that $\mu(A)=0=\nu(A^c)$ then they are said to be singular (or \textbf{orthogonal}). This is denoted by $\mu\perp\nu$.
    }
    \begin{theorem}[Lebesgue's decomposition theorem]
        Consider two $\sigma$-finite measures $\mu,\nu$. There exist two other $\sigma$-finite measures $\nu_a, \nu_s$ such that $\nu=\nu_a+\nu_s$ where $\nu_a\ll\mu$ and $\nu_s\perp\mu$.
    \end{theorem}

    Property \ref{lebesgue:theorem:measure_by_integral} can be generalized to arbitrary measure spaces as follows:
    \begin{property}
        Let $(X,\Sigma, \mu)$ be a measure space. Let $f:X\rightarrow\mathbb{R}$ be a measurable function such that $\int f\ d\mu$ exists. Then $\nu(f) = \int_Ffd\ \mu$ defines an absolutely continuous measure $\nu\ll\mu$.
    \end{property}

    \begin{definition}[Dominated measure]\index{measure!dominated}
            Let $\mu, \nu$ be two measures defined on a measurable space $(X, \Sigma)$. Then $\mu$ is said to \textbf{dominate} $\nu$ if $0\leq\nu(F)\leq\mu(F)$ for every $F\in\Sigma$.
    \end{definition}

    \begin{theorem}[Radon-Nikodym theorem for dominated measures]\index{Radon-Nikodym}~\newline
        Let $\mu$ be a measure on $(X, \Sigma)$ such that $\mu(X) = 1$ and let $\nu$ be a measure dominated by $\mu$. There exists a non-negative $\Sigma$-measurable function $h$ such that $\nu(F) = \int_F h\ d\mu$ for all $F\in\Sigma$.
    \end{theorem}
    \sremark{The assumption $\mu(X) = 1$ is non-restrictive as every other finite measure $\phi$ can be normalized by taking $\mu = \frac{\phi}{\phi(X)}$.}

    \newdef{Radon-Nikodym derivative}{\index{Radon-Nikodym!derivative}\index{derivative|seealso{Radon-Nikodym}}
        The function $h$ whose existence is implied by the previous theorem is called the Radon-Nikodym derivative of $\nu$ with respect to $\mu$ and we denote it by $\deriv{\nu}{\mu}$.
    }

    \begin{theorem}[Radon-Nikodym theorem]\index{Radon-Nikodym}
        Let $(X,\Sigma)$ be a measurable space and let $\mu,\nu$ be two $\sigma$-finite measures defined on this space such that $\nu\ll\mu$. There exists a non-negative measurable function $g:X\rightarrow\mathbb{R}$ such that $\nu(F) = \int_F g\ d\mu$ for all $F\in\Sigma$.
    \end{theorem}
    \remark{The function $g$ in the previous theorem is unique up to a $\mu$-null (and thus $\nu$-null) set.}

    \begin{property}[Change of variables]
        Let $\mu, \nu$ be finite measures such that $\mu$ dominates $\nu$ and let $\deriv{\nu}{\mu}$ be the associated Radon-Nikodym derivative. For every $\nu$-integrable function $f$ we have
        \begin{gather}
            \int_X f\ d\nu = \int_X fh_\nu\ d\mu.
        \end{gather}
    \end{property}

    \begin{property}\index{chain!rule}
        Let $\lambda,\nu$ and $\mu$ be $\sigma$-finite measures. If $\lambda\ll\mu$ and $\nu\ll\mu$ then we have
        \begin{itemize}
            \item \textbf{Linearity}: $\ds\deriv{(\lambda+\nu)}{\mu} = \deriv{\lambda}{\mu} + \deriv{\lambda}{\mu}$ a.e.
            \item \textbf{Chain rule}: if $\lambda\ll\nu$ then $\ds\deriv{\lambda}{\mu} = \deriv{\lambda}{\nu}\deriv{\nu}{\mu}$ a.e.
        \end{itemize}
    \end{property}

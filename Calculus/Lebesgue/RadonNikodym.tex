\section{Radon-Nikodym theorem}\label{lebesgue:section:Radon-Nikodym}
	
	\begin{definition}\index{continuity!absolute continuity}\label{lebesgue:absolute_continuity}
		Let $(\Omega,\mathcal{F})$ be a measurable space. Let $\mu, \nu$ be two measures defined on this space. $\nu$ is said to be \textbf{absolutely continuous with respect to} $\mu$ if
        	\begin{gather}
	        	\forall A\in\mathcal{F}: \mu(A) = 0\implies\nu(A) = 0
		\end{gather}
	\end{definition}
	\begin{notation}
		This relation is denoted by $\nu \ll \mu$.
	\end{notation}
	\begin{theorem}[Absolute continuity]
    		Let $\mu, \nu$ be finite measures on a measurable space $(\Omega, \mathcal{F})$. Then $\nu\ll\mu$ if and only if
	        \begin{gather}
        		\forall\varepsilon>0:\exists\delta>0:\forall A\in\mathcal{F}:\mu(A)<\delta\implies\nu(A)<\varepsilon
	        \end{gather}
	\end{theorem}

	Property \ref{lebesgue:theorem:measure_by_integral} can be generalized to arbitrary measure spaces as follows:
	\begin{property}
		Let $(\Omega,\mathcal{F}, \mu)$ be a measure space. Let $f:\Omega\rightarrow\mathbb{R}$ be a measurable function such that $\int fd\mu$ exists. Then $\nu(f) = \int_Ffd\mu$ defines a measure $\nu\ll\mu$.
	\end{property}

	\begin{definition}[Dominated measure]\index{measure!dominated}
    		Let $\mu, \nu$ be two measures. $\mu$ is said to \textbf{dominate} $\nu$ if $0\leq\nu(F)\leq\mu(F)$ for every $F\in\mathcal{F}$.
	\end{definition}
    
	\begin{theorem}[Radon-Nikodym theorem for dominated measures]\index{Radon-Nikodym!theorem}~\newline
		Let $\mu$ be a measure such that $\mu(\Omega) = 1$. Let $\nu$ be a measure dominated by $\mu$. There exists a non-negative $\mathcal{F}$-measurable function $h$ such that $\nu(F) = \int_F hd\mu$ for all $F\in\mathcal{F}$.
	\end{theorem}
	\sremark{The assumption $\mu(\Omega) = 1$ is non-restrictive as every other finite measure $\phi$ can be normalized by putting $\mu = \frac{\phi}{\phi(\Omega)}$.}

	\newdef{Radon-Nikodym derivative}{\index{Radon-Nikodym!derivative}\index{derivative|seealso{Radon-Nikodym}}
    		The function $h$ as defined in previous theorem is called the Radon-Nikodym derivative of $\nu$ with respect to $\mu$ and we denote it by $\ds\deriv{\nu}{\mu}$.
	}

	\begin{theorem}[Radon-Nikodym theorem]\index{Radon-Nikodym!theorem}
		Let $(\Omega,\mathcal{F})$ be a measurable space. Let $\mu,\nu$ be two $\sigma$-finite measures defined on this space such that $\nu\ll\mu$. There exists a non-negative measurable function $g:\Omega\rightarrow\mathbb{R}$ such that $\nu(F) = \int_F gd\mu$ for all $F\in\mathcal{F}$.
	\end{theorem}
	\remark{The function $g$ in the previous theorem is unique up to a $\mu$-null (or $\nu$-null) set.}

	\begin{property}
		Let $\mu, \nu$ be finite measures such that $\mu$ dominates $\nu$. Let $\deriv{\nu}{\mu}$ be the associated Radon-Nikodym derivative. For every $\nu$-integrable function $f$ we have
        	\begin{gather}
			\int_\Omega fd\nu = \int_\Omega fh_\nu d\mu
		\end{gather}
	\end{property}

	\begin{property}
    		Let $\lambda,\nu,\mu$ be $\sigma$-finite measures. If $\lambda\ll\mu$ and $\nu\ll\mu$ then we have:
	        \begin{itemize}
			\item $\ds\deriv{(\lambda+\nu)}{\mu} = \deriv{\lambda}{\mu} + \deriv{\lambda}{\mu}$ a.e.
			\item Chain rule: if $\lambda\ll\nu$ then $\ds\deriv{\lambda}{\mu} = \deriv{\lambda}{\nu}\deriv{\nu}{\mu}$ a.e.
        	\end{itemize}
	\end{property}
\section{Measures}
\subsection{General definitions}

    \newdef{Measure}{\index{measure}\index{outer!measure}\index{$\sigma$!additivity}\label{lebesgue:measure}
        Let $X$ be a set and let $\Sigma$ be a $\sigma$-algebra over $X$. A function $\mu:\Sigma\rightarrow\overline{\mathbb{R}}$ is called a measure if it satisfies the following conditions:
        \begin{enumerate}
            \item \textbf{Non-negativity}: $\forall E\in\Sigma:\mu(E) \geq0$,
            \item \textbf{Measure zero}: $\mu(\emptyset) = 0$, and
            \item \textbf{Countable additivity}\footnote{This is also called \textbf{$\sigma$-additivity}.} : $\forall i\neq j:E_i\cap E_j=\emptyset\implies\mu\left(\bigcup_{i=1}^\infty E_i\right) = \sum_{i=1}^\infty \mu(E_i)$.
        \end{enumerate}
        When $\mu$ only satisfies countable subadditivity, i.e. the equality in the last condition becomes an inequality $\leq$, for any collection of sets (disjoint or not) it is called an \textbf{outer measure}.
    }

    \newdef{Measure space}{\label{lebesgue:measure_space}\index{measurable!set}
        The pair $(X, \Sigma)$ is called a measurable space. The elements $E\in\Sigma$ are called \textbf{measurable sets}. The triplet $(X, \Sigma, \mu)$ is called a measure space.
    }

    \begin{method}
        To show that two measures coincide on a $\sigma$-algebra, it suffices to show that they coincide on the generating sets and apply the monotone class theorem \ref{set:theorem:monotone_class}.
    \end{method}

    \newdef{Almost everywhere\footnotemark}{\label{lebesgue:almost_everywhere}\index{almost everywhere}
        \footnotetext{In probability theory this is often called \textbf{almost surely}.}
        Let $(X, \Sigma, \mu)$ be a measure space. A property $P$ is said to hold on X almost everywhere (a.e.) if it satisfies the following equation:
        \begin{gather}
            \mu\big(\{x\in X:\neg P(x)\}\big) = 0.
        \end{gather}
    }

    \newdef{Complete measure space}{\index{complete!measure space}
        The measure space $(X,\Sigma,\mu)$ is said to be complete if for every $E\in\Sigma$ with $\mu(E) = 0$ the following property holds for all $A\subset E$:
        \begin{gather}
            A\in\Sigma \qquad\text{and}\qquad \mu(A) = 0.
        \end{gather}
    }
    \newdef{Completion}{
        Let $\mathcal{F},\mathcal{G}$ be $\sigma$-algebras over a set $X$. $\mathcal{G}$ is said to be the completion of $\mathcal{F}$ if it is the smallest $\sigma$-algebra such that the measure space $(X,\mathcal{G},\mu)$ is complete.
    }

    \newdef{Borel measure}{\index{Borel!measure}
        Consider a topological space $X$ together with its Borel $\sigma$-algebra $\mathcal{B}$ (see definition \ref{topology:borel_set}). Any measure defined on the measurable space $(X, \mathcal{B})$ is called a Borel measure.
    }
    \newdef{Regular measure}{\label{lebesgue:regular_measure}
        Let $\mu$ be a measure on a measurable space $(X, \Sigma)$. $\mu$ is called a regular measure if it satisfies the following equations for every measurable set $B$:
        \begin{align}
            \mu(B) =& \inf\big\{\mu(O):O \text{ open and measurable}, O\supset B\big\}\\
            \mu(B) =& \sup\big\{\mu(F):F \text{ compact and measurable}, F\subset B\big\}.
        \end{align}
    }
    \newdef{Radon measure}{\index{Radon!measure}\index{locally!finite}\label{lebesgue:radon_measure}
        A regular Borel measure with the additional property that it is \textbf{locally finite}, i.e. every point has a neighbourhood of finite measure. If we restrict ourselves to locally compact Hausdorff spaces then this is equivalent to requiring that every compact subset has finite measure.
    }

    \newdef{\texorpdfstring{$\sigma$-}{sigma-}finite measure}{\index{$\sigma$!finite}\label{lebesgue:sigma_finite_measure}
        Let $(X,\Sigma,\mu)$ be a measure space. The measure $\mu$ is said to be $\sigma$-finite if there exists a sequence $\seq{A}$ of measurable sets such that $\bigcup_{n=1}^{+\infty}A_n = X$ with $\forall A_n:\mu(A_n) < +\infty$.
    }

    \newdef{Measure-preserving map}{
        Let $(X,\Sigma,\mu)$ be a measure space and consider a map $T:X\rightarrow X$. $T$ is said to be measure-preserving if it satisfies the following equation:
        \begin{gather}
            \mu\left(T^{-1}(A)\right) = \mu(A)
        \end{gather}
        for all $A\in\Omega$. This equation can also be written using a pushforward notation: $T_\ast\mu=\mu$. These form the morphisms in the category of measure spaces.
    }

    \newdef{Ergodic map}{\index{ergodic}
        Let $(X,\Omega)$ be a measure space. Consider a measure-preserving map $T:X\rightarrow X$. $T$ is said to be ergodic if the following conditions is satisfied:
        \begin{gather}
            T(A) = A\implies \mu(A) = 0 \lor \mu(X\backslash A) = 0.
        \end{gather}
        This is equivalent to stating that for every set $A\in\Sigma$ with positive measure the following condition holds:
        \begin{gather}
            \mu\left(\bigcup_{n=1}^\infty T^{-n}(A)\right) = 1.
        \end{gather}
    }

    \begin{property}
        Consider a topological space $X$ with Borel $\sigma$-algebra $\mathcal{B}$. Almost every $T$-orbit is dense in the support of $\mu$.
    \end{property}

    \newdef{Mixing}{\index{mixing}
        An endomorphism of a measure spaces $(X,\Sigma,\mu)$ is said to be mixing if for all measurable spaces $A,B$ the following equality holds:
        \begin{gather}
            \lim_{n\rightarrow+\infty}\mu\left(T^{-n}(A)\cap B\right) = \mu(A)\mu(B)
        \end{gather}
    }
    \begin{property}
        All mixing transformations are ergodic.
    \end{property}

\subsection{Lebesgue measure}

    \newdef{Null set}{\index{null!set}
        A set $A\subset\mathbb{R}$ is called a null set if it can be covered by a sequence of intervals of arbitrarily small length, i.e. $\forall\varepsilon>0$ there exists a sequence $\seq{I}$ such that
        \begin{gather}
            A \subseteq \bigcup_{n=1}^{+\infty}I_n
        \end{gather}
        and
        \begin{gather}
            \sum_{i=1}^{+\infty}l(I_n) < \varepsilon.
        \end{gather}
    }

    \begin{property}
        Let $(E_i)_{i\in\mathbb{N}}$ be a sequence of null sets. The union $\bigcup_{i=1}^{+\infty}E_i$ is also null.
    \end{property}
    \begin{result}\label{lebesgue:theorem:countable_set_is_null}
        Any countable set is null.
    \end{result}

    \newdef{Lebesgue outer measure}{\index{Lebesgue!outer measure}\label{lebesgue:outer_measure}
        Let $X\subseteq\mathbb{R}$ be a set. The (Lebesgue) outer measure of $X$ is defined as follows:
        \begin{gather}
            m^*(X) := \inf\left\{\sum_{i=1}^{+\infty} l(I_i)\text{ with }(I_i)_{i\in\mathbb{N}} \text{ a sequence of open intervals that covers }X\right\}.
        \end{gather}
    }

    \begin{property}
        Let $I$ be an interval. The outer measure equals the length: $m^*(I) = l(I)$.
    \end{property}
    \begin{property}
        The outer measure is translation-invariant: $m^*(A + t) = m^*(A)$ for all $A,t$.
    \end{property}
    \begin{property}
        The Lebesgue outer measure is an outer measure in the sense of definition \ref{lebesgue:measure}.
    \end{property}

    \begin{theorem}[Carath\'eodory's criterion]\index{Carath\'eodory!criterion}\index{Lebesgue!measure}\index{measurable!set}\label{lebesgue:lebesgue_measure}
        Let $X$ be a subset of $\mathbb{R}$. If $X$ satisfies the following equation, it is said to be \textbf{Lebesgue measurable}:
        \begin{gather}
            \forall E\subseteq\mathbb{R}:m^*(E) = m^*(E\cap X) + m^*(E\cap X^c).
        \end{gather}
        This is denoted by $X\in\mathcal{M}$ and the outer measure $m^*(X)$ is called the Lebesgue measure of $X$. It is denoted by $m(X)$.
    \end{theorem}
    \begin{property}
        All null sets and intervals are measurable.
    \end{property}
    \newprop{Countable additivity}{\index{countable!additivity}
        For every sequence $(E_i)_{i\in\mathbb{N}}$ with $E_i\in\mathcal{M}$ satisfying $i\neq j:E_i\cap E_j = \emptyset$ the following equation holds:
        \begin{gather}
            m\left(\bigcup_{i=1}^{+\infty}E_i\right) = \sum_{i=1}^{+\infty}m(E_i).
        \end{gather}
    }
    \begin{remark}
       Previous property, together with the properties of the outer measure, implies that the Lebesgue measure is indeed a proper measure as defined in \ref{lebesgue:measure}. Furthermore, $\mathcal{M}$ is a $\sigma$-algebra \ref{set:sigma_algebra} over $\mathbb{R}$.
    \end{remark}

    \begin{property}
        For every $A\subset\mathbb{R}$ there exists a sequence $(O_i)_{i\in\mathbb{N}}$ of open sets such that
        \begin{gather}
            \label{lebesgue:theorem:open_cover_existence}
            A\subset\bigcap_iO_i\qquad\text{and}\qquad m\left(\bigcap_iO_i\right) = m^*(A).
        \end{gather}
    \end{property}
    \begin{property}
        For every $E\in\mathcal{M}$ there exists a sequence $(F_i)_{i\in\mathbb{N}}$ of closed sets such that
        \begin{gather}
            \label{lebesgue:theorem:closed_cover_existence}
            \bigcup_iF_i\subset E\qquad\text{and}\qquad m\left(\bigcup_iF_i\right) = m(E).
        \end{gather}
    \end{property}
    \sremark{The previous 2 theorems imply that the Lebesgue measure is a regular Borel measure (see definition \ref{lebesgue:regular_measure}).}

    \begin{property}
        Let $E\subset\mathbb{R}$. $E\in\mathcal{M}$ if and only if for every $\varepsilon>0$ there exist an open set $O\supset E$ and a closed set $F\subset E$ such that $m^*(O\backslash E) < \varepsilon$ and $m^*(E\backslash F)<\varepsilon$.
    \end{property}

    \begin{property}
        Let $(A_i)_{i\in\mathbb{N}}$ be a sequence of sets with $\forall i:A_i\in\mathcal{M}$. The following two properties apply:
        \begin{align}
            \forall i: A_i\subseteq A_{i+1} &\implies m\left(\bigcup_{i=1}^{+\infty}A_i\right) = \lim_{i\rightarrow+\infty}m(A_i)\\
            \forall i: A_i\supseteq A_{i+1} \land m(A_1)<+\infty &\implies m\left(\bigcap_{i=1}^{+\infty}A_i\right) = \lim_{i\rightarrow+\infty}m(A_i).
        \end{align}
    \end{property}
    \remark{This property is not only valid for the Lebesgue measure but for every countably additive set function.}
    \begin{property}[Continuity]
        The Lebesgue measure $m(X)$ is continuous at $\emptyset$, i.e. if $\lim_{i\rightarrow\infty}A_i=\emptyset$ then $\lim_{i\rightarrow+\infty}m(A_i) = 0$.
    \end{property}

    \begin{property}[Relation between Lebesgue and Borel algebras]\label{lebesgue:completion_remark}
        The Lebesgue $\sigma$-algebra $\mathcal{M}$ is the completion of the Borel $\sigma$-algebra $\mathcal{B}$. (This is in fact how the Lebesgue $\sigma$-algebra was introduced historically.)
    \end{property}

    \begin{construct}[Restriction]\index{Lebesgue!restricted measure}\label{lebesgue:restricted_lebesgue_measure}
        Let $B\subset\mathbb{R}$ be a measurable set with measure $m(B)>0$. The restriction of the Lebesgue measure to the set $B$ is defined as follows:
        \begin{gather}
            \mathcal{M}_B := \left\{A\cap B:A\in\mathcal{M}\right\}\qquad\text{and}\qquad\forall E\in\mathcal{M}_B:m_B(E) := m(E).
        \end{gather}
        Furthermore, the measure space $(B,\mathcal{M}_B,m_B)$ is complete.
    \end{construct}

\subsection{Measurable functions}

    \newdef{Measurable function}{\index{measurable!function}\label{lebesgue:measurable_function}
        A function $f$ is (Lebesgue) measurable if for every interval $I\subset\mathbb{R}:f^{-1}(I)\in\mathcal{M}$.
    }
    \newdef{Borel measurable function}{\index{Borel!measurable function}\label{lebesgue:borel_measurable_function}
        A function $f$ is called Borel measurable\footnote{These functions are often simply called \textbf{Borel functions}.} if for every interval $I\subset\mathbb{R}:f^{-1}(I)\in\mathcal{B}$.
    }
    \remark{The inclusion $\mathcal{B}\subset\mathcal{M}$ implies that every Borel-measurable function is also Lebesgue-measurable.}

    \begin{property}
        The class of Borel/Lebesgue measurable functions defined on $E\in\mathcal{M}$ forms an algebra.
    \end{property}

    \begin{example}
        Following types of functions are measurable:
        \begin{itemize}
            \item monotonic functions,
            \item continuous functions, and
            \item indicator functions.
        \end{itemize}
    \end{example}
    \begin{result}
        Let $f,g$ be measurable functions and let $F:\mathbb{R}\times\mathbb{R}\rightarrow\mathbb{R}$ be a continuous function. The composition $F(f(x), g(x))$ is also measurable.
    \end{result}

    \begin{property}
        Let $f$ be a measurable function. The level set $\{x:f(x) = a\}$ is measurable for all $a\in\mathbb{R}$.
    \end{property}

    \begin{property}
        Define following functions, which are measurable if $f$ is measurable as a result of previous properties:
        \begin{gather}
            \label{lebesgue:positive_part}
            f^+(x) = \max(f,0) =
            \begin{cases}
                f(x)&f(x)>0\\
                0&f(x)\leq0
            \end{cases}
        \end{gather}
        \begin{gather}
            \label{lebesgue:negative_part}
            f^-(x) = \max(-f,0) =
            \begin{cases}
                0&f(x)>0\\
                -f(x)&f(x)\leq0.
            \end{cases}
        \end{gather}
        The function $f:E\rightarrow\mathbb{R}$ is measurable if and only if both $f^+$ and $f^-$ are measurable. Furthermore, $f$ is measurable if $|f|$ is measurable (the converse is false).
    \end{property}

\subsection{Limit operations}

    \begin{property}
        Let $\seq{f}$ be a sequence of Borel/Lebesgue measurable functions. The following functions are also measurable:
        \begin{itemize}
            \item $\ds\min_{i\leq k}(f_i)$ and $\ds\max_{i\leq k}(f_i)$,
            \item $\ds\inf_{i\in\mathbb{N}}(f_i)$ and $\ds\sup_{i\in\mathbb{N}}(f_i)$, and
            \item $\ds\liminf_{i\rightarrow+\infty}(f_i)$ and $\ds\limsup_{i\rightarrow+\infty}(f_i)$.
        \end{itemize}
    \end{property}

    \begin{property}
        Let $f$ be a measurable function and let $g$ be a function such that $f=g$ almost everywhere. Then $g$ is measurable as well.
    \end{property}
    \result{A result of the previous two properties is the following: if a sequence of measurable functions converges pointwise a.e. then the limit is also a measurable function.}

    \newdef{Essential supremum}{\index{essential!supremum}\label{lebesgue:essential_supremum}
        \begin{gather}
            \esssup(f) := \sup\{z:f\geq z\text{ a.e.}\}
        \end{gather}
    }
    \newdef{Essential infimum}{\index{essential!infimum}\label{lebesgue:essential_infimum}
        \begin{gather}
            \essinf(f) := \inf\{z:f\leq z\text{ a.e.}\}
        \end{gather}
    }
    \begin{property}
        Every measurable function $f$ satisfies the following inequalities:
        \begin{itemize}
            \item $f\leq\esssup(f)\text{ a.e.}$ and $f\geq\essinf(f)\text{ a.e.}$
            \item $\esssup(f)\leq\sup(f)$ and $\essinf(f)\geq\inf(f)$.
        \end{itemize}
        The latter pair of inequalities becomes a pair of equalities if $f$ is continuous.
    \end{property}
    \begin{property}
        If $f,g$ are measurable functions then $\esssup(f+g)\leq\esssup(f) + \esssup(g)$. An analogous inequality holds for the essential infimum.
    \end{property}
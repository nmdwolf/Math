\section{Measures}
\subsection{General definitions}

    \newdef{Measure}{\index{measure}\index{outer!measure}\index{$\sigma$!additivity}\label{lebesgue:measure}
        Let $X$ be a set and let $\Sigma$ be a $\sigma$-algebra over $X$. A function $\mu:\Sigma\rightarrow\overline{\mathbb{R}}$ is called a measure if it satisfies the following conditions:
        \begin{enumerate}
            \item\textbf{Nonnegativity}: $\forall E\in\Sigma:\mu(E) \geq0$,
            \item\textbf{Empty set is null}: $\mu(\emptyset) = 0$, and
            \item\textbf{$\sigma$-additivity}: $\forall i\neq j:E_i\cap E_j=\emptyset\implies\mu\left(\bigcup_{n=1}^\infty E_n\right) = \sum_{i=n}^\infty\mu(E_n)$.
        \end{enumerate}
        When $\mu$ only satisfies countable subadditivity, i.e. the equality in the last condition becomes an inequality $\leq$, it is called an \textbf{outer measure}.
    }
    \begin{remark}
        To show that two measures coincide on a $\sigma$-algebra, it suffices to show that they coincide on the generating sets and apply the monotone class theorem \ref{set:monotone_class}.
    \end{remark}

    \newdef{Measure space}{\label{lebesgue:measure_space}\index{measurable!set}
        The pair $(X,\Sigma)$ is called a measurable space and the triple $(X,\Sigma,\mu)$ is called a measure space. The elements $E\in\Sigma$ are called \textbf{measurable sets}.
    }

    \newdef{Null set}{\index{null!set}
        A set $A\subset\mathbb{R}$ is said to be null if $\mu(A)=0$.
    }

    \newdef{Almost everywhere\footnotemark}{\index{almost everywhere}\label{lebesgue:almost_everywhere}
        \footnotetext{In probability theory this is often called \textbf{almost surely}.}
        Let $(X,\Sigma,\mu)$ be a measure space. A property $P$ is said to hold on $X$ almost everywhere (abbreviated as \textbf{a.e.}) if it satisfies the following equation:
        \begin{gather}
            \mu\big(\{x\in X\,\big\vert\,\neg P(x)\}\big) = 0,
        \end{gather}
        i.e. it holds everywhere except for a null set.
    }

    \newdef{Complete measure space}{\index{complete!measure space}
        A measure space $(X,\Sigma,\mu)$ is said to be complete if for every $E\in\Sigma$ with $\mu(E)=0$ the implication $A\subset E\implies A\in\Sigma$ holds. Additivity then necessarily implies that $\mu(A)=0$.
    }
    \newdef{Completion}{
        Let $\mathcal{F}\subseteq\mathcal{G}$ be $\sigma$-algebras over a set $X$. $(X,\mathcal{G},\overline{\mu})$ is called the completion of $(X,\mathcal{F},\mu)$ if:
        \begin{enumerate}
            \item $\forall A\in\mathcal{F}:\overline{\mu}(A)=\mu(A)$,
            \item $(X,\mathcal{G},\overline{\mu})$ is complete, and
            \item $\mathcal{G}$ is the smallest $\sigma$-algebra for which the foregoing conditions hold.
        \end{enumerate}
    }

    \newdef{Borel measure}{\index{Borel!measure}
        Consider a topological space together with its Borel $\sigma$-algebra \ref{topology:borel_set}. Any measure defined on this measurable space is called a Borel measure.
    }
    \newdef{Regular measure}{\label{lebesgue:regular_measure}
        Let $\mu$ be a measure on a measurable space $(X,\Sigma)$. It is called a regular measure if it satisfies the following equations for every measurable set $B$:
        \begin{align}
            \mu(B) =& \inf\big\{\mu(O)\,\big\vert\,O\text{ open and measurable},O\supset B\big\}\\
            \mu(B) =& \sup\big\{\mu(F)\,\big\vert\,F\text{ compact and measurable},F\subset B\big\}.
        \end{align}
        A Borel regular measure can also be characterized as a Borel measure such that for every subset $A\subseteq X$ there exists a Borel set $B$ with $A\subseteq B$ and $\inf\big\{\mu(E)\,\big\vert\,A\subseteq E\in\Sigma\big\}=\mu(B)$.
    }
    \newdef{Radon measure}{\index{Radon!measure}\index{locally!finite}\label{lebesgue:radon_measure}
        A Borel measure on a Hausdorff space that is outer regular, inner regular on open sets and \textbf{locally finite}, i.e. every point has a neighbourhood of finite measure. When restricted to locally-compact Hausdorff spaces, this is equivalent to requiring that every compact subset has finite measure.
    }

    \newdef{\texorpdfstring{$\sigma$-}{sigma-}finite measure}{\index{$\sigma$!finite}\label{lebesgue:sigma_finite_measure}
        Let $(X,\Sigma,\mu)$ be a measure space. The measure $\mu$ is said to be $\sigma$-finite if there exists a sequence $\seq{A}$ of measurable sets such that $\bigcup_{n=1}^\infty A_n = X$ with $\forall n\in\mathbb{N}:\mu(A_n)<\infty$.
    }

\subsection{Lebesgue measure}

    \newdef{Lebesgue outer measure}{\index{Lebesgue!outer measure}\label{lebesgue:outer_measure}
        Let $X\subseteq\mathbb{R}$ be a set. The (Lebesgue) outer measure of $X$ is defined as follows:
        \begin{gather}
            \lambda^*(X) := \inf\left\{\sum_{n=1}^\infty l(I_n)\,\middle\vert\,\seq{I}\text{ a sequence of open intervals that covers }X\right\}.
        \end{gather}
    }

    \begin{property}[Intervals]
        The outer measure of an interval $I$ equals its length: $\lambda^*(I) = l(I)$.
    \end{property}
    \begin{property}[Translation]\label{lebesgue:translation_invariant}
        The outer measure is translation-invariant: $\lambda^*(A+t) = \lambda^*(A)$ for all $A\subset\mathbb{R}$ and $t\in\mathbb{R}$.
    \end{property}
    \begin{property}
        The Lebesgue outer measure is an outer measure in the sense of Definition \ref{lebesgue:measure}.
    \end{property}

    \begin{theorem}[Carath\'eodory's criterion]\index{Carath\'eodory!criterion}\index{Lebesgue!measure}\index{measurable!set}\label{lebesgue:lebesgue_measure}
        Let $X$ be a subset of $\mathbb{R}$. If $X$ satisfies the following equation, it is said to be \textbf{Lebesgue measurable}:
        \begin{gather}
            \forall A\subseteq\mathbb{R}:\lambda^*(A) = \lambda^*(A\cap X) + \lambda^*(A\cap X^c).
        \end{gather}
        The collection of all Lebesgue-measurable sets is denoted by $\mathcal{M}$ and the outer measure $\lambda^*(X)$, now denoted by $\lambda$, is called the \textbf{Lebesgue measure} of $X$.
    \end{theorem}
    \begin{construct}[Carath\'eodory's extension theorem]\index{premeasure}\label{lebesgue:caratheodory}
        In fact, the above construction can be generalized to arbitrary sets. Every outer measure $\mu^*$ gives rise to a $\sigma$-algebra consisting of those sets that satisfy Carath\'eodory's criterion \ref{lebesgue:lebesgue_measure} with respect to $\mu^*$. Furthermore, consider a \textbf{premeasure} $\mu_0$, i.e. a $\sigma$-additive function defined on an algebra of sets \ref{set:algebra_of_sets} such that $\mu_0(\emptyset) = 0$. Definition \ref{lebesgue:outer_measure} can be used to define an outer measure $\mu^*$ in terms of the premeasure $\mu_0$ by replacing intervals with elements from the given algebra of sets. The $\sigma$-algebra generated by this outer measure contains the given algebra of sets and $\mu^*$ restricts to $\mu_0$. This shows that any premeasure can be extended to a genuine measure. Moreover, it can be shown that this measure is complete.
    \end{construct}
    \begin{result}\label{lebesgue:completion_remark}
        The Lebesgue $\sigma$-algebra $\mathcal{M}$ is the completion of the Borel $\sigma$-algebra $\mathcal{B}$. (This is how the Lebesgue $\sigma$-algebra was introduced historically.)
    \end{result}

    \begin{property}\label{lebesgue:countable_set_is_null}
        Any countable set is null with respect to the Lebesgue outer measure.
    \end{property}

    \begin{property}
        The Lebesgue measure is a regular Borel measure. For every $A\subset\mathbb{R}$ there exists a sequence $\seq{O}$ of open sets such that
        \begin{gather}
            \label{lebesgue:open_cover_existence}
            A\subset\bigcap_{n=1}^\infty O_n\qquad\text{and}\qquad\lambda\left(\bigcap_{n=1}^\infty O_n\right) = \lambda^*(A),
        \end{gather}
        and for every $E\in\mathcal{M}$ there exists a sequence $\seq{F}$ of closed sets such that
        \begin{gather}
            \label{lebesgue:closed_cover_existence}
            \bigcup_{n=1}^\infty F_n\subset E\qquad\text{and}\qquad\lambda\left(\bigcup_{n=1}^\infty F_n\right) = \lambda(E).
        \end{gather}
    \end{property}

    \begin{property}
        Consider a set $A\subset\mathbb{R}$. $A\in\mathcal{M}$ if and only if for every $\varepsilon>0$ there exist an open set $O\supset A$ and a closed set $F\subset A$ such that $\lambda^*(O\backslash A) < \varepsilon$ and $\lambda^*(A\backslash F)<\varepsilon$.
    \end{property}

    \begin{property}
        Let $\seq{A}$ be a sequence of sets in $\mathcal{M}$. The following two properties apply:
        \begin{align}
            \forall i\in\mathbb{N}:A_i\subseteq A_{i+1} &\implies \lambda\left(\bigcup_{n=1}^\infty A_n\right) = \lim_{n\rightarrow+\infty}\lambda(A_n)\\
            \forall i\in\mathbb{N}:A_i\supseteq A_{i+1}\land\lambda(A_1)<+\infty &\implies \lambda\left(\bigcap_{i=n}^\infty A_n\right) = \lim_{n\rightarrow+\infty}\lambda(A_n).
        \end{align}
    \end{property}
    \remark{This property is valid for every $\sigma$-additive set function.}

    \begin{construct}[Restriction]\index{Lebesgue!restricted measure}\label{lebesgue:restricted_lebesgue_measure}
        Let $A\in\mathcal{M}$ have nonzero measure. The restriction of the Lebesgue measure to the set $B$ is defined as follows:
        \begin{gather}
            \mathcal{M}_A := \{A\cap B\mid B\in\mathcal{M}\} \qquad\text{and}\qquad \forall E\in\mathcal{M}_A:\lambda_A(E) := \lambda(E).
        \end{gather}
        It can be shown that the measure space $(A,\mathcal{M}_A,\lambda_A)$ is complete.
    \end{construct}

\subsection{Measurable functions}

    \newdef{Measurable function}{\index{measurable!function}
        Consider two measurable spaces $(X,\Sigma_X)$ and $(Y,\Sigma_Y)$. A function $f:X\rightarrow Y$ is said to be measurable if for every measurable set $A\in\Sigma_Y$ the preimage $f^{-1}(A)$ is also measurable. Equivalently, the $\sigma$-algebra generated by the preimages of measurable sets in $\Sigma_Y$ should be a sub-$\sigma$-algebra of $\Sigma_X$.
    }

    Two important examples are given below:
    \begin{example}[Borel-measurable function]\label{lebesgue:borel_measurable_function}
        A continuous function $f:X\rightarrow Y$ such that for every open set $O\in\mathcal{B}_Y:f^{-1}(O)\in\mathcal{B}_X$.
    \end{example}
    \begin{example}[Lebesgue-measurable function]\label{lebesgue:measurable_function}
        A function $f:\mathbb{R}\rightarrow\mathbb{R}$ such that for every interval $I\subset\mathbb{R}:f^{-1}(I)\in\mathcal{M}$.
    \end{example}
    \remark{The inclusion $\mathcal{B}\subset\mathcal{M}$ implies that every Borel-measurable function is also Lebesgue-measurable.}

    \begin{property}
        The class of Borel/Lebesgue-measurable functions defined on $E\in\mathcal{M}$ forms an algebra.
    \end{property}

    \begin{example}
        The following types of functions are Lebesgue-measurable:
        \begin{itemize}
            \item monotonic functions,
            \item continuous functions,
            \item indicator functions, and
            \item compositions of measurable functions.
        \end{itemize}
    \end{example}
    \begin{result}
        Let $f,g$ be (Lebesgue-)measurable functions and let $F:\mathbb{R}\times\mathbb{R}\rightarrow\mathbb{R}$ be a continuous function. The composition $F(f(x),g(x))$ is also measurable.
    \end{result}

    \begin{property}
        Let $f$ be a Lebesgue-measurable function. The level set $\{x\mid f(x) = a\}$ is measurable for all $a\in\mathbb{R}$.
    \end{property}

    \begin{property}
        Define the following functions (which are measurable if $f$ is measurable as a result of the previous properties):
        \begin{align}
            \label{lebesgue:positive_part}
            f^+(x) &:= \max(f,0) =
            \begin{cases}
                f(x)&f(x)>0\\
                0&f(x)\leq0,
            \end{cases}\\\nonumber\\
            \label{lebesgue:negative_part}
            f^-(x) &:= \max(-f,0) =
            \begin{cases}
                0&f(x)>0\\
                -f(x)&f(x)\leq0.
            \end{cases}
        \end{align}
        The function $f:\mathbb{R}\rightarrow\mathbb{R}$ is measurable if and only if both $f^+$ and $f^-$ are measurable. Furthermore, $f$ is measurable if $|f|$ is measurable (the converse is false in general).
    \end{property}

    \newdef{Pushforward}{\index{pushforward!measure}
        Consider two measurable spaces $(X_1,\Sigma_1)$ and $(X_2,\Sigma_2)$ together with a measurable function $f:X_1\rightarrow X_2$. For every measure $\mu$ on $X_1$ one can define the pushforward measure $f_*\mu$ on $X_2$ as follows:
        \begin{gather}
            f_*\mu(A) := \mu\big(f^{-1}(A)\big).
        \end{gather}
    }

    \newdef{Measure-preserving function}{
        \nomenclature[S_Meas]{$\mathbf{Meas}$}{category of measure spaces and measure-preserving functions}
        Let $(X,\Sigma,\mu)$ be a measure space and consider a measurable function $T:X\rightarrow X$. $T$ is said to be measure-preserving if it satisfies
        \begin{gather}
            \mu\left(T^{-1}(A)\right) = \mu(A)
        \end{gather}
        for all $A\in\Sigma$. This can also be concisely written as $T_\ast\mu=\mu$. These functions form the morphisms in the category $\mathbf{Meas}$ of measure spaces.\footnote{The notation $\mathbf{Meas}$ is also sometimes used to denote the larger category of measurable spaces and measurable functions.}
    }

    \newdef{Ergodic function}{\index{ergodicity}
        Let $(X,\Sigma,\mu)$ be a measure space and consider a measure-preserving function $T:X\rightarrow X$. It is said to be ergodic if the following condition is satisfied:
        \begin{gather}
            T(A) = A\implies\mu(A) = 0\lor\mu(X\backslash A) = 0.
        \end{gather}
        This is equivalent to stating that for every set $A\in\Sigma$ with positive measure the following condition holds:
        \begin{gather}
            \mu\left(\bigcup_{n=1}^\infty T^{-n}(A)\right) = 1.
        \end{gather}
    }

    \begin{property}
        Consider a topological space $X$ with Borel $\sigma$-algebra $\mathcal{B}$ and let $T$ be an ergodic function. Almost every $T$-orbit is dense in the support of $\mu$.
    \end{property}

    \newdef{Mixing}{\index{mixing}
        An endomorphism of a measure spaces $(X,\Sigma,\mu)$ is said to be mixing if for all measurable spaces $A,B$ the following equality holds:
        \begin{gather}
            \lim_{n\rightarrow+\infty}\mu\left(T^{-n}(A)\cap B\right) = \mu(A)\mu(B)
        \end{gather}
    }
    \begin{property}
        All mixing transformations are ergodic.
    \end{property}

\subsection{Limit operations}

    \begin{property}
        Let $\seq{f}$ be a sequence of measurable functions. The following functions are also measurable:
        \begin{itemize}
            \item $\ds\min_{i\leq k}(f_i)$ and $\ds\max_{i\leq k}(f_i)$
            \item $\ds\inf_{n\in\mathbb{N}}(f_n)$ and $\ds\sup_{n\in\mathbb{N}}(f_n)$
            \item $\ds\liminf_{n\rightarrow+\infty}(f_n)$ and $\ds\limsup_{n\rightarrow+\infty}(f_n)$
        \end{itemize}
    \end{property}

    \begin{property}
        If $f$ is a measurable function and $g$ is a function such that $f=g$ almost everywhere, then $g$ is measurable as well.
    \end{property}
    \result{As a result of the previous two properties, if a sequence of measurable functions converges pointwise a.e., the limit is also a measurable function.}

    \newdef{Essential supremum}{\index{essential!supremum}\label{lebesgue:essential_supremum}
        \begin{gather}
            \esssup(f) := \inf\{z\in\mathbb{R}\mid f\leq z\text{ a.e.}\}
        \end{gather}
    }
    \newdef{Essential infimum}{\index{essential!infimum}\label{lebesgue:essential_infimum}
        \begin{gather}
            \essinf(f) := \sup\{z\in\mathbb{R}\mid f\geq z\text{ a.e.}\}
        \end{gather}
    }
    \begin{property}
        Every measurable function $f$ satisfies the following inequalities:
        \begin{itemize}
            \item $f\leq\esssup(f)\text{ a.e.}$ and $f\geq\essinf(f)\text{ a.e.}$
            \item $\esssup(f)\leq\sup(f)$ and $\essinf(f)\geq\inf(f)$.
        \end{itemize}
        The latter pair of inequalities becomes a pair of equalities if $f$ is continuous.
    \end{property}
    \begin{property}
        If $f,g$ are measurable functions, then $\esssup(f+g)\leq\esssup(f) + \esssup(g)$. An analogous inequality holds for the essential infimum.
    \end{property}
\section{Radon-Nikodym theorem}\label{lebesgue:section:Radon-Nikodym}

    \begin{definition}\index{continuity!absolute continuity}\label{lebesgue:absolute_continuity}
        Let $(X,\Sigma)$ be a measurable space and let $\mu, \nu$ be two measures defined on this space. Then $\nu$ is said to be \textbf{absolutely continuous} with respect to $\mu$ if
        \begin{gather}
            \forall A\in\Sigma: \mu(A) = 0\implies\nu(A) = 0.
        \end{gather}
    \end{definition}
    \begin{notation}
        This relation is often denoted by $\nu\ll\mu$.
    \end{notation}

    The following property relates the notion of absolute continuity above with that of definition \ref{calculus:absolute_continuity}:
    \begin{property}[Absolute continuity]
        Let $\mu, \nu$ be finite measures on a measurable space $(X, \Sigma)$. Then $\nu\ll\mu$ if and only if
        \begin{gather}
            \forall\varepsilon>0:\exists\delta>0:\forall A\in\Sigma:\mu(A)<\delta\implies\nu(A)<\varepsilon.
        \end{gather}
    \end{property}

    \newdef{Singular measures}{\index{measure!singular}\index{orthogonal!measure|see{measure, singular}}
        Consider two measures $\mu,\nu$. If there exists a set $A$ such that $\mu(A)=0=\nu(A^c)$ then they are said to be singular (or \textbf{orthogonal}). This is denoted by $\mu\perp\nu$.
    }
    \begin{theorem}[Lebesgue's decomposition theorem]
        Consider two $\sigma$-finite measures $\mu,\nu$. There exist two other $\sigma$-finite measures $\nu_a, \nu_s$ such that $\nu=\nu_a+\nu_s$ where $\nu_a\ll\mu$ and $\nu_s\perp\mu$.
    \end{theorem}

    Property \ref{lebesgue:theorem:measure_by_integral} can be generalized to arbitrary measure spaces as follows:
    \begin{property}
        Let $(X,\Sigma, \mu)$ be a measure space. Let $f:X\rightarrow\mathbb{R}$ be a measurable function such that $\int f\ d\mu$ exists. Then $\nu(f) = \int_Ffd\ \mu$ defines an absolutely continuous measure $\nu\ll\mu$.
    \end{property}

    \begin{definition}[Dominated measure]\index{measure!dominated}
            Let $\mu, \nu$ be two measures defined on a measurable space $(X, \Sigma)$. Then $\mu$ is said to \textbf{dominate} $\nu$ if $0\leq\nu(F)\leq\mu(F)$ for every $F\in\Sigma$.
    \end{definition}

    \begin{theorem}[Radon-Nikodym theorem for dominated measures]\index{Radon-Nikodym}~\newline
        Let $\mu$ be a measure on $(X, \Sigma)$ such that $\mu(X) = 1$ and let $\nu$ be a measure dominated by $\mu$. There exists a non-negative $\Sigma$-measurable function $h$ such that $\nu(F) = \int_F h\ d\mu$ for all $F\in\Sigma$.
    \end{theorem}
    \sremark{The assumption $\mu(X) = 1$ is non-restrictive as every other finite measure $\phi$ can be normalized by taking $\mu = \frac{\phi}{\phi(X)}$.}

    \newdef{Radon-Nikodym derivative}{\index{Radon-Nikodym!derivative}\index{derivative|seealso{Radon-Nikodym}}
        The function $h$ whose existence is implied by the previous theorem is called the Radon-Nikodym derivative of $\nu$ with respect to $\mu$ and we denote it by $\deriv{\nu}{\mu}$.
    }

    \begin{theorem}[Radon-Nikodym theorem]\index{Radon-Nikodym}
        Let $(X,\Sigma)$ be a measurable space and let $\mu,\nu$ be two $\sigma$-finite measures defined on this space such that $\nu\ll\mu$. There exists a non-negative measurable function $g:X\rightarrow\mathbb{R}$ such that $\nu(F) = \int_F g\ d\mu$ for all $F\in\Sigma$.
    \end{theorem}
    \remark{The function $g$ in the previous theorem is unique up to a $\mu$-null (and thus $\nu$-null) set.}

    \begin{property}[Change of variables]
        Let $\mu, \nu$ be finite measures such that $\mu$ dominates $\nu$ and let $\deriv{\nu}{\mu}$ be the associated Radon-Nikodym derivative. For every $\nu$-integrable function $f$ we have
        \begin{gather}
            \int_X f\ d\nu = \int_X fh_\nu\ d\mu.
        \end{gather}
    \end{property}

    \begin{property}\index{chain!rule}
        Let $\lambda,\nu$ and $\mu$ be $\sigma$-finite measures. If $\lambda\ll\mu$ and $\nu\ll\mu$ then we have
        \begin{itemize}
            \item \textbf{Linearity}: $\ds\deriv{(\lambda+\nu)}{\mu} = \deriv{\lambda}{\mu} + \deriv{\lambda}{\mu}$ a.e.
            \item \textbf{Chain rule}: if $\lambda\ll\nu$ then $\ds\deriv{\lambda}{\mu} = \deriv{\lambda}{\nu}\deriv{\nu}{\mu}$ a.e.
        \end{itemize}
    \end{property}
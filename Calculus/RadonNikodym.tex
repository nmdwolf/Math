\section{Radon-Nikodym theorem}\label{section:Radon-Nikodym}

    \begin{definition}[Absolute continuity]\index{continuity!absolute}\label{lebesgue:absolute_continuity}
        Let $(X,\Sigma)$ be a measurable space and let $\mu,\nu$ be two measures defined on this space. Then $\nu$ is said to be absolutely continuous with respect to $\mu$ if
        \begin{gather}
            \forall A\in\Sigma:\mu(A) = 0\implies\nu(A) = 0.
        \end{gather}
        This relation is often denoted by $\nu\ll\mu$.
    \end{definition}

    The following property relates the notion of absolute continuity above with that of Definition \ref{calculus:absolute_continuity}:
    \begin{property}[Absolute continuity]
        Let $\mu,\nu$ be finite measures on a measurable space $(X,\Sigma)$. Then $\nu\ll\mu$ if and only if
        \begin{gather}
            \forall\varepsilon>0:\exists\delta>0:\forall A\in\Sigma:\mu(A)<\delta\implies\nu(A)<\varepsilon.
        \end{gather}
    \end{property}

    \newdef{Singular measures}{\index{measure!singular}\index{orthogonal!measure|see{measure, singular}}
        Consider two measures $\mu,\nu$. If there exists a set $A$ such that $\mu(A)=0=\nu(A^c)$, they are said to be singular (or \textbf{orthogonal}). This is denoted by $\mu\perp\nu$.
    }
    \begin{theorem}[Lebesgue's decomposition theorem]
        Let $\mu,\nu$ be two $\sigma$-finite measures. There exist two other $\sigma$-finite measures $\nu_a,\nu_s$ such that $\nu=\nu_a+\nu_s$, where $\nu_a\ll\mu$ and $\nu_s\perp\mu$.
    \end{theorem}

    \begin{definition}[Dominated measure]\index{measure!dominated}
        Let $\mu,\nu$ be two measures defined on a measurable space $(X,\Sigma)$. Then $\mu$ is said to \textbf{dominate} $\nu$ if $0\leq\nu(F)\leq\mu(F)$ for every $F\in\Sigma$.
    \end{definition}

    \begin{theorem}[Radon-Nikodym theorem for dominated measures]\index{Radon-Nikodym}
        Let $\mu$ be a finite measure on a measurable space $(X,\Sigma)$ and let $\nu$ be a measure dominated by $\mu$. There exists a nonnegative, measurable function $f$ such that $\nu(A) = \int_Af\,d\mu$ for all $A\in\Sigma$.
    \end{theorem}
    \newdef{Radon-Nikodym derivative}{\index{Radon-Nikodym!derivative}\index{derivative|seealso{Radon-Nikodym}}
        The function $f$ in the previous theorem is called the Radon-Nikodym derivative of $\nu$ with respect to $\mu$. It is generally denoted by $\deriv{\nu}{\mu}$.
    }

    \begin{theorem}[Radon-Nikodym theorem]\index{Radon-Nikodym}
        Let $(X,\Sigma)$ be a measurable space and let $\mu,\nu$ be two $\sigma$-finite measures defined on $\Sigma$ such that $\nu\ll\mu$. There exists a nonnegative, measurable function $f:X\rightarrow\mathbb{R}$ such that $\nu(A) = \int_Af\,d\mu$ for all $A\in\Sigma$.
    \end{theorem}
    \remark{The function $f$ in this theorem is unique up to a $\mu$-null (and thus $\nu$-null) set.}
    \begin{property}
        In general the Radon-Nikodym derivative is not integrable (unless the measures are finite). However, it is always locally integrable \ref{lebesgue:locally_integrable}. Together with Property \ref{lebesgue:measure_by_integral} this implies that (densities of) absolutely continuous measures are in bijection with locally integrable functions.
    \end{property}

    \begin{property}[Change of variables]
        Let $\mu,\nu$ be finite measures such that $\nu\ll\mu$ and let $\deriv{\nu}{\mu}$ be the associated Radon-Nikodym derivative. For every $\nu$-integrable function $f$ the following equality holds
        \begin{gather}
            \int_A f\,d\nu = \int_A fh_\nu\,d\mu
        \end{gather}
        for all $A\in\Sigma$.
    \end{property}

    \begin{property}\index{chain!rule}
        Let $\lambda,\nu$ and $\mu$ be $\sigma$-finite measures. If $\lambda\ll\mu$ and $\nu\ll\mu$, the following two properties hold:
        \begin{itemize}
            \item\textbf{Linearity}: $\ds\deriv{(\lambda+\nu)}{\mu} = \deriv{\lambda}{\mu} + \deriv{\lambda}{\mu}$.
            \item\textbf{Chain rule}: If $\lambda\ll\nu$, then $\ds\deriv{\lambda}{\mu} = \deriv{\lambda}{\nu}\deriv{\nu}{\mu}$ a.e.
        \end{itemize}
    \end{property}
\chapter{Bundle theory}

\section{Fibre bundles}

	\newdef{Fibered manifold}{\index{fibered manifold}
		A fibered manifold is a surjective submersion\footnote{See definition \ref{manifolds:submersion}.} $\pi:E\rightarrow B$ where $E$ is called the \textbf{total space} and $B$ the \textbf{base space}.
	}

	The most important example of a fibered manifold is a fibre bundle:
	\newdef{Fibre bundle}{\index{fibre!bundle}\index{local!trivialization}\index{structure group}
		\label{manifolds:fibre_bundle}
		A fibre bundle is a tuple $(E, B, \pi, F, G)$ where $E, B$ and $F$ are topological spaces and $G$ is a topological group (called the \textbf{structure group}), such that there exists a smooth surjective map $\pi:E\rightarrow B$ and an open cover $\{U_i\}_{i\in I}$ of $B$ for which there exists a family of homeomorphisms $\{\varphi_i:\pi^{-1}(U_i)\rightarrow U_i\times F\}_{i\in I}$ that make the following diagram commute:
		\begin{figure}[ht!]
			\centering
			\begin{tikzpicture}
				\matrix (m) [matrix of math nodes,row sep=2em,column sep=3em,minimum width=2em, ampersand replacement=\&]{
					\pi^{-1}(U_i) \& \& U_i\times F\\
					\& U_i\&\\
				};
				\path[-stealth]
					(m-1-1) edge node [above] {$\varphi_i$} (m-1-3)
					(m-1-1) edge node [below left] {$\pi$} (m-2-2)
					(m-1-3) edge node [below right] {$\text{pr}_1$} (m-2-2);
			\end{tikzpicture}
		\end{figure}
		
		We call $E$ and $B$ the \textbf{total space} and \textbf{base space} respectively, $F$ the \textbf{(typical) fibre}, $\varphi_i$ a \textbf{local trivialization}\footnote{This name follows from the fact that the bundle is locally homeomorphic to a product space: $E\cong U\times\mathbb{R}^n$, which is the definition of a trivial bundle (see \ref{manifolds:trivial_bundle}).}, $(U_i, \varphi_i)$ a \textbf{bundle chart}\footnote{This is due to the similarities with the charts as defined for manifolds.} and the set $\{(U_i, \varphi_i)\}_{i\in I}$ a \textbf{trivializing cover}.
		
		The transition maps $\varphi_j\circ\varphi_i^{-1}:(U_i\cap U_j)\times F\rightarrow (U_i\cap U_j)\times F$ can be identified with the cocycle\footnote{See definition \ref{group:cocycle}.} $g_{ji}:U_i\cap U_j\rightarrow G$, associated to the (left) action, which we require to be faithful\footnote{See definition \ref{group:faithful_action}.}, of $G$ on every fibre, by the following relation:
		\begin{equation}
			\varphi_j\circ\varphi_i^{-1}(b, x) = (b, g_{ji}(b)\cdot x)
		\end{equation}
	}
	\begin{remark}
		One should pay attention that the bundle charts are not coordinate charts in the original sense \ref{manifolds:chart} because the image of $\varphi_i$ is not an open subset of $\mathbb{R}^n$. However they serve the same purpose and we can still use them to locally inspect the total space $P$.
	\end{remark}
	\begin{notation}
		A fibre bundle $(E, B, \pi, F, G)$ is often indicated by the following diagram:
		
		\begin{figure}[ht!]
			\centering
			\begin{tikzpicture}
				\matrix (m) [matrix of math nodes,row sep=2em,column sep=2em,minimum width=2em, ampersand replacement=\&]{
				F \& E\\
				\& B\\
			};
			\path[-stealth]
				(m-1-1) [right hook ->] edge node [above] {} (m-1-2)
				(m-1-2) [->] edge node [right] {$\pi$} (m-2-2);
			\end{tikzpicture}
		\end{figure}
		or more compactly $F\hookrightarrow E\xrightarrow{\ \pi\ }{B}$. A drawback of these notations is that we do not immediately know what the structure group of the bundle is.
	\end{notation}

	\newdef{Fibre}{\index{fibre}
		Let $F\hookrightarrow E\xrightarrow{\ \pi\ }B$ be a fibre bundle over a base space $B$. The fibre over $b\in B$ is defined as the set $\pi^{-1}(b)$.
	}
	
	\newdef{Smooth fibre bundle}{
		A smooth fibre bundle is a fibre bundle\\$(E, B, \pi, F, G)$ with the following constraints:
		\begin{itemize}
			\item The base space $B$ and typical fibre $F$ are smooth manifolds.
			\item The structure goup $G$ is a Lie group.
			\item The projection map, trivializing maps and transition functions are diffeomorphisms.
		\end{itemize}
	}
	\begin{remark}
		A smooth fibre bundle is also a smooth manifold.
	\end{remark}
	
	\newdef{Compatible\footnotemark\ bundle charts}{\index{compatible!bundle charts}
		\footnotetext{Also called an \textbf{admissible chart}.}
		A bundle chart $(U, \varphi)$ is compatible with a trivializing cover $\{(U_i, \varphi_i)\}_{i\in I}$ if whenever $U\cap U_i\neq\emptyset$ their exists a map $h_i:U\cap U_i\rightarrow G$ such that:
		\begin{equation}
			\varphi\circ\varphi_i^{-1}(b, x) = (b, h_i(b)x)
		\end{equation}
		for all $b\in U\cap U_i$ and $x\in F$. Two trivializing covers are \textit{equivalent} if all bundle charts are cross-compatible. As in the case of manifolds, this gives rise to the notion of a \textbf{G-atlas}. A \textbf{G-bundle} is then defined as a fibre bundle eqipped with an equivalence class of $G$-atlases.
	}
	
	\newdef{Equivalent fibre bundles}{
		Two fibre bundles $\pi_1:F_1\rightarrow B$ and $\pi_2:F_2\rightarrow B$ (over the same base space $B$) are equivalent if there exist trivializing covers\footnotemark\ $\{(U_i, \varphi_i)\}_{i\in I}$ and $\{(U_i, \varphi'_i)\}_{i\in I}$ and a family of smooth functions $\{\rho_i:U_i\rightarrow G\}_{i\in I}$ such that:
		\begin{equation}
			g'_{ji}(b) = \rho_j(b)\circ g_{ji}(b)\circ\rho_i^{-1}(b)
		\end{equation}
		for every $b\in U_i\cap U_j$.
		\footnotetext{Remark that the collection $\{U_i\}_{i\in I}$ is the same for both trivializing covers.}
	}
	\begin{property}
		Two fibre bundles over the same base space are equivalent if and only if they are isomorphic\footnotemark. Furthermore, if there exists a bundle map between two fibre bundles over the same base space, then they are equivalent.
		\footnotetext{Two fibre bundles $F$ and $G$ are isomorphic if there exist bundle maps $f:F\rightarrow G$ and $g:G\rightarrow F$ such that $f\circ g = \mathbbm{1}_G$ and $g\circ f = \mathbbm{1}_F$.}
	\end{property}
	
	\newdef{Trivial bundle}{\label{manifolds:trivial_bundle}
		A fibre bundle $(E, B, \pi, F)$ is trivial if $E = B\times F$.
	}
	\newdef{Trivialization}{
		A trivialization of a fibre bundle $\xi$ is an equivalence $\xi\rightarrow B\times F$. Bundles for which a trivialization can be found are also called \textit{trivial bundles}.
	}
	
	\newdef{Fibre product}{\index{fibre!product}
		Let $(F_1, B, \pi_1)$ and $(F_2, B, \pi_2)$  be two fibre bundles on a base space $B$. Their fibre product is defined as:
		\begin{equation}
			\label{manifolds:fibre_product}
			F_1\diamond F_2 = \{f\times g\in F_1\times F_2: \pi_1(f) = \pi_2(g)\}
		\end{equation}
	}
	
\subsection{Sections}

	\newdef{Section}{\index{section}
		A \textbf{global} section on a fibre bundle $\pi:E\rightarrow B$ is a smooth function $s:B\rightarrow E$ such that $\pi\circ s = \mathbbm{1}_B$. For any open subset $U\subset B$ we define a local section as a smooth function $s_U:U\rightarrow E$ such that $\pi\circ s_U(b) = b$ for all $b\in U$.
	}
	\begin{notation}
		The set of all global sections on a bundle $E$ is denoted by $\Gamma(E)$. The set of local sections on $U\subset E$ is similarly denoted by $\Gamma(U)$.
	\end{notation}

\subsection{Reduction of the structure group}

	\begin{construct}
		Consider a fibre bundle $\mathcal{F} = (E, B, \pi, F, G)$. Let $H$ be a subgroup of $G$. If there exists a fibre bundle with structure group $H$ equivalent to $\mathcal{F}$ then we say that the structure group $G$ can be reduced to $H$.
	\end{construct}
	
	\begin{property}\index{orientation}
		An $n$-dimensional manifold is orientable if and only if the structure group $GL(\mathbb{R}^n)$ of its frame bundle $F(M)$ is reducible to $GL^+(\mathbb{R}^n)$, i.e. the group of invertible matrices with positive determinant. 
	\end{property}
	
	\newdef{$G$-structure}{\index{$G$-structure}
		Consider a manifold $M$. A $G$-structure on $M$ is the reduction of the structure group $GL(\mathbb{R}^n)$ of the frame bundle $F(M)$ to a subgroup $G\subset GL(\mathbb{R}^n)$.
	}
	\begin{example}
		An $O(n)$-structure on $M$ turns the manifold into a Riemannian manifold\footnote{See definition \ref{riemann:riemannian_manifold}.}.
	\end{example}

\subsection{Jet bundles}

	\newdef{Jet}{\index{jet}
		Consider a fibre bundle $(E, B, \pi)$ with its sections $\Gamma(E)$. Two sections $\sigma, \xi\in\Gamma(E)$ with local coordinates $(\sigma^i)$ and $(\xi^i)$ define the same $r$-jet at a point $p\in B$ if and only if:
		\begin{equation}
			\left.\mpderiv{\alpha}{\sigma^i}{x}\right|_p = \left.\mpderiv{\alpha}{\xi^i}{x}\right|_p
		\end{equation}
		for all $0\leq i\leq \dim E$ and every multi-index $\alpha$ such that $0\leq|\alpha|\leq r$. It is clear that this relation defines an equivalence relation. The $r$-jet at $p\in B$ with representative $\sigma$ is denoted by $j_p^r\sigma$. The number $r$ is called the \textbf{order} of the jet.
	}
	
	\newdef{Jet manifold}{
		Consider a fibre bundle $(E, B, \pi)$. The $r$-jet manifold $J^r(\pi)$ of the projection $\pi$ is defined as:
		\begin{equation}
			J^r(\pi) = \{j_p^r\sigma: \sigma\in\Gamma(E), p\in B\}
		\end{equation}
		The set $J^0(\pi)$ is identified with the total space $E$.
	}
	
	\newdef{Jet projections}{
		Let $(E, B, \pi)$ be a fibre bundle with $r$-jet manifolds $J^r(\pi)$. The \textbf{source projection} $\pi_r$ and \textbf{target projection} $\pi_{r, 0}$ are defined as the maps
		\begin{align}
			\pi_r&:J^r(\pi)\rightarrow B:j_p^r\sigma\mapsto p\\
			\pi_r&:J^r(\pi)\rightarrow E:j_p^r\sigma\mapsto \sigma(p)
		\end{align}
		These projections satisfy $\pi_r = \pi\circ\pi_{r, 0}$. We can also define a \textbf{$k$-jet projection} $\pi_{r, k}$ as the map
		\begin{equation}
			\pi_{r, k}:J^r(\pi)\rightarrow J^k(\pi):j_p^r\sigma\mapsto j_p^k\sigma
		\end{equation}
		where $k\leq r$. The $k$-jet projections satisfy a transitivity property $j_{k, m} = j_{r, m}\circ j_{k, r}$.
	}
	\newdef{Jet prolongation}{
		Let $\sigma$ be a section on a fibre bundle $(E, B, \pi)$. The $r$-jet prolongation $j^r\sigma$ corresponding to $\sigma$ is defined as the following map:
		\begin{equation}
			j^r\sigma:B\rightarrow J^r(\pi):p\mapsto j_p^r\sigma
		\end{equation}
	}
	
	\newdef{Jet bundle}{
		The $r$-jet bundle corresponding to the projection $\pi$ is then defined as the triple $(J^r(\pi), B, \pi_r)$. The bundle charts $(U_i, \varphi_i)$\footnote{Where $\varphi_i = (x^k, u^\alpha)$ with $x^k$ the base space coordinates and $u^\alpha$ the total space coordinates.} on $E$ define \textit{induced} bundle charts on $J^r(\pi)$ in the following way:
		\begin{align}
			U_i^r &= \{j_p^r\sigma: \sigma(p)\in U_i\}\\
			\varphi_i^r &= \left(x^k, u^\alpha, \left.\mpderiv{I}{u^\alpha}{x}\right|_p \right)
		\end{align}
		where $I$ is a multi-index such that $0\leq|I|\leq r$. The partial derivatives $\left.\mpderiv{I}{u^\alpha}{x}\right|_p$ are called the \textbf{derivative coordinates} on $J^r(\pi)$.
	}

\section{Principal bundles}

	\begin{definition}[Principal bundle]\index{principal bundle}\index{local trivializations}\index{fibre}\index{structure group}
		A principal bundle is a fibre bundle $(E, B, \pi, G, G)$ such that the structure group and the typical fibre are the same, i.e. we identify the structure group with the group of left translations of $G$.
	\end{definition}
	\begin{remark}\index{torsor}
		We remark that although the fibres are homeomorphic to $G$, they do not carry a group structure due to the lack of a distinct identity element. This turns them into \textbf{G-torsors}. However it is possible to locally (i.e. in a neighbourhood of a point $p\in M$), but not globally, endow the fibres with a group structure by choosing an element of every fibre to be identity element.
	\end{remark}
	
	\begin{property}
		The dimension of $P$ is given by:
		\begin{equation}
			\label{manifolds:principal_bundle_dimension}
			\dim P = \dim M + \dim G
		\end{equation}
	\end{property}

	\begin{property}
		Let $\pi:P\rightarrow B$ be a principal $G$-bundle with local trivializations $\{(U_i, \varphi_i)\}_{i\in I}$. There exists a (faithful) right action of $G$ on $P$ given by:
		\begin{equation}
			z\cdot g = \varphi_i^{-1}(b, hg)
		\end{equation}
		for all $g, h\in G$ and $z\in\pi(U_i)$. This action preserves fibres ($y\cdot g\in F_b$ for all $y\in F_b, g\in G$). Furthermore, it is free\footnotemark\ and it is transitive. It follows that the fibres over $B$ are exactly the orbits of the right action on $P$.
		\footnotetext{See definition \ref{group:free_action}.}
		
		Every local trivialization $\varphi_i$ is also $G$-equivariant:
		\begin{equation}
			\varphi_i(z\cdot g) = \varphi_i(z)\cdot g
		\end{equation}
	\end{property}
	
	\newdef{Bundle map}{\index{bundle map}
		A bundle map $F:P_1\rightarrow P_2$ between principal $G$-bundles is a pair of smooth maps $(f_B, f_E)$ such that:
		\begin{enumerate}
			\item The diagram \ref{tikz:principal_bundle_map} below commutes.
			\item $f_E$ is $G$-equivariant\footnotemark.
		\end{enumerate}
		The map $f_E$ is said to \textbf{cover} $f_B$.
		\footnotetext{See definition \ref{group:equivariant}.}

		\begin{figure}[ht!]
		\centering	
		\begin{tikzpicture}
			\matrix (m) [matrix of math nodes,row sep=3em,column sep=4em,minimum width=2em, ampersand replacement=\&]{
				E_1 \& E_2\\
				B_1 \& B_2\\
			};
			\path[-stealth]
				(m-1-1) edge node [left] {$\pi_1$} (m-2-1)
					edge node [above] {$f_E$} (m-1-2)
				(m-2-1) edge node [below] {$f_B$} (m-2-2)
				(m-1-2) edge node [right] {$\pi_2$} (m-2-2);
		\end{tikzpicture}
		\caption{Bundle map between principal $G$-bundles.}
		\label{tikz:principal_bundle_map}
		\end{figure}
	}
	
	\begin{example}[Associated principal bundle]
		For every fibre bundle $(E, B, \pi, F, G)$ we can construct an associated principal $G$-bundle by replacing the fibre $F$ by $G$ itself.
	\end{example}
	\begin{property}
		A fibre bundle $\xi$ is trivial if and only if the associated principal bundle is trivial. More generally, two fibre bundles are isomorphic if and only if their associated principal bundles are isomorphic.
	\end{property}
	
	\begin{example}[Frame bundle]\index{frame bundle}\index{frame}
		Let $V$ be an $n$-dimensional vector space. Denote the set of ordered bases, also called \textbf{frames}, of $V$ by $F(V)$. This set is isomorphic to the group $GL(\mathbb{R}^n)$ which follows from the fact that every basis transformation is given by the action of an element of the general linear group. We can thus construct a principal bundle associated to the vector bundle $E$ by replacing every fibre $\pi^{-1}(b)$ by $F(\pi^{-1}(b))\cong$ GL$(\mathbb{R}^n)$.
	\end{example}
	
\subsection{Sections}
	Where every vector bundle has at least one global section, the \textbf{zero section}\footnotemark, a general principal bundle does not necessarily have a global section. This is made clear by the following property:
	\footnotetext{This is the map $s:b\rightarrow\vector{0}$ for all $b\in B$.}
	\begin{property}
		A principal $G$-bundle $P$ is trivial if and only if there exists a global section on $P$. Furthermore, there exists a bijection between the set of all global sections $\Gamma(P)$ and the set of trivializations $\text{Triv}(P)$.
	\end{property}

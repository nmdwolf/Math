\chapter{Bundle theory}

\section{Fibre bundles}

	\newdef{Fibre bundle}{\index{fibre!bundle}\index{local!trivialization}\index{structure group}
		\label{manifolds:fibre_bundle}
		A fibre bundle is a tuple $(E, B, \pi, F, G)$ where $E, B$ and $F$ are topological spaces and $G$ is a topological group (called the \textbf{structure group}), such that there exists a smooth surjective map $\pi:E\rightarrow B$ and an open cover $\{U_i\}_{i\in I}$ of $B$ for which there exists a family of homeomorphisms $\{\varphi_i:\pi^{-1}(U_i)\rightarrow U_i\times F\}_{i\in I}$ that make the following diagram commute:
		\begin{figure}[ht!]
			\centering
			\begin{tikzpicture}
				\matrix (m) [matrix of math nodes,row sep=2em,column sep=3em,minimum width=2em, ampersand replacement=\&]{
					\pi^{-1}(U_i) \& \& U_i\times F\\
					\& U_i\&\\
				};
				\path[-stealth]
					(m-1-1) edge node [above] {$\varphi_i$} (m-1-3)
					(m-1-1) edge node [below left] {$\pi$} (m-2-2)
					(m-1-3) edge node [below right] {$\text{pr}_1$} (m-2-2);
			\end{tikzpicture}
		\end{figure}
		
		We call $E$ and $B$ the \textbf{total} space and \textbf{base} space respectively, $F$ the \textbf{(typical) fibre}, $\varphi_i$ a \textbf{local trivialization}\footnote{This name follows from the fact that locally we have $E\cong U\times\mathbb{R}^n$, which is the definition of a trivial bundle (see \ref{manifolds:trivial_bundle}).}, $(U_i, \varphi_i)$ a \textbf{bundle chart}\footnote{This is due to the similarities with the charts as defined for manifolds.} and the set $\{(U_i, \varphi_i)\}_{i\in I}$ a \textbf{trivializing cover}.
		
		The transition maps $\varphi_j\circ\varphi_i^{-1}:(U_i\cap U_j)\times F\rightarrow (U_i\cap U_j)\times F$ can be identified with the cocycle\footnote{See definition \ref{group:cocycle}.} $g_{ji}:U_i\cap U_j\rightarrow G$, associated to the (left) action, which we require to be faithful\footnote{See definition \ref{group:faithful_action}.}, of $G$ on every fibre, by the following relation:
		\eq{
			\varphi_j\circ\varphi_i^{-1}(b, x) = (b, g_{ji}(b)\cdot x)
		}
	}
	\begin{remark}
		One should pay attention that the bundle charts are not coordinate charts in the original sense \ref{manifolds:chart} because the image of $\varphi_i$ is not an open subset of $\mathbb{R}^n$. However they serve the same purpose and we can still use them to locally inspect the total space $P$.
	\end{remark}
	\begin{notation}
		A fibre bundle $(E, B, \pi, F, G)$ is often indicated by the following diagram:
		
		\begin{figure}[ht!]
			\centering
			\begin{tikzpicture}
				\matrix (m) [matrix of math nodes,row sep=2em,column sep=2em,minimum width=2em, ampersand replacement=\&]{
				F \& E\\
				\& B\\
			};
			\path[-stealth]
				(m-1-1) [right hook ->] edge node [above] {} (m-1-2)
				(m-1-2) [->] edge node [right] {$\pi$} (m-2-2);
			\end{tikzpicture}
		\end{figure}
		or more compactly $F\hookrightarrow E\xrightarrow{\ \pi\ }{B}$. A drawback of these notations is that we do not immediately know what the structure group of the bundle is.
	\end{notation}

	\newdef{fibre}{\index{fibre}
		Let $F\hookrightarrow E\xrightarrow{\ \pi\ }B$ be a fibre bundle over a base space $B$. The fibre over $b\in B$ is defined as the set $\pi^{-1}(b)$.
	}
	
	\newdef{Smooth fibre bundle}{
		A smooth fibre bundle is a fibre bundle\\$(E, B, \pi, F, G)$ with the following constraints:
		\begin{itemize}
			\item The base space $B$ and typical fibre $F$ are smooth manifolds.
			\item The structure goup $G$ is a Lie group.
			\item The projection map, trivializing maps and transition functions are diffeomorphisms.
		\end{itemize}
	}
	\begin{remark}
		A smooth fibre bundle is also a smooth manifold.
	\end{remark}
	
	\newdef{Compatible\footnotemark\ bundle charts}{\index{compatible!bundle charts}
		\footnotetext{Also called an \textbf{admissible chart}.}
		A bundle chart $(U, \varphi)$ is compatible with a trivializing cover $\{(U_i, \varphi_i)\}_{i\in I}$ if whenever $U\cap U_i\neq\emptyset$ their exists a map $h_i:U\cap U_i\rightarrow G$ such that:
		\eq{
			\varphi\circ\varphi_i^{-1}(b, x) = (b, h_i(b)x)
		}
		for all $b\in U\cap U_i$ and $x\in F$. Two trivializing covers are \textit{equivalent} if all bundle charts are cross-compatible. As in the case of manifolds, this gives rise to the notion of a \textbf{G-atlas}. A \textbf{G-bundle} is then defined as a fibre bundle eqipped with an equivalence class of $G$-atlases.
	}
	
	\newdef{Equivalent fibre bundles}{
		Two fibre bundles $\pi_1:F_1\rightarrow B$ and $\pi_2:F_2\rightarrow B$ (over the same base space $B$) are equivalent if there exist trivializing covers\footnotemark\ $\{(U_i, \varphi_i)\}_{i\in I}$ and $\{(U_i, \varphi'_i)\}_{i\in I}$ and a family of smooth functions $\{\rho_i:U_i\rightarrow G\}_{i\in I}$ such that:
		\eq{
			g'_{ji}(b) = \rho_j(b)\circ g_{ji}(b)\circ\rho_i^{-1}(b)
		}
		for every $b\in U_i\cap U_j$.
		\footnotetext{Remark that the collection $\{U_i\}_{i\in I}$ is the same for both trivializing covers.}
	}
	\begin{property}
		Two fibre bundles over the same base space are equivalent if and only if they are isomorphic\footnotemark. Furthermore, if there exists a bundle map between two fibre bundles over the same base space, then they are equivalent.
		\footnotetext{Two fibre bundles $F$ and $G$ are isomorphic if there exist bundle maps $f:F\rightarrow G$ and $g:G\rightarrow F$ such that $f\circ g = \mathbbm{1}_G$ and $g\circ f = \mathbbm{1}_F$.}
	\end{property}
	
	\newdef{Trivial bundle}{\label{manifolds:trivial_bundle}
		A fibre bundle $(E, B, \pi, F)$ is trivial if $E = B\times F$.
	}
	\newdef{Trivialization}{
		A trivialization of a fibre bundle $\xi$ is an equivalence $\xi\rightarrow B\times F$. Bundles for which a trivialization can be found are also called \textit{trivial bundles}.
	}
	
	\newdef{Fibre product}{\index{fibre!product}
		Let $(F_1, B, \pi_1)$ and $(F_2, B, \pi_2)$  be two fibre bundles on a base space $B$. Their fibre product is defined as:
		\eq{
			\label{manifolds:fibre_product}
			F_1\diamond F_2 = \{f\times g\in F_1\times F_2: \pi_1(f) = \pi_2(g)\}
		}
	}



\section{Principal bundles}

	The following construction is very similar to the construction of a vector bundle (see \ref{manifolds:vector_bundle_construction}).
	\begin{definition}[Principal bundle]\index{principal bundle}\index{local trivializations}\index{fibre}\index{structure group}
		A principal bundle is a fibre bundle $(E, B, \pi, G, G)$ such that the structure group and the typical fibre are the same, i.e. we identify the structure group with the group of left translations of $G$.
	\end{definition}	
	\begin{remark}\index{torsor}
		We remark that although the fibres are homeomorphic to $G$, they do not carry a group structure due to the lack of a distinct identity element. This turns them into \textbf{G-torsors}. However it is possible to locally (i.e. in a neighbourhood of a point $p\in M$), but not globally, endow the fibres with a group structure by choosing an element of every fibre to be identity element.
	\end{remark}
	
	\begin{property}
		The dimension of $P$ is given by:
		\begin{equation}
			\label{manifolds:principal_bundle_dimension}
			\dim P = \dim M + \dim G
		\end{equation}
	\end{property}

	\begin{property}
		Let $\pi:P\rightarrow B$ be a principal $G$-bundle with local trivializations $\{(U_i, \varphi_i)\}_{i\in I}$. There exists a (faithful) right action of $G$ on $P$ given by:
		\eq{
			z\cdot g = \varphi_i^{-1}(b, hg)
		}
		for all $g, h\in G$ and $z\in\pi(U_i)$. This action preserves fibres ($y\cdot g\in F_b$ for all $y\in F_b, g\in G$). Furthermore, it is free\footnotemark\ and it is transitive. It follows that the fibres over $B$ are exactly the orbits of the right action on $P$.
		\footnotetext{See definition \ref{group:free_action}.}
		
		 Every local trivialization $\varphi_i$ is also $G$-equivariant:
		\eq{
			\varphi_i(z\cdot g) = \varphi_i(z)\cdot g
		}
	\end{property}
	
	\newdef{Bundle map}{\index{bundle map}
		A bundle map $F:P_1\rightarrow P_2$ between principal $G$-bundles is a pair of smooth maps $(f_B, f_E)$ such that:
		\begin{enumerate}
			\item The diagram \ref{tikz:principal_bundle_map} below commutes.
			\item $f_E$ is $G$-equivariant\footnotemark.
		\end{enumerate}
		The map $f_E$ is said to \textbf{cover} $f_B$.
		\footnotetext{See definition \ref{group:equivariant}.}

		\begin{figure}[ht!]
		\centering	
		\begin{tikzpicture}
			\matrix (m) [matrix of math nodes,row sep=3em,column sep=4em,minimum width=2em, ampersand replacement=\&]{
				E_1 \& E_2\\
				B_1 \& B_2\\
			};
			\path[-stealth]
				(m-1-1) edge node [left] {$\pi_1$} (m-2-1)
					edge node [above] {$f_E$} (m-1-2)
				(m-2-1) edge node [below] {$f_B$} (m-2-2)
				(m-1-2) edge node [right] {$\pi_2$} (m-2-2);
		\end{tikzpicture}
		\caption{Bundle map between principal $G$-bundles.}
		\label{tikz:principal_bundle_map}
		\end{figure}
	}
	
	\begin{example}[Associated principal bundle]
		For every fibre bundle $(E, B, \pi, F, G)$ we can construct an associated principal $G$-bundle by replacing the fibre $F$ by $G$ itself.
	\end{example}
	\begin{property}
		A fibre bundle $\xi$ is trivial if and only if the associated principal bundle is trivial. More generally, two fibre bundles are isomorphic if and only if their associated principal bundles are isomorphic.
	\end{property}
	
	\begin{example}[Frame bundle]\index{frame bundle}\index{frame}
		Let $V$ be an $n$-dimensional vector space. Denote the set of ordered bases, also called \textbf{frames}, of $V$ by $F(V)$. This set is isomorphic to the group $GL(\mathbb{R}^n)$ which follows from the fact that every basis transformation is given by the action of an element of the general linear group. We can thus construct a principal bundle associated to the vector bundle $E$ by replacing every fibre $\pi^{-1}(b)$ by $F(\pi^{-1}(b))\cong$ GL$(\mathbb{R}^n)$.
	\end{example}
	
\subsection{Sections}

	Following definition is analogous to that for vector bundles.
	\newdef{Section}{\index{section}
		A \textbf{global} section on a fibre bundle $\pi:E\rightarrow B$ is a smooth function $s:B\rightarrow E$ such that $\pi\circ s = \mathbbm{1}_B$. For any open subset $U\subset B$ we define a local section as a smooth function $s_U:U\rightarrow E$ such that $\pi\circ s_U(b) = b$ for all $b\in U$.
	}
	\begin{notation}
		The set of all global sections on a bundle $E$ is denoted by $\Gamma(E)$. The set of local sections on $U\subset E$ is similarly denoted by $\Gamma(U)$.
	\end{notation}
	
	Where every vector bundle has at least one global section, the \textbf{zero section}\footnotemark, a general principal bundle does not necessarily have a global section. This is made clear by the following property:
	\footnotetext{This is the map $s:b\rightarrow\vector{0}$ for all $b\in B$.}
	\begin{property}
		A principal $G$-bundle $P$ is trivial if and only if there exists a global section on $P$. Furthermore, there exists a bijection between the set of all global sections $\Gamma(P)$ and the set of trivializations $\text{Triv}(P)$.
	\end{property}

\section{Vector bundles}
	The tangent space and tangent bundle, as introduced in subsection \ref{diff:section:tangent_space}, can also be introduced in a more topological way:
	
\subsection{Tangent bundles}

	\begin{construct}[Tangent bundle]\index{tangent!bundle}
		Let $M$ be a manifold with atlas $\{(U_i, \varphi_i)\}_{i\leq n}$. Consider for every open set $U$ an associated set $TU = U\times\mathbb{R}^n$. For every smooth function $f$ we can define an associated smooth function on $TU$, called the differential of $f$, by:
		\begin{equation}
			\label{diff:manifolds:T_function}
			Tf:U\times\mathbb{R}^n\rightarrow f(U)\times\mathbb{R}^n:(x, v)\mapsto(f(x), Df(x)v)
		\end{equation}
		where $Df(x):\mathbb{R}^n\rightarrow\mathbb{R}^n$ is the linear operator associated with the Jacobian matrix of $f$ in $x$. Applying this definition to the transition functions $\psi_{ji}$ we obtain a new set of functions $\widetilde{\psi}_{ji} := T\psi_{ji}$ given by:
		\begin{equation}
			\widetilde{\psi}_{ji}(\varphi_i(x), v) = \left(\varphi_j(x), (\varphi_j\circ\varphi_i^{-1})'(\varphi_i(x))v\right)
		\end{equation}
		Because the transition functions are diffeomorphisms the Jacobians are invertible. This implies that the maps $\widetilde\psi_{ji}$ are elements of $GL(\mathbb{R}^n)$. The tangent bundle is now obtained by applying the fibre bundle construction theorem \ref{manifolds:theorem:fibre_bundle_construction_theorem} to the triple $(M, \mathbb{R}^n, GL(\mathbb{R}^n))$ together with the cover $\{U_i\}_{i\leq n}$ and the cocycle $\{\widetilde\psi_{ji}\}_{i,j\in I}$.
	\end{construct}
	\begin{remark*}
		The charts in the atlas of the constructed bundle are sometimes called \textbf{natural charts}.
	\end{remark*}
	
	\begin{property}
		Let $M$ be an $n$-dimensional manifold. Using the natural charts on $TM$ which give a local homeomorphism \[\psi_i:TM\rightarrow U_i\times\mathbb{R}^n\cong\mathbb{R}^n\times\mathbb{R}^n\] we can see that $TM$ is isomorphic to $\mathbb{R}^{2n}$. This implies that the tangent bundle is a manifold of dimension $2n$.
	\end{property}
	
	\newdef{Tangent space}{\index{tangent!space}
		Let $x\in M$. The topological definition of the tangent space is given by the fibre
		\begin{equation}
			T_xM := \tau_M^{-1}(x)
		\end{equation}
		If we use the natural charts to map $T_xM$ to the set $\varphi_i(x)\times\mathbb{R}^n$, we see that $T_xM$ is isomorphic to $\mathbb{R}^n$ and thus also to $M$ itself. Furthermore, we can equip every fibre with the following vector space structure:
		\begin{align*}
			(x, v_1)+(x, v_2)&:=(x, v_1 + v_2)\\
			r(x, v)&:=(x, rv)
		\end{align*}
	}
	\begin{remark}
		Now it is clear that the rule "\textit{a vector is something that transforms like a vector}" stems from the fact that:
		\[\text{a vector }v\in T_xM\text{ is tangent to }\varphi_i(x)\text{ in a chart }(U_i, \varphi_i)\]
		if and only if
		\[D(\varphi_j\circ\varphi_i^{-1})(\varphi_i(x))v\text{ is tangent to }\varphi_j(x)\text{ in a chart }(U_j, \varphi_j)\]
		Comparing this property to \ref{diff:manifolds:tangent_curve_transformation}, we see that tangent vectors defined through equivalence classes of tangent curves are indeed tangent vectors according to our new construction.
	\end{remark}
	
	\newdef{Differential}{\index{differential}\label{manifolds:differential}
		The map $T$ from \ref{diff:manifolds:T_function} can be generalized to arbitrary smooth manifolds as the map $Tf:TM\rightarrow TN$. Furthermore, let $x\in U\subseteq M$ and let $V = f(U)$. By looking at the restriction of $Tf$ to $T_xM$, denoted by $T_xf$, we see that it maps $T_xU$ to $T_{f(x)}V$ (where $V=f(U)$) linearly. So $T_xf$ is a linear map on fibres.
	}
	
	\begin{property}
		The map $Tf: TM\rightarrow TN$ (see \ref{diff:manifolds:T_function}) has following properties\footnotemark:
		\begin{itemize}
			\item $T(\mathbbm{1}_M) = \mathbbm{1}_{TM}$
			\item Let $f, g$ be two smooth functions on smooth manifolds. Then $T(f\circ g) = Tf\circ Tg$.
		\end{itemize}
	\end{property}
	\footnotetext{This turns the map $T$ into a functor on the category of smooth manifolds. We can view $T$ as a functorial derivative.}
	
	\begin{remark*}
		We can also use a construction similar to that of the tangent bundle to reconstruct the original manifold $M$ from the sets $\varphi_i(U_i)$.
	\end{remark*}
	
	\newdef{Rank}{\index{rank}
		\label{manifolds:rank}
		Let $f:M\rightarrow N$ be a differentiable map between smooth manifolds. Using the fact that $Tf$ is a linear map of fibres\footnotemark, we define the rank of $f$ at $p\in M$ as the rank (as in \ref{linalgebra:image_rank}) of the differential $Tf:T_pM\rightarrow T_{f(p)}N$.
	}
	\footnotetext{See definition \ref{manifolds:differential}.}
	
	\begin{theorem}[Inverse function theorem]\index{inverse function theorem}\label{manifolds:theorem:inverse_function_theorem}
		A $C^\infty$ map $f:M\rightarrow N$ between smooth manifolds is locally homeomorphic (resp. locally diffeomorphic) if and only if its differential $Tf:T_pM\rightarrow T_pN$ is an isomorphism (resp. diffeomorphism) at $p$.
	\end{theorem}

\subsection{Vector bundles}

	Instead of restricting ourselves by letting the typical fibre be a Euclidean space with the same dimension as the base manifold, we can generalize the construction of the tangent bundle in the following way:
		
	\begin{construct}[Vector bundle]\index{vector!bundle}\index{fibre}\index{local!trivialization}
		\label{manifolds:vector_bundle_construction}
		Consider a smooth $n$-dimensional manifold $M$ with atlas $\{(U_i, \varphi_i)\}_{i\leq n}$, a cocycle $\{g_{ji}: U_i\cap U_j\rightarrow G\}_{i,j\leq n}$ with values in a Lie group $G$ and a smooth representation $\rho:G\rightarrow GL(V)$, where $V$ is a vector space.
		
		Now we can construct a new topological space $E$, similar to the construction of the tangent bundle, by taking the disjoint union of the sets $\varphi_i(U_i)\times V$ and quotienting out using the functions $\widetilde{g}_{ji}:(\varphi_i(x), v)\mapsto(\varphi_j(x), g_{ji}(x)\cdot v)$, where $g\cdot v\equiv \rho(g)v$. This gives us a set of natural charts\footnotemark\ $\{(\widetilde{U}_i, \widetilde{\varphi}_i)\}_{i\leq n}$, a projection map $\pi:E\rightarrow M$ induced by the local projection $\varphi_i(U_i)\times V\rightarrow\varphi_i(U_i)$ and a naturally defined vector space on every fibre $V_x:=\pi^{-1}(x)$. Furthermore every fibre $V_x$ is (although not necessarilly canonically) isomorphic to $V$.
		
		This set $E$ is called a \textbf{smooth vector bundle} over $M$ with \textit{typical fibre} $V$ and \textit{projection map} $\pi$.
		\footnotetext{We could instead use any other kind of topological space. The point is that a vector bundle is a fibre bundle \ref{manifolds:fibre_bundle} for which the typical fibres are vector spaces.}
	\end{construct}

	\begin{remark}
		As is also the case for tangent bundles (which are specific cases of vector bundles where the typical fibre has the same dimension as the manifold) the choice of charts on $E$ is not random. To preserve the structure of fibres, the use of the natural charts is imperative.
	\end{remark}
	\begin{remark}
		Vector bundles are smooth fibre bundles where the typical fibre is a vector space $V$ and the structure group is given by $GL(V)$.
	\end{remark}
	
	\newdef{Associated vector bundle}{\label{manifolds:associated_vector_bundle}
		Consider a representation\newline $\rho:GL(\mathbb{R}^n)\rightarrow GL(\mathbb{R}^l)$ and the cocycle $t_{ji} := D(\psi_{ji})\circ\varphi_i$ as defined for tangent bundles. The composition $\rho\circ t_{ji}:U_i\cap U_j\overset{t_{ji}}{\rightarrow} GL(\mathbb{R}^n) \overset{\rho}{\rightarrow} GL(\mathbb{R}^l)$ is again a cocycle and can thus be used to define a new vector bundle on $M$. The vector bundle $E = \rho(TM)$ so obtained is called the associated bundle of the tangent bundle induced by $\rho$.
	}
	\begin{remark}
		\label{manifolds:vector_principal_correspondence}
		It should also be noted that every vector bundle is associated to a principle $GL(V)$-bundle where the cocycles $g_{ji}$ now act by left multiplication on elements of $GL(V)$.
	\end{remark}

	\begin{example}[Contravariant vectors]\index{contravariant}
		By noting that the $k^{th}$ tensor power $\otimes^k$ induces a representation given by the tensor product of the representations, we can construct the bundle of $k^{th}$ order contravariant vectors $\otimes^k(TM)$ with the cocycle given by $x\mapsto t_{ji}(x)\otimes\cdots\otimes t_{ji}(x)$.
	\end{example}
	\begin{example}[Cotangent bundle]\index{covariant}\label{manifolds:cotangent_bundle}
		Another (smooth) representation is given by $A\mapsto (A^T)^{-1}=(A^{-1})^T$ for every linear map $A$. The vector bundle constructed this way, where the cocycle is given by $(t_{ji}^T)^{-1}$, is called the cotangent bundle on $M$ and is denoted by $T^*M$. Elements of the fibres are called covariant vectors or covectors.
	\end{example}
	\begin{notation}
		A combination of the cocycle $t_{ji}$ and its dual $(t_{ji}^T)^{-1}$ can also be used to define the bundle of $k^{th}$ order contravariant and $l^{th}$ order covariant vectors on $M$. This bundle is denoted by $T^{(k, l)}M$.
	\end{notation}
	
	\begin{example}[Pseudovectors]\index{pseudovector}
		If we consider the representation
		\begin{equation}
			\rho:A\mapsto \sgn\det(A)A
		\end{equation}
		we can construct a bundle similar to the tangent bundle. The sign of the cocycle functions $t_{ji}$ now has an influence on the fibres. Elements of these fibres are called \textbf{pseudovectors}.
	\end{example}
	
	\newdef{Subbundle}{\index{subbundle}
		A subbundle of a vector bundle $\pi:E\rightarrow M$ is a collection of subspaces $U_x$ of fibres $E_x$ that make up a vector bundle on their own.
	}
	
	\newdef{Whitney sum}{\index{Whitney!sum}
		Consider two vector bundles $W, W'$ with fibres $E, E'$ respectively. Then we can construct a new vector bundle $W\oplus W'$ by defining the new typical fibre to be the direct sum $E\oplus E'$, i.e. the fibre above b is given by $E_b\oplus E_b'$.  This operation is called the Whitney sum or direct sum of vector bundles.
	}
	
\subsection{Sections}
	\begin{remark*}
		Vector fields can be regarded as sections of the tangent bundle. Similarly, 1-forms can be regarded as sections of the cotangent bundle.
	\end{remark*}
	
	\newdef{Frame}{\index{frame}\label{diff:frame}
		A frame of a vector bundle $E$ is a tuple $(s_1, ..., s_n)$ of smooth sections such that $(s_1(b), ..., s_n(b))$ is a basis of the fibre $\pi^{-1}(b)$ for all $b\in B$.
	}
	\begin{property}
		A vector bundle is trivial if and only if there exists a frame of global sections.
	\end{property}

\section{Vector fields}
	\newdef{Vector field}{\index{vector!field}
		A smooth section $s\in\Gamma(TM)$ of the tangent bundle is called a vector field.
	}
	\begin{notation}
		The set of all vector fields on a manifold $M$ is often denoted by $\mathfrak{X}(M)$.
	\end{notation}
	
	\newdef{Pullback}{\index{pullback}
		Let $X$ be vector field on $M$ and let $\varphi:M\rightarrow N$ be a diffeomorphism between smooth manifolds. The pullback of $X$ along $\varphi$ is defined as:
		\begin{equation}
			\label{manifolds:pullback}
			(\varphi^*X)_p = T\varphi^{-1}(X_{\varphi(p)})
		\end{equation}
	}
	\newdef{Pushforward}{\index{pushforward}
		Let $X\in\mathfrak{X}(M)$ and let $\varphi:M\rightarrow N$ be a diffeomorphism between smooth manifolds. Using the differential $T\varphi$ we can define the pushforward of $X$ along $\varphi$ as:
		\begin{equation}
			\label{manifolds:pushforward}
			(\varphi_*X)_{\varphi(p)} = T\varphi(X_p)
		\end{equation}
		which we can rewrite using the pullback as:
		\begin{equation}
			\label{manifolds:pullback_pushforward}
			\varphi_*X = \varphi^{-1*}X
		\end{equation}
		Or equivalently we can define a vector field on $N$ by:
		\begin{equation}
			(\varphi_*X)_q(f) = X_{\varphi^{-1}(q)}(f\circ\varphi)
		\end{equation}
		for all smooth functions $f:N\rightarrow\mathbb{R}$ and points $q\in N$.
	}
	
	\begin{remark*}
		For both the pullback and pushforward, we need the map $\varphi$ to be a diffeomorphism. For differential forms this is only necessary for the definition of pushforwards. (See definitions \ref{forms:pullback} and \ref{forms:pushforward}).
	\end{remark*}
	
\subsection{Integral curves}
	\newdef{Integral curve}{\index{integral!curve}
		Let $X\in\mathfrak{X}(M)$ and let $\gamma:\ ]a, b[\rightarrow M$ be a smooth curve on $M$. $\gamma$ is said to be an integral curve of $X$ if:
		\begin{equation}
			\label{manifolds:integral_curve}
			\boxed{\gamma'(t) = X(\gamma(t))}
		\end{equation}
		for all $t\in]a,b[$ where we defined $\gamma'(t) := T\gamma(t, 1)$ using the functorial derivative \ref{diff:manifolds:T_function}.
		
		This equation can be seen as a system of ordinary differential equations in the second argument. Using Picard's existence theorem\footnotemark\ together with the initial value condition $\gamma(0) = p$ we can find a unique curve on $]a, b[$ satisfying the defining equation \ref{manifolds:integral_curve}. Furthermore we can extend the interval $]a, b[$ to a maximal interval such that the solution is still unique. This solution, denoted by $\gamma_p$, is called the \textbf{integral curve of $X$ through $p$}.
		\footnotetext{Also Picard-Lindel\"of theorem.}
	}
	
	\newdef{Flow}{\index{flow}
		Let $X\in\mathfrak{X}(M)$. The function $\sigma_t$:
		\begin{equation}
			\label{manifolds:flow}
			\sigma_t(p) = \gamma_p(t)
		\end{equation}
		is called the flow of $X$ at time $t$. The flow domain is defined as the set $D(X) = \{(t, p)\in\mathbb{R}\times M\ |\ t\in ]a_p, b_p[\}$ where $]a_p, b_p[$ is the maximal interval on which $\gamma_p(t)$ is defined.
	}
	\begin{property}
		Suppose that $D(X) = \mathbb{R}\times M$. The flow $\sigma_t$ has following properties for all $s, t\in\mathbb{R}$:
		\begin{itemize}
			\item $\sigma_0 = \mathbbm{1}_M$
			\item $\sigma_{s+t} = \sigma_s\circ\sigma_t$
			\item $\sigma_{-t} = (\sigma_t)^{-1}$
		\end{itemize}
		These three properties\footnote{The third property follows from the other two.}\ say that $\sigma_t$ is a bijective group action from $M$ to the additive group of real numbers. This implies that $\sigma_t$ is indeed a \textbf{flow} in the general mathematical sense.
	\end{property}
	
	\newdef{Complete vector field}{\index{complete!vector field}
		A vector field $X$ is called complete if the flow domain for every flow is whole $\mathbb{R}$.
	}
	
	\begin{property}
		The flow $\sigma_t$ of a vector field is of class $C^\infty$. If $X$ is complete it follows from previous definition that the flow is a diffeomorphism from $M$ onto itself.
	\end{property}

	
\subsection{Lie derivative}

	\newformula{Lie derivative for smooth functions}{\index{Lie!derivative}
		Let $X\in\mathfrak{X}(M)$ and let $f:M\rightarrow\mathbb{R}$ be a smooth function. The Lie derivative of $f$ with respect to $X$ at $p\in M$ is defined as:
		\begin{equation}
			\label{manifolds:lie_derivative_functions}
			\boxed{(\mathcal{L}_Xf)(p) = \lim_{t\rightarrow0}\stylefrac{f(\gamma_p(t)) - f(p)}{t}}
		\end{equation}
		which closely resembles the standard derivative in Euclidean space.
	}
	
	\begin{formula}[$\dag$]\label{manifolds:ex:lie_derivative_function}
		Working out previous formula and rewriting it as an operator equality gives:
		\begin{equation}
			\label{manifolds:lie_derivative_function_expansion}
			\boxed{\mathcal{L}_X = \sum_kX_k\pderiv{}{x^k}}
		\end{equation}
		It is clear that this is just the vector field $X$ expanded in the basis \ref{diff:manifolds:tangent_vector_partial}. We also recover the behaviour of a tangent vector as a derivation. So for smooth functions $f:M\rightarrow\mathbb{R}$ we obtain:
		\begin{equation}
			\mathcal{L}_Xf(p) = X_p(f)
		\end{equation}
	\end{formula}
	
	\newformula{Lie derivative for vector fields$^\dag$}{\label{manifolds:ex:lie_derivative_vector_fields}
		Let $X, Y\in\mathfrak{X}(M)$
		\begin{equation}
			\label{manifolds:lie_derivative_vector_field}
			\boxed{\mathcal{L}_XY = \left.\deriv{}{t}(\sigma_t^*X)(\gamma_p(t))\right|_{t=0}}
		\end{equation}
	}
	\begin{property}\index{Lie!bracket}
		Let $X, Y\in\mathfrak{X}(M)$ be vector fields of class $C^k$. The Lie derivative has following properties:
		\begin{itemize}
			\item $\mathcal{L}_XY$ is a vector field.
			\item \textbf{Lie bracket}:
				\begin{equation}
					\label{manifolds:lie_bracket}
					\mathcal{L}_XY = [X, Y]
				\end{equation}
				which is also a derivation on $C^{k-1}(M, \mathbb{R})$ due to the cancellation of second-order derivatives in the local representation.
			\item The Lie derivative is antisymmetric:
				\begin{equation}
					\mathcal{L}_XY = -\mathcal{L}_YX
				\end{equation}
				This follows from the previous property.
		\end{itemize}
	\end{property}
	
\subsection{Frobenius theorem}

	Looking at property \ref{linalgebra:grassmannian_construction} and noting that GL$_n(\mathbb{C})$ is a Lie group, we can endow the Grassmannian Gr$(k, \mathbb{R}^n)$ \ref{linalgebra:grassmannian} with a differentiable structure, turning it into a smooth manifold. With this we can construct a new bundle\footnotemark\ by applying the usual 'gluing' process:
	\footnotetext{Due to the fact that the Grassmannian is not a vector space, we construct a general fibre bundle and not a vector bundle.}
	
	\begin{construct}[Grassman bundle]\index{Grassman!bundle}
		First define the transition functions:
		\begin{equation}
			\psi_{ji}:\varphi_i(U_i\cap U_j)\times \text{Gr}(k, \mathbb{R}^n) \rightarrow \varphi_j(U_i\cap U_j)\times \text{Gr}(k, \mathbb{R}^n):(\varphi_i(x), V)\mapsto(\varphi_j(x), t_{ji}(x)\cdot V)
		\end{equation}
		where $\{t_{ji}\}_{i, j\leq n}$ is the tangent bundle cocycle, but now with an action on the compact manifold Gr$(k, \mathbb{R}^n)$ instead of the vector space $\mathbb{R}^n$. Using this set of transition functions we use a construction similar to \ref{manifolds:vector_bundle_construction} to create a new (fibre) bundle where every fibre is diffeomorphic to Gr$(k, \mathbb{R}^n)$, namely it is the Grassmannian Gr$(k, T_pM)$ associated to the tangent space in every point $p\in M$.
	\end{construct}
	\begin{notation}
		The Grassman $k$-plane bundle is denoted by Gr$(k, TM)$.
	\end{notation}
	
	\newdef{Distribution}{\index{distribution!of $k$-planes}\label{manifolds:distribution}
		A smooth section of the Grassman $k$-plane bundle is called a distribution of $k$-planes.
	}
	
	\begin{definition}[Integrable]\index{integrable!manifold}
		Let $M$ be a smooth manifold and let $W\in\Gamma(\text{Gr}(k, TM))$ be a distribution of $k$-planes. A submanifold $N\subseteq M$ is said to integrate $W$ with initial condition $p_0\in M$ if for every $p\in N$ we find that $W(p) = T_pN$ and $p_0\in N$. $W$ is said to be integrable if there exists such a submanifold $N$.
	\end{definition}
	
	\newdef{Frobenius integrability condition}{\index{Frobenius!integrability condition}
		A distribution of $k$-planes $W$ over a smooth manifold $M$ is said to satisfy the Frobenius integrability condition in an open set $U\subseteq M$ if for every two vector fields $X, Y$ defined on $U$, such that $X(p)\in W(p)$ and $Y(p)\in W(p)$ for all $p\in U$, there Lie bracket $[X, Y](p)$ is also an element of $W(p)$ for all $p\in U$.
	}
	\begin{theorem}[Frobenius' integrability theorem]\index{Frobenius!integrability theorem}\label{manifolds:frobenius}
		Let $W$ be a distribution of $k$-planes over a smooth manifold $M$. Then $W$ is integrable if and only if $W$ satisfies the Frobenius integrability condition.
	\end{theorem}

\section{Differential \texorpdfstring{$k$}{k}\ -forms}

	\newdef{Differential form}{\index{differential form}
		A differential $k$-form is a map
		\begin{equation}
			\boxed{\omega: T^{\diamond k}M\rightarrow \mathbb{R}}
		\end{equation}
		such that the restriction of $\omega$ to each fibre of the fibre product\footnotemark\ $T^{\diamond k}M$ is multilinear and antisymmetric.
		
		The space of all differential $k$-forms on a manifold $M$ is denoted by $\Omega^k(M)$. $\Omega^0(M)$ is defined as the space of smooth functions $C^\infty:M\rightarrow\mathbb{R}$.
	}
	\footnotetext{See definition \ref{manifolds:fibre_product}.}
	
	\begin{adefinition}
		An alternative definition goes as follows. Consider the representation \[\rho_k:GL(R^{m*})\rightarrow GL(\Lambda^k(\mathbb{R}^{m*})): T\mapsto T\wedge...\wedge T\] where $T$ is a linear map. This representation induces an associated vector bundle\footnotemark\ $\rho_k(\tau_M^*)$ of the cotangent bundle on $M$. A differential $k$-form is then given by a section of $\rho_k(\tau_M^*)$. $\Omega^k(M)$ can then be defined as follows: \[\Omega^k(M) = \Gamma(\rho_k(\tau_M^*))\]
	\end{adefinition}
	\footnotetext{See definition \ref{manifolds:associated_vector_bundle}.}
	
	\begin{construct}
		We can construct a graded algebra by equipping the graded vector space
		\begin{equation}
			\Omega(M) = \bigoplus_{k\geq0}\Omega^k(M)
		\end{equation}
		with the wedge product of differential forms (which is induced by the wedge product on $\Lambda^k(\mathbb{R}^{m*})$ through the alternative definition). This graded algebra is associative, graded-commutative and unital with the constant function $1\in C^{\infty}(M)$ as identity element.
	\end{construct}

	\newdef{Pullback}{\index{pullback}
		Let $f:M\rightarrow N$ be a smooth function between smooth manifolds and let $\omega$ be a differential $k$-form on $N$. The pullback of $\omega$ by $f$ is defined as:
		\begin{equation}
			\label{forms:pullback}
			\boxed{f^*(\omega) = \omega\circ Tf:TM\rightarrow\mathbb{R}}
		\end{equation}
		So $f^*$ can be seen as a map pulling elements from $T^*N$ back to $T^*M$.
	}
	\newdef{Pushforward}{\index{pushforward}
		Let $f:M\rightarrow N$ be a diffeomorphism between smooth manifolds and let $\omega$ be a differential $k$-form on $M$. The pushforward $\omega$ by $f$ is defined as:
		\begin{equation}
			\label{forms:pushforward}
			f_*(\omega): \omega\circ Tf^{-1}: TN\rightarrow\mathbb{R}
		\end{equation}
		or using the pullback:
		\begin{equation}
			f_*(\omega) = f^{-1*}(\omega)
		\end{equation}
	}
	\begin{remark*}
		Note that the pushforward of differential $k$-form is only defined for diffeomorphisms, in constrast to pullbacks which only require smooth functions. Furthermore this also explains why differential forms are the most valuable elements in differential geomeotry. Vector fields can't even be pulled back in general by smooth maps.
	\end{remark*}
	
	\newformula{Dual basis}{\index{basis}
		Consider the basis $\{\left.\pderiv{}{x_i}\right|_p\}_{i\leq n}$ from definition \ref{diff:manifolds:tangent_vector_partial} for the tangent space $T_pM$. From this set we can construct a natural dual basis for the cotangent space $T_p^*M$ using the natural pairing of these spaces:
		\begin{equation}
			\label{forms:basis}
			\left\langle\pderiv{}{x_i}, dx_j\right\rangle = \delta_{ij}
		\end{equation}
	}
	
\subsection{Exterior derivative}

	\newdef{Exterior derivative}{\index{exterior!derivative}\index{differential}\label{forms:def:exterior_derivative}
		The exterior derivative $d_k$ is a map defined on the graded algebra of differential $k$-forms:
		\begin{equation}
			d_k:\Omega^k(M)\rightarrow\Omega^{k+1}(M)
		\end{equation}
		For $k=0$ it is given by\footnotemark:
		\begin{equation}
			\label{forms:function_derivative}
			df = \sum_{i=1}^n\pderiv{f}{x_i}dx_i
		\end{equation}
		where we remark that the `infinitesimals' are in fact unit vectors with norm $1$. This formula can be generalized to higher dimensions as follows:
		\begin{equation}
			\label{forms:exterior_derivative}
			\boxed{d(fdx_{i_1}\wedge...\wedge dx_{i_k}) = df\wedge dx_{i_1}\wedge...\wedge dx_{i_k}}
		\end{equation}
	}
	\footnotetext{For $f\in\Omega^0(M)$, we call $df$ the \textbf{differential} of $f$.}
	
	\begin{result}
		It follows immediately from \ref{forms:exterior_derivative} that
		\begin{equation}
			d(dx_i) = 0
		\end{equation}
		for all $i\leq n$.
	\end{result}
	
	\begin{property}\index{Leibniz!rule}\label{forms:exterior_derivative_properties}
		The exterior derivatives have following properties:
		\begin{itemize}
			\item For all $k\geq 0$, for all $\omega\in\Omega^k(M)$: $d_k\circ d_{k+1} = 0$, so $\text{im}(d_k)\subseteq\text{ker}(d_{k+1})$.
			\item The exterior derivative is an $\mathbb{R}$-linear map.
			\item Graded Leibniz rule:
				\begin{equation}
					d(\omega_1\wedge\omega_2) = d\omega_1\wedge\omega_2 + (-1)^j\omega_1\wedge d\omega_2
				\end{equation}
				where $\omega_1\in\Omega^j(M), \omega_2\in\Omega^k(M)$.
			\item Let $f\in C^\infty(M)$: $f^*(d\omega) = d(f^*\omega)$ where $f^*$ denotes the pullback \ref{forms:pullback}.
		\end{itemize}
	\end{property}
	
	\begin{remark}[$\dag$]\index{gradient}\index{rotor}\index{divergence}\label{forms:vector_calculus}
		The gradient, rotor (curl) and divergence from standard vector calculus\footnote{See section \ref{vectorcalculus:nabla}.} can be rewritten using exterior derivatives as follows: Let $\vector{f} = (f_1, f_2, f_3)$ with $f_i$ smooth for every $i$ and let $f$ be a smooth function. Denote the canonical isomorphism between $\mathbb{R}^3$ and $\mathbb{R}^{3*}$ by $\sim$.
		\begin{empheq}[box=\fbox]{align}
			\sim df &= \nabla f \\
			\sim (\ast d\alpha) &= \nabla\times\vector{f} \\
			\ast d\omega &= \nabla\cdot\vector{f}
		\end{empheq}
		The properties in section \ref{vectorcalculus:mixed_properties} then follows from the identity $d^2 = 0 $.
	\end{remark}
	
	\begin{example}
		Let $f\in C^\infty(M, \mathbb{R})$. Let $\gamma$ be a curve on $M$. From the definition \ref{forms:basis} of the basis $\{dx_k\}_{k\leq n}$ we obtain following result:
		\begin{equation}
			\langle df(x), \gamma'(t) \rangle = \sum_k \pderiv{f}{x_k}(x)\gamma_k'(t) = (f\circ\gamma')(t)
		\end{equation}
	\end{example}
	
	\begin{example}
		An explicit formula for the exterior derivative of a 1-form $\Phi$ is:
		\begin{equation}
			\label{forms:1form_exterior_derivative}
			d\Phi(X, Y) = X(\Phi(Y)) - Y(\Phi(X)) - \Phi([X, Y])
		\end{equation}
	\end{example}

\subsection{Lie derivative}
	
	\newformula{Lie derivative of differential forms}{\index{Lie!derivative}
		\begin{equation}
			\label{manifolds:lie_derivative_forms}
			\boxed{\mathcal{L}_X\omega(p) = \lim_{t\rightarrow0}\stylefrac{\sigma_t^*\omega - \omega}{t}(p)}
		\end{equation}
	}

	\begin{formula}[Lie derivative of smooth functions]
		Using the definition of the exterior derivative of smooth functions \ref{forms:function_derivative} and the definition of the dual (cotangent) basis \ref{forms:basis} we can rewrite the Lie derivative \ref{manifolds:lie_derivative_function_expansion} as:
		\begin{equation}
			Xf(p)= df_p(X(p))
		\end{equation}
	\end{formula}
	
	\begin{property}
		The Lie derivative also has following Leibniz-type property with respect to differential forms:
		\begin{equation}
			\mathcal{L}_X(\omega (Y)) = (\mathcal{L}_X\omega)(Y) + \omega(\mathcal{L}_XY)
		\end{equation}
		where $X, Y$ are two vector fields and $\omega$ is a 1-form.
	\end{property}
	
	\newformula{Lie derivative of tensor fields}{
		By comparing the definitions of the Lie derivatives of vector fields \ref{manifolds:lie_derivative_vector_field} and differential forms \ref{manifolds:lie_derivative_forms} we can see that both definitions are identical upon replacing $X$ by $\omega$. This leads to the definition of a Lie derivative of a general tensor field $\mathcal{T}\in\Gamma(T^{(k, l)}M)$:
		\begin{equation}
			\boxed{\mathcal{L}_X\mathcal{T}(p) = \left.\deriv{}{t}\sigma_t^*\mathcal{T}(\gamma_p(t))\right|_{t=0}}
		\end{equation}
	}
	
\subsection{Interior product}

	\newdef{Interior product}{\index{interior!product}
		Aside from the differential (exterior derivative) we can also define another operation on the algebra of differential forms:
		\begin{equation}
			\label{forms:interior_derivative}
			\iota_X:(\iota_X\omega)(v_1, ..., v_{k-1})\mapsto\omega(X, v_1, ..., v_{k-1})
		\end{equation}
		This antiderivation (of degree $-1$) from $\Omega^k(M)$ to $\Omega^{k-1}(M)$ is called the \textbf{interior product} or \textbf{interior derivative}. This can be seen as a generalization of the contraction map \ref{tensor:contraction}.
	}
	
	\newformula{Cartan\footnotemark}{\index{Cartan!formula for Lie derivative}
		\footnotetext{Sometimes called \textit{Cartan's magic formula} or \textit{Cartan's (infinitesimal) homotopy formula}.}
		Let $X$ be a vector field and let $\omega$ be a differential $k$-form. The Lie derivative of $\omega$ along $X$ is given by the following formula:
		\begin{equation}
			\label{forms:cartan_magic_formula}
			\mathcal{L}_X\omega = \iota_X(d\omega) + d(\iota_X\omega)
		\end{equation}
	}
	
\subsection{de Rham Cohomology}

	\newdef{Exact form}{\index{exact}
		Let $\omega\in\Omega^k(M)$. If $\omega$ can be written as $\omega = d\chi + 0$ for some $\chi\in\Omega^{k-1}(M)$ and $0\in\Omega^0(M)$ the zero function then $\omega$ is said to be exact. It follows that
		\begin{equation}
			\text{im}(d_k) = \{\omega\in\Omega^{k+1}(M):\omega\text{ is exact}\}
		\end{equation}
	}
	\newdef{Closed form}{\index{closed}
		Let $\omega\in\Omega^k(M)$. If $d\omega = 0$ then $\omega$ i said to be closed. It follows that
		\begin{equation}
			\{\omega\in\Omega^k(M):\omega\text{ is closed}\}\subseteq\text{ker}(d_k)
		\end{equation}
	}
	
	\begin{remark}\label{forms:remark:closed_exact}
		From the first item of property \ref{forms:exterior_derivative_properties} it follows that every exact form is closed. The converse however is not true\footnote{See result \ref{forms:theorem:poincare} for more information.}.
	\end{remark}

	
	\newdef{Cochain complex}{\index{cochain!complex}\index{boundary}\index{differential}\index{cocycle}
		Let $(A_k)_{k\in\mathbb{N}}$ be a sequence of Abelian groups or modules together with a sequence $(\partial_k)_{k\in\mathbb{N}}$ of homomorphisms, called the \textbf{boundary operators} or \textbf{differentials},  such that for all $k$:
		\begin{equation}
			\partial_k:A_k\rightarrow A_{k+1}
		\end{equation}
		Furthermore let $\partial_k^{\ 2} = 0$ for every $k\in\mathbb{N}$. This structure is called a cochain complex\footnotemark. Elements in $\text{im}(\partial_k)$ are called \textbf{coboundaries} and elements in $\text{ker}(\partial_k)$ are called \textbf{cocycles}.
		\footnotetext{A chain complex is constructed similarly. For this structure we consider a descending order, i.e.: $\partial_k:A_k\rightarrow A_{k-1}$.}
	}
	
	\newdef{de Rham complex}{\index{de Rham!complex}
		The structure given by the chain
		\begin{equation}
			0\rightarrow\Omega^0(M)\overset{d}{\rightarrow}\Omega^1(M)\overset{d}{\rightarrow}...
		\end{equation}
		together with the sequence of exterior derivatives $d_k$ forms a cochain complex. This complex is called the de Rham complex.
	}

	The relation between closed and exact forms can be used to define the de Rham cohomology groups.
	\newdef{de Rham cohomology}{\index{de Rham!cohomology}
		The $k^{th}$ de Rham cohomology group on $M$ is defined as the following quotient space:
		\begin{equation}
			\label{forms:de_rham_cohomology}
			\boxed{H^k_{\text{dr}}(M) = \frac{\text{ker}(d_{k+1})}{\text{im}(d_k)}}
		\end{equation}
		This quotient space is a vector space. Two elements of the same equivalence class in $H^k_{dr}(M)$ are said to be \textbf{cohomologous}.
		
		One can construct a graded ring \ref{group:graded_ring} from these cohomology groups, called the cohomology ring $H^*$. The product is called the \textbf{cup product} $\smile$ and it is a graded-commutative product (see \ref{group:graded_commutativity}).
	}
	\newdef{Cup product}{\index{cup product}
		Let $[\nu]\in H^k_{\text{dr}}$ and $[\omega]\in H^l_{\text{dr}}$, where we used $[\cdot]$ to show that the elements are in fact equivalence relations belonging to differential forms $\nu$ and $\omega$. The cup product is defined as follows: $[\nu]\smile[\omega] = [\nu\wedge\omega]$.
	}
	
	\begin{theorem}[Poincar\'e's lemma\footnotemark]\index{Poincar\'e!lemma for de Rham cohomology}\label{forms:theorem:poincare}
		\footnotetext{The original theorem states that on a contractible space (see definition \ref{topology:contractible_space}) every closed form is exact.}
		 For every point $p\in M$ there exists a neighbourhood on which the de Rham cohomology is trivial:
		\begin{equation}
			\forall p\in M:\exists U\subseteq M: H^k_{\text{dr}}(U) = 0
		\end{equation}
		This implies that every closed form is locally exact.
	\end{theorem}

\subsection{Vector-valued differential forms}

	\newdef{Vector-valued differential form}{\index{vector-valued differential form}
		Let $V$ be a vector space and $E$ a vector bundle with $V$ as typical fibre. A vector-valued differential form can be defined in two ways. Firstly we can define a vector-valued $k$-form as a map $\omega:\bigotimes^kTM\rightarrow V$. A more general definition is based on sections of a corresponding vector bundle:
		\begin{equation}
			\Omega^k(M, E) = \Gamma(E\otimes\Lambda^kT^*M)
		\end{equation}
	}

	\newformula{Wedge product}{\index{wedge product}
		Let $\omega\in\Omega^k(M, E_1)$ and $\nu\in\Omega^p(M, E_2)$. The wedge product of these differential forms is defined as:
		\begin{equation}
			\omega\wedge\nu(v_1, ..., v_{k+p}) = \stylefrac{1}{(k+p)!}\sum_{\sigma\in S_{k+p}}\sgn(\sigma)\omega(v_{\sigma(1)}, ..., v_{\sigma(k)})\otimes\nu(v_{\sigma(k+1)}, ..., v_{\sigma(p)})
		\end{equation}
		This is a direct generalization of the formula for the wedge product of ordinary differential forms where we replaced the (scalar) product (product in the algebra $\mathbb{R}$) by the tensor product (product in the general tensor algebra). It should be noted that result of this operation is not an element of any of the original bundles $E_1$ or $E_2$ but of the product bundle $E_1\otimes E_2$.
	}

	\newdef{Lie-algebra-valued differential form}{
		A vector-valued differential form where the vector space $V$ is equipped with a Lie algebra structure.
	}

	\newformula{Wedge product}{\index{wedge product}
		Let $\omega\in\Omega^k(M, \mathfrak{g})$ and $\nu\in\Omega^p(M, \mathfrak{g})$. The wedge product of these differential forms is defined as:
		\begin{equation}
			[\omega\wedge\nu](v_1, ..., v_{k+p}) = \stylefrac{1}{(k+p)!}\sum_{\sigma\in S_{k+p}}\sgn(\sigma)[\omega(v_{\sigma(1)}, ..., v_{\sigma(k)}),\nu(v_{\sigma(k+1)}, ..., v_{\sigma(p)})]
		\end{equation}
		where $[\cdot, \cdot]$ is the Lie bracket in $\mathfrak{g}$.
	}

\section{Connections}
\subsection{Vertical vectors}
	
	Because smooth fibre bundles (which include smooth principal $G$-bundles) are also smooth manifolds we can define the traditional notions for them, such as the tangent bundle. We use these to construct the horizontal and vertical (sub)bundles:
	\newdef{Vertical vector}{\index{vertical!vector}
		Let $\pi:E\rightarrow B$ be a smooth fibre bundle. The subbundle $\ker(\pi_\ast)$ of $TE$ is called the vertical bundle of $E$. Fibrewise this gives us $V_x = T_x(E_{\pi(x)})$.
	}

	For principal $G$-bundles we can use an equivalent definition:
	\begin{adefinition}
		Consider a smooth principal $G$-bundle $G\hookrightarrow P\xrightarrow{\pi} M$. We first construct a map $\iota_p$ for every element $p\in P$:
		\begin{equation}
			\iota_p:G\rightarrow P: g\mapsto p\cdot g
		\end{equation}
		We then define a tangent vector $v\in T_p P$ to be vertical if it lies in the image of $\iota_{p,\ast}$, i.e. $\text{Vert}(T_pP) = \text{im}(\iota_{p,\ast})$. This construction is supported by the exactness of following short sequence:
		\begin{equation}
			0\xrightarrow{} \mathfrak{g} \xrightarrow{\iota_{p,\ast}} T_p P\xrightarrow{\pi_\ast} T_xM \xrightarrow{} 0
		\end{equation}
	\end{adefinition}
	
	\begin{property}[Dimension]
		It follows from the second definition that the vertical vectors of a principal $G$-bundle are nothing but the pushforward of the Lie algebra $\mathfrak{g}$ under the right action of $G$ on $P$. Furthermore, the exactness of the sequence implies that $\iota_{p,\ast}:\mathfrak{g}\rightarrow\text{Vert}(T_pP)$ is an isomorphism of vector spaces. In particular, it implies that
		\begin{equation}
			\label{manifolds:vertical_dimension}
			\dim\text{Vert}(T_pP) = \dim\mathfrak{g} = \dim G
		\end{equation}
	\end{property}
	
	\newdef{Fundamental vector field}{
		Consider a principal $G$-bundle. Let $A\in\mathfrak{g}$, where $\mathfrak{g}$ is the Lie algebra corresponding to $G$. The vertical vector field $A^\#:P\rightarrow TP$ given by
		\begin{equation}
			\label{manifolds:fundamental_vector_field}
			A^\#(p) = \iota_{p,\ast}(A)\in\text{Vert}(T_pP)
		\end{equation}
		is called the fundamental vector field associated to $A$.
	}
	\begin{adefinition}
		An equivalent definition of the fundamental vector field $A^\#(p)$ is given by:
		\begin{equation}
			A^\#_p(f) = \left.\deriv{}{t}f(p\cdot\exp(tA))\right|_{t=0}
		\end{equation}
		where $f\in C^\infty(P)$.
	\end{adefinition}
	
	\begin{property}
		The map $(\cdot)^\#:\mathfrak{g}\rightarrow\Gamma(TP)$ is a Lie algebra morphism:
		\begin{equation}
			[A, B]^\# = [A^\#, B^\#]
		\end{equation}
		where the Lie bracket on the left is that in $\mathfrak{g}$ and the Lie bracket on the right is that in $\mathfrak{X}(M)$ given by \ref{manifolds:lie_bracket}.
	\end{property}
	
	\begin{property}
		The vertical bundle satisfies the following $G$-equivariance condition:
		\begin{equation}
			\label{diff:vert_g_equivariance}
			R_{g, \ast}(\text{Vert}(T_pP)) = \text{Vert}(T_{pg}P)
		\end{equation}
		
		By differentiating the equality \[R_g\circ\iota_p = \iota_{pg}\circ\text{ad}_{g^{-1}}\] and using \ref{lie:adjoint_representation}, \ref{manifolds:fundamental_vector_field} we obtain the following algebraic formulation of the $G$-equivariance condition:
		\begin{equation}
			R_{g, \ast}\left(A^\#(p)\right) = \left(\text{Ad}_{g^{-1}}A\right)^\#(pg)
		\end{equation}
	\end{property}
	
\subsection{Ehresmann connections}

	\newdef{Ehresmann connection}{\index{connection!Ehresmann}\index{horizontal!vector}\label{manifolds:connection}
		Consider a smooth fibre bundle $P$. An (Ehresmann) connection on $P$ is the selection of a subspace $\text{Hor}(T_pP)\leq T_pP$ for every $p\in P$ such that:
		\begin{itemize}
			\item $\text{Vert}(T_pP)\oplus\text{Hor}(T_pP) = T_pP$
			\item The selection depends smoothly on $p$.\footnote{See the definiton of a (smooth) distribution \ref{manifolds:distribution}.}
		\end{itemize}
		The elements of $\text{Hor}(T_pP)$ are said to be \textbf{horizontal vectors} with respect to the connection.
	}
	
	\newdef{Principal connection}{\index{connection!principal}\label{manifolds:principal_connection}
		A principal connection on a smooth principal $G$-bundle $P$ is a $G$-equivariant Ehresmann connection, i.e. an Ehresmann connection for which the horizontal subspaces satisfy following $G$-equivariance condition:
		\begin{equation}
			R_{g, \ast}(\text{Hor}(T_pP)) = \text{Hor}(T_{pg}P)
		\end{equation}
		where $R_g$ denotes the right action of $G$ on $P$
	}
	\begin{remark}
		Note that this condition is automatically satisfied for vertical bundles (see equation \ref{diff:vert_g_equivariance}).
	\end{remark}
	
	\newdef{Horizontal bundle}{
		The horizontal (sub)bundle $\text{Hor}(TP)$ is defined as $\bigsqcup_{p\in P}\text{Hor}(T_pP)$. The $G$-equivariance condition then implies that this subbundle is invariant under (the pushforward of) the right action of $G$.
	}
	
	\begin{property}[Dimension]\label{manifolds:connection_dimensions}
		Properties \ref{manifolds:principal_bundle_dimension}, \ref{manifolds:vertical_dimension} and the direct sum decomposition of $T_pP$ imply the following relation:
		\begin{equation}
			\dim\text{Hor}(T_pP) = \dim M
		\end{equation}
		Here we briefly summarize all dimensional relations between the components of a principal $G$-bundle over a base manifold $M$:
		\begin{empheq}[box=\widefbox]{align}
			\dim P &= \dim M + \dim G\\
			\dim M &= \dim\text{Hor}(T_pP)\\
			\dim G &= \dim\text{Vert}(T_pP)
		\end{empheq}
		for all $p\in P$.
	\end{property}
	
	\newdef{Horizontal and vertical forms}{\index{horizontal!form}\index{vertical!form}\label{forms:horizontal_form}
		Let $\theta\in\Omega^k(P)$ be a differential $k$-form. We define following notions:
		\begin{itemize}
			\item $\theta$ is said to be horizontal if
			\begin{equation}
				\theta(v_1, ..., v_k) = 0
			\end{equation}
			whenever at least 1 of the $v_i$ lies in $\text{Vert}(T_pP)$.
			\item $\theta$ is said to be vertical if
			\begin{equation}
				\theta(v_1, ..., v_k) = 0
			\end{equation}
			whenever at least 1 of the $v_i$ lies in $\text{Hor}(T_pP)$.
		\end{itemize}
		For functions $f\in\Omega^0(P)$ it is vacuously true that they are both vertical and horizontal.
	}
	\newdef{Dual connection}{\index{dual!connection}
		First we define the dual of the horizontal bundle:
		\begin{equation}
			\text{Hor}(T_p^*P) = \{h^*\in T_p^*P|h^*(v)=0, v\in\text{Vert}(T_pP)\}
		\end{equation}
		
		It is the set horizontal 1-forms. A dual connection can then be defined as the selection of a vertical covector bundle $\text{Vert}(T_p^*P)$ satisfying the conditions of definition \ref{manifolds:connection} and \ref{manifolds:principal_connection} (where $\text{Vert}$ and $\text{Hor}$ should be interchanged).
	}
	
\subsection{Connection form}

	\newdef{Connection one-form}{\index{connection!form}
		Let $\prb$ be a principal bundle. A connection one-form, related to a given principal connection, is a $\mathfrak{g}$-valued 1-form $\omega:\Gamma(TP)\rightarrow\mathfrak{g}$ that satisfies the following 2 conditions:
		
		\begin{enumerate}
			\item Cancellation of fundamental vector fields:
			\begin{equation}
				\omega(A^\#) = A
			\end{equation}
			\item $G$-equivariance:
			\begin{equation}
				\omega\circ R_{g, \ast} = \text{Ad}_{g^{-1}}\circ\omega
			\end{equation}
		\end{enumerate}
		The horizontal subspaces are then defined as $\text{Hor}(T_pP) = \ker\omega|_p$.
	}	
	\begin{formula}
		Consider a principal $G$-bundle $P$. Given a principal connection on $P$, the associated connection one-form is given by the following map:
		\begin{equation}
			\omega = (\iota_{p,\ast})^{-1}\circ\pr_V
		\end{equation}
		where $\pr_V$ is the projection $TP\rightarrow\text{Vert}(TP)$ associated to the decomposition from definition \ref{manifolds:connection}.
	\end{formula}
	
	\begin{formula}
		Consider a principal bundle $\prb$ and an associated vector bundle $P\times_G V$. For every $G$-equivariant map $\phi:P\rightarrow V$ and any $X\in\mathfrak{g}$ we find that
		\begin{equation}
			d\phi(X^\#) + [\omega\wedge\phi](X^\#) = 0
		\end{equation}
		where the left action of $\mathfrak{g}$ is induced by the representation of $G$ on $V$.
	\end{formula}
	
	\begin{property}
		Consider two principal $G$-bundles $\xi_1$ and $\xi_2$. Let $\omega$ be a connection one-form on $\xi_1$ and let $F:\xi_1\rightarrow \xi_2$ be a bundle map. The map $F^*\omega$ defines an Ehresmann connection on $\xi_2$.
	\end{property}
	
	\newdef{Reducible connection}{
		Consider a principal $G$-bundle $P$ equipped with a connection one-form $\omega$. If the bundle map $\theta$ induces an $H$-reduction of $P$ then the connection $\omega$ is said to be reducible if $\theta^*\omega$ takes values in $\mathfrak{h}$.
	}
	
\subsection{Maurer-Cartan form}

	\newdef{Maurer-Cartan form}{\index{Maurer-Cartan form}\index{Cartan!(connection) form|see{Maurer-Cartan}}
		For every $g\in G$ we have that the tangent space $T_gG$ is isomorphic to $T_eG = \mathfrak{g}$. The isomorphism $T_gG\rightarrow\mathfrak{g}$ is given by the Maurer-Cartan form:
		\begin{equation}
			\boxed{\Omega := L_{g^{-1},\ast}}
		\end{equation}
	}
	
	\begin{construct}
		Consider a manifold $M = \{x\}$. When constructing a principal $G$-bundle over $M$ we see that the total space $P = \{x\}\times G$ can be identified with the structure group $G$. From the relations in property \ref{manifolds:connection_dimensions} it follows that the horizontal spaces are null-spaces (which indeed defines a smooth distribution and thus a connection according to \ref{manifolds:connection}) and that the vertical spaces are equal to the tangent spaces, i.e. $\text{Vert}(T_gG) = T_gG$ (where we used the association $P\cong G$).
		
		The simplest way to define a connection form $\omega$ on this bundle would be the trivial projection $TP\rightarrow\text{Vert}(TP) = \mathbbm{1}_{TP}$. The image of this map would however be $T_gG$ and not $\mathfrak{g}$ as required. This can be solved by using the Maurer-Cartan form $\Omega:T_gG\rightarrow\mathfrak{g}$, i.e. we define $\omega(v) = \Omega(v)$.
	\end{construct}
	
	\begin{property}
		The Maurer-Cartan form is the unique Principal connection on the bundle $G\hookrightarrow G\rightarrow \{x\}$.
	\end{property}

\subsection{Local representation}

	\newdef{Yang-Mills field}{\index{Yang-Mills!field}
		Consider a principal bundle $\prb$ and an open subset $U\subseteq M$. Given an Ehresmann connection $\omega$ on $P$ and a local section $\sigma:U\rightarrow P$, we define the Yang-Mills field $\omega^U:\Gamma(TU)\rightarrow\mathfrak{g}$ as follows:
		\begin{equation}
			\label{diff:prin:yang_mills_field}
			\omega^U = \sigma^*\omega
		\end{equation}
	}
	
	\newdef{Local representation}{
		Consider a principal bundle $\prb$. Let $(U, \varphi)$ be a bundle chart on $P$. The local representation of an Ehresmann connection $\omega$ on $P$ with respect to the chart $(U, \varphi)$ is given by $(\varphi^{-1})^*\omega$.
	}
	
	\begin{formula}
		Consider an Ehresmann connection $\omega$ on a principal bundle $\prb$. According to property \ref{diff:prin_section_triv} every local section $\sigma:U\rightarrow P$ induces both a Yang-Mills field $\omega^U$ and a local representation of $\omega$. These two forms are related by the following equation:
		\begin{equation}
			h^*\omega_{(m, g)}(v, X) = \text{Ad}_{g^{-1}}(\omega^U_m(v)) + \Omega_g(X)
		\end{equation}
		where $v\in T_mU, X\in\mathfrak{g}$, $\Omega$ is the Maurer-Cartan form on $G$ and $h$ is the local trivialization induced by $\sigma$.
	\end{formula}
	
	\begin{formula}[Compatibility condition]
		Consider a principal bundle $\prb$ and 2 open subsets $U, V$ of $M$. Given 2 local sections $\sigma_U:U\rightarrow P$ and $\sigma_V:V\rightarrow P$ and an Ehresmann connection $\omega$ on $P$, we can define two Yang-Mills field $\omega^U$ and $\omega^V$ on $M$.
		
		On the intersection $U\cap V$ we can find a (unique) gauge transformation $\xi:U\cap V\rightarrow G$ such that $\sigma_V(m) = \sigma_U(m)\cdot\xi(m)$. Using this gauge transformation we can relate $\omega^U$ and $\omega^V$ as follows:
		\begin{equation}
			\label{diff:prin:local_compatibility}
			\omega^V_m = \text{Ad}_{\xi(m)^{-1}}\omega^U_m + (\xi^*\Omega)_m
		\end{equation}
		where $\Omega$ is the Maurer-Cartan form on $G$.
	\end{formula}
	
	\begin{example}[General linear group\footnotemark]
		\footnotetext{A derivation can be found in lecture 22 of \cite{schuller}.}
		Let $G=\text{GL}(\mathbb{R}^n)$. The second term in equation \ref{diff:prin:local_compatibility} can be written as follows:
		\begin{equation}
			(\xi^*\Omega)^i_{\ j} = (\xi(m)^{-1})^i_{\ k}\pderiv{}{x^\mu}\xi(p)^k_{\ j}dx^\mu
		\end{equation}
		at every point $m\in M$. Formally this can be written coordinate-independently as:
		\begin{equation}
			\label{diff:prin:mc_pullback}
			\xi^*\Omega = \xi^{-1}d\xi
		\end{equation}
	\end{example}
	
	\begin{example}[Christoffel symbols]\index{Christoffel!symbols}
		Let $\Gamma^i_{\ j\mu}, \overline{\Gamma}^k_{\ l\nu}$ be the Yang-Mills fields corresponding to a connection of a frame bundle, where the sections are induced by a choice of coordinates ($x^i$ and $y^i$ respectively). In this case, the expansion coefficients of the Yang-Mills field are called the \textbf{Christoffel symbols}\footnote{See also equation \ref{diff:christoffel_symbol}.}. Using equations \ref{diff:prin:local_compatibility} and \ref{diff:prin:mc_pullback} this becomes:
		\begin{equation}
			\overline{\Gamma}^i_{\ j\mu} = \pderiv{y^\nu}{x^\mu}\left(\pderiv{x^i}{y^k}\Gamma^k_{\ l\nu}\pderiv{y^l}{x^j} + \pderiv{x^i}{y^k}\frac{\partial^2y^k}{\partial x^j\partial x^\nu}\right)
		\end{equation}
	\end{example}

\subsection{Horizontal lifts and parallel transport}
	
	\begin{definition}[Horizontal lift]\index{horizontal!lift}
		Consider a principal $G$-bundle $G\hookrightarrow P\rightarrow M$ and a curve $\gamma:[0, 1]\rightarrow M$. Let $p_0\in \pi^{-1}(\gamma(0))$. There exists a unique curve $\widetilde{\gamma}_{p_0}:[0, 1]\rightarrow P$ satisfying the following conditions:
		\begin{itemize}
			\item $\widetilde{\gamma}_{p_0}(0) = p_0$
			\item $\pi\circ\widetilde{\gamma}_{p_0} = \gamma$
			\item $\widetilde{\gamma}_{p_0}'(t)\in\text{Hor}(TP)$ for all $t\in[0, 1]$
		\end{itemize}
		The curve $\widetilde{\gamma}_{p_0}$ is said to be the \textbf{horizontal lift} of $\gamma$ starting at $p_0$. When it is clear from the context what the basepoint $p_0$ is, the subscript is often ommited and we write $\widetilde{\gamma}$ instead of $\widetilde{\gamma}_{p_0}$.
	\end{definition}
	\begin{remark}[Horizontal curve]\index{horizontal!curve}
		Curves satisfying the last condition in the above property are said to be horizontal.
	\end{remark}
	
	\begin{method}
		Consider a principal bundle $G\hookrightarrow P\rightarrow M$. Let $\gamma(t)$ be a curve in $M$ and let $\omega$ be an Ehresmann connection on $P$. For general structure groups $G$, the horizontal lift can be found as follows: Let $\delta(t)$ be a curve in $P$ that projects onto $\gamma(t)$, i.e. $\pi\circ\delta=\gamma$, such that $\widetilde\gamma_{p_0}(t)=\delta(t)\cdot g(t)$ for some curve $g(t)$ in $G$. The curve $g(t)$ can then be found as the unique solution of the following first order ODE:
		\begin{equation}
			\label{diff:prin:horizontal_ode}
			\text{Ad}_{g(t)^{-1}}\omega_{\delta(t)}(X_{\delta, \delta(t)}) + \Omega_{g(t)}(Y_{g, g(t)}) = 0
		\end{equation}
		where $X_\delta, Y_g$ are tangent vectors to respectively the curves $\delta(t)$ and $g(t)$ and where $\Omega$ is the Maurer-Cartan form on $G$. As initial value condition we use $\delta(0)\cdot g(0) = p_0$.
	\end{method}
	\begin{remark}
		When given a local section $\sigma:U\rightarrow P$ we can rewrite the ODE in a more explicit form. First we remark that the section induces a curve $\delta = \sigma\circ\gamma$. Taking the derivative yields $X_\delta = \sigma_*(X_\gamma)$. Using this we can rewrite the ODE as
		\begin{equation}
			\text{Ad}_{g(t)^{-1}}\omega_{\delta(t)}(\sigma_*X_{\gamma, \gamma(t)}) + \Omega_{g(t)}(Y_{g, g(t)}) = 0
		\end{equation}
		By using the equality $f^*\omega = \omega\circ f_*$ and introducing the Yang-Mills field $A = \sigma^*\omega$ this becomes:
		\begin{equation}
			\text{Ad}_{g(t)^{-1}}A(X_{\gamma, \gamma(t)}) + \Omega_{g(t)}(Y_{g, g(t)}) = 0
		\end{equation}
	\end{remark}
	
	\begin{example}
		 For matrix Lie groups the above ODE can be reformulated as follows: Given the trivial section $s:U\rightarrow U\times G:x\mapsto (x, e)$, where $U$ is an open subset of $M$, the horizontal lift of $\gamma(t)$ can locally be parametrized as $\widetilde{\gamma}(t) = \underbrace{(s\circ\gamma)(t)}_{\delta(t)}\cdot g(t) = (\gamma(t), g(t))$ where $g(t)$ is a curve in $G$. To determine $\widetilde{\gamma}(t)$ it is thus sufficient to find $g(t)$. The ODE \ref{diff:prin:horizontal_ode} then becomes:
		\begin{equation}
			\label{diff:prin:horizontal_ode_matrix}
			g'(t) = -\omega(\gamma(t), e, \gamma'(t), 0)g(t)
		\end{equation}
		
		Using the trivial section we can rewrite this formula. First we consider the action of the Yang-Mills field $s^*\omega$ on the derivative $\gamma_* = (\gamma(t), \gamma'(t))$. Using the fact that it is linear in the second argument we can write: \[s^*\omega(\gamma(t), \gamma'(t)) = A(\gamma(t))\gamma'(t)\] where $A:M\rightarrow\text{Hom}(\mathbb{R}^{\dim M}, \mathfrak{g})$ gives a linear map for each point $\gamma(t)\in M$. The action can also be rewritten using the relation $f^*\omega = \omega\circ f_\ast$ as\[s^*\omega(\gamma(t), \gamma'(t)) = \omega\Big(s_\ast(\gamma(t), \gamma'(t))\Big) = \omega(\gamma(t), e, \gamma'(t), 0)\]
		Combining these relations with the ODE \ref{diff:prin:horizontal_ode_matrix} gives:
		\begin{equation}
			\label{diff:prin:horizontal_ode_derivative}
			\left(\deriv{}{t} + A(\gamma(t))\gamma'(t)\right)g(t) = 0
		\end{equation}
		where $\deriv{}{t}$ is the matrix given by the scalar multiplication of the derivative $\deriv{}{t}$ and the identity matrix $I$.
	\end{example}
	
	\begin{method}
		The ODE \ref{diff:prin:horizontal_ode} can now be solved. We explicitly assume that $G$ is a matrix Lie group such that we can start from equation \ref{diff:prin:horizontal_ode_derivative}. Direct intergation and iteration gives us:
		\begin{equation}
			g(t) = \left[I - \int_0^tdt_1A(\gamma'(t_1)) + \int_0^tdt_1\int_0^{t_1}dt_2A(\gamma'(t_1))A(\gamma'(t_2)) - ...\right]g(0)
		\end{equation}
		where $A$ is the Yang-Mills field corresponding to the local section $\sigma$. This can be rewritten using the standard "square integration" trick\footnote{Well known from the Dyson series \ref{QM:dyson_series}.} as:
		\begin{equation}
			g(t) = \left[I - \int_0^tdt_1A(\gamma'(t_1)) + \frac{1}{2!}\int_0^tdt_1\int_0^{\textcolor{red}{t}}dt_2\mathcal{T}\Big(A(\gamma'(t_1))A(\gamma'(t_2))\Big) - ...\right]g(0)
		\end{equation}
		By noting that this formula is equal to the path-ordered exponential series we find:
		\begin{equation}
			\boxed{g(t) = \mathcal{T}\exp\left(-\int_0^tdt'A(\gamma'(t'))\right)g(0)}
		\end{equation}
	\end{method}
	
	\newdef{Parallel transport}{\index{parallel transport!on principal bundles}
		\nomenclature[O_Par]{$\text{Par}_t^\gamma$}{Parallel transport map with respect to the curve $\gamma$.}
		The parallel transport map with respect to the curve $\gamma$ is defined as follows:
		\begin{equation}
			\text{Par}_t^\gamma:\pi^{-1}(\gamma(0))\rightarrow\pi^{-1}(\gamma(t)):p_0\mapsto \widetilde{\gamma}_{p_0}(t)
		\end{equation}
		This map is $G$-equivariant and it is an isomorphism of fibres.
	}
	
\subsection{Holonomy}
	
	\newdef{Holonomy group}{\index{holonomy}
		\nomenclature[S_Hol]{$\text{Hol}_p(\omega)$}{Holonomy group at $p$ with respect to the connection $\omega$.}
		Consider a principal bundle $\prin{G}{P}{M}$. Let $\Omega^{ps}_mM\subset\Omega_m M$ be the subset of the loop space with basepoint $m\in M$ consisting of piecewise smooth loops. The holonomy group $\text{Hol}_p(\omega)$ based at $p\in\pi^{-1}(m)\subset P$ with respect to the connection form $\omega$ is given by:
		\begin{equation}
			\text{Hol}_p(\omega) = \{g\in G: p \sim p\cdot g\}
		\end{equation}
		where two point $p, q\in P$ are equivalent if there exists a loop $\gamma\in\Omega_m^{ps}M$ such that the horizontal lift $\widetilde{\gamma}$ connects $p$ and $q$.
	}
	\newdef{Reduced holonomy group}{
		The reduced holonomy group $\text{Hol}_p^0(\omega)$ is defined as the subset of $\text{Hol}_p(\omega)$ using only contractible loops.
	}
	
\subsection{Koszul connections and covariant derivatives}

	\newdef{Horizontal lifts on associated bundles}{\index{horizontal!lift}
		Let $P_F = P\times_G F$ be an associated bundle of a principal bundle $\prb$. Let $\gamma$ be a curve in $M$ with horizontal lift $\widetilde{\gamma}_p$ in $P$. The horizontal lift of $\gamma$ in $P_F$ through a point $[p, f]\in P_F$ is defined as follows:
		\begin{equation}
			\widetilde{\gamma}^{P_F}_{[p, f]}(t) = [\widetilde{\gamma}_p(t), f]
		\end{equation}
		Although the fibre element $f$ seems to stay constant along the horizontal lift, it in fact changes according to formula \ref{diff:prin:associated_bundle_equivalence}.
	}
	\newdef{Parallel transport on associated bundles}{
		Similar to the case of principal bundles $P$, the parallel transport map on an associated bundle $P_F$ is defined as:
		\begin{equation}
			\text{Par}_t^\gamma:\pi_F^{-1}(\gamma(0))\rightarrow\pi_F^{-1}(\gamma(t)):[p, f]\mapsto \widetilde{\gamma}^{P_F}_{[p, f]}(t)
		\end{equation}
	}

	\begin{example}[Parallel transport on vector bundles]
		Consider a principal bundle $G\hookrightarrow P\rightarrow M$. Suppose that the Lie group $G$ acts on a vector space $V$ by a representation $\rho:G\rightarrow\text{GL}_m$ . We can then construct an associated vector bundle $\pi_1:P\times_{GL(V)} V\rightarrow M$. Assume further that we work on a chart $(U, \varphi)$ such that we can locally write $P, P_V$ as product bundles.
		
		Parallel transport on this vector bundle is then defined as follows. Let $\gamma(t)$ be a curve in $M$ such that $\gamma(0)=x_0$ and $\gamma(1) = x_1$. Furthermore, let the horizontal lift $\widetilde{\gamma}(t) = (\gamma(t), g(t))$ satisfy $\widetilde{\gamma}(0)=(x_0, h)$ as initial condition. The parallel transport of the point $(x_0, v_0)\in U\times V$ along $\gamma$ is given by the following map:
		\begin{equation}
			\text{Par}^\gamma_t:\pi^{-1}_1(x_0)\rightarrow\pi^{-1}_1(\gamma(t)):(x_0, v_0)\mapsto \big(\gamma(t), \rho\big(g(t)h^{-1}\big)v_0\big)
		\end{equation}
		It should be noted that this map is independent of the initial element $h\in G$ (despite the presence of the factor $h^{-1}$). Furthermore, $\text{Par}^\gamma_t$ is an isomorphism of vector spaces and can thus be used to identify distant fibers (as long as they lie in the same path-component).
	\end{example}
	\begin{remark}
		For every vector bundle one can construct the frame bundle and use the parallel transport map on this bundle to define parallel transport of vectors. Hence the previous construction is valid for any vector bundle.
	\end{remark}
	
	\newdef{Covariant derivative}{\index{covariant!derivative}
		Consider a vector bundle with model fibre space $V$ and its associated principal GL$(V)$-bundle with Ehresmann connection $\omega$, both over a base manifold $M$. Let $\sigma:M\rightarrow E$ be a section of the vector bundle and let $X$ be a vector field on $M$. The covariant derivative of $\sigma$ with respect to $X$ is defined as:
		\begin{equation}
			\boxed{\nabla_X\sigma|_{x_0} = \lim_{t\rightarrow+\infty}\stylefrac{(\text{Par}_t^\gamma)^{-1}\sigma(\gamma(t)) - \sigma(x_0)}{t}}
		\end{equation}
		where $\gamma(t)$ is any curve such that $\gamma(0) = x_0$ and $\gamma'(0) = X(x_0)$.
	}
	
	\begin{property}\index{Koszul!connection}
		The map
		\begin{equation}
			\Gamma(TM)\times\Gamma(E)\rightarrow\Gamma(E):(X, \sigma)\mapsto\nabla_X\sigma
		\end{equation}
		gives a Koszul connection \ref{manifolds:koszul_connection}. It follows that every Ehresmann connection on a principal bundle induces a Koszul connection on all of its associated vector bundles.
	\end{property}
	
\subsection{Exterior covariant derivative}

	\newdef{Exterior covariant derivative}{\index{exterior!covariant derivative}
		Consider a principal bundle $\prin{G}{P}{M}$ equipped with an Ehresmann connection $\omega$. Let $\theta\in \Omega^k(P)$ be a differential $k$-form. The exterior covariant derivative $D\theta$ is defined as follows:
		\begin{equation}
			D\theta(v_0, ..., v_k) = d\theta(v_0^H, ..., v_k^H)
		\end{equation}
		where $d$ is the exterior derivative \ref{forms:def:exterior_derivative} and $v_i^H$ is the projection of $v_i$ on the horizontal subspace $\text{Hor}(T_pP)$ associated to the Ehresmann connection $\omega$. From the definition it follows that the exterior covariant derivative $D\theta$ is a horizontal form\footnote{See definition \ref{forms:horizontal_form}.}.
	}
	\begin{remark}
		The exterior covariant derivative can also be defined for general $W$-valued $k$-forms where $W$ is a vector space. This can be done by defining it component-wise with respect to a given basis on $W$. Afterwards one can prove that the choice of basis plays no role.
	\end{remark}
	
	\begin{property}
		If $\phi$ is equivariant than $D\phi$ is a tensorial form.
	\end{property}
	
	\begin{formula}
		Using the Koszul connection on the tangent bundle $TP$ we can rewrite the action of the exterior covariant derivative as follows:
		\begin{equation}
			\boxed{D\theta(v_0, ..., v_k) = \sum_i^k(-1)^i\nabla_{v_i}\theta(v_0, ..., \hat{v}_i, ..., v_k) + \sum_{i<j}^k(-1)^{i+j}\theta([v_i, v_j], v_0, ..., \hat{v}_i, ..., \hat{v}_j, ..., v_k)}
		\end{equation}
		where $\hat{v}_i$ means that this vector is omitted. As an example we explicitly give the formula for a 1-form $\Phi$:
		\begin{equation}
			D\Phi(X, Y) = \nabla_X(\Phi(Y)) - \nabla_Y(\Phi(X)) - \Phi([X, Y])
		\end{equation}
		which should remind the reader of the analogous formula for the ordinary exterior derivative \ref{forms:k_form_exterior_derivative}.
	\end{formula}
	
	By property \ref{diff:prin:section_bijection} we can use the following construction to find an explicit expression for the covariant derivative on an associated vector bundle:
	\begin{construct}\index{covariant!derivative}
		Let $\prb$ be a principal bundle and let $P_V := P\times_G V$ be an associated vector bundle. Given a section $\sigma:M\rightarrow P_V$ we can construct a $G$-equivariant map $\phi:P\rightarrow V$ using formula \ref{diff:prin:section_bijection_phi}.
		
		First we construct the exterior covariant derivative of $\phi$:
		\begin{equation}
			D\phi(X) = d\phi(X) + [\omega\wedge\phi](X)
		\end{equation}
		where $X\in T_pP$. Now given an additional (local) section $\varphi:U\subseteq M\rightarrow P$ we can pull back the previous structure. This gives us:
		\begin{equation}
			(\varphi^*D\phi)(Y) = d(\varphi^*\phi)(Y) + [\varphi^*\omega\wedge\varphi^*\phi](Y)
		\end{equation}
		where $Y\in T_mM$. After introducing the notations $S:=\varphi^*\phi$ and $\nabla_YS:=(\varphi^*D\phi)(Y)$ and remembering the definition of the Yang-Mills field \ref{diff:prin:yang_mills_field} this becomes:
		\begin{equation}
			\label{diff:prin:local_covariant_derivative}
			\boxed{\nabla_YS = dS(Y) + \omega^U(Y)\cdot S}
		\end{equation}
	\end{construct}
	\begin{example}
		Let $G=\text{GL}(\mathbb{R}^n)$. In local coordinates equation \ref{diff:prin:local_covariant_derivative} becomes:
		\begin{equation}
			(\nabla_YS)^i = \pderiv{S^i}{x^k}Y^k + \Gamma^i_{\ jk}S^jY^k
		\end{equation}
		which is exactly the formula known from classic differential geometry and relativity. 
	\end{example}

\subsection{Curvature}

	\newdef{Curvature}{\index{curvature}
		Let $\omega$ be an Ehresmann connection on a principal bundle $\prin{G}{P}{M}$. The curvature $\Omega$ of $\omega$ is defined as the exterior covariant derivative $D\omega$.
	}
	\newdef{Flat connection}{\index{connection!flat}
		An Ehresmann connection $\omega$ is said to be flat if its curvature $\Omega$ vanishes everywhere.
	}
	\begin{example}
		Let $\omega_G$ be the Maurer-Cartan form on a Lie group $G$. It follows from the fact that the only horizontal vector on the bundle $\prin{G}{G}{\{x\}}$ is the zero vector, that the curvature of $\omega_G$ is 0. Hence the Maurer-Cartan form is a flat connection.
	\end{example}
	
	\begin{property}[Second Bianchi identity]\index{Bianchi identity}
		Let $\omega$ be an Ehresmann connection with curvature $\Omega$.
		\begin{equation}
			\boxed{D\Omega = 0}
		\end{equation}
	\end{property}
	\begin{remark}
		One should however pay attention not to generalize this result\mnote{\dbend} to general differential forms. Only the exterior derivative satisfies the coboundary condition $d^2 \equiv 0$, the exterior covariant derivative does not.
	\end{remark}
	
	\newformula{Cartan structure equation}{\index{Cartan!structure equation}
		Let $\omega$ be an Ehresmann connection and let $\Omega$ be its curvature form.
		\begin{equation}
			\boxed{\Omega = d\omega + \frac{1}{2}[\omega\wedge\omega]}
		\end{equation}
	}
	
	The following property is an immediate consequence of the Frobenius integrability theorem \ref{manifolds:frobenius} and the fact that an Ehresmann connection vanishes on the horizontal subbundle.
	\begin{property}\index{integrable}
		Let $\omega$ be an Ehresmann connection. The associated horizontal distribution\footnote{See \ref{manifolds:distribution} for the definition of a distribution of vector spaces.}\[p\mapsto\text{Hor}(T_pP)\]is integrable if and only if the connection $\omega$ is flat. Furthermore, the vertical distribution is always integrable.
	\end{property}
	
	\newdef{Yang-Mills field strength}{\index{Yang-Mills!field strength}
		Let $\prb$ be a principal bundle equipped with an Ehresmann connection $\omega$. Given a local section $\sigma:U\subseteq M\rightarrow P$ we define the Yang-Mills field strength $F$ as the pullback $\sigma^*\Omega$, where $\Omega=D\omega$ is the curvature of $\omega$.
	}
	
\subsection{Torsion}

	\newdef{Solder form}{\index{solder form}
		Let $\prb$ be a principal bundle. Let $V$ be a $\dim M$-dimensional vector space equipped with a representation\footnote{In general this will be $V=\mathbb{R}^{\dim M}$ and $G=\text{GL}(n, \mathbb{R})$.} $\rho:G\rightarrow\text{GL}(V)$ such that $TM\cong P\times_G V$ in the sense of associated bundles. A solder(ing) form $\theta$ on $P$ is a $V$-valued one-form that satisfies the following conditions:
		\begin{itemize}
			\item Horizontal: $\forall p\in P$: $\ker\theta_p\leq\text{Vert}(T_pP)$
			\item $G$-equivariant: $R_g^*\theta = \rho(g)^{-1}\theta$
		\end{itemize}
		where $R_g$ is the right action of $G$ on $P$.
	}
	
	\newdef{Torsion}{\index{torsion}
		Let $\prb$ be a principal bundle equipped with an Ehresmann connection $\omega$ and a solder form $\theta$. The torsion $\Theta$ is defined as the exterior covariant derivative $D\theta$.
	}
	
	\begin{formula}[Cartan structure equation]\index{Cartan!structure equation}
		Let $\omega$ be an Ehresmann connection, $\theta$ a solder form and $\Theta$ its torsion form.
		\begin{equation}
			\Theta = d\theta + \omega\barwedge\theta
		\end{equation}
		where the wedge product is defined analogously\footnote{For forms with $\deg\geq1$ we sum over all permutations of the arguments.} to \ref{forms:vector_valued_wedge} and \ref{forms:lie_algebra_valued_wedge} using the representation of $\mathfrak{g}$ on $V$ induced by the representation $\rho:G\rightarrow\text{GL}(V)$:
		\begin{equation}
			\omega\barwedge\theta(v, w) = \Big[\omega(v)\cdot\theta(w) - \omega(w)\cdot\theta(v)\Big]
		\end{equation}
		where $\cdot$ denotes the representation of $\mathfrak{g}$ on $V$.
	\end{formula}
	
	\begin{property}[First Bianchi identity]\index{Bianchi identity}
		Let $\omega$ be an Ehresmann connection, $\Omega$ its curvature, $\theta$ a solder form and $\Theta$ its torsion.
		\begin{equation}
			\boxed{D\Theta = \Omega\barwedge\theta}
		\end{equation}
	\end{property}
	
	\begin{property}
		Consider a smooth manifold $M$ equipped with a $G$-structure. If this structure is integrable then it admits a torsion-free connection.
	\end{property}


\chapter{Complex Geometry}\label{chapter:complex_geometry}

\section{Complex structures}

    \newdef{Almost complex structure}{
        Let $M$ be a smooth manifold. An almost complex structure on $M$ is a (complexified) smooth $(1,1)$-tensor field $J:TM^\mathbb{C}\rightarrow TM^\mathbb{C}$ such that $J|_p:T_pM^\mathbb{C}\rightarrow T_pM^\mathbb{C}$ satisfies $J|_p^2 = -1$ for all $p\in M$.
    }

    This definition implies the following property:
    \begin{property}
        An almost complex manifold is even-dimensional and orientable.
    \end{property}

    An almost complex structure induces a decomposition of the tangent bundle in so-called holomorphic and antiholomorphic components:\[TM^\mathbb{C} = TM^+\oplus TM^-,\] where both bundles have the same dimension. When the coordinates on $M$ are denoted by $\{x^k\}_{k\leq 2n}$, bases for these two subbundles are given by \[\left\{\pderiv{}{z^k} := \frac{1}{2}\left(\pderiv{}{x}_{^{2k-1}} - i\pderiv{}{x}_{^{2k}}\right)\right\}_{k\leq n}\] and \[\left\{\pderiv{}{\overline{z}^k} := \frac{1}{2}\left(\pderiv{}{x}_{^{2k-1}} + i\pderiv{}{x}_{^{2k}}\right)\right\}_{k\leq n},\] respectively.
    \remark{The reason that the almost complex structure is defined on the complexified tangent bundle has to do with the fact that $J$ is only diagonalizable on a complex vector space (because it squares to a negative value).}

    \begin{example}[Complex vector spaces]
        Consider a complex vector space $V$. By looking at Property \ref{vector:complexification_decomposition} and using the canonical isomorphism $V\cong T_vV$ for vector spaces, one can see that the automorphism $v\mapsto iv$ induced by the imaginary unit gives rise to an almost complex structure on $V$.
    \end{example}

    \begin{property}[Reduction of structure group]
        A $2m$-dimensional manifold $M$ admits an almost complex structure if and only if the structure group of the tangent bundle $TM$ can be reduced from $\GL(\mathbb{R}^{2n})$ to $\GL(\mathbb{C}^n)$.
    \end{property}

    \newdef{Complex manifold}{\index{manifold!complex}
        A topological space $M$ for which there exists an open cover $\{U_i\}_i$ such that for every $U_i$ there exists a homeomorphism $\varphi_i:U_i\rightarrow \mathbb{C}^n$ onto some open subset of $\mathbb{C}^n$. The transition functions $\varphi_{ji}:\varphi_i(U_i\cap U_j)\rightarrow\varphi_j(U_i\cap U_j)$ are also required to be holomorphic.
    }
    \newdef{Complex dimension}{
        The integer $n$ in previous definition is called the complex dimension of $M$. It is denoted by $\dim_\mathbb{C}(M)$.
    }

    \begin{property}
        An almost complex manifold is complex if and only if the $\GL(\mathbb{C}^n)$-structure is integrable.
    \end{property}
    The integrability condition can be rephrased algebraically as follows:
    \begin{theorem}[Newlander-Nirenberg]\index{Nijenhuis!tensor}\index{integrable!complex structure}
        An almost complex manifold is complex if and only if the \textbf{Nijenhuis tensor} $N_J$ vanishes for all vector fields:
        \begin{gather}
            \label{complex:integrable_structure}
            N_J(X,Y) = [JX,JY] - J[JX,Y] - J[X,JY] - [X,Y] = 0.
        \end{gather}
        In a local coordinate-induced basis this becomes
        \begin{gather}
            J_\rho^\nu\partial_\nu J_\sigma^\mu - J_\sigma^\nu\partial_\nu J_\rho^\mu - J_\nu^\mu\partial_\rho J_\sigma^\nu + J_\nu^\mu\partial_\sigma J_\rho^\mu = 0.
        \end{gather}
    \end{theorem}

    \newdef{Metalinear structure}{\index{meta!linear group}\label{complex:metalinear_structure}
        Consider the complex linear group $\GL(n,\mathbb{C})$ together with the morphism $\det:\GL(n,\mathbb{C})\rightarrow\mathbb{C}^\times$. The metalinear group can be considered as the domain of the holomorphic square root of $\det$:
        \begin{gather}
            \mathrm{ML}(n,\mathbb{C}) := \big\{(A,z)\in\GL(n,\mathbb{C})\times\mathbb{C}^\times\,\big\vert\,\det(A)=z^2\big\}.
        \end{gather}
        An equivalent definition, which will be used in the remainder of the text, makes use of the special linear group:
        \begin{gather}
            \mathrm{ML}(n,\mathbb{C}) = \frac{\mathrm{SL}(n,\mathbb{C})\times\mathbb{C}}{2\mathbb{Z}},
        \end{gather}
        where $\mathbb{Z}$ acts on the product group as $k:(A,z)\mapsto(e^{-2\pi ik/n})A, z + 2\pi ik/n)$. This group is the double cover of $\GL(n,\mathbb{C})$.

        Similar to the definition of spinor and metasymplectic structures (Definitions \ref{riemann:spin_structure} and \ref{symplectic:metaplectic_structure}), one can also define metalinear structures on a manifold. The metalinear frame bundle is a lift of the (complex) frame bundle along the canonical morphism $\mathrm{ML}(n,\mathbb{C})\rightarrow\GL(n,\mathbb{C})$ such that it ``commutes'' with the bundle map $F_\mathrm{ML}M\rightarrow FM$.
    }
    \begin{property}[Existence]
        A manifold $M$ admits a metalinear structure if and only if its first Stiefel-Whitney class $w_1\in H^1(M;\mathbb{Z}_2)$ squares to 0. In particular, every orientable manifold admits a metalinear structure. The set of nonequivalent metalinear structures is parametrized by $H^1(M;\mathbb{Z}_2)$.
    \end{property}
    \begin{remark}
        The above definitions can be restricted to real manifolds and real metalinear structures.
    \end{remark}

    \newdef{Half-form}{\index{half-form}
        Consider a smooth manifold $M$ equipped with a metalinear frame bundle $F_\mathrm{ML}M$. The bundle of half-forms $M^{1/2}$ is defined as the associated $\mathbb{C}$-line bundle defined by the action $(g,\lambda)\mapsto e^{nz/2}\lambda$, where $g\equiv(A,z)\in\mathrm{ML}(n,\mathbb{C})$.

        Now, consider the bundle of 1-densities $|\Omega^1|(M)$ from Definition \ref{diff:honest_density}. There exists a map $\Gamma(M^{1/2})\times\Gamma(M^{1/2})\rightarrow\Gamma(|\Omega|^1(M))$ defined by sending the pair $(\mu,\nu)$ to the (tensor) product $\mu\overline{\nu}$ along the covering map $F_\mathrm{ML}M\rightarrow FM$. If one does not use the conjugation, a section of the ordinary $n$-form bundle $\Omega^n(M)$ is obtained. In this sense the existence of a metalinear structure is equivalent to the existence of a square root of the determinant line bundle.
    }

    \begin{property}[Metaplectic structure]\index{metaplectic!structure}\label{complex:metaplectic}
        Let $(M,\omega)$ be a symplectic manifold and consider a Lagrangian subbundle $L\subset TM$. The tangent bundle $TM$ admits a metaplectic structure if and only if $L$ admits a metalinear structure.
    \end{property}

\section{Complex differential forms}

    \begin{property}
        On a complex manifold there exist coordinates $\{z^i\}_{i\leq n}$ such that the almost complex structure $J$ can be written as
        \begin{gather}
            \label{complex:complex_structure}
            J = i\partial_k\otimes dz^k - i\partial_{\overline{k}}\otimes d\overline{z}^k.
        \end{gather}
        This coordinate expression can be used to find a coordinate transformation from the real coordinates $\{x^i\}_{i\leq2n}$ to the complex coordinates $\{z^i,\overline{z}^i\}_{i\leq n}$.
    \end{property}

    Using the basis forms $dz^i,d\overline{z}^i$ one can also define complex Grassmann spaces $\Omega^{p,q}(M)$, analogous to $\Omega^k(X)$ for smooth manifolds, for any point $m\in M$:
    \begin{align}
        \Omega^{1,0}_m(M) &:= \mathrm{span}_\mathbb{C}\{dz^i_m\}\\
        \Omega^{0,1}_m(M) &:= \mathrm{span}_\mathbb{C}\{d\overline{z}^i_m\}\\
        \Omega^{p,q}_m(M) &:= \left(\bigwedge_{i=1}^p\Omega^{1,0}_m\right)\wedge\left(\bigwedge_{j=1}^q\Omega^{0,1}_m\right).
    \end{align}

    \begin{property}
        The spaces $\Omega^{1,0}(M)$ and $\Omega^{0,1}(M)$ are stable, i.e. they transform tensorially, under holomorphic coordinate transformations. On the space \[\Omega^k(M) = \bigoplus_{p+q=k}\Omega^{p,q}(M)\] of forms of total degree $k$ one can then define the canonical projection maps $\pi^{p,q}:\Omega^k\rightarrow\Omega^{p,q}$.
    \end{property}

    \newdef{Dolbeault operator}{\index{Dolbeault!operator}
        Consider a general $(p+q)$-form $\omega\in\Omega^{p,q}(M)$. The de Rham differential maps this form to a $(p+q+1)$-form. This form is in general an element of $\sum_{r+s=p+q+1}\Omega^{r,s}(M)$. Using the projection maps $\pi^{p,q}$ one defines the Dolbeault operators as follows:
        \begin{align}
            \partial &:= \pi^{p+1,q}\circ d\\
            \overline{\partial} &:= \pi^{p,q+1}\circ d.
        \end{align}
    }
    \begin{property}
        By explicitly writing out the action of the de Rham differential $d$ on a general $(p,q)$-form one obtains the following decomposition:
        \begin{gather}
            d = \partial + \overline{\partial}.
        \end{gather}
        By using the coboundary property of $d$ one also obtains
        \begin{align}
            \partial^2 = \overline{\partial}^2 &= 0\\
            \partial\overline{\partial} + \overline{\partial}\partial &= 0.
        \end{align}
    \end{property}
    \begin{remark}[Integrability]
        It can be shown that $J$ is integrable, i.e. the almost complex structure is complex, if and only if the induced Dolbeault operator $\overline{\partial}$ squares to zero.
    \end{remark}

    \begin{formula}
        Analogous to the definition of the de Rham codifferential \eqref{riemann:codifferential}, one can define the adjoints of the Dolbeault operators:
        \begin{align}
            \partial^\dag &:= -\ast\partial\ast\\
            \overline{\partial}^\dag &:= -\ast\overline{\partial}\ast,
        \end{align}
        where the fact that the real dimension of a complex manifold is even is used: $(-1)^{n(k+1)+1} = -1$.
    \end{formula}
    \begin{result}
        Using these definitions one can write the Hodge Laplacian \eqref{riemann:hodge_laplacian} as:
        \begin{gather}
            \Delta = 2(\partial\partial^\dag + \partial^\dag\partial) = 2(\overline{\partial}\overline{\partial}^\dag + \overline{\partial}^\dag\overline{\partial}).
        \end{gather}
    \end{result}

\section{K\"ahler manifolds}\label{section:kahler}

    In analogy with the definition of Riemannian manifolds \ref{riemann:riemannian_manifold} one can also define metrics for complex vector bundles:
    \newdef{Hermitian manifold}{\index{Hermitian|seealso{manifold}}\index{manifold!Hermitian}
        A complex vector bundle equipped with a Hermitian bundle metric. A connection that is compatible with this metric is called a Hermitian connection.
    }

    \newdef{K\"ahler manifold}{\index{K\"ahler!manifold}
        Consider a smooth manifold equipped with a Riemannian structure $(M,g)$, a symplectic structure $(M,\omega)$ and an almost complex structure $J$. This manifold is called a K\"ahler manifold if the structures satisfy any of the following (equivalent) sets of compatibility conditions:
        \begin{enumerate}
            \item The almost complex structure $J$ is integrable\footnote{If not, the manifold is said to be almost K\"ahler.}, and
            \item The symplectic form is compatible with the almost complex structure:
                \begin{gather}
                    \omega(v,w) = \omega(Jv,Jw)
                \end{gather}
                and
                \begin{gather}
                    \omega(v,Jv)>0;
                \end{gather}
        \end{enumerate}
        or
        \begin{enumerate}
            \item $M$ is Hermitian with metric $h(v,w) := g(v,w) + ig(v,Jw)$, and
            \item The fundamental two-form $\omega(v,w) := g(v,Jw)$ is closed and hence symplectic\footnote{The nondegeneracy condition is automatically satisfied because of the nondegeneracy of the metric.};
        \end{enumerate}
        or\footnote{Here the symplectic structure can be recovered using the K\"ahler form defined below.}
        \begin{enumerate}
            \item $M$ is Hermitian with metric $h(v,w) := g(v,w) + ig(v,Jw)$, and
            \item $J$ is parallel with respect to the Levi-Civita connection on $(M,g)$:
            \begin{gather}
                \nabla_XJ = 0.
            \end{gather}
        \end{enumerate}
    }
    \remark{The property that says that $J$ acts isometrically can be interchanged for the statement that $J$ acts as a symplectomorphism. These two statements are equivalent on a K\"ahler manifold.}

    \newdef{K\"ahler form}{\index{Hermitian!form}\index{fundamental!form}
        The central object in all these definitions is the K\"ahler form (also called the \textbf{Hermitian form} or \textbf{fundamental form}):
        \begin{gather}
            \omega(v,w) := g(v,Jw).
        \end{gather}
    }

    \begin{formula}
        The metric $g = g_{ij}dx^i\otimes dx^j$ can be rewritten as
        \begin{gather}
            g = g_{i\overline{j}}\big(dz^i\otimes d\overline{z}^j + d\overline{z}^j\otimes dz^i\big).
        \end{gather}
        The K\"ahler form can then be written as
        \begin{gather}
            \label{complex:kahler_form}
            \omega = ig_{i\overline{j}}dz^i\wedge d\overline{z}^j.
        \end{gather}
    \end{formula}

    \newdef{K\"ahler potential}{\index{K\"ahler!potential}
        Using the $\partial\overline{\partial}$-lemma \ref{complex:del_delbar_lemma} one can locally write the K\"ahler form as
        \begin{gather}
            \label{complex:kahler_potential}
            \omega = i\partial\overline{\partial}K(z,\overline{z}),
        \end{gather}
        where the real function $K\in\Omega^0(M)$ is called the \textbf{K\"ahler potential}.
    }
    \begin{result}
        Formula \ref{complex:kahler_form} implies that one can locally rewrite the metric as
        \begin{gather}
            g_{i\overline{j}} = \partial_i\partial_{\overline{j}}K(z,\overline{z}).
        \end{gather}
    \end{result}

    \begin{property}
        The Christoffel symbols associated to the Levi-Civita connection on $(M,g)$ admit a simple expression when $M$ is K\"ahler. Only the $\Gamma^{\ \ k}_{ij}$ and $\Gamma^{\ \ \overline{k}}_{\overline{i}\overline{j}}$ components do not vanish. They are given by
        \begin{align}
            \Gamma^{\ \ k}_{ij} &= g^{k\overline{m}}\partial_ig_{j\overline{m}}\\
            \Gamma^{\ \ \overline{k}}_{\overline{i}\overline{j}} &= g^{\overline{k}m}\partial_{\overline{i}}g_{\overline{j}m}.
        \end{align}
        Accordingly, the only nonvanishing component of the Riemann curvature tensor is
        \begin{gather}
            R_{\overline{i}j\overline{k}l} = g_{\overline{k}m}\partial_{\overline{i}}\Gamma^{\ \ m}_{jl}.
        \end{gather}
    \end{property}

    \newdef{K\"ahler transformation}{
        From Definition \ref{complex:kahler_potential} one can conclude that the K\"ahler potential is not unambiguously defined. The following transformation leaves the K\"ahler form invariant:
        \begin{gather}
            K'(z,\overline{z}) = K(z,\overline{z}) + f(z) + \overline{f}(\overline{z}).
        \end{gather}
        On overlapping coordinate charts the transformation between K\"ahler potentials is exactly of this form.
    }

    \newdef{Hyperk\"ahler manifold}{
        A manifold is said to be hyperk\"ahler if it is \textit{hypercomplex} and if it admits a (Riemannian) metric that is K\"ahler with respect to all complex structures. Explicitly this means that:
        \begin{enumerate}
            \item there exist distinct complex structures $I,J,K$ such that $I^2 = J^2 = K^2 = IJK = -1$, and
            \item the K\"ahler forms induced by $I,J$ and $K$ are closed.
        \end{enumerate}
    }

\subsection{Killing vectors}

    \newdef{Holomorphic Killing vector}{\index{Killing!vector}
        Consider the set of Killing vector fields $X_A$ associated to the metric $g$. Within this set of vector fields one can consider the set of vector fields $k_A$ satisfying
        \begin{gather}
            \mathcal{L}_{k_A}J = 0.
        \end{gather}
        or, equivalently by the K\"ahler condition,
        \begin{gather}
            \mathcal{L}_{k_A} \omega = 0.
        \end{gather}
        These are called holomorphic Killing vector fields because their components are holomorphic in the sense of complex analysis. This can easily be shown by writing the Killing condition in terms of covariant derivatives and by using expression \ref{complex:complex_structure}.
    }

    \newdef{Moment map}{\index{moment!map}\index{generating!function}
        Let $k$ be a holomorphic Killing vector field. From $d\omega = 0$ one can, using Cartan's magic formula \ref{diff:cartan_magic_formula} and the above condition, derive that $\iota_k\omega$ is closed. Poincar\'e's lemma then implies that there exists a real function $\mathcal{P}(z,\overline{z})$ such that
        \begin{gather}
            \iota_k\omega = d\mathcal{P}.
        \end{gather}
        Using Equation \ref{complex:kahler_form} one can then find the following expression for the Killing vector fields:
        \begin{gather}
            k^i = -ig^{i\overline{j}}\partial_{\overline{j}}\mathcal{P}.
        \end{gather}
    }

\section{Cohomology}
%\subsection{Hodge-de Rham cohomology}
%
    %\newdef{Hodge-de Rham cohomology}{\index{Hodge!cohomology}
    %    Let $M$ be a complex manifold. The Hodge-de Rham cohomology class $H^{p, q}(M)$ is defined as follows:
    %    \begin{gather}
    %        H^{p, q}(M) = \frac{\ker(d_k)}{\im(d_{k-1})}
    %    \end{gather}
    %}

    ?? COMPLETE ??

\subsection{Dolbeault cohomology}

    \begin{theorem}[Hodge decomposition]\index{Hodge!decomposition}
        Let $M$ be a compact K\"ahler manifold.
        \begin{gather}
            H^k_{dR}(M)\cong\bigoplus_{p+q=k}H^{p,q}(M)
        \end{gather}
        for all $k\in\mathbb{N}$
    \end{theorem}

    By analogy with the Poincar\'e lemma for smooth manifolds one can prove the following theorems:
    \begin{theorem}[$\partial$-lemma]\index{$\partial$-lemma}
        Let $\alpha\in\Omega^{p,q}(M)$. If $\partial\alpha = 0$, then locally there exists a complex form $\beta\in\Omega^{p-1,q}$ such that $\alpha = \partial\beta$.
    \end{theorem}
    \begin{theorem}[$\overline{\partial}$-lemma]
        Let $\alpha\in\Omega^{p,q}(M)$. If $\overline{\partial}\alpha = 0$, then locally there exists a complex form $\beta\in\Omega^{p,q-1}$ such that $\alpha = \overline{\partial}\beta$.
    \end{theorem}
    \begin{theorem}[$\partial\overline{\partial}$-lemma]\label{complex:del_delbar_lemma}
        Let $\alpha\in\Omega^{p,q}(M)$. If $d\alpha = 0$, then locally there exists a complex form $\beta\in\Omega^{p-1,q-1}$ such that $\alpha = \partial\overline{\partial}\beta$.
    \end{theorem}

    ?? COMPLETE ??
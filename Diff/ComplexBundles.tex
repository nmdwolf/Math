\chapter{Complex Geometry}\label{chapter:complex_geometry}

\section{Complex structures}

    \newdef{Almost complex structure}{
        Let $M$ be a smooth manifold. An almost complex structure on $M$ is a (complexified) smooth $(1, 1)$-tensor field $J:TM^{\mathbb{C}}\rightarrow TM^{\mathbb{C}}$ such that $J|_p:T_pM^{\mathbb{C}}\rightarrow T_pM^{\mathbb{C}}$ satisfies $J|_p^2 = -1$ for all $p\in M$.
    }

    This definition implies the following property:
    \begin{property}
        An almost complex manifold is even-dimensional and orientable.
    \end{property}

    An almost complex structure induces a decomposition in so-called holomorphic and anti-holomorphic components:\[TM^{\mathbb{C}} = TM^+\oplus TM^-\] where both bundles have the same dimension. When the coordinates on $M$ are denoted by $\{x^k\}_{k\leq 2n}$, the bases for these two subbundles are given by \[\left\{\pderiv{}{z^k} := \frac{1}{2}\left(\pderiv{}{x}_{^{2k-1}} - i\pderiv{}{x}_{^{2k}}\right)\right\}_{k\leq n}\] and \[\left\{\pderiv{}{\overline{z}^k} := \frac{1}{2}\left(\pderiv{}{x}_{^{2k-1}} + i\pderiv{}{x}_{^{2k}}\right)\right\}_{k\leq n}\] respectively.

    \remark{The reason that we define the almost complex structure on the complexified tangent bundle has to do with the fact that $J$ is only diagonalizable on a complex vector space (because it squares to a negative value).}

    \begin{example}[Complex vector spaces]
        Consider a complex vector space $V$. If we look at property \ref{tensor:complexification_decomposition} and use the canonical isomorphism $V\cong T_vV$ for vector spaces, we see that the automorphism $v\mapsto iv$ induced by the imaginary unit gives rise to an almost complex structure on $V$.
    \end{example}

    \begin{property}
        A manifold $M$ admits an almost complex structure if and only if the structure group of the tangent bundle $TM$ can be reduced from $\text{GL}(\mathbb{R}^{2n})$ to $\text{GL}(\mathbb{C}^n)$.
    \end{property}

    \newdef{Complex manifold}{
        A topological space $M$ for which there exists an open cover $\{U_i\}_i$ such that for every $U_i$ there exists a homeomorphism $\varphi_i:U_i\rightarrow \mathbb{C}^n$ onto some open subset of $\mathbb{C}^n$. The transition functions $\varphi_{ji}:\varphi_i(U_i\cap U_j)\rightarrow \varphi_j(U_i\cap U_j)$ are also required to be holomorphic.
    }
    \newdef{Complex dimension}{
        The integer $n$ in previous definition is called the complex dimension of $M$, denoted by $\dim_{\mathbb{C}}(M)$.
    }

    \begin{property}
        An almost complex manifold is complex if and only if the $\text{GL}(\mathbb{C}^n)$-structure is integrable. The integrability condition can be rephrased algebraically as follows:
    \end{property}
    \begin{theorem}[Newlander-Nirenberg]\index{Nijenhuis!tensor}\index{integrable!complex structure}
        An almost complex manifold is complex if and only if the \textbf{Nijenhuis tensor} $N_J$ vanishes for all vector fields:
        \begin{gather}
            \label{complex:integrable_structure}
            N_J(X, Y) = [JX, JY] - J[JX, Y] - J[X, JY] - [X, Y] = 0.
        \end{gather}
    \end{theorem}
    When working in a local coordinate-induced basis we obtain the following condition:
    \begin{gather}
        J_\rho^\nu\partial_\nu J_\sigma^\mu - J_\sigma^\nu\partial_\nu J_\rho^\mu - J_\nu^\mu\partial_\rho J_\sigma^\nu + J_\nu^\mu\partial_\sigma J_\rho^\mu = 0.
    \end{gather}

\section{Complex differential forms}

    \begin{property}
        On a complex manifold there exist coordinates $\{z^i\}_{i\leq n}$ such that the almost complex structure $J$ can be written as
        \begin{gather}
            \label{complex:complex_structure}
            J = i\partial_k\otimes dz^k - i\partial_{\overline{k}}\otimes d\overline{z}^k.
        \end{gather}
        This coordinate expression can be used, given an almost complex manifold satisfying \ref{complex:integrable_structure}, to find a coordinate transformation from the real coordinates $\{x^i\}_{i\leq2n}$ to the complex coordinates $\{z^i, \overline{z}^i\}_{i\leq n}$.
    \end{property}

    Using the basis forms $dz^i, d\overline{z}^i$ one can also define complex Grassmann spaces $\Omega^{p, q}_m(M)$, analogous to $\Omega^k(X)$ for smooth manifolds, for any point $m\in M$:
    \begin{align}
        \Omega^{1, 0}_m(M) &:= \text{span}_{\mathbb{C}}\{dz^i_m\}\\
        \Omega^{0, 1}_m(M) &:= \text{span}_{\mathbb{C}}\{d\overline{z}^i_m\}\\
        \Omega^{p, q}_m(M) &:= \left(\bigwedge_{i=1}^p\Omega^{1, 0}_m\right)\wedge\left(\bigwedge_{j=1}^q\Omega^{0, 1}_m\right).
    \end{align}

    \begin{property}
        The spaces $\Omega^{1, 0}(M)$ and $\Omega^{0, 1}(M)$ are stable under holomorphic coordinate transformations, i.e. they transform tensorially. As such they are complex vector bundles. On the bundle \[\Omega^k(M) = \bigoplus_{p+q=k}\Omega^{p, q}(M)\] of forms of total degree $k$ one can then define the canonical projection maps $\pi^{p, q}:\Omega^k\rightarrow\Omega^{p, q}$.
    \end{property}

    \newdef{Dolbeault operator}{\index{Dolbeault!operator}
        Consider a general $(p+q)$-form $\omega\in\Omega^{p, q}(M)$. The de Rham differential maps this form to a $(p+q+1)$-form. This form is in general an element of $\sum_{r+s=p+q+1}\Omega^{r, s}(M)$. Using the projection maps $\pi^{p, q}$ one defines the Dolbeault operators:
        \begin{align}
            \partial &:= \pi^{p+1, q}\circ d\\
            \overline{\partial} &:= \pi^{p, q+1}\circ d.
        \end{align}
    }
    \begin{property}
        By explicitly writing out the action of the de Rham differential $d$ on a general $(p, q)$-form one obtains the following decomposition:
        \begin{gather}
            d = \partial + \overline{\partial}.
        \end{gather}
        By using the coboundary property of $d$ one also obtains
        \begin{align}
            \partial^2 = \overline{\partial}^2 &= 0\\
            \partial\overline{\partial} + \overline{\partial}\partial &= 0.
        \end{align}
    \end{property}
    \begin{remark}
        It can be shown that $J$ is integrable if and only if the induced Dolbeault operator $\overline{\partial}$ squares to zero.
    \end{remark}

    \begin{formula}
        Analogous to the definition of the de Rham codifferential \ref{manifolds:codifferential} one can define the adjoint of Dolbeault operators:
        \begin{align}
            \partial^\dag &:= -\ast\partial\ast\\
            \overline{\partial}^\dag &:= -\ast\overline{\partial}\ast
        \end{align}
        where we used the fact that the real dimension of a complex manifold is even: $(-1)^{n(k+1)+1} = -1$.
    \end{formula}
    \begin{result}
        Using these definitions one can write the Hodge Laplacian \ref{diff:hodge_laplacian} as:
        \begin{gather}
            \Delta = 2(\partial\partial^\dag + \partial^\dag\partial) = 2(\overline{\partial}\overline{\partial}^\dag + \overline{\partial}^\dag\overline{\partial}).
        \end{gather}
    \end{result}

\section{K\"ahler manifolds}

    \newdef{K\"ahler manifold}{\index{K\"ahler!manifold}
        Consider a smooth manifold equipped with a Riemannian structure $(M, g)$, a symplectic structure $(M, \omega)$ and an almost complex structure $J$. This manifold is a K\"ahler manifold if the structures satisfy any of the following (equivalent) sets of compatibility conditions:
        \begin{enumerate}
            \item The almost complex structure $J$ is integrable\footnote{If not the manifold is said to be almost K\"ahler.}.
            \item The symplectic form is compatible with the almost complex structure:
                \begin{gather}
                    \omega(v, w) = \omega(Jv, Jw)
                \end{gather}
                and
                \begin{gather}
                    \omega(v, Jv)>0.
                \end{gather}
        \end{enumerate}
        or
        \begin{enumerate}
            \item $M$ is Hermitian:
                \begin{gather}
                    g_{\mathbb{C}}(v, w) = g_{\mathbb{C}}(Jv, Jw)
                \end{gather}
                where $g_{\mathbb{C}}$ is defined as the linear extension of $g$ from $TM$ to $T^{\mathbb{C}}M$.
            \item The fundamental two-form $\omega(v, w) = g_{\mathbb{C}}(v, Jw)$ is closed\footnote{The nondegeneracy condition is automatically satisfied because of the nondegeneracy of the metric.}.
        \end{enumerate}
        or\footnote{Here the symplectic structure can be recovered using the K\"ahler form defined below.}
        \begin{enumerate}
            \item $M$ is Hermitian.
            \item $J$ is parallel with respect to the Levi-Civita connection on $(M, g)$:
            \begin{gather}
                \nabla_XJ = 0.
            \end{gather}
        \end{enumerate}
    }
    \remark{The property, which says that $J$ acts isometrically, can be interchanged for the statement that $J$ acts as a symplectomorphism. These two statements are equivalent on a K\"ahler manifold.}

    \newdef{K\"ahler form}{\index{Hermitian!form}\index{fundamental!form}
        The central object in all these definitions is the K\"ahler form\footnote{Sometimes called the \textbf{Hermitian form} or \textbf{fundamental form}.}:
        \begin{gather}
            \omega(v, w) := g_{\mathbb{C}}(v, Jw).
        \end{gather}
        The K\"ahler form and metric together define a Hermitian metric $h=g+i\omega$.
    }

    \begin{formula}
        The metric $g = g_{ij}dx^i\otimes dx^j$ can be rewritten as
        \begin{gather}
            g = g_{i\overline{j}}\big(dz^i\otimes d\overline{z}^j + d\overline{z}^j\otimes dz^i\big).
        \end{gather}
        The K\"ahler form can then be written as
        \begin{gather}
            \label{complex:kahler_form}
            \omega = ig_{i\overline{j}}dz^i\wedge d\overline{z}^j.
        \end{gather}
    \end{formula}

    \newdef{K\"ahler potential}{\index{K\"ahler!potential}
        Using the $\partial\overline{\partial}$-lemma \ref{complex:del_delbar_lemma} we can locally write the K\"ahler form as follows:
        \begin{gather}
            \label{complex:kahler_potential}
            \omega = i\partial\overline{\partial}K(z, \overline{z})
        \end{gather}
        where the real function $K\in\Omega^0(M)$ is called the \textbf{K\"ahler potential}.
    }
    \begin{result}
        Formula \ref{complex:kahler_form} implies that we can (locally) rewrite the metric as
        \begin{gather}
            g_{i\overline{j}} = \partial_i\partial_{\overline{j}}K(z, \overline{z}).
        \end{gather}
    \end{result}

    \begin{property}
        The Christoffel symbols associated to the Levi-Civita connection on $(M, g)$ admits a simple expression when $M$ is K\"ahler. Only the $\Gamma^{\ \ k}_{ij}$ and $\Gamma^{\ \ \overline{k}}_{\overline{i}\overline{j}}$ components remain and they are given by
        \begin{align}
            \Gamma^{\ \ k}_{ij} &= g^{k\overline{m}}\partial_ig_{j\overline{m}}\\
            \Gamma^{\ \ \overline{k}}_{\overline{i}\overline{j}} &= g^{\overline{k}m}\partial_{\overline{i}}g_{\overline{j}m}.
        \end{align}
        Accordingly, the only nonvanishing component of the Riemann curvature tensor is
        \begin{gather}
            R_{\overline{i}j\overline{k}l} = g_{\overline{k}m}\partial_{\overline{i}}\Gamma^{\ \ m}_{jl}.
        \end{gather}
    \end{property}

    \newdef{K\"ahler transformation}{
        From definition \ref{complex:kahler_potential} one can conclude that the K\"ahler potential is not defined unambiguously. The following transformation leaves the K\"ahler form invariant:
        \begin{gather}
            K'(z, \overline{z}) = K(z, \overline{z}) + f(z) + \overline{f}(\overline{z}).
        \end{gather}
        On overlapping coordinate charts the transformation between K\"ahler potentials is exactly of this form.
    }

    \newdef{Hyperk\"ahler manifold}{
        A manifold is said to be hyperk\"ahler if it \textit{hypercomplex} and if it admits a (Riemannian) metric which is K\"ahler with respect to all complex structures. Explicitly this means that:
        \begin{enumerate}
            \item There exist complex structures $I, J, K$ such that $I^2 = J^2 = K^2 = IJK = -1$.
            \item The K\"ahler forms induced by $I, J$ and $K$ are closed.
        \end{enumerate}
    }

\subsection{Killing vectors}

    \newdef{Holomorphic Killing vector}{\index{Killing!vector}
        Consider the set of Killing vectors $X_A$ associated to the metric $g$, i.e. $\mathcal{L}_{X_A}g = 0$. Within this set of vectors one can consider the set of vectors $k_A$ satisfying
        \begin{gather}
            \mathcal{L}_{k_A}J = 0.
        \end{gather}
        These are called holomorphic Killing vectors because their components are holomorphic in the sense of complex analysis. This can easily be shown by writing the Killing condition in terms of covariant derivatives and by using expression \ref{complex:complex_structure}.
    }
    \begin{remark}
        Using the K\"ahler condition one can equivalently require the following condition:
        \begin{gather}
            \mathcal{L}_{k_A} \omega = 0.
        \end{gather}
    \end{remark}

    \newdef{Moment map}{\index{moment!map}\index{generating!function}
        Let $k$ be a holomorphic Killing vector. From $d\omega = 0$ one can, using Cartan's magic formula \ref{forms:cartan_magic_formula} and the above condition, derive that $\iota_k\omega$ is closed. Poincar\'e's lemma then implies that there exists a real function $\mathcal{P}(z, \overline{z})$ such that
        \begin{gather}
            \iota_k\omega = d\mathcal{P}.
        \end{gather}
        Using the expression \ref{complex:kahler_form} one can then find the following expression for the Killing vector:
        \begin{gather}
            k^i = -ig^{i\overline{j}}\partial_{\overline{j}}\mathcal{P}.
        \end{gather}
    }

\section{Cohomology}
%\subsection{Hodge-de Rham cohomology}
%
    %\newdef{Hodge-de Rham cohomology}{\index{Hodge!cohomology}
    %    Let $M$ be a complex manifold. The Hodge-de Rham cohomology class $H^{p, q}(M)$ is defined as follows:
    %    \begin{gather}
    %        H^{p, q}(M) = \frac{\ker(d_k)}{\im(d_{k-1})}
    %    \end{gather}
    %}

\subsection{Dolbeault cohomology}

    \begin{theorem}[Hodge decomposition]\index{Hodge!decomposition}
        Let $M$ be a compact K\"ahler manifold. For all $k\in\mathbb{N}$ we have
        \begin{gather}
            H^k_{dR}(M) \cong \bigoplus_{p+q=k}H^{p, q}(M).
        \end{gather}
    \end{theorem}

    By analogy with the Poincar\'e lemma for smooth manifolds we have the following theorems:
    \begin{theorem}[$\partial$-lemma]\index{$\partial$-lemma}
        Let $\alpha\in\Omega^{p, q}(M)$. If $\partial\alpha = 0$ then locally there exists a complex form $\beta\in\Omega^{p-1, q}$ such that $\alpha = \partial\beta$.
    \end{theorem}
    \begin{theorem}[$\overline{\partial}$-lemma]
        Let $\alpha\in\Omega^{p, q}(M)$. If $\overline{\partial}\alpha = 0$ then locally there exists a complex form $\beta\in\Omega^{p, q-1}$ such that $\alpha = \overline{\partial}\beta$.
    \end{theorem}
    \begin{theorem}[$\partial\overline{\partial}$-lemma]\label{complex:del_delbar_lemma}
        Let $\alpha\in\Omega^{p, q}(M)$. If $d\alpha = 0$ then locally there exists a complex form $\beta\in\Omega^{p-1, q-1}$ such that $\alpha = \partial\overline{\partial}\beta$.
    \end{theorem}

    ?? COMPLETE ??
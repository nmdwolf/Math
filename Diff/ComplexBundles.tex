\chapter{Complex Geometry}

\section{Complex structures}

	\newdef{Almost complex structure}{
		Let $M$ be a smooth manifold. An almost complex structure on $M$ is a smooth $(1, 1)$-tensor field $J:TM^{\mathbb{C}}\rightarrow TM^{\mathbb{C}}$ such that $J|_p:T_pM^{\mathbb{C}}\rightarrow T_pM^{\mathbb{C}}$ satisfies $J|_p^2 = -1$ for all $p\in M$.
		
		This structure induces a decomposition in so-called holomorphic and anti-holomorphic components:\[TM^{\mathbb{C}} = TM^+\oplus TM^-\] where both bundles have the same dimension. When the coordinates on $M$ are denoted by $\{x^k\}_{k\leq 2n}$ the basis for these two subbundles is given by $\{\pderiv{}{z^k} = \frac{1}{2}\left(\pderiv{}{x}_{^{2k-1}} - i\pderiv{}{x}_{^{2k}}\right)\}_{k\leq n}$ and $\{\pderiv{}{\overline{z}^k} = \frac{1}{2}\left(\pderiv{}{x}_{^{2k-1}} + i\pderiv{}{x}_{^{2k}}\right)\}_{k\leq n}$ respectively.
	}
	\remark{The reason that we define the almost complex structure on the complexified tangent bundle has to do with the fact that $J$ is only diagonalizable on a complex vector space (because they square to a negative value).}
	
	\begin{property}
		An almost complex manifold is even-dimensional and orientable.
	\end{property}
	
	\begin{property}
		A manifold $M$ admits an almost complex structure if and only if the structure group of the tangent bundle $TM$ can be reduced from $GL(\mathbb{R}^{2n})$ to $GL(\mathbb{C}^n)$.
	\end{property}
	
	\newdef{Complex manifold}{
		A topological space $M$ for which there exists an open cover $\{U_i\}_i$ such that for every $U_i$ there exists a homeomorphism $\varphi_i:U_i\rightarrow \mathbb{C}^n$ onto some open subset of $\mathbb{C}^n$.	Furthermore the transition functions $\varphi_{ji}:\varphi_i(U_i\cap U_j)\rightarrow \varphi_j(U_i\cap U_j)$ are also required to be holomorphic.
	}
	\newdef{Complex dimension}{
		The integer $n$ in previous definition is called the complex dimension of $M$, denoted by $\dim_{\mathbb{C}}(M)$.
	}
	
	\begin{theorem}[Newlander-Nirenberg]\index{Nijenhuis!tensor}\index{integrable!complex structure}
		An almost complex manifold is complex if and only if the \textbf{Nijenhuis tensor} $N_J$ vanishes for all vector fields:
		\begin{equation}
			\label{complex:integrable_structure}
			N_J(X, Y) = [JX, JY] - J[JX, Y] - J[X, JY] - [X, Y] = 0
		\end{equation}
	\end{theorem}
	When working in a local coordinate-induced basis we obtain the following condition:
	\begin{equation}
		J_\rho^\nu\partial_\nu J_\sigma^\mu - J_\sigma^\nu\partial_\nu J_\rho^\mu - J_\nu^\mu\partial_\rho J_\sigma^\nu + J_\nu^\mu\partial_\sigma J_\rho^\mu = 0
	\end{equation}
	An almost complex structure satisfying this condition is also said to be integrable. (See also next property.)
	
\section{Complex differential forms}
	
	\begin{property}
		On a complex manifold there exist coordinates $\{z^i\}_i$ such that the almost complex structure $J$ can be written as:
		\begin{equation}
			J = i\partial_k\otimes dz^k - i\partial_{\overline{k}}\otimes d\overline{z}^k
		\end{equation}
		This coordinate expansion can be used to, given an almost complex manifold satisfying \ref{complex:integrable_structure}, find a coordinate transformation from the real coordinates $\{x^i\}_{i\leq2n}$ to the complex coordinates $\{z^i\}_{i\leq n}$.
	\end{property}

	Using the basis forms $dz^i, d\overline{z}^i$ one can also define complex Grassmann spaces $\Omega^{p, q}_m(M)$, analogous to $\Omega^k(X)$ for smooth manifolds, for any point $m\in M$:
	\begin{align}
		\Omega^{1, 0}_m(M) &= \text{span}_{\mathbb{C}}\{dz^i_m\}\\
		\Omega^{0, 1}_m(M) &= \text{span}_{\mathbb{C}}\{d\overline{z}^i_m\}\\
		\Omega^{p, q}_m(M) &= \left(\bigwedge_{i=1}^p\Omega^{1, 0}_m\right)\wedge\left(\bigwedge_{j=1}^q\Omega^{0, 1}_m\right)
	\end{align}
	
	\begin{property}
		The spaces $\Omega^{1, 0}(M)$ and $\Omega^{0, 1}(M)$ are stable under holomorphic coordinate transformations, i.e. they transform tensorially and hence they are complex vector bundles.
	\end{property}
	On the bundle $\Omega^k(M) = \bigoplus_{p+q=k}\Omega^{p, q}(M)$ of forms of total degree $k$ one can then define the canonical projection maps $\pi^{p, q}:\Omega^k\rightarrow\Omega^{p, q}$.
	
	\newdef{Dolbeault operator}{\index{Dolbeault!operator}
		Consider a general $(p+q)$-form $\omega\in\Omega^{p, q}(M)$. The de Rham differential maps this form to a $(p+q+1)$-form. This form is in general an element of $\sum_{r+s=p+q+1}\Omega^{r, s}(M)$. Using the projection maps $\pi^{p, q}$ one obtains the Dolbeault operators:
		\begin{align}
			\partial &= \pi^{p+1, q}\circ d\\
			\overline{\partial} &= \pi^{p, q+1}\circ d
		\end{align}
	}
	\begin{property}
		By explicitly writing out the action of the de Rham differential $d$ on a general $(p, q)$-form one obtains the following decomposition:
		\begin{equation}
			d = \partial + \overline{\partial}
		\end{equation}
		and by using the coboundary property of $d$:
		\begin{align}
			\partial^2 = \overline{\partial}^2 &= 0\\
			\partial\overline{\partial} + \overline{\partial}\partial &= 0
		\end{align}
	\end{property}
	
	\begin{formula}
		Analogous to the definition of the de Rham codifferential \ref{manifolds:codifferential} one can define the adjoint of the $\partial$ and $\overline{\partial}$ operators:
		\begin{align}
			\partial^\dag &= -\ast\partial\ast\\
			\overline{\partial}^\dag &= -\ast\overline{\partial}\ast
		\end{align}
		where we used the fact that the real dimension of a complex manifold is even: $(-1)^{n(k+1)+1} = -1$.
	\end{formula}
	\begin{result}
		Using these definitions one can rewrite the Hodge Laplacian as:
		\begin{equation}
			\Delta = 2(\partial\partial^\dag + \partial^\dag\partial) = 2(\overline{\partial}\overline{\partial}^\dag + \overline{\partial}^\dag\overline{\partial})
		\end{equation}
	\end{result}
	
\section{K\"ahler manifolds}

	\newdef{K\"ahler manifold}{\index{K\"ahler!manifold}
		Consider a smooth manifold equipped with a Riemannian structure $(M, g)$, a symplectic structure $(M, \omega)$ and an almost complex structure $J$. This manifold is a K\"ahler manifold if the structures satisfy following equivalent sets of compatibility conditions:
		\begin{itemize}
			\item The almost complex structure $J$ is integrable\footnote{If not then the manifold is said to be almost K\"ahler.}.
			\item The symplectic form is compatible with the almost complex structure:
				\begin{equation}
					\omega(v, w) = \omega(Jv, Jw)
				\end{equation}
				and
				\begin{equation}
					\omega(v, Jv)>0
				\end{equation}
		\end{itemize}

		or
		\begin{itemize}
			\item $M$ is Hermitian\footnote{$g_{\mathbb{C}}$ is the extension (by linearity) of $g$ from $TM$ to $T^{\mathbb{C}}M$. This map has $\mathbb{C}$ as codomain.}:
				\begin{equation}
					g_{\mathbb{C}}(v, w) = g_{\mathbb{C}}(Jv, Jw)
				\end{equation}
			\item The fundamental two-form $\omega(v, w) = g_{\mathbb{C}}(Jv, w)$ is closed\footnote{The non-degeneracy follows from the non-degeneracy of the metric.}.
		\end{itemize}
		or
		\begin{itemize}
			\item $M$ is Hermitian (and in particular complex).
			\item $J$ is stable under parallel transport with respect to the Levi-Civita connection on $(M, g)$:
			\begin{equation}
				\nabla_XJ = 0
			\end{equation}
		\end{itemize}
	}
	\remark{The property, which says that $J$ acts isometrically, can be interchanged for the statement that $J$ acts as a symplectomorphism. These two statements are equivalent if the second property holds.}
	
	\newdef{K\"ahler form}{
		The central object in all these definitions is the K\"ahler form:
		\begin{equation}
			\boxed{\omega(v, w) = g_{\mathbb{C}}(Jv, w)}
		\end{equation}
	}
	
	\begin{formula}
		The metric $g = g_{ij}dx^i\otimes dx^j$ can be rewritten as:
		\begin{equation}
			g = 2g_{i\overline{j}}dz^i\otimes d\overline{z}^j
		\end{equation}
		The K\"ahler form can be written as:
		\begin{equation}
			\omega = ig_{i\overline{j}}dz^i\wedge d\overline{z}^j
		\end{equation}
	\end{formula}
	
	\newdef{K\"ahler potential}{\index{K\"ahler!potential}
		Using the $\partial\overline{\partial}$-lemma \ref{complex:del_delbar_lemma} we can locally write the K\"ahler form as:
		\begin{equation}
			\omega = i\partial\overline{\partial}K(z, \overline{z})
		\end{equation}
		where $K\in\Omega^0(M)$ is called the \textbf{K\"ahler potential}.
	}
	\begin{result}
		Using the K\"ahler potential we can locally rewrite the Hermitian metric as:
		\begin{equation}
			g_{i\overline{j}} = \partial_i\partial_{\overline{j}}K(z, \overline{z})
		\end{equation}
	\end{result}
	
	\begin{property}
		The Christoffel symbols associated to the Levi-Civita connection on $(M, g)$ are also quite easy when $M$ is K\"ahler. Only the $\Gamma^{\ \ k}_{ij}$ and $\Gamma^{\ \ \overline{k}}_{\overline{i}\overline{j}}$ components remain and they are given by:
		\begin{align}
			\Gamma^{\ \ k}_{ij} &= g^{k\overline{m}}\partial_ig_{j\overline{m}}\\
			\Gamma^{\ \ \overline{k}}_{\overline{i}\overline{j}} &= g^{\overline{k}m}\partial_{\overline{i}}g_{\overline{j}m}
		\end{align}
		Accordingly the only non-vanishing component of the Riemann curvature tensor is:
		\begin{equation}
			R_{\overline{i}j\overline{k}l} = g_{\overline{k}m}\partial_{\overline{i}}\Gamma^{\ \ m}_{jl}
		\end{equation}
	\end{property}
	
\section{Cohomology}
\subsection{Hodge-de Rham cohomology}

	%\newdef{Hodge-de Rham cohomology}{\index{Hodge!cohomology}
	%	Let $M$ be a complex manifold. The Hodge-de Rham cohomology class $H^{p, q}(M)$ is defined as follows:
	%	\begin{equation}
	%		H^{p, q}(M) = \frac{\ker(d_k)}{\im(d_{k-1})}
	%	\end{equation}
	%}

	\begin{theorem}[Hodge decomposition]\index{Hodge!decomposition}
		Let $M$ be a K\"ahler manifold. For all $k\in\mathbb{N}$ we have:
		\begin{equation}
			H^k(M) = \bigoplus_{p+q=k}H^{p, q}(M)
		\end{equation}
	\end{theorem}

\subsection{Dolbeault cohomology}
	
	\begin{theorem}[Hodge decomposition]\index{Hodge!decomposition}
		Let $M$ be a K\"ahler manifold. For all $k\in\mathbb{N}$ we have:
		\begin{equation}
			H^k(M) = \bigoplus_{p+q=k}H^{p, q}(M)
		\end{equation}
	\end{theorem}
	
	By analogy with the Poincar\'e lemma for smooth manifolds we have following theorems:
	\begin{theorem}[$\partial$-lemma]\index{$\partial$-lemma}
		Let $\alpha\in\Omega^{p, q}(M)$. If $\partial\alpha = 0$ then there exists locally a complex form $\beta\in\Omega^{p-1, q}$ such that $\alpha = \partial\beta$.
	\end{theorem}
	\begin{theorem}[$\overline{\partial}$-lemma]
		Let $\alpha\in\Omega^{p, q}(M)$. If $\overline{\partial}\alpha = 0$ then there exists locally a complex form $\beta\in\Omega^{p, q-1}$ such that $\alpha = \overline{\partial}\beta$.
	\end{theorem}
	\begin{theorem}[$\partial\overline{\partial}$-lemma]\label{complex:del_delbar_lemma}
		Let $\alpha\in\Omega^{p, q}(M)$. If $d\alpha = 0$ then there exists locally a complex form $\beta\in\Omega^{p-1, q-1}$ such that $\alpha = \partial\overline{\partial}\beta$.
	\end{theorem}

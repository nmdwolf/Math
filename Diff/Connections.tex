\section{Connections}
\subsection{Vertical bundle}
	
	Because smooth fibre bundles (which include smooth principal G-bundles) are also smooth manifolds we can define the traditional notions for them, such as the tangent bundle. We use this to construct the notions of horizontal and vertical bundles:
	\newdef{Vertical vector}{\index{vertical vector}
		Let $\pi:E\rightarrow B$ be a smooth fibre bundle. The subbundle $\ker(T\pi)$ of $TE$ is called the vertical bundle of $E$. Fibrewise this gives us $V_x = T_x(E_{\pi(x)})$.
	}

	For principal G-bundles we can use an equivalent definition:
	\begin{adefinition}
		Consider a smooth principal $G$-bundle $G\hookrightarrow P\xrightarrow{\pi} M$. We first construct a map $\iota_p$ for every element $p\in P$:
		\eq{
			\iota_p:G\rightarrow P: g\mapsto p\cdot g
		}
		We then define a tangent vector $v\in T_p P$ to be vertical if it lies in the image of $T_e\iota_p$, i.e. $\text{Vert}(T_pP) = \text{im}(T_e\iota_p)$. This construction is supported by the exactness of following short sequence:
		\begin{equation}
			0\xrightarrow{} \mathfrak{g} \xrightarrow{T_e\iota_p} T_p P\xrightarrow{T_p\pi} T_xM \xrightarrow{} 0
		\end{equation}
	\end{adefinition}
	\begin{property}[Dimension]
		It follows from the second definition that the vertical vectors of a principal G-bundle are nothing but the pushforward of the Lie algebra $\mathfrak{g}$ under the right action of $G$ on $P$. Furthermore, the exactness of the sequence implies that $T_e\iota_p:\mathfrak{g}\rightarrow\text{Vert}(T_pP)$ is an isomorphism of vector spaces. In particular, it implies that
		\begin{equation}
			\label{manifolds:vertical_dimension}
			\dim\text{Vert}(T_pP) = \dim\mathfrak{g} = \dim G
		\end{equation}
	\end{property}
	
	\newdef{Fundamental vector field}{
		Consider a principle $G$-bundle. Let $A\in\mathfrak{g}$, where $\mathfrak{g}$ is the Lie algebra corresponding to $G$. The vertical vector field $A^\#:P\rightarrow TP$ given by
		\begin{equation}
			\label{manifolds:fundamental_vector_field}
			A^\#(p) = \iota_{p,\ast}(A)\in\text{Vert}(T_pP)
		\end{equation}
		is called the fundamental vector field associated to $A$.
	}
	\begin{property}
		The map $(\cdot)^\#:\mathfrak{g}\rightarrow\Gamma(TP)$ is a Lie algebra morphism:
		\begin{equation}
			[A, B]^\# = [A^\#, B^\#]
		\end{equation}
		where the Lie bracket on the left is that in $\mathfrak{g}$ and the Lie bracket on the right is that in $\mathfrak{X}(M)$ given by \ref{manifolds:lie_bracket}.
	\end{property}
	
		\begin{property}
		The vertical bundle satisfies the following $G$-equivariance condition:
		\begin{equation}
			R_{g, \ast}(\text{Vert}(T_pP)) = \text{Vert}(T_{pg}P)
		\end{equation}
		
		By differentiating the equality \[R_g\circ\iota_p = \iota_{pg}\circ\text{ad}_{g^{-1}}\] and using \ref{manifolds:fundamental_vector_field} and \ref{lie:adjoint_rep_of_group} we obtain the following algebraic formulation of the $G$-equivariance condition:
		\begin{equation}
			R_{g, \ast}\left(A^\#(p)\right) = \left(\text{Ad}_{g^{-1}}A\right)^\#(pg)
		\end{equation}
	\end{property}
	
\subsection{Horizontal bundle}

	\newdef{Connection}{\index{connection}
		\label{manifolds:connection}
		Consider a principal bundle $P$ with structure group $G$. A connection on $P$ is the selection of a subspace $\text{Hor}(T_pP)\leq T_pP$ for every $p\in P$ such that:
		\begin{itemize}
			\item $\text{Vert}(T_pP)\oplus\text{Hor}(T_pP) = T_pP$
			\item The selection depends smoothly on $p$.\footnote{See the definiton of a (smooth) distribution \ref{manifolds:distribution}.}
			\item The subspace $\text{Hor}(T_pP)$ is $G$-equivariant:
			\begin{equation}
				R_{g, \ast}(\text{Hor}(T_pP)) = \text{Hor}(T_{pg}P)
			\end{equation}
		\end{itemize}
		The elements of $\text{Hor}(T_pP)$ are said to be \textbf{horizontal vectors} with respect to the connection.
	}
	\begin{remark}
		Remark that the $G$-invariance condition for vertical bundles is an intrinsic property while we have to require it by definition for the horizontal bundle.
	\end{remark}
	
	\newdef{Horizontal bundle}{
		The horizontal bundle $\text{Hor}(TP)$ is defined as $\bigcup_{p\in P}\text{Hor}(T_pP)$. It is a subbundle of $TP$. The $G$-invariance condition then implies that this subbundle is invariant under (the pushforward of) the  right action of $G$.
	}
	
	\begin{property}[Dimension]
		Properties \ref{manifolds:principal_bundle_dimension}, \ref{manifolds:vertical_dimension} and the direct sum decomposition of $T_pP$ imply the following relation:
		\begin{equation}
			\dim\text{Hor}(T_pP) = \dim M
		\end{equation}
	\end{property}
	\begin{property}
		\label{manifolds:connection_dimensions}
		We take some time to summarize all dimensional relations between the components of a principal $G$-bundle over a base manifold $M$:
		\begin{empheq}[box=\widefbox]{align}
			\dim P &= \dim M + \dim G\\
			\dim M &= \dim\text{Hor}(T_pP)\\
			\dim G &= \dim\text{Vert}(T_pP)
		\end{empheq}
		for all $p\in P$.
	\end{property}
	
	\newdef{Dual connection}{\index{dual!connection}
		First we define the dual of the horizontal bundle:
		\begin{equation}
			\text{Hor}(T_p^*P) = \{h^*\in T_p^*P|h^*(v)=0, v\in\text{Vert}(T_pP)\}
		\end{equation}
		Equivalently, the horizontal covector bundle is defined as the set of linear functionals that annihilate vertical vectors. Just as with the vertical bundle this structure is independent of any connection on $P$.
		
		A dual connection can then be defined as the selection of a vertical covector bundle $\text{Vert}(T_p^*P)$ satisfying the conditions of definition \ref{manifolds:connection} where $\text{Vert}$ and $\text{Hor}$ should be interchanged.
	}
	
\subsection{Ehresmann connection}\index{Ehresmann connection}

	\newdef{Ehresmann connection}{
		Let $(P, M, \pi, G)$ be a principal bundle. An Ehresmann connection is a $\mathfrak{g}$-valued 1-form $\omega:TP\rightarrow\mathfrak{g}$ that satisfies following 2 conditions:
		
		\begin{enumerate}
			\item
			\begin{equation}
				\omega\circ R_{g, \ast} = \text{Ad}_{g^{-1}}\circ\omega
			\end{equation}
			\item
			\begin{equation}
				\omega(A^\#) = A
			\end{equation}
		\end{enumerate}
		The horizontal subspaces are then defined as $\text{Hor}(T_pP) = \ker\omega|_p$.
	}
	
	\begin{property}
		Consider two principal $G$-bundles $\xi_1$ and $\xi_2$. Let $\omega$ be an Ehresmann connection on $\xi_1$ and let $F;\xi_1\rightarrow \xi_2$ be a bundle map covering a smooth map $f$. The map $F^*\omega$ defines an Ehresmann connection on $\xi_2$.
	\end{property}
	
	\begin{example}
		Consider a principal $G$-bundle. An Ehresmann connection on this bundle is given by the following map:
		\begin{equation}
			\omega = (T_e\iota_p)^{-1}\circ\pr_V
		\end{equation}
		where $\pr_V$ is the projection $TP\rightarrow\text{Vert}(TP)$ associated to the decomposition from definition \ref{manifolds:connection}.
	\end{example}
	
	\newdef{Horizontal and vertical forms}{\index{horizontal!form}\index{vertical!form}\label{forms:horizontal_form}
		Let $\omega$ be an Ehresmann connection on a principal bundle $\prin{G}{P}{M}$. Let $\theta\in\Omega^k(P)$ be a $k$-form. We define following notions:
		\begin{itemize}
			\item $\theta$ is said to be horizontal if
			\begin{equation}
				\theta(v_1, ..., v_k) = 0
			\end{equation}
			whenever at least 1 of the $v_i$ lies in $\text{Vert}(T_pP)$.
			\item $\theta$ is said to be vertical if
			\begin{equation}
				\theta(v_1, ..., v_k) = 0
			\end{equation}
			whenever at least 1 of the $v_i$ lies in $\text{Hor}(T_pP)$.
		\end{itemize}
		For functions $f\in\Omega^0(P)$ it is vacuously true that they are both vertical and horizontal.
	}
	
\subsection{Maurer-Cartan form}

	\newdef{Maurer-Cartan form}{\index{Maurer-Cartan form}\index{Cartan!(connection) form|see{Maurer-Cartan}}
		For every $g\in G$ we have that the tangent space $T_gG$ is isomorphic to $T_eG\cong\mathfrak{g}$. The isomorphism $T_gG\rightarrow\mathfrak{g}$ is given by the Maurer-Cartan form:
		\begin{equation}
			\boxed{\Omega := L_{g^{-1},\ast}}
		\end{equation}
	}
	
	\begin{definition}
		Consider a manifold $M = \{x\}$. When constructing a principal $G$-bundle over $M$ we see that the total space $P = \{x\}\times G$ can be identified with the structure group $G$. From the relations in property \ref{manifolds:connection_dimensions} we see that the horizontal spaces are null-spaces (which defines a smooth distribution and thus a connection according to \ref{manifolds:connection}) and that the vertical spaces are equal to the tangent spaces, i.e. $\text{Vert}(T_gG) = T_gG$ (where we already made the association $P\cong G$).
		
		The simplest way to define a connection form $\omega$ on this bundle would be the trivial projection $TP\rightarrow\text{Vert}(TP) = \mathbbm{1}_{TP}$. The image of this map would however be $T_gG$ and not $\mathfrak{g}$ as required. This can be solved by using the Maurer-Cartan form $\Omega:T_gG\rightarrow\mathfrak{g}$, i.e. we define $\omega(v) = \Omega(v)$.
	\end{definition}
	
	\begin{property}
		The Maurer-Cartan form is the unique Ehresmann connection on the bundle $G\hookrightarrow G\rightarrow \{x\}$.
	\end{property}
	
\subsection{Horizontal lifts and parallel transport}
	
	\begin{property}\index{horizontal!lift}
		Consider a principal $G$-bundle $G\hookrightarrow P\rightarrow M$ and a curve $\gamma:[0, 1]\rightarrow M$. Let $p_0\in \pi^{-1}(\gamma(0))$. There exists a unique curve $\widetilde{\gamma}_{p_0}:[0, 1]\rightarrow P$ satisfying the following conditions:
		\begin{itemize}
			\item $\widetilde{\gamma}_{p_0}(0) = p_0$
			\item $\pi\circ\widetilde{\gamma}_{p_0} = \gamma$
			\item $\widetilde{\gamma}_{p_0}'(t)\in\text{Hor}(TP)$ for all $t\in[0, 1]$
		\end{itemize}
		The curve $\widetilde{\gamma}_{p_0}$ is said to be the horizontal lift of $\gamma$ starting at $p_0$. When it is clear from the context what the basepoint $p_0$ is, the subscript is often ommited and we write $\widetilde{\gamma}$ instead of $\widetilde{\gamma}_{p_0}$.
	\end{property}
	\begin{remark}[Horizontal curve]\index{horizontal!curve}
		Curves satisfying the last condition are said to be horizontal.
	\end{remark}
	
	\newdef{Parallel transport on principal bundles}{\index{parallel transport!on principal bundles}
		The parallel transport map is defined as follows:
		\begin{equation}
			\text{Par}_t^\gamma:\pi^{-1}(\gamma(0))\rightarrow\pi^{-1}(\gamma(t)):p_0\mapsto \widetilde{\gamma}_{p_0}(t)
		\end{equation}
		This map is $G$-equivariant and it is a diffeomorphism of fibres.
	}
	
	\begin{formula}
		Consider a principal bundle $G\hookrightarrow P\rightarrow M$. Let $\gamma(t)$ be a curve in $M$ and let $\omega$ be an Ehresmann connection on this bundle. The horizontal lift of $\gamma(t)$ can be parametrized as $(\gamma(t), g(t)$ where $g(t)$ is some unique curve in $G$. To determine $\widetilde{\gamma}(t)$ it is thus sufficient to find $g(t)$. The following uniquely (and globally) characterizes $g(t)$:
		\begin{equation}
			g'(t) = -\omega(\gamma(t), e, \gamma'(t), 0)g(t)
		\end{equation}
		Using the trivial section $s:U\rightarrow U\times G$ where $U$ is an open subset of $M$ we can rewrite this formula. First we consider the action of the pullback $s^*\omega$ on the derivative $T\gamma(t, 1) = (\gamma(t), \gamma'(t))$. Using the fact that it is linear (because it is a 1-form) we can write\[s^*\omega(\gamma(t), \gamma'(t)) = A(\gamma(t))\gamma'(t)\]where $A:M\rightarrow\text{Hom}(\mathbb{R}^m, \mathfrak{g})$ gives a linear map for each point $\gamma(t)\in M$. The action can also be rewritten using the relation $f^*\omega = \omega\circ f_\ast$ as\[s^*\omega(\gamma(t), \gamma'(t)) = \omega[s_\ast(\gamma(t), \gamma'(t)))] = \omega(\gamma(t), e, \gamma'(t), 0)\]
		Combining these relations with the ODE for $g(t)$ gives
		\begin{equation}
			\left(\deriv{}{t} + A(\gamma(t))\gamma'(t)\right)g(t) = 0
		\end{equation}
		where $\deriv{}{t}$ is a matrix given by the scalar multiplication of the derivative $\deriv{}{t}$ and the unit matrix $I$.
	\end{formula}
	
\subsection{Koszul connections and covariant derivatives}

	\newdef{Parallel transport on vector bundles}{
		Consider a principal bundle $G\hookrightarrow P\rightarrow M$ where we explicitly require $P$ to be trivial, i.e. $P = M\times G$. Suppose that the Lie group $G$ acts on a vector space $V$ by a representation $\rho:G\rightarrow GL_m$ . We can then construct an associated vector bundle $\pi_1:M\times V\rightarrow M$.
		
		Parallel transport on this vector bundle is then defined as follows. Let $\gamma(t)$ be a curve in $M$ such that $\gamma(0)=x_0$ and $x_1 = \gamma(1)$. Furthemore, let the horizontal lift $\widetilde{\gamma}(t)$ have $\widetilde{\gamma}(0)=(x_0, h)$ as initial condition. The parallel transport of the point $(x_0, v_0)\in M\times V$ along $\gamma$ is given by the following map:
		\begin{equation}
			\text{Par}^\gamma_t:\pi^{-1}_1(x_0)\rightarrow\pi^{-1}_1(\gamma(t)):(x_0, v_0)\mapsto \big(\gamma(t), \rho\big(g(t)h^{-1}\big)v_0\big)
		\end{equation}
		It should be noted that this map is independent of the initial element $h\in G$. Furthermore, $\text{Par}^\gamma_t$ is an isomorphism of vector spaces and can thus be used to identify distant fibers (as long as they lie in the same path-component).
	}
	\begin{remark}
		Two remarks have to be made. First of all, although the previous construction explicitly used trivial bundles, it is also valid for general non-trivial vector bundles. Secondly, following remark \ref{manifolds:vector_principal_correspondence} we can construct a principal bundle for any vector bundle and use the parallel transport on this bundle to define parallel transport of vectors. The previous construction is thus possible for every vector bundle.
	\end{remark}
	
	\newdef{Covariant derivative}{\index{covariant!derivative}
		Consider a vector bundle with model fibre space $V$ and its associated principal $GL(V)$-bundle with Ehresmann connection $\omega$, both over a base manifold $M$. Let $\sigma:M\rightarrow E$ be a section of the vector bundle and let $X$ be a vector field on $M$. The covariant derivative of $\sigma$ with respect to $X$ is defined as:
		\begin{equation}
			\nabla_X\sigma(x_0) = \lim_{t\rightarrow+\infty}\stylefrac{(\text{Par}_t^\gamma)^{-1}\sigma(\gamma(t)) - \sigma(x_0)}{t}
		\end{equation}
		where $\gamma(t)$ is any curve such that $\gamma(0) = x_0$ and $\gamma'(0) = X(x_0)$.
	}
	\begin{property}
		Let $\pi:E\rightarrow M$ be a vector bundle. Let $\sigma, X$ and $f$ be respectively a section on $E$, a vector field on $M$ and a $C^\infty$ function on $M$. The covariant derivative along $X$ satisfies following properties:
		\begin{itemize}
			\item $\nabla_X\sigma$ is a smooth section on $E$.
			\item The map $(X, \sigma)\mapsto\nabla_X\sigma$ is bilinear over $\mathbb{R}$.
			\item $\nabla_{(fX)}\sigma = f\nabla_X\sigma$
			\item $\nabla_X(f\sigma) = f\nabla_X\sigma + X(f)\sigma$
		\end{itemize}
	\end{property}
	\begin{remark}
		The last two properties show the major difference between the Lie derivative and the covariant derivative when $\sigma$ is a section of the tangent bundle, i.e. a vector field. Lie derivatives depend on the local behaviour of both $X$ and $\sigma$. The covariant derivative on the other hand only depends on the value of $X$ at $p\in M$ and on the local behaviour of $\sigma$.
	\end{remark}
	
	\newdef{Koszul connection}{\index{Koszul!connection}
		The map
		\begin{equation}
			\Gamma(TM)\times\Gamma(E)\rightarrow\Gamma(E):(X, \sigma)\mapsto\nabla_X\sigma
		\end{equation}
		is called a Koszul connection if the above properties hold. From the above constructions it also follows that every Ehresmann connection on a principal bundle induces a Koszul connection on all of its associated vector bundles.
	}
	
	\newdef{Exterior covariant derivative}{\index{exterior!covariant derivative}
		Consider a principal bundle $\prin{G}{P}{M}$ equipped with an Ehresmann connection $\omega$. Let $\theta\in \Omega^k(P)$ be a differential $k$-form. The exterior covariant derivative $D\theta$ is defined as follows:
		\begin{equation}
			D\theta(v_0, ..., v_k) = d\theta(v_0^H, ..., v_k^H)
		\end{equation}
		where $d$ is the exterior derivative \ref{forms:def:exterior_derivative} and $v_i^H$ is the projection of $v_i$ on the horizontal subspace $\text{Hor}(T_pP)$ associated to the Ehresmann connection $\omega$. From the definition it follows that the covariant derivative $D\theta$ is a horizontal form \ref{forms:horizontal_form}.
	}
	\begin{remark}
		The exterior covariant derivative can also be defined for general $W$-valued $k$-forms where $W$ is a vector space. This can be done by defining it component-wise with respect to a given basis on $W$. Afterwards one can prove that the choice of basis plays no role.
	\end{remark}
	
	\begin{formula}
		Using the Koszul connection on the tangent bundle $TP$ we can rewrite the action of the exterior covariant derivative as follows:
		\begin{equation}
			D\theta(v_0, ..., v_k) = \sum_i^k(-1)^i\nabla_{v_i}\theta(v_0, ..., \hat{v}_i, ..., v_k) + \sum_{i<j}^k(-1)^{i+j}\theta([v_i, v_j], v_0, ..., \hat{v}_i, ..., \hat{v}_j, ..., v_k)
		\end{equation}
		where $\hat{v}_i$ means that this vector is omitted. As an example we explicitly give the formula for a 1-form $\Phi$:
		\begin{equation}
			D\Phi(X, Y) = \nabla_X(\Phi(Y)) - \nabla_Y(\Phi(X)) - \Phi([X, Y])
		\end{equation}
		which should remind the reader of the analogous formula for the ordinary exterior derivative \ref{forms:1form_exterior_derivative}.
	\end{formula}
	
\subsection{Curvature of a connection}

	\newdef{Curvature}{\index{curvature}
		Let $\omega$ be an Ehresmann connection on a principal bundle $\prin{G}{P}{M}$. The curvature of $\omega$ is defined as the exterior covariant derivative $D\omega$.
	}
	\newdef{Flat connection}{
		An Ehresmann connection $\omega$ is said to be flat if its curvature $D\omega$ vanishes everywhere.
	}
	
	The following property is an immediate consequence of the Frobenius integrability theorem \ref{manifolds:frobenius} and the fact that an Ehresmann connection vanishes on the horizontal subbundle.
	\begin{property}\index{integrable}
		Let $\omega$ be an Ehresmann connection. The associated horizontal distribution\footnote{See \ref{manifolds:distribution} for the definition of a distribution of vector spaces.}\[p\mapsto\text{Hor}(T_pP)\]is integrable if and only if the connection $\omega$ is flat. The vertical distribution is always integrable.
	\end{property}

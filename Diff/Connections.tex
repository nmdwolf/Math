\section{Connections}
\subsection{Vertical vectors}
	
	Because smooth fibre bundles (which include smooth principal $G$-bundles) are also smooth manifolds we can define the traditional notions for them, such as the tangent bundle. We use these to construct the horizontal and vertical (sub)bundles:
	\newdef{Vertical vector}{\index{vertical!vector}
		Let $\pi:E\rightarrow B$ be a smooth fibre bundle. The subbundle $\ker(\pi_\ast)$ of $TE$ is called the vertical bundle of $E$. Fibrewise this gives us $V_x = T_x(E_{\pi(x)})$.
	}

	For principal $G$-bundles we can use an equivalent definition:
	\begin{adefinition}
		Consider a smooth principal $G$-bundle $G\hookrightarrow P\xrightarrow{\pi} M$. We first construct a map $\iota_p$ for every element $p\in P$:
		\begin{gather}
			\iota_p:G\rightarrow P: g\mapsto p\cdot g
		\end{gather}
		We then define a tangent vector $v\in T_p P$ to be vertical if it lies in the image of $\iota_{p,\ast}$, i.e. $\text{Vert}(T_pP) = \text{im}(\iota_{p,\ast})$. This construction is supported by the exactness of following short sequence:
		\begin{gather}
			0\xrightarrow{} \mathfrak{g} \xrightarrow{\iota_{p,\ast}} T_p P\xrightarrow{\pi_\ast} T_xM \xrightarrow{} 0
		\end{gather}
	\end{adefinition}
	
	\begin{property}[Dimension]
		It follows from the second definition that the vertical vectors of a principal $G$-bundle are nothing but the pushforward of the Lie algebra $\mathfrak{g}$ under the right action of $G$ on $P$. Furthermore, the exactness of the sequence implies that $\iota_{p,\ast}:\mathfrak{g}\rightarrow\text{Vert}(T_pP)$ is an isomorphism of vector spaces. In particular, it implies that
		\begin{gather}
			\label{manifolds:vertical_dimension}
			\dim\text{Vert}(T_pP) = \dim\mathfrak{g} = \dim G
		\end{gather}
	\end{property}
	
	\newdef{Fundamental vector field}{
		Consider a principal $G$-bundle. Let $A\in\mathfrak{g}$, where $\mathfrak{g}$ is the Lie algebra corresponding to $G$. The vertical vector field $A^\#:P\rightarrow TP$ given by
		\begin{gather}
			\label{manifolds:fundamental_vector_field}
			A^\#(p) = \iota_{p,\ast}(A)\in\text{Vert}(T_pP)
		\end{gather}
		is called the fundamental vector field associated to $A$.
	}
	\begin{adefinition}
		An equivalent definition of the fundamental vector field $A^\#(p)$ is given by:
		\begin{gather}
			A^\#_p(f) = \left.\deriv{}{t}f(p\cdot\exp(tA))\right|_{t=0}
		\end{gather}
		where $f\in C^\infty(P)$.
	\end{adefinition}
	
	\begin{property}
		The map $(\cdot)^\#:\mathfrak{g}\rightarrow\Gamma(TP)$ is a Lie algebra morphism:
		\begin{gather}
			[A, B]^\# = [A^\#, B^\#]
		\end{gather}
		where the Lie bracket on the left is that in $\mathfrak{g}$ and the Lie bracket on the right is that in $\mathfrak{X}(M)$ given by \ref{manifolds:lie_bracket}.
	\end{property}
	
	\begin{property}
		The vertical bundle satisfies the following $G$-equivariance condition:
		\begin{gather}
			\label{diff:vert_g_equivariance}
			R_{g, \ast}(\text{Vert}(T_pP)) = \text{Vert}(T_{pg}P)
		\end{gather}
		
		By differentiating the equality \[R_g\circ\iota_p = \iota_{pg}\circ\text{ad}_{g^{-1}}\] and using \ref{lie:adjoint_representation}, \ref{manifolds:fundamental_vector_field} we obtain the following algebraic formulation of the $G$-equivariance condition:
		\begin{gather}
			R_{g, \ast}\left(A^\#(p)\right) = \left(\text{Ad}_{g^{-1}}A\right)^\#(pg)
		\end{gather}
	\end{property}
	
\subsection{Ehresmann connections}

	\newdef{Ehresmann connection}{\index{connection!Ehresmann}\index{horizontal!vector}\label{manifolds:connection}
		Consider a smooth fibre bundle $P$. An (Ehresmann) connection on $P$ is the selection of a subspace $\text{Hor}(T_pP)\leq T_pP$ for every $p\in P$ such that:
		\begin{itemize}
			\item $\text{Vert}(T_pP)\oplus\text{Hor}(T_pP) = T_pP$
			\item The selection depends smoothly on $p$.\footnote{See the definiton of a (smooth) distribution \ref{manifolds:distribution}.}
		\end{itemize}
		The elements of $\text{Hor}(T_pP)$ are said to be \textbf{horizontal vectors} with respect to the connection.
	}
	
	\newdef{Principal connection}{\index{connection!principal}\label{manifolds:principal_connection}
		A principal connection on a smooth principal $G$-bundle $P$ is a $G$-equivariant Ehresmann connection, i.e. an Ehresmann connection for which the horizontal subspaces satisfy following $G$-equivariance condition:
		\begin{gather}
			R_{g, \ast}(\text{Hor}(T_pP)) = \text{Hor}(T_{pg}P)
		\end{gather}
		where $R_g$ denotes the right action of $G$ on $P$
	}
	\begin{remark}
		Note that this condition is automatically satisfied for vertical bundles (see equation \ref{diff:vert_g_equivariance}).
	\end{remark}
	
	\newdef{Horizontal bundle}{
		The horizontal (sub)bundle $\text{Hor}(TP)$ is defined as $\bigsqcup_{p\in P}\text{Hor}(T_pP)$. The $G$-equivariance condition then implies that this subbundle is invariant under (the pushforward of) the right action of $G$.
	}
	
	\begin{property}[Dimension]\label{manifolds:connection_dimensions}
		Properties \ref{manifolds:principal_bundle_dimension}, \ref{manifolds:vertical_dimension} and the direct sum decomposition of $T_pP$ imply the following relation:
		\begin{gather}
			\dim\text{Hor}(T_pP) = \dim M
		\end{gather}
		Here we briefly summarize all dimensional relations between the components of a principal $G$-bundle over a base manifold $M$:
		\begin{empheq}[box=\widefbox]{align}
			\dim P &= \dim M + \dim G\\
			\dim M &= \dim\text{Hor}(T_pP)\\
			\dim G &= \dim\text{Vert}(T_pP)
		\end{empheq}
		for all $p\in P$.
	\end{property}
	
	\newdef{Horizontal and vertical forms}{\index{horizontal!form}\index{vertical!form}\label{forms:horizontal_form}
		Let $\theta\in\Omega^k(P)$ be a differential $k$-form. We define following notions:
		\begin{itemize}
			\item $\theta$ is said to be horizontal if
			\begin{gather}
				\theta(v_1, ..., v_k) = 0
			\end{gather}
			whenever at least 1 of the $v_i$ lies in $\text{Vert}(T_pP)$.
			\item $\theta$ is said to be vertical if
			\begin{gather}
				\theta(v_1, ..., v_k) = 0
			\end{gather}
			whenever at least 1 of the $v_i$ lies in $\text{Hor}(T_pP)$.
		\end{itemize}
		For functions $f\in\Omega^0(P)$ it is vacuously true that they are both vertical and horizontal.
	}
	\newdef{Tensorial form}{\index{tensorial}\label{diff:tensorial_form}
		Consider a differential form on a principal $G$-bundle $P$ with values in a vector space $V$ equipped with a representation $\rho:G\rightarrow V$. This form is said to be \textbf{tensorial of type $(V, \rho)$} if it is horizontal and if it satisfies the equivariancy condition
		\begin{gather}
			R_g^*\theta = \rho(g^{-1})\theta
		\end{gather}
	}

	\newdef{Dual connection}{\index{dual!connection}
		First we define the dual of the horizontal bundle:
		\begin{gather}
			\text{Hor}(T_p^*P) = \{h^*\in T_p^*P|h^*(v)=0, v\in\text{Vert}(T_pP)\}
		\end{gather}
		
		It is the set horizontal 1-forms. A dual connection can then be defined as the selection of a vertical covector bundle $\text{Vert}(T_p^*P)$ satisfying the conditions of definition \ref{manifolds:connection} and \ref{manifolds:principal_connection} (where $\text{Vert}$ and $\text{Hor}$ should be interchanged).
	}
	
\subsection{Connection form}

	\newdef{Connection one-form}{\index{connection!form}
		Let $\prb$ be a principal bundle. A connection one-form, related to a given principal connection, is a $\mathfrak{g}$-valued 1-form $\omega:\Gamma(TP)\rightarrow\mathfrak{g}$ that satisfies the following 2 conditions:
		\begin{enumerate}
			\item Cancellation of fundamental vector fields:
			\begin{gather}
				\omega(A^\#) = A
			\end{gather}
			\item $G$-equivariance:
			\begin{gather}
				\omega\circ R_{g, \ast} = \text{Ad}_{g^{-1}}\circ\omega
			\end{gather}
		\end{enumerate}
		The horizontal subspaces are then defined as $\text{Hor}(T_pP) = \ker\omega|_p$.
	}	
	\begin{formula}
		Consider a principal $G$-bundle $P$. Given a principal connection on $P$, the associated connection one-form is given by the following map:
		\begin{gather}
			\omega = (\iota_{p,\ast})^{-1}\circ\pr_V
		\end{gather}
		where $\pr_V$ is the projection $TP\rightarrow\text{Vert}(TP)$ associated to the decomposition from definition \ref{manifolds:connection}.
	\end{formula}
	
	\begin{formula}
		Consider a principal bundle $\prb$ and an associated vector bundle $P\times_G V$. For every $G$-equivariant map $\phi:P\rightarrow V$ and any $X\in\mathfrak{g}$ we find that
		\begin{gather}
			d\phi(X^\#) + [\omega\wedge\phi](X^\#) = 0
		\end{gather}
		where the left action of $\mathfrak{g}$ is induced by the representation of $G$ on $V$.
	\end{formula}
	
	\begin{property}
		Consider two principal $G$-bundles $\xi_1$ and $\xi_2$. Let $\omega$ be a connection one-form on $\xi_1$ and let $F:\xi_1\rightarrow \xi_2$ be a bundle map. The map $F^*\omega$ defines an Ehresmann connection on $\xi_2$.
	\end{property}
	
	\newdef{Reducible connection}{
		Consider a principal $G$-bundle $P$ equipped with a connection one-form $\omega$. If the bundle map $\theta$ induces an $H$-reduction of $P$ then the connection $\omega$ is said to be reducible if $\theta^*\omega$ takes values in $\mathfrak{h}$.
	}
	
\subsection{Maurer-Cartan form}

	\newdef{Maurer-Cartan form}{\index{Maurer-Cartan form}\index{Cartan!(connection) form|see{Maurer-Cartan}}
		For every $g\in G$ we have that the tangent space $T_gG$ is isomorphic to $T_eG = \mathfrak{g}$. The isomorphism $T_gG\rightarrow\mathfrak{g}$ is given by the Maurer-Cartan form:
		\begin{gather}
			\boxed{\Omega := L_{g^{-1},\ast}}
		\end{gather}
	}
	
	\begin{construct}
		Consider a manifold $M = \{x\}$. When constructing a principal $G$-bundle over $M$ we see that the total space $P = \{x\}\times G$ can be identified with the structure group $G$. From the relations in property \ref{manifolds:connection_dimensions} it follows that the horizontal spaces are null-spaces (which indeed defines a smooth distribution and thus a connection according to \ref{manifolds:connection}) and that the vertical spaces are equal to the tangent spaces, i.e. $\text{Vert}(T_gG) = T_gG$ (where we used the association $P\cong G$).
		
		The simplest way to define a connection form $\omega$ on this bundle would be the trivial projection $TP\rightarrow\text{Vert}(TP) = \mathbbm{1}_{TP}$. The image of this map would however be $T_gG$ and not $\mathfrak{g}$ as required. This can be solved by using the Maurer-Cartan form $\Omega:T_gG\rightarrow\mathfrak{g}$, i.e. we define $\omega(v) = \Omega(v)$.
	\end{construct}
	
	\begin{property}
		The Maurer-Cartan form is the unique Principal connection on the bundle $G\hookrightarrow G\rightarrow \{x\}$.
	\end{property}

\subsection{Local representation}

	\newdef{Yang-Mills field}{\index{Yang-Mills!field}\label{diff:prin:yang_mills_field}
		Consider a principal bundle $\prb$ and an open subset $U\subseteq M$. Given an Ehresmann connection $\omega$ on $P$ and a local section $\sigma:U\rightarrow P$, we define the Yang-Mills field $\omega^U:\Gamma(TU)\rightarrow\mathfrak{g}$ as follows:
		\begin{gather}
			\omega^U = \sigma^*\omega
		\end{gather}
	}
	
	\newdef{Local representation}{
		Consider a principal bundle $\prb$. Let $(U, \varphi)$ be a bundle chart on $P$. The local representation of an Ehresmann connection $\omega$ on $P$ with respect to the chart $(U, \varphi)$ is given by $(\varphi^{-1})^*\omega$.
	}
	
	\begin{formula}
		Consider an Ehresmann connection $\omega$ on a principal bundle $\prb$. According to property \ref{diff:prin_section_triv} every local section $\sigma:U\rightarrow P$ induces both a Yang-Mills field $\omega^U$ and a local representation of $\omega$. These two forms are related by the following equation:
		\begin{gather}
			h^*\omega_{(m, g)}(v, X) = \text{Ad}_{g^{-1}}(\omega^U_m(v)) + \Omega_g(X)
		\end{gather}
		where $v\in T_mU, X\in\mathfrak{g}$, $\Omega$ is the Maurer-Cartan form on $G$ and $h$ is the local trivialization induced by $\sigma$.
	\end{formula}
	
	\begin{formula}[Compatibility condition]
		Consider a principal bundle $\prb$ and 2 open subsets $U, V$ of $M$. Given 2 local sections $\sigma_U:U\rightarrow P$ and $\sigma_V:V\rightarrow P$ and an Ehresmann connection $\omega$ on $P$, we can define two Yang-Mills field $\omega^U$ and $\omega^V$ on $M$.
		
		On the intersection $U\cap V$ we can find a (unique) gauge transformation $\xi:U\cap V\rightarrow G$ such that $\sigma_V(m) = \sigma_U(m)\cdot\xi(m)$. Using this gauge transformation we can relate $\omega^U$ and $\omega^V$ as follows:
		\begin{gather}
			\label{diff:prin:local_compatibility}
			\omega^V_m = \text{Ad}_{\xi(m)^{-1}}\omega^U_m + (\xi^*\Omega)_m
		\end{gather}
		where $\Omega$ is the Maurer-Cartan form on $G$.
	\end{formula}
	
	\begin{example}[General linear group\footnotemark]
		\footnotetext{A derivation can be found in lecture 22 of \cite{schuller}.}
		Let $G=\text{GL}(\mathbb{R}^n)$. The second term in equation \ref{diff:prin:local_compatibility} can be written as follows:
		\begin{gather}
			(\xi^*\Omega)^i_{\ j} = (\xi(m)^{-1})^i_{\ k}\pderiv{}{x^\mu}\xi(p)^k_{\ j}dx^\mu
		\end{gather}
		at every point $m\in M$. Formally this can be written coordinate-independently as:
		\begin{gather}
			\label{diff:prin:mc_pullback}
			\xi^*\Omega = \xi^{-1}d\xi
		\end{gather}
	\end{example}
	
	\begin{example}[Christoffel symbols]\index{Christoffel!symbols}
		Let $\Gamma^i_{\ j\mu}, \overline{\Gamma}^k_{\ l\nu}$ be the Yang-Mills fields corresponding to a connection of a frame bundle, where the sections are induced by a choice of coordinates ($x^i$ and $y^i$ respectively). In this case, the expansion coefficients of the Yang-Mills field are called the \textbf{Christoffel symbols}\footnote{See also equation \ref{diff:christoffel_symbol}.}. Using equations \ref{diff:prin:local_compatibility} and \ref{diff:prin:mc_pullback} this becomes:
		\begin{gather}
			\overline{\Gamma}^i_{\ j\mu} = \pderiv{y^\nu}{x^\mu}\left(\pderiv{x^i}{y^k}\Gamma^k_{\ l\nu}\pderiv{y^l}{x^j} + \pderiv{x^i}{y^k}\frac{\partial^2y^k}{\partial x^j\partial x^\nu}\right)
		\end{gather}
	\end{example}

\subsection{Horizontal lifts and parallel transport}
	
	\begin{definition}[Horizontal lift]\index{horizontal!lift}
		Consider a principal bundle $\prb$ and a curve $\gamma:[0, 1]\rightarrow M$. For every point $p_0\in \pi^{-1}(\gamma(0))$ there exists a unique curve $\widetilde{\gamma}_{p_0}:[0, 1]\rightarrow P$ satisfying the following conditions:
		\begin{itemize}
			\item $\widetilde{\gamma}_{p_0}(0) = p_0$
			\item $\pi\circ\widetilde{\gamma}_{p_0} = \gamma$
			\item $\widetilde{\gamma}_{p_0}'(t)\in\text{Hor}(TP)$ for all $t\in[0, 1]$.
		\end{itemize}
		The curve $\widetilde{\gamma}_{p_0}$ is said to be the \textbf{horizontal lift} of $\gamma$ starting at $p_0$. When it is clear from the context what the basepoint $p_0$ is, the subscript is often ommited and we write $\widetilde{\gamma}$ instead of $\widetilde{\gamma}_{p_0}$.
	\end{definition}
	\begin{remark}[Horizontal curve]\index{horizontal!curve}
		Curves satisfying the last condition in the above property are said to be horizontal.
	\end{remark}
	
	\begin{method}
		Consider a principal bundle $\prb$. Let $\gamma(t)$ be a curve in $M$ and let $\omega$ be a principal connection on $P$. For general structure groups $G$, the horizontal lift can be found as follows:
		
		\qquad Let $\delta(t)$ be a curve in $P$ that projects onto $\gamma(t)$, i.e. $\pi\circ\delta=\gamma$, such that $\widetilde\gamma_{p_0}(t)=\delta(t)\cdot g(t)$ for some curve $g(t)$ in $G$. The curve $g(t)$ can then be found as the unique solution of the following first order ODE:
		\begin{gather}
			\label{diff:prin:horizontal_ode}
			\text{Ad}_{g(t)^{-1}}\omega_{\delta(t)}(X_{\delta, \delta(t)}) + \Omega_{g(t)}(Y_{g, g(t)}) = 0
		\end{gather}
		where $X_\delta, Y_g$ are tangent vectors to respectively the curves $\delta(t)$ and $g(t)$ and where $\Omega$ is the Maurer-Cartan form on $G$. As initial value condition we use $\delta(0)\cdot g(0) = p_0$.
	\end{method}
	\begin{remark}
		When given a local section $\sigma:U\rightarrow P$ we can rewrite the ODE in a more explicit form. First we remark that the section induces a curve $\delta = \sigma\circ\gamma$. Taking the derivative yields $X_\delta = \sigma_*(X_\gamma)$. Using this we can rewrite the ODE as
		\begin{gather}
			\text{Ad}_{g(t)^{-1}}\omega_{\delta(t)}(\sigma_*X_{\gamma, \gamma(t)}) + \Omega_{g(t)}(Y_{g, g(t)}) = 0
		\end{gather}
		By using the equality $f^*\omega = \omega\circ f_*$ and introducing the Yang-Mills field $A = \sigma^*\omega$ this becomes:
		\begin{gather}
			\text{Ad}_{g(t)^{-1}}A(X_{\gamma, \gamma(t)}) + \Omega_{g(t)}(Y_{g, g(t)}) = 0
		\end{gather}
	\end{remark}
	
	\begin{example}
		 For matrix Lie groups the above ODE can be reformulated as follows: Given the trivial section $s:U\rightarrow U\times G:x\mapsto (x, e)$, where $U$ is an open subset of $M$, the horizontal lift of $\gamma(t)$ can locally be parametrized as $\widetilde{\gamma}(t) = \underbrace{(s\circ\gamma)(t)}_{\delta(t)}\cdot g(t) = (\gamma(t), g(t))$ where $g(t)$ is a curve in $G$. To determine $\widetilde{\gamma}(t)$ it is thus sufficient to find $g(t)$. The ODE \ref{diff:prin:horizontal_ode} then becomes:
		\begin{gather}
			\label{diff:prin:horizontal_ode_matrix}
			g'(t) = -\omega(\gamma(t), e, \gamma'(t), 0)g(t)
		\end{gather}
		
		Using the trivial section we can rewrite this formula. First we consider the action of the Yang-Mills field $s^*\omega$ on the derivative $\gamma_* = (\gamma(t), \gamma'(t))$. Using the fact that it is linear in the second argument we can write: \[s^*\omega(\gamma(t), \gamma'(t)) = A(\gamma(t))\gamma'(t)\] where $A:M\rightarrow\text{Hom}(\mathbb{R}^{\dim M}, \mathfrak{g})$ gives a linear map for each point $\gamma(t)\in M$. The action can also be rewritten using the relation $f^*\omega = \omega\circ f_\ast$ as\[s^*\omega(\gamma(t), \gamma'(t)) = \omega\Big(s_\ast(\gamma(t), \gamma'(t))\Big) = \omega(\gamma(t), e, \gamma'(t), 0)\]
		Combining these relations with the ODE \ref{diff:prin:horizontal_ode_matrix} gives:
		\begin{gather}
			\label{diff:prin:horizontal_ode_derivative}
			\left(\deriv{}{t} + A(\gamma(t))\gamma'(t)\right)g(t) = 0
		\end{gather}
		where $\deriv{}{t}$ is the matrix given by the scalar multiplication of the derivative $\deriv{}{t}$ and the identity matrix $I$.
	\end{example}
	
	\begin{method}
		The ODE \ref{diff:prin:horizontal_ode} can now be solved. We explicitly assume that $G$ is a matrix Lie group such that we can start from equation \ref{diff:prin:horizontal_ode_derivative}. Direct intergation and iteration gives us:
		\begin{gather}
			g(t) = \left[I - \int_0^tdt_1A(\gamma'(t_1)) + \int_0^tdt_1\int_0^{t_1}dt_2A(\gamma'(t_1))A(\gamma'(t_2)) - ...\right]g(0)
		\end{gather}
		where $A$ is the Yang-Mills field corresponding to the local section $\sigma$. This can be rewritten using the standard "square integration" trick\footnote{Well known from the Dyson series \ref{QM:dyson_series}.} as:
		\begin{gather}
			g(t) = \left[I - \int_0^tdt_1A(\gamma'(t_1)) + \frac{1}{2!}\int_0^tdt_1\int_0^{\textcolor{red}{t}}dt_2\mathcal{T}\Big(A(\gamma'(t_1))A(\gamma'(t_2))\Big) - ...\right]g(0)
		\end{gather}
		By noting that this formula is equal to the path-ordered exponential series we find:
		\begin{gather}
			\boxed{g(t) = \mathcal{T}\exp\left(-\int_0^tdt'A(\gamma'(t'))\right)g(0)}
		\end{gather}
	\end{method}
	
	\newdef{Parallel transport}{\index{parallel transport!on principal bundles}
		\nomenclature[O_Par]{$\text{Par}_t^\gamma$}{Parallel transport map with respect to the curve $\gamma$.}
		The parallel transport map with respect to the curve $\gamma$ is defined as follows:
		\begin{gather}
			\text{Par}_t^\gamma:\pi^{-1}(\gamma(0))\rightarrow\pi^{-1}(\gamma(t)):p_0\mapsto \widetilde{\gamma}_{p_0}(t)
		\end{gather}
		This map is $G$-equivariant and it is an isomorphism of fibres.
	}
	
\subsection{Holonomy}
	
	\newdef{Holonomy group}{\index{holonomy}
		\nomenclature[S_Hol]{$\text{Hol}_p(\omega)$}{Holonomy group at $p$ with respect to the connection $\omega$.}
		Consider a principal bundle $\prin{G}{P}{M}$. Let $\Omega^{ps}_mM\subset\Omega_m M$ be the subset of the loop space consisting of piecewise smooth loops with basepoint $m\in M$. The holonomy group $\text{Hol}_p(\omega)$ based at $p\in\pi^{-1}(m)\subset P$ with respect to the connection form $\omega$ is given by
		\begin{gather}
			\text{Hol}_p(\omega) = \{g\in G: p \sim p\cdot g\}
		\end{gather}
		where two points $p, q\in P$ are equivalent if there exists a loop $\gamma\in\Omega_m^{ps}M$ such that the horizontal lift $\widetilde{\gamma}$ connects $p$ and $q$.
	}
	\newdef{Reduced holonomy group}{
		The reduced holonomy group $\text{Hol}_p^0(\omega)$ is defined as the subset of $\text{Hol}_p(\omega)$ consisting of contractible loops.
	}
	
\subsection{Koszul connections and covariant derivatives}

	\newdef{Horizontal lifts on associated bundles}{\index{horizontal!lift}
		Let $P_F := P\times_G F$ be an associated bundle of a principal bundle $\prb$ and let $\gamma$ be a curve in $M$ with horizontal lift $\widetilde{\gamma}_p$ in $P$. The horizontal lift of $\gamma$ to $P_F$ through the point $[p, f]\in P_F$ is defined as follows:
		\begin{gather}
			\widetilde{\gamma}^{P_F}_{[p, f]}(t) = [\widetilde{\gamma}_p(t), f].
		\end{gather}
		Although the element $f$ seems to stay constant along the horizontal lift, it in fact changes according to formula \ref{diff:prin:associated_bundle_equivalence}.
	}
	\newdef{Parallel transport on associated bundles}{
		Similar to the case of principal bundles $P$, the parallel transport map on an associated bundle $P_F$ is defined as
		\begin{gather}
			\text{Par}_t^\gamma:\pi_F^{-1}(\gamma(0))\rightarrow\pi_F^{-1}(\gamma(t)):[p, f]\mapsto \widetilde{\gamma}^{P_F}_{[p, f]}(t).
		\end{gather}
	}

	\begin{example}[Parallel transport on vector bundles]
		Consider a principal bundle $\prb$. Suppose that the Lie group $G$ acts on a vector space $V$ through a representation $\rho:G\rightarrow\text{GL}(V)$ . We can then construct an associated vector bundle $\pi_1:P\times_{GL(V)} V\rightarrow M$. Assume further that we work on a chart $(U, \varphi)$ such that we can locally write $P$ and $P_V$ as product bundles.
		
		Parallel transport on this vector bundle is then defined as follows. Let $\gamma(t)$ be a curve in $M$ such that $\gamma(0)=x_0$ and $\gamma(1) = x_1$. Furthermore, let the horizontal lift $\widetilde{\gamma}(t) = (\gamma(t), g(t))$ satisfy $\widetilde{\gamma}(0)=(x_0, h)$ as initial condition. Parallel transport of the point $(x_0, v_0)\in U\times V$ along $\gamma$ is given by the following map:
		\begin{gather}
			\text{Par}^\gamma_t:\pi^{-1}_1(x_0)\rightarrow\pi^{-1}_1(\gamma(t)):(x_0, v_0)\mapsto \big(\gamma(t), \rho\big(g(t)h^{-1}\big)v_0\big)
		\end{gather}
		It should be noted that this map is independent of the initial element $h\in G$ (despite the presence of the factor $h^{-1}$). Furthermore, $\text{Par}^\gamma_t$ is an isomorphism of vector spaces and can thus be used to identify distant fibers (as long as they lie in the same path-component).
	\end{example}
	\begin{remark}
		For every vector bundle one can construct the frame bundle and use the parallel transport map on this bundle to define parallel transport of vectors. Hence the previous construction is applicable to any vector bundle.
	\end{remark}
	
	\newdef{Covariant derivative}{\index{covariant!derivative}
		Consider a vector bundle with model fibre space $V$ and its associated principal GL$(V)$-bundle with principal connection $\omega$ (both over a base manifold $M$). Let $\sigma:M\rightarrow E$ be a section of the vector bundle and let $X$ be a vector field on $M$. The covariant derivative of $\sigma$ with respect to $X$ is defined as
		\begin{gather}
			\nabla_X\sigma|_{x_0} = \lim_{t\rightarrow0}\stylefrac{(\text{Par}_t^\gamma)^{-1}\sigma(\gamma(t)) - \sigma(x_0)}{t}
		\end{gather}
		where $\gamma(t)$ is any curve satisfying $\gamma(0) = x_0$ and $\gamma'(0) = X(x_0)$.
	}
	
	\begin{property}\index{Koszul!connection}
		The map
		\begin{gather}
			\Gamma(TM)\times\Gamma(E)\rightarrow\Gamma(E):(X, \sigma)\mapsto\nabla_X\sigma
		\end{gather}
		gives a Koszul connection \ref{manifolds:koszul_connection}. It follows that every principal connection on a principal bundle induces a Koszul connection on all of its associated vector bundles.
	\end{property}
	
\subsection{Exterior covariant derivative}

	\newdef{Exterior covariant derivative}{\index{exterior!covariant derivative}
		Consider a principal bundle $\prin{G}{P}{M}$ equipped with a principal connection $\omega$ and let $\theta\in \Omega^k(P)$ be a differential $k$-form. The exterior covariant derivative $D\theta$ is defined as follows:
		\begin{gather}
			D\theta(v_0, ..., v_k) = d\theta(v_0^H, ..., v_k^H)
		\end{gather}
		where $d$ is the exterior derivative \ref{forms:def:exterior_derivative} and $v_i^H=\pi_*v_i$ is the projection of $v_i$ on the horizontal subspace $\text{Hor}(T_pP)$. From the definition it follows that the exterior covariant derivative $D\theta$ is a horizontal form\footnote{See definition \ref{forms:horizontal_form}.}.
	}
	\begin{remark}
		The exterior covariant derivative can also be defined for general $W$-valued $k$-forms (where $W$ is a vector space). This can be done by defining it component-wise with respect to a given basis on $W$. Afterwards one can prove that the choice of basis plays no role.
	\end{remark}
	
	\begin{property}
		If $\Phi$ is an equivariant form then $D\Phi$ is a tensorial form.
	\end{property}
	
	\begin{formula}
		Using the Koszul connection on the tangent bundle $TP$ we can rewrite the action of the exterior covariant derivative as follows:
		\begin{gather}
			D\theta(v_0, ..., v_k) = \sum_i^k(-1)^i\nabla_{v_i}\theta(v_0, ..., \hat{v}_i, ..., v_k) + \sum_{i<j}^k(-1)^{i+j}\theta([v_i, v_j], v_0, ..., \hat{v}_i, ..., \hat{v}_j, ..., v_k)
		\end{gather}
		where as usual  $\hat{v}_i$ means that this vector is omitted. As an example we explicitly give the formula for a 1-form $\Phi$:
		\begin{gather}
			D\Phi(X, Y) = \nabla_X(\Phi(Y)) - \nabla_Y(\Phi(X)) - \Phi([X, Y])
		\end{gather}
		which should remind the reader of the analogous formula for the ordinary exterior derivative \ref{forms:k_form_exterior_derivative}.
	\end{formula}
	
	By property \ref{diff:prin:section_bijection} we can use the following construction to find an explicit expression for the covariant derivative on an associated vector bundle:
	\begin{construct}\index{covariant!derivative}
		Let $\prb$ be a principal bundle and let $P_V := P\times_G V$ be an associated vector bundle. Given a section $\sigma:M\rightarrow P_V$ we can construct a $G$-equivariant map $\phi:P\rightarrow V$ using formula \ref{diff:prin:section_bijection_phi}.
		
		The exterior covariant derivative of $\phi$ is given by
		\begin{gather}
			D\phi(X) = d\phi(X) + \omega\barwedge\phi(X)
		\end{gather}
		where $X\in T_pP$. Now, given an additional (local) section $\varphi:U\subseteq M\rightarrow P$, we can pull back this derivative to the base manifold $M$. This gives
		\begin{gather}
			(\varphi^*D\phi)(Y) = d(\varphi^*\phi)(Y) + \varphi^*\omega\barwedge\varphi^*\phi(Y)
		\end{gather}
		where $Y=\pi_*X\in T_mM$. After introducing the notations $S:=\varphi^*\phi$ and $\nabla_YS:=(\varphi^*D\phi)(Y)$ and remembering the definition of the Yang-Mills field \ref{diff:prin:yang_mills_field} this becomes
		\begin{gather}
			\label{diff:prin:local_covariant_derivative}
			\nabla_YS = dS(Y) + \omega^U(Y)\triangleright S.
		\end{gather}
	\end{construct}
	\begin{example}
		Let $G=\text{GL}(\mathbb{R}^n)$. In local coordinates equation \ref{diff:prin:local_covariant_derivative} can be rewritten as follows:
		\begin{gather}
			(\nabla_YS)^i = \pderiv{S^i}{x^k}Y^k + \Gamma^i_{\ jk}S^jY^k.
		\end{gather}
		This is exactly the formula known from classic differential geometry and relativity. 
	\end{example}

\subsection{Curvature}

	\newdef{Curvature}{\index{curvature}
		Let $\omega$ be a principal connection on a principal bundle $\prin{G}{P}{M}$. The curvature $\Omega$ of $\omega$ is defined as the exterior covariant derivative $D\omega$.
	}
	\newdef{Flat connection}{\index{connection!flat}
		A principal connection $\omega$ is said to be flat if its curvature $\Omega$ vanishes everywhere.
	}
	\begin{example}
		Let $\omega_G$ be the Maurer-Cartan form on a Lie group $G$. Because the only horizontal vector field on the bundle $\prin{G}{G}{\{x\}}$ is the zero vector, the curvature of $\omega_G$ is 0. Hence the Maurer-Cartan form is a flat connection.
	\end{example}
	
	\begin{property}[Second Bianchi identity]\index{Bianchi identity}
		Let $\omega$ be a principal connection with curvature $\Omega$. The curvature is covariantly constant:
		\begin{gather}
			D\Omega = 0.
		\end{gather}
	\end{property}
	\begin{remark}
		One should however pay attention to the fact that this result\mnote{\dbend} does not generalize to arbitrary differential forms. Only the exterior derivative satisfies the coboundary condition $d^2 \equiv 0$, the exterior covariant derivative does not.
	\end{remark}
	
	\newformula{Cartan structure equation}{\index{Cartan!structure equation}
		Let $\omega$ be an Ehresmann connection and let $\Omega$ be its curvature form. The curvature can be expressed in terms of the connection as follows:
		\begin{gather}
			\Omega = d\omega + \frac{1}{2}[\omega\wedge\omega].
		\end{gather}
	}
	
	The following property is an immediate consequence of the Frobenius integrability theorem \ref{manifolds:frobenius} and the fact that an connection vanishes on the horizontal subbundle:
	\begin{property}\index{integrable}
		Let $\omega$ be a principal connection. The associated horizontal distribution\footnote{See \ref{manifolds:distribution} for the definition of a distribution of vector spaces.}\[p\mapsto\text{Hor}(T_pP)\]is integrable if and only if the connection $\omega$ is flat. Furthermore, the vertical distribution is always integrable.
	\end{property}
	
	Similar to definition \ref{diff:prin:yang_mills_field} we can also define the Yang-Mills field strength:
	\newdef{Yang-Mills field strength}{\index{Yang-Mills!field strength}
		Let $\prb$ be a principal bundle equipped with an Ehresmann connection $\omega$. Given a local section $\sigma:U\subseteq M\rightarrow P$ we define the Yang-Mills field strength $F$ as the pullback $\sigma^*\Omega$, where $\Omega=D\omega$ is the curvature of $\omega$.
	}
	
\subsection{Torsion}

	\newdef{Solder form}{\index{solder form}
		Let $\prb$ be a principal bundle. Let $V$ be a $\dim M$-dimensional vector space equipped with a representation\footnote{In general this will be $V=\mathbb{R}^{\dim M}$ and $G=\text{GL}(n, \mathbb{R})$.} $\rho:G\rightarrow\text{GL}(V)$ such that $TM\cong P\times_G V$ in as associated bundles. A solder(ing) form $\theta$ on $P$ is a tensorial\footnote{See definition \ref{diff:tensorial_form}.} one-form of type $(V, \rho)$.
	}
	
	\newdef{Torsion}{\index{torsion}
		Let $\prb$ be a principal bundle equipped with an Ehresmann connection $\omega$ and a solder form $\theta$. The torsion $\Theta$ is defined as the exterior covariant derivative $D\theta$.
	}
	
	\begin{formula}[Cartan structure equation]\index{Cartan!structure equation}
		Let $\omega$ be an Ehresmann connection, $\theta$ a solder form and $\Theta$ its torsion form.
		\begin{gather}
			\Theta = d\theta + \omega\barwedge\theta
		\end{gather}
		where the wedge product is defined analogously\footnote{For forms with $\deg\geq1$ we sum over all permutations of the arguments.} to \ref{forms:vector_valued_wedge} and \ref{forms:lie_algebra_valued_wedge} using the representation of $\mathfrak{g}$ on $V$ induced by the representation $\rho:G\rightarrow\text{GL}(V)$:
		\begin{gather}
			\omega\barwedge\theta(v, w) = \omega(v)\triangleright\theta(w) - \omega(w)\triangleright\theta(v).
		\end{gather}
	\end{formula}
	
	\begin{property}[First Bianchi identity]\index{Bianchi identity}
		Let $\omega$ be an Ehresmann connection, $\Omega$ its curvature, $\theta$ a solder form and $\Theta$ its torsion.
		\begin{gather}
			D\Theta = \Omega\barwedge\theta.
		\end{gather}
	\end{property}
	
	\begin{property}
		Consider a smooth manifold $M$ equipped with a $G$-structure. If this structure is integrable then it admits a torsion-free connection.
	\end{property}

\section{Characteristic classes}

	\newdef{Characteristic class}{\index{characteristic!class}
		Let $M$ be a smooth manifold. A characteristic class is a map\footnote{The coefficient ring is often assumed to be the base field ($\mathbb{R}$ or $\mathbb{C}$) but this is not always the case (e.g. Stiefel-Whitney classes).} $c: \text{Vect}(M)\rightarrow H^\ast(M)$ such that if $E, E'\in\text{Vect}(M)$ are equivalent, then $c(E) = c(E')$.
	}

\subsection{Chern-Weil theory}

	The characteristic classes of a vector bundle can be constructed from the connection and curvature forms on the vector bundle. This is done using a class of polynomials in the Lie algebra $\mathfrak{g}$ of the structure group.

	\newdef{Chern-Weil morphism}{\index{Chern-Weil}
		Let $E\rightarrow M$ be a vector bundle with connection $A$ and curvature $F$. There exists a morphism of algebras $K[\mathfrak{g}]^G\rightarrow H^*_{dR}(M):P\mapsto P(F)$ satisfying:
		\begin{itemize}
			\item $P(F)$ is closed.
			\item $P(F)$ pulls back uniquely to a (closed) form $\overline{P}(F) := \pi^\ast P(F)$ on $M$.
			\item $\overline{P}(F)$ does not depend on the connection $A$, i.e. for connections $A, A'$ the difference $\overline{P}(F_A) - \overline{P}(F_{A'})$ is exact.
		\end{itemize}
	}

\subsection{Chern classes}

	\newdef{Canonical class}{\index{canonical!class}
		Consider a smooth manifold $M$. The first Chern class of the canonical bundle $\bigwedge^nT^*M$ is called the canonical class of $M$.
	}
	\newdef{Theta characteristic}{
		Consider a smooth manifold $M$ together with its canonical class $K_M$. The theta characteristic, if it exists, is a charactertic class $\Theta$ such that $\Theta\cup\Theta=K_M$ where $\cup$ is the cup-product in cohomology.
	}

\subsection{Cohomology of Lie groups}\index{cohomology!Lie group}

	Using the language of characteristic classes we can find a concise description of the (continuous) group cohomology of Lie groups. First of all we use the isomorphism between continuous group cohomology and cohomology of classifying spaces:
	\begin{gather}
		H^*(BG; \mathbb{Z})\cong H^*_c(G; \mathbb{Z})
	\end{gather}

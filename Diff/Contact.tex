\chapter{Contact Geometry}\label{chapter:contact}

\section{Contact structures}
\subsection{Contact form}

    \newdef{Contact element}{\index{contact!element}
        Let $M$ be a smooth $n$-dimensional manifold. A contact element at the point $p\in M$, called the \textbf{contact point}, is a $(n-1)$-dimensional subspace of the tangent space $T_pM$.
    }
    \begin{property}
        Because every $(n-1)$-dimensional subspace of the tangent space can be constructed as the kernel of a linear functional (an element of $T^*_pM$), one can construct the space of contact elements as a quotient of the cotangent bundle:
        \begin{equation}
            PT^*M = (T^*M\backslash\{0_M\})/\sim,
        \end{equation}
        where the equivalence relation $\sim$ is defined by $\omega\sim\rho\iff\exists\lambda\in\mathbb{R}_0:\omega = \lambda\rho$.
    \end{property}

    \newdef{Contact structure}{\index{contact!structure}
        Let $M$ be a $(2n+1)$-dimensional smooth manifold. A distribution $\xi$ of contact elements on $M$ is called a contact structure on $M$ if the (locally) defining one-form $\alpha$ satisfies the following non-integrability condition\footnote{In fact it is maximally non-integrable. (Compare with Frobenius' theorem ?? TODO (FORM VERSION) ??.)}:
        \begin{equation}
            \alpha\wedge(d\alpha)^n\neq0.
        \end{equation}
        If the one-form $\alpha$ is defined globally on $M$, it is called a \textbf{contact form} and the pair $(M,\alpha)$ is accordingly called a \textbf{contact manifold}.
    }

    \newprop{Coorientable distribution}{\index{coorientable}
        A contact form $\alpha$ such that $\xi=\ker(\alpha)$ can be defined globally if and only if the distribution $\xi$ is coorientable, i.e. the line bundle $TM/\xi$ is trivial (or orientable).
    }

\subsection{Reeb vector fields}

    \newdef{Reeb vector field}{\index{Reeb vector field}
        Let $(M,\alpha)$ be a contact manifold. A Reeb vector field on $M$ is a vector field $X$ such that $\alpha(X) = 1$ and $\iota_Xd\alpha = 0$.
    }
    \begin{property}
        Given a contact manifold, there exists a unique Reeb vector field associated to it.
    \end{property}

    ?? COMPLETE ??
\chapter{\texorpdfstring{$G$-Structures}{G-Structures}}
    In the following table we give an overview of the more common $G$-structures one can define on a smooth (simply-connected) manifolds $M^n$.
    \begin{center}
        \begin{tabularx}{\textwidth}{|l|c|X|}
             \hline
                 Geometric structure&Structure group&Remarks\\
             \hline
                 Orientation&$\text{SL}(n,\mathbb{R})$&$\text{GL}^+(n, \mathbb{R})$ is sufficient for orientability. The special linear group gives rise to a volume form.\\
                 Riemannian metric&$O(n)$&\\&&\\
                 Almost-symplectic structure*&$\text{Sp}(n,\mathbb{R})$&Integrability (in the form of a closed form) gives a symplectic manifold.\\&&\\
                 Almost-complex structure*&$\text{GL}(k, \mathbb{C})$&Integrability (in the form of Newlander-Nirenberg) gives a complex manifold.\\&&\\
                 Almost-Hermitian structure*&$\text{U}(k)$&Integrability gives a K\"ahler manifold.\\&&\\
                 Calabi-Yau*&$\text{SU}(k)$&\\&&\\
                 Hyperk\"ahler**&$\text{Sp}(k)$&Hyperk\"ahler implies Calabi-Yau.\\&&\\
                 Almost quaternionic**&$(\text{GL}(k,\mathbb{H})\times\mathbb{H}^\times)/\mathbb{R}^\times$&Integrability gives a quaternionic manifold. We also require $k\geq2$ because for $k=1$ we would otherwise obtain that every orientable 4-manifold is quaternionic (amongst other things).\\&&\\
                 Quaternionic-K\"ahler**&$(\text{Sp}(k)\times\text{Sp}(1))/\mathbb{Z}_2$&These manifolds are not strictly K\"ahler since the structure group is not a subgroup of $\text{U}(2k)$.\\
             \hline
        \end{tabularx}
    \end{center}
    Structures marked with $\ast$ require the real dimension $n=2k$ to be even. Structures marked with $\ast\ast$ require the real dimension $n=4k$ to be a multiple of 4.

    \begin{remark*}
        This table is strongly related\footnote{The SL$(n, \mathbb{R})$-structure is technically not part of the original classification since it is not a subgroup of O$(n)$ and hence the manifold is not necessarily Riemannian.} to the classification of (\textit{irreducible} simply-connected \textit{nonsymmetric}) Riemannian manifolds by \textit{Berger}. A more general classification for manifolds which are not necessarily Riemannian was initiated by Berger and finished by others. However we will only mention this extension here, for references see \cite{diffgeom_physics}.

        Since not all concepts from this classification were defined throughout the compendium we will explain them here:
        \begin{itemize}
            \item \textbf{Irreducible}: A Riemannian manifold is said to be irreducible if it is not locally isomorphic to a product of Riemannian manifolds.
            \item \textbf{Symmetric}\footnote{or \textbf{locally symmetric}}: A smooth manifold, locally modelled on $V\cong\mathbb{R}^n$, is said to be symmetric if the curvature mapping $F(M)\rightarrow\Lambda^2V^*\otimes\mathfrak{g}$ is covariantly constant.
        \end{itemize}
    \end{remark*}

    \begin{remark}
        Although most manifolds from the above list admit an explicit definition, the quaternionic K\"ahler manifolds are exactly defined by their structure group/holonomy group.

        It is also clear that hyperk\"ahler manifolds are a specific class of quaternionic K\"ahler manifolds since Sp$(k)$ can be embedded in Sp$(k)\cdot\text{Sp}(1)$. To exclude this class we can just require the holonomy group to be all of Sp$(k)\cdot\text{Sp}(1)$. This is equivalent to requiring that quaternionic K\"ahler manifolds have a nonvanishing scalar curvature. This is related to the following property:
    \end{remark}
    \begin{property}
        Every quaternionic K\"ahler manifold is Einstein \ref{diff:einstein_manifold}. The hyperk\"ahler manifolds are then exactly the quaternionic K\"ahler manifolds with vanishing scalar curvature (which is constant by the manifold being Einstein).
    \end{property}
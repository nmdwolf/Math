\chapter{Integration on manifolds}

For the theory on measure spaces and Lebesgue integration see chapter \ref{chapter:lebesgue}.

\section{Orientation}
	\newdef{Orientation}{\index{orientation}\index{volume}
		Similar to definition \ref{tensor:orientation} we can define an orientation on a differentiable manifold $M$. First we modify the definition of the volume element a little bit. A \textbf{volume form} on $M$ is a nowhere-vanishing top-dimensional differential form $\text{Vol}\in\Omega^n(M)$ where $n = \dim(M)$. The definition of an orientation is then equivalent to that in \ref{tensor:orientation}.
		
		An \textbf{oriented atlas} is given by all charts of $M$ for which the pullback of the Euclidean volume form is a positive multiple of $\text{Vol}$. This also means that the transition functions have a positive Jacobian determinant\footnotemark. The existence of a volume form turns a differentiable manifold into an \textbf{orientable manifold}.
		\footnotetext{This is in fact an equivalent definition.}
		
		Alternatively an orientable manifold with volume form $\omega$ is said to be \textbf{positively oriented} if $\omega(v_1, ..., v_n)>0$ where $(v_1, ..., v_n)$ is a basis for $T_pM$.
	}
	
	\begin{example}\index{determinant}
		Let $M=\mathbb{R}^n$. The canonical Euclidean volume form is given by the determinant map
		\begin{equation}
			\det:(u_1, ..., u_n)\mapsto\det(u_1, ..., u_n)
		\end{equation}
		where the $u_n$'s are expressed in the canonical basis $(e_1, ..., e_n)$. The name `volume form' is justified by noting that the determinant map gives the signed volume of the n-dimensional parallelotope spanned by the vectors $\{u_1, ..., u_n\}$.
	\end{example}
	
	\begin{property}
		Let $\omega_1, \omega_2$ be two volume forms on $M$. Then there exists a smooth function $f$ such that \[\omega_1 = f\omega_2\] Furthermore, the sign of this function is constant on every connected component of $M$.
	\end{property}

\section{Integration of top-dimensional forms}\index{Lebesgue!integral}\index{measure}

	\newdef{Measure zero}{
		A subset $U\subset M$ of an orientable manifold is said to be of measure zero if it is the countable union of inverse images of null sets in $\mathbb{R}^n$.
	}
	
	\newdef{Integrable form}{
		A differential form is said to be integrable if its components with respect to any basis of $\Omega^k(M)$ are Lebesgue integrable on $\mathbb{R}^n$.
	}

	\begin{formula}[Compact support]
		Let $\omega$ be a top-dimensional form on $M$ with compact support on $U\subset M$.
		\begin{equation}
			\label{forms:integration_compact_support}
			\int_M\omega = \int_U\omega = \int_{-\infty}^{+\infty}\cdots\int_{-\infty}^{+\infty}\omega_{12...n}(x)dx^1dx^2...dx^n
		\end{equation}
		This integral is well defined because under a (orientation preserving) change of coordinates the component $\omega_{1...n}$ transforms as $\omega'_{1...n} = J\omega_{1...n}$ where $J$ is the Jacobian determinant. Inserting this in the integral and replacing $dx_i$ by $dx'_i$ then gives us the well-known change-of-variables formula from Lebesgue integration theory.
	\end{formula}

	\begin{formula}[General]
		Let $\omega$ be a top-dimensional form on $M$ and let $\{\varphi_i\}_i$ be a partition of unity\footnotemark\ subordinate to an atlas on $M$.
		\footnotetext{This always exists if we require $M$ to be paracompact. See definition \ref{topology:partition_of_unity} and property \ref{topology:paracompact_partition_unity}.}
		\begin{equation}
			\label{forms:integration}
			\int_M\omega = \sum_i\int_M\varphi_i\omega
		\end{equation}
		where the integrals on the right-hand side are defined as in equation \ref{forms:integration_compact_support}.
	\end{formula}
	
	\newprop{Compact manifolds}{
		Let $M$ be a smooth compact manifold. Because every continuous form on $M$ is obviously compactly supported all continuous forms are integrable on $M$.
	}
	\newprop{Pullback}{
		Let $f:M\rightarrow N$ be an orientation-preserving diffeomorphism.
		\begin{equation}
			\int_Mf^*\omega = \int_N\omega
		\end{equation}
	}
	
	\begin{notation}
		Because the integral of differential forms satisfies properties similar to the ones listed in \ref{lebesgue:general_properties} we introduce the following notation:
		\begin{equation}
			\int_M\omega = \langle M, \omega \rangle
		\end{equation}
	\end{notation}
	
\section{General integration}

\section{Stokes' theorem}	

	\begin{theorem}[Stokes' theorem]\index{Stokes!theorem for differential forms}
		Let $\Sigma$ be an orientable smooth manifold. Denote the boundary of $\Sigma$ by $\partial\Sigma$. Let $\omega$ be a differential $k$-form on $\Sigma$. We have the following equality:
		\begin{equation}
			\label{forms:theorem:stokes_theorem}
			\boxed{\int_{\partial\Sigma}\omega = \int_\Sigma d\omega}
		\end{equation}
	\end{theorem}
	\begin{result}
		The Kelvin-Stokes theorem \ref{vectorcalculus:stokes_theorem}, the divergence theorem \ref{vectorcalculus:divergence_theorem} and Green's identity \ref{vectorcalculus:green_indentity} are immediate results of this (generalized) Stokes' theorem.
	\end{result}

\section{de Rham Cohomology}\index{cochain}\index{de Rham!cohomology}

	Now we can also give a little side note about why the de Rham cohomology groups \ref{forms:de_rham_cohomology} really form a cohomology theory. For this we need some concepts from homology which can be found in section \ref{section:homology}. Let $M$ be a compact differentiable manifold and let $\{\lambda_i:\Delta^k\rightarrow M\}$ be the set of singular $k$-simplexes on $M$.
	
	Now suppose that we want to integrate over a singular $k$-chain $C$ on $M$, i.e. $C = \sum_{i=0}^ka_i\lambda_i$. Formula \ref{forms:integration} says that we can pair the $k$-form $\omega$ and the chain $C$ such that they act as duals to each other (hence $p$-forms are also called $p$-\textbf{cochains}), producing a real number\footnote{This requires the chain group to have real coefficients instead of integer coefficients as is mostly used in homology.}:
	\begin{equation}
		\langle\cdot,C\rangle:\Omega^n(M)\rightarrow\mathbb{R}:\omega\mapsto\int_C\omega = \sum_{i=0}^ka_i\int_{\Delta_k}\lambda_i^{*}\omega
	\end{equation}
	where $\lambda_i^*$ pulls back $\omega$ to $\Delta^k$ which is a subset of $\mathbb{R}^k$ as required. Now Stokes' theorem \ref{forms:theorem:stokes_theorem} tells us that
	\begin{equation}
		\int_Cd\omega = \int_{\partial C}\omega
	\end{equation}
	Using the paring $\langle\cdot,\cdot\rangle$ this becomes
	\begin{equation}
		\langle d\omega, C\rangle = \langle \omega, \partial C\rangle
	\end{equation}
	The operators $d$ and $\partial$ can thus be interpreted as formal adjoints. After checking (again using Stokes' theorem) that all chains $C$ and cochains $\omega$ belonging to the same equivalence classes $[C]\in H_k(M, \mathbb{R})$ and $[\omega]\in H^k(M, \mathbb{R})$ give rise to the same number\footnote{Suppose that $A, B\in [C]$ and $\phi, \chi\in[\omega]$ then $\langle \phi, A \rangle = \langle \chi, B \rangle$.} $\langle\omega, C\rangle$ we see that the singular homology groups and the de Rham cohomology groups on $M$ are well defined dual groups. The name cohomology is thus wel chosen for \ref{forms:de_rham_cohomology}.

\chapter{K-theory}

    The main reference for this chapter is \cite{K}.

    In this chapter all topological (base) spaces are supposed to be both compact and Hausdorff. This ensures that the complex of K-theories satisfies the Eilenberg-Steenrod axioms \ref{topology:eilenberg_steenrod_axioms}.

\section{Basic definitions}

    \newdef{K-theory}{\index{K-theory}
        Let Vect$(X)/\sim$ be the set of isomorphism classes of finite-dimensional vector bundles over a topological space $X$. Because this set is well-behaved with respect to Whitney sums, the structure $(\text{Vect}(X)/\sim, \oplus)$ forms an Abelian monoid. The Grothendieck completion\footnote{See definition \ref{group:grothendieck_completion}.} of $(\text{Vect}(X)/\sim, \oplus)$ is called the (real) $K$-theory of $X$.
    }

    \begin{notation}
        \nomenclature[S_K]{$K^0(X)$}{K-theory over a (compact Hausdorff) space $X$.}
        The $K$-theory of a space $X$ is denoted by $K^0(X)$.
    \end{notation}

    \begin{example}
        Let $\{x_0\}$ be a one-point space. The K-theory $K^0(\{x_0\})$ is isomorphic to the additive group of integers $(\mathbb{Z}, +)$.
    \end{example}

    \newdef{Virtual vector bundle}{\index{virtual!vector bundle}\index{vector!bundle}
        The elements of $K^0(X)$ are pairs $([E], [E'])$ that can formally be written as a difference $[E] - [E']$. These elements are called virtual (vector) bundles.
    }
    \newdef{Virtual rank}{\index{virtual!rank}\index{rank}
        The virtual rank of the virtual bundle $([E], [E'])$ is defined as follows:
        \begin{gather}
            \text{rk}([E],[E']) := \text{rk}(E) - \text{rk}(E').
        \end{gather}
    }

    \begin{property}
        Property \ref{bundles:prop:hausdorff} implies that every virtual bundle is of the form $[E] - [X\times\mathbb{R}^n]$ for some vector bundle $E$ and integer $n\in\mathbb{N}$.
    \end{property}

    \newdef{Reduced K-theory}{
        Let $(X, x_0)$ be a pointed space. The inclusion $\{x_0\}\hookrightarrow X$ induces a group morphism $M: K^0(X)\rightarrow K^0(x_0)$ given by the restriction of virtual bundles to the basepoint $x_0$. The reduced K-theory $\widetilde{K^0}(X)$ is given by $\ker(M)$.
    }

    \begin{adefinition}
        One can define the reduced K-theory $\widetilde{K}(X)$ equivalently as follows: Consider the stable isomorphism classes\footnote{See definition \ref{bundle:stable_isomorphism}.} of vector bundles over $X$. Under Whitney sums these define a commutative group $(\text{Vect}(X)/\sim_{stable}, \oplus)$ which is (naturally) isomorphic to $\widetilde{K^0}(X)$.
    \end{adefinition}
\chapter{\difficult{K-theory}}\index{K-theory}

    The main reference for this chapter is \cite{K}.

    In this chapter all topological (base) spaces are supposed to be both compact and Hausdorff. This ensures that the complex of K-theories satisfies the Eilenberg-Steenrod axioms \ref{topology:eilenberg_steenrod_axioms}.

\section{Topological K-theory}
\subsection{Introduction}

    \newdef{K-theory}{
        Let Vect$(X)/\sim$ be the set of isomorphism classes of finite-dimensional vector bundles over a topological space $X$. Because this set is well-behaved with respect to Whitney sums, the structure $(\text{Vect}(X)/\sim, \oplus)$ forms an Abelian monoid. The Grothendieck completion\footnote{See definition \ref{group:grothendieck_completion}.} of $(\text{Vect}(X)/\sim, \oplus)$ is called the (real) $K$-theory of $X$.
    }

    \begin{notation}
        \nomenclature[S_K]{$K^0(X)$}{K-theory over a (compact Hausdorff) space $X$.}
        The $K$-theory of a space $X$ is denoted by $K^0(X)$.
    \end{notation}

    \begin{example}
        Let $\{x_0\}$ be a one-point space. The K-theory $K^0(\{x_0\})$ is isomorphic to the additive group of integers $(\mathbb{Z}, +)$.
    \end{example}

    \newdef{Virtual vector bundle}{\index{virtual!vector bundle}\index{vector!bundle}
        The elements of $K^0(X)$ are pairs $([E], [E'])$ that can formally be written as a difference $[E] - [E']$. These elements are called virtual (vector) bundles.
    }
    \newdef{Virtual rank}{\index{virtual!rank}\index{rank}
        The virtual rank of the virtual bundle $([E], [E'])$ is defined as follows:
        \begin{gather}
            \text{rk}([E],[E']) := \text{rk}(E) - \text{rk}(E').
        \end{gather}
    }

    \begin{property}
        Property \ref{bundles:prop:hausdorff} implies that every virtual bundle is of the form $[E] - [X\times\mathbb{R}^n]$ for some vector bundle $E$ and integer $n\in\mathbb{N}$.
    \end{property}

    \newdef{Reduced K-theory}{
        Let $(X, x_0)$ be a pointed space. The inclusion $\{x_0\}\hookrightarrow X$ induces a group morphism $M: K^0(X)\rightarrow K^0(x_0)$ given by the restriction of virtual bundles to the basepoint $x_0$. The reduced K-theory $\widetilde{K^0}(X)$ is given by $\ker(M)$.
    }

    \begin{adefinition}
        One can define the reduced K-theory $\widetilde{K}(X)$ equivalently as follows: Consider the stable isomorphism classes\footnote{See definition \ref{bundle:stable_isomorphism}.} of vector bundles over $X$. Under Whitney sums these define a commutative group $(\text{Vect}(X)/\sim_{stable}, \oplus)$ which is (naturally) isomorphic to $\widetilde{K^0}(X)$.
    \end{adefinition}

\section{Algebraic K-theory}
\subsection{Determinant}

    Over noncommutative rings $R$ the determinant of a matrix is not as easily defined as over commutative rings such as field. For example in the $2\times2$ case one could choose either $ad-bc$ or $da-bc$ (or any other permutation), there exists no canonical choice. To fix this we take a look at the most important properties of the determinant map:
    \begin{itemize}
        \item It is invariant under elementary row/column operations (see item 3 of property \ref{linalgebra:determinant_properties}).
        \item It is invariant under augmentation by the identity, i.e. under the transformation $A\mapsto A\oplus\mathbbm{1}$.
    \end{itemize}
    To implement the second property we will move from the finite-dimensional general linear groups $\text{GL}(n, R)$ to the so-called ''stable'' version:
    \newdef{Stable general linear group}{\index{stable!group}
        For every two integers $m<n$ there exists a canonical inclusion map $\text{GL}(m, R)\hookrightarrow\text{GL}(n, R)$ through augmentation by $\mathbbm{1}_{n-m}$. This allows us to define the stable general linear group (or infinite general linear group) as a direct limit:
        \begin{gather}
            \text{GL}(R) := \varinjlim\text{GL}(n, R).
        \end{gather}
    }

    On this stable group one can then define an equivalence relation by saying that two matrices are equivalent if they belong to the same coset with respect to the subgroup $E(R)$ generated by the elementary matrices \ref{linalgebra:elementary_matrix}. It can also be shown that $E(R)$ is equal to the commutator subgroup $[\text{GL}(R), \text{GL}(R)]$.

    The determinant map is then abstractly defined as the quotient map from the following definition:
    \newdef{$K_1$}{\index{determinant}
        The first algebraic K-group of a ring $R$ is defined as the Abelianization of its stable general linear group:
        \begin{gather}
            K_1(R) := \text{GL}(R)/[\text{GL}(R),\text{GL}(R)].
        \end{gather}
        The quotient map $\pi:\text{GL}(R)\rightarrow K_1(R)$ is called the \textbf{determinant map}.
    }

    To obtain lower K-groups we will define a ''suspension functor'':
    \newdef{Suspension}{\index{suspension}
        Let $R$ be a ring. By $\text{Mat}(R)$ we now denote the infinite matrix ring over $R$, i.e. the set of matrices with a finite number of nonzero entries in each row and column. This ring contains an ideal $\text{Mat}_{\text{fin}}(R)$ generated by all matrices that are zero outside a block of finite size. The suspension of $R$ is then defined as follows:
        \begin{gather}
            \Sigma R := \text{Mat}(R) / \text{Mat}_{\text{fin}}(R).
        \end{gather}
    }
    \newdef{Lower K-groups}{
        For all integers $n\geq1$ one defines the K-groups as follows:
        \begin{gather}
            H_n(R) := K_{1-n}(\Sigma^nR).
        \end{gather}
    }

    \begin{example}[$K_0$]
        It can be shown that $K_0(R)$ corresponds to the Grothendieck group associated to the monoid of finitely-generated projective $R$-modules. The relation to its topological counterpart is given by the Serre-Swan theorem \ref{diff:serre_swan}.
    \end{example}
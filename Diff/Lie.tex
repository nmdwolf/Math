\chapter{Lie groups and Lie algebras}\label{chapter:lie}

    References for this chapter are \cite{jeevanjee, fultonharris}. For some concepts such as vector fields and pushforwards we refer to chapter \ref{chapter:vector_bundles} below.

\section{Lie groups}

    \newdef{Lie group}{\index{Lie!group}\label{group:lie_group}
        A Lie group is a group that is also a differentiable manifold such that both the multiplication and inversion are smooth functions.\footnote{For complex Lie groups one needs the definition of a complex manifold (see chapter \ref{chapter:complex_geometry}).}
    }

    \newdef{Lie subgroup}{
        A subset of a Lie group is a Lie subgroup if it is both a subgroup and an immersed submanifold. If it is a regular submanifold then it is sometimes called a regular Lie subgroup.
    }
    \begin{theorem}[Closed subgroup theorem\footnotemark]\index{Cartan}
        \footnotetext{Sometimes called \textbf{Cartan's theorem}.}
        If $H$ is a subgroup of a Lie group $G$, closed with respect to the topology of $G$, then $H$ is an regular Lie subgroup of $G$.
    \end{theorem}

    \begin{property}\label{lie:prop_connected}
        Let $G$ be a connected Lie group. Every neighbourhood $U_e$ of the identity $e$ generates $G$, i.e. every element $g\in G$ can be written as a word in $U_e$.
    \end{property}

    \newdef{Isogeny}{
        Let $G, H$ be two Lie groups. $G$ and $H$ are said to be isogenous if one is a covering space\footnote{See definition \ref{topology:covering_space}.} of the other. The covering map is then called an isogeny between $G$ and $H$.
    }

\subsection{Left invariant vector fields}

    \newdef{Left-invariant vector field}{\index{vector field!left-invariant}
        \nomenclature[A_LIVF]{LIVF}{Left-invariant vector field}
        \nomenclature[S_LIVF]{$\mathfrak{X}^L$}{Space of left-invariant vector fields on a Lie group $G$.}
        Let $G$ be a Lie group and let $X$ be a vector field on $G$. $X$ is left-invariant if the following equivariance relation holds for all $g\in G$:
        \begin{gather}
            L_{g,\ast}X(h) = X(g\cdot h)
        \end{gather}
        where $L_g$ denotes the left action map associated with $g$. Left-invariant vector fields are often abbreviated as \textbf{LIVF}'s.
    }
    \begin{property}
        The set $\mathfrak{X}^L(G)$ of LIVF's on a Lie group $G$ is a vector space over $\mathbb{R}$.
    \end{property}
    \begin{property}\label{lie:livf_prop}
        The map $L_{g,\ast}$ is an isomorphism for every $g\in G$. It follows that a LIVF is uniquely determined by its value at the identity of $G$. Furthermore, for every $v\in T_e(G)$, there exists a LIVF $X\in\mathfrak{X}^L(G)$ such that $X(e) = v$ and this mapping is an isomorphism from $T_e(G)$ to $\mathfrak{X}^L(G)$.
    \end{property}

\subsection{One-parameter subgroups}

    \newdef{One-parameter subgroup}{\index{one-parameter subgroup}\label{group:one_parameter_subgroup}
        A one-parameter (sub)group is a Lie group morphism $\Phi:\mathbb{R}\rightarrow G$ from the additive group of real numbers to a Lie group $G$.
    }

    \begin{property}\label{group:OPS_composition}
        Let $\Phi:\mathbb{R}\rightarrow G$ be a one-parameter subgroup of $G$ and let $\Psi:G\rightarrow H$ be a continuous group morphism. Then $\Psi\circ\Phi:\mathbb{R}\rightarrow H$ is a one-parameter subgroup of $H$.
    \end{property}

    \begin{property}\label{lie:livf_subgroup}
        All LIVF's $X$ are complete\footnote{See definition \ref{manifold:complete_vector_field}.}. Hence for every LIVF $X$ we can find an integral curve $\gamma_X$ with initial condition $\gamma_X(0) = e$ for which the maximal flow domain\footnote{See definition \ref{manifolds:flow}.} $D(X)$ is $]-\infty, +\infty[$. This implies that the associated flow $\sigma_t$ determines a one-parameter subgroup of $G$. Conversely, for every one-parameter subgroup $\phi(t)$ we can construct a LIVF $X := \phi'(0)$. This correspondence is a bijection.
    \end{property}

\subsection{Cocycles}

    \newdef{Cocycle}{\index{cocycle}\label{group:cocycle}
        Let $M$ be a smooth manifold and $G$ a Lie group. A ($G$-valued) cocycle on $M$ with values in $G$ is a family of smooth functions $g_{ij}:U_i\cap U_j\rightarrow G$ that satisfy the following condition:
        \begin{gather}
            \label{group:cocycle_condition}
            g_{ij} = g_{ik}\circ g_{kj}.
        \end{gather}
    }
    \begin{property}
        Let $\{g_{ij}\}_{i,j}$ be a cocycle on $M$. We have the following properties for all $x\in M$:
        \begin{itemize}
            \item $g_{ii}(x) = e$
            \item $g_{ij}(x) = (g_{ji}(x))^{-1}$
        \end{itemize}
    \end{property}


\section{Lie algebras}

    There are two ways to define a Lie algebra. The first one is a stand-alone definition as a vector space equipped with a multiplication operation satisfying certain conditions. The second one establishes a direct relation between Lie groups (see \ref{group:lie_group}) and Lie algebras.

\subsection{Definitions}

    \newdef{Lie algebra}{\index{Lie!algebra}\index{Lie!bracket}\index{Jacobi!identity}\label{linalgebra:lie_algebra}
        Let $V$ be a vector space equipped with a binary operation $[\cdot, \cdot]:V\times V\rightarrow V$, called the \textbf{Lie bracket}. $(V, [\cdot,\cdot])$ is a Lie algebra if the Lie bracket satisfies the following conditions:
        \begin{enumerate}
            \item \textbf{Bilinearity}: $[ax + y, z] = a[x, z] + [y, z]$
            \item \textbf{Alternativity}: $[v, v] = 0$
            \item \textbf{Jacobi identity}: $[a, [b, c]] + [b, [c, a]] + [c, [a, b]] = 0$
        \end{enumerate}
    }
    \remark{Note that often the alternativity condition is replaced by an antisymmetry condition. However, this is only well-defined in a field of characteristic $\neq2$, for $\text{char}=2$ we only have that alternativity implies antisymmetry (not the other way around). Since we will almost exclusively work over fields such as $\mathbb{R}$ or $\mathbb{C}$, this won't pose a problem and hence we will always use the antisymmetry condition as the defining property.}

    \newdef{Structure constants}{\index{structure!constants}
        Because Lie algebras are closed under the Lie bracket, the Lie bracket of two elements can be expressed in term of a basis $\{X_k\}_{k\in I}$ as follows:
        \begin{gather}
            [X_i, X_j] = \sum_{k\in I}c_{ij}^{\ \ k}X_k
        \end{gather}
        where the factors $c_{ij}^{\ \ k}$ are called the structure constants\footnote{Note that these constants are basis-dependent.} of the Lie algebra.
    }
    \begin{property}
        Two Lie algebras $\mathfrak{g}, \mathfrak{h}$ are isomorphic if one can find bases $\mathcal{B}$ for $\mathfrak{g}$ and $\mathcal{C}$ for $\mathfrak{h}$ such that the associated structure constants are equal for all indices $i, j$ and $k$.
    \end{property}

    \begin{example}[Lie algebra of LIVF's]
        Consider the vector space $\mathfrak{X}^L(G)$ of LIVF's on a Lie group G. Using property \ref{manifolds:lie_bracket} we can show that the commutator (Lie bracket) also defines a LIVF on $G$. It follows that $\mathfrak{X}^L(G)$ is closed under Lie brackets and hence is a Lie algebra.
    \end{example}
    For the following alternative definition we use property \ref{lie:livf_prop} to relate the above Lie algebra of left-invariant vector fields and the tangent space to the identity:
    \newadef{Lie algebra of Lie group}{
        Let $G$ be a Lie group. The tangent space $\mathfrak{g}:=T_eG$ has the structure of a Lie algebra where the Lie bracket is induced by the commutator of vector fields \ref{manifolds:lie_bracket} in the following way:
        \begin{gather}
            \text{\textlbrackdbl} A, B \text{\textrbrackdbl} := l_{g^{-1}, *}\left[l_{g, *}A, l_{g, *}B\right]
        \end{gather}
        where $A, B\in T_eG$ and where $[\cdot, \cdot]$ is the Lie bracket on $\mathfrak{X}^L(G)$. This induces an isomorphism of Lie algebras: $\mathfrak{g}\cong_{\text{Lie}}\mathfrak{X}^L(G)$. We will freely use this isomorphism throughout this text.
    }

    \begin{notation}
        Lie algebras are generally denoted by fraktur symbols. For example, the Lie algebra associated with the Lie group $G$ is often denoted by $\mathfrak{g}$.
    \end{notation}

    \begin{theorem}[Ado]\index{Ado}\label{lie:theorem:ado}
        Every finite-dimensional Lie algebra can be embedded as a subalgebra of $\mathfrak{gl}_n \cong \textup{M}_n$.
    \end{theorem}
    \begin{theorem}[Lie's third theorem]\index{Lie!third theorem}
        Every finite-dimensional Lie algebra $\mathfrak{g}$ is the Lie algebra of a unique simply-connected Lie group $G$.
    \end{theorem}

    \newdef{Lie algebra morphism}{\index{morphism!of Lie algebras}
        A map $\Phi:\mathfrak{g}\rightarrow\mathfrak{h}$ is a Lie algebra morphism if it satisfies the following condition
        \begin{gather}
            \Phi([X, Y]) = [\Phi(X), \Phi(Y)]
        \end{gather}
        for all $X, Y\in\mathfrak{g}$.
    }
    \begin{property}[Homomorphism theorem\footnotemark]\label{lie:prop_hom}
        \footnotetext{See also formula \ref{lie:induced_homomorphism}.}
        Let $G, H$ be Lie groups with $G$ simply-connected. If a linear map $\Phi:\mathfrak{g}\rightarrow\mathfrak{h}$ is a Lie algebra morphism then there exists a unique Lie group morphism $\phi:G\rightarrow H$ such that $\Phi = \phi_*$. The converse is trivial: every Lie group morphism induces a Lie algebra morphism through its differential.
    \end{property}

\subsection{Exponential map}

    \newformula{Exponential map}{\index{exponential map}
        Let $X$ be a LIVF on $G$. We define the exponential map $\exp:\mathfrak{g}\rightarrow G$ as
        \begin{gather}
            \exp(X) := \gamma_X(1)
        \end{gather}
        where $\gamma_X$ is the associated one-parameter subgroup defined in property \ref{lie:livf_subgroup}.
    }

    \begin{property}
        The exponential map is the unique map $\mathfrak{g}\rightarrow G$ such that exp$(0) = e$ and for which the restrictions to the lines through the origin in $\mathfrak{g}$ are one-parameter subgroups of $G$.
    \end{property}
    \begin{result}\label{lie:exp_result}
        Because the identity element $\mathbbm{1}_\mathfrak{g} = (\exp_\ast)_e$ is an isomorphism, the inverse function theorem \ref{manifolds:theorem:inverse_function_theorem} implies that the image of $\exp$ will contain a neighbourhood of the identity $e\in G$. If $G$ is connected then property \ref{lie:prop_connected} implies that $\exp$ generates all of $G$.

        Together with the property that $\psi\circ\exp = \exp\circ\ \psi_\ast$ for every Lie group morphism $\psi:G\rightarrow H$ it follows that \textit{if $G$ is connected, a Lie group morphism $\psi:G\rightarrow H$ is completely determined by its differential $\psi_\ast$ at the identity $e\in G$}.
    \end{result}

    \begin{example}[Matrix Lie groups]\index{matrix!exponentiation}
        For matrix Lie groups we can define the classic matrix exponential:
        \begin{gather}
            e^{tX} := \sum_{k=0}^{+\infty}\stylefrac{(tX)^k}{k!}.
        \end{gather}
        This operation defines a curve $\gamma(t)$ which can be used as a one-parameter subgroup on $G$. It should be noted that this formula converges for every $X\in M_{m,n}$ and that it is invertible with inverse given by $\exp(-X)$. Using Ado's theorem \ref{lie:theorem:ado} one can then use this matrix exponential to represent the exponential map for any (finite-dimensional) Lie algebra.
    \end{example}

    \begin{remark}
        Let $G$ be a compact Lie group. The exponential map is surjective. However, because the asocciated Lie algebra $\mathfrak{g}$ is clearly noncompact, the exponential map cannot be homeomorphic and hence cannot be injective.
    \end{remark}

    \newformula{Baker-Campbell-Hausdorff formula}{\index{Baker-Campbell-Hausdorff formula}
        This formula is the solution of the equation
        \begin{gather}
            Z = \log(\exp(X)\exp(X))
        \end{gather}
        for $X, Y\in\mathfrak{g}$. The solution is given by the following formula:
        \begin{gather}
            \label{linalgebra:bch_formula}
            e^Xe^Y = \exp\left(X + Y + \frac{1}{2}[X, Y] + \frac{1}{12}[X, [X, Y]] - \frac{1}{12}[Y, [X, Y]] + \cdots\right).
        \end{gather}
        One should note that this formula will only converge if $X, Y$ are sufficiently small (for matrix Lie algebras this means that $||X|| + ||Y|| < \frac{\ln(2)}{2}$ under the Hilbert-Schmidt norm \ref{linalgebra:hilbert_schmidt_norm}). Due to the closure under the Lie bracket the exponent in the BCH formula is also an element of the Lie algebra. So the formula gives an expression for Lie group multiplication in terms of Lie algebra elements (whenever the formula converges).
    }
    \begin{result}[Lie product formula\footnotemark]\index{Lie!product formula}\index{Trotter expansion}
        \footnotetext{Also called the \textbf{Lie-Trotter formula}. Later, extension where given by \textit{Kato} and \textit{Suzuki}.}
        Let $\mathfrak{g}$ be a Lie algebra. The following formula applies to any $X, Y\in\mathfrak{g}$:
        \begin{gather}
            \label{linalgebra:lie_product_formula}
            e^{X + Y} = \lim_{n\rightarrow+\infty}\left(e^{\frac{X}{n}}e^{\frac{Y}{n}}\right)^n.
        \end{gather}
    \end{result}

\subsection{Examples}

    \begin{example}
        The cross product $\times:\mathbb{R}^3\times\mathbb{R}^3\rightarrow\mathbb{R}^3$ turns $\mathbb{R}^3$ into a Lie algebra.
    \end{example}
    \begin{example}
        An interesting example is the Lie algebra associated to the Lie group of invertible complex\footnote{As usual, this result is also valid for real matrices.} matrices $\text{GL}(n, \mathbb{C})$. This Lie group is a subset of its own Lie algebra $\mathfrak{gl}(n, \mathbb{C}) = M_n(\mathbb{C})$. It follows that for every $A\in\text{GL}(n, \mathbb{C})$ and every $B\in\mathfrak{gl}(n, \mathbb{C})$ the following equality holds:
        \begin{gather}
            L_{A,\ast}(B) = L_A(B).
        \end{gather}
    \end{example}
    \begin{result}\label{lie:end_as_lie_algebra}
        By noting that the endomorphism ring End$(V)$ of an $n$-dimensional vector space $V$ is given by the matrix ring $M_n(K)$, we see that End$(V)$ also forms a Lie algebra when equipped with the commutator of linear maps.
    \end{result}

    \begin{example}[Isometries]\index{isometry}
        The Lie algebra associated with the group of isometries $\text{Isom}(V)$ of a nondegenerate Hermitian form satisfies the following condition
        \begin{gather}
            \label{lie:lie_isometry}
            \langle Xv, w \rangle = -\langle v, Xw \rangle
        \end{gather}
        for all Lie algebra elements $X$. It follows that the Lie algebra consists of all skew-Hermitian operators.
    \end{example}
    Two explicit cases are:
    \begin{example}[Lie algebra of \normalfont{O}$(3)$]\label{lie:so3}
        The set of $3\times3$ skew-symmetric matrices. It is important to note that $\mathfrak{o}(3) = \mathfrak{so}(3)$. The structure constants of this Lie algebra are given by the Levi-Civita symbol \ref{tensor:levi_civita_symbol}, i.e. $c_{ijk} = \varepsilon_{ijk}$.
    \end{example}
    \begin{example}[Lie algebra of \normalfont{SU}$(2)$]
        The set of $2\times2$ traceless skew-Hermitian matrices. This result can be generalized to arbitrary $n\in\mathbb{N}$.
    \end{example}

    Another important example is obtained by restricting $\text{GL}(2, \mathbb{C})$ to the subset of matrices with unit determinant:
    \begin{example}[Special linear group]
        To compute the Lie bracket of the Lie algebra $\mathfrak{sl}(2, \mathbb{C}) = T_e(\text{SL}(2, \mathbb{C}))$ we need to find the action of $l_{g, *}$ on any vector $Y\in\mathfrak{sl}(2, \mathbb{C})$. This is given by:
        \begin{gather}
            l_{\left(\begin{smallmatrix}a&b\\c&d\end{smallmatrix}\right), *}\left(\left.\pderiv{}{x^i}\right|_e\right)
            = \left(\begin{matrix}a&0&b\\-b&a&0\\c&0&\frac{1+bc}{a}\end{matrix}\right)^m_i\left.\pderiv{}{x^m}\right|_{\left(\begin{smallmatrix}a&b\\c&d\end{smallmatrix}\right)}
        \end{gather}
        where we used the coordinate chart $(U, \phi)$ defined by: \[U = \left\{\left(\begin{matrix}a&b\\c&d\end{matrix}\right)\in\text{SL}(2, \mathbb{C}): a\neq0\right\}\] and \[\phi:U\rightarrow\mathbb{C}^3:\left(\begin{matrix}a&b\\c&d\end{matrix}\right)\mapsto(a, b, c).\] One can then use this formula to work out the Lie bracket of the basis vectors $X_i = \left.\pderiv{}{x^i}\right|_e$ to obtain the following structure constants:
        \begin{gather}
            \label{lie:sl2c_lie_brackets}
            \begin{cases}
                \text{\textlbrackdbl}X_1, X_2\text{\textrbrackdbl} = 2X_2\\
                \text{\textlbrackdbl}X_1, X_3\text{\textrbrackdbl} = -2X_3.\\
                \text{\textlbrackdbl}X_2, X_3\text{\textrbrackdbl} = X_1
            \end{cases}
        \end{gather}
    \end{example}

\subsection{Solvable Lie algebras}

    \newdef{Normalizer}{\index{normalizer}
        The normalizer of a subset of a Lie algebra $S\subset\mathfrak{g}$ is the space of elements $x\in\mathfrak{g}$ that satisfy $[x,S]\subseteq S$.
    }
    \newdef{Centralizer}{\index{centralizer}
        The centralizer of a subset of a Lie algebra $S\subset\mathfrak{g}$ is the space of elements $x\in\mathfrak{g}$ that satisfy $[g,xS]=0$.
    }

    \newdef{Derived algebra}{\index{derived!algebra}
        Let $\mathfrak{g}$ be a Lie algebra. The derived Lie algebra is defined as follows:
        \begin{gather}
            [\mathfrak{g}, \mathfrak{g}] := \big\{[x, y]:x, y\in\mathfrak{g}\big\}.
        \end{gather}
    }
    \newdef{Solvable Lie algebra}{\index{solvable}
        Consider the sequence of derived Lie algebras
        \begin{gather}
            \mathfrak{g}\geq [\mathfrak{g}, \mathfrak{g}] \geq [[\mathfrak{g}, \mathfrak{g}], [\mathfrak{g}, \mathfrak{g}]] \geq \cdots
        \end{gather}
        If this sequence ends in the zero-space then the Lie algebra $\mathfrak{g}$ is said to be solvable.
    }
    \remark{In general one can define the derived series for any ideal $\mathfrak{J}\leq\mathfrak{g}$ and accordingly define solvability for ideals.}

    \newdef{Radical}{\index{radical}
        Let $\mathfrak{g}$ be a Lie algebra. The radical of $\mathfrak{g}$ is the largest solvable ideal in $\mathfrak{g}$.
    }

\subsection{Simple Lie algebras}

    \newdef{Direct sum}{\index{direct sum}
        The direct sum of two Lie algebras $\mathfrak{g}, \mathfrak{h}$ is defined as the direct sum in the sense of vector spaces (see \ref{linalgebra:direct_sum}) together with the condition
        \begin{gather}
            [x, y] = 0
        \end{gather}
        for all $x\in\mathfrak{g}$ and $y\in\mathfrak{h}$.
    }

    \newdef{Semidirect product}{\index{semidirect product}
        The semidirect product (or sum) $\mathfrak{g}\ltimes\mathfrak{h}$ of two Lie algebras $\mathfrak{g}, \mathfrak{h}$ is defined as the direct sum in the sense of vector spaces together with the condition that $\mathfrak{g}$ is an ideal of $\mathfrak{h}$ under the Lie bracket.
    }

    \newdef{Simple Lie algebra}{\index{simple!Lie algebra}
        A Lie algebra is said to be simple if it is non-Abelian and if it has no nontrivial ideals.
    }
    \newdef{Semisimple Lie algebra}{
        A Lie algebra is said to be semisimple if it is the direct sum of simple Lie algebras.
    }

    \begin{theorem}[Levi decomposition]\index{Levi!decomposition}
        Let $\mathfrak{g}$ be a finite-dimensional Lie algebra. This algebra can be decomposed as follows:
        \begin{gather}
            \mathfrak{g} = \mathfrak{R} \ltimes (\mathfrak{L}_1 \oplus \cdots \oplus \mathfrak{L}_n)
        \end{gather}
        where $\mathfrak{R}$ is the radical of $\mathfrak{g}$ and the algebras $\mathfrak{L}_i$ are simple subalgebras.
    \end{theorem}
    \begin{definition}\index{Levi!subalgebra}
        The semisimple subalgebra $\mathfrak{L}_1 \oplus \cdots \oplus \mathfrak{L}_n$ in the Levi decomposition of $\mathfrak{g}$ is called the \textbf{Levi subalgebra} or \textbf{Levi factor} of $\mathfrak{g}$.
    \end{definition}

\subsection{Central extensions}\label{section:central_extension_algebra}

    \newdef{Central extension}{\index{central extension!Lie algebra}
        A central extension of a Lie algebra $\mathfrak{g}$ by an Abelian Lie algebra $\mathfrak{a}$ is an exact sequence of Lie algebras of the form
        \begin{gather}
            0\longrightarrow\mathfrak{a}\longrightarrow\mathfrak{h}\longrightarrow\mathfrak{g}\longrightarrow0
        \end{gather}
        where the image of $\mathfrak{a}$ lies in the center of $\mathfrak{h}$.
    }

    Now consider a Lie algebra morphism $\Theta:\mathfrak{g}\times\mathfrak{g}\rightarrow\mathfrak{a}$ with the following properties:
    \begin{enumerate}
        \item $\Theta$ is bilinear
        \item $\Theta$ is antisymmetric
        \item $\Theta([a, b], c) + \Theta([b, c], a) + \Theta([c, a], b) = 0$
    \end{enumerate}
    Such a morphism is called a \textbf{Lie algebra 2-cocycle}.\index{cocycle!Lie algebra} Now, every 2-cocycle $\Theta:\mathfrak{g}\times\mathfrak{g}\rightarrow\mathfrak{a}$ induces a central extension of $\mathfrak{g}$ by $\mathfrak{a}$ in the following way: As a vector space we take\footnote{It always holds that $\mathfrak{h}=\mathfrak{g}\oplus\mathfrak{a}$ as vector spaces.} $\mathfrak{h}=\mathfrak{g}\oplus\mathfrak{a}$ and we define a Lie bracket on this space as follows:
    \begin{gather}
        [g\oplus\lambda, g'\oplus\mu] := [g, g']_{\mathfrak{g}} \oplus \Theta(g, g').
    \end{gather}

\section{Representation theory}
\subsection{Lie groups}

    \newdef{Representation of Lie groups}{\index{representation}
        Let $G$ be a Lie group and let $V$ be a vector space. A representation of $G$ on $V$ is a Lie group morphism $\rho:G\rightarrow\text{GL}(V)$.
    }

    \newdef{Adjoint representation of Lie groups}{\index{adjoint!representation}\label{lie:adjoint_representation}
        \nomenclature[O_Adg]{$\text{Ad}_g$}{Adjoint representation of a Lie group $G$.}
        Let $G$ be a Lie group. Consider the conjugation map (adjoint action) $\Psi_g:h\mapsto ghg^{-1}$. The adjoint representation of $G$ on $\mathfrak{g}$ is defined by the differential $T_e\Psi_g$. For matix Lie groups this becomes
        \begin{gather}
            \text{Ad}_g:T_eG\rightarrow T_eG: X\mapsto gXg^{-1}.
        \end{gather}
    }

\subsection{Lie algebras}

    \newdef{Representation of Lie algebras}{
        Let $\mathfrak{g}$ be a Lie algebra and let $V$ be a vector space. A representation of $\mathfrak{g}$ on $V$ is a Lie algebra morphism $\rho:\mathfrak{g}\rightarrow\text{End}(V)$.
    }

    \newformula{Adjoint representation of Lie algebras}{
        \nomenclature[O_adX]{$\text{ad}_X$}{Adjoint representation of a Lie algebra $\mathfrak{g}$.}
        Using the fact that the adjoint representation of Lie groups is smooth we can define the adjoint representation of Lie algebras as
        \begin{gather}
           \text{ad}_X := T_e(\text{Ad}_g)
        \end{gather}
        where $g = e^{tX}$. Explicitly, let $\mathfrak{g}$ be a Lie algebra. For every element $X\in\mathfrak{g}$ the adjoint map is given by
        \begin{gather}
            \label{lie:bracket_as_adjoint_rep}
            \text{ad}_X(Y) = [X, Y].
        \end{gather}
    }

    \begin{property}
        The adjoint representation ad$_X$ is faithful.
    \end{property}

    \begin{property}
        Given the antisymmetry of the Lie bracket, the Jacobi identity is equivalent to ad$:\mathfrak{g}\rightarrow$ End$(\mathfrak{g})$ being a Lie algebra morphism, i.e. ad$_{[X, Y]} = [$ad$_X, $ad$_Y]$.
    \end{property}

    \begin{formula}
        Let $\{e_i\}_{i\leq n}$ be a basis of a Lie algebra $\mathfrak{g}$. The structure coefficients can be calculated using the adjoint map as follows:
        \begin{gather}
            \label{lie:ad_structure_coefficient}
            (\text{ad}_{e_i})^j_k = C_{ik}^{\ \ j}
        \end{gather}
    \end{formula}

    \newformula{Induced morphism}{\index{morphism!induced}\label{lie:induced_homomorphism}
        Let $\phi:G\rightarrow H$ be a Lie group morphism\footnote{Continuity (inherent to the definition of a Lie group morphism) is needed to ensure that $\phi(e^{tX})$ is also a one-parameter subgroup (see \ref{group:OPS_composition}).} with $G$ connected and simply-connected. This morphism induces a
 a Lie algebra morphism\footnote{See also property \ref{lie:prop_hom}.} $\Phi:\mathfrak{g}\rightarrow\mathfrak{h}$ given by
        \begin{gather}
            \Phi(X) := \left.\deriv{}{t}\phi\left(e^{tX}\right)\right|_{t=0}
        \end{gather}
        or equivalently
        \begin{gather}
            \phi\left(e^{tX}\right) = e^{t\Phi(X)}.
        \end{gather}
    }

    \begin{example}
        The morphism induced by $\text{Ad}:G\rightarrow H$ is precisely $\text{ad}:\mathfrak{g}\rightarrow\mathfrak{h}$. Informally we can thus say that the infinitesimal version of the similarity transformation is given by the commutator (when $G=\textup{GL}_n$):
    \end{example}
    \begin{result}[Commutator]\index{commutator}
        For the general linear group GL$_n$ the Lie bracket is given by the commutator:
        \begin{gather}
            [X, Y] := XY - YX.
        \end{gather}
        This follows from definition \ref{lie:bracket_as_adjoint_rep}: $[X, Y] = \left.\deriv{}{t}\text{Ad}_{\gamma(t)}(Y)\right|_{t=0}$ with $\gamma(0) = e$ and $\gamma'(0) = X$.
    \end{result}

\subsection{Coadjoint orbits}

    The main reference for this section is the paper \cite{witten_coadjoint}.

    \newdef{Coadjoint representation}{\index{coadjoint}
        This is the representation of a Lie group $G$ on the dual space $\mathfrak{g}^*$:
        \begin{gather}
            \langle \text{Ad}_g^*(\omega), v \rangle := \langle \omega, \text{Ad}_g^{-1}(v) \rangle.
        \end{gather}
        Infinitesimally this induces a representation of $\mathfrak{g}$ on its linear dual. It is given by
        \begin{gather}
            K_*(X):\omega\mapsto -\omega\circ\text{ad}_X.
        \end{gather}
    }

    \newdef{Coadjoint orbit}{
        Given an element $\omega\in\mathfrak{g}^*$ we define the coadjoint orbit $\Omega_\omega$ as the orbit of $\omega$ under the action of $G$. One can also define this orbit as the homogeneous space $G/G_\omega$.
    }

    The following important construction shows that every coadjoint orbit is in fact canonically a symplectic manifold\footnote{See chapter \ref{chapter:symplectic}.}:
    \newdef{Kirillov-Kostant-Souriau form}{\index{Kirillov-Kostant-Souriau form}
        First we define an antisymmetric bilinear map $B_\alpha:\mathfrak{g}\times\mathfrak{g}\rightarrow\mathbb{C}$ by the following formula:
        \begin{gather}
            B_\alpha(X, Y) := \langle \alpha, [X, Y] \rangle.
        \end{gather}
        Now consider a coadjoint orbit $\Omega_\alpha$. Since $\alpha$ is an element of the coadjoint representation of $G$, the tangent vectors to $\Omega_\alpha$ (at $\alpha$) are naturally elements of the induced representation of $\mathfrak{g}$. Hence for any tangent vector $v$ one can consider $v:=K_*(X)\alpha$ for some $X\in\mathfrak{g}$. We then define a 2-form on $\Omega_\alpha$ by the following formula:
        \begin{gather}
            \omega_\alpha(v, w) = B_\alpha(X, Y)
        \end{gather}
        where $X, Y$ are the adjoint vectors generating the tangent vectors $v, w$. That this definition in fact defines a symplectic form is not hard to prove using the above definitions.
    }

\section{Structure}
\subsection{Killing form}

    \newdef{Killing form\footnotemark}{\index{Killing!form}
        \footnotetext{Also called the \textbf{Cartan-Killing form}.}
        Let $\mathfrak{g}$ be a finite-dimensional Lie algebra. The Killing form on $\mathfrak{g}$ is defined as the following symmetric bilinear form\footnote{i.e. a symmetric $(0,2)$-tensor in $\mathfrak{g}^*\otimes\mathfrak{g}^*$ (See definition \ref{tensor:type})}:
        \begin{gather}
            \label{linalgebra:killing_form}
            K(X, Y) := \text{tr}(\text{ad}_X\circ\text{ad}_Y).
        \end{gather}
        The trace can be found by representing the Lie algebra elements as matrices using Ado's theorem \ref{lie:theorem:ado}. From equation \ref{lie:ad_structure_coefficient} we can work out the action of the Killing form on the basis $\{e_i\}_{i\leq n}$:
        \begin{gather}
            K_{ij} = c_{ik}^{\ \ l}c_{jl}^{\ \ k}
        \end{gather}
        where $c_{ij}^{\ \ k}$ are the structure constants of the Lie algebra.
    }

    \begin{theorem}[Cartan's criterion]\index{Cartan!criterion}
        A Lie algebra is semisimple if and only if its Killing form is nondegenerate.
    \end{theorem}

    \begin{property}
        If a Lie group $G$ is compact then the Killing form of its associated Lie algebra $\mathfrak{g}$ is negative-definite.
    \end{property}
    \begin{result}
        Let $G$ be a compact Lie group. If its Lie algebra is semisimple the Killing form $K$ induces a metric
        \begin{gather}
            g:(X, Y)\mapsto -\ \text{tr}(\text{ad}_X\circ\text{ad}_Y) = -K(X, Y)
        \end{gather}
        which turns the corresponding Lie group $G$ into a Riemannian manifold.
    \end{result}

    \begin{property}
        The adjoint map $\text{ad}_Z$ is antisymmetric with respect to the Killing form:
        \begin{gather}
            \label{lie:ad_killing_form}
            K(\text{ad}_ZX, Y) = -K(X, \text{ad}_ZY)
        \end{gather}
        or equivalently:
        \begin{gather}
            K([X, Z], Y) = K(X, [Z, Y]).
        \end{gather}
    \end{property}
    \begin{property}
        The Killing-form is $\text{Ad}$-invariant, i.e.
        \begin{gather}
            K(\text{Ad}_g(X), \text{Ad}_g(Y)) = K(X, Y)
        \end{gather}
        for all $g\in G$. From this it follows that $\text{Ad}$ is a map from $G$ to the isometry group $\text{Isom}(\mathfrak{g})$.
    \end{property}

    \begin{property}\label{lie:killing_trace}
        For a simple Lie algebra, every invariant, i.e. satisfying equation \ref{lie:ad_killing_form}, symmetric bilinear form is a scalar multiple of the Killing form.
    \end{property}
    \begin{example}
        For $\mathfrak{su}(n)$ the relation is given by
        \begin{gather}
            \text{tr}(XY) = 2nK(X, Y).
        \end{gather}
    \end{example}

    \begin{property}\index{structure!constants}
        When the Lie algebra $\mathfrak{g}$ is compact and semisimple, hence when the Killing form induces a metric, one can find a basis of $\mathfrak{g}$, constructed by orthonormalising a given basis with respect to the Killing metric\footnote{The proof uses the ad-invariance of the Killing form.}, such that the structure constants are invariant under cyclic permutation of the indices:
        \begin{gather}
            c_{ijk} = c_{jki}.
        \end{gather}
        A corollary of this property is also that the structure constants become totally antisymmetric.
    \end{property}

    \begin{construct}[Induced Killing form]
        Let $\mathfrak{g}$ be a Lie algebra and let $V$ be a vector space equipped with a Lie algebra representation $\rho:\mathfrak{g}\rightarrow\text{End}(V)$. One can then define a Killing form associated with $\rho$ in the following way:
        \begin{gather}
            \label{lie:rho_killing_form}
            K_\rho(X, Y) = \text{tr}\Big(\rho(X)\circ\rho(Y)\Big)
        \end{gather}
        This definition is a generalization of \ref{linalgebra:killing_form} which reduces to the Killing form $K$ when choosing $V$ to be $\mathfrak{g}$ in the adjoint representation.
    \end{construct}

\subsection{Weights, roots and Dynkin diagrams}

    From here on we assume the base field to be algebraically closed (let us choose $\mathbb{C}$ for simplicity).

    \newdef{Cartan subalgebra}{\index{Cartan!subalgebra}
        Let $\mathfrak{g}$ be a Lie algebra. A subalgebra $\mathfrak{h}$ is called a Cartan subalgebra if satisfies the following two conditions:
        \begin{enumerate}
            \item \textbf{Nilpotent}: Its lower central series terminates:
                \begin{gather}
                    \exists n\in\mathbb{N}: \underbrace{[\mathfrak{h}, [\mathfrak{h}, [\mathfrak{h}, \ldots]]]}_{n \text{ times}} = 0
                \end{gather}
            \item \textbf{Self-normalizing}:
                \begin{gather}
                    \forall X\in\mathfrak{h}: [X, Y]\in\mathfrak{h} \implies Y\in\mathfrak{h}
                \end{gather}
        \end{enumerate}
    }

    From here one we will assume that all Lie algebras are finite-dimensional. This assumption is motivated by the following property:
    \begin{property}
        Every finite-dimensional Lie algebra contains a Cartan subalgebra.
    \end{property}
    \begin{property}
       If $\mathfrak{g}$ is semisimple then its Cartan subalgebra is Abelian.
    \end{property}

    \begin{construct}
        Let $\mathfrak{g}$ be a semisimple Lie algebra. A Cartan subalgebra $\mathfrak{h}$ can be constructed as follows: Choose linearly independent vectors $\{h_i\}_{i\in I}$ such that for all\footnote{The existence of such a choice, equivalent to requiring simultaneous diagonalization, is only guaranteed for semisimple Lie algebras.} $i, j\in I: [h_i, h_j] = 0$. If this set can be extended to a basis $\{h_i\}_{i\in I}\cup\{g_j\}_{j\in J}$ of $\mathfrak{g}$ such that every $g_j$ is a nontrivial eigenvector of the adjoint map $\text{ad}_{h_i}$ for all $i\in I$ then the algebra $\mathfrak{h} = \text{span}\{h_i\}_{i\in I}$ is a Cartan subalgebra.
    \end{construct}

    \newdef{Weight space}{\index{weight}\index{module!weight}\label{lie:weight_space}
        Let $V$ be a representation of a Lie algebra $\mathfrak{g}$ with Cartan subalgebra $\mathfrak{h}$. Let $\lambda$ be a linear functional on $\mathfrak{h}$. The weight space $V_\lambda$ of $V$ with \textbf{weight} $\lambda$ is defined as follows:
        \begin{gather}
            V_\lambda := \{v\in V: h\cdot v = \lambda(H)v, \forall h\in\mathfrak{h}\}.
        \end{gather}
        Nonzero elements of a weight space are called \textbf{weight vectors}. If the representation $V$ can be decomposed as a direct sum of weigth spaces it is called a \textbf{weight module}:
        \begin{gather}
            V = \bigoplus_{\lambda\in\mathfrak{h}^*}V_\lambda.
        \end{gather}
    }

    In the case that $V$ is the adjoint representation, the (nonzero) weights are called \textbf{roots}:
    \newdef{Root}{\index{root}
        Let $\mathfrak{g}$ be a Lie algebra with Cartan subalgebra $\mathfrak{h}$. From the definition of a Cartan subalgebra it follows that for all $h\in\mathfrak{h}$:
        \begin{gather}
            [h, g_j] = \alpha_j(h)g_j
        \end{gather}
        where $\{g_j\}_{j\in J}$ is the basis extension of $\mathfrak{g}$ with respect to $\mathfrak{h}$. Because $\alpha_j(h)$ is an eigenvalue it is an element of the base field $\mathbb{C}$ and hence we can view $\alpha_j$ as a linear map $\mathfrak{h}\rightarrow\mathbb{C}$, or equivalently as a weight of the adjoint representation. The no-zero linear maps are called the roots of $\mathfrak{g}$ and form the so-called \textbf{root system} $\Phi$.

        It follows that there exists a weight space decomposition of $\mathfrak{g}$:
        \begin{gather}
            \mathfrak{g} = \mathfrak{h} \oplus \bigoplus_{\lambda\in\Phi}\mathfrak{g}_\lambda
        \end{gather}
        where the one-dimensional spaces $\mathfrak{g}_\lambda$ are the weight spaces associated to the roots $\lambda$ ($\mathfrak{h}$ is equal to $\mathfrak{g}_0$ in this notation).
    }

    \begin{property}
        If $\alpha\in\Phi$ then $-\alpha\in\Phi$. Furthermore, if $\alpha\in\Phi$ and $c\alpha\in\Phi$ then $c=\pm1$.
    \end{property}
    We conclude that the root system $\Phi$ is not linearly independent. To introduce some kind of basis we define the following notion:
    \newdef{Simple root}{
        The set of simple roots\footnote{For every root system $\Phi$ one can find a set of simple roots.} $\Delta$ is a linearly independent subset of $\Phi$ such that every element $\lambda\in\Phi$ can be written as
        \begin{gather}
            \lambda = \pm\sum_i^na_i\lambda_i
        \end{gather}
        where $a_i\in\mathbb{N}$ and $\lambda_i\in\Delta$. This definition requires the expansion coefficients $a_i$ of a certain root $\lambda$ to be either all positive or all negative.
    }

    More generally one can define an equivalence relation on the root system $\Phi$:
    \newdef{Positive roots}{
        Let $\Phi$ be the root system of a given Lie algebra $\mathfrak{g}$. Because the only scalar multiples of a root $\lambda\in\Phi$ in the root system are $\pm\lambda$ we can define a set of positive roots $\Phi^+$ as follows:
        \begin{enumerate}
            \item $\lambda\in\Phi^+\implies-\lambda\not\in\Phi^+$
            \item $\alpha, \beta\in\Phi^+\land\alpha+\beta\in\Phi\implies\alpha+\beta\in\Phi^+$
        \end{enumerate}
        The simple roots are then exactly the elements in $\Phi^+$ that cannot be written as a sum of other elements in $\Phi^+$.
    }

    \newdef{Triangular decomposition}{\index{Borel!subalgebra}
        Given a choice of positive roots $\Phi^+$ one can decompose the Lie algebra $\mathfrak{g}$ as follows:
        \begin{gather}
            \mathfrak{g} = \mathfrak{n}_-\oplus\mathfrak{h}\oplus\mathfrak{n}_+
        \end{gather}
        where $\mathfrak{n}_\pm = \bigoplus_{\alpha\in\Phi^\pm}\mathfrak{g}_\alpha$. The subalgebra $\mathfrak{h}\oplus\mathfrak{n}_+$ is called the \textbf{Borel subalgebra}. It is the maximal solvable subalgebra in $\mathfrak{g}$.
    }

    \begin{property}[Rank]\index{rank}
        Let $\mathfrak{h}$ be a Cartan subalgebra. The set of simple roots $\Delta$ forms a basis for the dual space $\mathfrak{h}^*$ (over $\mathbb{C}$) and hence the cardinality of $\Delta$ is equal to the dimension of the Cartan subalgebra. This dimension is called the \textbf{rank} of the Lie algebra.
    \end{property}

    \newdef{Weyl group}{\index{Weyl!group}
        For every simple root $\lambda$ we construct the Householder transformation\footnote{See definition \ref{linalgebra:householder_transformation}.} $\sigma_\lambda$ as follows:
        \begin{gather}
            \sigma_\lambda:\text{span}_{\mathbb{R}}(\Delta)\rightarrow\text{span}_{\mathbb{R}}(\Delta):\mu\mapsto \mu - 2\frac{\langle\lambda,\mu\rangle}{\langle\lambda,\lambda\rangle}\lambda.
        \end{gather}
         The Weyl group $W$ is then defined as the group generated by all the $\sigma_\lambda$'s.
    }
    \begin{property}
        Every root $\phi\in\Phi$ can be written as $\phi = \sigma(\mu)$ for some $\mu\in\Delta$ and $\sigma\in W$. Furthermore, the root system $\Phi$ is closed under the action of $W$. In particular we obtain that the Weyl group $W$ is exactly the symmetry group of the root system $\Phi$ and the isometry group of the Killing form (and its dual).
    \end{property}

    \newdef{Coroot}{\index{coroot}
        Consider the real span $\mathfrak{h}_0^*$ of the roots of $\mathfrak{g}$. Using the restriction of the (dual) Killing form to this real subspace one can construct a dual space $\mathfrak{h}_0\subset\mathfrak{h}$. The coroot $\alpha^\vee\in\mathfrak{h}_0$ associated to a root $\alpha$ is then defined by the following formula:
        \begin{gather}
            \alpha^\vee := 2\frac{\langle\alpha,\cdot\rangle}{\langle\alpha,\alpha\rangle}.
        \end{gather}
        With this definition the Weyl transformations can be rewritten as follows:
        \begin{gather}
            \sigma_\lambda:\text{span}_{\mathbb{R}}(\Delta)\rightarrow\text{span}_{\mathbb{R}}(\Delta):\mu\mapsto \mu - \mu(\lambda^\vee)\lambda.
        \end{gather}
        In the remainder of this chapter we will often implicitly use this identification between $\mathfrak{h}_0$ and $\mathfrak{h}_0^*$.
    }
    \begin{notation}[Coroot]
        Sometimes it is more favourable to denote the coroot associated to $\alpha$ by $H^\alpha$. We will adopt this convention in the remainder of this chapter.
    \end{notation}

       \newdef{Weyl chamber}{\index{Weyl!chamber}\index{weight}
        Given a choice of positive roots $\Phi^+$ one defines the closed (fundamental) Weyl chamber associated to this ordering as the subset $\mathcal{W}\subset\mathfrak{h}_0$ which contains the elements $w$ satisfying the following equation for all $\gamma\in\Phi^+$:
        \begin{gather}
            w(H^\gamma)\geq0.
        \end{gather}
        Elements of this Weyl chamber are called \textbf{dominant weights}.
    }
    \begin{property}
        The Weyl group acts transitively on the set of Weyl chambers and accordingly on the orderings of the root system.
    \end{property}

    \begin{property}
        Let $\alpha\in\Phi$ be a root. Choose a generating element $E^\alpha$ of the weight space $\mathfrak{g}_\alpha$ associated to $\alpha$ and let $F^\alpha$ be the generator of the weight space $\mathfrak{g}_{-\alpha}$ such that $\{E^\alpha, F^\alpha, [E^\alpha, F^\alpha]\}$ defines a one-dimensional simple Lie algebra. Then the following relations hold (for $\beta\neq\pm\alpha$):
        \begin{itemize}
            \item $\beta(H^\alpha) = 2\frac{\langle\alpha, \beta\rangle}{\langle\alpha, \alpha\rangle}\in\mathbb{Z}$
            \item $[H^\alpha, E^\alpha] = \alpha(H^\alpha)E^\alpha = 2E^\alpha$
            \item $[H^\alpha, F^\alpha] = -\alpha(H^\alpha)F^\alpha = -2F^\alpha$
        \end{itemize}
        where the inner product $\langle\cdot,\cdot\rangle$ is given by the dual Killing form.\footnote{Consider the \textit{sharp} map \ref{manifolds:sharp_map} where one replaces the metric $g$ by the Killing form $K$. The dual Killing form $K^*$ is then a proper inner product (when restricted to the real span of $\Delta$) defined as: \[K^*(\cdot, \cdot) = K(\cdot^\sharp, \cdot^\sharp)\]}
    \end{property}

    \newdef{Cartan matrix}{\index{Cartan!matrix}
        Let $\lambda_i, \lambda_j\in\Delta$ be simple roots. Because the Weyl group is the symmetry group of the root system \[\sigma_{\lambda_i}(\lambda_j) = \lambda_j - 2\frac{\langle\lambda_i,\lambda_j\rangle}{\langle\lambda_i,\lambda_i\rangle}\lambda_i\] is a root. From the properties above it then follows that the quantity
        \begin{gather}
            C_{ij} := 2\frac{\langle\lambda_i,\lambda_j\rangle}{\langle\lambda_i,\lambda_i\rangle} = \lambda_j(H^{\lambda_i})
        \end{gather}
        is an integer. The matrix formed by these numbers is called the Cartan matrix.
    }
    \begin{property}\label{lie:cartan_prop}
        The Cartan matrix $C_{ij}$ is an integral matrix with the following properties:
        \begin{itemize}
            \item $C_{ii} = 2$
            \item $C_{ij} \leq 0$ if $i\neq j$
            \item $C_{ij} = 0\iff C_{ji} = 0$
        \end{itemize}
        This last property however does not imply that the Cartan matrix is symmetric. The fact that it is not symmetric can immediately be seen from its definition. However it is \textit{symmetrizable}, i.e. there exist a positive diagonal matrix $D$ and a symmetric matrix $S$ such that $C = DS$. Furthermore, one also finds that $C$ is positive definite.
    \end{property}

    \newdef{Bond number}{
        For all indices $i\neq j$ the bond number $n_{ij}$ is defined as follows:
        \begin{gather}
            n_{ij} := C_{ij}C_{ji}.
        \end{gather}
        Using the definition of the coefficients $C_{ij}$ we see that $n_{ij}$ is an integer equal to $4\cos^2\sphericalangle(\lambda_i, \lambda_j)$. This implies that $n_{ij}$ can only take on the values $0, 1, 2, 3$.\footnote{The value 4 would only be possible if the angle between $\lambda_i$ and $\lambda_j$ is $0$ but this can only occur in the case that $i=j$, which was excluded from the definition.}
    }
    \begin{remark}
        In the case of $n_{ij} = 2$ or $n_{ij} = 3$ there arise two possibilities. Namely that $C_{ij}>C_{ji}$ or $C_{ij}<C_{ji}$. From the definition of the Cartan integers and the symmetry of the dual Killing form these cases correspond to $\langle\lambda_i, \lambda_i\rangle<\langle\lambda_j, \lambda_j\rangle$ and $\langle\lambda_i, \lambda_i\rangle>\langle\lambda_j, \lambda_j\rangle$.
    \end{remark}

    \begin{construct}[Dynkin diagram]\index{Dynkin diagram}\label{lie:construct_dynkin}
        For a semisimple Lie algebra $\mathfrak{g}$ with simple root set $\Delta$ one can draw a so-called Dynkin diagram by using the following rules:
        \begin{enumerate}
            \item For every simple root $\lambda\in\Delta$ draw a circle $\bigcirc$.
            \item Draw $n_{ij}$ lines between the circles associated to $\lambda_i$ and $\lambda_j$.
            \item If $n_{ij} = 2$ or $n_{ij} = 3$, add a $<$ or $>$ sign to relate the roots based on their lengths (see previous remark).
        \end{enumerate}
    \end{construct}

    \begin{property}
        The Dynkin diagrams can be classified as follows (for every type the first three examples are given):
        \begin{itemize}
            \item A$_n$: \begin{center}\dynk \dynkA{2} \dynkA{3} \end{center}
        \item B$_n, n\geq2$: \begin{center}\dynkB{2} \dynkB{3} \dynkB{4} \end{center}
        \item C$_n, n\geq2$: \begin{center}\dynkC{2} \dynkC{3} \dynkC{4} \end{center}
        \item D$_n, n\geq4$: \begin{center}\dynkD{4} \dynkD{5} \dynkD{6} \end{center}
        \end{itemize}
        These are the only possible diagrams for simple Lie algebras.\footnote{With exception of $E_6, E_7, E_8, F_4$ and $G_2$, the so-called \textit{exceptional Lie algebras}.}
    \end{property}

    \begin{example}[Special linear group]
        By looking at the Lie brackets in \ref{lie:sl2c_lie_brackets} we see that the one-element set $\{X_1\}$ forms a Cartan subalgebra of $\mathfrak{sl}(2, \mathbb{C})$. From \ref{lie:sl2c_lie_brackets} it is also immediately clear that the simple root set $\Delta$ is given by the one-element set $\{\lambda\in\mathfrak{sl}^*(2, \mathbb{C}): \lambda(X_1)\mapsto 2\}$. Hence the Dynkin diagram for $\mathfrak{sl}(2, \mathbb{C})$ is $A_1$.
    \end{example}

    \begin{theorem}[Cartan \& Killing]
        Every finite-dimensional simple Lie algebra (over $\mathbb{C}$) can be reconstructed from its set of simple roots $\Delta$.
    \end{theorem}
    \begin{construct}[Chevalley-Serre]\index{Chevalley-Serre}\label{lie:reconstruction}
        Given a Dynkin diagram (of a simple Lie algebra) one can reconstruct the original Lie algebra $\mathfrak{g}$ (over $\mathbb{C})$ up to isomorphism. The number of nodes is equal to the number of simple roots and hence gives us the rank of $\mathfrak{g}$.

        Let $n$ denote the rank. We first construct the free Lie algebra on $3n$ generators $\{E_i, F_i, H_i\}_{i\leq n}$. The Cartan subalgebra $\mathfrak{h}\leq\mathfrak{g}$ is constructed from the generators $H_i$ by imposing the following relations:
        \begin{itemize}
            \item $[H_i, H_j] = 0$
            \item $[H_i, E_j] = a_{ij}E_j$
            \item $[H_i, F_j] = -a_{ij}F_j$
            \item $[E_i, F_j] = \delta_{ij}H_j$
        \end{itemize}
        where the numbers $a_{ij}$ form the Cartan matrix obtained by reversing construction \ref{lie:construct_dynkin}. To complete the reconstruction one imposes following additional constraints:
        \begin{itemize}
            \item $\text{ad}_{E_i}^{|a_{ij}|+1}(E_j) = 0$
            \item $\text{ad}_{F_i}^{|a_{ij}|+1}(F_j) = 0$
        \end{itemize}
        The first set of relations are called the \textbf{Chevalley} relations and the last two are called the \textbf{Serre} relations. The full construction is called the \textbf{Chevalley-Serre presentation}.
    \end{construct}
    \remark{For composite diagrams (semisimple Lie algebras) one first constructs the Lie algebra corresponding to every simple diagram and then takes the direct sum.}

\subsection{Highest weight theory}\index{weight}

    For this section we recall definition \ref{lie:weight_space} of weight spaces and weight vectors.

    \newdef{Algebraically integral}{
        An element $H\in\mathfrak{h}_0$ is said to be algebraically integral if its value on every root is an integer. The set of all algebraically integral elements is called the \textbf{weight lattice}.
    }
    \newdef{Fundamental weight}{
        Let $\Delta=\{\alpha_i\}$ be the set of simple roots. The fundamental weights $\{\omega_i\}_{i\leq|\Delta|}$ are defined as the elements of $\mathfrak{h}_0^*$ for which the following formula is satisfied for all $i,j\leq|\Delta|$:
        \begin{gather}
            \omega_i(H^{\alpha_j})=\delta_{ij}.
        \end{gather}
        Hence an element $\lambda\in\mathfrak{h}_0$ is algebraically integral if it is an integral combination of fundamental weights.
    }

    \newdef{Ordering of weights}{
        Let $\Phi^+$ be a choice of positive roots. We can define a partial ordering on the set of weights $\mathfrak{h}_0$ in the following way:
        \begin{gather}
            \lambda\geq\mu\iff\lambda-\mu\in\text{span}_{\mathbb{N}}\left(\Phi^+\right).
        \end{gather}
    }

    \newdef{Highest weight vector}{
        Consider a representation $V$ of a Lie algebra $\mathfrak{g}$. An element $v\in V$ is said to be a highest weight vector if it is a weight vector which is annihilated by all positive roots. A highest weight module is a weight module which is generated by a highest weight vector.
    }

    \begin{theorem}[Highest weight theorem]
        Let $\mathfrak{g}$ be a finite-dimensional Lie algebra. The following statements hold:
        \begin{itemize}
            \item Every finite-dimensional irreducible representation of $\mathfrak{g}$ has a unique dominant integral highest weight.
            \item If two irreducible representations have the same highest weight then they are isomorphic.
            \item Every dominant integral weight is the highest weight of an irreducible finite-dimensional representation.
        \end{itemize}
    \end{theorem}

\subsection{Kac-Moody algebras}

    \newdef{Kac-Moody algebra}{\index{Kac-Moody algebra}\index{Cartan!matrix}\label{lie:kac_moody}
        Consider the Cartan matrix $A$ associated to a finite-dimensional (semi)simple Lie algebra. This matrix has the properties listed in \ref{lie:cartan_prop}. If one only retains the first three points from \ref{lie:cartan_prop} then one obtains a \textbf{generalized Cartan matrix}. Given such a generalized Cartan matrix $A'$ one can construct a (possibly infinite-dimensional) Lie algebra using an analog of the Chevalley-Serre relations.

        Since a generalized Cartan matrix might have $\det(A)=0$ one cannot use the Chevalley-Serre construction as such, because the constructed roots might be linearly dependent. However this problem can be easily solved: Let $A$ be an $n\times n$ (generalized) Cartan matrix. First we construct a complex vector space $\mathfrak{h}$ of dimension $2n-\rk(A)$ and choose for every $i\leq n$ a simple root $\alpha_i$ (resp. coroot $H^{\alpha_i}$) in $\mathfrak{h}^*$ (resp. $\mathfrak{h}$) such that these are linearly independent under the condition $\alpha_i(H^{\alpha_j})=A_{ij}$.\footnote{This construction is always possible and it is unique up to isomorphism.} Now we construct the free Lie algebra $\tilde{\mathfrak{g}}$ on $4n-\rk(A)$ generators $\{E_i, F_i\}_{i\leq n}\cup\mathfrak{h}$ and quotient out by the following relations:
        \begin{itemize}
            \item $[E_i, F_j] = \delta_{ij}H^{\alpha_i}$
            \item $[H, H']=0$ for $H, H'\in\mathfrak{h}$
            \item $[H, E_i]=\alpha_i(H)E_i$ for $H\in\mathfrak{h}$
            \item $[H, F_i]=-\alpha_i(H)F_i$ for $H\in\mathfrak{h}$
        \end{itemize}
        Given this algebra one can find the unique maximal ideal $\mathfrak{m}\leq\tilde{\mathfrak{g}}$ for which $\mathfrak{m}\cap\mathfrak{h}=\{0\}$. The quotient algebra $\tilde{\mathfrak{g}}/\mathfrak{m}$ is called the Kac-Moody algebra associated to $A$.\footnote{Kac has proven that for symmetrizable $A$ this construction is equivalent to a construction through the same generators but with the Chevalley-Serre presentation.} Although we did not add an analogon of the Serre relations in the construction it can be shown that these are valid (see for example \cite{aminiinfinite}).
    }
    \begin{remark}
        Assume that the generalized Cartan matrix $A$ is still symmetrizable. Then one can have three distinct classes of Kac-Moody algebras based on the definiteness of $A$:
        \begin{itemize}
            \item If $A$ is positive definite then one obtains a finite-dimensional (semi)simple Lie algebra.
            \item If $A$ is positive semi-definite\footnote{It can be shown for a generalized Cartan matrix $A$ that affinity is equivalent to the existence of a unique real vector $v$ (up to scaling) such that $Av=0$.} then one obtains a Kac-Moody algebra of affine type.
            \item If $A$ is indefinite then one obtains a Kac-Moody of indefinite type.
        \end{itemize}
    \end{remark}

    \begin{construct}[Loop algebra]\index{loop!algebra}
    Consider a finite-dimensional Lie algebra $\mathfrak{g}$. From this Lie algebra one can construct the so-called \textbf{loop algebra} $L\mathfrak{g}$ (sometimes denoted by $\mathfrak{g}[t, t^{-1}]$). The underlying vector space is given by $\mathfrak{g}\otimes\mathbb{C}[t, t^{-1}]$ and the Lie bracket is given by $[g\otimes t^k, g'\otimes t^l] = [g, g']_{\mathfrak{g}}\otimes t^{k+l}$ (extended by linearity).

    Equivalently one can obtain the loop algebra as the space of polynomial maps from $S^1$ to $\mathfrak{g}$ (hence the name). If $G$ is the Lie group associated to $\mathfrak{g}$ and $LG$ its (free) loop group\footnote{See property \ref{topology:loop_group}.}, then $LG$ has the natural structure of a (infinite-dimensional) Lie group and its Lie algebra is exactly given by $L\mathfrak{g}$.
    \end{construct}

    \begin{construct}[Affine Lie algebra]
        Given a simple Lie algebra $\mathfrak{g}$ one constructs the affine Lie algebra $\hat{\mathfrak{g}}$ as the central extension of the loop algebra $L\mathfrak{g}$ by $\mathbb{C}$ associated to the cocycle \[\Theta:(g\otimes t^k, g'\otimes t^l)\mapsto kK(g, g')\delta_{k+l, 0}\] where $K(\cdot, \cdot)$ is the Killing form on $\mathfrak{g}$. This cocycle can also be defined using the residue\footnote{See definition \ref{complexcalculus:residue_def}.} of a Laurent polynomial:
        \begin{gather}
            \Theta(f, g) := \text{Res}\left[K(f, g)\right]_t
        \end{gather}
        where we extended the Killing form on $\mathfrak{g}$ to $L\mathfrak{g}$ by $K(a\otimes t^k, b\otimes t^l)=K(a, b)t^{k+l}$.

        However to obtain a well-behaved affine Kac-Moody algebra one needs to extend this affine Lie algebra by a derivation. First we observe that the loop algebra and accordingly the affine Lie algebra is $\mathbb{Z}$-graded. A well-defined derivation is then obtained through multiplication by the grading. To this intent we add a formal generator $d$ together with the following relations (the central element from the previous step will be denoted by $c$):
        \begin{align}
            [d, g\otimes P(t)] &= g\otimes t\deriv{}{t}P(t)\\
            [d, c] &= 0
        \end{align}
        where $P(t)\in\mathbb{C}[t, t^{-1}]$.
    \end{construct}
    \newdef{Twisted Kac-Moody algebra}{
        Given a simple Lie algebra $\mathfrak{g}$ with associated Cartan matrix $A$ one can construct an affine Cartan matrix $\hat{A}$, called the \textit{extended} Cartan matrix. It can then be shown that the affine Kac-Moody algebra associated to $\hat{A}$, as defined in \ref{lie:kac_moody}, is isomorphic to the affine Kac-Moody algebra as constructed above starting from the simple Lie algebra $\mathfrak{g}$.

        All affine Kac-Moody algebra which are isomorphic to algebras defined in this way are said to be \textbf{untwisted}. All other affine Kac-Moody algebras are said to be \textbf{twisted}.
    }

\subsection{Universal enveloping algebra}

    \newdef{Universal enveloping algebra}{\index{universal!enveloping algebra}
        \nomenclature[S_Ug]{$U(\mathfrak{g})$}{Universal enveloping algebra of a Lie algebra $\mathfrak{g}$.}
        Let $(\mathfrak{g}, [\cdot, \cdot])$ be a Lie algebra and consider its tensor algebra $T(\mathfrak{g})$. The universal enveloping algebra $U(\mathfrak{g})$ is defined as the quotient of $T(\mathfrak{g})$ by the two-sided ideal generated by the elements $g\otimes h - h\otimes g - [g, h]$, where $g, h$ range over $\mathfrak{g}$.
    }

    \begin{construct}\label{lie:uea_construct}
        If one considers Chevalley-Serre presentation from the reconstruction theorem \ref{lie:reconstruction} as an (unital associative) algebra presentation instead of a Lie algebra presentation\footnote{Hence by replacing the Lie bracket by the commutator constructed from the formal multiplication in $U(\mathfrak{g})$.} then one obtains the universal enveloping algebra $U(\mathfrak{g})$ of $\mathfrak{g}$.
    \end{construct}

    \begin{theorem}[Poincar\'e-Birkhoff-Witt]\index{Poincar\'e-Birkhoff-Witt}
        Let $\mathfrak{g}$ be a Lie algebra with a totally ordered basis $\{g_i\}$. The monomials of the form $g_1^{m_1}g_2^{m_2}\cdots g_N^{m_N}$ constitute a basis for $U(\mathfrak{g})$.
    \end{theorem}

    \newdef{Casimir invariant\footnotemark}{\index{Casimir!invariant}\label{lie:casimir_invariant}
        \footnotetext{Also known as a \textbf{Casimir operator} or \textbf{Casimir element}.}
        Let $\mathfrak{g}$ be a Lie algebra. A Casimir invariant $J$ is an element of the center of $U(\mathfrak{g})$.
    }

    \newformula{Quadratic Casimir invariant}{
        Consider a Lie algebra representation $\rho:\mathfrak{g}\rightarrow\text{End}(V)$ on an $n$-dimensional vector space $V$. Let $\{X_i\}_{i\leq n}$ be a basis for $\mathfrak{g}$. The (quadratic) Casimir invariant associated with $\rho$ is given by
        \begin{gather}
            \Omega_\rho = \sum_{i=0}^n\rho(X_i)\circ\rho(\xi_i)
        \end{gather}
        where the set $\{\xi_i\}_{i\leq n}$ is defined by the relation $K_\rho(X_i, \xi_j) = \delta_{ij}$ using the Killing form \ref{lie:rho_killing_form}.
    }

    \begin{property}
        When the representation $\rho:\mathfrak{g}\rightarrow\text{End}(V)$ is irreducible, Schur's lemma \ref{rep:schurs_lemma} tells us that
        \begin{gather}
            \Omega_\rho = c_\rho\mathbbm{1}_V.
        \end{gather}
        By taking the trace of this formula and using formula \ref{lie:rho_killing_form} we see that $c_\rho = \frac{\dim\mathfrak{g}}{\dim V}$.
    \end{property}

    \newdef{Verma module}{\index{Verma module}\label{lie:verma_module}
        Consider a finite-dimensional Lie algebra $\mathfrak{g}$ with Borel subalgebra $\mathfrak{b}$. The Verma module with highest weight $\lambda$ is defined as follows\footnote{This can be seen as an "extension of scalars" procedure where we turn a $U(\mathfrak{b})$-module in a $U(\mathfrak{g})$-module.}:
        \begin{gather}
            V(\lambda) := U(\mathfrak{g})\otimes_{U(\mathfrak{b})}\mathbb{C}_\lambda
        \end{gather}
        where $\mathbb{C}_\lambda$ is the one-dimensional left $\mathfrak{b}$-module where the Cartan subalgebra acts by weight $\lambda$ and $\mathfrak{n}_+\subset\mathfrak{b}$ acts trivially. $U(\mathfrak{g})$ contains $U(\mathfrak{b})$ as a subalgebra by the PBW theorem and hence is a right $U(\mathfrak{b})$-module through right multiplication. Since $U(\mathfrak{g})$ is trivially a left module over itself, the Verma module also becomes a left $U(\mathfrak{g})$-module.
    }
    \begin{adefinition}
        The Verma module with highest weight $\lambda$ can also be defined using a quotient construction. Let $I_\lambda\subset U(\mathfrak{g})$ be the left ideal generated by the following elements\footnote{These relations exactly give the conditions for a highest weight vector.}:
        \begin{itemize}
            \item $X_\alpha\in\mathfrak{g_\alpha}$ for all positive roots $\alpha$
            \item $H-\lambda(H)\mathbf{1}$ for all $H\in\mathfrak{h}$
        \end{itemize}
        The Verma module $V(\lambda)$ is then defined as the quotient $U(\mathfrak{g})/I_\lambda$.
    \end{adefinition}

    The importance of Verma modules is given by the following property:
    \begin{property}
        The Verma module $V(\lambda)$ is a highest weight module with highest weight vector $\mathbf{1}\otimes\mathbf{1}$ (here we used the first definition of Verma modules). Furthermore, every highest weight module with highest weight $\lambda$ is a quotient of the Verma module $V(\lambda)$.
    \end{property}

    \begin{property}
         A basis for $V(\lambda)$ (let the highest weight vector be $v_\lambda$) is given by the monomials $F_{\alpha_1}^{m_1}F_{\alpha_2}^{m_2}\cdots F_{\alpha_N}^{m_N}v_\lambda$ where $\alpha_i$ are negative roots, $m_i\in\mathbb{N}$ and $F_{\alpha_i}\in\mathfrak{g}_{\alpha_i}$.
    \end{property}

\subsection{Group contractions}

    \newdef{In\"on\"u-Wigner contraction}{\index{In\"on\"u-Wigner}
        Consider an $n$-dimensional Lie group $G$ with Lie algebra $\mathfrak{g}$. Choose a basis $\{e_i\}_{i\leq n}$ for $\mathfrak{g}$. A nonsingular transformation of the basis would leave the structure of the group unchanged. However one can rewrite this nonsingular transformation in terms of a singular transformation:
        \[U = u + \varepsilon w.\]
        The group contraction is obtained by taking the limit $\varepsilon\rightarrow0$. In terms of the structure constants this is equivalent to setting a set of structure constants to zero such that one obtains a smaller subalgebra (and associated subgroup). It can then be shown that there exists a bijection between continuous subgroups and group contractions. The Lie algebra elements belonging to the contracted subalgebra form an Abelian invariant subalgebra and hence generate an Abelian invariant subgroup. The group contraction $\tilde{G}$ is obtained as the factor group of $G$ with respect to this Abelian subgroup.
    }

    \begin{example}[Galilei group]
        The Galilei group in $d$ dimensions can be obtained as a group contraction of the inhomogeneous Lorentz group in $d+1$ (spacetime) dimensions with respect to time displacements and spatial rotations.
    \end{example}

\subsection{Lie algebra cohomology}

    \newdef{Chevalley-Eilenberg algebra}{\index{Chevalley-Eilenberg algebra}
        Let $\mathfrak{g}$ be a finite-dimensional\footnote{This construction can be generalized almost verbatim to the infinite-dimensional case.} Lie algebra. Consider a basis $\{t_a\}_{a\leq n}$ of $\mathfrak{g}$ and let $\{t^a\}_{a\leq n}$ be its linear dual. The Chevalley-Eilenberg algebra CE$(\mathfrak{g})$ is defined the Grassmann algebra $\Lambda^\bullet\mathfrak{g}^*$ with a dg-algebra structure induced by the differential\footnote{In section \ref{section:bv_formalism} we explain how this differential is obtained as the dual of the Lie bracket.}
        \begin{gather}
            dt^a:=-\frac{1}{2}C_{bc}^{\ \ a}t^b\wedge t^c
        \end{gather}
        where $C_{bc}^{\ \ a}$ are the structure constants of $\mathfrak{g}$.
    }

    By analogy to the case of group (co)homology we define the (co)homology of a Lie algebra using the Tor en Ext functors associated to a certain category of rings and modules. The natural choice of ring in the case of Lie algebras is the universal envelopping algebra $U(\mathfrak{g})$. The tensor and Hom operations underlying the construction are with respect to the trivial $U(\mathfrak{g})$-module $k$ (the underlying field of the Lie algebra). This then gives:
    \begin{align}
        H^i_{\text{Lie}}(k;M) &:= \text{Ext}^i_{U(\mathfrak{g})}(k,M)\\
        H_i^{\text{Lie}}(k;M) &:= \text{Tor}_i^{U(\mathfrak{g})}(k,M)
    \end{align}
    where $M$ is a $\mathfrak{g}$-module and as such also a $U(\mathfrak{g})$-module.

    For the remainder of this section we will focus on cohomology. The chapter on homological algebra now tells us that we have to find a projective resolution of $k$ to determine the Ext-functor in terms of Hom-sets Hom$(\mathcal{F}, M)$ (see section \ref{section:tor_ext}). It can be shown that the relevant object is given by the tensor product $U(\mathfrak{g})\otimes\Lambda^\bullet\mathfrak{g}$:
    \begin{gather}
        \text{Ext}^i_{U(\mathfrak{g})}(k, M) = H^i(\text{Hom}_{\mathfrak{g}}(U(\mathfrak{g)}\otimes_k\Lambda^\bullet\mathfrak{g}, M)) \cong H^i(\text{Hom}_k(\Lambda^\bullet\mathfrak{g}, M))
    \end{gather}
    where the differential of the (middle) complex is given by
    \begin{gather}
        d(u\otimes g_1\wedge\cdots\wedge g_n) := \sum_i(-1)^{i+1}ug_i\otimes(g_1\wedge\cdots\wedge\hat{g}_i\wedge\cdots\wedge g_n)\\+ \sum_{i<j}(-1)^{i+j}u\otimes[g_i,g_j]\wedge\cdots\wedge\hat{g}_i\wedge\cdots\wedge\hat{g}_j\wedge\cdots\wedge g_n.\nonumber
    \end{gather}
    In the case $M=k$ the Hom-complex can easily be seen to be the Chevalley-Eilenberg algebra CE$(\mathfrak{g})=\Lambda^\bullet\mathfrak{g}^*$. Using a change of coefficients the general case is shown to be isomorphic to $M\otimes\Lambda^\bullet\mathfrak{g}^*$ where the differential on $U(\mathfrak{g})\otimes\text{CE}(\mathfrak{g})$ gets extended by the extra term $(-1)^n dm\otimes (u\otimes g_1\wedge\cdots\wedge g_n)$ where $dm(g)$ is given by $g\cdot m$.

    \newdef{Weil algebra}{\index{Weil!algebra}
        Consider a Lie algebra $\mathfrak{g}$. Its Weil algebra is defined as the dg-algebra $\Lambda^\bullet(\mathfrak{g}^*\oplus\mathfrak{g}^*[1])$ with the diferential $d_W=d_{CE}+\mathbf{d}$ where $d_{CE}$ is the differential on the Chevalley-Eilenberg subalgebra $CE(\mathfrak{g})\subset W(\mathfrak{g})$ and $\mathbf{d}$ shifts the degree of $\mathfrak{g}^*$ by $1$ and acts as 0 on $\mathfrak{g}^*[1]$. The action of $d_{CE}$ to shifted generators is defined using graded commutativity, i.e. $[\mathbf{d}, d_{CE}]=0$.
    }
    \newdef{Horizontal elements}{\index{horizontal!elements}
        The elements of the subalgebra $\Lambda^\bullet\mathfrak{g}^*[1]$ are sometimes called the \textbf{horizontal elements}.
    }

    \sremark{From here one we shall drop the subscript and denote the differential of the Weil algebra by $d$.}

    It is clear that we have a short exact sequence
    \begin{gather}
        0\rightarrow\ker(p)\rightarrow W(\mathfrak{g})\overset{p}{\twoheadrightarrow} CE(\mathfrak{g})\rightarrow0
    \end{gather}
    where $p$ is the obvious projection map. An important subset of $\mathfrak{h}$ is given by the algebra of invariant polynomials inv$(\mathfrak{g})$:
    \newdef{Invariant polynomial}{\index{invariant!polynomial}
        Horizontal element $\omega$ for which $d\omega$ is also horizontal. (Sometimes one replaces the horizontality condition by $d\omega = 0$.)

        It should be noted that although this definition might seem complicated it is (for ordinary Lie algebras\footnote{The above definition leads to an easy generalization in the context of $L_\infty$-algebras.}) equivalent to the ordinary definition in terms of $ad$-invariant polynomials.
    }
    \newadef{Invariant polynomial}{
        Let $G$ be a Lie group with Lie algebra $\mathfrak{g}$. A polynomial $P\in K[\mathfrak{g}]$, where $K$ is the base field, is said to be invariant (or sometimes $Ad$-invariant) if
        \begin{gather}
        P(X) = P(gXg^{-1})
        \end{gather}
        for all $X\in\mathfrak{g}$ and $g\in G$. This subalgebra of $K[\mathfrak{g}]$ is denoted by $K[\mathfrak{g}]^G$.
    }

    A concept that will be important later on for the study of characteristic classes on fibre bundles is the transgression map:
    \newdef{Transgression}{\index{transgression}
        The exact sequence above induces a long exact sequence in cohomology and an invariant polynomial is set to be in transgression with a cocycle in $CE(\mathfrak{g})$ if their cohomology classes are related by the connecting homomorphism. More explicitly, by definition of invariant polynomials we have $d\omega = 0$ and since $W(\mathfrak{g})$ has vanishing cohomology there exists an element $c_\omega$ such that $\omega=dc_\omega$. By restricting $c_\omega$ to $CE(\mathfrak{g})$ we obtain a $\mathfrak{g}$-cocycle since $d_{CE}c_\omega=0$.
    }

    \begin{example}[Killing form]\label{lie:killing_transgression}
        Consider the invariant polynomial $\langle\cdot,\cdot\rangle$ induced by the Killing form on a semisimple Lie algebra. By transgression one obtains the canonical 3-cocycle $\langle\cdot, [\cdot,\cdot]\rangle$.
    \end{example}

\section{Poisson algebras and Lie superalgebras}

    \newdef{Internal Lie algebra}{\index{Lie!algebra}
        Let $(\textbf{C}, \otimes, \mathbf{1})$ be a symmetric monoidal category with braiding $\sigma$. A Lie algebra internal to \textbf{C} is an object $A\in\ob{C}$ and a morphism \[[\cdot, \cdot]:A\otimes A\rightarrow A\] satisyfing the following conditions:
        \begin{enumerate}
            \item \textbf{Antisymmetry}: $[\cdot, \cdot] + [\cdot, \cdot]\circ\sigma_{A, A} = 0$
            \item \textbf{Jacobi identity}: $[\cdot, [\cdot, \cdot]] + [\cdot, [\cdot, \cdot]]\circ\tau + [\cdot, [\cdot, \cdot]]\circ\tau^2= 0$
        \end{enumerate}
        where $\tau = (\mathbbm{1}\otimes\sigma_{A, A})\circ(\sigma_{A, A}\otimes\mathbbm{1})$ denotes cyclic permutation.
    }
    \begin{example}[Lie superalgebra]
        If we use the braiding $\sigma(a\otimes b) = (-1)^{|a||b|}b\otimes a$ in the category \textbf{sVect} we obtain the notion of a Lie superalgebra.
    \end{example}

    \newdef{Poisson algebra}{\index{Poisson!algebra}\index{Poisson!bracket}\label{lie:poisson_algebra}
        Let $V$ be a vector space equipped with two bilinear operations $\star$ and $\{\cdot, \cdot\}$ that satisfy the following conditions:
        \begin{enumerate}
            \item The couple $(V, \star)$ is an associative algebra.
            \item The couple $(V, \{\cdot, \cdot\})$ is a Lie algebra.
            \item The \textbf{Poisson bracket} $\{\cdot,\cdot\}$ acts as a derivation\footnote{See definition \ref{manifolds:derivation}.} with respect to the operation $\star$, i.e. \[\{x, y\star z\} = \{x, y\}\star z + y\star\{x, z\}\]
        \end{enumerate}
    }

\subsection{Batalin-Vilkovisky formalism}\label{section:bv_formalism}

    The introduction to this section is based on lecture notes by D. Fiorenza \cite{bv_formalism}.

    One can also obtain a different generalization of Lie algebras by rewriting it in terms of superspaces. Using the content of section \ref{section:graded_spaces} one can regard the Lie bracket as a linear map on the \textit{odd} symmetric algebra:
    \begin{align*}
        [\cdot,\cdot]&:\mathfrak{g}\wedge\mathfrak{g}\rightarrow\mathfrak{g}\\
        &\quad\ \rotatebox[origin=c]{90}{$\xleftarrow{\qquad}$}=\\
        &\text{Sym}^2(\Pi\mathfrak{g})\rightarrow\Pi\mathfrak{g}
    \end{align*}
    By dualizing and extending through the (graded) Leibniz rule one obtains a (degree 1) derivation \[\delta:\text{Sym}^\bullet(\Pi\mathfrak{g}^*)\rightarrow\text{Sym}^\bullet(\Pi\mathfrak{g}^*)\] on the space of regular functionals $\mathcal{F}(\mathfrak{g}):=\text{Sym}^\bullet(\Pi\mathfrak{g}^*)$. Now it can be shown that the Jacobi identity on $\mathfrak{g}$ is equivalent to $\delta$ being a (co)differential, i.e. $\delta^2\equiv0$.

    \newadef{Lie algebra}{\index{Lie!algebra}
        A vector space $\mathfrak{g}$ equipped with a degree 1 derivation which is also a differential on the function space $\mathcal{F}(\mathfrak{g})$.
    }

    By dropping the condition that the differential is of degree 1 we obtain the following notion:    A (graded) vector space $V$ equipped with a derivation which is also a codifferential on the function space $\mathcal{F}(V)$.

    By the (graded) Leibniz rule the derivation $\delta$ is completely defined if we know its restriction to the subspace $\Pi V^*\leq\mathcal{F}(V)$. For every $n\in\mathbb{N}$ one can construct the projection of $\delta$ onto the $n^{th}$ symmetric power: \[\delta|_i:\Pi V^*\rightarrow\text{Sym}^n(\Pi V^*).\] These morphisms have degree $n-2$. The relation $\delta^2=0$ implies a list of (quadratic) relations on the differentials $\delta_n$:
    \begin{align*}
        \delta_1^2 &= 0\\
        \delta_1\delta_2+\delta_2\delta_1 &= 0\\
        \delta_1\delta_3+\delta_2^2+\delta_3\delta_1 &= 0\\
        &\ \ \vdots
    \end{align*}
    By dualizing we recover some Lie algebra-like structures (we put $[\cdot]_1:=d$):
    \begin{align*}
        d^2&=0\\
        d[\cdot,\cdot]_2 &= [d\cdot,\cdot]_2+[\cdot,d\cdot]_2\\
        [[v_1,v_2],v_3]_2+\text{cyc. perm.} &= d[v_1,v_2,v_3]_3-[dv_1,v_2,v_3]_3-[v_1,dv_2,v_3]_3-[v_1,v_2,dv_3]_3\\
        &\ \ \vdots
    \end{align*}
    These relations can be interpreted as follows:
    \begin{itemize}
        \item $d$ is a differential.
        \item $d$ acts as a derivative on the binary bracket.
        \item The Jacobi identity holds up to a chain homotopy (given by the ternary bracket).
        \item The higher relations are similar to the chain homotopy for the Jacobi identity. This gives
    \end{itemize}

    \newdef{$L_\infty$-algebra\footnotemark}{\index{$L_\infty$-algebra}
        \footnotetext{Also called a \textbf{strong(ly) homotopy Lie algebra}.}
        The above idea leads to the definition of an $L_\infty$-algebra as a graded vector space $V$ equipped with a collection of morphisms $l_n:\Lambda^n(V)\rightarrow V$ of degree $n-2$ subject to the relations
        \begin{gather}
            \sum_{i+j=n+1}\sum_{\sigma\in S_n}(-1)^{i(j-1)}\chi(\sigma;v_1,...,v_n)l_i\big(l_j(v_{\sigma(1)}\wedge\cdots\wedge v_{\sigma(j)})\wedge v_{\sigma(j+1)}\wedge\cdots\wedge v_{\sigma(n)}\big)=0
        \end{gather}
        where $\chi(\sigma;v_1,...,v_n)$ denotes the antisymmetric Koszul sign.
    }

    \begin{example}[Lie algebra]\index{Lie!algebra}
        It can easily be checked that the $L_\infty$-algebra with $V$ concentrated in degree 0 is equivalent to the structure of an ordinary Lie algebra. Similarly one obtains the notion of a Lie $n$-algebra by truncating an $L_\infty$-algebra at degree $n$.
    \end{example}
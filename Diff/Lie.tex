\chapter{Lie groups and Lie algebras}\label{chapter:lie}
\section{Lie groups}
    
	\newdef{Lie group}{\index{Lie!group}\label{group:lie_group}
		A Lie group is a group that is also a differentiable manifold such that both the multiplication and inversion are smooth functions.
	}
	
	\newdef{Lie subgroup}{
		A subset of a Lie group is a Lie subgroup if it is both a subgroup and a closed submanifold.
	}
	\begin{theorem}[Closed subgroup theorem\footnotemark]
		\footnotetext{Sometimes called \textbf{Cartan's theorem}.}
		If $H$ is a closed\footnotemark\ subgroup of a Lie group $G$ then $H$ is a Lie subgroup of $G$.
		\footnotetext{With respect to the group topology on $G$.}
	\end{theorem}
	
	\begin{property}\label{lie:prop_connected}
		Let $G$ be a connected Lie group. Every neighbourhood $U_e$ of the identity $e$ generates $G$, i.e. every element $g\in G$ can be written as a word in $U_e$.
	\end{property}

\subsection{Left invariant vector fields}
	
	\newdef{Left Invariant Vector Field (LIVF)}{\index{LIVF (left invariant vector field)}
		Let $G$ be a Lie group. Let $X$ be a vector field on $G$. $X$ is left invariant if the following equivariance relation holds for all $g\in G$:
		\begin{equation}
			L_{g,\ast}X(h) = X(g\cdot h)
		\end{equation}
		where $L_g$ denotes the left action map associated with $g$.
	}
	\begin{property}
		The set $\mathcal{L}(G)$ of LIVF's on a Lie group $G$ is a vector space over $\mathbb{R}$.
	\end{property}
	\begin{property}
		\label{lie:livf_prop}
		The map $L_{g,\ast}$ is an isomorphism for every $g\in G$. It follows that a LIVF is uniquely determined by its value at the identity of $G$. Furthermore, for every $v\in T_e(G)$, there exists a LIVF $X\in\mathcal{L}(G)$ such that $X(e) = v$ and this mapping is an isomorphism from $T_e(G)$ to $\mathcal{L}(G)$.
	\end{property}

\subsection{One-parameter subgroups}

        \newdef{One-parameter subgroup}{\index{one-parameter subgroup}\label{group:one_parameter_subgroup}
        	A one-parameter (sub)group is a Lie group homomorphism $\Phi:\mathbb{R}\rightarrow G$ from the additive group of real numbers to a Lie group $G$.
        }

        \begin{property}\label{group:OPS_composition}
        	Let $\Phi:\mathbb{R}\rightarrow G$ be a one-parameter subgroup of $G$. Let $\Psi:G\rightarrow H$ be a continuous group homomorphism. Then $\Psi\circ\Phi:\mathbb{R}\rightarrow H$ is a one-parameter subgroup of $H$.
        \end{property}
        
        \begin{property}\label{lie:livf_subgroup}
        	All LIVF's $X$ are complete\footnote{See definition \ref{manifold:complete_vector_field}.}. Hence for every LIVF $X$ we can find an integral curve $\gamma^X$ with initial condition $\gamma^X(0) = e$ for which the maximal flow domain\footnotemark\ $D(X)$ is $]-\infty, +\infty[$. This implies that the associated flow $\sigma_t$ determines a one-parameter subgroup of $G$. Conversely, for every one-parameter subgroup $\phi(t)$ we can construct a LIVF $X = \phi'(0)$. This correspondence is a bijection.
        	\footnotetext{See definition \ref{manifolds:flow}.}
        \end{property}

\subsection{Cocycles}

	\newdef{Cocycle}{\index{cocycle}\label{group:cocycle}
		Let $M$ be a smooth manifold and $G$ a Lie group. A cocycle on $M$ with values in $G$ is a family of smooth functions $g_{ij}:U_i\cap U_j\rightarrow G$ that satisfy the following condition:
		\begin{equation}
			\label{group:cocycle_condition}
			g_{ij} = g_{ik}\circ g_{kj}
		\end{equation}
	}
	\begin{property}
		Let $\{g_{ij}\}_{i,j}$ be a cocycle on $M$. We have the following properties:
		\begin{itemize}
			\item $g_{ii}(x) = \mathbbm{1}_M$
			\item $g_{ij}(x) = (g_{ji}(x))^{-1}$
		\end{itemize}
		for all $x\in M$.
	\end{property}


\section{Lie algebras}
    	There are two ways to define a Lie algebra. The first one is a stand-alone definition using a vector space equipped with a multiplication operation. The second one establishes a direct relation between Lie groups (see \ref{group:lie_group}) and real Lie algebras.
        
\subsection{Definitions}

        \newdef{Lie algebra}{\index{Lie!algebra}\index{Lie!bracket}\index{Jacobi!identity}\label{linalgebra:lie_algebra}
        	Let $V$ be a vector space equipped with a binary operation $[\cdot, \cdot]:V\times V\rightarrow V$ is a Lie algebra if the Lie bracket $[\cdot, \cdot]$ satisfies the following conditions:
            \begin{enumerate}
            	\item Bilinearity: $[ax + y, z] = a[x, z] + [y, z]$
                \item Alternativity: $[v, v] = 0$
                \item Jacobi identity: $[a, [b, c]] + [b, [c, a]] + [c, [a, b]] = 0$
            \end{enumerate}
        }
        
        \newdef{Lie algebra of LIVF's}{
        	Consider the vector space $\mathcal{L}(G)$ of LIVF's on a Lie group G. Using property \ref{manifolds:lie_bracket} we can show that the commutator (Lie bracket) also defines a LIVF on $G$. It follows that $\mathcal{L}(G)$ is closed under Lie brackets and hence is also a Lie algebra.
        }
	\newdef{Lie algebra of Lie group}{
		Let $G$ be a Lie group. The tangent space $\mathfrak{g} := T_eG$ has the structure of a Lie algebra where the Lie bracket is induced by the commutator of vector fields \ref{manifolds:lie_bracket} in the following way:
		\begin{equation}
			\text{\textlbrackdbl} A, B \text{\textrbrackdbl} := \left.\left[l_{g, *}A, l_{g, *}B\right]\right|_{g=e}
		\end{equation}
		where $A, B\in T_eG$ and where $[\cdot, \cdot]$ is the Lie bracket on $\mathcal{L}(G)$. This induces an isomorphism of Lie algebras: $\mathfrak{g}\cong_{\text{Lie}}\mathcal{L}(G)$.
	}
        
        \begin{notation}
        	Lie algebras are generally denoted by fraktur symbols. For example, the Lie algebra associated with the Lie group $G$ is often denoted by $\mathfrak{g}$.
        \end{notation}
        
	\begin{theorem}[Ado]\index{Ado}\label{lie:theorem:ado}
		Every finite-dimensional Lie algebra can be embedded as a subalgebra of $\mathfrak{gl}_n$.
	\end{theorem}
	\begin{theorem}[Lie's third theorem]\index{Lie!third theorem}
		Every finite-dimensional Lie algebra $\mathfrak{g}$ is the Lie algebra of a unique simply-connected Lie group $G$.
	\end{theorem}
	
	\newdef{Lie algebra homomorphism}{\index{homomorphism!of Lie algebras}
		A map $\Phi:\mathfrak{g}\rightarrow\mathfrak{h}$ is a Lie algebra homomorphism if it satisfies following condition
		\begin{equation}
			\Phi([X, Y]) = [\Phi(X), \Phi(Y)]
		\end{equation}
		for all $X, Y\in\mathfrak{g}$.
	}
	\begin{property}[Homomorphisms theorem\footnotemark]\label{lie:prop_hom}
		\footnotetext{See also formula \ref{lie:induced_homomorphism}.}
		Let $G, H$ be Lie groups with $G$ simply-connected. If a linear map $\Phi:\mathfrak{g}\rightarrow\mathfrak{h}$ is a Lie algebra homomorphism then there exists a unique Lie group homomorphism $\phi:G\rightarrow H$ such that $\Phi = \phi_*$.\footnote{The converse is trivial: every Lie group homomorphism induces a Lie algebra homomorphism through its differential.}
	\end{property}

\subsection{Exponential map}
	\newformula{Exponential map}{\index{exponential map}
		Let $X\in\mathfrak{g}$ be a LIVF on $G$. We define the exponential map $\exp:\mathfrak{g}\rightarrow G$ as:
		\begin{equation}
			\boxed{\exp(X) := \gamma_X(1)}
		\end{equation}
		where $\gamma_X$ is the associated one-parameter subgroup defined in property \ref{lie:livf_subgroup}.
	}
	
	\begin{property}
		The exponential map is the unique map $\mathfrak{g}\rightarrow G$ such that exp$(0) = e$ and for which the restrictions to the lines through the origin in $\mathfrak{g}$ are one-parameter subgroups of $G$.
	\end{property}
	\begin{result}
		Because the identity element $\mathbbm{1}_\mathfrak{g} = (\exp_\ast)_e$ is an isomorphism, the inverse function theorem \ref{manifolds:theorem:inverse_function_theorem} implies that the image of $\exp$ will contain a neighbourhood of the identity $e\in G$. If $G$ is connected then property \ref{lie:prop_connected} implies that $\exp$ generates all of $G$.
		
		Together with the property that $\psi\circ\exp = \exp\circ\ \psi_\ast$ for every Lie group homomorphism $\psi:G\rightarrow H$ it follows that \textit{if $G$ is connected, a Lie group homomorphism $\psi:G\rightarrow H$ is completely determined by its differential $\psi_\ast$ at the identity $e\in G$}.
	\end{result}
	
	\begin{example}[Matrix Lie groups]\index{matrix!exponentiation}
		For matrix Lie groups we define the classic matrix exponential:
		\begin{equation}
			e^{tX} = \sum_{k=0}^{+\infty}\stylefrac{(tX)^k}{k!}
		\end{equation}
		This operation defines a curve $\gamma(t)$ which can be used as a one-parameter subgroup on $G$. It should be noted that this formula converges for every $X\in M_{m,n}$ and is invertible with the inverse given by $\exp(-X)$. Using Ado's theorem \ref{lie:theorem:ado} one can then use this matrix exponential to represent the exponential map for any (finite-dimensional) Lie algebra.
	\end{example}
	
	\begin{property}
		Let $G$ be a compact Lie group. The exponential map is surjective. However, because the asocciated Lie algebra $\mathfrak{g}$ is non-compact, the exponential map cannot be homeomorphic and hence cannot be injective.
	\end{property}

\subsection{Structure}  
        
        \newdef{Structure constants}{\index{structure constants}
        	As Lie algebras are closed under Lie brackets, every Lie bracket can be expanded in term of a basis $\{X_k\}_{k\in I}$ as follows:
        	\begin{equation}
            		[X_i, X_j] = \sum_{k\in I}c_{ij}^{\ \ k}X_k
	        \end{equation}
        	where the factors $c_{ij}^{\ \ k}$ are called the structure constants\footnotemark\ of the Lie algebra.
		\footnotetext{Note that these constants are basis-dependent.}
        }
        \begin{property}
        	Two Lie algebras $\mathfrak{g}, \mathfrak{h}$ are isomorphic if one can find bases $\mathcal{B}$ for $\mathfrak{g}$ and $\mathcal{C}$ for $\mathfrak{h}$ such that the associated structure constants are equal for all indices $i, j$ and $k$.
        \end{property}

        \newformula{Baker-Campbell-Hausdorff formula}{\index{Baker-Campbell-Hausdorff formula}
        	This formula is the solution of the equation
        	\begin{equation}
            		Z = \log(\exp(X)\exp(X))
	        \end{equation}
        	for $X, Y\in\mathfrak{g}$. The solution is given by following formula
	        \begin{equation}
        	    	\label{linalgebra:bch_formula}
        	        e^Xe^Y = \exp\left(X + Y + \frac{1}{2}[X, Y] + \frac{1}{12}[X, [X, Y]] - \frac{1}{12}[Y, [X, Y]] + \cdot\right)
        	\end{equation}
	        One should note that this formula will only converge if $X, Y$ are sufficiently small (for matrix Lie algebras this means that $||X|| + ||Y|| < \frac{\ln(2)}{2}$ under the Hilbert-Schmidt norm \ref{linalgebra:hilbert_schmidt_norm}). Due to the closure under commutators (see Lie algebra definition) the exponent in the BCH formula is also an element of the Lie algebra. So the formula gives an expression for Lie group multiplication in terms of Lie algebra elements (whenever the formula converges). 
        }
        \begin{result}[Lie product formula\footnotemark]\index{Lie!product formula}
        	\footnotetext{Also called the Lie-Trotter formula.}
        	Let $\mathfrak{g}$ be a Lie algebra. The following formula applies to any $X, Y\in\mathfrak{g}$:
        	\begin{equation}
            		\label{linalgebra:lie_product_formula}
	                e^{X + Y} = \lim_{n\rightarrow+\infty}\left(e^{\frac{X}{n}}e^{\frac{Y}{n}}\right)^n
        	\end{equation}
        \end{result}


\subsection{Examples}
        \begin{example}
        	The cross product $\times:\mathbb{R}^3\times\mathbb{R}^3\rightarrow\mathbb{R}^3$ turns $\mathbb{R}^3$ into a Lie algebra.
        \end{example}
        \begin{example}
        	An interesting example is the Lie algebra associated to the Lie group of invertible complex\footnotemark\ matrices $GL(n, \mathbb{C})$. This Lie group is a subset of its own Lie algebra $\mathfrak{gl}(n, \mathbb{C}) = M_n(\mathbb{C})$. It follows that for every $A\in GL(n, \mathbb{C})$ and every $B\in\mathfrak{gl}(n, \mathbb{C})$ the following equality holds:
        	\begin{equation}
        		L_{A,\ast}(B) = L_A(B)
        	\end{equation}
        	\footnotetext{As usual, this result is also valid for real matrices.}
        \end{example}
        \begin{result}
        	By noting that the endomorphism ring End$(V)$ of an $n$-dimensional vector space $V$ is given by the matrix ring $M_n(K)$, we see that End$(V)$ also forms a Lie algebra when equipped with the commutator of linear maps.
        \end{result}
        
        Following two examples of Lie algebras can be checked using condition \ref{linalgebra:lie_isometry}:
        \begin{example}[Lie algebra of $O(3)$]
        	The set of $3\times3$ anti-symmetric matrices. It is also important to note that $\mathfrak{o}(3) = \mathfrak{so}(3)$. The structure constants of this Lie algebra are given by \ref{tensor:levi_civita_symbol}, i.e. $C_{ijk} = \varepsilon_{ijk}$.
        \end{example}
        \begin{example}[Lie algebra of $SU(2)$]
        	The set of $2\times2$ traceless anti-Hermitian matrices. This result can be generalized to arbitrary $n\in\mathbb{N}$.
        \end{example}
        
        \begin{example}[$SL(2, \mathbb{C})$]
        	\nomenclature[S]{$SL(2, \mathbb{C})$}{Special linear group of dimension 2 over the field of complex numbers. Is isomorphic to $\text{Spin}(1, 3)$ and hence forms a universal double cover of the Lorentz group $SO(1, 3)$. Important in general relativity as the transformation group of spinors.}
        	To compute the Lie bracket in the Lie algebra $\mathfrak{sl}(2, \mathbb{C}) = T_e(SL(2, \mathbb{C}))$ we need to find the action of $l_{g, *}$ on any vector $Y\in\mathfrak{sl}(2, \mathbb{C})$. This is given by:
        	\begin{equation}
        		l_{\left(\begin{smallmatrix}a&b\\c&d\end{smallmatrix}\right), *}\left(\left.\pderiv{}{x^i}\right|_e\right)
        		= \left(\begin{matrix}a&0&b\\-b&a&0\\c&0&\frac{1+bc}{a}\end{matrix}\right)^m_i\left.\pderiv{}{x^m}\right|_{\left(\begin{smallmatrix}a&b\\c&d\end{smallmatrix}\right)}
        	\end{equation}
        	where we used the coordinate chart $(U, \phi)$ defined by: \[U = \left\{\left(\begin{matrix}a&b\\c&d\end{matrix}\right)\in SL(2, \mathbb{C}): a\neq0\right\}\] and \[\phi:U\rightarrow\mathbb{C}^3:\left(\begin{matrix}a&b\\c&d\end{matrix}\right)\mapsto(a, b, c)\]
        	One can then use this formula to work out the Lie bracket of the basis vectors $X_i = \left.\pderiv{}{x^i}\right|_e$ to obtain the structure constants:
        	\begin{equation}
        		\label{lie:sl2c_lie_brackets}
        		\begin{cases}
        			\text{\textlbrackdbl}X_1, X_2\text{\textrbrackdbl} = 2X_2\\
        			\text{\textlbrackdbl}X_1, X_3\text{\textrbrackdbl} = -2X_3\\
        			\text{\textlbrackdbl}X_2, X_3\text{\textrbrackdbl} = X_1
        		\end{cases}
        	\end{equation}
        \end{example}

\subsection{Solvable Lie algebras}

	\newdef{Derived algebra}{
		Let $\mathfrak{g}$ be a Lie algebra with Lie bracket $[\cdot, \cdot]$. The derived Lie algebra is defined as follows:
		\begin{equation}
			[\mathfrak{g}, \mathfrak{g}] = \{[x, y]:x, y\in\mathfrak{g}\}
		\end{equation}
	}
	\newdef{Solvable Lie algebra}{\index{solvable}
		Consider the sequence of derived Lie algebras
		\begin{equation}
			g\geq [\mathfrak{g}, \mathfrak{g}] \geq [[\mathfrak{g}, \mathfrak{g}], [\mathfrak{g}, \mathfrak{g}]] \geq \cdots
		\end{equation}
		If this sequence ends in the zero-space then the Lie algebra $\mathfrak{g}$ is said to be solvable.
	}
	
	\newdef{Radical}{\index{radical}
		Let $\mathfrak{g}$ be a Lie algebra. The radical of $\mathfrak{g}$ is the largest solvable ideal in $\mathfrak{g}$.
	}
	
\subsection{Simple Lie algebras}

	\newdef{Direct sum}{\index{direct!sum}
        	The direct sum of two Lie algebras $\mathfrak{g}, \mathfrak{h}$ is defined as the direct sum in the sense of vector spaces (see \ref{linalgebra:direct_sum}) together with the condition
        	\begin{equation}
        		[x, y] = 0
        	\end{equation}
        	for all $x\in\mathfrak{g}$ and $y\in\mathfrak{h}$.
        }
        
        \newdef{Semidirect sum}{
        	The semi direct product $\mathfrak{g}\ltimes\mathfrak{h}$ of two Lie algebras $\mathfrak{g}, \mathfrak{h}$ is defined as the direct sum in the sense of vector spaces (see \ref{linalgebra:direct_sum}) together with the condition that $\mathfrak{g}$ is an ideal of $\mathfrak{h}$ under the Lie bracket.
        }
        
        \newdef{Simple Lie algebra}{\index{simple!Lie algebra}
        	A Lie algebra is said to be simple if it is non-Abelian and if it has no non-trivial ideals.
        }
        \newdef{Semisimple Lie algebra}{
        	A Lie algebra is said to be semisimple if it is the direct sum of simple algebras.
        }
        
        \begin{theorem}[Levi decomposition]\index{Levi!decomposition}
        	Let $\mathfrak{g}$ be a finite-dimensional Lie algebra. This algebra can be decomposed as follows:
        	\begin{equation}
        		\mathfrak{g} = \mathfrak{R} \ltimes (\mathfrak{L}_1 \oplus \cdots \oplus \mathfrak{L}_n)
        	\end{equation}
        	where $\mathfrak{R}$ is a solvable ideal and the $\mathfrak{L}_i$ are simple subalgebras.
        \end{theorem}

\section{Representation theory}
\subsection{Lie groups}

	\newdef{Representation of Lie groups}{\index{representation}
		Let $G$ be a Lie group and let $V$ be a vector space. A representation of $G$ on $V$ is a Lie group homomorphism $\rho:G\rightarrow GL(V)$.
	}

	\newdef{Adjoint representation of Lie groups}{\index{adjoint!representation}\label{lie:adjoint_representation}
		\nomenclature[O]{$\text{Ad}_g$}{Adjoint representation of a Lie group $G$.}
		Let $G$ be a Lie group. Consider the conjugation map $\Psi_g:h\mapsto ghg^{-1}$. The adjoint representation of $G$ is defined by the differential of the conjugation $T_e\Psi_g$:
		\begin{equation}
			\text{Ad}_g:T_eG\rightarrow T_eG: X\mapsto gXg^{-1}
		\end{equation}
		It is a representation of $G$ on its own tangent space $T_eG\equiv\mathfrak{g}$.
	}

\subsection{Lie algebras}

	\newdef{Representation of Lie algebras}{
		Let $\mathfrak{g}$ be a Lie algebra and let $V$ be a vector space. A representation of $\mathfrak{g}$ on $V$ is a Lie algebra homomorphism $\rho:\mathfrak{g}\rightarrow \text{End}(V)$.
	}

       	\newformula{Adjoint representation of Lie algebras}{
       		\nomenclature[O]{$\text{ad}_X$}{Adjoint representation of a Lie algebra $\mathfrak{g}$.}
       		Using the fact that the adjoint representation of Lie groups is smooth we can define the adjoint representation of Lie algebras as:
       		\begin{equation}
       			\text{ad}_X := T_e(\text{Ad}_g)
		\end{equation}
            	where $g = e^{tX}$. Explicitly, let $\mathfrak{g}$ be a Lie algebra. For every element $X\in\mathfrak{g}$ the adjoint map is given by:
                \begin{equation}
                	\label{lie:bracket_as_adjoint_rep}
                	\text{ad}_X(Y) = [X, Y]
                \end{equation}
                This representation is faithful.
	}
	
        \begin{property}
		Given the antisymmetry of the Lie bracket the Jacobi identity is equivalent to ad$:\mathfrak{g}\rightarrow$ End$(\mathfrak{g})$ being a Lie algebra homomorphism, i.e. ad$_{[X, Y]} = [$ad$_X, $ad$_Y]$.
	\end{property}
	
	\begin{formula}
		Let $\{e_i\}_{i\leq n}$ be a basis of a Lie algebra $\mathfrak{g}$. The structure coefficients can be calculated using the adjoint map as follows:
		\begin{equation}
			\label{lie:ad_structure_coefficient}
			(\text{ad}_{e_i})^j_k = C_{ik}^{\ \ j}
		\end{equation}
	\end{formula}

        \newformula{Induced homomorphism}{\index{homomorphism!induced}\label{lie:induced_homomorphism}
            	Let $\phi:G\rightarrow H$ be a Lie group homomorphism\footnote{Continuity (inherent to the definition of a Lie group homomorphism) is needed to ensure that $\phi(e^{tX})$ is also a one-parameter subgroup (see \ref{group:OPS_composition}).} with $G$ connected and simply-connected. This homomorphism induces a
 a Lie algebra homomorphism\footnote{See also property \ref{lie:prop_hom}.} $\Phi:\mathfrak{g}\rightarrow\mathfrak{h}$ given by:
 		\begin{equation}
 			\Phi(X) = \left.\deriv{}{t}\phi\left(e^{tX}\right)\right|_{t=0}
 		\end{equation}
                or equivalently:
                \begin{equation}
                	\phi\left(e^{tX}\right) = e^{t\Phi(X)}
                \end{equation}
 	}
            
        \begin{remark}
            	The homomorphism induced by $\text{Ad}:G\rightarrow H$ is precisely $\text{ad}:\mathfrak{g}\rightarrow\mathfrak{h}$. Informally we can thus say that the infinitesimal version of the similarity transformation is given by the commutator (in case of $G=$ GL$_n$).
        \end{remark}
 	\begin{result}[Commutator]\index{commutator}
        	For the general linear group GL$_n$ the Lie bracket is given by the commutator:
	        \begin{equation}
        	    	\boxed{[X, Y] = XY - YX}
        	\end{equation}
	        This follows from definition \ref{lie:bracket_as_adjoint_rep}: $[X, Y] = \left.\deriv{}{t}\text{Ad}_{\gamma(t)}(Y)\right|_{t=0}$ with $\gamma(0) = e$ and $\gamma'(0) = X$.
        \end{result}

\subsection{Killing form}

	\newdef{Killing form\footnotemark}{\index{Killing!form}
		\footnotetext{Also called the \textbf{Cartan-Killing form}.}
		Let $\mathfrak{g}$ be a finite-dimensional Lie algebra. The Killing form on $\mathfrak{g}$ is defined as the following symmetric bilinear form\footnotemark:
                \begin{equation}
                	\label{linalgebra:killing_form}
                	\boxed{K(X, Y) = \text{tr}(\text{ad}_X\circ\text{ad}_Y)}
                \end{equation}
                The trace can be found by representing the Lie algebra elements as matrices using Ado's theorem \ref{lie:theorem:ado}. From equation \ref{lie:ad_structure_coefficient} we can work out the action of the Killing form on the basis $\{e_i\}_{i\leq n}$:
                \begin{equation}
                	K_{ij} = C_{ik}^{\ \ l}C_{jl}^{\ \ k}
                \end{equation}
                where $C_{ij}^k$ are the structure constants of the Lie algebra.
	}
	\footnotetext{i.e. a symmetric $(0,2)$-tensor in $\mathfrak{g}^*\otimes\mathfrak{g}^*$ (See definition \ref{tensor:type})}
	
	\begin{theorem}[Cartan's criterion]\index{Cartan!criterion}
		A Lie algebra is semisimple if and only if its Killing form is non-degenerate.
	\end{theorem}
	\begin{result}
		If a Lie algebra is semisimple its Killing form induces a metric
                \begin{equation}
                	g:(X, Y)\mapsto -\ \text{tr}(\text{ad}_X, \text{ad}_Y)
                \end{equation}
                which turns the corresponding Lie group $G$ into a Riemannian manifold.
	\end{result}
	
	\begin{theorem}
		A Lie group $G$ is compact if and only if the Killing form of its associated Lie algebra $\mathfrak{g}$ is negative-definite.
	\end{theorem}
            
	\begin{property}
		The adjoint map $\text{ad}_Z$ is antisymmetric with respect to the Killing form:
		\begin{equation}
			\label{lie:ad_killing_form}
			K(\text{ad}_ZX, Y) = -K(X, \text{ad}_ZY)
		\end{equation}
	\end{property}
	\begin{property}
		The Killing-form is $\text{Ad}$-invariant, i.e.
                \begin{equation}
                    K(\text{Ad}_g(X), \text{Ad}_g(Y)) = K(X, Y)
		\end{equation}
                for all $g\in G$. From this it follows that $\text{Ad}$ is a map from $G$ to the isometry group $\text{Isom}(\mathfrak{g})$.
	\end{property}
	
	\begin{definition}
		Let $\mathfrak{g}$ be a Lie algebra and let $V$ be a vector space equipped with a Lie algebra representation $\rho:\mathfrak{g}\rightarrow\text{End}(V)$. One can then define a Killing form associated with $\rho$ in the following way:
		\begin{equation}
			\label{lie:rho_killing_form}
			K_\rho(X, Y) = \text{tr}\Big(\rho(X)\circ\rho(Y)\Big)
		\end{equation}
	\end{definition}
	\begin{remark}
		This definition is clearly more general than \ref{linalgebra:killing_form} as $K$ is simply given by $K_{\text{ad}}$.
	\end{remark}

\subsection{Roots and Dynkin diagrams}

        \newdef{Cartan subalgebra}{\index{Cartan!subalgebra}
        	Let $\mathfrak{g}$ be a Lie algebra. A subalgebra $\mathfrak{h}$ is called a Cartan subalgebra if there exists a basis $\{h_i\}_{i\in I}$ of $\mathfrak{h}$ that can be extended to a basis $\{h_i\}_{i\in I}\cup\{g_j\}_{j\in J}$ of $\mathfrak{g}$ such that every $g_j$ is an eigenvector of the adjoint map $\text{ad}_h$ for all $h\in\mathfrak{h}$.
        }
        
        \begin{property}
        	Every finite-dimensional Lie algebra contains a Cartan subalgebra.
        \end{property}
        \begin{property}
        	If $\mathfrak{g}$ is semisimple then its Cartan subalgebra is Abelian.
        \end{property}
        
        \newdef{Root}{\index{root}
        	Let $\mathfrak{g}$ be a Lie algebra with Cartan subalgebra $\mathfrak{h}$. From the definition of a Cartan subalgebra it follows that for all $h\in\mathfrak{h}$:
        	\begin{equation}
        		[h, g_j] = \alpha_j(h)g_j
        	\end{equation}
        	where $\{g_j\}_{j\in J}$ is the basis extension of $\mathfrak{g}$ with respect to $\mathfrak{h}$. Because $\alpha_j(h)$ is an eigenvector it is an element of the base field $\mathbb{C}$ and hence we can view $\alpha_j$ as a linear map $\mathfrak{h}\rightarrow\mathbb{C}$, or equivalently $\alpha_j\in\mathfrak{h}^*$. These linear maps are called the roots of $\mathfrak{g}$.
        }
        
        \begin{property}
        	From equation \ref{lie:ad_killing_form} it follows that if $\lambda$ is a root of $\mathfrak{g}$ then $-\lambda$ is also a root of $\mathfrak{g}$.
        \end{property}
        Previous property implies that the root set $\Phi$ is not linearly independent. Therefore we introduce following concept:
        \newdef{Simple root}{
        	The set of simple roots\footnote{For every root set $\Phi$ one can find a set of simple roots.} $\Delta$ is a linearly independent subset of $\Phi$ such that every element $\lambda\in\Phi$ can be written as:
        	\begin{equation}
        		\lambda = \pm\sum_i^na_i\lambda_i
        	\end{equation}
        	where $a_i\in\mathbb{N}$ and $\lambda_i\in\Delta$. This definition requires the expansion coefficients $a_i$ of a certain root $\lambda$ to be either all positive or all negative.
        }
        
        \begin{property}
        	Let $\mathfrak{h}$ be a Cartan subalgebra. The set of simple roots $\Delta$ forms a basis for the dual space $\mathfrak{h}^*$ (over $\mathbb{C}$).
        \end{property}
        
        \newdef{Weyl group}{\index{Weyl!group}
        	For every simple root $\lambda$ we define the linear map $\sigma_\lambda$ as follows\footnote{This comes down to a reflection through the hyperplane orthogonal to the vector $\lambda$.}:
        	\begin{equation}
        		\sigma_\lambda:\text{span}_{\mathbb{R}}(\Delta)\rightarrow\text{span}_{\mathbb{R}}(\Delta):\mu\mapsto \mu - 2\frac{\langle\lambda,\mu\rangle}{\langle\lambda,\lambda\rangle}\lambda
        	\end{equation}
        	where the inner product $\langle\cdot,\cdot\rangle$ is given by the dual Killing form.\footnotemark\ The Weyl group $W$ is then defined as the group generated by all the $\sigma_\lambda$'s.
        	\footnotetext{Consider the \textit{sharp} map \ref{manifolds:sharp_map} where one replaces the metric $g$ by the Killing form $K$. The dual Killing form $K^*$ is then a proper inner product defined as: \[K^*(\cdot, \cdot) = K(\cdot^\sharp, \cdot^\sharp)\]}
        }
        
        \begin{property}
        	Every root $\phi\in\Phi$ can be written as $\phi = \sigma(\mu)$ for some $\mu\in\Delta$ and $\sigma\in W$. Furthermore, the root set $\Phi$ is closed under the action of $W$.
        \end{property}

        \newdef{Cartan matrix}{\index{Cartan!matrix}
        	Let $\lambda_i, \lambda_j\in\Delta$ be simple roots. It follows from previous property that \[\sigma_{\lambda_i}(\lambda_j) = \lambda_j - 2\frac{\langle\lambda_i,\lambda_j\rangle}{\langle\lambda_i,\lambda_i\rangle}\lambda_i\] is a root. From the definition of simple roots it then follows that the quantity
        	\begin{equation}
        		C_{ij} = 2\frac{\langle\lambda_i,\lambda_j\rangle}{\langle\lambda_i,\lambda_i\rangle}
        	\end{equation}
        	is an integer. The matrix formed by these numbers is called the Cartan matrix.
        }
        \begin{property}
        	The Cartan matrix $C_{ij}$ has the following properties:
        	\begin{itemize}
        		\item $C_{ii} = 2$
        		\item $C_{ij} \leq 0$ if $i\neq j$
        		\item $C_{ij} = 0\iff C_{ji} = 0$
        	\end{itemize}
        	This last property however does not imply that the Cartan matrix is symmetric. The fact that it is not symmetric can immediately be seen from its definition.
        \end{property}
        
        \newdef{Bond number}{
        	For all indices $i\neq j$ the bond number $n_{ij}$ is defined as follows:
        	\begin{equation}
        		n_{ij} = C_{ij}C_{ji}
        	\end{equation}
        	Using the definition of the coefficients $C_{ij}$ we see that $n_{ij}$ is an integer equal to $4\cos^2\sphericalangle(\lambda_i, \lambda_j)$. This implies that $n_{ij}$ can only take on the values $0, 1, 2, 3$.\footnote{The value 4 would only be possible if the angle between $\lambda_i$ and $\lambda_j$ is $0$ but this can only occur in the case that $i=j$, which was excluded from the definition.}
        }
        \begin{remark}
        	In the case of $n_{ij} = 2$ or $n_{ij} = 3$ there arise two possibilities. Namely that $C_{ij}>C_{ji}$ or $C_{ij}<C_{ji}$. From the definition of the Cartan integers and the symmetry of the dual Killing form these cases correspond to $\langle\lambda_i, \lambda_i\rangle<\langle\lambda_j, \lambda_j\rangle$ and $\langle\lambda_i, \lambda_i\rangle>\langle\lambda_j, \lambda_j\rangle$.
        \end{remark}
        
        \begin{construct}[Dynkin diagram]\index{Dynkin diagram}
        	For a semisimple Lie algebra $\mathfrak{g}$ with simple root set $\Delta$ one can draw a so-called Dynkin diagram by using the following rules:
        	\begin{enumerate}
        		\item For every simple root $\lambda\in\Delta$ draw a circle: $\bigcirc$
        		\item If $\bigcirc$ and $\bigcirc$ denote the simple roots $\lambda_i$ and $\lambda_j$, draw $n_{ij}$ lines between them.
        		\item When $n_{ij} = 2$ or $n_{ij} = 3$ add a $<$ or $>$ sign to relate the roots based on their lengths (see previous remark).
        	\end{enumerate}
        \end{construct}
        
        \begin{theorem}[Cartan \& Killing]
        	Every finite-dimensional simple $\mathbb{C}$-Lie algebra can be reconstructed from its set of simple roots $\Delta$.
        \end{theorem}
        
        \begin{method}
        	The Dynkin diagrams can be classified as follows (for every type the first three examples are given):
        	\begin{itemize}
        		\item A$_n$: \begin{center}\dynk \dynkA{2} \dynkA{3} \end{center}
			\item B$_n, n\geq2$: \begin{center}\dynkB{2} \dynkB{3} \dynkB{4} \end{center}
			\item C$_n, n\geq2$: \begin{center}\dynkC{2} \dynkC{3} \dynkC{4} \end{center}
			\item D$_n, n\geq4$: \begin{center}\dynkD{4} \dynkD{5} \dynkD{6} \end{center}
        	\end{itemize}
        	These are the only possible diagrams for simple Lie algebras.\footnote{With exception of $E_6, E_7, E_8, F_4$ and $G_2$, the so-called \textit{exceptional Lie algebras}.}
        \end{method}
        
        \begin{example}[$SL(2, \mathbb{C})$]
        	By looking at the Lie brackets in \ref{lie:sl2c_lie_brackets} we see that the one-element set $\{X_1\}$ forms a Cartan subalgebra of $\mathfrak{sl}(2, \mathbb{C})$. From \ref{lie:sl2c_lie_brackets} it is also immediately clear that the simple root set $\Delta$ is given by the one-element set $\{\lambda\in\mathfrak{sl}^*(2, \mathbb{C}): \lambda(X_1)\mapsto 2\}$. Hence the Dynkin diagram for $\mathfrak{sl}(2, \mathbb{C})$ is $A_1$.
        \end{example}

\section{Universal enveloping algebra}

	\newdef{Universal enveloping algebra}{\index{universal!enveloping algebra}
		\nomenclature[S]{$U(\mathfrak{g})$}{The universal enveloping algebra of a Lie algebra $\mathfrak{g}$.}
		Let $\mathfrak{g}$ be a Lie algebra with Lie bracket $[\cdot, \cdot]$. First construct the tensor algebra $T(\mathfrak{g})$. The universal enveloping algebra $U(\mathfrak{g})$ is defined as quotient of $T(\mathfrak{g})$ by the two-sided ideal generated by the elements $g\otimes h - h\otimes g - [g, h]$.
	}
	
	\newdef{Casimir invariant\footnotemark}{\index{Casimir!invariant}\label{lie:casimir_invariant}
		\footnotetext{Also known as a \textbf{Casimir operator} or \textbf{Casimir element}.}
		Let $\mathfrak{g}$ be a Lie algebra. A Casimir invariant $J$ is an element of the center of $U(\mathfrak{g})$.
	}
	
	\newformula{Quadratic Casimir invariant}{
		Consider a Lie algebra representation $\rho:\mathfrak{g}\rightarrow\text{End}(V)$ on an $n$-dimensional vector space $V$. Let $\{X_i\}_{i\leq n}$ be a basis for $\mathfrak{g}$. The (quadratic) Casimir invariant associated with $\rho$ is given by:
		\begin{equation}
			\Omega_\rho = \sum_{i=0}^n\rho(X_i)\circ\rho(\xi_i)
		\end{equation}
		where the set $\{\xi_i\}_{i\leq n}$ is defined by the relation $K_\rho(X_i, \xi_j) = \delta_{ij}$ using the Killing form \ref{lie:rho_killing_form}.
	}
	
	\begin{property}
		When the representation $\rho:\mathfrak{g}\rightarrow\text{End}(V)$ is irreducible Schur's lemma \ref{rep:schurs_lemma} tells us that:
		\begin{equation}
			\Omega_\rho = c_\rho\mathbbm{1}_V
		\end{equation}
		By taking the trace of this formula and using formula \ref{lie:rho_killing_form} we see that $c_\rho = \frac{\dim\mathfrak{g}}{\dim V}$.
	\end{property}

\section{Poisson algebras and Lie superalgebras}

	\newdef{Poisson algebra}{\index{Poisson!algebra}
		Let $V$ be a vector space equipped with two bilinear operations $\star$ and $\{\cdot, \cdot\}$ that satisfy the following conditions:
		\begin{itemize}
			\item The couple $(V, \star)$ is an associative algebra.
			\item The couple $(V, \{\cdot, \cdot\})$ is a Lie algebra.
			\item the \textbf{Poisson bracket} $\{\cdot,\cdot\}$ acts as a derivation\footnote{See definition \ref{manifolds:derivation}.} with respect to the operation $\star$, i.e. \[\{x, y\star z\} = \{x, y\}\star z + y\star\{x, z\}\]
		\end{itemize}
	}

\chapter{Quantum Algebra}

\section{Hopf algebras}

	\newdef{Hopf algebra}{\index{Hopf!algebra}\index{antipode}
		Let $(A, \Delta, \nabla, \eta, \varepsilon)$ be a bi-algebra. $A$ is called a Hopf algebra if it is equipped with a linear map $S:A\rightarrow A$ that satisfies:
		\begin{equation}
			\nabla\circ(\mathbbm{1}_A\otimes S)\circ\Delta = \nabla\circ(S\otimes\mathbbm{1}_A)\circ\Delta = \eta\circ\varepsilon
		\end{equation}
		The map $S$ is also called the \textbf{antipode} or \textbf{coinverse}.
	}

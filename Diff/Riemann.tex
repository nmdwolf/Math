\chapter{Riemannian Geometry}\label{diff:chapter:riemann}

	The main reference for this chapter is \cite{petersen}.

\section{Riemannian manifolds}
\subsection{Metric}

	\begin{definition}[Bundle metric]\index{metric!bundle}
		Consider the bundle of second order covariant vectors. Following from \ref{tensor:tensor_product} every section $g$ of this bundle gives a bilinear map \[g_x:T_xM\times T_xM\rightarrow\mathbb{R}\]
		for all $x\in M$. If this map is symmetric and non-degenerate and if it depends smoothly on $p$ it is called a \textbf{(Lorentzian}) metric.\footnote{See also the section about Hermitian forms and metric forms \ref{linalgebra:innerproduct}.}
		
		The maps $\{g_x\}_{x\in M}$ can be `glued' together to form a global metric $g$, defined on the fibre product\footnotemark\ $TM\diamond TM$. Defining this map on $TM\times TM$ is not possible as tangent vectors belonging to different points in $M$ cannot be `compared'. The collection $\{\langle\cdot|\cdot\rangle_x|x\in M\}$ is called a \textbf{bundle metric}\index{metric!bundle}.
		\footnotetext{See definition \ref{manifolds:fibre_product}.}
	\end{definition}

	A Riemannian metric also induces a duality between $TM$ and $T^*M$. This is given by the \textit{flat} and \textit{sharp} isomorphisms:
	\newdef{Musical isomorphisms}{\index{musical isomorphism}\label{riemann:musical_isomorphisms}
		Let $g:TM\times TM\rightarrow\mathbb{R}$ be the Riemannian metric on $M$. The \textbf{flat} isomorphism is defined as:
		\begin{gather}
			\label{manifolds:flat_map}
			\flat:v\mapsto g(v, \cdot)
		\end{gather}
		The \textbf{sharp} isomorphism is defined as the inverse map:
		\begin{gather}
			\label{manifolds:sharp_map}
			\sharp:g(v, \cdot)\mapsto v
		\end{gather}
		These 'musical' isomorphisms can be used to lower and raise tensor indices.
	}

\subsection{Riemannian manifold}

	\newdef{Pseudo-Riemannian manifold}{\index{Riemann!manifold}\label{riemann:riemannian_manifold}
		Let $M$ be a smooth manifold. This manifold is called pseudo-Riemannian if it is equipped with a pseudo-Riemannian metric. A \textbf{Riemannian manifold} is similarly defined.
	}
	
	\newdef{Riemannian isometry}{\index{isometry}
		Let $(M, g_M)$ and $(N, g_N)$ be two Riemannian manifolds. An isometry \ref{diff:isometry_def} $f:M\rightarrow N$ is said to be Riemannian if $F^*g_N = g_M$.
	}
	
	\begin{property}\index{index}
		Let $M$ be a pseudo-Riemannian manifold. For every $p\in M$ there exists a splitting $T_pM = P\oplus N$ where $P$ is a subspace on which the pseudometric is positive definite and $N$ is a subspace on which the pseudometric is negative-definite. This splitting is however not unique, only the dimensions of the two subspaces are well-defined.
	\end{property}
	Due to the continuity of the pseudometric, the dimensions of this splitting wil be the same for points in the same neighbourhood. For connected manifolds this amounts to a global invariant:
	\newdef{Index}{
		Let $M$ be a connected pseudo-Riemannian manifold. The dimension of the \textit{negative} subspace $N$ in the above splitting $T_pP = P\oplus N$ is called the index of the pseudo-Riemannian manifold.
	}
	
	\begin{theorem}[Whitney's embedding theorem]\index{Whitney!embedding theorem}
		Every smooth paracompact\footnotemark\ manifold $M$ can be embedded in $\mathbb{R}^{2\dim M}$.
		\footnotetext{See definition \ref{topology:paracompact}.}
	\end{theorem}
	\begin{theorem}[Whitney's immersion theorem]\index{Whitney!immersion theorem}
		Every smooth paracompact manifold $M$ can be immersed in $\mathbb{R}^{2\dim M - 1}$.
	\end{theorem}
	\begin{theorem}[Immersion conjecture]\index{immersion!conjecture}
		Every smooth paracompact manifold $M$ can be immersed in $\mathbb{R}^{2\dim M - a(\dim M)}$ where $a(n)$ is the number of 1's in the binary expansion of $n$.
	\end{theorem}
	
	\newdef{Riemannian cone}{\index{Riemann!cone}\label{riemann:riemannian_cone}
		Let $(M, g)$ be a Riemannian manifold. Consider the product manifold $M\times]0, +\infty[$. This manifold can also be turned into a Riemannian manifold by equipping it with the metric $t^2g+dt^2$. This manifold is called the Riemannian cone or \textbf{metric cone} of $(M, g)$.
	}
	
\subsection{Levi-Civita connection}

	\newdef{Riemannian connection}{\index{Riemann!connection}\index{Levi-Civita!connection|see{Riemann connection}}\label{riemann:levi_civita_connection}
		An affine connection $\nabla$ on a Riemannian manifold $(M, g)$ is said to be Riemannian if it satisfies following two conditions:
		\begin{enumerate}
			\item $\nabla$ is metric:
			\begin{gather}
				X(g(Y, Z)) = g(\nabla_XY, Z) + g(Y, \nabla_XZ)
			\end{gather}
			\item $\nabla$ is torsion-free:
			\begin{gather}
				\nabla_XY - \nabla_YX = [X, Y]
			\end{gather}
		\end{enumerate}
		A Riemannian connection is often called a \textbf{Levi-Civita connection}. 
	}
	
	\begin{theorem}[Fundamental theorem of Riemannian geometry]
		The Levi-Civita connection on a Riemannian manifold $(M, g)$ is unique.
	\end{theorem}
	
	\newformula{Koszul formula}{\index{Koszul!formula}
		The Levi-Civita connection $\nabla$ on a Riemannian manifold $(M, g)$ is implicitly (and uniquely\footnote{Any connection satisfying this formula necessarily coincides with the Levi-Civita connection.}) given by the following formula:
		\begin{align}
			2g(\nabla_XY, Z) &= \mathcal{L}_Xg(Y, Z) + d(\iota_Xg)(Y, Z)\\
			&= X(g(Y, Z)) + Y(g(Z, X)) - Z(g(X, Y))\nonumber\\
			&\hspace{3cm}+ g([X, Y], Z) - g([Z, X], Y) - g([Y, Z], X)
		\end{align}
	}
	
	The following (local) formula can be useful, especially in general relativity:
	\newformula{Divergence}{
		Let $\nabla$ be the Levi-Civita connection of a given Riemannian manifold. Using the metric determinant one can write the (covariant) divergence in terms of the ordinary partial derivatives:
		\begin{gather}
			\label{diff:divergence_partial}
			\nabla_\mu V^\mu = \frac{1}{\sqrt{g}}\partial_\mu(\sqrt{g}V^\mu)
		\end{gather}
	}

\subsection{Killing vectors}

	\newdef{Killing vector}{\index{Killing!vector}\label{diff:killing_vector}
		Let $(M, g)$ be a Riemannian manifold. A vector field $X$ satisfying
		\begin{gather}
			\boxed{\mathcal{L}_Xg = 0}
		\end{gather}
		is called a Killing vector field.
		
		A simple calculation gives us the following coordinate expression:
		\begin{gather}
			(\mathcal{L}_Xg)_{\mu\nu} = X^\lambda\partial_\lambda g_{\mu\nu} + g_{\lambda\mu}\partial_\nu X^\lambda + g_{\lambda\nu}\partial_\mu X^\lambda
		\end{gather}
	}
	\begin{formula}
		Given a Levi-Civita connection $\nabla$ on $(M, g)$ we can rewrite the Killing condition as follows:
		\begin{gather}
			\nabla_{(m}X_{n)} = 0
		\end{gather}
	\end{formula}

	\newdef{Killing tensor}{\index{Killing!tensor}
		Let $\nabla$ be the Levi-Civita connection on $(M, g)$. A tensor $T$ satisyfing
		\begin{gather}
			\label{diff:killing_tensor}
			\nabla_{(m_N}T_{m_1...m_{N-1})} = 0
		\end{gather}
		is called a Killing tensor. It is obvious that this \textbf{generalized Killing condition} is a direct generalization of the Killing condition as given above.
	}

\section{Curvature}\label{diff:section:curvature}

	\newformula{Riemann curvature tensor}{\index{Riemann!curvature}
		The Riemann (curvature) tensor $R$ is defined as following $(1,3)$-tensor:
		\begin{gather}
			R(v, w)z = [\nabla_v, \nabla_w]z - \nabla_{[v, w]}z
		\end{gather}
		where $\nabla$ is the Levi-Civita connection. In index notation it is given by:
		\begin{gather}
			R^i_{jkl} = dx^i\big(R(e_k, e_l)e_j\big)
		\end{gather}
	}
	
	\newformula{Directional curvature operator\footnotemark}{\index{tidal force operator}
		\footnotetext{Also called the \textbf{tidal force operator} (mostly in physics).}
		\begin{gather}
			R_v(w) = R(w, v)v
		\end{gather}
	}
	
	\newformula{Sectional curvature}{\index{sectional!curvature}
		\begin{gather}
			\text{sec}(v, w) = \frac{g(R(w, v)v, w)}{g(v, v)g(w,w) - g(v, w)^2} = \frac{g(R_v(w), w)}{g(v\wedge w, v\wedge w)}
		\end{gather}
		An important result states that the sectional curvature only depends on the span of $v, w$.
	}
	\remark{For surfaces the sectional curvature coincides with the Gaussian curvature $K$ (see Theorema Egregium \ref{diff:theorema_egregium}). Generally the sectional curvature gives the Gaussian curvature of the plane spanned by the vector $v, w$.}
	
	\newformula{Ricci tensor}{\index{Ricci!tensor}
	    	\begin{gather}
    			\label{diff:manifolds:ricci_tensor}
        	R_{\mu\nu} = R^\lambda_{\ \mu\lambda\nu}
	    	\end{gather}
	}
    
	\newformula{Ricci scalar}{\index{Ricci!scalar}\index{scalar!curvature}
	    	\begin{gather}
	    		\label{diff:manifolds:ricci_scalar}
	        R = R^\mu_{\ \mu}
	    	\end{gather}
	        This scalar quantity is also called the \textbf{scalar curvature}.
	}

	\newformula{Einstein tensor}{\index{Einstein!tensor}
		\begin{gather}
			\label{diff:manifolds:einstein_tensor}
			\boxed{G_{\mu\nu} = R_{\mu\nu} - \frac{1}{2}g_{\mu\nu}R}
		\end{gather}
	}
	\begin{theorem}
	    	For 4-dimensional manifolds the Einstein tensor $G_{\mu\nu}$ is the only tensor containing at most second derivatives of the metric $g_{\mu\nu}$ and satisfying:
	        \begin{gather}
	        	\nabla_\mu G^{\mu\nu} = 0
	        \end{gather}
	\end{theorem}

\section{Spinor bundles}\label{section:spinor_bundles}

	In this subsection we only work with Riemannian manifolds\footnote{This also works for Lorentzian manifolds and Spin$(1, n-1)$ groups.} $(M, g)$ because this ensures the existence of an O$(n)$ reduction of the tangent bundle $TM$ (see the example above).

	\newdef{Spin structure}{\index{spin!structure}\index{spin!frame}
		Consider the (oriented) orthonormal frame bundle $\pi_{SO}:F_{SO}(M)\rightarrow M$ which is obtained by reducing the structure group of the frame bundle $F(M)$ from GL$(n)$ to SO$(n)$. Furthermore, let $\pi_{spin}:P_{spin}\rightarrow M$ be a principal Spin$(n)$-bundle over $M$.
		
		The smooth manifold $M$ is said to have a spin structure if there exists an equivariant 2-fold lifting of $F_{SO}$ to $P_{spin}$, i.e. a morphism $\xi:P_{spin}\rightarrow F_{SO}(M)$ together with the 2-fold covering map $\rho:\text{Spin}(n)\rightarrow\text{SO}(n)$ that satisfy:
		\begin{itemize}
			\item $\pi_{SO}\circ\xi = \pi_{spin}$
			\item $\xi(p\vartriangleleft g) = \xi(p)\cdot\rho(g)$
		\end{itemize}
		for all $g\in\text{Spin}(n)$, where $\vartriangleleft$ and $\cdot$ denote the right actions of the respective structure groups. If $M$ admits a spin structure it is often called a \textbf{spin manifold} and the principal Spin$(n)$-bundle $P$ is called the \textbf{spin frame bundle}.
	}
	
	\begin{property}
		A smooth orientable manifold $M$ is spin if and only if its second Stiefel-Whitney class vanishes.
	\end{property}
	\begin{property}
		A special case occurs when $\dim M = 3$. We then have that $M$ is spin if it is compact and orientable.
	\end{property}
	
	\newdef{Spin bundle}{\index{spin!bundle}
		A spin bundle is a vector bundle associated to a spin frame bundle.
	}
	\newdef{Spinor field}{\index{spinor}
		A spinor field is a (smooth) section of a spin bundle.
	}

\section{Sphere bundles}

	\newdef{Unit sphere bundle}{\index{unit!sphere bundle}
		Let $V$ be a normed vector space. Consider a vector bundle $\prin{V}{E}{B}$. From this bundle we can derive a new bundle where we replace the typical fibre $V$ by the unit sphere $\{v\in V : ||v|| = 1\}$. It should be noted that this new bundle is not a vector bundle since the unit sphere is not a vector space.
	}
	\begin{remark}[Unit disk bundle]\index{unit!disk bundle}
		A similar construction can be made by replacing the unit sphere by the unit disk $\{v\in V : ||v||\leq1\}$.
	\end{remark}

\section{Hilbert bundles}
	
	\newdef{Hilbert bundle}{\index{Hilbert!bundle}
		A vector bundle for which the typical fibre is a Hilbert space is called a Hilbert bundle.
	}
	\newdef{Compatible Hilbert bundle}{
		Consider the isomorphisms
		\begin{gather}
			l_x:F_x\rightarrow\mathcal{H}:h\mapsto\varphi_i(x, h)\in\pi(x)
		\end{gather}
		where $\mathcal{H}$ is the typical fibre and where $\{(U_i, \varphi_i)\}_{i\in I}$ is a trivializing cover. These maps $l_x$ are called \textbf{point-trivializing maps}.
		
		Using these maps we can extend the metric structure of the typical fibre $\mathcal{H}$ to the fibres $F_x$ for all $x$ by:
		\begin{gather}
			\langle v|w \rangle_x = \langle l_x(v)|l_x(w) \rangle_{\mathcal{H}}
		\end{gather}
		The Hilbert bundle is said to be compatible (with the metric structure on $\mathcal{H}$) if the above extension is valid for all $v, w \in F_x$.
	}
	
	\begin{remark*}
		For compatible Hilbert bundles, the transition maps $l_{x\rightarrow y} = l_y^{-1}\circ l_x:\pi^{-1}(x)\rightarrow\pi^{-1}(y)$ are easily seen to be isometries.
	\end{remark*}

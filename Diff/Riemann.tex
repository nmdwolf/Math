\chapter{Riemannian Geometry}

\section{Riemannian manifolds}

	\newdef{Pseudo-Riemannian manifold}{\index{Riemann!Riemannian manifold}
		Let $M$ be a smooth manifold with a non-degenerate symmetric bilinear form on every tangent space $T_pM, p\in M$. This manifold is called pseudo-Riemannian. If the form is also positive-definite, i.e. it is a metric form, then the manifold is said to be Riemannian.
	}
	
	\begin{property}\index{index}
		Let $M$ be a pseudo-Riemannian manifold. For every $p\in M$ there exists a splitting $T_pM = P\oplus N$ where $P$ is a subspace on which the pseudometric is positive-definite and $N$ is a subspace on which the pseudometriv is negative-definite. This splitting is however not unique, only the dimensions of the two subspaces are well-defined.
	\end{property}
	Due to the continuity of the pseudometric, the dimensions of this splitting wil be the same for points in the same neighbourhood. For connected manifolds this amounts to a global invariant:
	\newdef{Index}{
		Let $M$ be a connected pseudo-Riemannian manifold. The dimension of the \textit{negative} subspace $N$ in the above splitting $T_pP = P\oplus N$ is called the index of the pseudo-Riemannian manifold.
	}

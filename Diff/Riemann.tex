\chapter{Riemannian Geometry}\label{chapter:riemann}

    The main reference for this chapter is \cite{petersen}.

\section{Riemannian manifolds}
\subsection{Metric}

    \begin{definition}[Bundle metric]\index{metric}
        Consider the bundle of second order covariant vectors. From definition \ref{tensor:tensor_product} we know that every section $g$ of this bundle gives a bilinear map \[g_p:T_pM\times T_pM\rightarrow\mathbb{R}\] for all $p\in M$. If this map is symmetric and nondegenerate it is called a \textbf{(Lorentzian}) metric.\footnote{See also the section about Hermitian forms and metric forms \ref{linalgebra:innerproduct}.}

        The maps $\{g_p\}_{p\in M}$ can be glued together to form a global metric $g$, defined on the fibre product\footnote{See definition \ref{manifolds:fibre_product}.} $TM\diamond TM$. Defining this map on $TM\times TM$ is not possible as tangent vectors belonging to different points in $M$ cannot be `compared'. The collection $\{\langle\cdot|\cdot\rangle_p:p\in M\}$ is called a \textbf{bundle metric} or \textbf{fibre metric}.
    \end{definition}

    A Riemannian metric induces a duality between $TM$ and $T^*M$. This is given by the \textit{flat} and \textit{sharp} isomorphisms:
    \newdef{Musical isomorphisms}{\index{musical isomorphism}\label{riemann:musical_isomorphisms}
        Let $g:TM\times TM\rightarrow\mathbb{R}$ be a Riemannian metric on $M$. The \textbf{flat} isomorphism is defined as
        \begin{gather}
            \label{manifolds:flat_map}
            \flat:v\mapsto g(v, \cdot).
        \end{gather}
        The \textbf{sharp} isomorphism is defined as the inverse map. For an arbitrary covector field $\omega$ it is implicitly given by
        \begin{gather}
            \label{manifolds:sharp_map}
            g(\omega^\sharp, v) = \omega(v).
        \end{gather}
        These \textbf{musical isomorphisms} can be used to raise and lower tensor indices.
    }

    \newdef{Codifferential}{\index{codifferential}
        Using the de Rham differential $d$ and the Hodge star operator \ref{tensor:hodge_star} (here extended to manifolds) we can define a boundary operator\footnote{This means that $\delta^2 = 0$ (see \ref{topology:boundary_operator_relation}).} $\delta:\Omega^k(M)\rightarrow \Omega^{k-1}(M)$:
        \begin{gather}
            \label{manifolds:codifferential}
            \delta := (-1)^k\ast^{-1} d \ast = (-1)^{n(k+1)+1}\ast d \ast.
        \end{gather}
    }
    \newdef{Hodge Laplacian\footnotemark}{\index{Hodge!Laplacian}
        \footnotetext{Sometimes called the \textbf{Hodge-de Rham} or \textbf{Laplace-de Rham} operator.}
        Using the de Rham differential and codifferential one can define the following derivation $\Delta:\Omega^k(M)\rightarrow\Omega^k(M)$:
        \begin{gather}
            \label{diff:hodge_laplacian}
            \Delta := d\delta + \delta d.
        \end{gather}
        It should be noted that, in contrast to the differential $d$, the Hodge Laplacian does depend on the metric through the definition of the codifferential.
    }

    \newdef{Harmonic form}{\index{harmonic!form}
        Element of the kernel of the Hodge Laplacian $\Delta$. The space of harmonic $k$-forms is denoted by $\mathcal{H}^k(M)$.
    }

    \begin{theorem}[Hodge decomposition]\index{Hodge!decomposition}
        Let $M$ be a compact manifold equipped with a Hodge Laplacian $\Delta$. Every differential $k$-form admits a decomposition of the form
        \begin{gather}
            \omega = d\alpha+\delta\beta+h
        \end{gather}
        where $h\in\mathcal{H}^k(M)$.
    \end{theorem}
    \begin{result}
        The $k^{th}$ de Rham cohomology group is isomorphic (as Abelian groups) to the space of harmonic $k$-forms:
        \begin{gather}
            H^k(M;\mathbb{R})\cong\mathcal{H}^k(M).
        \end{gather}
    \end{result}

\subsection{Riemannian manifold}

    \newdef{Pseudo-Riemannian manifold}{\index{Riemann!manifold}\label{riemann:riemannian_manifold}
        Let $M$ be a smooth manifold. This manifold is said to be pseudo-Riemannian if it is equipped with a pseudo-Riemannian metric. A \textbf{Riemannian manifold} is defined similarly.
    }

    \newdef{Riemannian isometry}{\index{isometry}
        Let $(M, g_M)$ and $(N, g_N)$ be two Riemannian manifolds. An isometry \ref{diff:isometry_def} $f:M\rightarrow N$ is said to be Riemannian if $f^*g_N = g_M$.
    }

    \begin{property}\index{index}
        Let $M$ be a pseudo-Riemannian manifold. For every $p\in M$ there exists a splitting $T_pM = P\oplus N$ where $P$ is a subspace on which the pseudometric is positive definite and $N$ is a subspace on which the pseudometric is negative-definite. This splitting is however not unique, only the dimensions of the two subspaces are well-defined.
    \end{property}
    Due to the continuity of the pseudometric, the dimensions of this splitting will be the same for all points in a connected neighbourhood. For connected manifolds this amounts to a global invariant:
    \newdef{Index}{
        Let $M$ be a connected pseudo-Riemannian manifold. The dimension of the ''negative'' subspace $N$ in the above splitting $T_pP = P\oplus N$ is called the index of the pseudo-Riemannian manifold.
    }

    \begin{theorem}[Whitney's embedding theorem]\index{Whitney!embedding theorem}
        Every smooth paracompact\footnote{See definition \ref{topology:paracompact}.} manifold $M$ can be embedded in $\mathbb{R}^{2\dim M}$.
    \end{theorem}
    \begin{theorem}[Whitney's immersion theorem]\index{Whitney!immersion theorem}
        Every smooth paracompact manifold $M$ can be immersed in $\mathbb{R}^{2\dim M - 1}$.
    \end{theorem}
    The following theorem is slightly stronger:
    \begin{theorem}[Immersion conjecture]\index{immersion!conjecture}
        Every smooth paracompact manifold $M$ can be immersed in $\mathbb{R}^{2\dim M - a(\dim M)}$ where $a(n)$ is the number of 1's in the binary expansion of $n$.
    \end{theorem}

    \newdef{Riemannian cone}{\index{Riemann!cone}\label{riemann:riemannian_cone}
        Let $(M, g)$ be a Riemannian manifold. Consider the product manifold $M\times\ ]0, +\infty[$. This manifold can also be turned into a Riemannian manifold by equipping it with the metric $t^2g+dt^2$. This manifold is called the Riemannian cone or \textbf{metric cone} of $(M, g)$.
    }

\subsection{Levi-Civita connection}

    \newdef{Riemannian connection}{\index{Riemann!connection}\index{Levi-Civita!connection|see{Riemann connection}}\label{riemann:levi_civita_connection}
        An affine connection $\nabla$ on a Riemannian manifold $(M, g)$ is said to be Riemannian if it satisfies the following two conditions:
        \begin{enumerate}
            \item $\nabla$ is \textbf{metric}:
            \begin{gather}
                X(g(Y, Z)) = g(\nabla_XY, Z) + g(Y, \nabla_XZ).
            \end{gather}
            \item $\nabla$ is \textbf{torsion-free}:
            \begin{gather}
                \nabla_XY - \nabla_YX = [X, Y].
            \end{gather}
        \end{enumerate}
        A Riemannian connection is often called a \textbf{Levi-Civita connection}.
    }

    Because the existence of a metric is equivalent to the integrability of an $\text{O}(n)$-structure, properties \ref{diff:prin:reducible_holonomy}, \ref{diff:prin:connection_reducibility} and \ref{diff:prin:integrable_torsion_free} imply that there exists a torsion-free connection that preserves this structure. The following theorem gives an even sharper result:
    \begin{theorem}[Fundamental theorem]
        Every Riemannian manifold $(M, g)$ admits a unique Levi-Civita connection.
    \end{theorem}

    \newformula{Koszul formula}{\index{Koszul!formula}
        The Levi-Civita connection $\nabla$ on a Riemannian manifold $(M, g)$ is implicitly defined by the following formula:
        \begin{align}
            2g(\nabla_XY, Z) &= \mathcal{L}_Xg(Y, Z) + d(\iota_Xg)(Y, Z)\\
            &= X(g(Y, Z)) + Y(g(Z, X)) - Z(g(X, Y))\nonumber\\
            &\hspace{3cm}+ g([X, Y], Z) - g([Z, X], Y) - g([Y, Z], X).
        \end{align}
    }

    The following (local) formula can be useful (especially in general relativity):
    \newformula{Divergence}{
        Consider the Riemannian volume form $\text{Vol}=\sqrt{g}dx^1\wedge\cdots\wedge dx^n$. The divergence of a vector field $X$ is defined as follows:
        \begin{gather}
            \mathcal{L}_X\text{Vol} =: \text{div}(X)\text{Vol}.
        \end{gather}
        In an orthonormal frame field on can also prove that this is equivalent to
        \begin{gather}
            \text{div}(X) = \frac{1}{2}\sum_{i=1}^n(\mathcal{L}_Xg)(e_i,e_i).
        \end{gather}
        Let $\nabla$ be the Levi-Civita connection on a given Riemannian manifold. Using the Koszul formula one can show that the above formula implies the following equality
        \begin{gather}
            \text{div}(X) = \text{tr}(Y\mapsto \nabla_YX) \equiv \nabla_\mu X^\mu.
        \end{gather}
        where $\text{tr}$ denotes the contraction (or trace) induced by $g$. This makes it clear that the covariant divergence is indeed a good generalization of the divergence \ref{vectorcalculus:divergence} from vector calculus.

        Using the metric determinant one can locally write the divergence in terms of the ordinary partial derivatives:
        \begin{gather}
            \label{diff:divergence_partial}
            \nabla_\mu V^\mu = \frac{1}{\sqrt{g}}\partial_\mu(\sqrt{g}V^\mu).
        \end{gather}
    }

    The geodesic equation \ref{diff:geodesic_equation} can be generalized as follows:
    \newdef{Geodesic}{\index{geodesic}\label{riemann:geodesic}
        A curve $\gamma$ on a Riemannian manifold $(M, g)$ that is autoparallel with respect to the Levi-Civita connection:
        \begin{gather}
            \nabla_{\dot{\gamma}}\dot{\gamma} = 0.
        \end{gather}
    }

    ?? COMPLETE (DIVERGENCE, HESSIAN, ...) ??

\subsection{Killing vectors}

    \newdef{Killing vector}{\index{Killing!vector}\label{diff:killing_vector}
        Let $(M, g)$ be a Riemannian manifold. A vector field $X$ satisfying
        \begin{gather}
            \mathcal{L}_Xg = 0
        \end{gather}
        is called a Killing vector field. A straightforward calculation gives us the following coordinate expression:
        \begin{gather}
            (\mathcal{L}_Xg)_{\mu\nu} = X^\lambda\partial_\lambda g_{\mu\nu} + g_{\lambda\mu}\partial_\nu X^\lambda + g_{\lambda\nu}\partial_\mu X^\lambda.
        \end{gather}
    }
    \begin{formula}
        Given a Levi-Civita connection $\nabla$ on $(M, g)$ we can rewrite the Killing condition as follows:
        \begin{gather}
            \nabla_{(\mu}X_{\nu)} = 0.
        \end{gather}
    \end{formula}

    \newdef{Killing tensor}{\index{Killing!tensor}
        Let $\nabla$ be the Levi-Civita connection on $(M, g)$. A tensor $T$ satisfying
        \begin{gather}
            \label{diff:killing_tensor}
            \nabla_{(\mu_N}T_{\mu_1\ldots\mu_{N-1})} = 0
        \end{gather}
        is called a Killing tensor. It is obvious that this \textbf{generalized Killing condition} is a direct generalization of the Killing condition as given above.
    }

\section{Curvature}\label{diff:section:curvature}

    \newformula{Riemann curvature tensor}{\index{Riemann!curvature}
        The Riemann (curvature) tensor $R$ is defined as the following $(1,3)$-tensor:
        \begin{gather}
            R(v, w)z := [\nabla_v, \nabla_w]z - \nabla_{[v, w]}z
        \end{gather}
        where $\nabla$ is the Levi-Civita connection. In index notation it is given by
        \begin{gather}
            R^i_{jkl}\nabla_i = R(e_k, e_l)e_j.
        \end{gather}
    }

    \begin{property}[Bianchi identity]\index{Bianchi identity}
        The first (or algebraic) Bianchi identity reads
        \begin{gather}
            R(X, Y)Z + R(Y, Z)X + R(Z, X)Y = 0.
        \end{gather}
        The second (or differential) Bianchi identity is a similar identity for the covariant derivative:
        \begin{gather}
            (\nabla_ZR)_{X, Y}W + (\nabla_XR)_{Y, Z}W + (\nabla_YR)_{Z, X}W = 0.
        \end{gather}
    \end{property}

    \newformula{Directional curvature operator\footnotemark}{\index{tidal force operator}
        \footnotetext{Also called the \textbf{tidal force operator} (mostly in physics).}
        \begin{gather}
            R_v(w) := R(w, v)v
        \end{gather}
    }

    \newformula{Sectional curvature}{\index{sectional!curvature}
        \begin{gather}
            \text{sec}(v, w) := \frac{g(R(w, v)v, w)}{g(v, v)g(w,w) - g(v, w)^2} = \frac{g(R_v(w), w)}{g(v\wedge w, v\wedge w)}
        \end{gather}
        An important result states that the sectional curvature only depends on the span of $v, w$.
    }
    \remark{For surfaces the sectional curvature coincides with the Gaussian curvature $K$ (see Gauss' Theorema Egregium \ref{diff:theorema_egregium}). Generally, the sectional curvature gives the Gaussian curvature of the plane spanned by the vectors $v, w$.}

    \newformula{Ricci tensor}{\index{Ricci!tensor}
        In coordinate-free notation one defines the Ricci tensor as the following trace:
        \begin{gather}
            R(v, w) := \text{tr}\big(x\mapsto R(x, v)w\big)
        \end{gather}
        or equivalently
        \begin{gather}
            R(v, w) = \sum_{i=1}^n g(R(e_i, v)w, e_i).
        \end{gather}
        In index notation this becomes
        \begin{gather}
            \label{diff:manifolds:ricci_tensor}
            R_{\mu\nu} := R^\lambda_{\ \mu\nu\lambda}.
        \end{gather}
        The Ricci tensor is also often denoted by $\text{Ric}$ (especially in Riemannian geometry). A $(1,1)$-tensor can be obtained by dropping the metric:
        \begin{gather}
            R(v) = \sum_{i=1}^n R(v, e_i)e_i.
        \end{gather}
    }
    \begin{property}
        The $(0,2)$-Ricci tensor can also be rewritten in terms of the sectional curvature in the case where $||v||=1$. Let $\{e_2,\ldots,e_n\}$ be a set of orthonormal vectors such that $\{v, e_2, \ldots, e_n\}$ forms an orthonormal basis.
        \begin{gather}
            R(v,v) = \sum_{i=2}^n \text{sec}(v, e_i).
        \end{gather}
        The Ricci tensor can thus be interpreted as an averaged (sectional) curvature.
    \end{property}

    \newformula{Ricci scalar}{\index{Ricci!scalar}\index{scalar!curvature}
        \begin{gather}
            \label{diff:manifolds:ricci_scalar}
            R := R^\mu_{\ \mu}
        \end{gather}
        This (scalar) quantity is also called the \textbf{scalar curvature}.
    }

    \newformula{Einstein tensor}{\index{Einstein!tensor}
        \begin{gather}
            \label{diff:manifolds:einstein_tensor}
            G_{\mu\nu} := R_{\mu\nu} - \frac{1}{2}g_{\mu\nu}R
        \end{gather}
    }
    \begin{property}
        For 4-dimensional manifolds the Einstein tensor $G_{\mu\nu}$ is the only tensor containing at most second derivatives of the metric $g_{\mu\nu}$ and satisfying
        \begin{gather}
            \nabla_\mu G^{\mu\nu} = 0.
        \end{gather}
    \end{property}

    \newdef{Einstein manifold}{\index{Einstein!manifold}\label{diff:einstein_manifold}
        A Riemannian manifold for which the Ricci tensor $R_{\mu\nu}$ is proportional to the metric $g_{\mu\nu}$.
    }
    \sremark{This name is justified by the fact that Einstein manifolds are exactly the solutions of the Einstein field equations \ref{rel:einstein_field_equations} with a cosmological constant.}

\section{Spinor bundles}\label{section:spinor_bundles}

    \sremark{In this section we only work with orientable (pseudo-)Riemannian manifolds because this assures that $M$ allows a reduction to a special orthogonal group and hence we can look at the associated $\text{Spin}$-group. For the more geenral definition of $\text{Pin}$-bundles we refer to \cite{AMP2}.}

\subsection{Spin structures}

    From definition \ref{linalgeba:vector_alternative} we know that one can define a vector $v\in V$ as an equivalence class of couples $(c, \mathfrak{b})$ where $\lambda\in\mathbb{C}^{\dim(V)}$ is a (coordinate) vector and $\psi$ is a (linear) frame of $V$. We will now extend this definition to Clifford algebras and spinors.\footnote{We haven't put this section in the chapter on Clifford algebras because the theory of spinors is in general always used in a manifold setting.}

    \newdef{Spinor}{\index{spinor}
        Let $V$ be a vector space equipped with a metric form $g$ of signature $(p,q)$. Consider the set $\mathbb{C}^{2^k}\times F_{SO}V\times\text{Spin}(p,q)$ where $k=\lfloor\frac{p+q}{2}\rfloor$ and $F_{SO}V$ is the set of orthonormal frames in $V$. One can define an equivalence relation on this set as follows: Two triples $(c_1, \mathfrak{b}_1, \Lambda_1)$, $(c_2, \mathfrak{b}_2, \Lambda_2)$ are said to be equivalent if and only if
        \begin{gather}
            c_2 = \Lambda c_1\qquad\qquad\mathfrak{b}_1=L\mathfrak{b}_2\qquad\qquad\Lambda=\Lambda_2\Lambda_1^{-1}\qquad\qquad\rho(\Lambda)=L
        \end{gather}
        where $\rho$ is the (2-to-1) covering map $\text{Spin}(p, q)\rightarrow\text{SO}(p, q)$. such an equivalence class is called a \textbf{spinor}. The $2^k$ numbers in $c_1$ are the components of this spinor in the \textbf{spin frame} $(\mathfrak{b}_1, \Lambda_1)$.

        It should be noted that the two components of a spin frame $(\mathfrak{b}, \Lambda)$ are not independent. Let us choose a \textbf{fiducial frame}\footnote{Different choices of fiducial frame give different, yet isomorphic, spinor spaces.} $(\mathfrak{b}_0, e)$ where $e$ is the identity element of $\text{Spin}(p, q)$. The couple $(\mathfrak{b}, \Lambda)$ is a well-defined spin frame if and only if $\rho(\Lambda)=L$ whenever $\mathfrak{b}=L\mathfrak{b}_0$.
    }

    \newdef{Spinor field}{
        Let $(M, g)$ be a (pesudo-)Riemannian manifold. Every tangent space $T_pM, p\in M$ is a vector space equipped with a nondegenerate bilinear form and hence we can use the above definition to construct a spinor space at $p$. If the orthonormal frame bundle $F_{SO}(M)$ is trivial then we can smoothly choose a section $p\mapsto\mathfrak{b}(p)$ and define the fiducial frame field to be $p\mapsto(\mathfrak{b}(p), e)$.

        However, if $F_{SO}(M)$ is not trivial, this construction only works on a local chart. To be able to extend it to the whole manifold, we need to patch the different frame fields together. The required compatibility conditions reads as follows:
        \begin{gather}
            \rho(\Lambda_i(x)\Lambda_j^{-1}(x))=L_{ij}(x)
        \end{gather}
        if
        \begin{gather}
            \mathfrak{b}_i(x) = L(x)\mathfrak{b}_j(x)
        \end{gather}
        for all $x\in U_i\cap U_j$.
    }

    One can show that a manifold admits the definition of a global spin frame field if and only if it admits a spin structure:
    \newdef{Spin structure}{\index{spin!structure}\index{spin!frame}
        Consider the (oriented) orthonormal frame bundle \[\pi_{SO}:F_{SO}(M)\rightarrow M\] which is obtained by reducing the structure group of the frame bundle $F(M)$ from GL$(n)$ to SO$(n)$. Furthermore, let $\pi_{spin}:P_{spin}\rightarrow M$ be a principal Spin$(n)$-bundle over $M$.

        The smooth manifold $M$ is said to have a spin structure if there exists an equivariant 2-fold lifting of $F_{SO}$ to $P_{spin}$, i.e. a morphism $\xi:P_{spin}\rightarrow F_{SO}(M)$ together with the 2-fold covering map $\rho:\text{Spin}(n)\rightarrow\text{SO}(n)$ that satisfy
        \begin{itemize}
            \item $\pi_{SO}\circ\xi = \pi_{spin}$, and
            \item $\xi(p\vartriangleleft g) = \xi(p)\cdot\rho(g)$
        \end{itemize}
        for all $g\in\text{Spin}(n)$, where $\vartriangleleft$ and $\cdot$ denote the right actions of the respective structure groups. If $M$ admits a spin structure it is often called a \textbf{spin manifold} and the principal Spin$(n)$-bundle $P_{spin}$ is called the \textbf{spin frame bundle}.
    }

    \newdef{Spin bundle}{\index{spin!bundle}\index{spinor}
        A spin bundle is a vector bundle associated to a spin frame bundle. Sections of a spin bundle are called spinor fields.
    }

    The classification of spin manifolds can be phrased in terms of characteristic classes. Instead of the usual $\mathbb{R}$- or $\mathbb{Z}$-valued cohomology classes, we will now introduce some classes in $\mathbb{Z}_2$-cohomology.
    \begin{property}
        A smooth manifold is orientable if and only if its first Stiefel-Whitney class vanishes.
    \end{property}
    \begin{property}
        A smooth orientable manifold $M$ is spin if and only if its second Stiefel-Whitney class vanishes. Furthermore, the distinct spin structures form an affine space over $H^1(M, \mathbb{Z}_2)$.
    \end{property}
    \begin{property}
        A special case occurs when $\dim M = 3$. We then have that $M$ is spin if it is compact and orientable.
    \end{property}

\subsection{Dirac operators}

    In this section we will generalize the partial derivatives $\partial_i$ and gradient operator $\sum_{i=1}^ne_i\partial_i$ to Clifford algebras and Clifford modules.

    \newdef{Clifford bundle}{\index{Clifford!bundle}
        Consider a (pseudo-)Riemannian manifold $(M, g)$ of signature $(p, q)$. For every point $p\in M$ one can construct a Clifford algebra associated to the tangent space $(T_pM, g_p)$. Using these Clifford algebras one can construct an associated bundle to $TM$ which has $C\ell_{p,q}(\mathbb{R})$ as its typical fibre. In general one calls a vector bundle with a Clifford algebra as typical fibre, for which the local trivializations respect the algebra structure, a Clifford bundle.\footnote{Note that one can use this construction to turn any vector bundle that admits a fibre metric into a Clifford bundle.}
    }

    \newdef{Clifford module bundle}{\index{Clifford!module}
        Consider a (pseudo-)Riemannian manifold $(M, g)$ with its associated Clifford bundle $C\ell(TM)$. Any vector bundle that admits a left $C\ell(TM)$-action is called a Clifford module (bundle) over $M$.
    }

    Let us first define the Dirac operator on Euclidean space $\mathbb{R}^n$. The Dirac operator is obtained by composing the ordinary gradient
    \begin{gather}
        \partial:=\sum_{i=1}^ne_i\partial_i
    \end{gather}
    with the morphism $\iota_{C\ell}:e_i\mapsto\gamma_i$ that sends a basis for $\mathbb{R}^n$ to the corresponding generators of $C\ell_n(\mathbb{R})$:
    \begin{gather}
        \underline{\partial} := \sum_{i=1}^n\gamma_i\partial_i.
    \end{gather}

    We would now like to extend this definition to Clifford modules. As usual this will be done by replacing partial derivatives by covariant derivatives.
    \begin{property}
        Let $(M, g)$ be a (pseudo-)Riemannian manifold and let $\nabla$ be the associated Levi-Civita connection. For every Clifford module $E$ over $M$ there exists a unique connection $\nabla^E:\Gamma(E)\rightarrow \Gamma(T^*M\otimes E)$ that respects the Clifford action:
        \begin{gather}
            \nabla^E(\iota_{C\ell}(X)\cdot\sigma) = \iota_{C\ell}(\nabla X)\cdot\sigma + \iota_{C\ell}(X)\cdot\nabla^E\sigma
        \end{gather}
        where $\iota_{C\ell}:TM\rightarrow C\ell(TM)$ is the canonical map that embeds a vector field in $C\ell(TM)$.
    \end{property}

    \newdef{Dirac operator}{\index{Dirac!operator}
        Consider a (pseudo-)Riemannian manifold $(M, g)$ together with a Clifford module $E$. If $\nabla^E$ is the compatible connection from the previous property then we define the Dirac operator as follows:
        \begin{gather}
            \underline{D} := \sum_{i=1}^n\iota_{C\ell}(e_i)\cdot\nabla^E_{e_i}\sigma.
        \end{gather}
        where $\{e_i\}_{i\leq n}$ is a local (orthonormal) frame field.
    }
    \begin{property}
        The Dirac operator is a self-adjoint elliptic differential operator.
    \end{property}

\subsection{Index theorem}

    \begin{property}
        The $\hat{A}$-genus \ref{diff:prin:a_roof_genus} of a spin manifold is an integer.
    \end{property}

    ?? COMPLETE ??

\section{Conformal structures}
\subsection{Conformal transformations}

    \newdef{Conformal transformation}{
        Consider two (pseudo-)Riemannian manifolds $(M, g)$ and $(M', g')$. A smooth map $f:M\rightarrow M'$ is said to be conformal if it leaves the metric invariant up to a scale transformation:\footnote{Compare this to definition \ref{diff:conformal_map}.}
        \begin{gather}
            f^*g' = \Omega g
        \end{gather}
        for some smooth positive function $\Omega:M\rightarrow\mathbb{R}^+$. If $f$ is a diffeomorphism then it is called a \textbf{conformal transformation}. These (restricted) diffeomorphisms are the morphisms in the category of smooth manifolds equipped with a conformal structure.
    }

    Infinitesimally these maps are described by a specific type of vector field:
    \newdef{Conformal Killing vector}{\index{Killing!conformal vector}
        Consider a pseudo-Riemannian manifold $(M, g)$. A vector field $X$ is called a conformal Killing vector field (with conformal factor $\kappa:M\rightarrow\mathbb{R}$) if it satisfies
        \begin{equation}
            \mathcal{L}_Xg = \kappa g.
        \end{equation}
        In local coordinates this amounts to
        \begin{equation}
            \nabla_\mu X_\nu + \nabla_\nu X_\mu = \kappa g_{\mu\nu}
        \end{equation}
        where $\nabla$ is the Levi-Civita connection associated to $(M, g)$. Equivalently, a vector field is a conformal Killing vector field if its flow determines a conformal transformation.
    }

    By parametrizing an infinitesimal transformation as $x^\mu\rightarrow x^\mu+\varepsilon^\mu$ one obtains the following infinitesimal generators:
    \begin{itemize}
        \item \textbf{Translations}: $a^\mu\partial_\mu$,
        \item \textbf{Rotations} (orthogonal transformations): $\omega^\mu_{\ \nu} x^\nu\partial_\mu$,
        \item \textbf{Dilations}: $\lambda x^\mu\partial_\mu$, and
        \item \textbf{Special conformal transformations}: $x^2b^\mu\partial_\mu - 2(b\cdot x)x^\mu\partial_\mu$.
    \end{itemize}
    As usual, exponentiating these generators gives the finite transformations. One immediately notices that the Poincar\'e group is a subgroup of the conformal group.

    \newdef{Conformal group}{\index{conformal!group}
        \nomenclature[S_CONF]{$\text{Conf}(M)$}{conformal group of (pseudo-)Riemannian manifold $M$}
        The conformal group of a pseudo-Riemannian manifold $M$ is not just the group of conformal transformations of $M$. The conformal group $\text{Conf}(M)$ is defined as the connected component of the identity element in the conformal diffeomorphism group of the conformal compactification of $M$.
    }
    \begin{property}
        The conformal group of (pseudo-)Euclidean space in signature $(p, q)$ is isomorphic to $\text{SO}(p+1, q+1)$.
    \end{property}

\section{\difficult{Hilbert bundles}}

    \newdef{Hilbert bundle}{\index{Hilbert!bundle}
        A vector bundle where the typical fibre is a Hilbert space.
    }
    \newdef{Compatible Hilbert bundle}{
        Consider the isomorphisms
        \begin{gather}
            l_x^{-1}:\mathcal{H}\rightarrow F_x:h\mapsto\varphi_i^{-1}(x, h)\in\pi^{-1}(x)
        \end{gather}
        where $\mathcal{H}$ is the typical fibre and where $\{(U_i, \varphi_i)\}_{i\in I}$ is a trivializing cover. The maps $l_x$ are called \textbf{point-trivializing maps}.

        Using these maps we can extend the metric structure of the typical fibre $\mathcal{H}$ to the fibres $F_x$ for all $x$ by:
        \begin{gather}
            \langle v|w \rangle_x := \langle l_x(v)|l_x(w) \rangle_{\mathcal{H}}.
        \end{gather}
        The Hilbert bundle is said to be compatible (with the metric structure on $\mathcal{H}$) if the above extension is valid for all $v, w \in F_x$.
    }

    \begin{remark*}
        For compatible Hilbert bundles, the transition maps $l_{x\rightarrow y} = l_y^{-1}\circ l_x:\pi^{-1}(x)\rightarrow\pi^{-1}(y)$ are easily seen to be isometries.
    \end{remark*}

    ?? COMPLETE (B. Iliev OR S. Lang) ??
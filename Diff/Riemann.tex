\chapter{Riemannian Geometry}

\section{Riemannian manifolds}
\subsection{Metric}

	\begin{definition}[Bundle metric]\index{metric!bundle}
		Consider the bundle of second order covariant vectors. Following from \ref{tensor:tensor_product} every section $g$ of this bundle gives a bilinear map \[g_x:T_xM\times T_xM\rightarrow\mathbb{R}\]
		for all $x\in M$. If this map is symmetric and non-degenerate and if it depends smoothly on $p$ it is called a \textbf{(Lorentzian}) metric.\footnote{See also the section about Hermitian forms and metric forms \ref{linalgebra:innerproduct}.}
		
		The maps $\{g_x\}_{x\in M}$ can be `glued' together to form a global metric $g$, defined on the fibre product\footnotemark\ $TM\diamond TM$. Defining this map on $TM\times TM$ is not possible as tangent vectors belonging to different points in $M$ cannot be `compared'. The collection $\{\langle\cdot|\cdot\rangle_x|x\in M\}$ is called a \textbf{bundle metric}\index{metric!bundle}.
		\footnotetext{See definition \ref{manifolds:fibre_product}.}
	\end{definition}

	A Riemannian metric also induces a duality between $TM$ and $T^*M$. This is given by the \textit{flat} and \textit{sharp} isomorphisms:
	\newdef{Musical isomorphisms}{\index{musical isomorphism}
		Let $g:TM\times TM\rightarrow\mathbb{R}$ be the Riemannian metric on $M$. The \textbf{flat} isomorphism is defined as:
		\begin{equation}
			\label{manifolds:flat_map}
			\flat:v\mapsto g(v, \cdot)
		\end{equation}
		The \textbf{sharp} isomorphism is defined as the inverse map:
		\begin{equation}
			\label{manifolds:sharp_map}
			\sharp:p\mapsto v
		\end{equation}
		such that $p(\cdot) = g(v, \cdot)$. These 'musical' isomorphisms can be used to lower and raise tensor indices.
	}

\subsection{Riemannian manifold}

	\newdef{Pseudo-Riemannian manifold}{\index{Riemann!Riemannian manifold}\label{riemann:riemannian_manifold}
		Let $M$ be a smooth manifold. This manifold is called pseudo-Riemannian if it is equipped with a pseudo-Riemannian metric. A \textbf{Riemannian manifold} is similarly defined.
	}
	
	\newdef{Riemannian isometry}{\index{isometry}
		Let $(M, g_M)$ and $(N, g_N)$ be two Riemannian manifolds. An isometry \ref{diff:isometry_def} $f:M\rightarrow N$ is said to be Riemannian if $F^*g_N = g_M$.
	}
	
	\begin{property}\index{index}
		Let $M$ be a pseudo-Riemannian manifold. For every $p\in M$ there exists a splitting $T_pM = P\oplus N$ where $P$ is a subspace on which the pseudometric is positive-definite and $N$ is a subspace on which the pseudometriv is negative-definite. This splitting is however not unique, only the dimensions of the two subspaces are well-defined.
	\end{property}
	Due to the continuity of the pseudometric, the dimensions of this splitting wil be the same for points in the same neighbourhood. For connected manifolds this amounts to a global invariant:
	\newdef{Index}{
		Let $M$ be a connected pseudo-Riemannian manifold. The dimension of the \textit{negative} subspace $N$ in the above splitting $T_pP = P\oplus N$ is called the index of the pseudo-Riemannian manifold.
	}
	
	\begin{theorem}[Whitney]\index{Whitney's theorem}
		Every smooth paracompact\footnotemark\ manifold admits a Riemannian metric.
		\footnotetext{See definition \ref{topology:paracompact}.}
	\end{theorem}

\section{Sphere bundle}

	\newdef{Unit sphere bundle}{\index{unit!sphere bundle}
		Let $V$ be a normed vector space. Consider a vector bundle $\prin{V}{E}{B}$. From this bundle we can derive a new bundle where we replace the typical fibre $V$ by the unit sphere $\{v\in V : ||v|| = 1\}$. It should be noted that this new bundle is not a vector bundle as the unit sphere is not a vector space.
	}
	\begin{remark}[Unit disk bundle]\index{unit!disk bundle}
		A similar construction can be made by replacing the unit sphere by the unit disk $\{v\in V : ||v||\leq1\}$.
	\end{remark}

\section{Hilbert bundles}
	
	\newdef{Hilbert bundle}{\index{Hilbert!bundle}
		A vector bundle for which the typical fibre is a Hilbert space is called a Hilbert bundle.
	}
	\newdef{Compatible Hilbert bundle}{
		Consider the isomorphisms
		\begin{equation}
			l_x:F_x\rightarrow\mathcal{H}:h\mapsto\varphi_i(x, h)\in\pi(x)
		\end{equation}
		where $\mathcal{H}$ is the typical fibre and where $\{(U_i, \varphi_i)\}_{i\in I}$ is a trivializing cover. These maps $l_x$ are called \textbf{point-trivializing maps}.
		
		Using these maps we can extend the metric structure of the typical fibre $\mathcal{H}$ to the fibres $F_x$ for all $x$ by:
		\begin{equation}
			\langle v|w \rangle_x = \langle l_x(v)|l_x(w) \rangle_{\mathcal{H}}
		\end{equation}
		The Hilbert bundle is said to be compatible (with the metric structure on $\mathcal{H}$) if the above extension is valid for all $v, w \in F_x$.
	}
	
	\begin{remark*}
		For compatible Hilbert bundles, the transition maps $l_{x\rightarrow y} = l_y^{-1}\circ l_x:\pi^{-1}(x)\rightarrow\pi^{-1}(y)$ are easily seen to be isometries.
	\end{remark*}

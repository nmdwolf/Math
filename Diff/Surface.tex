\chapter{Curves and Surfaces}\label{chapter:curves_surfaces}
\section{Curves}

    \newdef{Regular curve}{\index{regular!curve}
        Let $c(t):I\rightarrow\mathbb{R}^n$ be a curve defined on an interval $I$. $c(t)$ is said to be regular\footnote{See also property \ref{manifolds:regular_point}.} if $\deriv{c}{t}(t)\neq\mathbf{0}$ for all $t\in I$.
    }

    \newdef{Geometric property}{A geometric property is a property that is invariant under:
        \begin{enumerate}
            \item parameter transformations
            \item orientation-preserving (orthonormal) basis transformations
        \end{enumerate}
    }

    \begin{property}
        Let $c(t), d(t)$ be two curves with the same image. The following relation holds for all $t$:
        \begin{gather}
            c(t)\text{ regular}\iff d(t)\text{ regular}.
        \end{gather}
    \end{property}

\subsection{Arc length}

    \newdef{Natural parameter}{\index{natural!parameter}\label{diff:natural_parameter}
        Let $c(t)$ be a curve. The parameter $t$ is said to be a natural parameter if
        \begin{gather}
            \left|\left|\deriv{c}{t}\right|\right|=1
        \end{gather}
        for all values of $t$.
    }

    \begin{formula}[Arc length]\index{arc!length}\label{diff:arc_length_integral}
        The following function is a bijective map and a natural parameter for the curve $c$:
        \begin{gather}
            \phi(t) := \int_{t_0}^t||\dot{c}(t)||dt.
        \end{gather}
    \end{formula}
    \sremark{The arc length as defined above is often denoted by the letter $s$.}

    \begin{property}
        Let $c(t)$ be a curve and let $u$ be an alternative parameter of $c(t)$. It is a natural parameter if and only if there exists a constant $\alpha$ such that:\[u = \pm s + \alpha\] where $s$ is the integral as defined in equation \ref{diff:arc_length_integral}.
    \end{property}
    \begin{remark*}
        This property implies that there does not exist a unique natural parameter for any curve.
    \end{remark*}

\subsection{Frenet-Serret frame}

    \newdef{Tangent vector}{\index{tangent!vector}\label{diff:tangent_vector}
        Let $c(s)$ be a curve parametrized by arc length. The tangent vector (field) $t(s)$ is defined as follows:
        \begin{gather}
            t(s) := c\,'(s).
        \end{gather}
    }
    \begin{property}\label{diff:unit_tangent_vector}
        From the definition of the natural parametrization \ref{diff:natural_parameter} and the previous definition it follows that the tangent vector is automatically a unit vector.
    \end{property}

    \newdef{Principal normal vector}{\index{normal!vector}\label{diff:principal_normal_vector}
        Let $c(s)$ be a curve parametrized by arc length. The principal normal vector (field) is defined as follows:
        \begin{gather}
            n(s) := \stylefrac{t\,'(s)}{||t\,'(s)||}.
        \end{gather}
    }
    \begin{property}
        From property \ref{diff:unit_tangent_vector} and the definition of the principal normal vector it follows that the tangent vector and principal normal vector are always orthogonal.
    \end{property}

    \newdef{Binormal vector}{\label{diff:binormal_vector}
        Let $c(s)$ be a curve parametrized by arc length. The binormal vector (field) is defined as follows:
        \begin{gather}
            b(s) := t(s)\times n(s).
        \end{gather}
    }

    \newdef{Frenet-Serret frame}{\index{Frenet-Serret}\label{diff:frenet_serret_frame}
        Because the vectors $t(s),n(s)$ and $b(s)$ are mutually orthonormal and linearly independent, we can use them to construct an oriented orthonormal basis. The ordered basis $\big\{t(s),n(s),b(s)\big\}$ is called the Frenet-Serret frame.
    }
    \sremark{This basis does not have to be the same for every value of the parameter $s$.}

    \newdef{Curvature}{\index{curvature}\label{diff:curvature}
        Let $c(s)$ be a curve parametrized by arc length. The curvature of $c(s)$ is defined as follows:
        \begin{gather}
            \stylefrac{1}{\rho(s)} := ||t\,'(s)||.
        \end{gather}
    }
    \newdef{Torsion}{\index{torsion}\label{diff:torsion}
        Let $c(s)$ be a curve parametrized by arc length. The torsion of $c(s)$ is defined as follows:
        \begin{gather}
            \tau(s) := \rho(s)^2(t\ \ t\,'\ \ t\,'')
        \end{gather}
        where $(a b c)$ denotes the \textit{triple product} $a\cdot(b\times c)$.
    }

    \newformula{Frenet formulas}{\index{Frenet formulas}\label{diff:frenet_formulas}
        The derivatives of the tangent, principal normal and binormal vector fields can be written as a linear combination of those vectors themselves:
        \begin{gather}
            \left\{
            \begin{array}{ccccc}
                t\,'(s) &=&& \stylefrac{1}{\rho(s)}n(s)&\\
                n\,'(s) &=& -\stylefrac{1}{\rho(s)}t(s) &+& \tau(s)b(s).\\
                b\,'(s) &=& &-\tau(s)n(s)&
            \end{array}
            \right.
        \end{gather}
    }

    \begin{theorem}[Fundamental theorem for curves]\index{fundamental theorem!for curves}\label{diff:theorem:fundamental_theorem_of_curves}
        Let $k(s),w(s):U\rightarrow\mathbb{R}$ be two $C^1$-functions with $k(s)\geq 0,\forall s\in U$. There exists an interval $]-\varepsilon, \varepsilon[\ \subset U$ and a curve $c(s):\ ]-\varepsilon,\varepsilon[\ \rightarrow\mathbb{R}^3$ with natural parameter $s$ such that $c(s)$ has $k(s)$ as its curvature and $w(s)$ as its torsion.
    \end{theorem}

\section{Surfaces}

    \begin{notation}
        Let $\sigma(q^1, q^2)$ be the parametrization of a surface\footnote{The symbol $\sigma$ denotes the surface as a vector field while $\Sigma$ denotes the geometric image of $\sigma$.}. The derivative of $\sigma$ with respect to the coordinate $q^i$ is written as follows:
        \begin{gather}
            \label{diff:derivative_of_surface}
            \sigma_i:=\pderiv{\sigma}{q^i}.
        \end{gather}
    \end{notation}

    \newdef{Tangent plane}{\index{tangent!plane}\label{diff:tangent_space}
        Let $P(q_0^1,q_0^2)$ be a point on the surface $\Sigma$. The tangent space $T_P\Sigma$ to $\Sigma$ in $P$ is defined as follows:
        \begin{gather}
            T_P\Sigma:=\left\{v\in\mathbb{R}^3:\left[v - \sigma(q_0^1,q_0^2)\right]\cdot\left[\sigma_1(q_0^1,q_0^2)\times\sigma_2(q_0^1,q_0^2)\right] = 0\right\}.
        \end{gather}
    }

    \newdef{Normal vector}{\index{normal!vector}
        The cross product in equation \ref{diff:tangent_space} is closely related to the normal vector to $\Sigma$ at the point $P$. The normal vector at the point $(q_0^1, q_0^2)$ is defined as
        \begin{gather}
            \label{diff:surface_normal_vector}
            N := \stylefrac{\sigma_1\times\sigma_2}{||\sigma_1\times\sigma_2||}.
        \end{gather}
        This way we see that the tangent plane $T_P\Sigma$ is exactly the set of vectors that are orthogonal to the normal vector $N(q^1_0, q^2_0)$.
    }

\subsection{First fundamental form}

    \newdef{Metric coefficients}{\index{metric!coefficients}
        Let $\sigma$ be the parametrization of a surface. The metric coefficients $g_{ij}$ are defined as follows:
        \begin{gather}
            \label{diff:metric_coefficient}
            g_{ij} := \sigma_i\cdot\sigma_j.
        \end{gather}
    }
    \newdef{Scale factor}{\index{scale!factor}\label{diff:scale_factor}
        The following factors are often used in vector calculus:
        \begin{gather}
            g_{ii} =: h_i^2.
        \end{gather}
    }

    \newdef{First fundamental form}{\index{fundamental!form}\label{diff:first_fundamental_form}\index{metric}
        Let $\sigma$ be the parametrization of a surface. We can define a bilinear form $I_P:T_P\Sigma\times T_P\Sigma\rightarrow\mathbb{R}$ that restricts the inner product on $mathbb{R}^3$ to $T_P\Sigma$:
        \begin{gather}
           I_P(v,w) := v\cdot w.
       \end{gather}
        This bilinear form is called the first fundamental form or \textbf{metric}.
    }
    \begin{result}
        All vectors $v,w\in T_P\Sigma$ are linear combinations of the tangent vectors $\sigma_1,\sigma_2$. This enables us to relate the first fundamental form and the metric coefficients \ref{diff:metric_coefficient}:
        \begin{gather}
            I_P(v, w) = v^i\sigma_i\cdot w^j\sigma_j = g_{ij}v^iw^j.
        \end{gather}
    \end{result}

    \begin{notation}
        The (arc) length \ref{diff:arc_length_integral} of a curve $c(t)$ can be written as follows:
        \begin{gather}
            s = \int||\dot{c}(t)||dt = \int\sqrt{ds^2}
        \end{gather}
        where the second equality is formally defined. The two equalities together can be combined into the following notation for the metric (which is often used in physics):
        \begin{gather}
            ds^2 := g_{ij}dq^idq^j.
        \end{gather}
    \end{notation}

    \begin{formula}[Inverse metric]\label{diff:inverse_metric_matrix}
        Let $(g_{ij})$ denote the metric tensor. We define the matrix $(g^{ij})$ as its inverse:
        \begin{gather}
            (g^{ij}) := \stylefrac{1}{\det(g_{ij})} \begin{pmatrix} g_{22}&-g_{12}\\ -g_{12}&g_{11} \end{pmatrix}.
        \end{gather}
    \end{formula}

\subsection{Isometries}

    \newdef{Isometry}{\index{isometry}\label{diff:isometry_def}
        An isometry is a distance-preserving map, i.e. a smooth map $\Phi:\Sigma\rightarrow\Sigma'$ that maps arc segments in $\Sigma$ to arc segments with the same length in $\Sigma'$. In differential geometry this map is usually assumed to be diffeomorphic.
    }
    \begin{property}\label{diff:isometry}
        A diffeomorphism $\Phi$ is an isometry if and only if the metric coefficients of $\sigma$ and $\sigma'$ are the same.
    \end{property}

    \newdef{Conformal map}{\index{conformal}\label{diff:conformal_map}
        A diffeomorphism $\Phi:\Sigma\rightarrow\Sigma'$ is said to be conformal or \textbf{isogonal} if it maps two intersecting curves in $\Sigma$ to intersecting curves in $\Sigma'$ with the same intersection angle.
    }
    \begin{property}
        A diffeomorphism $\Phi$ is conformal if and only if the metric coefficients of $\sigma$ and $\sigma'$ are proportional.
    \end{property}

    \newdef{Area-preserving map}{
        A diffeomorphism $\Phi:\Sigma\rightarrow\Sigma'$ is said to be area-preserving if it maps a subset of $\Sigma$ to a subset of $\Sigma'$ with the same area.
    }
    \begin{property}
        A diffeomorphism $\Phi$ is area-preserving if and only if the metric coefficients of $\sigma$ and $\sigma'$ satisfy
        \begin{gather}
            g_{11}'g_{22}' - (g_{12}')^2 = g_{11}g_{22} - g_{12}^2
        \end{gather}
        for all points $(q^1, q^2)$, i.e. if it preserves the determinant of the metric.
    \end{property}
    \result{A map that is area-preserving and conformal is also isometric.}

\subsection{Second fundamental form}

    \newdef{Second fundamental form}{\index{fundamental!form}\label{diff:second_fundamental_form}
        Let $\sigma$ be the parametrization of a surface. The second fundamental form is the bilinear form $II_P:T_P\Sigma\times T_P\Sigma\rightarrow\mathbb{R}$ defined as follows:
        \begin{gather}
            II_P(v,w) := L_{ij}(q^1, q^2)v^iw^j
        \end{gather}
        where $L_{ij} := N\cdot\sigma_{ij}$.
    }

    \newdef{Normal curvature}{\index{normal!curvature}\index{curvature!normal}
        Let $c$ be a curve embedded as \[c(s) := \sigma\left(q^1(s), q^2(s)\right).\] The normal curvature of $c$ at the point $\left(q^1(s), q^2(s)\right)$ is defined as
        \begin{gather}
            \label{diff:normal_curvature}
            \stylefrac{1}{\rho_n(s)} := c\ ''(s)\cdot N(s).
        \end{gather}
        From the definition of the second fundamental form it follows that the normal curvature can be written as follows:
        \begin{gather}
            \stylefrac{1}{\rho_n(s)} = II(t, t) = \stylefrac{II\left(\dot{c}(\lambda), \dot{c}(\lambda)\right)}{I\left(\dot{c}(\lambda), \dot{c}(\lambda)\right)}
        \end{gather}
        where the last equality holds for any given parameter $\lambda$.
    }

    \begin{theorem}[Meusnier]\index{Meusnier}\index{osculating circle}
        Let $c,d$ be two curves defined on a surface $\sigma$. The curves have the same normal curvature at the point $\left(q^1(t_0), q^2(t_0)\right)$ if the following two conditions are satisfied:
        \begin{itemize}
            \item $c(t_0) = d(t_0)$
            \item $\dot{c}(t_0)\ ||\ \dot{d}(t_0)$
        \end{itemize}
        Furthermore, the \textit{osculating circles} of all curves with the same normal curvature at a given point form a sphere.
    \end{theorem}

    \begin{property}\index{normal!section}
        The normal curvature of a \textbf{normal section}\footnote{The intersection of the surface with a normal plane at the point.} at a given point is equal to the curvature of the section at that point.
    \end{property}

    \newdef{Geodesic curvature}{\index{curvature!geodesic}
        Let $c$ be a curve embedded as \[c(s) := \sigma\left(q^1(s), q^2(s)\right).\] The geodesic curvature of $c$ at the point $\left(q^1(s), q^2(s)\right)$ is defined as follows:
        \begin{gather}
            \label{diff:geodesic_curvature}
            \stylefrac{1}{\rho_g(s)} := \left(N(s)\ t(s)\ t'(s)\right).
        \end{gather}
    }

    \begin{formula}
        Let $c$ be a curve defined on a surface $\sigma$. From the definitions of the normal and geodesic curvature it follows that
        \begin{gather}
            \stylefrac{1}{\rho^2} = \stylefrac{1}{\rho^2_n} + \stylefrac{1}{\rho^2_g}.
        \end{gather}
    \end{formula}

\subsection{Curvature of a surface}

    \newdef{Weingarten map\footnotemark}{\index{Weingarten map}\index{shape!operator}
        \footnotetext{Sometimes called the \textbf{shape operator}.}
        Let $P$ be a point on the surface $\Sigma$. The Weingarten map $L_P:T_P\Sigma\rightarrow T_P\Sigma$ is the linear map defined as follows:
        \begin{gather}
            \label{diff:weingarten_map}
            L_P(\sigma_1) := -N_1\qquad\text{and}\qquad L_P(\sigma_2) := -N_2.
        \end{gather}
    }
    \begin{formula}
        Let $v, w\in T_P\Sigma$. The following equalities relate the second fundamental form and the Weingarten map:
        \begin{gather}
            L_P(v)\cdot w = L_P(w)\cdot v = II_P(v, w).
        \end{gather}
    \end{formula}

    \begin{formula}
        Let $\left(g^{ij}\right)$ be the inverse of the metric. The matrix elements of $L_P$ can be expressed as follows:
        \begin{gather}
            L^k_j = g^{ki}L_{ij}.
        \end{gather}
    \end{formula}
    \begin{formula}[Weingarten formulas]\index{Weingarten!formulas}
        \begin{gather}
            N_j = -L^k_j\sigma_k
        \end{gather}
    \end{formula}

    \newdef{Principal curvatures}{\index{curvature!principal}\label{diff:theorem:principal_directions}
        The eigenvalues of the Weingarten map are called the principal curvatures of the surface and they are denoted by \[\stylefrac{1}{R_1}\quad\text{and}\quad\stylefrac{1}{R_2}.\] Let $h_1, h_2$ denote the eigenvectors of $L_P$. By the formulas above, the principal curvatures are given by $II_P(h_i, h_i)$ and they are the extreme values of the normal curvature. The associated tangent vectors are called the \textbf{principal directions}. Furthermore, they form a basis for the tangent plane.
    }
    \begin{property}\index{umbilical point}
        If the principal curvatures at a point $P$ are not equal, the principal directions are orthogonal. If they are equal, the point $P$ is said to be an \textbf{umbilical point} or \textbf{umbilic}.
    \end{property}

    \newdef{Line of curvature}{\index{curvature!line}
        A curve is said to be a line of curvature if the tangent vector at every point $P$ is a principal direction of the surface at $P$.
    }

    \newformula{Rodrigues' formula}{\index{Rodrigues' formula}
        A curve is a line of curvature if and only if it is a solution of the following differential equation:
        \begin{gather}
            \label{diff:rodrigues_formula}
            \deriv{N}{t}(t) = -\stylefrac{1}{R(t)}\deriv{c}{t}(t).
        \end{gather}
        If the curve satisfies this formula, then the scalar function $1/R(t)$ coincides with the principal curvature along the curve.
    }
    \newformula{Differential equation for curvature lines}{
        \begin{gather}
            \left|
            \begin{array}{ccc}
                (\dot{q}^2)^2&-\dot{q}^1\dot{q}^2&(\dot{q}^1)^2\\
                g_{11}&g_{12}&g_{22}\\
                L_{11}&L_{12}&L_{22}
            \end{array}
            \right| = 0
        \end{gather}
    }
    \begin{property}
        From definition \ref{diff:theorem:principal_directions} we know that the principal directions are determined by orthogonal vectors. It follows that on a surface containing no umbilics the curvature lines form an orthogonal web.
    \end{property}

    \newdef{Gaussian curvature}{\index{curvature!Gaussian}\label{diff:gaussian_curvature}
        The Gaussian curvature $K$ of a surface is defined as the determinant of the Weingarten map:
       \begin{gather}
            K := \stylefrac{1}{R_1R_2}.
        \end{gather}
    }
    \newdef{Mean curvature}{\index{curvature!mean}\label{diff:mean_curvature}
        The mean curvature $H$ of a surface is defined as the trace of the Weingarten map:
        \begin{gather}
            H := \stylefrac{1}{2}\left(\stylefrac{1}{R_1} + \stylefrac{1}{R_2}\right).
        \end{gather}
    }
    \begin{property}
        The principal curvatures are the solutions of the following equation:
        \begin{gather}
            x^2 - 2Hx + K = 0.
        \end{gather}
            This is the characteristic equation \ref{linalgebra:characteristic_equation} of the Weingarten map.
    \end{property}

    \begin{definition}
        Let $P$ be a point on the surface $\Sigma$.
        \begin{itemize}
            \item $P$ is said to be \textbf{elliptic} if $K > 0$ in $P$.
            \item $P$ is said to be \textbf{hyperbolic} if $K < 0$ in $P$.
            \item $P$ is said to be \textbf{parabolic} if $K = 0$ and $\stylefrac{1}{R_1}$ or $\stylefrac{1}{R_2}\neq0$ in $P$.
            \item $P$ is said to be \textbf{flat} if $\stylefrac{1}{R_1} = \stylefrac{1}{R_2} = 0$ in $P$.
            \item $P$ is said to be \textbf{umbilical} if $\stylefrac{1}{R_1} = \stylefrac{1}{R_2}$ in $P$.
        \end{itemize}
    \end{definition}
    \sremark{From previous definition it follows that a flat point is a special type of umbilic.}

    \begin{property}
        A surface $\Sigma$ containing only umbilics is either part of a sphere or part of a plane.
    \end{property}
    \begin{formula}
        In the neighbourhood of a point of a surface with principal curvatures $1/R_1$ and $1/R_2$, the surface is locally given by the quadric
        \begin{gather}
            x_3 = \stylefrac{1}{2}\left(\stylefrac{x_1^2}{R_1} + \stylefrac{x_2^2}{R_2}\right)
        \end{gather}
        up to order $O(x^2)$.
    \end{formula}

    \begin{formula}[Euler's formula]\index{Euler!formula for normal curvature}
        Let $h_1, h_2$ be the eigenvectors of the Weingarten map. The normal curvature of a couple $(P, v)$ where $v = h_1\cos\theta + h_2\sin\theta \in T_P\Sigma$ is given by
        \begin{gather}
            \label{diff:euler_formula}
            \stylefrac{1}{\rho_n} = \stylefrac{\cos^2\theta}{R_1} + \stylefrac{\sin^2\theta}{R_2}.
        \end{gather}
    \end{formula}

    \newdef{Asymptotic curve}{\index{asymptotic!curve}
        A curve which is at every point tangent to a direction with zero normal curvature.
    }
    \newformula{Differential equation for asymptotic curves}{
        \begin{gather}
            L_{11}\left(\dot{q}^1(t)\right)^2 + 2L_{12}\dot{q}^1(t)\dot{q}^2(t) + L_{22}\left(\dot{q}^2(t)\right)^2 = 0
        \end{gather}
    }
    \begin{property}
        A curve on a surface is an asymptotic curve if and only if the tangent plane and the \textit{osculation plane} coincide at every point $P$ on the curve.
    \end{property}

\subsection{Christoffel symbols and geodesics}

    \newformula{Gauss' formulas}{\index{Gauss!formula for surfaces}\index{Christoffel symbols}
        The second derivatives of the surface $\sigma$ are given by
        \begin{gather}
            \label{diff:gauss_formulas}
            \sigma_{ij} = L_{ij}N + \Gamma^k_{\ ij}\sigma_k
        \end{gather}
        where the \textbf{Christoffel symbols} $\Gamma^k_{\ ij}$ are defined as
        \begin{gather}
            \label{diff:christoffel_symbol}
            \Gamma^k_{\ ij} := g^{kl}\sigma_l\cdot\sigma_{ij}.
        \end{gather}
    }
    \begin{result}
        From the expression of the Christoffel symbols we can derive an alternative expression only in terms of the metric $g_{ij}$:
        \begin{gather}
            \Gamma^k_{\ ij} = \stylefrac{1}{2}g^{kl}\left(\pderiv{g_{il}}{q^j} - \pderiv{g_{ij}}{q^l} + \pderiv{g_{jl}}{q^i}\right).
        \end{gather}
    \end{result}

    \newdef{Geodesic}{\index{geodesic}
        A curve with zero geodesic curvature\footnote{See definition \ref{diff:geodesic_curvature}.}.
    }
    \begin{property}
        A curve on a surface is a geodesic if and only if the tangent plane and the \textit{osculation plane} are orthogonal at every point of the surface.
    \end{property}
    \newformula{Differential equation for geodesic}{
        If the curve is parametrized by arc length, it is a geodesic if the functions $q^1(s)$ and $q^2(s)$ satisfy the following differential equation:
        \begin{gather}
            \label{diff:geodesic_equation}
            q''^k + \Gamma^k_{\ ij}q'^iq'^j = 0.
        \end{gather}
    }

\subsection{Theorema Egregium}

    \newformula{Codazzi-Mainardi equations}{\index{Codazzi-Mainardi}
        \begin{gather}
            \label{diff:codazzi_mainardi}
            \pderiv{L_{ij}}{q^k} - \pderiv{L_{ik}}{q^j} = \Gamma^l_{\ ik}L_{lj} - \Gamma^l_{\ ij}L_{lk}
        \end{gather}
    }

    \newdef{Riemann curvature tensor}{\index{Riemann!curvature tensor}\label{diff:riemann_curvature}
        \begin{gather}
            R^l_{\ ijk} := \pderiv{\Gamma^l_{\ ik}}{q^j} - \pderiv{\Gamma^l_{\ ij}}{q^k} + \Gamma^s_{\ ik}\Gamma^l_{\ sj} - \Gamma^s_{\ ij}\Gamma^l_{\ ks}
        \end{gather}
    }

    \newformula{Gauss' equations}{\index{Gauss!equation for Riemann curvature}
        \begin{gather}
            \label{diff:gauss_equations}
            R^l_{\ ijk} = L_{ik}L^l_j - L_{ij}L^l_k
        \end{gather}
    }

    \begin{theorem}[Theorema Egregium]\index{Theorema Egregium}\label{diff:theorema_egregium}
        The Gaussian curvature\footnote{See formula \ref{diff:gaussian_curvature}.} $K$ is completely determined by the metric tensor $g_{ij}$ and its derivatives:
        \begin{gather}
            K = \stylefrac{R^l_{\ 121}g_{l2}}{g_{11}g_{22} - g_{12}^2}.
        \end{gather}
    \end{theorem}
    \sremark{This theorem is remarkable due to the fact that the coefficients $L_{ij}$, which appear in the general formula of the Gaussian curvature, cannot be expressed in terms of the metric.}

    \begin{property}
        From the condition of isometries \ref{diff:isometry} and the previous theorem it follows that if two surfaces are connected by an isometric map, the corresponding points have the same Gaussian curvature.
    \end{property}
    \begin{result}
        There exists no isometric projection from the sphere to the plane. This also implies that a perfect (i.e. isometric) map of the Earth cannot be created.
    \end{result}
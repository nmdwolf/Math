\chapter{Curves and Surfaces}\label{chapter:curves_surfaces}
\section{Curves}

	\newdef{Regular curve}{\index{regular!curve}
		Let $\vector{c}(t):I\rightarrow\mathbb{R}^n$ be a curve defined on an interval $I$. $\vector{c}(t)$ is said to be regular\footnote{See also property \ref{manifolds:regular_point}.} if $\deriv{\vector{c}}{t}(t) \neq \vector{0}$ for all $t\in I$.
	}

	\newdef{Geometric property}{A geometric property is a property that is invariant under:
    		\begin{enumerate}
			\item parameter transformations
        		\item orientation-preserving (orthonormal) basis transformations
		\end{enumerate}
	}

	\begin{property}
		Let $\vector{c}(t), \vector{d}(t)$ be two curves with the same image. The following relation holds for all $t$:
	        \begin{gather}
			\vector{c}(t)\text{ regular}\iff\vector{d}(t)\text{ regular}.
		\end{gather}
	\end{property}

\subsection{Arc length}

    	\newdef{Natural parameter}{\index{natural!parameter}\label{diff:natural_parameter}
        	Let $\vector{c}(t)$ be a curve. The parameter $t$ is said to be a natural parameter if
	        \begin{gather}
	            	\left|\left|\deriv{\vector{c}}{t}\right|\right|\equiv1.
		\end{gather}
	}

        \begin{formula}[Arc length]\index{arc!length}\label{diff:arc_length_integral}
		The following function $\phi(t)$ is a bijective map and a natural parameter for the curve $\vector{c}$:
	        \begin{gather}
        	        \phi(t) = \int_{t_0}^t||\dot{\vector{c}}(t)||dt.
		\end{gather}
	\end{formula}
        \sremark{The arc length as defined above is often denoted by the letter $s$.}

        \begin{property}
		Let $\vector{c}(t)$ be a curve and let $u$ be an alternative parameter of $\vector{c}(t)$. It is a natural parameter if and only if there exists a constant $\alpha$ such that:\[u = \pm s + \alpha\] where $s$ is the integral as defined in equation \ref{diff:arc_length_integral}.
	\end{property}
        \begin{remark*}
		This property implies there does not exist a unique natural parameter or arc length for any curve.
	\end{remark*}

\subsection{Frenet-Serret frame}

    	\newdef{Tangent vector}{\index{tangent!vector}\label{diff:tangent_vector}
    		Let $\vector{c}(s)$ be a curve parametrized by arc length. The tangent vector (field) $\vector{t}(s)$ is defined as follows:
        	\begin{gather}
			\vector{t}(s) = \vector{c}\,'(s).
		\end{gather}
        }
        \begin{property}\label{diff:unit_tangent_vector}
		From the definition of the natural parametrization \ref{diff:natural_parameter} and the previous definition it follows that the tangent vector is automatically a unit vector.
	\end{property}

        \newdef{Principal normal vector}{\index{normal!vector}\label{diff:principal_normal_vector}
	        Let $\vector{c}(s)$ be a curve parametrized by arc length. The principal normal vector (field) is defined as follows:
        	\begin{gather}
			\vector{n}(s) = \stylefrac{\vector{t}\,'(s)}{||\vector{t}\,'(s)||}.
		\end{gather}
        }
        \begin{property}
		From property \ref{diff:unit_tangent_vector} and the definition of the principal normal vector it follows that the tangent vector and principal normal vector are always orthogonal.
	\end{property}
        
        \newdef{Binormal vector}{\label{diff:binormal_vector}
	        Let $\vector{c}(s)$ be a curve parametrized by arc length. The binormal vector (field) is defined as follows:
        	\begin{gather}
			\vector{b}(s) = \vector{t}(s)\times\vector{n}(s).
		\end{gather}
        }

        \newdef{Frenet-Serret frame}{\index{Frenet-Serret}\label{diff:frenet_serret_frame}
	        As the vectors $\vector{t}(s), \vector{n}(s)$ and $\vector{b}(s)$ are mutually orthonormal and linearly independent, we can use them to construct an oriented orthonormal basis. The ordered basis $\Big\{\vector{t}(s),\vector{n}(s), \vector{b}(s)\Big\}$ is called the Frenet-Serret frame.
	}
        \sremark{This basis does not have to be the same for every value of the parameter $s$.}

        \newdef{Curvature}{\index{curvature}\label{diff:curvature}
	        Let $\vector{c}(s)$ be a curve parametrized by arc length. The curvature of $\vector{c}(s)$ is defined as follows:
        	\begin{gather}
			\stylefrac{1}{\rho(s)} = ||\vector{t}\,'(s)||.
		\end{gather}
        }
        \newdef{Torsion}{\index{torsion}\label{diff:torsion}
        	Let $\vector{c}(s)$ be a curve parametrized by arc length. The torsion of $\vector{c}(s)$ is defined as follows:
        	\begin{gather}
			\tau(s) = \rho(s)^2(\vector{t}\ \ \vector{t}\,'\ \ \vector{t}\,'')
		\end{gather}
		where $(a b c)$ denotes the \textit{triple product} $a\cdot(b\times c)$.
        }

        \newformula{Frenet formulas}{\index{Frenet formulas}\label{diff:frenet_formulas}
        	The derivatives of the tangent, principal normal and binormal vector fields can be written as a linear combination of the those vectors themselves:
        	\begin{gather}
			\left\{
                	\begin{array}{ccccc}
				\vector{t}\,'(s) &=&& \stylefrac{1}{\rho(s)}\vector{n}(s)&\\
        	                \vector{n}\,'(s) &=& -\stylefrac{1}{\rho(s)}\vector{t}(s) &+& \tau(s)\vector{b}(s)\\
        	                \vector{b}\,'(s) &=& &-\tau(s)\vector{n}(s)&
			\end{array}
        	        \right.
		\end{gather}
        }

        \begin{theorem}[Fundamental theorem of curves]\index{Fundamental theorem!of curves}\label{diff:theorem:fundamental_theorem_of_curves}
	        Let $k(s), w(s): U \rightarrow \mathbb{R}$ be two $\mathcal{C}^1$-functions with $k(s) \geq 0, \forall s\in U$. There exists an interval $]-\varepsilon, \varepsilon[\ \subset U$ and a curve $\vector{c}(s) :\ ]-\varepsilon, \varepsilon[\ \rightarrow \mathbb{R}^3$ with natural parameter $s$ such that $\vector{c}(s)$ has $k(s)$ as its curvature and $w(s)$ as its torsion.
	\end{theorem}

\section{Surfaces}

	\begin{notation}
	    	Let $\vector{\sigma}$ be the parametrization of a surface\footnote{The symbol $\vector{\sigma}$ denotes the surface as a vector field while $\Sigma$ denotes the geometric image of $\vector{\sigma}$.}. The derivative of $\vector{\sigma}$ with respect to the coordinate $q^i$ is written as follows:
	    	\begin{gather}
	    		\label{diff:derivative_of_surface}
			\pderiv{\vector{\sigma}}{q^i} = \vector{\sigma}_i.
		\end{gather}
	\end{notation}

    	\newdef{Tangent plane}{\index{tangent!plane}
        	Let $P(q_0^1,q_0^2)$ be a point on the surface $\Sigma$. The tangent space $T_P\Sigma$ to $\Sigma$ in $P$ is defined as follows:
	        \begin{gather}
			\label{diff:tangent_space}
        	        \forall \vector{r}\in T_P\Sigma : \left[\vector{r} - \vector{\sigma}(q_0^1,q_0^2)\right]\cdot\left[\vector{\sigma}_1(q_0^1,q_0^2)\times\vector{\sigma}_2(q_0^1,q_0^2)\right] = 0.
		\end{gather}
        }

        \newdef{Normal vector}{\index{normal!vector}
        	The cross product in equation \ref{diff:tangent_space} is closely related to the normal vector to $\Sigma$ in $P$. The normal vector in the point $(q_0^1, q_0^2)$ is defined as:
		\begin{gather}
			\label{diff:surface_normal_vector}
                	\vector{N} = \stylefrac{1}{||\vector{\sigma}_1\times\vector{\sigma}_2||}\left(\vector{\sigma}_1\times\vector{\sigma}_2\right)
		\end{gather}
		This way we see that the tangent plane $T_P\Sigma$ is exactly the set of vectors that are orthogonal to the normal vector $\vector{N}(q^1_0, q^2_0)$.
        }

\subsection{First fundamental form}

    	\newdef{Metric coefficients}{\index{metric}
        	Let $\vector{\sigma}$ be the parametrization of a surface. The metric coefficients $g_{ij}$ are defined as follows:
	        \begin{gather}
	        	\label{diff:metric_coefficient}
	                g_{ij} = \vector{\sigma}_i\cdot\vector{\sigma}_j.
		\end{gather}
        }
        \newdef{Scale factor}{\index{scale!factor}\label{diff:scale_factor}
        	The following factors are often used in vector calculus:
        	\begin{gather}
			g_{ii} = h_i^2.
		\end{gather}
        }

    	\newdef{First fundamental form}{\index{fundamental!form}\label{diff:first_fundamental_form}
        	Let $\vector{\sigma}$ be the parametrization of a surface. We can define a bilinear form $I_P:T_P\Sigma\times T_P\Sigma\rightarrow\mathbb{R}$ that restricts\footnote{Since $\Sigma$ is embedded in a higher-dimensional Euclidean space we can restrict the standard inner product on $\mathbb{R}^n$.} the inner product to $T_P\Sigma$:
	        \begin{gather}
	                I_P(\vector{v},\vector{w}) = \vector{v}\cdot\vector{w}.
		\end{gather}
	        This bilinear form is called the first fundamental form or \textbf{metric}.
        }
        \begin{result}
		All vectors $\vector{v},\vector{w}\in T_P\Sigma$ are linear combinations of the tangent vectors $\vector{\sigma}_1,\vector{\sigma}_2$. This enables us to relate the first fundamental form and the metric coefficients \ref{diff:metric_coefficient}:
        	\begin{gather}
        		I_P(\vector{v}, \vector{w}) = v^i\vector{\sigma}_i\cdot w^j\vector{\sigma}_j = g_{ij}v^iw^j.
        	\end{gather}
	\end{result}

        \begin{notation}
	        The (arc) length \ref{diff:arc_length_integral} of a curve $\vector{c}(t)$ can be written as follows:
        	\begin{gather}
	            	s = \int||\dot{\vector{c}}(t)||dt = \int\sqrt{ds^2}
		\end{gather}
	        where the second equality is formally defined. The two equalities together can be combined into the following notation for the metric which is often used in physics:
		\begin{gather}
			ds^2 = g_{ij}dq^idq^j.
		\end{gather}
	\end{notation}

        \begin{formula}[Inverse metric]\label{diff:inverse_metric_matrix}
		Let $(g_{ij})$ denote the metric tensor. We define the matrix $(g^{ij})$ as its inverse:
		\begin{gather}
			(g^{ij}) = \stylefrac{1}{\det(g_{ij})} \begin{pmatrix} g_{22}&-g_{12}\\ -g_{12}&g_{11} \end{pmatrix}.
		\end{gather}
	\end{formula}

\subsection{Isometries}

    	\newdef{Isometry}{\index{isometry}\label{diff:isometry_def}
        	An isometry is a distance-preserving map, i.e. a diffeomorphism $\Phi:\Sigma\rightarrow\Sigma'$ that maps arc segments in $\Sigma$ to arc segments with the same length in $\Sigma'$.
        }
        \begin{property}\label{diff:isometry}
        	A diffeomorphism $\Phi$ is an isometry if and only if the metric coefficients of $\sigma$ and $\sigma'$ are the same.
        \end{property}

        \newdef{Conformal map}{\index{conformal}\label{diff:conformal_map}
        	A diffeomorphism $\Phi:\Sigma\rightarrow\Sigma'$ is said to be conformal or \textbf{isogonal} if it maps two intersecting curves in $\Sigma$ to intersecting curves in $\Sigma'$ with the same intersection angle.
        }
        \begin{property}
        	A diffeomorphism $\Phi$ is conformal if and only if the metric coefficients of $\sigma$ and $\sigma'$ are proportional.
        \end{property}

        \newdef{Area-preserving map}{
        	A diffeomorphism $\Phi:\Sigma\rightarrow\Sigma'$ is said to be area-preserving if it maps a segment of $\Sigma$ to a segment of $\Sigma'$ with the same area.
        }
        \begin{property}
        	A diffeomorphism $\Phi$ is area-preserving if and only if the metric coefficients of $\sigma$ and $\sigma'$ satisfy:
	        \begin{gather}
        	    	g_{11}'g_{22}' - (g_{12}')^2 = g_{11}g_{22} - g_{12}^2
        	\end{gather}
	        for all points $(q^1, q^2)$, i.e. it preserves the determinant of the metric.
        \end{property}
        \result{A map that is area-preserving and conformal is also isometric.}

\subsection{Second fundamental form}

    	\newdef{Second fundamental form}{\index{fundamental!form}
        	Let $\vector{\sigma}$ be the parametrization of a surface. The second fundamental form is a bilinear form $II_P:T_P\Sigma\times T_P\Sigma\rightarrow\mathbb{R}$ defined as follows:
        	\begin{gather}
			\label{diff:second_fundamental_form}
                	II_P(\vector{v},\vector{w}) = L_{ij}(q^1, q^2)v^iw^j
		\end{gather}
	        where $L_{ij} = \vector{N}\cdot\vector\sigma_{ij}$.
        }

        \newdef{Normal curvature}{\index{normal!curvature}\index{curvature!normal}
        	Let $\vector{c}$ be a curve embedded as \[\vector{c}(s) = \vector{\sigma}\left(q^1(s), q^2(s)\right).\] The normal curvature of $\vector{c}$ at a point $\left(q^1(s), q^2(s)\right)$ is defined as
		\begin{gather}
            		\label{diff:normal_curvature}
	                \stylefrac{1}{\rho_n(s)} = \vector{c}\ ''(s)\cdot\vector{N}(s).
        	\end{gather}
	        From the definition of the second fundamental form it follows that the normal curvature can be written as follows:
        	\begin{gather}
        	    	\stylefrac{1}{\rho_n(s)} = II(\vector{t}, \vector{t}) = \stylefrac{II\left(\dot{\vector{c}}(t), \dot{\vector{c}}(t)\right)}{I\left(\dot{\vector{c}}(t), \dot{\vector{c}}(t)\right)}.
        	\end{gather}
	        where the last equality holds for any given parameter $t$.
        }

        \begin{theorem}[Meusnier]\index{Meusnier}\index{osculating circle}
        	Let $\vector{c}, \vector{d}$ be two curves defined on a surface $\vector{\sigma}$. The curves have the same normal curvature in a point $\left(q^1(t_0), q^2(t_0)\right)$ if the following two conditions are satisfied:
        	\begin{itemize}
        		\item $\vector{c}(t_0) =  \vector{d}(t_0)$
        		\item $\dot{\vector{c}}(t_0)\ ||\ \dot{\vector{d}}(t_0)$
        	\end{itemize}
        	Furthermore, the osculating circles of all curves with the same normal curvature at a given point form a sphere.
        \end{theorem}

        \begin{property}\index{normal!section}
        	The normal curvature of A \textit{normal section}\footnote{The intersection of the surface with a normal plane at the point} at a given point is equal to the curvature of the section at that point.
        \end{property}

        \newdef{Geodesic curvature}{\index{curvature!geodesic}
        	Let $\vector{c}$ be a curve embedded as \[\vector{c}(s) = \vector{\sigma}\left(q^1(s), q^2(s)\right).\] The geodesic curvature of $\vector{c}$ at the point $\left(q^1(s), q^2(s)\right)$ is defined as follows:
        	\begin{gather}
            		\label{diff:geodesic_curvature}
                	\stylefrac{1}{\rho_g(s)} = \left(\vector{N}(s)\ \vector{t}(s)\ \vector{t}\ '(s)\right).
		\end{gather}
        }

        \begin{formula}
        	Let $\vector{c}$ be a curve defined on a surface $\vector{\sigma}$. From the definitions of the normal and geodesic curvature it follows that
        	\begin{gather}
            		\stylefrac{1}{\rho^2} = \stylefrac{1}{\rho^2_n} + \stylefrac{1}{\rho^2_g}.
	        \end{gather}
        \end{formula}

\subsection{Curvature of a surface}

	\newdef{Weingarten map\footnotemark}{\index{Weingarten}\index{shape!operator}
		\footnotetext{Sometimes called the \textbf{shape operator}.}
        	Let $P$ be a point on the surface $\Sigma$. The Weingarten map $L_P:T_P\Sigma\rightarrow T_P\Sigma$ is the linear map defined as follows:
	        \begin{gather}
	            	\label{diff:weingarten_map}
	            	L_P(\vector{\sigma}_1) = -\vector{N}_1\qquad\text{and}\qquad L_P(\vector{\sigma}_2) = -\vector{N}_2.
	        \end{gather}
        }
        \begin{formula}
        	Let $\vector{v}, \vector{w}\in T_P\Sigma$. The following equalities relate the second fundamental form and the Weingarten map:
        	\begin{gather}
            		L_P(\vector{v})\cdot\vector{w} = L_P(\vector{w})\cdot\vector{v} = II_P(\vector{v}, \vector{w})
	        \end{gather}
        \end{formula}

        \begin{formula}
        	Let $\left(g^{ij}\right)$ be the inverse of the metric. The matrix elements of $L_P$ can be expressed as follows:
		\begin{gather}
            		L^k_j = g^{ki}L_{ij}
            	\end{gather}
        \end{formula}
        \begin{formula}[Weingarten formulas]\index{Weingarten!formulas}
        	\begin{gather}
        		\vector{N}_j = -L^k_j\vector{\sigma}_k
        	\end{gather}
        \end{formula}

        \newdef{Principal curvatures}{\index{curvature!principal}\label{diff:theorem:principal_directions}
        	The eigenvalues of the Weingarten map are called the principal curvatures of the surface and they are denoted by \[\stylefrac{1}{R_1}\quad\text{and}\quad\stylefrac{1}{R_2}.\] TBy the formulas above they are given by $II_P(\vector{h}_i, \vector{h}_i)$ and they are the extreme values of the normal curvature. The associated tangent vectors are called the \textbf{principal directions}. Furthermore, they form a basis for the tangent plane.
        }
        \begin{property}\index{umbilical point}
        	If the principal curvatures are not equal, the principal directions are orthogonal. If they are equal, the point $P$ is said to be an \textbf{umbilical point} or \textbf{umbilic}.
        \end{property}

        \newdef{Line of curvature}{\index{curvature!line}
        	A curve is a line of curvature if the tangent vector in every point $P$ is a principal direction of the surface in $P$.
        }

        \newformula{Rodrigues' formula}{\index{Rodrigues' formula}
        	A curve is a line of curvature if and only if it satisfies the following formula:
		\begin{gather}
        		\label{diff:rodrigues_formula}
	                \deriv{\vector{N}}{t}(t) = -\stylefrac{1}{R(t)}\deriv{\vector{c}}{t}(t).
        	\end{gather}
        	If the curve satisfies this formula, then the scalar function $1/R(t)$ coincides with the principal curvature along the curve.
        }
        \newformula{Differential equation for curvature lines}{
        	\begin{gather}
        		\left|
        	        \begin{array}{ccc}
        	        	(\dot{q}^2)^2&-\dot{q}^1\dot{q}^2&(\dot{q}^1)^2\\
        		        g_{11}&g_{12}&g_{22}\\
		                L_{11}&L_{12}&L_{22}
	                \end{array}
                	\right| = 0
        	\end{gather}
        }
        \begin{property}
        	From definition \ref{diff:theorem:principal_directions} we know that the principal directions are determined by orthogonal vectors. It follows that on a surface containing no umbilics the curvature lines form an orthogonal web.
        \end{property}

        \newdef{Gaussian curvature}{\index{curvature!Gaussian}\label{diff:gaussian_curvature}
        	The Gaussian curvature $K$ of a surface is defined as the determinant of the Weingarten map:
           	\begin{gather}
        	        K = \stylefrac{1}{R_1R_2}.
        	\end{gather}
        }
        \newdef{Mean curvature}{\index{curvature!mean}\label{diff:mean_curvature}
        	The mean curvature $H$ of a surface is defined as the trace of the Weingarten map:
	        \begin{gather}
        	        H = \stylefrac{1}{2}\left(\stylefrac{1}{R_1} + \stylefrac{1}{R_2}\right).
        	\end{gather}
        }
        \begin{property}
        	The principal curvatures are the solutions of the following equation:
        	\begin{gather}
        		x^2 - 2Hx + K = 0.
        	\end{gather}
        	This is the characteristic equation \ref{linalgebra:characteristic_equation} of the Weingarten map.
        \end{property}

        \begin{definition}
        	Let $P$ be a point on the surface $\Sigma$.
	        \begin{itemize}
			\item $P$ is said to be \textbf{elliptic} if $K > 0$ in $P$.
	                \item $P$ is said to be \textbf{hyperbolic} if $K < 0$ in $P$.
        	        \item $P$ is said to be \textbf{parabolic} if $K = 0$ and $\stylefrac{1}{R_1}$ or $\stylefrac{1}{R_2}\neq0$ in $P$.
        	        \item $P$ is said to be \textbf{flat} if $\stylefrac{1}{R_1} = \stylefrac{1}{R_2} = 0$ in $P$.
        	        \item $P$ is said to be \textbf{umbilical} if $\stylefrac{1}{R_1} = \stylefrac{1}{R_2}$ in $P$.
        	\end{itemize}
        \end{definition}
        \sremark{From previous definition it follows that a flat point is a special type of umbilic.}

        \begin{property}
        	A surface $\Sigma$ containing only umbilics is either part of a sphere or part of a plane.
        \end{property}
        \begin{formula}
        	In the neighbourhood of a point $P$ of a surface with principal curvatures $1/R_1$ and $1/R_2$, the surface is locally given by the following quadric
	        \begin{gather}
            		x_3 = \stylefrac{1}{2}\left(\stylefrac{x_1^2}{R_1} + \stylefrac{x_2^2}{R_2}\right)
        	\end{gather}
	        up to order $O(x^2)$.
        \end{formula}

        \begin{formula}[Euler's formula]\index{Euler!formula for normal curvature}
        	The normal curvature of a couple $(P, \vector{e})$ where is $\vector{e} = \vector{h}_1\cos\theta + \vector{h}_2\sin\theta \in T_P\Sigma$ is given by
            	\begin{gather}
            		\label{diff:euler_formula}
	                \stylefrac{1}{\rho_n} = \stylefrac{\cos^2\theta}{R_1} + \stylefrac{\sin^2\theta}{R_2}.
        	\end{gather}
        \end{formula}

        \newdef{Asymptotic curve}{\index{asymptotic!curve}
        	An asymptotic curve is a curve which is in every point $P$ tangent to a direction with zero normal curvature.
        }
        \newformula{Differential equation for asymptotic curves}{
        	\begin{gather}
        		L_{11}\left(\dot{q}^1(t)\right)^2 + 2L_{12}\dot{q}^1(t)\dot{q}^2(t) + L_{22}\left(\dot{q}^2(t)\right)^2 = 0
        	\end{gather}
        }
        \begin{property}
        	A curve on a surface is an asymptotic curve if and only if the tangent plane and the \textit{osculation plane} coincide in every point $P$ on the curve.
        \end{property}

\subsection{Christoffel symbols and geodesics}

	\newformula{Gauss' formulas}{\index{Gauss!formula for surfaces}\index{Christoffel symbols}
		The second derivatives of the surface $\vector{\sigma}$ are given by
	    	\begin{gather}
    			\label{diff:gauss_formulas}
        		\vector{\sigma}_{ij} = L_{ij}\vector{N} + \Gamma^k_{\ ij}\vector{\sigma}_k
	    	\end{gather}
        	where the \textbf{Christoffel symbols} $\Gamma^k_{\ ij}$ are defined as
        	\begin{gather}
        		\label{diff:christoffel_symbol}
        		\Gamma^k_{\ ij} = g^{kl}\vector{\sigma}_l\cdot\vector{\sigma}_{ij}.
	        \end{gather}
	}
	\begin{result}
    		From the expression of the Christoffel symbols we can derive an alternative expression using only the metric tensor $g_{ij}$:
        	\begin{gather}
        		\Gamma^k_{\ ij} = \stylefrac{1}{2}g^{kl}\left(\pderiv{g_{il}}{q^j} - \pderiv{g_{ij}}{q^l} + \pderiv{g_{jl}}{q^i}\right).
        	\end{gather}
	\end{result}

	\newdef{Geodesic}{\index{geodesic}
    		A geodesic is a curve with zero geodesic curvature\footnote{See definition \ref{diff:geodesic_curvature}.}.
	}
	\begin{property}
    		A curve on a surface is a geodesic if and only if the tangent plane and the \textit{osculation plane} are orthogonal in every point $P$ of the surface.
	\end{property}
	\newformula{Differential equation for geodesic}{
    		If the curve is parametrized by arc length, then it is a geodesic if the functions $q^1(s)$ and $q^2(s)$ satisfy the following differential equation:
    		\begin{gather}
    			\label{diff:geodesic_equation}
        		q''^k + \Gamma^k_{\ ij}q'^iq'^j = 0.
	    	\end{gather}
	}

\subsection{Theorema Egregium}

	See also section \ref{diff:section:curvature} for a generalization to Riemannian manifolds.

	\newformula{Codazzi-Mainardi equations}{\index{Codazzi-Mainardi}
    		\begin{gather}
    			\label{diff:codazzi_mainardi}
		        \pderiv{L_{ij}}{q^k} - \pderiv{L_{ik}}{q^j} = \Gamma^l_{\ ik}L_{lj} - \Gamma^l_{\ ij}L_{lk}
	    	\end{gather}
	}

	\newdef{Riemann curvature tensor}{\index{Riemann!curvature tensor}\label{diff:riemann_curvature}
    		\begin{gather}
        		R^l_{\ ijk} = \pderiv{\Gamma^l_{\ ik}}{q^j} - \pderiv{\Gamma^l_{\ ij}}{q^k} + \Gamma^s_{\ ik}\Gamma^l_{\ sj} - \Gamma^s_{\ ij}\Gamma^l_{\ ks}
	    	\end{gather}
	}

	\newformula{Gauss' equations}{\index{Gauss!equation for Riemann curvature}
    		\begin{gather}
    			\label{diff:gauss_equations}
        		R^l_{\ ijk} = L_{ik}L^l_j - L_{ij}L^l_k
	    	\end{gather}
	}

	\begin{theorem}[Theorema Egregium]\index{Theorema Egregium}\label{diff:theorema_egregium}
    		The Gaussian curvature\footnote{See formula \ref{diff:gaussian_curvature}.} $K$ is completely determined by the metric tensor $g_{ij}$ and its derivatives:
        	\begin{gather}
	        	K = \stylefrac{R^l_{\ 121}g_{l2}}{g_{11}g_{22} - g_{12}^2}.
        	\end{gather}
	\end{theorem}
	\sremark{This theorem is remarkable due to the fact that the coefficients $L_{ij}$, which appear in the general formula of the Gaussian curvature, cannot be expressed in terms of the metric tensor.}

	\begin{property}
    		From the condition of isometries \ref{diff:isometry} and the previous theorem it follows that if two surfaces are connected by an isometric map, the corresponding points in $\Sigma$ and $\Sigma'$ have the same Gaussian curvature.
	\end{property}
	\begin{result}
		There exists no isometric projection from the sphere to the plane. This also implies that a perfect (i.e. isometric) map of the Earth cannot be created.
	\end{result}
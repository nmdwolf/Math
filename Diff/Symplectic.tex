\chapter{Symplectic Geometry}\label{chapter:symplectic}
\section{Symplectic manifolds}

	\newdef{Symplectic form}{\index{symplectic!form}
		Let $\omega\in\Omega^2(M)$ be a differential 2-form. $\omega$ is said to be symplectic if it satisfies following properties:
		\begin{itemize}
			\item Closedness: $d\omega = 0$
			\item Non-degeneracy: if $\omega(u, v) = 0, \forall u\in TM$ then $v=0$
		\end{itemize}
	}
	\newdef{Symplectic manifold}{\index{symplectic!manifold}
		A manifold $M$ equipped with a symplectic 2-form $\omega$ is called a symplectic manifold. This structure is often denoted as the pair $(M, \omega)$.
	}
	
	\begin{property}[Dimension]
		From the antisymmetry and the non-degeneracy of the symplectic form it follows that $M$ is even-dimensional.
	\end{property}
	
	\begin{theorem}[Darboux]\index{Darboux!chart}
		Let $(M, \omega)$ be a symplectic manifold. For every neighbourhood $\Omega$ in $T^*M$ there exists an adapted coordinate system $(q^i, p^i)$ such that
		\begin{equation}
			\left.\omega\right|_\Omega = \sum_idp^i\wedge dq^i
		\end{equation}		
		{\normalfont The charts in Darboux's theorem are called \textbf{Darboux charts} and they cover $M$.}
	\end{theorem}
	\begin{remark}
		This theorem shows that all symplectic manifolds of the same dimension are locally isomorphic, i.e. there exist no local invariants, in contrast to for example Riemannian manifolds.
	\end{remark}
	
	\begin{formula}
		In Darboux coordinates the components of the symplectic form $\omega$ read:
		\begin{equation}
			\omega_{ij} = \left(
			\begin{array}{c|c}
				0&\ \mathbbm{1}\ \ \\
				\hline
				-\mathbbm{1}&\ 0\ \ 
			\end{array}
			\right)
		\end{equation}
		By the non-degeneracy we can define the 'dual' $\omega^\sharp$ as:
		\begin{equation}
			(\omega^\sharp)^{ij} = \left(
			\begin{array}{c|c}
				\ 0\ \ &-\mathbbm{1}\\
				\hline
				\ \mathbbm{1}\ \ &0
			\end{array}
			\right)
		\end{equation}
	\end{formula}
	
\subsection{Symplectomorphisms}

	\newdef{Symplectomorphism}{\index{symplectomorphism}
		A symplectomorphism is a morphism of symplectic manifolds, i.e. a diffeomorphism $f:(M, \omega_M)\rightarrow (N, \omega_N)$ satisfying:
		\begin{equation}
			f^*\omega_N = \omega_M
		\end{equation}
		These maps form a \textit{semigroup} called the \textbf{symplectomorphism group}\footnote{Not to be confused with the symplectic group $\text{Sp}(n)$.}
	}
	\newdef{Symplectic vector field}{
		A vector field is said to be symplectic if its flow preserves the symplectic form $\omega$:
		\begin{equation}
			\mathcal{L}_X\omega = 0
		\end{equation}
		Equivalently a vector field is symplectic if its flow is a symplectomorphism. These vector fields form a Lie subalgebra of $\mathfrak{X}(M)$.
	}
	
	\newdef{Hamiltonian vector field}{\index{Hamilton!vector field}
		Let $(M, \omega)$ be a symplectic manifold. For every function $f\in C^\infty(M)$ we define the associated Hamiltonian vector field $X_f$ by the following relation\footnote{A lot of different conventions exist in the literature. We use the one compatible with the Hamiltonian equations \ref{lagrange:hamilton_equations} which are universally accepted.}:
		\begin{equation}
			\omega(X_f, \cdot) = -df(\cdot)
		\end{equation}
		or by using $\omega^\sharp$:
		\begin{equation}
			X_f(\cdot) = \omega^\sharp(-df, \cdot)
		\end{equation}
		These vector fields form a Lie subalgebra of the Lie algebra of symplectic vector fields. The flow associated to a Hamiltonian vector field is sometimes called a \textbf{Hamiltonian symplectomorphism}.\footnote{The fact that the Hamiltonian flow indeed preserves the symplectic form follows from the closedness of $\omega$.}
	}
	
	\newdef{Poisson bracket}{\index{Poisson!bracket}
		Let $(M, \omega)$ be a symplectic manifold. The Poisson bracket of two functions $f, g\in C^\infty(M)$ is defined as:
		\begin{equation}
			\{f, g\} = X_f(g)
		\end{equation}
		or equivalently:
		\begin{equation}
			X_{\{f, g\}} = [X_f, X_g]
		\end{equation}
	}
	\begin{property}
		The Poisson brackets induced by the symplectic form turns the structure $(C^\infty(M), \{\cdot,\cdot\})$ into a Lie algebra\footnote{The antisymmetry follows from equation \ref{diff:lie_derivative_antisymmetry} and the Jacobi-identity follows from the closedness of $\omega$.} and the second equation in fact gives a (surjective) Lie algebra morphism\footnote{The kernel is given by the constant functions.} $(C^\infty(M), \{\cdot,\cdot\})\rightarrow(\{\text{X : X is a HVF on M}\}, [\cdot, \cdot])$. Furthermore, together with the pointwise multiplication the structure becomes a Poisson algebra\footnote{See definition \ref{lie:poisson_algebra}.}.
	\end{property}
	
	\newdef{Poisson manifold}{
		A smooth manifold on which the algebra of smooth functions can be equipped with a Poisson algebra structure.
	}
	\begin{property}
		From the property above every symplectic manifold is a Poisson manifold. The converse however is not true.
	\end{property}
	
\section{Lagrangian submanifolds}

	\newdef{Symplectic complement}{
		Let $(M, \omega)$ be a symplectic manifold and let $S\subset M$ be an embedded submanifold $\iota: S\hookrightarrow M$. The symplectic orthogonal complement $T^\bot_pS$ at the point $p\in S$ is defined as\footnote{The notation $T^\omega_pS$ is also sometimes used.}:
		\begin{equation}
			T^\bot_pS = \{v\in T_pM: \omega(v, \iota_* w) = 0, \forall w\in T_pS\}
		\end{equation}
	}
	
	\newdef{Isotropic submanifold}{\index{isotropic}
		Let $(M, \omega)$ be a symplectic manifold. An embedded submanifold $\iota:S\hookrightarrow M$ is called isotropic if $T_pS\subset T^\bot_pS$.
	}
	\newdef{Isotropic submanifold}{
		Let $(M, \omega)$ be a symplectic manifold. An embedded submanifold $\iota:S\hookrightarrow M$ is called co-isotropic if $T^\bot_pS\subset T_pS$.
	}
	\newdef{Larangian submanifold}{\index{Lagrange!Lagrangian submanifold}
		Let $(M, \omega)$ be a symplectic manifold. An embedded submanifold $\iota:S\hookrightarrow M$ is called Lagrangian if $T_pS = T^\bot_pS$. Therefore they are sometimes called maximal isotropic submanifolds.
	}
	
\section{Cotangent bundle}

	\begin{construct}[Liouville one-form]\index{Liouville!1-form}\index{canonical!1-form|see{Liouville}}
		Let $M$ be a smooth manifold. The cotangent bundle $T^*M$ comes equipped with a canonical symplectic form:
		
		Let $(q, p)$ denote the local coordinates on $T^*M$. Define a 1-form $\alpha$ by \[\alpha = p_idq^i\] In coordinate-free notation this can be written as
		\begin{equation}
			\alpha(z) = \pi_2(z)\Big(\pi_*(z)\Big)
		\end{equation}
		where $z\in TT^*M$, $\pi: T^*M\rightarrow M$ and $\pi_2:TT^*M\rightarrow T^*M$. This 1-form, called the \textbf{canonical 1-form}\footnote{Also known as the \textbf{Liouville 1-form}.}, induces a symplectic form in the following way: \[\omega = d\alpha\]
	\end{construct}
	
\section{Hamiltonian dynamics}

	\newdef{Completely integrable system}{\index{CIS}
		Consider a multimap \[F \equiv (F_1, ..., F_n):M\rightarrow\mathbb{R}^n\] on a symplectic manifold $(M, \omega)$. This map defines a completely integrable system (CIS) if it satisfies the following conditions:
		\begin{itemize}
			\item $\dim M = 2n$
			\item The Hamiltonian vector fields $\{X_{F_i}\}_{i\leq n}$ are almost everywhere linearly independent.\footnote{Equivalently we require that the Jacobian $DF$ has full rank almost everywhere.}
			\item For every $i, j\leq n$ we have that $\{F_i, F_j\} = 0$.
		\end{itemize}
	}

\subsection{Hamiltonian actions}

	\newdef{Hamiltonian torus action}{\index{Hamilton!action}
		Let $(M, \omega)$ be a symplectic manifold. Let $T^n$ be a torus group acting on $M$. The action of $T^n$ is said to be Hamiltonian if there exists a CIS on $M$ such that the action arises as the Hamiltonian flow of $F$.
	}

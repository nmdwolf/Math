\chapter{Symplectic Geometry}\label{chapter:symplectic}

    References for this chapter are \cite{mcduff, symplectic}.

\section{Symplectic manifolds}

    \newdef{Symplectic form}{\index{symplectic!form}
        Le $M$ be a smooth manifold and consider a 2-form $\omega\in\Omega^2(M)$. $\omega$ is said to be symplectic\footnote{If one drops the closedness condition then one obtains a \textbf{almost symplectic manifold}.} if it satisfies following properties:
        \begin{itemize}
            \item \textbf{Closedness}: $d\omega = 0$
            \item \textbf{Nondegeneracy}: if $\omega(u, v) = 0, \forall u\in TM$ then $v=0$
        \end{itemize}
    }
    \newdef{Symplectic manifold}{\index{symplectic!manifold}
        A manifold $M$ equipped with a symplectic 2-form $\omega$ is called a symplectic manifold. This structure is often denoted by the pair $(M, \omega)$.
    }

    \begin{property}[Dimension]
        From the antisymmetry and the nondegeneracy of the symplectic form it follows that $M$ is even-dimensional.
    \end{property}

    \begin{theorem}[Darboux]\index{Darboux}
        Let $(M, \omega)$ be a symplectic manifold. For every neighbourhood $\Omega$ in $T^*M$ there exists an adapted coordinate system $(q^i, p^i)$ such that
        \begin{gather}
            \left.\omega\right|_\Omega = \sum_idp^i\wedge dq^i.
        \end{gather}
    \end{theorem}
    The adapted charts from Darboux' theorem are called \textbf{Darboux charts}. A symplectic manifold admits a covering by Darboux charts.
    \begin{remark}
        This theorem shows that all symplectic manifolds of the same dimension are locally isomorphic, i.e. there exist no local invariants. This is in stark contrast to for example Riemannian manifolds.
    \end{remark}

    \begin{formula}
        In Darboux coordinates the components of the symplectic form $\omega$ are given by
        \begin{gather}
            \omega_{ij} = \left(
            \begin{array}{c|c}
                \ 0\ \ &-\mathbbm{1}\\
                \hline
                \ \mathbbm{1}\ \ &0
            \end{array}
            \right).
        \end{gather}
        Using the nondegeneracy condition we can define the ''dual'' $\omega^\sharp$ as
        \begin{gather}
            (\omega^\sharp)^{ij} = \left(
            \begin{array}{c|c}
                0&\ \mathbbm{1}\ \ \\
                \hline
                -\mathbbm{1}&\ 0\ \
            \end{array}
            \right).
        \end{gather}
    \end{formula}

    \begin{property}\label{diff:symplectic_G_structure}
        In the language of $G$-structures\footnote{See section \ref{section:G-structure}.} one can restate the definition of symplectic manifolds. A smooth $2n$-dimensional manifold is \textbf{almost symplectic} exactly if it admits an Sp$(2n)$-structure. It is symplectic exactly if the Sp$(2n)$-structure is integrable, which by Darboux' theorem is equivalent to first order integrability, i.e. $d\omega = 0$.
    \end{property}

    \newdef{Symplectic potential}{\index{symplectic!potential}
        By the Poincaré lemma \ref{forms:theorem:poincare} the symplectic form $\omega$ (locally) defines a one-form $\theta$:
        \begin{gather}
            \omega = d\theta.
        \end{gather}
        This one-form is sometimes called the sympletic potential of $(M, \omega)$.
    }
    \begin{construct}[Liouville one-form\footnotemark]\index{Liouville!one-form}\index{canonical!one-form|see{Liouville}}
        \footnotetext{Also known as the \textbf{canonical one-form}.}
        Let $M$ be a smooth manifold. The cotangent bundle $T^*M$ comes equipped with a canonical symplectic form: Let $(q, p)$ denote the local coordinates on $T^*M$ and define a one-form $\alpha$ by \[\alpha := p_idq^i.\] In coordinate-free notation this can be written as follows:
        \begin{gather}
            \alpha(z) = \pi_2(z)\Big(\pi_*(z)\Big)
        \end{gather}
        where $z\in TT^*M$ and where $\pi: T^*M\rightarrow M$ and $\pi_2:TT^*M\rightarrow T^*M$ are the obvious bundle projections. This one-form serves as a symplectic potential for the cotangent bundle: \[\omega = d\alpha.\]
    \end{construct}

\subsection{Symplectomorphisms}

    \newdef{Symplectomorphism}{\index{symplectomorphism}
        A symplectomorphism is an isomorphism of symplectic manifolds, i.e. a diffeomorphism $f:(M, \omega_M)\rightarrow (N, \omega_N)$ satisfying
        \begin{gather}
            f^*\omega_N = \omega_M.
        \end{gather}
        These maps form a semigroup\footnote{See definition \ref{group:semigroup}.} called the \textbf{symplectomorphism group}. This should not be confused with the symplectic group $\text{Sp}(n)$.
    }
    \newdef{Symplectic vector field}{
        A vector field is said to be symplectic if its flow preserves the symplectic form $\omega$:
        \begin{gather}
            \mathcal{L}_X\omega = 0.
        \end{gather}
        Equivalently, a vector field is symplectic if its flow is a symplectomorphism. These vector fields form a Lie subalgebra of $\mathfrak{X}(M)$.
    }

    \newdef{Hamiltonian vector field}{\label{diff:hamilton_vectorfield}\index{Hamilton!vector field}
        Let $(M, \omega)$ be a symplectic manifold. For every function $f\in C^\infty(M)$ we define the associated Hamiltonian vector field $X_f$ by the following relation\footnote{A lot of different conventions exist in the literature. We use the one compatible with the Hamiltonian equations \ref{lagrange:hamilton_equations} which are universally accepted.}:
        \begin{gather}
            \omega(X_f, \cdot) = -df(\cdot).
        \end{gather}
        This can be rewritten by using $\omega^\sharp$ as
        \begin{gather}
            X_f(\cdot) = \omega^\sharp(-df, \cdot).
        \end{gather}
        These vector fields form a Lie subalgebra of the Lie algebra of symplectic vector fields. The flow associated to a Hamiltonian vector field is sometimes called a \textbf{Hamiltonian symplectomorphism}.\footnote{The fact that the Hamiltonian flow indeed preserves the symplectic form follows from the closedness of $\omega$.}
    }

    \newdef{Poisson bracket}{\index{Poisson!bracket}
        Let $(M, \omega)$ be a symplectic manifold. The Poisson bracket of two functions $f, g\in C^\infty(M)$ is defined as
        \begin{gather}
            \{g, f\} := X_f(g)
        \end{gather}
        or equivalently as
        \begin{gather}
            X_{\{g, f\}} := [X_f, X_g]
        \end{gather}
        where $X_f, X_g$ are the Hamiltonian vector fields as defined above.
    }
    \begin{property}
        The Poisson bracket induced by the symplectic form turns the structure $(C^\infty(M), \{\cdot,\cdot\})$ into a Lie algebra and the second equation in fact gives a (surjective) Lie algebra morphism\footnote{The kernel is given by the constant functions and hence it is not a bijection.} $(C^\infty(M), \{\cdot,\cdot\})\rightarrow(\{\text{X : X is a HVF on M}\}, [\cdot, \cdot])$. Furthermore, together with the pointwise multiplication the structure becomes a Poisson algebra\footnote{See definition \ref{lie:poisson_algebra}.}.
    \end{property}

    \newdef{Poisson manifold}{\index{Poisson!manifold}\label{diff:poisson_manifold}
        A smooth manifold on which the algebra of smooth functions can be equipped with a Poisson algebra structure.
    }
    \begin{property}
        From the property above every symplectic manifold is a Poisson manifold. The converse however is not true.
    \end{property}

\section{Lagrangian submanifolds}

    \newdef{Symplectic complement}{
        Let $(M, \omega)$ be a symplectic manifold and let $S\subset M$ be an embedded submanifold $\iota: S\hookrightarrow M$. The symplectic orthogonal complement $T^\bot_pS$ (sometimes also denoted by $T^\omega_pS$) at the point $p\in S$ is defined as the space
        \begin{gather}
            T^\bot_pS := \{v\in T_pM: \omega(v, \iota_* w) = 0, \forall w\in T_pS\}.
        \end{gather}
    }

    \newdef{Isotropic submanifold}{\index{isotropic}
        Let $(M, \omega)$ be a symplectic manifold. An embedded submanifold $\iota:S\hookrightarrow M$ is said to be isotropic if $T_pS\subset T^\bot_pS$.
    }
    \newdef{Isotropic submanifold}{
        Let $(M, \omega)$ be a symplectic manifold. An embedded submanifold $\iota:S\hookrightarrow M$ is said to be co-isotropic if $T^\bot_pS\subset T_pS$.
    }
    \newdef{Larangian submanifold}{\index{Lagrange!submanifold}
        Let $(M, \omega)$ be a symplectic manifold. An embedded submanifold $\iota:S\hookrightarrow M$ is said to be Lagrangian if $T_pS = T^\bot_pS$. Therefore they are sometimes called maximal isotropic submanifolds.
    }

    \begin{example}[Closed sections]\label{diff:closed_section_submanifold}
        Consider a closed section of a cotangent bundle $T^*Q$, i.e. a map $\sigma:Q\rightarrow T^*Q$ such that $d\sigma=0$. The graph of $\sigma$ is a Lagrangian submanifold.
    \end{example}

    \begin{theorem}[Maslov \& H\"ormander]\index{Maslov-H\"ormander}\index{Morse!family}\index{generating function}
        Let $Q$ be a smooth manifold and consider a smooth function $W:Q\times\mathbb{R}^k\rightarrow\mathbb{R}$ where $k\geq 0$. If 0 is a regular value of the map \[\pderiv{W}{u}:Q\times\mathbb{R}^k\rightarrow\mathbb{R}^k\] then the subset $\Lambda\subset T^*Q$, locally defined by the following equations
        \begin{gather}
            \pderiv{W}{u^\alpha} = 0\qquad\qquad p_i=\pderiv{W}{q^i},
        \end{gather}
        is a Lagrangian submanifold. Conversely, if $\Lambda\xhookrightarrow{\iota}T^*Q$ is a Lagrangian submanifold then at every $\lambda_0\in\Lambda$ there exists an integer \[k_0\geq\dim Q - \text{rk}\left(D(\pi_Q\circ\iota)|_{\lambda_0}\right)\] such that locally around $\lambda_0$ the submanifold $\Lambda$ is described by some function $W:Q\times\mathbb{R}^{k_0}$ satisfying the above equations.
    \end{theorem}
    Any function $W$ generating a Lagrangian submanifold through the above equations will be called a \textbf{generating function}.\footnote{Functions satisfying both the equations and the regularity condition are called \textbf{Morse families}.}

    \newdef{Real polarization}{\index{polarization}
        A (real) polarization of a symplectic manifold $(M, \omega)$ is a foliation by Lagrangian submanifolds, i.e. a subbundle\footnote{One often looks at a subbundle of the complexified tangent bundle $T_{\mathbb{C}}M$ (in this case the symplectic form is extended linearly to the complexified tangent spaces).} $P\subset TM$ such that the following conditions are satisfied:
        \begin{enumerate}
            \item \textbf{Maximality}: $\dim TM=2\dim P$
            \item \textbf{Isotropy}: $\iota_X\omega = 0$ for all $X\in P$
            \item \textbf{Involutivity}\footnote{This condition characterizes $P$ as an integrable subbundle by Frobenius' theorem.}: $[X, Y]=0$ for all $X, Y\in P$
        \end{enumerate}
    }
    More generally one can define a (complex) polarization:
    \newdef{Polarization}{
        An involutive Lagrangian subbundle $\mathcal{P}$ of the complexified tangent bundle $T_{\mathbb{C}}M$ with the additional property that $\dim(\mathcal{P}\cap\overline{\mathcal{P}}\cap T_{\mathbb{C}}M)$ is constant throughout the entire manifold.

        A real polarization is (after complexifying it) the same as a complex polarization for which $\mathcal{P}=\overline{\mathcal{P}}$.
    }

    \begin{example}[Vertical polarization]\index{vertical!polarization}
        Consider the cotangent bundle $T^*Q$ of a smooth manifold $Q$. We define the bundle $\mathcal{P}$ at every point $m\in T^*Q$ as the complexified tangent space \[T_mT^*_{\pi(m)}Q\otimes\mathbb{C}=\text{span}\left\{\pderiv{}{p_i}:p_i\text{ is a Darboux coordinate on }T^*Q\right\}\] where $\pi$ is the cotangent bundle projection. It can be shown that this polarization is real.
    \end{example}

\section{Hamiltonian dynamics}
\subsection{Dynamical systems}

    \newdef{Dynamical system}{\index{Hamilton!function}\index{dynamical!system}
        Let $(M, \omega)$ be a symplectic manifold and let $H\in C^\infty(M)$ be a distinguished observable. The triple $(M, \omega, H)$ is called a dynamical system with \textbf{Hamiltonian} $H$. The time derivative of any observable $F\in C^\infty(M)$ is defined by\footnote{Note that this construction can in fact be generalized to Poisson manifolds \ref{diff:poisson_manifold}.}
        \begin{gather}
            \dot{F} := \{H, F\}
        \end{gather}
        where $\{\cdot,\cdot\}$ is the Poisson bracket on $M$. The time evolution is then governed by the Hamiltonian flow $\exp(tX_H)$.
    }
    \newdef{Conserved quantity}{
        Let $(M, \omega, H)$ be a dynamical system. An observable $F\in C^\infty(M)$ is said to be conserved if it satisfies $\dot{F}=\{H, F\} = 0$.
    }

    \begin{result}[Noether's theorem]\index{Noether!theorem}
        \textit{Noether's theorem} (at least in its form from classical mechanics) is just an application of the antisymmetry of the Poisson bracket:
        \begin{gather}
            \{H, Q\} = 0\iff\{Q, H\} = 0.
        \end{gather}
    \end{result}

    From here on we consider a specific type of Hamiltonian function $H$, called a \textbf{mechanical Hamiltonian}. Let $(Q, g)$ be a Riemannian manifold and equip the cotangent bundle $M:=T^*Q\overset{\pi}{\rightarrow}Q$ with its canonical symplectic structure. The Hamiltonians we will consider are of the form (in local Darboux coordinates)
    \begin{gather}
        H(q, p) = \frac{1}{2}g^{ij}(q)p_ip_j + V(q)
    \end{gather}
    where $V(q)$ is a smooth function. These Hamiltonians have two types of symmetries (conserved quantities):
    \newdef{Kinematical}{\index{symmetry!kinematical}
        Consider a conserved quantity $C$. If $\pi_*(X_C)\in \Gamma(TQ)$ exists and $\mathcal{L}_{\pi_*(X_C)}g = 0$ then the symmetry is said to be kinematical.
    }
    \sremark{The second condition says that $\pi_*(X_C)$ is a Killing vector. (See definition \ref{diff:killing_vector}.)}
    \newdef{Dynamical}{\index{symmetry!dynamical}
        Any symmetry that is not a kinematical symmetry is said to be dynamical.
    }

    The following algorithm gives us a way to find conditions to check whether a given observable is conserved:
    \begin{method}[Van Holten's algorithm]\index{Van Holten}
        Let the conserved quantity be analytic, i.e. \[C(q, p) = \sum_{k=0}^N\frac{1}{k!}a^{(n_1\ldots n_k)}(q)p_{n_1}\ldots p_{n_k}\] for some $N\in\mathbb{N}$, where the brackets around indices denote symmetrization. For a flat manifold, i.e. where $g$ does not depend on $q$, we can rewrite $\{C, T+V\} = 0$ as \[\sum_{n = 1}^N\left[\frac{1}{(k-1)!}a^{n_1\ldots n_{k-1}i}p_{n_1}\ldots p_{n_{k-1}}\pderiv{V}{q^i} - \frac{2}{k!}\pderiv{}{q^i}a^{n_1\ldots n_k}p_{n_1}\ldots p_{n_k}g^{im}p_m\right] = 0\] Because two polynomials are equal if and only if their corresponding coefficients are equal, we obtain the following equations:
        \begin{enumerate}
            \item $0^{th}$ order: $\displaystyle a^k\pderiv{V}{q^k} = 0$
            \item $1^{st}$ order: $\displaystyle a^{(n_1i)}\pderiv{V}{q^i} - 2\pderiv{a}{q^i}g^{in_1} = 0$
            \item $N^{th}$ order: $\displaystyle \frac{1}{N!}a^{(n_1\ldots n_Ni)}\pderiv{V}{q^i} - \frac{2}{(N-1)!}\pderiv{}{q^i}a^{(n_1\ldots n_{N-1}}g^{i)n_N} = 0$
        \end{enumerate}
        where one should pay attention to the symmetrization brackets in the second term of the last equation. Pulling down the indices by multiplying with the metric $g_{n_im_i}$ gives
        \begin{gather}
            a_{(m_1\ldots m_N)}^{\phantom{(m_1\ldots m_N)}i}\partial_iV - 2N\partial_{(m_N}a_{m_1\ldots m_{N-1})} = 0.
        \end{gather}
        The upper bound $N$ in the series expansion is determined by the generalized Killing condition\footnote{See equation \ref{diff:killing_tensor}.}:
        \begin{gather}
            \partial_{(m_{N+1}}a_{m_1\ldots m_N)} = 0\implies a_{(m_1\ldots m_{N+1})} = 0.
        \end{gather}
    \end{method}
    \begin{remark}
        The above algorithm still holds for curved manifolds when replacing all partial derivatives $\partial_i$ by (Levi-Civita) covariant derivatives $\nabla_i$.
    \end{remark}

\subsection{Hamilton-Jacobi equation}\index{Hamilton-Jacobi}\label{section:hamilton_jacobi}

    \newdef{Hamilton-Jacobi equation}{
        Consider a smooth manifold $Q$ equipped with and let the cotangent bundle be equipped with a Hamiltonian function $H:T^*Q\rightarrow\mathbb{R}$. The Hamilton-Jacobi equation (associated to $H$) is the differential equation (for a function $S:Q\rightarrow\mathbb{R}$) of the following form:
        \begin{gather}
            H\circ dS = 0.
        \end{gather}
        This can be rewritten as
        \begin{gather}
            \label{diff:hamilton_jacobi}
            H\left(q, \pderiv{S}{q}\right) = 0.
        \end{gather}
    }

    Because of property \ref{diff:closed_section_submanifold} we immediately see that the solutions of the Hamilton-Jacobi equation define a Lagrangian submanifold of the cotangent bundle. Furthermore, they have the property that they are transversal to the fibres of the projection $\pi:T^*Q\rightarrow Q$. By relaxing this transversality condition one obtains the following more general notion:
    \newdef{Geometric solution}{
        A Lagrangian submanifold of the level set $H^{-1}(0)$ for some smooth function $H:T^*Q\rightarrow\mathbb{R}$.
    }
    \remark{It can be proven that geometric solutions can locally be described by a solution of the Hamilton-Jacobi equation.}

\subsection{Integrability}

    \newdef{Completely integrable system}{\index{CIS}
        \nomenclature[A_CIS]{CIS}{completely integrable system}
        Consider a (smooth) vector-valued function \[F\equiv(F_1,\ldots,F_n):M\rightarrow\mathbb{R}^n\] on a symplectic manifold $(M, \omega)$. This map defines a completely integrable system (CIS) if it satisfies the following conditions:
        \begin{enumerate}
            \item The dimension is maximal, i.e. $\dim M = 2n$.
            \item The Hamiltonian vector fields $\{X_{F_i}\}_{i\leq n}$ are almost everywhere linearly independent.\footnote{Equivalently, we can require that the Jacobian $DF$ has full rank almost everywhere.}
            \item For every $i, j\leq n$:  $\{F_i, F_j\} = 0$.
        \end{enumerate}
    }

\subsection{Hamiltonian actions}

    \newdef{Hamiltonian torus action}{\index{Hamilton!action}
        Let $(M, \omega)$ be a symplectic manifold and let $T^n$ be a torus group acting on $M$. The action of $T^n$ is said to be Hamiltonian if there exists a CIS on $M$ such that the action arises as the Hamiltonian flow of $F$.
    }

    ?? COMPLETE ??

\section{Symplectic reduction}

    \newdef{Reduced vector field}{
        Let $M$ be a smooth manifold and $G$ a Lie group which acts freely and properly on $M$ (this implies that the quotient space $M/G$ is a smooth manifold). Now suppose that $G$ acts as a symmetry group on some vector field $X\in\mathfrak{X}(M)$, i.e. $\Phi_g^*X=X$ for all $g\in G$. The reduced vector field $\overline{X}\in\mathfrak{X}(M/G)$ is defined through the following equation:
        \begin{gather}
            \overline{X}(\pi(m)) := \pi_*X(m)
        \end{gather}
        where $\pi:M\rightarrow M/G$ is the quotient map of the associated principal $G$-bundle.
    }

\subsection{Poisson reduction}

    \newdef{Poisson map}{
        Let $(M, \{\cdot, \cdot\})$ and $(N, [\cdot, \cdot])$ be two Poisson manifolds. A Poisson map $\Phi:M\rightarrow N$ is a map satisfying the following equality for all $f, g\in C^\infty(N)$:
        \begin{gather}
            \Phi^*[f, g] = \{\Phi^*f, \Phi^*g\}.
        \end{gather}
    }
    \newdef{Poisson action}{
        Let $G$ be a Lie group and let $(M, \{\cdot, \cdot\})$ be a Poisson manifold. A $G$-action on $M$ is called a Poisson action or \textbf{canonical action} if every $g\in G$ acts by a Poisson map.
    }

    \begin{theorem}[Poisson reduction]
        Let $G$ be a Lie group which acts freely and properly on a Poisson manifold $(M, \{\cdot, \cdot\})$. If the action is canonical then the Poisson bracket on $M$ descends (uniquely) to a Poisson bracket on the quotient manifold $M/G$. Furthermore, the projection $\pi:M\rightarrow M/G$ is a Poisson map with respect to this structure.
    \end{theorem}
    \begin{property}
        Let $H:M\rightarrow\mathbb{R}$ be a $G$-invariant Hamiltonian function. Its Hamiltonian vector field $X_H$ is also $G$-invariant and the Hamiltonian vector field of the reduced Hamiltonian $h:M/G\rightarrow\mathbb{R}$, defined by $H:=h\circ\pi$, is given by the reduced vector field of $X_H$.
    \end{property}

\subsection{Lie-Poisson reduction}

    In the case of Lie-Poisson reductions one considers the cotangent bundle $T^*G$ of a Lie group $G$ as the configuration manifold. It is not too hard to show that $T^*G/G\cong\mathfrak{g}^*$.

    \newformula{Lie-Poisson equations}{
        We first assign to any vector field $X:Q\rightarrow TQ$ a linear function $\mu_X:T^*Q\rightarrow\mathbb{R}$ by the following formula:
        \begin{gather}
            \mu_X\left(\alpha|_q\right) := \alpha(X)|_q.
        \end{gather}
        For these functions one has $\{\mu_X, \mu_Y\} = -\mu_{[X, Y]}$.

        Now, choose a basis $\{E^i\}_{i\leq\dim(\mathfrak{g})}$ for $\mathfrak{g}^*$. This basis induces a basis $\{(E^i)_L\}$ of left-invariant one-forms on $G$. The projection of a one-form $\alpha\in T^*G$ onto its component associated to the basis element $(E^i)_L$ gives a map $\mu_i:\mathfrak{g}^*\rightarrow\mathbb{R}$ by the following formula:
        \begin{gather}
            \mu_i\circ\pi:\alpha_k(E^k)_L\mapsto \alpha_i
        \end{gather}
        where $\pi:T^*G\rightarrow T^*G/G\cong\mathfrak{g}$ is the quotient map defined by $(E^i)_L\mapsto E^i$. It can be shown that $\mu_i\circ\pi$ is exactly the linear function associated to the corresponding left-invariant vector field $(E_i)_L$.

        The Lie-Poisson equations for $G$ are the following set of equations:
        \begin{gather}
            \dot{\mu}_i = \{\mu_i, h\}_{\mathfrak{g}^*} = -C^k_{ij}\mu_k\pderiv{h}{\mu_j}
        \end{gather}
        where the Poisson bracket on $\mathfrak{g}^*$ is defined by applying the Poisson reduction theorem to $T^*G$.
    }

\section{Metaplectic structures}

    \newdef{Metaplectic group}{\index{metaplectic!group}
        Consider the symplectic group Sp$(2n, \mathbb{R})$ as defined in \ref{linalgebra:symplectic_group}. This group admits a double covering called the metaplectic group Mp$(2n, \mathbb{R})$.
    }
    \remark{In contrast to the group\footnote{Which is the double cover of SO$(n)$ (see property \ref{clifford:pin_group}).} Spin$(n)$, the metaplectic group is not a matrix group. As such it does not admit a faithful finite-dimensional representation.}

    \newdef{Metaplectic structure}{\index{metaplectic!structure}
        Consider a symplectic manifold $(M, \omega)$ of dimension $2n$. By property \ref{diff:symplectic_G_structure} the frame bundle $F(M)$ can be reduced to a Sp$(2n)$-bundle $\pi_{Sp}:F_{Sp}(M)\rightarrow M$. Now, let $\pi_{meta}:P_{meta}\rightarrow M$ be a principal Mp$(2n)$-bundle over $M$.

        The smooth manifold $M$ is said to have a metaplectic structure if there exists an equivariant 2-fold lifting of $F_{Sp}$ to $P_{meta}$, i.e. a morphism $\xi:P_{meta}\rightarrow F_{Sp}(M)$ together with the 2-fold covering map $\rho:\text{Mp}(2n)\rightarrow\text{Sp}(2n)$ that satisfy:
        \begin{itemize}
            \item $\pi_{Sp}\circ\xi = \pi_{meta}$
            \item $\xi(p\vartriangleleft g) = \xi(p)\cdot\rho(g)$
        \end{itemize}
        for all $g\in\text{Mp}(2n)$, where $\vartriangleleft$ and $\cdot$ denote the right actions of the respective structure groups.
    }
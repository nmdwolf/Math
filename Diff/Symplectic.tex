\chapter{Symplectic Geometry}\label{chapter:symplectic}
\section{Symplectic manifolds}

	\newdef{Symplectic form}{\index{symplectic!form}
		Let $\omega\in\Omega^2(M)$ be a differential 2-form. $\omega$ is said to be symplectic if it satisfies following properties:
		\begin{itemize}
			\item Closedness: $d\omega = 0$
			\item Non-degeneracy: if $\omega(u, v) = 0, \forall u\in TM$ then $v=0$
		\end{itemize}
	}
	\newdef{Symplectic manifold}{\index{symplectic!manifold}
		A manifold $M$ equipped with a symplectic 2-form $\omega$ is called a symplectic manifold. This structure is often denoted as the pair $(M, \omega)$.
	}
	
	\begin{property}[Dimension]
		From the antisymmetry and the non-degeneracy of the symplectic form it follows that $M$ is even-dimensional.
	\end{property}
	
	\begin{theorem}[Darboux]\index{Darboux!chart}
		Let $(M, \omega)$ be a symplectic manifold. For every neighbourhood $\Omega$ in $T^*M$ there exists an adapted coordinate system $(q^i, p^i)$ such that:
		\begin{equation}
			\left.\omega\right|_\Omega = \sum_idp^i\wedge dq^i
		\end{equation}		
		{\normalfont The charts in Darboux's theorem are called \textbf{Darboux charts} and they cover $M$.}
	\end{theorem}
	\begin{remark}
		This theorem shows that all symplectic manifolds of the same dimension are locally isomorphic, i.e. there exist no local invariants, in contrast to for example Riemannian manifolds.
	\end{remark}
	
	\begin{formula}
		In Darboux coordinates the components of the symplectic form $\omega$ read:
		\begin{equation}
			\omega_{ij} = \left(
			\begin{array}{c|c}
				\ 0\ \ &-\mathbbm{1}\\
				\hline
				\ \mathbbm{1}\ \ &0
			\end{array}
			\right)
		\end{equation}
		By the non-degeneracy we can define the 'dual' $\omega^\sharp$ as:
		\begin{equation}
			(\omega^\sharp)^{ij} = \left(
			\begin{array}{c|c}
				0&\ \mathbbm{1}\ \ \\
				\hline
				-\mathbbm{1}&\ 0\ \ 
			\end{array}
			\right)
		\end{equation}
	\end{formula}
	
\subsection{Symplectomorphisms}

	\newdef{Symplectomorphism}{\index{symplectomorphism}
		A symplectomorphism is an isomorphism of symplectic manifolds, i.e. a diffeomorphism $f:(M, \omega_M)\rightarrow (N, \omega_N)$ satisfying:
		\begin{equation}
			f^*\omega_N = \omega_M
		\end{equation}
		These maps form a \textit{semigroup} called the \textbf{symplectomorphism group}\footnote{Not to be confused with the symplectic group $\text{Sp}(n)$.}
	}
	\newdef{Symplectic vector field}{
		A vector field is said to be symplectic if its flow preserves the symplectic form $\omega$:
		\begin{equation}
			\mathcal{L}_X\omega = 0
		\end{equation}
		Equivalently a vector field is symplectic if its flow is a symplectomorphism. These vector fields form a Lie subalgebra of $\mathfrak{X}(M)$.
	}
	
	\newdef{Hamiltonian vector field}{\label{diff:hamilton_vectorfield}\index{Hamilton!vector field}
		Let $(M, \omega)$ be a symplectic manifold. For every function $f\in C^\infty(M)$ we define the associated Hamiltonian vector field $X_f$ by the following relation\footnote{A lot of different conventions exist in the literature. We use the one compatible with the Hamiltonian equations \ref{lagrange:hamilton_equations} which are universally accepted.}:
		\begin{equation}
			\omega(X_f, \cdot) = -df(\cdot)
		\end{equation}
		or by using $\omega^\sharp$:
		\begin{equation}
			X_f(\cdot) = \omega^\sharp(-df, \cdot)
		\end{equation}
		These vector fields form a Lie subalgebra of the Lie algebra of symplectic vector fields. The flow associated to a Hamiltonian vector field is sometimes called a \textbf{Hamiltonian symplectomorphism}.\footnote{The fact that the Hamiltonian flow indeed preserves the symplectic form follows from the closedness of $\omega$.}
	}
	
	\newdef{Poisson bracket}{\index{Poisson!bracket}
		Let $(M, \omega)$ be a symplectic manifold. The Poisson bracket of two functions $f, g\in C^\infty(M)$ is defined as:
		\begin{equation}
			\{g, f\} = X_f(g)
		\end{equation}
		or equivalently:
		\begin{equation}
			X_{\{g, f\}} = [X_f, X_g]
		\end{equation}
	}
	\begin{property}
		The Poisson brackets induced by the symplectic form turns the structure $(C^\infty(M), \{\cdot,\cdot\})$ into a Lie algebra\footnote{The antisymmetry follows from equation \ref{diff:lie_derivative_antisymmetry} and the Jacobi-identity follows from the closedness of $\omega$.} and the second equation in fact gives a (surjective) Lie algebra morphism\footnote{The kernel is given by the constant functions.} $(C^\infty(M), \{\cdot,\cdot\})\rightarrow(\{\text{X : X is a HVF on M}\}, [\cdot, \cdot])$. Furthermore, together with the pointwise multiplication the structure becomes a Poisson algebra\footnote{See definition \ref{lie:poisson_algebra}.}.
	\end{property}
	
	\newdef{Poisson manifold}{
		A smooth manifold on which the algebra of smooth functions can be equipped with a Poisson algebra structure.
	}
	\begin{property}
		From the property above every symplectic manifold is a Poisson manifold. The converse however is not true.
	\end{property}
	
\section{Lagrangian submanifolds}

	\newdef{Symplectic complement}{
		Let $(M, \omega)$ be a symplectic manifold and let $S\subset M$ be an embedded submanifold $\iota: S\hookrightarrow M$. The symplectic orthogonal complement $T^\bot_pS$ at the point $p\in S$ is defined as\footnote{The notation $T^\omega_pS$ is also sometimes used.}:
		\begin{equation}
			T^\bot_pS = \{v\in T_pM: \omega(v, \iota_* w) = 0, \forall w\in T_pS\}
		\end{equation}
	}
	
	\newdef{Isotropic submanifold}{\index{isotropic}
		Let $(M, \omega)$ be a symplectic manifold. An embedded submanifold $\iota:S\hookrightarrow M$ is called isotropic if $T_pS\subset T^\bot_pS$.
	}
	\newdef{Isotropic submanifold}{
		Let $(M, \omega)$ be a symplectic manifold. An embedded submanifold $\iota:S\hookrightarrow M$ is called co-isotropic if $T^\bot_pS\subset T_pS$.
	}
	\newdef{Larangian submanifold}{\index{Lagrange!Lagrangian submanifold}
		Let $(M, \omega)$ be a symplectic manifold. An embedded submanifold $\iota:S\hookrightarrow M$ is called Lagrangian if $T_pS = T^\bot_pS$. Therefore they are sometimes called maximal isotropic submanifolds.
	}
	
	\newdef{Polarization}{\index{polarization}
		A polarization of a symplectic manifold $(M, \omega)$ is a foliation by Lagrangian submanifolds, i.e. a subbundle\footnote{One often looks at a subbundle of the complexified tangent bundle $T_{\mathbb{C}}M$ (in this case the symplectic form is extended linearly to the complexified tangent spaces).} $P\subset TM$ such that the following conditions are satisfied:
		\begin{itemize}
			\item (Maximality): $\dim TM=2\dim P$
			\item (Isotropy): $\iota_X\omega = 0$ for all $X\in P$
			\item (Integrability\footnote{This condition characterizes that $P$ is a foliation by Frobenius' integrability theorem.}): $[X, Y]=0$ for all $X, Y\in P$
		\end{itemize}
	}
	
\section{Cotangent bundle}

	\begin{construct}[Liouville one-form]\index{Liouville!1-form}\index{canonical!1-form|see{Liouville}}
		Let $M$ be a smooth manifold. The cotangent bundle $T^*M$ comes equipped with a canonical symplectic form: Let $(q, p)$ denote the local coordinates on $T^*M$ and define a 1-form $\alpha$ by \[\alpha = p_idq^i\] In coordinate-free notation this can be written as follows:
		\begin{equation}
			\alpha(z) = \pi_2(z)\Big(\pi_*(z)\Big)
		\end{equation}
		where $z\in TT^*M$, $\pi: T^*M\rightarrow M$ and $\pi_2:TT^*M\rightarrow T^*M$. This 1-form, called the \textbf{canonical 1-form}\footnote{Also known as the \textbf{Liouville 1-form}.}, then induces a symplectic form in the by exterior derivation: \[\omega = d\alpha\]
	\end{construct}
	
\section{Hamiltonian dynamics}

	\newdef{Completely integrable system}{\index{CIS}
		Consider a multimap \[F \equiv (F_1, ..., F_n):M\rightarrow\mathbb{R}^n\] on a symplectic manifold $(M, \omega)$. This map defines a completely integrable system (CIS) if it satisfies the following conditions:
		\begin{itemize}
			\item $\dim M = 2n$
			\item The Hamiltonian vector fields $\{X_{F_i}\}_{i\leq n}$ are almost everywhere linearly independent.\footnote{Equivalently we require that the Jacobian $DF$ has full rank almost everywhere.}
			\item For every $i, j\leq n$ we have that $\{F_i, F_j\} = 0$.
		\end{itemize}
	}

\subsection{Hamiltonian actions}

	\newdef{Hamiltonian torus action}{\index{Hamilton!action}
		Let $(M, \omega)$ be a symplectic manifold. Let $T^n$ be a torus group acting on $M$. The action of $T^n$ is said to be Hamiltonian if there exists a CIS on $M$ such that the action arises as the Hamiltonian flow of $F$.
	}
	
\section{Symplectic reduction}

	\newdef{Reduced vector field}{
		Let $M$ be a smooth manifold and $G$ a Lie group which acts freely and properly on $M$. Now suppose that $G$ acts as a symmetry group on the vector field $X\in\mathfrak{X}(M)$, i.e. $\Phi_g^*X=X$ for all $g\in G$. The reduced vector field $\overline{X}$ is defined through the following equation:
		\begin{gather}
			\overline{X}(\pi(m)) = \pi_*X(m)
		\end{gather}
		where $\pi:M\rightarrow M/G$ is the quotient map of the associated principal $G$-bundle.
	}

\subsection{Poisson reduction}

	\newdef{Poisson map}{
		Let $(M, \{\cdot, \cdot\})$ and $(N, [\cdot, \cdot])$ be two Poisson manifolds. A Poisson map $\Phi:M\rightarrow N$ is a map satisfying the following equality for all $f, g\in C^\infty(N)$:
		\begin{gather}
			[f, g]\circ\Phi = \{f\circ\Phi, g\circ\Phi\}
		\end{gather}
	}
	\newdef{Poisson action}{
		Let $G$ be a Lie group and let $(M, \{\cdot, \cdot\})$ be a Poisson manifold. A $G$-action on $M$ is said to be a Poisson action or \textbf{canonical action} if every $g\in G$ acts by a Poisson map.
	}
	
	\begin{theorem}[Poisson reduction]
		Let $G$ be a Lie group which acts freely and properly on a Poisson manifold $(M, \{\cdot, \cdot\})$. If the action is canonical then Poisson bracket on $M$ descends (uniquely) to a Poisson bracket on the quotient manifold $M/G$. Furthermore the projection $\pi:M\rightarrow M/G$ is a Poisson map with respect to this structure.
	\end{theorem}
	\begin{property}
		Let $H:M\rightarrow\mathbb{R}$ be a $G$-invariant Hamiltonian function. Its Hamiltonian vector field $X_H$ is also $G$-invariant and the Hamiltonian vector field of the reduced Hamiltonian $h:M/G\rightarrow\mathbb{R}$, defined by $H\equiv h\circ\pi$, is given by the reduced vector field of $X_H$.
	\end{property}

\subsection{Lie-Poisson reduction}

	In the case of Lie-Poisson reduction one considers the cotangent bundle $T^*G$ of a Lie group $G$ as the configuration manifold. It is not to hard to show that $T^*G/G\cong\mathfrak{g}^*$.

	\newformula{Lie-Poisson equations}{
		We first assign to any vector field $X:Q\rightarrow TQ$ a linear function $\mu_X:T^*Q\rightarrow\mathbb{R}$ by the following formula:
		\begin{gather}
			\mu_X\left(\alpha|_q\right) = \alpha(X)|_q
		\end{gather}
		For these functions one has $\{\mu_X, \mu_Y\} = -\mu_{[X, Y]}$.
		
		Now choose a basis $\{E^i\}$ for $\mathfrak{g}^*$. This basis induces a basis $\{(E^i)_L\}$ of left-invariant 1-forms on $G$. The projection of a 1-form $\alpha\in T^*G$ onto its component associated to the basis element $(E^i)_L$ gives a map $\mu_i:\mathfrak{g}^*\rightarrow\mathbb{R}$ by the following formula:
		\begin{gather}
			\mu_i\circ\pi:\alpha_k(E^k)_L\mapsto \alpha_i
		\end{gather}
		where $\pi:T^*G\rightarrow T^*G/G$ is the quotient map defined by $(E^i)_L\mapsto E^i$.	It can be shown that $\mu_i\circ\pi$ is exactly the linear function associated to the corresponding left-invariant vector field $(E_i)_L$.
		
		The Lie-Poisson equations for $G$ are the following set of equations:
		\begin{gather}
			\dot{\mu}_i = \{\mu_i, h\}_{\mathfrak{g}^*} = -C^k_{ij}\mu_k\pderiv{h}{\mu_j}
		\end{gather}
		where the Poisson bracket on $\mathfrak{g}^*$ is defined using the Poisson reduction theorem on $T^*G$.
	}

	

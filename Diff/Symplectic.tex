\chapter{Symplectic Topology}\label{chapter:symplectic}
\section{Symplectic manifolds}

	\newdef{Symplectic form}{\index{symplectic!form}
		Let $\omega\in\Omega^2(M)$ be a differential 2-form. $\omega$ is said to be a symplectic form if it satisfies following properties:
		\begin{itemize}
			\item Closed: $d\omega = 0$
			\item Non-degeneracy: if $\omega(u, v) = 0, \forall u\in TM$ then $v=0$
		\end{itemize}
	}
	\newdef{Symplectic manifold}{\index{symplectic!manifold}
		A manifold $M$ equipped with a symplectic 2-form $\omega$ is called a symplectic manifold. This structure is often denoted as a pair $(M, \omega)$.
	}
	
	\begin{property}
		From the antisymmetry (property of all differential $k$-forms) and the non-degeneracy of the symplectic form, it follows that $M$ is even dimensional.
	\end{property}
	
	\begin{theorem}[Darboux]\index{Darboux!chart}
		Let $(M, \omega)$ be a symplectic manifold. For every neighbourhood $\Omega$ in $T^*M$ there exists a fibered coordinate system $(x^i, y^i)$ such that
		\begin{equation}
			\left.\omega\right|_\Omega = \sum_idx^i\wedge dy^i
		\end{equation}		
		{\normalfont The charts in Darboux's theorem are called \textbf{Darboux charts} and they form a cover of $M$.}
	\end{theorem}
	\begin{formula}
		In Darboux coordinates the symplectic form $\omega$ takes the form
		\begin{equation}
			\omega_{ij} = \left(
			\begin{array}{c|c}
				0&-\mathbbm{1}\\
				\hline
				\mathbbm{1}&0
			\end{array}
			\right)
		\end{equation}
		By the non-degeneracy we can define the 'dual' $\omega^\sharp$ as:
		\begin{equation}
			(\omega^\sharp)^{ij} = \left(
			\begin{array}{c|c}
				0&\mathbbm{1}\\
				\hline
				-\mathbbm{1}&0
			\end{array}
			\right)
		\end{equation}
	\end{formula}
	
	\newdef{Hamiltonian vector field}{\index{Hamilton!vector field}
		Let $(M, \omega)$ be a symplectic manifold. For every function $f\in C^\infty(M)$ we define the associated Hamiltonian vector field $X_f$ by the following relation:
		\begin{equation}
			\omega(X_f, \cdot) = df(\cdot)
		\end{equation}
		or by using $\omega^\sharp$:
		\begin{equation}
			X_f(\cdot) = \omega^\sharp(df, \cdot)
		\end{equation}
	}
	
	\newdef{Poisson bracket}{\index{Poisson!bracket}
		Let $(M, \omega)$ be a symplectic manifold. The Poisson bracket of two functions $f, g\in C^\infty(M)$ is defined as:
		\begin{equation}
			\{f, g\} = X_f(g)
		\end{equation}
		or equivalently:
		\begin{equation}
			X_{\{f, g\}} = [X_f, X_g]
		\end{equation}
		This turns the structure $(C^\infty(M), \{\cdot,\cdot\})$ into a Lie algebra\footnote{The antisymmetry follows from equation \ref{diff:lie_derivative_antisymmetry} and the Jacobi-identity follows from the closedness of $\omega$.} and the second equation in fact gives a Lie algebra isomorphism $(C^\infty(M), \{\cdot,\cdot\})\cong_{Lie}(\{\text{X : X is a HVF on M}\}, [\cdot, \cdot])$. Furthermore, together with the pointwise multiplication the structure becomes a Poisson algebra\footnote{See definition \ref{lie:poisson_algebra}.}.
	}
	
\section{Lagrangian submanifolds}

	\newdef{Symplectic complement}{
		Let $(M, \omega)$ be a symplectic manifold and let $S\subset M$ be an embedded submanifold $\iota: S\hookrightarrow M$. The symplectic orthogonal complement $T^\bot_pS$ at the point $p\in S$ is defined as:
		\begin{equation}
			T^\bot_pS = \{v\in T_pM: \omega(v, \iota_* w) = 0, \forall w\in T_pS\}
		\end{equation}
	}
	
	\newdef{Isotropic submanifold}{\index{isotropic}
		Let $(M, \omega)$ be a symplectic manifold. An embedded submanifold $\iota:S\hookrightarrow M$ is called isotropic if $T_pS\subset T^\bot_pS$.
	}
	\newdef{Isotropic submanifold}{
		Let $(M, \omega)$ be a symplectic manifold. An embedded submanifold $\iota:S\hookrightarrow M$ is called co-isotropic if $T^\bot_pS\subset T_pS$.
	}
	\newdef{Larangian submanifold}{\index{Lagrange!Lagrangian submanifold}
		Let $(M, \omega)$ be a symplectic manifold. An embedded submanifold $\iota:S\hookrightarrow M$ is called Lagrangian if $T_pS = T^\bot_pS$. Therefore they are sometimes called maximal isotropic submanifolds.
	}

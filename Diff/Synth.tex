\chapter{Synthetic Differential Geometry}

\section{Neighbourhoods}

    \newdef{Neighbourhood relation}{
        A reflexive and symmetric relation $\sim$ with the additional property that the morphisms in the category under consideration preserve this relationship.
    }
    \begin{example}[Monad]\index{monad}
        Let $M$ be a set. Given a neighbourhood relation $\sim$ on $M$, the (first order) monad around $x\in M$ is defined as
        \begin{gather}
            \underline{\mathfrak{M}}(x) := \{y\in M:y\sim x\}.
        \end{gather}
    \end{example}

    \newdef{Infinitesimal simplex}{\index{simplex!infinitesimal}
        An infinitesimal $k$-simplex with respect to a neighbour relation $\sim$ is a collection of $k+1$ points $\{x_i\}_{i\leq k}$ such that $x_i\sim x_j$ for every $i, j\leq k$.
    }

    \newdef{Geometric distribution}{\index{distribution!involutive}
        Let $M$ be a set equipped with a neighbourhood relation $\sim$. A (geometric) distribution on $M$ is a reflexive symmetric refinement $\approx$ of $\sim$. A distribution is said to be \textbf{involutive} if
        \begin{gather}
            (x\approx y)\land(y\approx z)\land(x\sim z)\implies x\approx z
        \end{gather}
        for all $x, y, z\in M$.
    }
    \newdef{Integral subset}{\index{integral!subset}
        Let $M$ be a set equipped with a neighbourhood relation $\sim$ and an associated distribution $\approx$. A subset $N\subseteq M$ is said to be integral with respect to $\approx$ if $\approx$ and $\sim$ coincide on $N$.
    }
    \begin{theorem}[Frobenius' theorem]\index{Frobenius!integrability theorem}\index{leaf}
        An involutive distribution admits maximal connected integral subsets, these are called \textbf{leaves}.
    \end{theorem}

\section{Affine connections}
    \newdef{Affine connection}{\index{connection!affine}
        An affine connection is a map $\lambda(x, y, z)$ which for every three points $x, y, z\in M$ such that $y\sim x$ and $z\sim x$ gives a point $w\in M$ such that $w\sim z$ and $w\sim y$. Graphically this is given by a completion of diagram \ref{fig:synth_connection} to diagram \ref{fig:synth_connection_complete}.
        \begin{figure}[ht!]
            \centering
            \begin{subfigure}{0.49\textwidth}
                \centering
                \begin{tikzpicture}
                    \matrix (m) [matrix of math nodes,row sep=4em,column sep=4em, minimum width=2em, ampersand replacement=\&]{
                        \&\&\phantom{w}\\
                        z\&\&y\\
                        x\&\&\\
                    };
                    \draw (m-2-1) -- (m-3-1);
                    \draw[->] (m-3-1) -- (m-2-3);
                \end{tikzpicture}
                \caption{Neighbouring points.}
                \label{fig:synth_connection}
            \end{subfigure}
            \begin{subfigure}{0.49\textwidth}
                \centering
                \begin{tikzpicture}
                    \matrix (m) [matrix of math nodes,row sep=4em,column sep=4em, minimum width=2em, ampersand replacement=\&]{
                        \&\&w\\
                        z\&\&y\\
                        x\&\&\\
                    };
                    \draw (m-2-1) -- (m-3-1);
                    \draw[->] (m-3-1) -- (m-2-3);
                    \draw[dashed, ->] (m-2-1) -- (m-1-3);
                    \draw[dashed] (m-2-3) -- (m-1-3);
                \end{tikzpicture}
                \caption{Connection in synthetic theories.}
                \label{fig:synth_connection_complete}
            \end{subfigure}
        \end{figure}
    }
    \sremark{By looking at these diagrams the concept of parallel transport can be made a lot more intuitive than in classic differential geometry, e.g. diagram \ref{fig:synth_connection_complete} shows the parallel transport of the point $z$ along $xy$.}

    \newdef{Symmetric connection}{\index{torsion}
        An affine connection $\lambda$ is said to be symmetric or \textbf{torsion-free} if $\lambda(x, y, z) = \lambda(x, z, y)$.
    }

    \newdef{Flat connection}{\index{connection!flat}
        An affine connection $\lambda$ is said to be flat or \textbf{curvature-free} if parallel transporting a point around an infinitesimal 2-simplex gives that same point again.
    }
    \newdef{Curvature}{\index{curvature}
        Let $\lambda$ be an affine connection on $M$. The curvature of $\lambda$ is the map $\mathcal{R}$ which assigns to every infinitesimal 2-simplex $\{x_0, x_1, x_2\}$ the following automorphism: \[\underline{\mathfrak{M}}(x_0)\rightarrow\underline{\mathfrak{M}}(x_0):z\mapsto\text{result of parallel transporting z around }\{x_0, x_1, x_2\}.\]
    }

    \newdef{Geodesic}{\index{geodesic}
        A subset $S\subseteq M$ stable under the affine connection $\lambda$.
    }

\section{Euclidean geometry}
\subsection{Infinitesimal elements}

    \newdef{Infinitesimal line}{\index{infinitesimal}
        Let $R$ be the line. By picking two distinct points, labelled $0$ and $1$, one can turn the line into a commutative ring\footnote{Or more explicitly an algebra over the rationals $\mathbb{Q}$.} $(R, +, \cdot)$. The infinitesimal line is then defined as the following set:
        \begin{gather}
            \Delta := \{x\in R: x^2 = 0\}.
        \end{gather}
        A neighbourhood relation on $R$ is then induced by setting $\underline{\mathfrak{M}}(0)\equiv\Delta$.
    }
    \remark{If one would follow the Euclidean point of view, this set would be $\{0\}$. However by not requiring $R$ to be a field we obtain a larger set.}

    \begin{axiom}\index{slope}\label{synth:axiom1}
        For every map $f:\Delta\rightarrow R$ there exists a unique element $b\in R$, called the \textbf{slope} of $f$, such that:
        \begin{gather}
            f(d) = f(0) + d\cdot b
        \end{gather}
        for all $d\in\Delta$.
    \end{axiom}
    \result{The map $\alpha:R\times R\rightarrow R^\Delta: (a, b)\mapsto (f:d\mapsto a+d\cdot b)$ is invertible and hence an isomorphism\footnote{If one equips the set $R\times R$ with the multiplication rule $(a, b)\cdot(a', b') = (a\cdot a', a\cdot b' + a'\cdot b)$ then this becomes an $R$-algebra isomorphism.}.}
    \result{Let $a, b\in R$. If $d\cdot a = d\cdot b$ for all $d\in\Delta$ then $a = b$.}

    \begin{notation}
        Analogous to the infinitesimal line one defines the subsets $D_k$ in the following way:
        \begin{gather}
            D_k := \{x\in R: x^{k+1} = 0\}.
        \end{gather}
    \end{notation}

\subsection{Calculus}

    \begin{formula}[Taylor expansion]\index{Taylor!expansion}\index{Leibniz!rule}
        From axiom \ref{synth:axiom1} we derive the following exact Taylor expansion:
        \begin{gather}
            f(x+d) = f(x) + d\cdot f'(x)
        \end{gather}
        where $f'(x)$ can be interpreted as the derivative of $f$ at the point $x\in R$.
    \end{formula}
    \sremark{If $f$ also depends on additional parameters in $R$, then we can define the partial derivatives in a similar fashion.}
    \begin{property}
        Using axiom \ref{synth:axiom1} it can be easily proven that the derivative is linear and satisfies Leibniz's rule.
    \end{property}

    Using the sets $D_k$ one can prove higher order expansions. First we generalize axiom \ref{synth:axiom1}:
    \begin{axiom}\label{synth:axiom1b}
        For every $g:D_k\rightarrow R$ there exist unique elements $\{b_1, \ldots, b_k\}$ in $R$ such that
        \begin{gather}
            f(d) = f(0) + \sum_jd^j\cdot b_j
        \end{gather}
        for all $d\in D_k$.
    \end{axiom}
    \begin{result}
        Let $f:R\rightarrow R$ and $d\in D_k$.
        \begin{gather}
            f(x+d) = f(x) + d\cdot f'(x) + \cdots + \frac{d^k}{k!}\cdot f^{(k)}(x)
        \end{gather}
    \end{result}

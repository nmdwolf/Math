\chapter{Calculus of Variations}\label{chapter:variation}

    The standard references for global variational calculus are \cite{var_bicomplex, takens}.

\section{Constrained systems}
\subsection{Holonomic constraints}

    \newdef{Holonomic constraint}{
        A constraint $f(q,t) = 0$ is said to be holonomic if it only depends on the coordinates $q^i$ and $t$.
    }
    \begin{method}[Holonomic constraints]
        The Euler-Lagrange equations of a system with $k$ holonomic constraints $f_k(q,t) = 0$ can be obtained from the generalized action functional
        \begin{gather}
            \int_a^b\Big[L(q(t),\dot{q}(t),t) + \sum_{j=1}^k\lambda_j(t)f_j(q(t),t)\Big]dt
        \end{gather}
        where $\lambda_j(t)$ are undetermined (Lagrange) multipliers.
    \end{method}

\section{Noether symmetries}
\subsection{Classical systems}

    \newdef{Noether symmetry}{
        Consider an integral quantity $I$ defined through some Lagrangian function $L(q,u)$:
        \begin{gather}
            I_M = \int_ML(q,u,\partial u)dq
        \end{gather}
        where $u$ are (analytic) functions of the variables $q$. A transformation $q\longrightarrow q',u\longrightarrow u'$ of the variables\footnote{The transformations of the derivatives $\partial u$ are induced by the ones for $u$.} is called a Noether symmetry for $L$ if it satisfies
        \begin{gather}
            \int_{M'}L(q',u',\partial u')dq' = \int_ML(q,u,\partial u)dq
        \end{gather}
        for arbitrary $M$.
    }

    Following Lie we introduce the notion of a group of transformations:
    \newdef{Finite continuous group}{
        A collection of analytic functions, closed under inverses and composition, such that every function depends analytically on a finite number of parameters. In this chapter we will denote these groups by $\mathfrak{G}_k$ (where $k$ is the number of independent parameters).
    }
    \remark{It should be clear that this is the same as a finite-dimensional Lie group (see definition \ref{lie:lie_group}).}

    Instead of a parameters, one can also generalize to functions:
    \newdef{Infinite continuous group}{
        A collection of analytic functions, closed under inverses and composition, such that every function depends analytically on a finite number of arbitrary (analytic) functions. In this chapter we will denote these groups by $\mathfrak{G}_{\infty,k}$ (where $k$ is the number of independent functions).
    }
    \remark{In physics terminology the infinite groups would be the symmetry groups obtained by gauging a global symmetry $\mathfrak{G}_k$.}

    \begin{theorem}[Noether]\index{Noether}
        Consider an integral quantity $I$ that is invariant under some group $\mathfrak{G}$.
        \begin{itemize}
            \item If $\mathfrak{G}$ is finite continuous (and hence of the form $\mathfrak{G}_k$) then there exist $k$ independent (linear) combinations among the Lagrangian expressions of $I$ that are equal to divergences. Conversely, if there exist $k$ independent combinations among the Lagrangian expressions that are divergences, then $I$ is invariant under a group of the form $\mathfrak{G}_k$.
            \item If $\mathfrak{G}$ is infinite continuous (and hence of the form $\mathfrak{G}_{\infty,k}$) then there exists $k$ independent relations among the Lagrangian expressions and their derivatives\footnote{The order up to which the derivatives occur is equal to the order of derivatives up to which the transformations depend on the $k$ arbitrary functions.}. Conversely, if $k$ such relations exist, then the integral $I$ is invariant under a group of the form $\mathfrak{G}_{\infty,k}$.
        \end{itemize}
    \end{theorem}
    \remark{In fact the first theorem is also valid in the limit of an infinite number of parameters.}

    \newdef{Improper relations}{
        Divergence relations $\sum_i\psi_i\delta u_i = \nabla\cdot B$ obtained in a variational problem with symmetry group $\mathfrak{G}_k$ can be classified into two groups:
        \begin{itemize}
            \item If the quantities $B$ are linear combinations of Lagrangian expressions (and their derivatives) then the divergence relations are said to be improper. It can be shown that this is the case if and only if $\mathfrak{G}_k$ is a subgroup of an infinite continuous symmetry group $\mathfrak{G}_{\infty,k}$.
            \item In all other cases the divergence relations are said to be proper.
        \end{itemize}
    }

    For Lagrangians describing ''point particles'', hence where $M\subseteq\mathbb{R}$, we can obtain the following result:
    \begin{example}[One dimension]
        Infinitesimal transformations
        \begin{align*}
            q^i \longrightarrow q^i& + \varepsilon\xi^i(q^k,t)\\
            t \longrightarrow t& + \varepsilon\tau(q^k,t)\\
            \dot{q}^i \longrightarrow \dot{q}^i& + \varepsilon(\dot{\xi}^i - \dot{q}^i\dot{\tau})
        \end{align*}
        generate Noether symmetries if they leave the Lagrangian invariant up to a total derivative (in first order) for every subinterval $[t_0, t_1]\subseteq[a, b]$ and for some function $f(q, t)$:
        \begin{gather}
            \int_{\tilde{t}_0}^{\tilde{t}_1}L(\tilde{q}, \dot{\tilde{q}}, \tilde{t})d\tilde{t} = \int_{t_0}^{t_1}L(q,\dot{q},t)dt + \varepsilon\int_{t_0}^{t_1}\deriv{f}{t}dt + O(\varepsilon^2).
        \end{gather}
        This is equivalent to requiring that the transformation is a solution of the following differential equation:
        \begin{gather}
            \pderiv{L}{\tau} + \pderiv{L}{q^i}\xi^i + \pderiv{L}{\dot{q}^i}(\dot{\xi}^i - \dot{q}^i\dot{\tau}) + L\dot{\tau} = \dot{f}.
        \end{gather}
        By Noether's (first) theorem we obtain for every such symmetry a conserved quantity of the following form
        \begin{gather}
            F := f - \left[L\tau + \pderiv{L}{\dot{q}^i}(\xi^i - \dot{q}^i\tau)\right].
        \end{gather}
    \end{example}

    ?? IS THIS A GENUINE GENERALIZATION OR NOT ??

\section{Jet bundles}\label{section:jet_bundles}

    Although the following constructions can be defined in the general context of fibred manifolds we will only consider them in the case of smooth fibre bundles. Only the notion of a jet will be defined in general for maps between smooth manifolds (such that we can define the Whitney topology on the space of smooth functions).

    \newdef{Jet}{\index{jet}\index{order}
        Consider two smooth manifolds $M,N$. Two morphism $\sigma,\xi\in C^\infty(M, N)$ with local coordinates $(\sigma^i)$ and $(\xi^i)$ define the same $r$-jet at a point $p\in M$ if and only if
        \begin{gather}
            \left.\mpderiv{\alpha}{\sigma^i}{x}\right|_p = \left.\mpderiv{\alpha}{\xi^i}{x}\right|_p
        \end{gather}
        for all $0\leq i\leq\dim M$ and every multi-index $\alpha$ with $0\leq|\alpha|\leq r$. It is clear that this relation defines an equivalence relation. The $r$-jet at $p\in M$ with representative $\sigma$ is denoted by $j_p^r\sigma$. The number $r$ is called the \textbf{order} of the jet.
    }

    \newdef{Whitney $C^k$-topology}{\index{Whitney!$C^k$-topology}\index{$C^k$-topology|seealso{Whitney}}
        Let $M, N$ be two smooth manifolds and consider the manifold\footnote{This manifold will be constructed further down this section in the case of fibre bundles.} of $k$-jets $J^k(M, N)$. A basis for the Whitney $C^k$-topology on $C^\infty(M, N)$ is given by the sets
        \begin{gather}
            S^k(U) := \{f\in C^\infty(M, N):J^kf\in U\}
        \end{gather}
        where $U$ is open in $J^k(M, N)$.
    }
    \begin{property}
        When the manifold $M$ is compact, the Whitney and compact-open topologies on $C^\infty(M, N)$ coincide. In general the Whitney topology is the topology of global uniform convergence.
    \end{property}

    \remark{The $k$-jet construction is easily carried over to the theory of bundles. To construct the jet space of sections on a (smooth) fibre bundle $(E, B, \pi)$ one just sets $M=B$ and $N=E$ in the previous definition. From this point on we will restrict to this case.}

    \newdef{Jet manifold}{
        Consider a fibre bundle $(E, B, \pi)$. The $r$-jet manifold $J^r(\pi)$ of the projection $\pi$ is defined as follows:
        \begin{gather}
            J^r(\pi) := \{j_p^r\sigma: \sigma\in\Gamma(E), p\in B\}.
        \end{gather}
        The set $J^0(\pi)$ is identified with the total space $E$.
    }

    \newdef{Jet projections}{
        Let $(E, B, \pi)$ be a fibre bundle with $r$-jet manifolds $J^r(\pi)$. The \textbf{source projection} $\pi_r$ and \textbf{target projection} $\pi_{r, 0}$ are defined as the maps
        \begin{align}
            \pi_r&:J^r(\pi)\rightarrow B:j_p^r\sigma\mapsto p\\
            \pi_{r, 0}&:J^r(\pi)\rightarrow E:j_p^r\sigma\mapsto \sigma(p).
        \end{align}
        These projections satisfy $\pi_r = \pi\circ\pi_{r, 0}$. We can also define a \textbf{$k$-jet projection} $\pi_{r, k}$ as the map
        \begin{gather}
            \pi_{r, k}:J^r(\pi)\rightarrow J^k(\pi):j_p^r\sigma\mapsto j_p^k\sigma
        \end{gather}
        where $k\leq r$. The $k$-jet projections satisfy the transitivity property $j_{k, m} = j_{r, m}\circ j_{k, r}$.
    }
    \begin{remark*}
        The names ''source'' and ''target'' come from the fact that for trivial bundles (or local trivializations) sections of $\pi:B\times F\rightarrow B$ are characterized by (global) functions $B\rightarrow F$. It follows that the jet bundles $J^k(\pi)$ and $J^k(B, F)$ are isomorphic.
    \end{remark*}
    \newdef{Prolongation}{\index{prolongation}
        Let $\sigma$ be a section of a fibre bundle $\pi:E\rightarrow B$. The $r$-jet prolongation $j^r\sigma$ corresponding to $\sigma$ is defined as the following map:
        \begin{gather}
            j^r\sigma:B\rightarrow J^r(\pi):p\mapsto j_p^r\sigma.
        \end{gather}
    }

    \newdef{Jet bundle}{
        The $r$-jet bundle corresponding to the projection $\pi$ is defined as the triple $(J^r(\pi), B, \pi_r)$. The bundle charts $(U_i, \varphi_i)$\footnote{Where $\varphi_i = (x^k, u^\alpha)$ with $x^k$ the base space coordinates and $u^\alpha$ the total space coordinates.} on $E$ define induced bundle charts on $J^r(\pi)$ in the following way:
        \begin{align}
            U_i^r &:= \{j_p^r\sigma: \sigma(p)\in U_i\}\\\nonumber\\
            \varphi_i^r &:= \left(x^k, u^\alpha, \left.\mpderiv{I}{u^\alpha}{x}\right|_p \right)
        \end{align}
        where $I$ is a multi-index such that $0\leq|I|\leq r$. The partial derivatives \[\left.\mpderiv{I}{u^\alpha}{x}\right|_p\] are called the \textbf{derivative coordinates} on $J^r(\pi)$.
    }

    \newdef{Holonomic section}{\index{section!holonomic}
        Consider a fibre bundle $\pi:E\rightarrow M$. A (local) section $\sigma$ of $\pi$ gives rise to a (local) section of $\pi_r$ given by the $r$-jet prolongation of $\sigma$. Sections of $J^r(E)$ which lie in the image of $j^r$ are said to be holonomic.
    }

    \newdef{Infinite jet bundle}{
        The inverse limit \ref{set:inverse_limit} of the projections $\pi_{k,k-1}:J^k(\pi)\rightarrow J^{k-1}(\pi)$. It can be shown (in an algebro-geometric fashion) that the smooth functions on the infinite jet bundle $J^\infty(\pi)$ are (at least locally) given by smooth functions on some finite jet bundle. This just means that we define the infinite jet bundle by taking its algebra of smooth functions to be the direct limit of those on the finite jet bundles. By extension it can be shown that any smooth morphism $J^\infty(E)\rightarrow E'$ into a finite-dimensional manifold factorizes through a finite jet bundle. Furthermore, we say a map $Q\rightarrow J^\infty(E)$ or $J^\infty(E)\rightarrow J^\infty(E')$ is smooth if the composition with any smooth map is again smooth.
    }

    \newdef{Differential operator}{\index{differential!operator}\label{diff:differential_operator}
        Let $E_1,E_2$ be two smooth fibre bundles over the same base manifold $M$. A differential operator $\widetilde{D}:E_1\rightarrow E_2$ is a bundle morphism $J^\infty(E_1)\rightarrow E_2$. This induces a map of sections $D:\Gamma(E_1)\rightarrow\Gamma(E_2)$ such that $\widetilde{D}=D\circ j^\infty$.
    }

    \begin{property}[Formal adjoints]\index{adjoint}\label{var:formal_adjoint}
        Consider two differential operators $D,D^\dagger:\Gamma(E)\rightarrow\Gamma(E^*\otimes\Lambda^{\dim(M)}(M))$. These operators are said to be formally adjoint if there exists a bilinear differential operator $K:\Gamma(E)\otimes\Gamma(E)\rightarrow\Omega^{\dim(M)}(M)$ such that the following condition is satisfied for all sections $\sigma_1,\sigma_2$ of $E$:
        \begin{gather}
            \langle D(\sigma_1),\sigma_2 \rangle - \langle \sigma_1,D^\dagger(\sigma_2) \rangle = dK(\sigma_1,\sigma_2).
        \end{gather}
        This formula can be interpreted as a generalization of Green's identities. In the case where $M$ is compact, Stokes's theorem \ref{forms:theorem:stokes_theorem} shows that $D$ and $D^\dagger$ are related through integration by parts.
    \end{property}

    \newdef{Generalized vector field}{\index{vector field!generalized}
        Consider a vector bundle $\pi:E\rightarrow M$. A generalized vector field on $E$ is locally defined by smooth functions on the infinite jet bundle $J^\infty(E)$.
    }

\subsection{Contact structure}

    Every jet space $J^k$ carries a natural constact structure\footnote{See chapter \ref{chapter:contact}.}:
    \newdef{Contact form}{\index{contact!form}\index{Cartan!distribution}
        Given a fibre bundle $\pi:E\rightarrow M$ and one of its jet bundles $J^k(E)$, we say that a differential form $\omega\in\Omega^\bullet(J^k(E))$ is a contact form if it is annihilated by all jet prolongations, i.e. $(j^{k+1}\sigma)^*\omega=0$ for all sections $\sigma\in\Gamma(E)$.

        The space of such contact forms generates a differential ideal and this in turn defines a distribution, called the \textbf{Cartan distribution}. For finite order jet spaces this distribution is completely non-integrable, but for infinite jet bundles (where Frobenius's theorem need not hold) the distribution becomes involutive and integrable.

        Locally the differential ideal is generated by the following contact forms:
        \begin{gather}
            \theta_I^\alpha := \mathbf{d}u^\alpha_I - \sum_i u^\alpha_{I, i}\mathbf{d}x^i
        \end{gather}
        where the $x^i$ are the independent variables and the $u^\alpha_I$ are the dependent variables.
    }

    Now consider a vector field $X$ on the total space $E$. There exists a vector field on $J^k(E)$, called the prolongation of $X$:
    \newdef{Prolongation}{\index{prolongation}\index{total!derivative}\label{var:prolongation}
        Given a generalized vector field $X$ on a fibre bundle $\pi:E\rightarrow M$, there exists a unique vector field $j^kE$ on the jet bundle $J^k(E)$ defined by the following conditions:
        \begin{enumerate}
            \item $X$ and $j^kX$ coincide on $C^\infty(E)$.
            \item $j^kX$ preserves the contact ideal, i.e. if $\theta$ is a contact form, then $\mathcal{L}_{j^kX}\theta$ is also a contact form.
        \end{enumerate}
        Locally, the prolongation of a vector field
        \begin{gather}
            X = X^\mu\partial_\mu + X^\alpha\partial_\alpha
        \end{gather}
        is given by $(|I|\leq k)$:
        \begin{gather}
            (j^k X)^\alpha_I = D_I(X^\alpha - u^\alpha_\mu X^\mu) + u^\alpha_{\mu I}X^\mu.
        \end{gather}
        where we introduced the ''total derivative'' operator\footnote{In fact this formula is virtually the same as the one for the true total derivative. However, partial derivatives are replaced by jet coordinates.} $D_\mu$:
        \begin{gather}
            D_\mu f := \pderiv{f}{x^\mu} + \pderiv{f}{u^\alpha}u^\alpha_{,\mu} + \pderiv{f}{u^\alpha_{,\nu}}u^\alpha_{,\nu\mu} + \cdots.
        \end{gather}
    }
    A similar definitions exists for vector fields on the base manifold:
    \newdef{Total vector field}{\index{total!vector field}
        Given a vector field $X\in\Gamma(TM)$, we define its total vector field $\text{tot}X$ by the following conditions:
        \begin{enumerate}
            \item $X$ and $\text{tot}X$ coincide on $C^\infty(M)$, and
            \item $\text{tot}X\intmul\omega = 0$ if $\omega$ is a contact form.
        \end{enumerate}
        An explicit formula can be obtained by replacing partial derivatives by total derivatives:
        \begin{gather}
            X = X^\mu\partial_\mu\longrightarrow\text{tot}X = X^\mu D_\mu.
        \end{gather}
        In particular, the total vector fields associated to a coordinate-induced basis $\partial_\mu$ are exactly the total derivatives $D_\mu$.
    }

\subsection{Partial differential equations}\index{PDE}

    In this section we consider partial differential equations of the form
    \begin{gather}
        \label{diff:pde_jet}
        f(x,u,u_I) = 0
    \end{gather}
    for a given smooth function $f$. It is obvious that we can regard $f$ as a function on the jet bundle $J^k(\mathbb{R}^m)$ where $k$ is the order of the highest derivative and $m$ is the number of independent variables.

    \newdef{Solution}{
        In this framework one can define a solution of the above PDE as a function $u$ satisfying $f\circ j^ku=0$.
    }

    ?? COMPLETE ??

\subsection{\texorpdfstring{Pseudogroups $\clubsuit$}{Pseudogroups}}

    \begin{example}[Diffeomorphism jets]\index{Fa\'a die Bruno formula}
        Let $M$ be a smooth manifold. Consider the set $\mathcal{D}^\omega(M)$ of local analytic diffeomorphisms $\phi:M\rightarrow M:z\mapsto\phi(z)\equiv Z$. The locality property turns this set into a (smooth) pseudogroup \ref{topology:pseudogroup}.

        By the inverse function theorem \ref{manifolds:theorem:inverse_function_theorem} we can define the diffeomorphism jet bundle $\mathcal{D}^r(M)$ as the subbundle of $J^r(M, M)$ for which \[\det\left(\pderiv{Z^\alpha}{z^\beta}\right)\neq0.\] It is also possible to endow this jet bundle with the structure of a groupoid \ref{cat:groupoid}. Using the source and target projections one can check that two elements $f,g\in D^r(M)$ can be multiplied if and only if $\pi_r(g) = \pi_{r,0}(f)$. The derivative coordinates can be found using the \textit{Fa\'a di Bruno formula}.

        Furthermore, every pseudogroup $\mathcal{G}\subset\mathcal{D}^\omega$ induces a subbundle $\mathcal{G}^{(r)}\subset\mathcal{D}^{(r)}$. This structure gives rise to the following notions:
    \end{example}
    \newdef{Regular pseudogroup}{\index{pseudo!group}\index{regular!pseudogroup}\index{order}
        Consider a smooth manifold $M$. Let $\mathcal{D}^\omega(M)$ be its diffeomorphism pseudogroup and let $\mathcal{G}\subset\mathcal{D}^\omega$ be another pseudogroup. If there exists an $N\in\mathbb{N}_0$, where $N$ is called the \textbf{order}, such that for all $r\geq N$ the jets $\pi_r:\mathcal{G}^{(r)}\rightarrow M$ form an embedded submanifold of $\Pi_r:\mathcal{D}^{(r)}\rightarrow M$ and such that the jet projections $\pi_{r+1, r}:\mathcal{G}^{(r+1)}\rightarrow\mathcal{G}^{(r)}$ are fibrations then we say that $\mathcal{G}$ is a regular pseudogroup.
    }
    \newdef{Lie pseudogroup}{\index{Lie!pseudogroup}
        Let $\mathcal{G}\subset\mathcal{D}^\omega$ be a regular analytic pseudogroup of order $k$. If every local diffeomorphism $\phi\in\mathcal{D}^\omega$ satisfying $j^k\phi\in\mathcal{G}^{(k)}$ is also an element of $\mathcal{G}$ then $\mathcal{G}$ is called a Lie pseudogroup.
    }
    \begin{property}
        Let $\mathcal{G}$ be a Lie pseudogroup of order $k$. The regularity condition implies that the jet bundle $\mathcal{G}^{(r)}$ is described by a set of $r^{th}$-order PDEs \[F\left(z, Z^{(r)}\right) = 0\] for all $r\geq n$. The (local) solutions to these equations are exactly the analytic functions which have $(z_0, Z^{(r)}_0)$ as local coordinates of their $r$-jet at $z_0\in M$.

        The Lie condition on $\mathcal{G}$ implies that every solution to the system is in fact an element of $\mathcal{G}$. This system of equations is called the \textbf{determining system} of the Lie pseudogroup.
    \end{property}

    \newdef{Lie completion}{\index{Lie!completion}
        Let $\mathcal{G}$ be a regular pseudogroup. The Lie completion $\overline{\mathcal{G}}$ of $\mathcal{G}$ is defined as the set of all (local) analytic diffeomorphisms solving the determining system of $\mathcal{G}$. This completion is itself a Lie pseudogroup. If $\mathcal{H}$ is a Lie pseudogroup then $\overline{\mathcal{H}}=\mathcal{H}$.
    }

    ?? COMPLETE (IS THIS EVEN USEFUL?) ??

\section{\difficult{Variational bicomplex}}

    In this section we heavily use the language of jet bundles as introduced in the previous section. A smooth function on the infinite jet bundle will be denoted by $f[u]$, i.e. the arguments will be written inside square brackets.

    In the remainder of this section we will consider a smooth fibre bundle $\pi:E\rightarrow M$ over a base manifold of dimension $m$.\footnote{In fact we can replace the fibre bundle with any smooth fibred manifold.} We begin by decomposing the de Rham operator $\mathbf{d}$ on the infinite jet bundle $J^\infty(E)$: \[\mathbf{d} = d+\delta.\] The \textbf{horizontal derivative} $d$ lifts the de Rham differential on $M$ to $E$ (hence the name). On smooth functions $C^\infty(J^\infty(E))$ it acts as follows:
    \begin{gather}
        df := D_\mu f\mathbf{d}x^\mu.
    \end{gather}
    The horizontal de Rham operator can be extended to all of $\Omega^\bullet(J^\infty(E))$ through the Leibniz property and the condition
    \begin{gather}
        d\circ\delta = -\delta\circ d
    \end{gather}
    which follows from the nilpotency of all differentials. The differentials $d,\delta$ turn $\Omega^\bullet(J^\infty(E))$ into a bigraded complex called the variational bicomplex.
    \begin{remark}
        Some authors use the term variational bicomplex for the bicomplex of \textbf{local forms} $\Omega^\bullet_{loc}(M\times\Gamma(E))$ which is defined as the image of $\Omega^\bullet(J^\infty(E))$ under the prolongation map $M\times\Gamma(E)\rightarrow J^\infty(E)$. This way they can work with forms over the (trivial) field bundle, while maintaining the property that all objects only depend on finite-order jets. Furthermore, when working in full generality they also twist the de Rham complex over $M$ by the orientation bundle \ref{diff:orientation_bundle} (see remark \ref{diff:honest_density} for more information).
    \end{remark}

    \newdef{Local Lagrangian}{\index{Lagrangian}
        A top-degree horizontal form on $J^\infty(E)$. Because $\Omega^{m,0}(J^\infty(E))$ is one-dimensional, every such form is proportional to the volume form:
        \begin{gather}
            \mathbf{L} = L\text{Vol}
        \end{gather}
        where $L$ is a smooth function on the infinite jet bundle. By its very nature this implies that $L$ (locally) only depends on partial derivatives up to some finite order.
    }


    \newdef{Evolutionary vector field}{\index{evolutionary vector field}
        A generalized vector field on $E$ that projects to 0 on $M$. The space of evolutionary vector fields is denoted by $\text{Ev}(J^\infty(E))$.

        The prolongation of an evolutionary vector field to $J^\infty(E)$ will still project to 0 on $M$. By extension we call all vector fields on $J^\infty(E)$ that preserve the contact ideal and project to 0 on $M$ evolutionary vector fields. Such vector fields are of the form
        \begin{gather}
            X = X^\alpha_I[u]\partial^I_\alpha.
        \end{gather}

        The name stems from the fact that such vector fields define PDEs that (locally) describe the evolution of the fibres.
    }
    \begin{property}[Prolongation of evolutionary vector fields]
        The prolongation of an evolutionary vector field can be written as follows:
        \begin{gather}
            j^\infty X = \sum_{|I|=0}^\infty (D_IX^\alpha)\partial^I_\alpha.
        \end{gather}
        Furthermore, by writing out Cartan's magic formula \ref{forms:cartan_magic_formula} (with respect to $\mathbf{d}$), we can prove that the prolongation of an evolutionary vector field also satisfies this formula with respect to $\delta$ and that $\iota_{j^\infty X}$ and $d$ anticommute.
    \end{property}
    \begin{property}[Evolutionary decomposition]
        Consider a generalized vector field $X$ on $\pi:E\rightarrow M$. By extending the $\text{tot}$-construction to generalized vector fields as
        \begin{gather}
            \text{tot}X:=\text{tot}(\pi_*X),
        \end{gather}
        we can define the evolutionary part of $X$ as follows:
        \begin{gather}
            X_{ev}:=X-(\pi_{\infty,0})_*(\text{tot}X).
        \end{gather}
        Locally this can be written as
        \begin{gather}
            X_{ev} = \left(X^\alpha - X^\mu u^\alpha_{,\mu}\right)\partial_\alpha
        \end{gather}
        for \[X = X^\mu\partial_\mu + X^\alpha\partial_\alpha.\] Using this definitions we can decompose the prolongation of $X$ as follows:
        \begin{gather}
            j^\infty X = j ^\infty X_{ev} + \text{tot}X.
        \end{gather}
    \end{property}

    \begin{property}[Total differential operators]\index{differential!operator}\index{Euler!operator}
        A total differential operator is a differential operator $P:\text{Ev}(J^\infty(E))\rightarrow\Omega^\bullet(J^\infty(E))$ that can locally be written as
        \begin{gather}
            P(X) = \sum_{|I|=0}^k\left(D_IX^\alpha\right)P^I_\alpha
        \end{gather}
        where the $P^I_\alpha$ are smooth forms. By a formal ''integration-by-parts'' operation we can rewrite this (again locally) as
        \begin{gather}
            \label{var:Q_forms}
            P(X) = \sum_{|I|=0}^kD_I\left(X^\alpha Q^I_\alpha\right)
        \end{gather}
        where the smooth forms $Q^I_\alpha$ can be expressed as follows:
        \begin{gather}
            Q^I_\alpha = \sum_{|J|=0}^{k-|I|}\binom{|I|+|J|}{|J|}(-1)^{|J|}D_JP^{IJ}_\alpha.
        \end{gather}
        The zeroth-order part $Q_\alpha$ defines itself a total differential operator:
        \begin{gather}
            E_P(X) := X^\alpha Q_\alpha.
        \end{gather}
        This operator is called the \textbf{Euler operator} (associated to $P$).
    \end{property}
    \begin{property}[Decomposition of total differential operators]\label{var:differential_operator_decomposition}
        Consider a total differential operator $P:\text{Ev}(J^\infty(E))\rightarrow\Omega^{m,n}(J^\infty(E))$. On each coordinate chart $U$ we can find a total differential operator $R_U$ that satisfies the following equation:
        \begin{gather}
            P(X) = E_P(X) + dR_U(X).
        \end{gather}
        Furthermore, the Euler operator $E_P$ is the unique (globally defined) total differential operator satisfying this equation. The operators $R_U$ can locally be expressed as follows:
        \begin{gather}
            R_U(X) = \sum_{|I|=0}^{k-1}D_I\left(X^\alpha D_\mu\intmul Q^{\mu I}_\alpha\right).
        \end{gather}
    \end{property}

    \begin{example}[Euler-Lagrange operator]\index{Euler-Lagrange!operator}\index{Lie-Euler operator}\index{on-shell}
        Consider a local Lagrangian $\mathbf{L}$. This form induces a total differential operator as follows:
        \begin{gather}
            P_{\mathbf{L}}(X) := \mathcal{L}_{j^\infty X}\mathbf{L}.
        \end{gather}
        The coefficients from \ref{var:Q_forms} associated to this operator are given by the following formula\footnote{To turn these coefficients in true forms we should multiple by the volume form.}:
        \begin{gather}
            \label{var:lie_euler_operator}
            E^I_\alpha(L) := \sum_{|J|=0}^{k-|I|}\binom{|I|+|J|}{|J|}(-1)^{|J|}D_J\left(\partial^{IJ}_\alpha L\right).
        \end{gather}
        The induced Euler operator is exactly the Euler-Lagrange operator associated to variational problems. For this reason and the fact that they are induced by a Lie derivative, we call the coefficients $E^I_\alpha$ \textbf{Lie-Euler operators}.

        Given a local Lagrangian $\mathbf{L}$ we define its \textbf{Euler-Lagrange form} as
        \begin{gather}
            \delta_{EL}\mathbf{L} := E_\alpha(L)\delta u^\alpha\wedge\text{Vol}.
        \end{gather}
        An explicit formula for the Euler-Lagrange derivative is given by the following formula (this is just equation \ref{var:lie_euler_operator} for $|I|=0$):
        \begin{gather}
            \delta_{EL}L := \left(\pderiv{L}{u^\alpha} - D_\mu\pderiv{L}{u^\alpha_\mu} + \cdots\right)\delta u^\alpha.
        \end{gather}
        Functions that satisfy $\delta_{EL}L=0$ are said to be \textbf{on-shell}.
    \end{example}

    The following property generalizes the property that the Euler-Lagrange equations remain invariant under addition of a divergence to the Lagrangian:
    \begin{property}
        If a smooth function is locally an order-$k$ divergence, i.e. $f=D_IA^I$ for smooth functions $A^I$ and $|I|=l$, then the Lie-Euler operators $E^J_\alpha$ vanish on $f$ for all $|J|<l$.
    \end{property}

    \begin{example}[Interior Euler operator]\index{Euler!operator}
        For every smooth form $\omega\in\Omega^{p,q}(J^\infty(E))$ we can define a total differential operator as follows:
        \begin{gather}
            P_\omega(X) := j^\infty X\intmul\omega.
        \end{gather}
        As for the previous example this operator induces (higher) Euler operators. Since in this case they arise from interior multiplication, we call them interior Euler operators:
        \begin{gather}
            \label{var:interior_euler_operators}
            F^I_\alpha(\omega) := \sum_{|J|=0}^{k-|I|}\binom{|I|+|J|}{|J|}(-1)^{|J|}D_J\left(\partial^{IJ}_\alpha\intmul\omega\right).
        \end{gather}
        For $p=\dim(M)$ we again have a globally defined Euler operator (also called the interior Euler operator):
        \begin{gather}
            I(\omega) := \frac{1}{q}\delta u^\alpha\wedge F_\alpha(\omega).
        \end{gather}
        The interior Euler operator defines a sequence of spaces, the so-called spaces of \textbf{functional forms}, as follows:
        \begin{gather}
            \mathcal{F}^q(J^\infty(E)) := \left\{\omega\in\Omega^{m,q}(J^\infty(E)):I(\omega)=\omega\right\}.
        \end{gather}
    \end{example}
    \begin{property}
        The interior Euler operator has the following important properties:
        \begin{itemize}
            \item It is a projector $I^2=I$.
            \item It vanishes on (locally) $d$-exact forms.
            \item $\delta_V := I\circ\delta$ gives $\mathcal{F}^\bullet(J^\infty(E))$ the structure of a cochain complex.
            \item $\delta_{EL}\mathbf{L} = \delta_V\mathbf{L}$ for all local Lagrangians $\mathbf{L}$.
        \end{itemize}
    \end{property}
    \begin{property}[Functional decomposition]\label{var:functional_decomposition}
        The de Rham spaces on $J^\infty(E)$ admit the following decomposition:
        \begin{gather}
            \Omega^{p,q}(J^\infty(E)) = \mathcal{F}^q(J^\infty(E))\oplus d\Omega^{p-1,q}(J^\infty(E))
        \end{gather}
        where the functional part is obtained by applying $I$ to a form.
    \end{property}

    The forms in $\mathcal{F}^q$ are said to be functional due to the following property:
    \begin{property}[Functionals]\index{degree!of functional}
        To every smooth $k$-form $\omega\in\Omega^\bullet(J^\infty(E))$, compact subset $V\subset E$ and $k-m$ generalized vector fields on $E$ we can assign a functional on $\Gamma(U)$, where $U$ is a chart containing $V$, by the following formula:
        \begin{gather}
            \label{var:functionals}
            W_\omega(X_1,\ldots,X_{k-m})[\sigma] := \int_V(j^\infty\sigma)^*\omega(j^\infty X_1,\ldots,j^\infty X_{k-m}).
        \end{gather}
        The number $k-m$ is called the \textbf{degree} of $W_\omega$. In general these functionals are invariant under the addition of a $d$-exact form, however, the assignment $\omega\mapsto W_\omega$ becomes a bijection (for a fixed subset $V$ and generalized vector fields $X_i$).
    \end{property}
    \begin{construct}
        The differential $\delta_V$ on $\mathcal{F}^\bullet$ induces a differential on the space of functionals of the above form:
        \begin{gather}
            \delta W_\omega := W_{\delta_V\omega}.
        \end{gather}
        An equivalent definition can be given by a formula similar to \ref{forms:k_form_exterior_derivative} where an evolutionary vector field $X$ acts on a degree-$0$ functional by Lie derivation:
        \begin{gather}
            X(W_\omega[\sigma]) := \int_V(j^\infty\sigma)^*(\mathcal{L}_{j^\infty X}\omega).
        \end{gather}
    \end{construct}

    The forms in $\mathcal{F}^1$ admit the following characterization:
    \newdef{Source form}{\index{source}
        A differential form $\omega\in\Omega^{m,1}(J^\infty(E))$ such that the evaluation on a vector field only depends on the projection $(\pi_{\infty,0})_*X\in TE$ where $\pi_{\infty,0}:J^\infty(E)\rightarrow E$ is the jet bundle target projection. After pulling back along the prolongation map $j^\infty$ this can be written as follows:
        \begin{gather}
            \Omega^{m,1}_{\text{source}}(E) := \delta C^\infty(E)\wedge\Omega^{m,0}(E),
        \end{gather}
        i.e. locally it can be written as
        \begin{gather}
            \omega = \omega_\alpha(x,u,u_I)\delta u^\alpha\wedge\text{Vol}.
        \end{gather}
        The PDEs associated to the source form $\omega$ are called \textbf{source equations}.
    }
    \begin{remark*}
        In fact one can extend the above definition to all of $\Omega^{\bullet,1}(J^\infty(E))$. So in general $\mathcal{F}^1$ is only a subspace of the space of source forms.
    \end{remark*}

    The degree-$2$ functional forms also admit a local characterization:
    \begin{property}
        Consider a collection of smooth functions $A^I_{\alpha\beta}\in C^\infty(J^\infty(U))$ on some chart $U\subset E$ such that
        \begin{gather}
            A^I_{\alpha\beta} = (-1)^{|I|+1}A^I_{\beta\alpha}.
        \end{gather}
        The (local) $(m,2)$-form
        \begin{gather}
            w^{|I|} := \delta u^\alpha\wedge\left(A^I_{\alpha\beta}\delta u^\beta_I + D_I(A^I_{\beta\alpha}\delta u^\beta)\right)\wedge\text{Vol}
        \end{gather}
        is an element of $\mathcal{F}^2(J^\infty(U))$. Furthermore, every degree-$2$ functional form can locally be expressed as a sum of the form
        \begin{gather}
            \omega = w^0 + w^1 + w^2 + \cdots.
        \end{gather}
    \end{property}

    Although local characterizations for functional forms $\omega\in\mathcal{F}^q(J^\infty(E))$ with $q\geq2$ are still not well-understood, there exists a more high-level characterization:
    \begin{property}[Characterization of functional forms]
        An $(m,q)$-form $\omega$ is functional if and only if there exists a linear \textit{formally skew-adjoint}\footnotemark\ differential operator $P:\text{Ev}(J^\infty(E))\rightarrow\Omega^{m,q-1}(J^\infty(E))$ such that
        \begin{gather}
            \omega = \delta u^\alpha\wedge P_\alpha.
        \end{gather}
        This representation is unique if it exists.
        \footnotetext{This is a generalization of \ref{var:formal_adjoint}: $P$ is said to be formally skew-adjoint if there exists an $(m-1,q-1)$-form $K$ such that
        \begin{gather}
            j^\infty X\intmul P(Y) + j^\infty Y\intmul P(X) = dK
        \end{gather}
        for all $X,Y\in\text{Ev}(J^\infty(E))$.}
    \end{property}

    ?? COMPLETE ALL BELOW ??
    \begin{property}[First variational formula]
        Given a local Lagrangian $\mathbf{L}$ we can look at solutions of the associated variational principle. This implies that we look at perturbations of a field configuration $[u_0]$. Such perturbations are generated by evolutionary vector fields and hence we can write the extremality condition as
        \begin{gather}
            \mathcal{L}_{j^\infty X}\mathbf{L} = 0\qquad\qquad\forall X\in\text{Ev}(J^\infty(E)).
        \end{gather}
        Because of property \ref{var:differential_operator_decomposition} we can decompose the above operator as follows:
        \begin{gather}
            \mathcal{L}_{j^\infty X}\mathbf{L} = j^\infty X\intmul\delta_{EL}\mathbf{L} + j^\infty X\intmul d\gamma.
        \end{gather}
        Using Cartan's magic formula, the fact that $\mathbf{L}$ is horizontal and that the above formula holds for all evolutionary vector fields, we arrive at the following decomposition (often called the first variational formula):
        \begin{gather}
            \delta\mathbf{L} = \delta_{EL}\mathbf{L} + d\gamma.
        \end{gather}
        (Such decompositions follow more generally from property \ref{var:functional_decomposition}.)
    \end{property}

    \newdef{Lepage form}{\index{Lepage equivalent}
        An $m$-form $\rho\in\Omega^\bullet(J^\infty(E))$ such that $\pi^{m,0}(X\intmul\mathbf{d}\rho)=0$ for all $\pi_{\infty,0}$-vertical vector fields $X$. Given a local Lagrangian $\mathbf{L}\in\Omega^{m,0}(J^\infty(E))$, we say that $\mathbf{L}$ and $\rho$ are Lepage equivalent if $\pi^{m,0}\rho = \mathbf{L}$.
    }
    \begin{property}[Lepage equivalent]
        Consider the first variational formula for a local Lagrangian $\mathbf{L}$. A Lepage equivalent is given by the form $\mathbf{L}+\gamma$.
    \end{property}

\subsection{Inverse problem}

    The inverse problem in the calculus of variations asks when a given system of PDEs can be obtained from a variational problem, i.e. when a source form $\Delta$ can be written in the form $\delta_{EL}\mathbf{L}$.

    \textit{Helmholtz} was the first to study the inverse problem, and hence the following sufficient conditions are named after him:
    \begin{property}[Helmholtz conditions]\index{Helmholtz!conditions}\index{variational!form}
        A source form $\Delta$ can be obtained from a local Lagrangian if
        \begin{gather}
            \mathcal{L}_{j^\infty X}\Delta = \delta_{EL}(X\intmul\Delta)
        \end{gather}
        for all evolutionary vector fields $X$. This can also be rewritten in terms of the variational differential $\delta_V$:
        \begin{gather}
            \delta_V\Delta = 0.
        \end{gather}
        Source forms satisfying this condition are said to be \textbf{locally variational}.
    \end{property}
    \begin{formula}[Local expression]
        Consider a source form that admits the local expression \[\Delta=P_\alpha[u]\delta u^\alpha\wedge\text{Vol}.\] The Helmholtz conditions can locally be expressed as follows:
        \begin{gather}
            (-1)^{|I|}\partial^I_\alpha P_\beta = E^I_\beta(P_\alpha)
        \end{gather}
        where $E^I_\beta$ are the Lie-Euler operators \ref{var:lie_euler_operator}.
    \end{formula}

    \begin{example}[Volterra-Vainberg formula]\index{Volterra-Vainberg formula}
        If $E$ is trivial and $\Delta\equiv F_\alpha\delta u^\alpha\wedge\text{Vol}$ satisfies the Helmholtz conditions, then
        \begin{gather}
            \label{var:trivial_helmholtz}
            \mathbf{L} = \int_0^1 u^\alpha F_\alpha[tu]dt
        \end{gather}
        satisfies $\Delta=\delta_{EL}\mathbf{L}$. If $\Delta$ is homogeneous of degree $k$ in $u$, then this can be expressed as
        \begin{gather}
            \mathbf{L} = \frac{1}{k+1}\iota_R\Delta
        \end{gather}
        with $R:=u^\alpha\partial_\alpha$ the (vertically) radial vector field on $E$.
    \end{example}

\subsection{Structure of the bicomplex}

    Morphisms of infinite jet bundles should factorize through finite jet bundles, so the pullback of a form $\omega\in\Omega^p(J^\infty(E'))$ of order $k$ along a morphism $\Phi:J^\infty(E)\rightarrow J^\infty(E')$ will be some form $\phi^*\omega\in\Omega^p(J^\infty(E))$ of degree $k'>k$. A well-defined morphism of variational bicomplexes should preserve the bigrading of forms, but this will clearly not be the case in general. Therefore we introduce a ''projected pullback'':
    \begin{gather}
        \Phi^\sharp:\Omega^{p,q}(J^\infty(E'))\rightarrow\Omega^{p,q}(J^\infty(E)):\omega\mapsto\pi^{p,q}(\Phi^*\omega).
    \end{gather}
    Here the projection $\pi^{p,q}$ is the projection of the variational bicomplex, not the one of jet bundles (which is denoted by subscripts).
    \begin{remark}\label{var:degree_raise_remark}
        The projection $\pi^{p,q}:\Omega^{p+q}(J^\infty(E))\rightarrow\Omega^{p,q}(J^\infty(E))$ is defined by substituting $\mathbf{d}u^\alpha_I\longrightarrow\delta u^\alpha_I + u^\alpha_{\mu I}dx^\mu$ and then projecting onto the correct horizontal and vertical degrees. Note that this does not preserve the order of forms due to the presence of the factor $u^\alpha_{\mu I}$.
    \end{remark}

    The main argument for introducing the projected pullback is that functionals of the form \ref{var:functionals} only care about $(m,q)$-forms for a specific $q$ (in particular \textbf{action functionals}, i.e. $q=0$).\index{action}

    \begin{formula}[Local Lagrangians]
        For local Lagrangians $\mathbf{L}\equiv L\text{Vol}_{E'}$ the projected pullback acts as follows:
        \begin{gather}
            \Phi^\sharp\mathbf{L} = (L\circ\Phi)\det(D_\mu f^\mu)\text{Vol}_E
        \end{gather}
        where $\pi'(\Phi[u])=(f^\mu)$. So we obtain the usual formula for pullbacks of top-dimensional forms, but with partial derivatives replaced by total derivatives.
    \end{formula}

    An important property of the de Rham differential is its naturality (see property \ref{forms:exterior_derivative_properties}). The following property states the ''naturality'' of the different operators on the variational bicomplex with respect to the projected pullback:
    \begin{property}
        Consider a morphism of infinite jet bundles $\Phi:J^\infty(E)\rightarrow J^\infty(E')$.
        \begin{itemize}
            \item $\Phi^\sharp$ and $\delta$ commute if and only if $\Phi$ covers a morphism of base manifolds.
            \item $\Phi^\sharp$ and $d$ commute if and only if $\Phi^*$ preserves the contact ideal.
            \item $\Phi^\sharp$ commutes with both differentials if and only if it coincides with $\Phi^*$
        \end{itemize}
        Furthermore, the projected pullback defines a contravariant functor on the subcategory on morphisms that satisfy at least one of the above properties.
    \end{property}

    The following property gives an infinitesimal analogue of the above considerations:
    \begin{property}
        Consider a vector field $X\in\mathfrak{X}(J^\infty(E))$. Its Lie derivative will in general not respect the bigrading of the variational bicomplex (unless $X$ is evolutionary on $E$) and therefore we introduce the ''projected Lie derivative'':
        \begin{gather}
            \mathcal{L}^\sharp_X:\Omega^{p,q}(J^\infty(E))\rightarrow\Omega^{p,q}(J^\infty(E)):\omega\mapsto\pi^{p,q}(\mathcal{L}_X\omega).
        \end{gather}
        This operator satisfies the following properties:
        \begin{itemize}
            \item It commutes with $\delta$ if and only if $X$ is $\pi_\infty$-related to a vector field on $M$.
            \item It commutes with $d$ if and only if $X$ is the prolongation of a generalized vector field on $E$.
        \end{itemize}
        Note that this does not simply follow from the previous property since $X$ does not necessarily define as flow on $J^\infty(E)$.
    \end{property}

    Aside from the differentials on the variational bicomplex, we should also look at the structure of the ''functional complex'' $(\mathcal{F}^\bullet,\delta_V)$. It can be shown that requiring both $[I,\Phi^\sharp]=0$ and $[\delta,\Phi^\sharp]=0$ is very restricting. Furthermore, a complete characterization of those morphisms $\Phi$ that satisfy only $[I,\Phi^\sharp]=0$ is not fully understood. However, the infinitesimal version is easier to handle since it only contains linear equations:
    \begin{property}
        Let $n$ be the rank of $E$ and consider a vector field $X$ on $J^\infty(E)$.
        \begin{itemize}
            \item If $n=1$, then $\mathcal{L}^\sharp_X$ commutes with $I$ if and only if $X$ is locally the prolongation of a generalized vector field on $E$ of the form
            \begin{gather}
                Y = -\pderiv{S}{u_\mu}\partial_\mu + \left(S - u_\mu\pderiv{S}{u_\mu}\right)\partial_u
            \end{gather}
            where $S$ is a function the first jet bundle $J^1(U)$.
            \item If $n>1$, then $\mathcal{L}^\sharp_X$ commutes with $I$ if and only if $X$ is the prolongation of vector field on $E$.
        \end{itemize}
    \end{property}

    The next step is to define operators that do preserve the functional complex. For this we use the projection property of $I$:
    \begin{align}
        &\Phi^\natural:\Omega^{m,q}(J^\infty(E))\rightarrow\mathcal{F}^q(J^\infty(E')):\omega\mapsto (I\circ\Phi^\sharp)\omega\\
        &\mathcal{L}^\natural_X:\Omega^{m,q}(J^\infty(E))\rightarrow\mathcal{F}^q(J^\infty(E)):\omega\mapsto (I\circ\mathcal{L}^\sharp_X)\omega.
    \end{align}
    \begin{property}
        The above operators satisfy the following properties:
        \begin{itemize}
            \item If $\Phi$ is a contact transformation, then $\Phi^\natural$ commutes with both $I$ and $\delta_V$.
            \item If $X$ is a generalized vector field on $E$, then $\mathcal{L}^\natural_{j^\infty X}$ commutes with both $I$ and $\delta_V$.
            \item $\Phi^\natural$ preserves locally variational forms.
        \end{itemize}
    \end{property}
    The projected and functionally projected operators also satisfy the following relations:
    \begin{property}[Euler-Lagrange operator]
        Let $\mathbf{L}\in\Omega^{m,0}(J^\infty(E'))$ be a local Lagrangian. If $\Phi$ is a contact transformation, then
        \begin{gather}
            \delta_{EL}(\Phi^\sharp\mathbf{L}) = \Phi^\natural(\delta_{EL}\mathbf{L}).
        \end{gather}
        If $X$ is a generalized vector field on $E'$, then
        \begin{gather}
            \delta_{EL}(\mathcal{L}^\sharp_{j^\infty X}\mathbf{L}) = \mathcal{L}^\natural_{j^\infty X}(\delta_{EL}\mathbf{L}).
        \end{gather}
    \end{property}

    \begin{formula}[Functionally projected Lie derivative]
        Let $X$ be a generalized vector field on $E$ and let $\omega\in\mathcal{F}^\bullet$ be a functional form.
        \begin{gather}
            \mathcal{L}^\natural_{j^\infty X}\omega = \delta_V(j^\infty X_{ev}\intmul\omega) + I(j^\infty X_{ev}\intmul\delta_V\omega).
        \end{gather}
        If $X$ is an evolutionary vector field, $X=X_{ev}$  and hence we can replace $X_{ev}$ by $X$ on the right-hand side. For projectable vector fields we can replace the left-hand side by the ordinary Lie derivative $\mathcal{L}_{j^\infty X}\omega$.
    \end{formula}

    \newdef{Distinguished symmetry}{\index{symmetry}
        A distinguished (generalized) symmetry of a source form $\Delta\in\mathcal{F}^1(J^\infty(E))$ is a (generalized) vector field $X$ on $E$ such that
        \begin{gather}
            \mathcal{L}^\natural_{j^\infty X}\Delta = 0.
        \end{gather}
        As for the formula above, we can replace the projected Lie derivative by an ordinary Lie derivative when $X$ is projectable.
    }
    \newdef{Bessel-Hagen symmetry}{\index{Bessel-Hagen|see{symmetry}}\index{divergence!symmetry}
        A Bessel-Hagen or \textbf{divergence} symmetry of a Euler-Lagrange form $\Delta\equiv\delta_{EL}(\mathbf{L})$ is a generalized vector field $X$ such that
        \begin{gather}
            \mathcal{L}^\natural_{j^\infty X}\mathbf{L} = d\eta
        \end{gather}
        for some $(m-1,1)$-form $\eta$.

        If this condition holds locally, then distinguished symmetries and Bessel-Hagen symmetries coincide. However, if the Bessel-Hagen condition is required to hold globally, then the Bessel-Hagen symmetries form only a subset of the distinguished symmetries.
    }

    \newdef{Local conservation law}{\index{conservation law}
        A generator of a local conservation law of a source form $\Delta\in\mathcal{F}^1(J^\infty(E))$ is an evolutionary vector field $X$ such that
        \begin{gather}
            \delta_{EL}(j^\infty X\intmul\Delta).
        \end{gather}
    }

    \begin{property}[Lie algebra of symmetries]
        Given a source form $\Delta\in\mathcal{F}^1(J^\infty(E))$, we can equip the vector space of generalized vector fields satisfying the following two conditions with the structure of a Lie algebra:
        \begin{itemize}
            \item They are distinguished symmetries.
            \item Their evolutionary part is a generator of local conservation laws.
        \end{itemize}
    \end{property}
    \begin{theorem}[Noether]\index{Noether}
        If $\Delta\in\mathcal{F}^1(J^\infty(E))$ is locally variational, then a generalized vector field on $E$ is a distinguished symmetry if and only if its evolutionary part is a generator of local conservation laws.
    \end{theorem}

    In the remainder of this section we will study the homological properties of the variational bicomplex over a local chart (or equivalently, over a trivial bundle). To prove the acyclicity of the various subcomplexes we will follow the usual approach of finding a null-homotopy (see property \ref{homalg:null_acyclic}), i.e. we look for a cochain map $h:C_\bullet\rightarrow C_\bullet$ such that
    \begin{gather}
        \text{id} = d\circ h + h\circ d.
    \end{gather}

    \begin{property}[Vertical complex is exact]
        Homotopy operators $h^{p,q}_V:\Omega^{p,q}\rightarrow\Omega^{p,q-1}$ for the vertical complex
        \begin{gather}
            0\longrightarrow\Omega^p(M)\xrightarrow{\,\pi_\infty^*\,}\Omega^{p,0}\xrightarrow{\ \delta\ }\Omega^{p,1}\xrightarrow{\ \delta\ }\cdots
        \end{gather}
        are given by the following formula
        \begin{gather}
            h^{p,q}_V(\omega) = \int_0^1\frac{1}{t}\Phi^*_{\log t}(j^\infty R\intmul\omega)dt
        \end{gather}
        where $R:=u^\alpha\partial_\alpha$ is the (vertically) radial vector field and $\Phi_\varepsilon:[x,u]\mapsto[x,e^\varepsilon u]$. It is not too difficult to check that for source forms this gives rise to formula \ref{var:trivial_helmholtz}.
    \end{property}
    The analogous statement for the horizontal complex is a bit more involved:
    \begin{property}[Augmented horizontal complex is exact]
        Homotopy operators $h^{p,q}_H:\Omega^{p,q}\rightarrow\Omega^{p-1,q}$ for the augmented horizontal complex
        \begin{gather}
            0\longrightarrow\Omega^{0,q}\xrightarrow{\ d\ }\Omega^{1,q}\longrightarrow\cdots\longrightarrow\Omega^{m,q}\xrightarrow{\ I\ }\mathcal{F}^q\xrightarrow{\ I\ }0
        \end{gather}
        are given by the following formula
        \begin{gather}
            h^{p,q}_H(\omega) = \frac{1}{q}\sum_{|I|=0}^{k-1}\frac{|I|+1}{m-p+|I|+1}D_I\left(\delta u^\alpha\wedge F^{I\mu}_\alpha(D_\mu\intmul\omega)\right)
        \end{gather}
        where the $F^I_\alpha$ are the interior Euler operators \ref{var:interior_euler_operators}.
    \end{property}

    Using the above properties we can prove the acyclicity of the \textbf{Euler-Lagrange complex} $\mathcal{E}$ (again over a local chart), i.e. the following sequence is exact:
    \begin{gather}
        0\longrightarrow\mathbb{R}\longrightarrow\Omega^{0,0}\xrightarrow{\ d\ }\Omega^{1,0}\longrightarrow\cdots\longrightarrow\Omega^{m,0}\xrightarrow{\delta_{EL}}\mathcal{F}^1\xrightarrow{\,\delta_V\,}\mathcal{F}^2\xrightarrow{\,\delta_V\,}\cdots.
    \end{gather}
    (We won't give explicit formulas since these are too complicated for what we want to achieve. See \cite{var_bicomplex} for a full account.)

    \begin{remark}
        Although we have stated (local) exactness of the variational bicomplex, it should be noted that we have not found an optimal solution. Consider the example of locally variational source forms. From the form of the homotopy operator $\mathcal{F}^1\rightarrow\Omega^{m,0}$ it should be clear that an order-$k$ source form is mapped to an order-$k$ Lagrangian. However, the Euler-Lagrange operator $\delta_{EL}$ will in general map order-$l$ Lagrangians to order-$2l$ source forms. Hence, we see that the homotopy operator will in general not result in a Lagrangians of minimal order.
    \end{remark}

    A last subject that we will consider in this section is the restriction of the variational bicomplex to finite jet bundles. However, as is clear from the definition of the horizontal differential, that forms of order $k$ are mapped to forms of order $k+1$. Therefore we will have to restrict our attention to a specific subcomplex of $\Omega^\bullet(J^\infty(E))$:
    \begin{gather}
        \Omega^\bullet_k(E)\subset\Omega^\bullet(J^{k+1}(E)) := \delta\text{-closure of }\Omega^\bullet(J^k(E)).
    \end{gather}
    From the basic definitions and properties of the two differentials $d,\delta$ it follows that $\Omega^\bullet_k$ is generated by $C^\infty(J^k(E))$, the horizontal basis $\{dx^\mu\}_{\mu\leq\dim(M)}$ and the contact basis $\{\delta u^\alpha_I\}_{\mu\leq\dim(M)}^{|I|\leq k}$.
    The next step is to further restrict to a horizontally closed subcomplex. Consider forms $\omega\in\Omega^{p,q}_k(E)$ of the form
    \begin{gather}
        \omega = \Big[du^{\alpha_1}_{I_1}\wedge\ldots\wedge du^{\alpha_r}_{I_r}\wedge d\delta u^{\beta_1}_{J_1}\wedge\ldots\wedge d\delta u^{\beta_s}_{J_s}\Big]\wedge fdx^{\kappa_1}\wedge\ldots\wedge dx^{\kappa_{p-r-s}}
    \end{gather}
    where $|I_i|,|J_i|=k-1$ and $f\in C^\infty(J^{k-1}(E))$. It is immediately clear that the subcomplex of such forms is also horizontally closed. The factor in between square brackets can also be rewritten as follows:
    \begin{gather}
        J = \frac{1}{(r+s)!}\frac{D(u^{\alpha_1}_{I_1},\ldots,u^{\alpha_r}_{I_r},\delta u^{\beta_1}_{J_1},\ldots,\delta u^{\beta_s}_{J_s})}{D(x^{\mu_1},\ldots,x^{\mu_r},x^{\nu_1},\ldots,x^{\nu_s})} dx^{\mu_1}\wedge\ldots\wedge dx^{\mu_r}\wedge dx^{\nu_1}\wedge\ldots\wedge dx^{\nu_s}.
    \end{gather}
    This factor has the form of a Jacobian determinant (with respect to total derivatives) and as such we call the subcomplex spanned by the above forms the \textbf{Jacobian (sub)complex} $\mathcal{J}^{\bullet,\bullet}_k(E)$.

    \begin{property}[Alternative characterizations]
        The Jacobian complexes can also be characterized as follows:
        \begin{itemize}
            \item Consider the projection $\pi^{\bullet,0}:\Omega^\bullet\rightarrow\Omega^{\bullet,0}$ (note that this maps forms in $\Omega^r(J^k(E))$ to forms in $\Omega^{r,0}_{k+1}(E)$ due to remark \ref{var:degree_raise_remark}).
            \begin{gather}
                \mathcal{J}^{p,q}_k(E) = \Omega^{p,q}(J^\infty(E))\cap\delta\text{-closure of }\pi^{\bullet,0}\left[\Omega^\bullet(J^{k-1}(E))\right].
            \end{gather}
            \item For $p<m$ the Jacobian complex contains those forms for which $d$ does not increase the order:
            \begin{gather}
                \mathcal{J}^{p,q}_k(E) = \{\omega\in\Omega^{p,q}_k:d\omega\in\Omega^{p+1,q}_k(E)\}.
            \end{gather}
            \item If $\omega\in\mathcal{J}^{p,q}_k(E)$, then $\omega$ is a polynomial in $u^\alpha_i$'s and $\delta u^\alpha_I$'s of degree $\leq r$ with $|I|=k$.
        \end{itemize}
    \end{property}

    It can be shown that the Jacobian subcomplex is (locally) exact and that it respects the structure of the Euler-Lagrange complex:
    \begin{property}[Exactness]
        Let $E$ be trivial. If $\delta_{EL}\omega=0$ for $\omega\in\Omega^{m,q=0}_k(E)$ or $I(\omega)=0$ for $\omega\in\Omega^{m,q\geq1}_k(E)$, then $\omega\in\mathcal{J}^{m,q}_k(E)$ and $\omega=d\eta$ for $\eta\in\mathcal{J}^{m-1,q}_k(E)$.
    \end{property}

    \begin{property}[Functional dependence of Lagrangians]
        If $\Delta$ is a locally variational source form of order $k$, then it is polynomial of degree $m$ in $k^{th}$-order derivatives of the $u^\alpha$.
    \end{property}
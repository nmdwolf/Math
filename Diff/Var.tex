\chapter{Calculus of variations}

\section{Constrained systems}
\subsection{Holonomic constraints}

	\newdef{Holonomic constraint}{
		A constraint $f(q, t) = 0$ is called holonomic if it only depends on the coordinates $q^i$ and $t$.
	}
	\begin{method}[Holonomic constraints]
		The Euler-Lagrange equations of a system with $k$ holonomic constraints $f_k(q, t) = 0$ can be obtained from the generalized action functional
		\begin{equation}
			\int_a^b\Big[L(q(t), \dot{q}(t), t) + \sum_{j=1}^k\lambda_j(t)f_j(q(t), t)\Big]dt
		\end{equation}
		where $\lambda_j(t)$ are undetermined (Lagrange) multipliers.
	\end{method}

\section{Noether symmetries}
\subsection{Classical systems}

	\newdef{Noether symmetry}{
		Let $L(q^i, \dot{q}^i, t)$ be the Lagrangian function describing some system $\mathcal{S}$. An infinitesimal transformation\footnote{The transformation for $\dot{q}^i$ is in fact induced by the ones for $q^i$ and $t$.}
		\begin{align*}
			q^i \longrightarrow q^i + \varepsilon\xi^i(q^k, t)\\
			t\longrightarrow t + \varepsilon\tau(q^k, t)\\\\
			\dot{q}^i\longrightarrow \dot{q}^i + \varepsilon(\dot{\xi}^i - \dot{q}^i\dot{\tau})
		\end{align*}
		is called a Noether symmetry for $L$ if it satisfies:
		\begin{equation}
			\int_{\tilde{t}_0}^{\tilde{t}_1}\tilde{L}(\tilde{q}^i, \dot{\tilde{q}}^i, \tilde{t})d\tilde{t} = \int_{t_0}^{t_1}L(q^i, \dot{q}^i, t)dt + \varepsilon\int_{t_0}^{t_1}\deriv{f}{t}dt
		\end{equation}
		for every curve $q:[a, b]\rightarrow\mathbb{R}$, for every subinterval $[t_0, t_1]\subseteq[a, b]$ and for some function $f(q^i, t)$.
	}

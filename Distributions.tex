\chapter{Distributions}
\section{Generalized function}

	\newdef{Schwartz space}{\index{Schwartz!space of rapidly decreasing functions}
	    	The Schwartz space or \textbf{space of rapidly decreasing functions}\footnotemark\ $S(\mathbb{R})$ is defined as:
	    	\footnotetext{These functions are said to be rapidly decreasing because every derivative $f^{(j)}(x)$ decays faster than any polynomial $x^i$ for $x\rightarrow+\infty$.}
		\begin{equation}
			\label{distribution:schwartz_space}
			S(\mathbb{R}) = \left\{f(x)\in C^\infty(\mathbb{R}):\forall i,j\in\mathbb{N}:\forall x\in\mathbb{R}:|x^if^{(j)}(x)|<+\infty\right\}
		\end{equation}
	}
    
    	\remark{This definition can be generalized to functions in $C^\infty(\mathbb{R}^n)$ or functions $f:\mathbb{R}^n\rightarrow\mathbb{C}$. The Schwartz space is then denoted by $S(\mathbb{R}^n,\mathbb{C})$.}
    
	\newdef{Functions of slow growth}{
	    	The set of functions of slow growth $N(\mathbb{R})$ is defined as:
	        \begin{equation}
			N(\mathbb{R}) = \{f(x)\in C^\infty(\mathbb{R}) : \forall i\in\mathbb{N},\exists M_i > 0: |f^{(i)}(x)| = O(|x|^i) \text{ for } |x|\rightarrow+\infty\}
		\end{equation}
	}
	\sremark{It is clear that all polynomials belong to $N(\mathbb{R})$ but not to $S(\mathbb{R})$.}
    
	\begin{property}
		If $f(x)\in S(\mathbb{R})$ and $a(x)\in N(\mathbb{R})$ then $a(x)f(x)\in S(\mathbb{R})$.
	\end{property}
    
	\newdef{Generalized function}{\index{generalized!function}
    		Let $g(x)\in S(\mathbb{R})$ be a test function. Let $\{f_n(x)\in S(\mathbb{R})\}, \{h_n(x)\in S(\mathbb{R})\}$ be sequences such that
        	\[
        		\lim_{n\rightarrow+\infty}\langle f_n(x)|g(x) \rangle = \lim_{n\rightarrow+\infty}\int_{-\infty}^{+\infty}f_n(x)g(x)dx
        	\]
        	and similarly for $h_n$.
        	Define the equivalence relation $\{f_n(x)\in S(\mathbb{R})\} \sim \{h_n(x)\in S(\mathbb{R})\}$ by saying that the two sequences, satisfying the previous condition, are equivalent if and only if
        	\[
        		\lim_{n\rightarrow+\infty}\langle f_n(x)|g(x) \rangle = \lim_{n\rightarrow+\infty}\langle h_n(x)|g(x) \rangle
		\]
        	A generalized function is defined as a complete equivalence class under previous relation.
	}
    
	\begin{notation}
		Let $\psi$ be a generalized function. Let $f\in S(\mathbb{R})$. The inner product \ref{lebesgue:L2_inner_product} is generalized by following functional:
	        \begin{equation}
			\langle \psi|f \rangle = \lim_{n\rightarrow+\infty}\int_{-\infty}^{+\infty}\psi_n(x)f(x)dx
		\end{equation}
	\end{notation}
    
	\begin{property}
		Let $\psi$ be a generalized function. Let $f(x)\in S(\mathbb{R})$. The previous functional has following properties:
   		\begin{itemize}
			\item \label{distribution:gf_prop1}$\forall i\in\mathbb{N}: \langle \psi^{(i)}|f \rangle = (-1)^i \langle \psi|f^{(i)} \rangle$
		        \item $\forall a,b\in\mathbb{R}, a\neq0: \langle \psi(ax+b)|f(x) \rangle = |a|^{-1}\langle \psi(x)|f(x - b/a) \rangle$
		        \item $\forall a(x)\in N(\mathbb{R}):\langle a\psi|f \rangle = \langle \psi|af \rangle$
		\end{itemize}
	\end{property}
    
	\newprop{Ordinary function as generalized function}{
	    	Let $f:\mathbb{R}\rightarrow\mathbb{C}$ be a function such that $\exists M\geq0:(1+x^2)^{-M}|f(x)| \in L(\mathbb{R},\mathbb{C})\footnotemark$. There exists a generalized function $\psi\sim\{f_n(x) \in S(\mathbb{R},\mathbb{C})\}$ such that for every $g(x)\in S(\mathbb{R},\mathbb{C})$: \[\langle \psi|g \rangle = \langle f|g \rangle\] Furthermore if $f(x)$ is continuous on an interval, then $\ds\lim_{n\rightarrow+\infty} f_n(x) = f(x)$ converges pointwise on that interval.
	    	\footnotetext{The space of Lebesgue integrable functions \ref{lebesgue:L1}.}
	}

\section{Dirac Delta distribution}

	\newdef{Heaviside function}{\index{Heaviside!function}
	    	Define the generalized function $H\sim\{H_n(x)\in S(\mathbb{R})\}$ as:
        	\begin{equation}
        		\label{distribution:heaviside_function}
			H(x) = \left\{
        		\begin{array}{ccc}
				0&if&x<0\\
		                1&if&x\geq0
			\end{array}\right.
		\end{equation}
	        From this definition it follows that for every $f\in S(\mathbb{R})$:
        	\begin{equation}
        		\label{distribution:heaviside_function_integral}
			\langle H|f \rangle = \int_0^{+\infty}f(x)dx
		\end{equation}
	}
	\remark{For the above integral to exist, $f(x)$ does not need to be an element of $S(\mathbb{R})$. It is a sufficient condition, but not a necessary one.}
    
	\newdef{Generalized delta function}{\index{Dirac!delta function}\label{distribution:dirac_delta}
	    	The Dirac delta function is defined as a representant of the equivalence class of generalized functions $\{H_n'(x)\in S(\mathbb{R})\}$. By equations \ref{distribution:gf_prop1} and \ref{distribution:heaviside_function_integral} we have for every $f\in S(\mathbb{R})$:
	        \begin{align*}
	            	\langle \delta|f \rangle&=\langle H'|f \rangle\\
        		&=-\langle H|h \rangle\\
	                &=-\ds\int_0^{+\infty}f'(x)dx\\
        		&=f(0)
		\end{align*}
	}
    
	\begin{property}[Sampling property]
	    	The result from previous definition can be generalized in the following way:
	        \begin{equation}
			\label{distribution:sieving_dirac_delta}
		        \boxed{f(x_0) = \int_\mathbb{R}f(x)\delta(x - x_0)dx}
		\end{equation}
	\end{property}
	\begin{example}[Dirac comb]\index{Dirac!comb}
	    	\begin{equation}
			\label{distribution:dirac_comb}
		        \uppercase\expandafter{\romannumeral 3}_b(x) = \sum_n\delta(x-nb)
		\end{equation}
	\end{example}
    
	\begin{property}
		Let $f(x)\in C^1(\mathbb{R})$ be a function with roots at $x_1,x_2,...,x_n$ such that $f'(x_i)\neq0$. The Dirac delta distribution has the following property:
		\begin{equation}
			\label{distribution:delta_of_function}
			\delta[f(x)] = \sum_{i=1}^n\stylefrac{1}{|f'(x_i)|}\delta(x-x_i)
		\end{equation}
	\end{property}
    
	\newprop{Convolution with delta function}{\index{convolution}
	    	Let $f(x)\in S(\mathbb{R})$. Let $\otimes$ denote the convolution.
	        \begin{equation}
			\delta(x)\otimes f(x) = \int_{-\infty}^{+\infty}\delta(x-\alpha)f(\alpha)d\alpha = f(x)
		\end{equation}
	}
    
	\newformula{Differentiation across discontinuities}{
	    	Let $f(x)$ be a piecewise continuous function with discontinuities at $x_1,...,x_n$. Let $f$ satisfy the conditions to be a generalized function. Define $\sigma_i = f^+(x_i) - f^-(x_i)$ which are the jumps of $f$ at its discontinuities. Next, define the (continuous) function \[f_c(x) = f(x) - \sum_{i=1}^n\sigma_iH(x-x_i)\] Differentiation gives \[f'(x) = f'_c(x) + \sum_{i=1}^n\sigma_i\delta(x-x_i)\] It follows that the derivative in the generalized sense of a piecewise continuous function equals the derivative in the classical sense plus a summation of delta functions at every jump discontinuity.
	}

\section{Fourier series}
   
	\newdef{Dirichlet kernel}{\index{Dirichlet!kernel}
   		The Dirichlet kernel is the collection of functions of the form
        \begin{equation}
        	\label{transforms:dirichlet_kernel}
            D_n(x) = \stylefrac{1}{2\pi}\sum_{k=-n}^ne^{ikx}
        \end{equation}
	}
    \newformula{Sieve property}{\index{sieve}
    	If $f\in C^1[-\pi, \pi]$ then
        \begin{equation}
        	\lim_{n\rightarrow+\infty}\int_{-\pi}^\pi f(x)D_n(x)dx = 0
        \end{equation}
    }
    \begin{formula}
    	For $2\pi$-periodic functions, the $n^{\text{th}}$ degree Fourier approximation is given by the following convolution:
    	\begin{equation}
    		s_n(x) = \sum_{k=-n}^n\widetilde{f}(k)e^{ikx} = (D_n \ast f)(x)
    	\end{equation}
    \end{formula}
    
    \begin{theorem}[Convergence of the Fourier series]
    	Let $f:\mathbb{R}\rightarrow\mathbb{R}$ be a function with period $2\pi$. If $f(x)$ is piecewise $C^1$ on $[-\pi, \pi]$ the the limit $\lim_{n\rightarrow+\infty}(D_n\ast f)(x)$ converges to $\frac{f(x+) + f(x-)}{2}$ for all $x\in\mathbb{R}$.
    \end{theorem}
	\newformula{Generalized Fourier series}{\index{Fourier!series}
    	Let $f(x)\in \mathcal{L}^2[-l, l]$ be a $2l$-periodic function. This function can be approximated by the following series:
        \begin{equation}
            \label{transforms:fourier_series}
            \boxed{f(x) = \sum_{n = -\infty}^{+\infty} \left(\frac{1}{2l}\int_{-l}^le^{-i\frac{n\pi x'}{l}}f(x')dx'\right) e^{i\frac{n\pi x}{l}}}
        \end{equation}
	}
    
    \begin{formula}[Fourier coefficients]
		As seen in the general formula, the Fourier coefficient $\widetilde{f}(n)$ can be calculated by taking the inner product \ref{hilbert:inner_product} of $f(x)$ and the $n$-th eigenfunction $e_n$:
		\begin{equation}
			\label{transforms:fourier_coefficients}
        		\widetilde{f}(n) = \left\langle e_n|f\right\rangle = \int_{-l}^le_n^*(x)f(x)dx \qquad\text{with}\qquad e_n = \sqrt\frac{1}{2l}e^{i\frac{n\pi x}{l}}
		\end{equation}
	\end{formula}
    
	\newdef{Periodic extension}{\index{periodic!extension}
    	Let $f(x)$ be piecewise $C^1$ on $[-L, L]$. The periodic extension $f^L(x)$ is defined by repeating the restriction of $f(x)$ to $[-L, L]$ every $2L$. The \textbf{normalized periodic extension} is defined as
        \begin{equation}
        	f^{L, \nu}(x) = \stylefrac{f^L(x+) + f^L(x-)}{2}
        \end{equation}
    }
    \begin{theorem}
    	If $f:\mathbb{R}\rightarrow\mathbb{R}$ is piecewise $C^1$ on $[-L, L]$ then the Fourier series approximation of $f(x)$ converges to $f^{L, \nu}(x)$ for all $x\in\mathbb{R}$.
    \end{theorem}

\section{Fourier transform}\index{Fourier!transform}
	
	The Fourier series can be used to expand a $2l$-periodic function as an infinite series of exponentials. For expanding a non-periodic function we need the Fourier integral: 
	\begin{equation}
		\label{transforms:fourier}
	        \boxed{\mathcal{F}(\omega) = \frac{1}{\sqrt{2\pi}} \int_{-\infty}^{\infty}f(t)e^{-i\omega t}dt}
	\end{equation}
    
	\begin{equation}
		\label{transforms:inverse_fourier}
	        f(t) = \mathcal{F}^{-1}(t) = \frac{1}{\sqrt{2\pi}} \Xint-_{-\infty}^{\infty}\mathcal{F}(\omega)e^{i\omega t}d\omega
	\end{equation}
    
	Equation \ref{transforms:fourier} is called the (forward) Fourier transform of $f(t)$ and equation \ref{transforms:inverse_fourier} is called the inverse Fourier transform.
    
	\begin{notation}
		The Fourier transform of a function $f(t)$, as seen in equation \ref{transforms:fourier}, is often denoted by $\widetilde{f}(\omega)$.
	\end{notation}
    
	\begin{theorem}[Convergence of the Fourier integral]
	    	If $f:\mathbb{R}\rightarrow\mathbb{R}$ is Lipschitz continuous (see \ref{calculus:lipschitz_continuity}) and if $\int_{-\infty}^{+\infty}|f(x)|dx$ is convergent then the Fourier integral converges to $f(x)$ for all $x\in\mathbb{R}$.
	\end{theorem}
	\begin{theorem}[Fourier inversion theorem]
	    	If both $f(t), \mathcal{F}(\omega)\in\mathcal{L}^1(\mathbb{R})$ are continuous then the Cauchy principal value in \ref{transforms:inverse_fourier} can be replaced by a normal integral.
	\end{theorem}
	\begin{remark}
    		Schwartz functions are continuous elements of $\mathcal{L}^1(\mathbb{R})$ and as such the Fourier inversion theorem also holds for these functions. This is interesting because checking the conditions for Schwartz functions is often easier then checking the more general conditions of the theorem.
	\end{remark}
    
	\begin{property}
	    	From the Riemann-Lebesgue lemma \ref{lebesgue:riemann_lebesue_lemma} it follows that
	        \begin{equation}
	        	\mathcal{F}(\omega)\rightarrow0 \qquad\text{if}\qquad |\omega|\rightarrow0
	        \end{equation}
	\end{property}
    
	\newprop{Parceval's theorem}{\index{Parceval}
	   	Let $(f, \widetilde{f})$ and $(g,\widetilde{g})$ be two Fourier transform pairs.
	        \begin{equation}
			\label{transforms:parcevals_theorem}
		        \int_{-\infty}^{+\infty}f(x)g(x)dx = \int_{-\infty}^{+\infty}\widetilde{f}(k)\widetilde{g}(k)dk
		\end{equation}
	}
	\begin{result}[Plancherel theorem]\index{Plancherel}
    		The integral of the square (of the modulus) of a Fourier transform is equal to the integral of the square (of the modulus) of the original function:
    		\begin{equation}
    			\label{transforms:plancherel_theorem}
        		\int_{-\infty}^{+\infty}|f(x)|^2dx = \int_{-\infty}^{+\infty}|\widetilde{f}(k)|^2dk
	   	\end{equation}
	\end{result}
	
	\begin{theorem}
		Let $f(x), F(k)$ be a Fourier transform pair. If $f(x)\in S(\mathbb{R},\mathbb{C})$, then $F(k)\in S(\mathbb{R},\mathbb{C})$. It follows that for a sequence $\{f_n(x)\in S(\mathbb{R},\mathbb{C})\}$ the sequence of Fourier transformed functions $\{F_n(x)\in S(\mathbb{R},\mathbb{C})\}$ is also a subset of the Schwartz space. Furthermore Parceval's theorem \ref{transforms:parcevals_theorem} gives \[\int_{-\infty}^{+\infty}f_n(x)g(x)dx = \int_{-\infty}^{+\infty}F_n(x)G(x)dx\in\mathbb{R}\] where $g(x)\in S(\mathbb{R},\mathbb{C})$. From these two properties it follows that the Fourier transform of a generalized function is also a generalized function.
	\end{theorem}
    
	\begin{property}
		Let $\psi$ be a generalized function with Fourier transform $\Psi$. Let $f(x)\in S(\mathbb{R},\mathbb{C})$ with Fourier transform $F(k)$. We have the following equality:
        \begin{equation}
			\langle \psi|F \rangle = \langle \Psi|f \rangle
		\end{equation}
	\end{property}

\subsection{Convolution}

        \newformula{Convolution}{\index{convolution}
	        \begin{equation}
	                \label{transforms:convolution}
	                \boxed{(f \ast g)(t) = \frac{1}{\sqrt{2\pi}} \int_{-\infty}^{\infty}f(\tau)g(t - \tau)d\tau}
	        \end{equation}
        }
        
        \begin{property}[Commutativity]
		\begin{equation}
			f \ast g = g \ast f
	        \end{equation}
	\end{property}
        
        \begin{theorem}[Convolution theorem]
		\begin{equation}
			\widetilde{f \ast g} = \widetilde{g} \widetilde{f}
	        \end{equation}
	\end{theorem}

\section{Laplace transform}

	\newformula{Laplace transform}{\index{Laplace!transform}
	        \begin{equation}
		        \label{transforms:laplace}
        		\mathcal{L}\{F(t)\}_{(s)} = \int_{0}^{\infty}f(t)e^{-st}dt
	        \end{equation}
	}
    
	\newformula{Bromwich integral}{\index{Bromwich!integral}
	        \begin{equation}
		        \label{transforms:inverse_laplace}
        		f(t) = \frac{1}{2\pi i} \int_{\gamma - i\infty}^{\gamma + i\infty}\mathcal{L}\{F(t)\}_{(s)}e^{st}ds
	        \end{equation}
	}
    
	\begin{notation}
		The Laplace transform as defined in equation \ref{transforms:laplace} is sometimes denoted by $f(s)$ .
	\end{notation}
    
\section{Mellin transform}

	\newformula{Mellin transform}{\index{Mellin!transform}
	    	\begin{equation}
			\label{transforms:mellin}
	    		\mathcal{M}\{f(x)\}(s) = \int_0^{+\infty}x^{s-1}f(x)dx
	    	\end{equation}
	}
	\newformula{Inverse Mellin transform}{
		\begin{equation}
			\label{transforms:inverse_mellin}
			f(x) = \frac{1}{2\pi i} \int_{\gamma - i\infty}^{\gamma + i\infty}\mathcal{M}\{f(x)\}_{(s)}x^{-s}ds
		\end{equation}
	}
    
\section{Integral representations}
	
	\newformula{Heaviside step function}{\index{Heaviside!step function}
		\begin{equation}
			\theta(x) = \frac{1}{2\pi i}\int_{-\infty}^\infty\frac{e^{ikx}}{x - i\varepsilon}dk
		\end{equation}
	}
	\newformula{Dirac delta function}{\index{Dirac!delta function}
		\begin{equation}
			\delta^{(n)}(\vector{x}) = \frac{1}{(2\pi)^n}\int_{-\infty}^\infty e^{i\vector{k}\cdot\vector{x}}d^nk
		\end{equation}
	}

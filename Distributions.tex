\chapter{Distributions}
\section{Generalized function}

	\newdef{Schwartz space}{\index{Schwartz!space of rapidly decreasing functions}
	    	The Schwartz space or \textbf{space of rapidly decreasing functions}\footnotemark\ $S(\mathbb{R})$ is defined as:
	    	\footnotetext{These functions are said to be rapidly decreasing because every derivative $f^{(j)}(x)$ decays faster than any polynomial $x^i$ for $x\rightarrow+\infty$.}
		\begin{equation}
			\label{distribution:schwartz_space}
			S(\mathbb{R}) = \left\{f(x)\in C^\infty(\mathbb{R}):\forall i,j\in\mathbb{N}:\forall x\in\mathbb{R}:|x^if^{(j)}(x)|<+\infty\right\}
		\end{equation}
	}
    
    	\remark{This definition can be generalized to functions of the class $C^\infty(\mathbb{R}^n)$ or functions $f:\mathbb{R}\rightarrow\mathbb{C}$. The Schwartz space is then denoted by $S(\mathbb{R},\mathbb{C})$.}
    
	\newdef{Functions of slow growth}{
	    	The set of functions of slow growth $N(\mathbb{R})$ is defined as:
	        \begin{equation}
			N(\mathbb{R}) = \{f(x)\in C^\infty(\mathbb{R}) : \forall i\in\mathbb{N},\exists M_i > 0: |f^{(i)}(x)| = O(|x|^i) \text{ for } |x|\rightarrow+\infty\}
		\end{equation}
	}
	\sremark{It is clear that all polynomials belong to $N(\mathbb{R})$ but not to $S(\mathbb{R})$.}
    
	\begin{property}
		If $f(x)\in S(\mathbb{R})$ and $a(x)\in N(\mathbb{R})$ then $a(x)f(x)\in S(\mathbb{R})$.
	\end{property}
    
	\newdef{Generalized function}{\index{generalized!function}
    		Let $g(x)\in S(\mathbb{R})$ be a test function. Let $\{f_n(x)\in S(\mathbb{R})\}, \{h_n(x)\in S(\mathbb{R})\}$ be sequences such that
        	\[
        		\lim_{n\rightarrow+\infty}\langle f_n(x)|g(x) \rangle = \lim_{n\rightarrow+\infty}\int_{-\infty}^{+\infty}f_n(x)g(x)dx
        	\]
        	and similarly for $h_n$.
        	Define the equivalence relation $\{f_n(x)\in S(\mathbb{R})\} \sim \{h_n(x)\in S(\mathbb{R})\}$ by saying that the two sequences, satisfying the previous condition, are equivalent if and only if
        	\[
        		\lim_{n\rightarrow+\infty}\langle f_n(x)|g(x) \rangle = \lim_{n\rightarrow+\infty}\langle h_n(x)|g(x) \rangle
		\]
        	A generalized function is defined as a complete equivalence class under previous relation.
	}
    
	\begin{notation}
		Let $\psi$ be a generalized function. Let $f\in S(\mathbb{R})$. The inner product \ref{lebesgue:L2_inner_product} is generalized by following functional:
        \begin{equation}
			\langle \psi|f \rangle = \lim_{n\rightarrow+\infty}\int_{-\infty}^{+\infty}\psi_n(x)f(x)dx
		\end{equation}
	\end{notation}
    
    \begin{property}
		Let $\psi$ be a generalized function. Let $f(x)\in S(\mathbb{R})$. The previous functional has following properties:
        \begin{itemize}
			\item \label{distribution:gf_prop1}$\forall i\in\mathbb{N}: \langle \psi^{(i)}|f \rangle = (-1)^i \langle \psi|f^{(i)} \rangle$
            \item $\forall a,b\in\mathbb{R}, a\neq0: \langle \psi(ax+b)|f(x) \rangle = |a|^{-1}\langle \psi(x)|f(x - b/a) \rangle$
            \item $\forall a(x)\in N(\mathbb{R}):\langle a\psi|f \rangle = \langle \psi|af \rangle$
		\end{itemize}
	\end{property}
    
    \newprop{Ordinary function as generalized function}{
    	Let $f:\mathbb{R}\rightarrow\mathbb{C}$ be a function such that $\exists M\geq0:(1+x^2)^{-M}|f(x)| \in L(\mathbb{R},\mathbb{C})\footnotemark$. There exists a generalized function $\psi\sim\{f_n(x) \in S(\mathbb{R},\mathbb{C})\}$ such that for every $g(x)\in S(\mathbb{R},\mathbb{C})$:
        \[
        	\langle \psi|g \rangle = \langle f|g \rangle
        \]
        Furthermore if $f(x)$ is continuous on an interval, then $\ds\lim_{n\rightarrow+\infty} f_n(x) = f(x)$ converges pointwise on that interval.
    }
    \footnotetext{The space of Lebesgue integrable functions \ref{lebesgue:L1}.}

\section{Dirac Delta distribution}

	\newdef{Heaviside function}{\index{Heaviside!function}
    	Define the generalized function $H\sim\{H_n(x)\in S(\mathbb{R})\}$ as:
        \begin{equation}
        	\label{distribution:heaviside_function}
			H(x) = \left\{
            \begin{array}{ccc}
				0&if&x<0\\
                1&if&x\geq0
			\end{array}\right.
		\end{equation}
        From this definition it follows that for every $f\in S(\mathbb{R})$:
        \begin{equation}
        	\label{distribution:heaviside_function_integral}
			\langle H|f \rangle = \int_0^{+\infty}f(x)dx
		\end{equation}
    }
    \remark{For the above integral to exist, $f(x)$ does not need to be an element of $S(\mathbb{R})$. It is a sufficient condition, but not a necessary one.}
    
    \newdef{Generalized delta function}{\index{Dirac!delta function}
    	The Dirac delta function is defined as a representant of the equivalence class of generalized functions $\{H_n'(x)\in S(\mathbb{R})\}$.
        By equations \ref{distribution:gf_prop1} and \ref{distribution:heaviside_function_integral} we have for every $f\in S(\mathbb{R})$:
        \begin{equation}
			\label{distribution:dirac_delta}
            \begin{array}{ccc}
            	\langle \delta|f \rangle&=&\langle H'|f \rangle\\
                &=&-\langle H|h \rangle\\
                &=&-\ds\int_0^{+\infty}f'(x)dx\\
                &=&f(0)
            \end{array}
		\end{equation}
    }
    
	\begin{property}[Sampling property]
    	The result from previous definition can be generalized in the following way:
        \begin{equation}
			\label{distribution:sieving_dirac_delta}
	        \boxed{f(x_0) = \int_\mathbb{R}f(x)\delta(x - x_0)dx}
		\end{equation}
	\end{property}
    \begin{example}[Dirac comb]\index{Dirac!comb}
    	\begin{equation}
			\label{distribution:dirac_comb}
            \uppercase\expandafter{\romannumeral 3}_b(x) = \sum_n\delta(x-nb)
		\end{equation}
    \end{example}
    
	\begin{property}
		Let $f(x)\in C^1(\mathbb{R})$ be a function with roots at $x_1,x_2,...,x_n$ such that $f'(x_i)\neq0$. The Dirac delta distribution has the following property:
		\begin{equation}
			\label{distribution:delta_of_function}
			\delta[f(x)] = \sum_{i=1}^n\stylefrac{1}{|f'(x_i)|}\delta(x-x_i)
		\end{equation}
	\end{property}
    
    \newprop{Convolution with delta function}{\index{convolution}
    	Let $f(x)\in S(\mathbb{R})$. Let $\otimes$ denote the convolution.
        \begin{equation}
			\delta(x)\otimes f(x) = \int_{-\infty}^{+\infty}\delta(x-\alpha)f(\alpha)d\alpha = f(x)
		\end{equation}
    }
    
    \newformula{Differentiation across discontinuities}{
    	Let $f(x)$ be a piecewise continuous function with discontinuities at $x_1,...,x_n$. Let $f$ satisfy the conditions to be a generalized function. Define $\sigma_i = f^+(x_i) - f^-(x_i)$ which are the jumps of $f$ at its discontinuities. Next, define the function
        \[
        	f_c(x) = f(x) - \sum_{i=1}^n\sigma_iH(x-x_i)
        \]
        which is a continuous function. Differentiation gives
        \[
        	f'(x) = f'_c(x) + \sum_{i=1}^n\sigma_i\delta(x-x_i)
        \]
        It follows that the derivative in a generalized sense of a piecewise continuous function equals the derivative in the classical sense plus a summation of delta functions at every jump discontinuity.
    }
    
\section{Fourier transform}\index{Fourier!transform}
	
    \begin{theorem}
		Let $f(x), F(k)$ be a Fourier transform pair. If $f(x)\in S(\mathbb{R},\mathbb{C})$, then $F(k)\in S(\mathbb{R},\mathbb{C})$. It follows that for a sequence $\{f_n(x)\in S(\mathbb{R},\mathbb{C})\}$ the sequence of Fourier transformed functions $\{F_n(x)\in S(\mathbb{R},\mathbb{C})\}$ is also a subset of the Schwartz space. Furthermore Parceval's theorem \ref{transforms:parcevals_theorem} gives
        \[
        	\int_{-\infty}^{+\infty}f_n(x)g(x)dx = \int_{-\infty}^{+\infty}F_n(x)G(x)dx\in\mathbb{R}
        \]
        where $g(x)\in S(\mathbb{R},\mathbb{C})$. From these two properties it follows that the Fourier transform of a generalized functions is also a generalized functions.
	\end{theorem}
    
    \begin{property}
		Let $\psi$ be a generalized function with Fourier transform $\Psi$. Let $f(x)\in S(\mathbb{R},\mathbb{C})$ with Fourier transform $F(k)$. We have the following equality:
        \begin{equation}
			\langle \psi|F \rangle = \langle \Psi|f \rangle
		\end{equation}
	\end{property}
    
    \newformula{Fourier representation of delta function}{
    	\begin{equation}
			\boxed{\delta(x-a) = \stylefrac{1}{2\pi}\int_{-\infty}^{+\infty}e^{ik(x-a)}dk}
		\end{equation}
    }

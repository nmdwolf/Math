\section{Space of integrable functions}
\subsection{Distance}\index{distance}
	To define a distance between functions, we first have to define some notion of length of a function. Normally this would not be a problem, because we now do know how to integrate integrable functions, however the fact that two functions differing on a null set have the same integral carries problems with it, i.e. a non-zero function could have a zero length. Therefore we will define the 'length' on a different vector space:\par
    
    \noindent Define the following set of equivalence classes $L^1(E) = \mathcal{L}^1(E)_{/\equiv}$ by introducing the equivalence relation: $f\equiv g$ if and only if $f=g$ a.e.
    \begin{property}
		$L^1(E)$ is a Banach space\footnotemark.
	\end{property}
    \footnotetext{See definition \ref{linalgebra:banach_space}.}
    
    \begin{formula}
		A norm on $L^1(E)$ is given by:
        \begin{equation}
			\label{lebesgue:L1_norm}
            ||f||_1 = \int_E |f|dm
		\end{equation}
	\end{formula}
    
\subsection{Hilbert space \texorpdfstring{$L^2$}\ }\index{Hilbert!space|see{L$^2$}}\index{L$^2$}
	\label{lebesgue:section:hilbert_space}
    
    \begin{property}
    	\label{lebesgue:L2_hilbert_space}
		$L^2$ is a Hilbert space\footnotemark.
	\end{property}
    \footnotetext{See definition \ref{hilbert:hilbert_space}.}
	\begin{formula}
		A norm on $L^2(E)$ is given by:
        \begin{equation}
			\label{lebesgue:L2_norm}
            ||f||_2 = \left(\int_E |f|^2dm\right)^{\frac{1}{2}}
		\end{equation}
        This norm is induced by the following inner product:
        \begin{equation}
			\label{lebesgue:L2_inner_product}
            \boxed{\langle f|g \rangle = \int_E f\overline{g}dm}
		\end{equation}
	\end{formula}
    Now instead of deriving $L^2$ from $\mathcal{L}^2$ we do the opposite. We define $\mathcal{L}^2$ as the set of measurable functions for which equation \ref{lebesgue:L2_norm} is finite.
    
    \newdef{Orthogonality}{\index{orthogonality}
    	As $L^2$ is a Hilbert space and thus has an inner product $\langle\cdot|\cdot\rangle$, it is possible to introduce the concept of orthogonality of functions in the following way:
        \begin{equation}
			\label{lebesgue:orthogonal_functions}
            \langle f|g \rangle = 0\implies\text{f and g are orthogonal}
		\end{equation}
        Furthermore it is also possible to introduce the angle between functions in the same way as equation \ref{linalgebra:angle}.
    }
    
    \begin{formula}[Cauchy-Schwarz inequality]\index{Cauchy!Cauchy-Schwarz inequality}
		Let $f,g\in L^2(E,\mathbb{C})$. We have that $fg\in L^1(E\mathbb{C})$ and:
        \begin{equation}
			\label{lebesgue:schwarz_inequality}
            \boxed{\left|\int_E f\overline{g}dm\right|\leq||fg||_1\leq||f||_2||g||_2}
		\end{equation}
	\end{formula}
    \sremark{This follows immediately from formula \ref{lebesgue:holders_inequality}.}
    
    \begin{property}
		If $E$ has finite Lebesgue measure then $L^2(E)\subset L^1(E)$.
	\end{property}
    
\subsection{\texorpdfstring{$L^p$}\ \ spaces}\index{L$^p$}
	Generalizing the previous two Lebesgue function classes leads us to the notion of $L^p$ spaces with the following norm:
    
    \begin{property}For all $1\leq p\leq+\infty$ $L^p(E)$ is a Banach space with a norm given by:
    	\begin{equation}
			\label{lebesgue:Lp_norm}
            ||f||_p = \left(\int_E |f|^p\ dm\right)^{\frac{1}{p}}
		\end{equation}
    \end{property}
    \remark{Note that $L^2$ is the only $L^p$ space that is also a Hilbert space. The other $L^p$ spaces do not have a norm induced by an inner product.}
    
    \newformula{H\"{o}lder's inequality}{\index{H\"older's inequality}
    	Let $\frac{1}{p} + \frac{1}{q} = 1$ with $p\geq1$. For every $f\in L^p(E)$ and $g\in L^q(E)$ we have that $fg\in L^1(E)$ and:
        \begin{equation}
        	\label{lebesgue:holders_inequality}
			||fg||_1\leq||f||_p||g||_q
		\end{equation}
    }
    \newformula{Minkowski's inequality}{\index{Minkowski!inequality}
    	For every $p\geq1$ and $f,g\in L^p(E)$ we have
        \begin{equation}
			\label{lebesgue:minkowskis_inequality}
            ||f+g||_p\leq||f||_p + ||g||_p
		\end{equation}
    }
    \begin{property}
		If $E$ has finite Lebesgue measure then $L^q(E)\subset L^p(E)$ when $1\leq p\leq q<+\infty$.
	\end{property}
    
\subsection{\texorpdfstring{$L^\infty$}\ \ space of essentially bounded measurable functions}
	\newdef{Essentially bounded function}{
    	Let $f$ be a measurable function satisfying $\esssup |f| <+\infty$. The function $f$ is said to be essentially bounded and the set of all such functions is denoted by $L^\infty(E)$.
    }
    
    \begin{formula}\index{supremum}
		A norm on $L^\infty$ is given by:
        \begin{equation}
			||f||_\infty = \esssup|f|
		\end{equation}	
        This norm is called the \textbf{supremum norm} and it induces the supremum metric \ref{topology:supremum_distance}.
	\end{formula}
    \begin{property}
		$L^\infty$ is a Banach space.
	\end{property}
\section{Lebesgue integral}
\subsection{Simple functions}

    	\newdef{Indicator function}{\index{indicator function}
        	An important function when working with sets is the following one:
	        \begin{equation}
        	    	\label{lebesgue:indicator_function}
			\boxed{\mathbbm{1}_A(x) = \left\{
        		\begin{array}{ccc}
				1&\text{if}&x\in A\\
		                0&\text{if}&x\not\in A
			\end{array}\right.}
		\end{equation}
        }
	\newdef{Simple function}{\index{simple function}
    		Let $f$ be a function that takes on a finite number of non-negative values $\{a_i\}$ with for every $i\neq j: f^{-1}(a_i)\cap f^{-1}(a_j) = \emptyset$. $f$ is called a simple function if it can be expanded in the following way:
        	\begin{equation}
			\label{lebesgue:simple_function}
			f(x) = \sum_{i=1}^n a_i\mathbbm{1}_{A_i}(x)
		\end{equation}
        	with $A_i = f^{-1}(a_i)\in\mathcal{M}$
    	}
        \begin{remark}[Step function]\index{step function}\label{lebesgue:step_function}
        	If the sets $A_i$ are intervals, the simple function is often called a 'step function'.
	\end{remark}
        
        \newformula{Lebesgue integral of simple functions}{\index{Lebesgue}
        	Let $\varphi$ be a simple function as defined in equation \ref{lebesgue:simple_function}. Let $\mu:\mathcal{M}\rightarrow\mathbb{R}$ be a Lebesgue measure and let $E$ be a measurable set. The Lebesgue integral of $\varphi$ over a $E$ with respect to $\mu$ is given by:
		\begin{equation}
			\label{lebesgue:integral_simple_function}
                	\int_E\varphi d\mu = \sum_{i=1}^na_i\mu(E\cap A_i)
		\end{equation}
        }
        \begin{example}
		Let $\mathbbm{1}_\mathbb{Q}$ be the indicator function of the set of rational numbers. This function is clearly a simple function. Previous formula makes it possible to integrate the rational indicator function over the real line, which is not possible in the sense of Riemann:
		\begin{equation}
			\int_\mathbb{R}\mathbbm{1}_\mathbb{Q}dm = 1\times m(\mathbb{Q}) + 0\times m(\mathbb{R}\backslash\mathbb{Q}) = 0
		\end{equation}
	        where the measure of the rational numbers is 0 because it is a countable set (see corollary \ref{lebesgue:theorem:countable_set_is_null}.
	\end{example}
        
\subsection{Measurable functions}

        \newformula{Lebesgue integral}{\index{Lebesgue!integral}
        	Let $f$ be a non-negative measurable function. Let $A$ be measurable set. The Lebesgue integral of $f$ over $E$ is defined as:
	        \begin{equation}
			\label{lebesgue:integral}
        	        \int_Efdm = \sup\left\{\int_E\varphi dm:\varphi \text{ a simple function such that } \varphi\leq f\right\}
		\end{equation}
        }
        \begin{property}
		The Lebesgue integral $\int_Efdm$ of a measurable function $f$ is always non-negative.
	\end{property}
        
        \begin{notation}
        	The following notation is frequently used (both in the sense of Riemann and Lebesgue):
		\begin{equation}
			\int fdm = \int_\mathbb{R}fdm
		\end{equation}
	\end{notation}
        \begin{formula}
		The following equality is easily proved as for every set $A\subseteq\mathbb{R}:A\cup A^c = \mathbb{R}$.
        	\begin{equation}
        	    	\label{lebesgue:interchanging_domains_with_indicator_function}
			\int_Afdm = \int f\mathbbm{1}_Adm
		\end{equation}
	\end{formula}
        
        \begin{theorem}
		Let $f$ be a non-negative measurable function. Then $f=0$ a.e. if and only if $\int_\mathbb{R} fdm = 0$.
	\end{theorem}
        
        \begin{property}
		The Lebesgue integral over a null set is 0.
	\end{property}
        \begin{property}\index{mean!value theorem}
		Let $f,g$ me measurable functions. The Lebesgue integral has the following properties:
        	\begin{itemize}
        	    	\item $f\leq g$ a.e. implies $\int fdm\leq\int gdm$.
        	        \item Let $A$ be a measurable set. Let $B\subset A$. Then $\int_B fdm\leq\int_A fdm$.
        	        \item The Lebesgue integral is linear.
        	        \item For every two disjoint measurable sets $A$ and $B$ we have that $\int_{A\cup B}fdm = \int_A fdm + \int_B fdm$.
        	        \item \textbf{Mean value theorem}: If $a\leq f(x)\leq b$, then $am(A)\leq\int_Afdm\leq bm(A)$.
		\end{itemize}
	\end{property}
        
        \begin{theorem}
			Let $f$ be a non-negative measurable function. There exists an increasing sequence $(\varphi_i)_{i\in\mathbb{N}}$ of simple functions such that $\varphi_i\nearrow f$.
		\end{theorem}
        \begin{theorem}
			Let $f$ be a bounded measurable function defined on the interval $[a,b]$. For every $\varepsilon>0$ there exists a step function\footnotemark\ $h$ such that $\int_a^b|f-h|dm<\varepsilon$.
		\end{theorem}
        \footnotetext{See remark \ref{lebesgue:step_function}.}
        

\subsection{Integrable functions}
        \newdef{Integrable function}{\index{integrable}
        	Let $E\in\mathcal{M}$. A measurable function $f$ is said to be integrable over $E$ if both $\int_Ef^+dm$ and $\int_Ef^-dm$ are finite. The Lebesgue integral of $f$ over $E$ is defined as:
            \begin{equation}
				\label{lebesgue:integrable_function}
                \int_E fdm = \int_E f^+dm - \int_E f^-dm
			\end{equation}
        }
        \sremark{The difference between the integral \ref{lebesgue:integral} and the integral of an integrable function is that with the latter $f$ does not have to be non-negative.}
        
        \begin{theorem}
			$f$ is integrable if and only if $|f|$ is integrable. Furthermore, $\int_E|f|dm = \int_E f^+dm + \int_E f^-dm$.
		\end{theorem}
        \begin{property}
			Let $f,g$ be integrable functions. The following important properties apply:
            \begin{itemize}
				\item $f+g$ is also integrable.
                \item $\forall E\in\mathcal{M}, \int_Efdm\leq\int_Egdm\implies f\leq g$ a.e.
                \item Let $c\in\mathbb{R}$. $\int_E(cf)dm = c\int_Efdm$.
                \item $f$ is finite a.e.
                \item $|\int fdm|\leq\int|f|dm$
                \item $f\geq0\land\int fdm=0\implies f=0$ a.e.
			\end{itemize}
		\end{property}
        
        \begin{theorem}
        	\label{lebesgue:L1}
			The set of functions integrable over a set $E\in\mathcal{M}$ forms a vector space. It is denoted by $\mathcal{L}^1(E)$.
		\end{theorem}
        
        \begin{property}
        	Let $f\in\mathcal{L}^1$ and $\varepsilon>0$. There exists a continuous function $g$, vanishing outside some finite interval, such that $\int|f-g|dm<\varepsilon$.
		\end{property}
        
        \begin{property}\index{continuity!absolute continuity}
        	\label{lebesgue:theorem:measure_by_integral}
			Let $f\geq0$. The mapping $E\mapsto\int_Efdm$ is a measure on $E$ (if it exists, hence if $f$ is integrable). Furthermore, this measure is said to be \textbf{absolutely continuous}.
		\end{property}
        \sremark{See section \ref{lebesgue:section:Radon-Nikodym} for further information.}

    \subsection{Convergence theorems}
    	\begin{theorem}[Fatou's lemma]\index{Fatou's lemma}
			Let $(f_n)_{n\in\mathbb{N}}$ be a sequence of non-negative measurable functions.
            \begin{equation}
				\label{lebesgue:theorem:fatous_lemma}
                \int_E\left(\liminf_{n\rightarrow\infty}f_n\right)dm \leq \liminf_{n\rightarrow\infty}\int_Ef_ndm
			\end{equation}
		\end{theorem}
        \begin{theorem}[Monotone convergence theorem]\index{monotone!convergence theorem}
			Let $E\in\mathcal{M}$. Let $(f_n)_{n\in\mathbb{N}}$ be an increasing sequence of non-negative measurable functions such that $f_n\nearrow f$ pointwise a.e. We have the following powerful equality:
            \begin{equation}
				\label{lebesgue:theorem:monotone_convergence_theorem}
                \boxed{\int_E fdm = \lim_{n\rightarrow\infty}\int_E f_n(x)dm}
			\end{equation}
		\end{theorem}
        
        \begin{method}
        	 \label{lebesgue:method:linear_proofs}
			To prove 'linear' results concerning integrable functions in spaces such as $\mathcal{L}^1(E)$ we proceed according to the following steps:
            \begin{enumerate}
            	\item Verify that the property holds for indicator functions. (This often follows by definition.)
                \item Use the linearity to extend the property to simple functions.
                \item Apply the monotone convergence theorem to show that the property holds for all non-negative measurable functions.
                \item Extend the property to all integrable functions by writing $f = f^+ - f^-$ and applying the linearity again.
			\end{enumerate}
		\end{method}
        
        \begin{theorem}[Dominated convergence theorem]\index{dominated convergence theorem}
			Let $E\in\mathcal{M}$. Let $(f_n)_{n\in\mathbb{N}}$ be a sequence of measurable functions with $\forall n:|f_n|\leq g$ a.e. for a function $g\in\mathcal{L}^1(E)$. If $f_n\rightarrow f$ pointwise a.e. then $f$ is integrable over $E$ and
            \begin{equation}
				\label{lebesgue:theorem:dominated_convergence_theorem}
                \int_E fdm = \lim_{n\rightarrow\infty}\int_E f_n(x)dm
			\end{equation}
		\end{theorem}
        
        \begin{property}
			Let $(f_n)_{n\in\mathbb{N}}$ be a sequence of non-negative measurable functions. The following equality applies:
            \begin{equation}
				\int\sum_{n=1}^{+\infty}f_n(x)dm = \sum_{n=1}^{+\infty}\int f_n(x)dm
			\end{equation}
            We cannot conclude that the right-hand side is finite a.e., so the series on the left-hand side need not be integrable.
		\end{property}
        
        \begin{theorem}[Beppo-Levi]\index{Levi|see{Beppo-Levi}}\index{Beppo-Levi}
			Suppose that
            \[\sum_{i=1}^\infty\int|f_n|(x)dm\text{ is finite.}\]
            The series $\sum_{i=1}^\infty f_n(x)$ converges a.e. Furthermore, the series is integrable and
            \begin{equation}
				\label{lebesgue:theorem:beppo_levi}
                \int\sum_{i=1}^\infty f_n(x)dm = \sum_{i=1}^\infty\int f_n(x)dm
			\end{equation}
		\end{theorem}
        
        \begin{theorem}[Riemann-Lebesgue lemma]\index{Riemann!Riemann-Lebesgue lemma}\label{lebesgue:riemann_lebesue_lemma}
			Let $f\in\mathcal{L}^1$. The sequences \[s_k = \int_{-\infty}^{+\infty}f(x)\sin(kx)dx\] and \[c_k = \int_{-\infty}^{+\infty}f(x)\cos(kx)dx\] both converge to 0.
		\end{theorem}
        \sremark{This theorem is useful in Fourier analysis.}

\subsection{Relation to the Riemann integral}
	\begin{theorem}[Fundamental theorem of calculus]
		If $f:[a,b]\rightarrow\mathbb{R}$ is continuous then $f$ is integrable and the function $F:x\mapsto\int_a^xfdm$ is differentiable for $x\in]a,b[$ such that $F'=f$.
	\end{theorem}
    
    \begin{theorem}
		Let $f:[a,b]\rightarrow\mathbb{R}$ be a bounded function.
        \begin{itemize}
        	\item $f$ is Riemann-integrable if and only if $f$ is continuous a.e. with respect to the Lebesgue measure on $[a,b]$.
            \item Riemann-integrable functions on $[a,b]$ are integrable with respect to the Lebesgue measure on $[a,b]$ and the integrals coincide.
		\end{itemize}
	\end{theorem}
    
    \begin{theorem}
		If $f\geq0$ and the improper Riemann integral \ref{calculus:improper_integral} exists, then the Lebesgue integral $\int fdm$ exists and the two integrals coincide.
	\end{theorem}

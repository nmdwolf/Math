\chapter{Algebra}

    For the later sections on (co)homology theory some terminology from Chapter \ref{chapter:hom_alg} is used.

\section{Algebraic structures}

    \newdef{Semigroup}{\index{semi-!group}\index{magma}\label{algebra:semigroup}
        A set $G$ equipped with a binary operation $\star$ such that the following axioms are satisfied:
        \begin{enumerate}
            \item\textbf{Closure}: $G$ is closed under $\star$.
            \item\textbf{Associativity}: $\star$ is asssociative.
        \end{enumerate}
        If the associativity axiom is dropped, a \textbf{magma} is obtained.
    }

    \newdef{Monoid}{\index{monoid}\label{algebra:monoid}
        A set $M$ equipped with a binary operation $\star$ such that the following axioms are satisfied:
        \begin{enumerate}
            \item\textbf{Closure}: $M$ is closed under $\star$.
            \item\textbf{Associativity}: $\star$ is associative.
            \item\textbf{Unitality}: $M$ contains an identity element with respect to $\star$.
        \end{enumerate}
    }
    \newdef{Nilpotent}{\index{nilpotent}\label{algebra:nilpotent}
        An element $x$ of a monoid for which there exists an integer $k\in\mathbb{N}$ such that $x^k=e$, where $e$ is the identity element.
    }
    \begin{property}[Eckmann-Hilton argument]\index{Eckmann-Hilton!argument}\label{algebra:eckmann_hilton}
        Let $(M,\circ),(M,\otimes)$ be two monoid structures (or even unital magma structures) on a set $M$ such that
        \begin{gather}
            (a\circ b)\otimes(c\circ d) = (a\otimes c)\circ(b\otimes d)
        \end{gather}
        for all $a,b,c,d\in M$. The two monoid structures coincide and are in fact Abelian. (This property admits a vast generalization, see \ref{cat:eckmann_hilton}.)
    \end{property}

    \newdef{Group}{\index{group}
        \nomenclature[S_Grp]{$\textbf{Grp}$}{category of groups and group homomorphisms}
        A set $G$ equipped with a binary operation $\star$ such that the following axioms are satisfied:
        \begin{enumerate}
            \item\textbf{Closure}: $G$ is closed under $\star$.
            \item\textbf{Associativity}: $\star$ is asssociative.
            \item\textbf{Unitality}: $G$ has an identity element with respect to $\star$.
            \item\textbf{Inverses}: Every element in $G$ has an inverse with respect to $\star$.
        \end{enumerate}
    }
    \newnot{Identity element}{
        The identity element of a general group will be denoted by $e$. In certain cases, where this makes sense, the identity element will be denoted by 0 or 1 (additive and multiplicative conventions).
    }
    \newdef{Abelian group}{\index{commutativity}\index{Abelian group}
        \nomenclature[S_Ab]{$\text{Ab}$}{category of Abelian groups}
        Let $(G,\star)$ be a group. If $\star$ is commutative, then $G$ is said to be an Abelian or \textbf{commutative} group.
    }

    \newdef{Morphism}{\index{morphism!of groups}
        A group \textbf{(homo)morphism} $\Phi:(G,\star)\rightarrow(H,\cdot)$ is a function $f:G\rightarrow H$ such that
        \begin{gather}
            \Phi(g\star g') = \Phi(g)\cdot\Phi(g')
        \end{gather}
        for all $g,g'\in G$.
    }

    \newdef{Kernel}{\index{kernel}\label{group:kernel}
        The kernel of a group morphism $\Phi:G\rightarrow H$ is defined as the set
        \begin{gather}
            \ker(\Phi) := \{g\in G\mid\Phi(g)=e_H\}.
        \end{gather}
        This set carries a group structure induced by that on $G$.
    }

    \begin{theorem}[First isomorphism theorem]\index{isomorphism theorem}\label{group:first_isomorphism_theorem}
        Consider a group morphism $\Phi:G\rightarrow H$. If $\Phi$ is surjective, then $G/\ker(\Phi)\cong H$.
    \end{theorem}

\section{Group theory}\label{section:groups}

    \newdef{Order of a group}{\index{order!of a group}
        The number of elements in the group. It is denoted by $|G|$ or $\mathrm{ord}(G)$.
    }
    \newdef{Order of an element}{
        The order of an element $a\in G$ is the smallest integer $n\in\mathbb{N}$ such that
        \begin{gather}
            a^n = e,
        \end{gather}
        where $e$ is the identity element of $G$.
    }

    \newdef{Torsion group}{\index{torsion!group}\label{group:torsion_group}
        A group in which all elements have finite order. The torsion set $\mathrm{Tor}(G)$ of a group $G$ is the set of all elements $a\in G$ that have finite order. If $G$ is Abelian, $\mathrm{Tor}(G)$ is a subgroup.
    }

    \begin{theorem}[Lagrange]\index{Lagrange}
        Let $G$ be a finite group and let $H$ be a subgroup. Then $|H|$ is a divisor of $|G|$.
    \end{theorem}
    \begin{result}
        The order of any element $g\in G$ is a divisor of $|G|$.
    \end{result}

    \begin{construct}[Grothendieck completion]\index{Grothendieck!completion}\label{group:grothendieck_completion}
        Let $(A,\boxplus)$ be a commutative monoid. From this monoid one can construct an Abelian group $G(A)$, called the Grothendieck completion of $A$, as the quotient of $A\times A$ by the equivalence relation
        \begin{gather}
            (a_1,a'_1)\sim (a_2,a'_2) \iff\exists c\in A: a_1 \boxplus a'_2 \boxplus c = a'_1 \boxplus a_2 \boxplus c.
        \end{gather}
        The identity element, denoted by 0, is given by the equivalence class of $(0,0)$. By the definition of $G(A)$, this class contains all elements $\alpha\in\Delta_A$. In particular, $[(a,b)]$ is the additive inverse of $[(b,a)]$: $[(a,b)] + [(b,a)] = 0$.
    \end{construct}
    \begin{uproperty}
        Let $G(A)$ be the Grothendieck completion of $A$. Every monoid morphism $m:A\rightarrow B$, between an Abelian monoid and an Abelian group, factors uniquely through a group morphism $\varphi:G(A)\rightarrow B$.
    \end{uproperty}

    \begin{example}[Integers]
        The Grothendieck completion of the natural numbers is the additive group of integers $\mathbb{Z}$.
    \end{example}

\subsection{Cosets}\index{co-!set}

    \newdef{Coset}{\index{normal!subgroup}\index{invariant!subgroup|see{normal subgroup}}\label{group:coset}
        Let $G$ be a group and let $H$ be a subgroup. The left coset of $H$ with respect to $g\in G$ is defined as the set
        \begin{gather}
            gH := \{gh\mid h\in H\}.
        \end{gather}
        The right coset is defined analogously. If for all $g\in G$ the left and right cosets coincide, the subgroup $H$ is said to be a \textbf{normal subgroup}, \textbf{normal divisor} or \textbf{invariant subgroup}.
    }
    \begin{notation}
        The set of left (resp. right) cosets is denoted by $G/H$ (resp. $H\backslash G$).
    \end{notation}

    \newdef{Index}{\index{index!of a subgroup}\label{group:index}
        Consider a group $G$ and a subgroup $H$. The index $|G:H|$ of $H$ in $G$ is defined as the number of cosets oh $H$ in $G$. This can also be expressed as
        \begin{gather}
            |G| = |G:H||H|.
        \end{gather}
    }

    \newadef{Normal subgroup}{\index{normal!subgroup}\index{conjugacy class}\label{group:normal_subgroup}
        Let $G$ be a group and let $H$ be a subgroup. Consider the \textbf{conjugacy classes} $gHg^{-1}$ for all $g\in G$. If all classes coincide with $H$ itself, then $H$ is said to be a normal subgroup.
    }
    \begin{notation}
        If $N$ is a normal subgroup of $G$, this is often denoted by $N\vartriangleleft G$.
    \end{notation}

    \newdef{Quotient group}{\index{quotient!group}\label{group:quotient_group}
        Let $G$ be a group and let $N\vartriangleleft$ be a normal subgroup. The coset space $G/N$ can be turned into a group by equipping it with the product induced by that on $G$.
    }

    \newdef{Center}{\index{center}\label{group:center}
        The center of a group $G$ is defined as follows:
        \begin{gather}
            Z(G) := \{z\in G\mid\forall g\in G:zg = gz\}.
        \end{gather}
        This is a normal subgroup of $G$.
    }
    \newdef{Normalizer}{\index{normalizer}
        The normalizer of a subgroup $H\subset G$ is defined as follows:
        \begin{gather}
            N_G(H) := \{g\in G\mid\forall gHg^{-1}=H\}\equiv\{g\in G\mid[G,H]\subseteq H\}.
        \end{gather}
    }
    \newdef{Centralizer}{\index{centralizer}
        The centralizer of a subgroup $H\subset G$ is defined as follows:
        \begin{gather}
            C_G(H) := \{g\in G\mid\forall h\in H:ghg^{-1}=h\}\equiv\{g\in G\mid[G,H]=0\}.
        \end{gather}
        This group clearly satisfies $C_G(H)\triangleleft N_G(H)$.
    }

\subsection{Sylow theorems}\index{Sylow}

    \newdef{Sylow $p$-subgroup}{\index{Sylow!subgroup}
        \nomenclature[S_Syl]{$\mathrm{Syl}_p(G)$}{set of Sylow $p$-subgroups of a finite group $G$}
        Consider a finite group $G$. For every prime $p$, a Sylow $p$-subgroup of $G$ is defined as a maximal $p$-group in $G$, i.e. every element has order a power of $p$ and the subgroup is maximal with resect to this property. The set of all Sylow $p$-subgroups of $G$ is denoted by $\mathrm{Syl}_p(G)$.
    }

    \begin{theorem}[Sylow I]
        Consider a finite group $G$. For every prime factor $p$ of $|G|$ with multiplicity $n$, there exists a Sylow $p$-subgroup of order $p^n$.
    \end{theorem}
    \begin{theorem}[Sylow II]
        Consider a finite group $G$ and a prime factor $p$ of $|G|$. All Sylow $p$-subgroups are conjugate and, in particular, isomorphic.
    \end{theorem}
    \begin{theorem}[Sylow III]
        Consider a finite group $G$ together with a prime factor $p$ of $|G|$ of multiplicity $n$ such that $|G|=p^nm$ for some $m\in\mathbb{N}$ and write $n_p:=\mathrm{card}(\mathrm{Syl}_p(G))$.
        \begin{itemize}
            \item $m=|G:H|$ for every Sylow $p$-subgroup $H$ of $G$.
            \item $n_p|m$, i.e. $n_p$ divides $m$.
            \item $n_p=1\bmod p$.
            \item $n_p=|G:N_G(H)|$ for any Sylow $p$-subgroup $H$ of $G$.
        \end{itemize}
    \end{theorem}

\subsection{Abelian groups}

    \newdef{Commutator subgroup\footnotemark}{\index{commutator!subgroup}\index{derived!subgroup}
        \footnotetext{Also called the \textbf{derived subgroup}.}
        The commutator subgroup $[G,G]$ of $G$ is defined as the group generated by the \textbf{commutators} \[[g,h] := g^{-1}h^{-1}gh,\] where $g,h\in G$. This group is a normal subgroup of $G$.
    }

    \begin{property}[Abelianization]\index{Abelianization}
        The Abelianization $G/[G,G]$ is an Abelian group. A group $G$ is Abelian if and only if $[G,G]$ is trivial.
    \end{property}

    \begin{property}[Abelian quotients]
        A quotient group $G/H$ is Abelian if and only if $[G,G]\leq H$.
    \end{property}


\subsection{Symmetric groups}

    \newdef{Symmetric group}{\index{symmetric!group}\index{degree!of symmetric group}\label{group:permutation_group}
        \nomenclature[S_symgr]{$S_n$}{symmetric group of degree $n$}
        \nomenclature[S_symgrX]{$\Sym(X)$}{symmetric group on the set $X$}
        The symmetric group $S_n$ (on $n$ elements) is defined as the set of all permutations of $\{1,\ldots,n\}$. The number $n$ is called the \textbf{degree} of the symmetric group. The symmetric group $\Sym(X)$ on a finite set $X$ is defined similarly (by first numbering the elements and then acting by $S_n$)\footnote{Two such choices are related through conjugation by a unique permutation. The resulting groups are isomorphic.}.

        When including infinite sets the symmetric group $\Sym(X)$ is defined as the group of all bijections from $X$ to itself (the multiplication is given by function composition).
    }

    \begin{theorem}[Cayley]\index{Cayley}
        Every group is isomorphic to a subgroup of the permutation group $\Sym(G)$.
    \end{theorem}

    \newdef{Cycle}{\index{cycle}
        A $k$-cycle is a permutation of the form $(a_1\ a_2\ \ldots\ a_k)$ sending $a_i$ to $a_{i+1}$ (and $a_k$ to $a_1$). A \textbf{cycle decomposition} of an arbitrary permutation is the decomposition into a product of disjoint cycles.
    }
    \begin{property}[Cycles are cyclic]
        Let $\tau$ be a $k$-cycle. Then $\tau$ is $k$-cyclic (hence the name):
        \begin{gather}
            \tau^k = e.
        \end{gather}
    \end{property}
    \begin{example}
        Consider the set $\{1,2,3,4,5,6\}$. The permutation $\sigma:x\mapsto x+2\bmod6$ can be decomposed as $\sigma = (1\ 3\ 5)(2\ 4\ 6)$.
    \end{example}

    \newdef{Transposition}{\index{transposition}
        A permutation that exchanges two elements but leaves the other ones unchanged.
    }
    \begin{property}[Decomposition]
        Any permutation can be decomposed as a product of transpositions. A permutation is said to be \textbf{even} (resp. \textbf{odd}) if the number of transpositions in its decomposition is even (resp. odd). One can prove that the parity of a permutation is well-defined, i.e. it is independent of the choice of decomposition.
    \end{property}

    \newdef{Alternating group}{\index{group!alternating}
        The alternating group $A_n$ is the subgroup of $S_n$ containing all even permutations, i.e. those permutations that can be decomposed as an even number of transpositions.
    }

    \newdef{Shuffle}{\index{shuffle}\label{group:shuffle}
        A permutation $\sigma\in S_n$ is called a $(p,q)$-shuffle (where $p+q=n$) if there exist disjoint increasing sequences $I=\{i_1<\ldots<i_p\}$ and $J=\{j_1<\ldots<j_q\}$ such that
        \begin{gather}
            \sigma(x) =
            \begin{cases}
                k&x=i_k\\
                k+p&x=j_k.
            \end{cases}
        \end{gather}
        The name stems from the idea of dividing a deck of cards into two piles and interleaving them. This way the order in each pile is strictly preserved.

        An unshuffle $\tau\in S_n$ is defined as a permuation such that $\tau^{-1}$ is a shuffle, i.e. there exist disjoint increasing sequences $I=\{i_1<\ldots<i_p\}$ and $J=\{j_1<\ldots<j_q\}$ such
        \begin{gather}
            \tau(k) =
            \begin{cases}
                i_k&k\leq p\\
                j_{k-p}&k>p.
            \end{cases}
        \end{gather}
    }

\subsection{Group presentations}

    \newdef{Generator}{\index{generator}
        A set of elements $\{g_i\}_{i\in I}\subset G$ (where $I$ can be infinite) is said to generate $G$ if every element in $G$ can be written as a product of the elements $g_i$. These elements are then called generators of $G$.
    }

    \newdef{Relations}{\index{relation}
        Let $G$ be a group. If the product of a number of elements $g\in G$ is equal to the identity $e$, this product is said to be a relation on $G$.
    }
    \newdef{Complete set of relations}{
        Let $G$ be a group generated by a set of elements $S$ (note that this set does not have be a group itself) and let $R$ be a set of relations on $S$. If $G$ is uniquely (up to an isomorphism) determined by $S$ and $R$, the set of relations is said to be complete.
    }

    \newdef{Presentation}{\index{presentation}\label{group:presentation}
        Let $G$ be a group with generators $S$ and let $R$ be a complete set of relations. The pair $(S,R)$ is called a presentation of $G$.

        If $R$ is finite, $G$ is said to be \textbf{finitely related}, while if $S$ is finite, $G$ is said to be \textbf{finitely generated}. If both $S$ and $R$ are finite, $G$ is said to be \textbf{finitely presented}.

        It is clear that every group can have many different presentations and that it is (very) difficult to tell if two groups are isomorphic by just looking at their presentations.
    }
    \begin{notation}
        The presentation of a group $G$ is often denoted by $\langle S|R \rangle$, where $S$ is the set of generators and $R$ the set of relations.
    \end{notation}

\subsection{Direct products}

    \newdef{Direct product}{\index{direct product!of groups}\label{group:direct_product}
        Let $G,H$ be two groups. The direct product $G\times H$ is defined as the set-theoretic Cartesian product $G\times H$ equipped with a group operation $\cdot$ defined as
        \begin{gather}
            (g_1,h_1)\cdot(g_2,h_2) = (g_1g_2,h_1h_2),
        \end{gather}
        where the operations on the right-hand side are the group operations in $G$ and $H$. This definition can be generalized to any number of groups, even an infinite number of them if one the $n$-tuples are replaced by elements of the infinite Cartesian product

        If $g\in G_1\times\cdots\times G_n$ can be written as $(g_1,\ldots,g_n)$ for $g_i\in G_i$, the $g_i$ are called the \textbf{components} of $g$.
    }

    \newdef{Weak direct product}{\index{direct sum!of groups}
        Consider the direct product of groups. The subgroup consisting of all elements for which all components, except finitely many of them, are the identity, is called the weak direct product. In the case of Abelian groups this is often called the \textbf{direct sum}. For a finite number of groups, the direct product and direct sum coincide.
    }
    \begin{notation}
        The direct sum is often denoted by $\oplus$ (in accordance with the notation for vector spaces and other algebraic structures that will be introduced further on).
    \end{notation}

    \newdef{Inner semidirect product}{\index{semi-!direct product}\index{split!product}
        Let $G$ be a group, $H\leq G$ a subgroup and $N\vartriangleleft G$ a normal subgroup. $G$ is said to be the inner semidirect product of $H$ and $N$, denoted by $N\rtimes H$, if it satifies the following equivalent statements:
        \begin{itemize}
            \item $G = NH := \{nh\mid n\in N,h\in H\}$, where $N\cap H = \{e\}$.
            \item For every $g\in G$ there exist a unique $n\in N,h\in H$ such that $g=nh$.
            \item For every $g\in G$ there exist a unique $n\in N,h\in H$ such that $g=hn$.
            \item There exists a group morphism $\rho:G\rightarrow H$ that satisfies $\rho|_H = e$ and $\ker(\rho)=N$.
            \item The composition of the natural embedding $i:H\rightarrow G$ and the projection $\pi:G\rightarrow G/N$ gives an isomorphism between $H$ and $G/N$.
        \end{itemize}
        Whenever $G$ is isomorphic to $N\rtimes H$, it is said to \textbf{split} over $N$.
    }
    \begin{property}[Normal subgroups]
        If both $H$ and $N$ in the above definition are normal, the inner semidirect product coincides with the direct product. In particular this includes the case of direct products. For a finite number of groups the direct product is generated by the elements of the groups.

        If the subgroups $H$ and $N$ have presentations $\langle S_H|R_H \rangle$ and $\langle S_N|R_N \rangle$, the direct product is given by
        \begin{gather}
            \label{group:direct_product_presentation}
            H\times N = \langle S_H\cup S_N|R_H\cup R_N \cup R_C \rangle,
        \end{gather}
        where $R_C$ is the set of relations that enforce the commutativity of $H$ and $N$.
    \end{property}

    \newdef{Outer semidirect product}{
        Let $G,H$ be two groups and let $\varphi:H\rightarrow\Aut(G)$ be a group morphism. The outer semidirect product $G\rtimes_\varphi H$ is defined as the set-theoretic Cartesian product $G\times H$ equipped with a binary relation $\cdot$ such that
        \begin{gather}
            (g_1,h_1)\cdot(g_2,h_2) = (g_1\varphi(h_1)(g_2),h_1h_2).
        \end{gather}
        The structure $(G\rtimes_\varphi H,\cdot)$ forms a group.

        By noting that the set $N=\{(g,e_H)\mid g\in G\}$ is a normal subgroup isomorphic to $G$, and that the set $B=\{(e_G,h)\mid h\in H\}$ is a subgroup isomorphic to $H$, one can also construct the outer semidirect product $G\rtimes_\varphi H$ as the inner semidirect product $N\rtimes B$.
    }

    \begin{remark}[Direct products]
        The direct product of groups is a special case of the outer semidirect product where the group morphism is given by the trivial map $\varphi:h\mapsto e_G$.
    \end{remark}

    Semidirect products can even be generalized further:
    \newdef{Bicrossed product of groups}{\index{bicrossed product}\index{matched pair}\label{group:bicrossed_product}
        Consider a group $G$ with two subgroups $K,H\leq G$ such that every element $g\in G$ can be uniquely decomposed as a product of an element in $K$ and an element in $H$. This implies that for $k\in K,h\in H$ there exists a unique decomposition of $kh$ of the form \[kh = (k\cdot h)k^h\] where $k\cdot h\in H$ and $k^h\in K$.

        It can be checked that the associativity of the product implies that $-\cdot-:K\times H\rightarrow H$ defines a left action of $K$ on $H$ and that $-^-:K\times H\rightarrow K$ defines a right action of $H$ on $K$. Some other properties are obtained in the same way:
        \begin{itemize}
            \item $e^h = e$,
            \item $k\cdot e = e$,
            \item $(kk')^h = k^{k'\cdot h}k'^h$, and
            \item $k\cdot(hh') = (k\cdot h)(k^h\cdot h')$.
        \end{itemize}
        Any two groups having this structure are said to form a \textbf{matched pair} (of groups). Given a matched pair of groups, one can define the bicrossed product $H \bowtie K$ as follows:
        \begin{gather}
            (h,k)(h',k') = \big(h(k\cdot h),k^{h'}k'\big).
        \end{gather}
    }

\subsection{Free groups}

    \newdef{Free group}{\index{free!group}\index{word}\label{group:free_group}
        Consider a set $S$. The free group on $S$ is the group generated by \textbf{words} in $S$, i.e. finite sequences of elements in $S$.
    }
    The definition of a group presentation \ref{group:presentation} can now be restated:
    \newadef{Presentation}{\index{presentation}
        A group $G$ is said to have a presentation $\langle S|R \rangle$ if it is isomorphic to the quotient of the free group on $S$ by the normal subgroup generated by $R$.
    }

    \newdef{Free product}{
        Consider two groups $G,H$. The free product $G\ast H$ is defined as the set consisting of all words composed of elements in $G$ and $H$ together with the concatenation (and reduction\footnote{Two elements of the same group, written next to each other, are replaced by their product.}) as multiplication. Due to the reduction, every element in $G\ast H$ has a unique expression of the form $g_1h_1g_2h_2\cdots g_nh_n$.
    }
    \begin{remark}[Cardinality]
        For nontrivial groups the free product is always infinite.
    \end{remark}
    \newadef{Free product}{\index{free!product}
        The free product of two groups $G$ and $H$ can equivalently be defined as the free group on the set $G\cup H$. It follows that if $G,H$ have presentations $\langle S_G\mid R_G \rangle$ and $\langle S_H|R_H \rangle$ respectively, the free product is given by
        \begin{gather}
            G\ast H = \langle S_G\cup S_H|R_G\cup R_H \rangle.
        \end{gather}
        By comparing to \eqref{group:direct_product_presentation} it can be seen that the free product is a generalization of the direct product.
    }
    \newdef{Free product with amalgamation}{\index{amalgamation}
        Consider three groups $F,G,H$ together with two group morphisms $\phi:F\rightarrow G$ and $\psi:F\rightarrow H$. The free product with amalgamation $G\ast_F H$ is defined by adding the following set of relations to the presentation of the free product $G\ast H$:
        \begin{gather}
            \big\{\phi(f)\psi(f)^{-1} = e\,\big\vert\,f\in F\big\}.
        \end{gather}
        This is the same as saying that the free product with amalgamation can be constructed as
        \begin{gather}
            G\ast_F H = (G\ast H) / N_F,
        \end{gather}
        where $N_F$ is the normal subgroup generated by elements of the form $\phi(f)\psi(f)^{-1}$.
    }

    \newdef{Free Abelian group}{\index{basis}\index{rank!of a group}
        An Abelian group $G$ is said to be freely generated on the generators $\{g_i\}_{i\in I}$ if every element $g\in G$ can be uniquely written as a formal linear combination of the generators:
        \begin{gather}
            G = \left\{\sum_{i\in J}a_ig_i\,\middle\vert\,a_i\in\mathbb{Z}, J\subseteq I\text{ is finite}\right\}.
        \end{gather}
        The set of generators is called a \textbf{basis}\footnote{In analogy with the basis of a vector space.} of $G$. The number of elements in the basis is called the \textbf{rank} of $G$.
    }
    \begin{property}[Nielsen-Schreier]\index{Nielsen-Schreier}
        Every subgroup of a free group is free.
    \end{property}

    \begin{theorem}[Fundamental theorem of finitely generated groups]\index{fundamental theorem!of finitely generated groups}\index{torsion!group}\label{group:theorem:free_group}
        Every finitely generated Abelian group $G$ of rank $n$ can either be decomposed as a quotient of two free, finitely generated, Abelian groups
        \begin{gather}
            G = F/F'
        \end{gather}
        or as a direct sum of a free, finitely generated, Abelian group and a torsion group \ref{group:torsion_group}:
        \begin{gather}
            G = A\oplus T \qquad\text{with}\qquad T\equiv Z_{h_1}\oplus\cdots\oplus Z_{h_m}.
        \end{gather}
        In the second decomposition $A$ has rank $n-m$ and all $Z_{h_i}$ are cyclic groups of order $h_i$, where $h_i$ is the power a prime. The group $T$ is called the \textbf{torsion subgroup}.
    \end{theorem}
    \begin{property}[Uniqueness]
        The rank $n-m$ and the numbers $h_i$ from previous theorem are unique.
    \end{property}

\section{Group actions}\label{section:group_actions}

    \newdef{Group action}{\index{group!action}\label{group:group_action}
        Let $G$ be a group and let $X$ be a set. A map $\rho:G\times X \rightarrow X$ is called an action of $G$ on $X$ if it satisfies the following conditions for all $g,h\in G$ and $x\in X$:
        \begin{enumerate}
            \item \textbf{Identity}: $\rho(e,x) = x$.
            \item \textbf{Compatibility}: $\rho(gh,x) = \rho(g,\rho(h,x))$.
        \end{enumerate}
        The set $X$ is called a (left) \textbf{$G$-space} or \textbf{$G$-set}. Right $G$-spaces are defined a similar way.
    }
    \remark{Note that this definition already makes sense for monoids \ref{algebra:monoid}.}
    \begin{adefinition}\label{group:permutation_remark}
        A group action can equivalently be defined as a group morphism $\rho:G\rightarrow\Sym(X)$. It assigns a permutation of $X$ to every element $g\in G$. If the set $X$ is equipped with some extra algebraic structure, one should replace $\Sym(X)$ by $\Aut(X)$, i.e. the action of $G$ should respect the structure.
    \end{adefinition}
    \begin{notation}
        The action $\rho(g,x)$ is often denoted by $g\cdot x$ or even $gx$ if no confusion can arise.
    \end{notation}

    \newdef{Equivariant map}{\index{equivariant!map}\index{morphism!of $G$-sets}\label{group:equivariant}
        Let $X,Y$ be two $G$-spaces. An equivariant map between these sets is a function $f:X\rightarrow Y$ satisfying
        \begin{gather}
            g\cdot f(x) = f(g\cdot x),
        \end{gather}
        where, by abuse of notation, the symbol $\cdot$ represents the group actions on both $X$ and $Y$. An equivariant map is sometimes called a \textbf{$G$-map}.\footnote{$G$-spaces together with the $G$-maps constitute a category.}
    }

    \begin{example}[$G$-module]\index{module!over a group}\index{intertwiner}\label{group:module}
        An Abelian group $M$ equipped with a left group action $\varphi:G\rightarrow\Aut(M)$, i.e. an action that acts by group morphisms. Equivariant maps of $G$-modules are also called \textbf{intertwining maps} or \textbf{intertwiners}.
    \end{example}

    \newdef{Orbit}{\index{orbit}\label{group:orbit}
        The orbit of an element $x\in X$ with respect to the action a group $G$ is defined as the set
        \begin{gather}
            G\cdot x\equiv\{g\cdot x\mid g\in G\}.
        \end{gather}
        The relation \[p\sim q\iff\exists g\in G:p = g\cdot q\] induces an equivalence relation for which the equivalence classes coincide with the orbits of the $G$-action. The set of equivalence classes $X/\!\sim$, often denoted by $X/G$, is called the \textbf{orbit space}.
    }
    \newdef{Stabilizer}{\index{stabilizer}\index{iso-!tropy group}\index{little!group}
        The stabilizer group (also called \textbf{isotropy group} or \textbf{little group}) of an element $x\in X$ with respect to the action of a group $G$ is defined as the set
        \begin{gather}
            G_x := \{g\in G\mid g\cdot x = x\}.
        \end{gather}
        This is a subgroup of $G$ for all $x\in X$.
    }
    \begin{theorem}[Orbit-stabilizer theorem]
        Let $G$ be a group acting on a set $X$ and let $G_x$ be the stabilizer of some $x\in X$. The following relation holds:
        \begin{gather}
            |G\cdot x||G_x| = |G|.
        \end{gather}
    \end{theorem}

    \newdef{Free action}{\index{free}\label{group:free_action}
        A group action is said to be free if $g\cdot x=x$ implies $g=e$ for any $x\in X$. Equivalently, a group action is free if the stabilizer groups of all elements is trivial.
    }
    \newdef{Faithful action}{\index{faithful!action}\index{effective!action|see{faithful}}\label{group:faithful_action}
        A group action is said to be faithful or \textbf{effective} if the morphism $G\rightarrow\Aut(X)$ is injective. Alternatively, a group action is faithful if for every two group elements $g,h\in G$ there exists an element $x\in X$ such that $g\cdot x\neq h\cdot x$.
    }

    \newdef{Transitive action}{\index{transitive!action}\label{group:transitive}
        A group action is said to be transitive if for every two elements $x, y\in X$ there exists a group element $g\in G$ such that $g\cdot x = y$. Equivalently, a group action is transitive if there is only one orbit.
    }
    \begin{property}\label{group:transitive_action_property}
        Let $X$ be a set equipped with a transitive action of a group $G$. There exists a bijection $X\cong G/G_x$, where $G_x$ is the stabilizer of any element $x\in X$.\\
        \begin{proof}
            \begin{mdframed}[roundcorner=10pt, linecolor=blue, linewidth=1pt]
                Choose an element $x\in X$. The stabilizer of $x$ with respect to $G$ is the set \[S_x := \{g\in G\mid g\cdot x = x\}.\] Due to the transitivity of the group action one obtains that \[\forall x, y\in X: \exists h\in G: h\cdot x = y.\] So, for every $z\in X$ one can choose a group element $g_z$ such that $g_z\cdot x = z$. For all elements in the coset $g_zS_x = \{g_zs\in G\mid s\in S_x\}$ the following equality is satisfied: \[(g_zs)\cdot x = g_z\cdot (s\cdot x) = g_z\cdot x = z.\] This implies that the map $\Phi:G/S_x \rightarrow X$ is surjective.

                Now, one needs to prove that $\Phi$ is also injective. A proof by contradiction is given. Choose two distinct cosets $gS_x$ and $hS_x$. There exist two elements $G,H\in X$ such that $g\cdot x = G$ and $h\cdot x = H$. Assume that $G = H$. This means that
                \begin{align*}
                    &g\cdot x = h\cdot x\\
                    \iff&(h^{-1}g)\cdot x = x\\
                    \iff&h^{-1}g\in S_x\\
                    \iff&hS_x\ni h(h^{-1}g) = g.
                \end{align*}
                This would imply that $gS_x = hS_x$, which is in contradiction with the assumptions. It follows that $G\neq H$ and, hence, that $\Phi$ is injective.\qed
            \end{mdframed}
        \end{proof}
    \end{property}

    \newdef{Homogeneous space}{\index{homogeneous!space}
        A set equipped with a transitive group action.
    }
    \newdef{Principal homogenous space}{\index{torsor}\label{group:torsor}
        If the action of a group $G$ on a homogeneous space $X$ is also free, then $X$ is said to be a principal homogeneous space or \textbf{$G$-torsor}.
    }
    \begin{example}[Affine space]
        The $n$-dimensional affine space $\mathbb{A}^n$ is an $\mathbb{R}^n$-torsor.
    \end{example}

    \newdef{Crossed module}{\index{module!crossed}\index{Peiffer}\label{group:crossed_module}
        A crossed module is a quadruple $(G,H,t,\alpha)$ where:
        \begin{itemize}
            \item $G,H$ are two groups,
            \item $t$ is a group morphism $H\rightarrow G$, and
            \item $\alpha$ is a group morphism $G\rightarrow\Aut(H)$.
        \end{itemize}
        These data are required to satisfy two compatibility conditions:
        \begin{enumerate}
            \item\textbf{$G$-equivariance}:
            \begin{gather}
                t(\alpha(g)h) = gt(h)g^{-1}.
            \end{gather}
            \item\textbf{Peiffer identity}:
            \begin{gather}
                \alpha(t(h))h' = hh'h^{-1}.
            \end{gather}
        \end{enumerate}
    }

\section{Group cohomology}\index{cohomology!group}

    \newdef{Group cohomology}{\label{group:cohomology}
        Consider a group $G$ together with a $G$-module $A$. Define the $k^{th}$ \textbf{chain group} as
        \begin{gather}
            C^k(G;A) := \big\{\text{all set-theoretic functions from }G^k\text{ to }A\big\}.
        \end{gather}
        The coboundary map $d^k:C^k(G;A)\rightarrow C^{k+1}(G;A)$ is defined as follows:
        \begin{align}
            d^kf(g_1,\ldots,g_k,g_{k+1}) = g_1\cdot f(g_2,\ldots,&g_k,g_{k+1}) + (-1)^{k+1}f(g_1,\ldots,g_k)\nonumber\\ &+\sum_{i=1}^k(-1)^{i+1} f(g_1,\ldots,g_ig_{i+1},\ldots,g_k,g_{k+1}).
        \end{align}
        The cohomology groups are defined as the following quotient groups:
        \begin{gather}
            H^k(G;A) := \frac{\ker(d^k)}{\im(d^k)}.
        \end{gather}
    }

    ?? COMPLETE (ADD e.g. classification of extensions) ??

\section{Hochschild and cyclic cohomology}\index{cohomology!Hochschild}

    \newdef{Hochschild homology groups}{\label{algebra:hochschild_homology}
        Let $A$ be an associative algebra and consider an $A$-bimodule $M$. For all $n\geq0$ define
        \begin{gather}
            HC_n(A;M) := M\otimes A^{\otimes n}.
        \end{gather}
        The boundary operator $d:HC_n(A;M)\rightarrow HC_{n-1}(A;M)$ is defined as follows:
        \begin{align}
            d_n:m\otimes a_1\otimes\cdots\otimes a_n\mapsto ma_1\otimes\cdots\otimes a_n + \sum_{i=1}^{n-1}(-1)^im\otimes a_1&\otimes\cdots\otimes a_ia_{i+1}\otimes\cdots\otimes a_n\\
            &+ (-1)^na_1m\otimes\cdots\otimes a_n.\nonumber
        \end{align}
        The Hochschild homology is then given by the homology of this chain complex:
        \begin{gather}
            HH_\bullet(A;M) := \ker(d_n)/\im(d_{n+1}).
        \end{gather}
        This complex can be turned into a graded-commutative algebra by equipping it with the \textit{shuffle product}.
    }
    \begin{remark}[Cohomology]
        Hochschild cohomology of $A$ can be obtained by dualizing: $HC^n(A):=\hom(HC_n(A),K)$, where $K$ is the field over which $A$ is defined.
    \end{remark}

    \begin{theorem}[Hochschild-Kostant-Rosenberg]\index{Hochschild-Kostant-Rosenberg}
        Let $A=C^\infty(M)$ for $M$ a smooth manifold.
        \begin{gather}
            HH_\bullet(A)\cong\Omega^n(M).
        \end{gather}
    \end{theorem}

    \newdef{Cyclic homology}{\index{cohomology!cyclic}
        The cyclic chain complex $CC_\bullet(A)$ is the subcomplex of the Hochschild complex $HC_\bullet(A)$ obtained as the kernel of $1-\lambda$, where $\lambda$ is the permutation operator
        \begin{gather}
            \lambda:HC_n(A)\rightarrow HC_n(A):a_0\otimes\cdots\otimes a_n\mapsto(-1)^na_n\otimes a_0\otimes\cdots\otimes a_{n-1}.
        \end{gather}
        It can be shown that the Hochschild boundary operator commutes with the permutation operator and, hence, descends to the cyclic complex. The resulting homology is called cyclic homology $CH_\bullet(A)$. By dualizing one obtains cyclic cohomology.
    }

    \begin{example}[Smooth algebras]
        Consider the case of $A=C^\infty(M)$, where $M$ is a smooth manifold. Then
        \begin{gather}
            CH_n(A)\cong Z^n_\text{dR}(M)\oplus\bigoplus_{\substack{i=1\\n-2i\geq0}}H^{n-2i}_\text{dR}(M).
        \end{gather}
        For this reason cyclic cohomology will be used in noncommutative geometry to generalize de Rham cohomology.
    \end{example}
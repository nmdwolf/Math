\chapter{Operator Theory}\label{chapter:operator_algebras}

    The main reference for this chapter is~\citet{blackadar_operator_2013}.

    \minitoc

\section{Operators}
\subsection{Operator topologies}\index{operator!topology}

    \newdef{Weak operator topology}{
        The topology generated by the seminorms $\bigl\{T\mapsto|\lambda(Tv)|\bigm\vert v\in V,\lambda\in V^*\bigr\}$. A net of linear operators $\net{T}$ on a space $V$ converges to a linear operator $T$ in the weak (operator) topology if $T_\alpha v\longrightarrow Tv$ for all $v\in V$ in the weak topology.

        In the case of Hilbert spaces, one can simplify the above definition using Riesz's representation theorem (\cref{functional:riesz}). The weak operator topology on a Hilbert space is generated by the seminorms $\bigl\{T\mapsto|\langle Tv\mid w \rangle|\bigm\vert v,w\in\mathcal{H}\bigr\}$.
    }

    \newdef{Strong operator topology}{
        The topology generated by the seminorms $\bigl\{T\rightarrow\|Tv\|\bigm\vert v\in V\bigr\}$. A net of linear operators $\seq{T}$ on a space $V$ converges to a linear operator $T$ in the strong (operator) topology if $T_\alpha v\longrightarrow Tv$ for all $v\in V$ in the norm (or strong) topology.
    }

    \newdef{Operator norm}{\index{norm!operator}\label{functional:operator_norm}
        The operator norm of $L$ is defined as follows:
        \begin{gather}
            \|T\|_{\text{op}} = \inf\bigl\{\lambda\in\mathbb{R}\bigm\vert\forall v\in V:\|Tv\|_W \leq \lambda\|v\|_V\bigr\}\,.
        \end{gather}
        Equivalent definitions of the operator norm are:
        \begin{gather}
            \|T\|_{\text{op}} = \sup_{\|v\|\leq1}\|Tv\| = \sup_{\|v\|=1}\|Tv\| = \sup_{v\neq0}\frac{\|Tv\|}{\|v\|}\,.
        \end{gather}
    }

    \newdef{Norm topology\footnotemark}{\index{topology!norm}
        \footnotetext{Also called the \textbf{uniform (operator) topology}.}
        A sequence of linear operators $\seq{T}$ on a space $V$ converges to a linear operator $T$ in the norm topology if the sequence $(\|T_n-T\|)_{n\in\mathbb{N}}$ converges to 0. (Sequences suffice since the norm topology is metrizable and, therefore, sequential by \cref{topology:first_countable_sequential}.)
    }

\subsection{Bounded operators}

    \newdef{Bounded operator}{\index{bounded!operator}\label{functional:bounded_operator}
        Let $T:V\rightarrow W$ be a linear operator between two normed spaces. The linear operator is said to be bounded if it satisfies
        \begin{gather}
            \|T\|_{\text{op}}<+\infty\,.
        \end{gather}
    }
    \begin{notation}
        The space of bounded linear operators from $V$ to $W$ is denoted by $\mathcal{B}(V,W)$.
    \end{notation}
    \begin{property}
        If $V$ is a Banach space, then $\mathcal{B}(V)$ is also a Banach space.
    \end{property}

    The following property reduces the problem of continuity to that of boundedness (or vice versa).
    \begin{property}\label{functional:bounded_continuous}
        Consider a linear operator $T\in\mathcal{L}(V, W)$. The following statements are equivalent:
        \begin{itemize}
            \item $T$ is bounded.
            \item $T$ is continuous at 0.
            \item $T$ is continuous on $V$.
            \item $T$ is uniformly continuous on $V$.
            \item $T$ maps bounded sets to bounded sets.
        \end{itemize}
    \end{property}

    \begin{property}[Bounded eigenvalues]
        The eigenvalues of a bounded operator are bounded by its operator norm. Furthermore, every bounded linear operator on a Banach space has at least one eigenvalue.
    \end{property}

    \begin{property}[BLT theorem\footnotemark]
        \footnotetext{BLT stands for 'bounded linear transformation'.}
        Consider a bounded linear operator $f:X\rightarrow W$, where $X$ is a dense subset of a normed space $V$ and $W$ is a Banach space. There exists a unique extension $F:V\rightarrow W$ such that $\|f\|_{\text{op}}=\|F\|_{\text{op}}$.
    \end{property}

    \newdef{Schatten class operator}{\index{Schatten class}\label{functional:schatten_class}
        Consider the space of bounded linear operators on a Hilbert space $\mathcal{H}$. The \textbf{Schatten p-norm}, for $p\in[1,+\infty[$, is defined as
        \begin{gather}
            \|T\|_p := \tr\left(\sqrt{T^*T}^{\,p}\right)^{1/p}\,.
        \end{gather}
        Linear operators for which this norm is finite constitute the $p^{\text{th}}$ Schatten class $\mathcal{I}_p$.
    }
    \begin{property}
        The Schatten classes are Banach spaces with respect to the associated Schatten norms.
    \end{property}

    \begin{example}[Trace-class operator]\index{trace!class}\label{functional:trace_class}
        The space of trace-class operators on a Hilbert space $\mathcal{H}$ is defined as follows:
        \begin{gather}
            \mathcal{B}_1(\mathcal{H}) := \{T\in\mathcal{B}(\mathcal{H})\mid\tr(|T|)<+\infty\}\,,
        \end{gather}
        where the trace functional was defined in \cref{functional:trace} and $|T|:=\sqrt{T^*T}$.
    \end{example}
    The following theorem can be seen as the analogue of Riesz's theorem for trace-class operators.
    \begin{property}
        For every bounded linear functional $\rho$ on the space of trace-class operators $\mathcal{B}_1(\mathcal{H})$, there exists a unique bounded linear operator $T\in\mathcal{B}(\mathcal{H})$ such that
        \begin{gather}
            \rho(S) = \tr(ST)
        \end{gather}
        for all $S\in\mathcal{B}_1(\mathcal{H})$. This implies that $\mathcal{B}^*_1(\mathcal{H})$ and $\mathcal{B}(\mathcal{H})$ are isometrically equivalent.
    \end{property}

    The previous property allows for the following definition.
    \newdef{Weak-* operator topology}{\index{topology!$\sigma$-weak}\index{topology!operator}\label{functional:sigma_weak_topology}
        The weak-* (also \textbf{ultraweak} or \textbf{$\sigma$-weak}) topology on $\mathcal{B}(\mathcal{H})$ with respect to the trace-class operators $\mathcal{B}_1(\mathcal{H})$. This is also called the $\sigma$-weak topology on $\mathcal{B}(\mathcal{H})$.
    }

    \begin{example}[Hilbert--Schmidt operator]\index{Hilbert--Schmidt!operator}\label{functional:hilbert_schmidt}
        Consider the Hilbert--Schmidt norm $\|\cdot\|_2$ from \cref{linalgebra:hilbert_schmidt_norm}. A linear operator $T\in\mathcal{B}(\mathcal{H})$ is said to be a Hilbert--Schmidt operator if it satisfies
        \begin{gather}
            \|T\|_2<+\infty\,.
        \end{gather}
        This space is closed under taking adjoints.
    \end{example}

    A more general, but still well-behaved, class of linear operators is the space of closed operators.
    \newdef{Closed operator}{\index{closed!operator}
        A linear operator $T:V\rightarrow W$ such that for every sequence $\seq{v}$ in $\dom(T)$ converging to $v\in V$, where $f(v_n)$ converges to $w\in W$, one finds that $v\in\dom(T)$ and $Tv=w$.

        Equivalently, one can define a closed linear operator as a linear operator for which its graph is a closed subset in the direct sum $V\oplus W$.
    }
    \newdef{Closure}{\index{closure}\label{functional:closure}
        Let $T:V\rightarrow W$ be a linear operator. Its closure (if it exists) is the closed linear operator $\overline{T}$ such that the graph of $\overline{T}$ is the closure of the graph of $T$ in $V\oplus W$.
    }

    \begin{theorem}[Closed graph theorem]\index{closed!graph theorem}
        A linear operator on a Banach space is closed if and only if it is bounded.
    \end{theorem}

\subsection{Self-adjoint operators}

    There is a multitude of different notions available in the literature that try to indicate in what sense a linear operator is related to its adjoint (not everyone agrees on the definitions). Here, an overview is given in the case of Hilbert spaces where linear operators are allowed to be unbounded.

    \Cref{linalgebra:adjoint_operator} for finite-dimensional spaces can be generalized as follows.
    \newdef{Adjoint}{\index{adjoint!Hermitian}
        Let $T$ be a linear operator on a Hilbert space $\mathcal{H}$. A linear operator $T^\dag$ is said to be the (Hermitian) adjoint of $T$ if the following conditions are satisfied:
        \begin{enumerate}
            \item $\langle v\mid Tw \rangle = \langle T^\dag v\mid w \rangle$ for all $v\in\dom(T^\dag)$ and $w\in\dom(T)$.
            \item Every other linear operator satisfying this property is a restriction of $T^\dag$ (i.e.~the domain of $T^*$ is maximal with respect to the above property).
        \end{enumerate}
    }
    \begin{property}
        Let $T$ be a bounded linear operator. Its adjoint is also bounded and $\|T\|_{\text{op}} = \|T^\dag\|_{\text{op}}$.
    \end{property}

    \newdef{Symmetric operator}{\index{symmetric!operator}
        A linear operator $T$ on a Hilbert space $\mathcal{H}$ such that $\dom(T)\subseteq\dom(T^\dag)$ and $T=T^\dag|_{\dom(T)}$.
    }
    \newdef{Self-adjoint operator}{\index{self-adjoint}
        A linear operator $T$ on a Hilbert space $\mathcal{H}$ such that $\dom(T)$ is dense in $\mathcal{H}$ and $T=T^\dag$.
    }

    The notion of Hermitian operator is the one where almost nobody agrees upon its definition. Here, the definition from~\citet{schreiber_nlab_2008} is chosen.
    \newdef{Hermitian operator}{\index{Hermitian!operator}\label{functional:hermitian}
        A bounded, symmetric operator.
    }

    \begin{theorem}[Hellinger--Toeplitz]\index{Hellinger--Toeplitz}
        A self-adjoint operator on a Hilbert space $\mathcal{H}$ is bounded if and only if its domain is all of $\mathcal{H}$.
    \end{theorem}

    \begin{theorem}[Stone]\index{Stone}\label{functional:stone}
        Consider a strongly continuous unitary one-parameter group, i.e.~a family of unitary operators $U:\mathbb{R}\rightarrow\mathrm{U}(\mathcal{H})$ such that
        \begin{itemize}
            \item $U$ is continuous in the strong operator topology:
            \begin{gather}
                \lim_{t\rightarrow t_0}U(t)v=U(t_0)v
            \end{gather}
            for all $t_0\in\mathbb{R},v\in\mathcal{H}$.
            \item $U$ is a \textit{one-parameter group} in the sense of \cref{lie:one_parameter_subgroup}.
        \end{itemize}
        There exists a self-adjoint operator $T$ such that $U(t)=e^{itT}$. Furthermore, the linear operator $T$ is bounded if and only if $U$ is continuous in the norm topology.
    \end{theorem}
    \newdef{Generator}{\index{generator}
        The linear operator $T$ in the preceding theorem is called the (infinitesimal) generator of the family $U$. It can be obtained through a formal derivative:
        \begin{gather}
            T = \left.\deriv{U(t)}{t}\right|_{t=0}\,.
        \end{gather}
    }

\subsection{Compact operators}

    \newdef{Compact operator}{\index{compact!operator}\label{functional:compact_operator}
        Let $V,W$ be Banach spaces. A linear operator $T:V\rightarrow W$ is said to be compact if the image of any bounded set in $V$ is relatively compact (\cref{topology:relatively_compact}).
    }

    \newadef{Compact operator}{
        Let $V,W$ be Banach spaces. A linear operator $T:V\rightarrow W$ is compact if, for every bounded sequence $\seq{v}$ in $V$, the sequence $\seq{Tv}\subset W$ has a convergent subsequence.
    }

    \begin{notation}
        The space of compact, bounded linear operators between Banach spaces $V,W$ is denoted by $\mathcal{B}_0(V,W)$. If $V=W$, this is abbreviated as $\mathcal{B}_0(V)$ as usual.
    \end{notation}
    \begin{property}
        $\mathcal{B}_0(V)$ is a two-sided ideal in the (Banach) algebra $\mathcal{B}(V)$. Moreover, $\mathcal{B}_0(V)=\mathcal{B}_1^*(V)$.
    \end{property}

    \begin{property}[Finite-rank operators]
        All finite-rank operators are compact. In fact, the space of compact operators is the norm closure of that of finite-rank operators.
    \end{property}

    \begin{property}
        Every compact operator is bounded.
    \end{property}
    \begin{result}
        Every linear map between finite-dimensional Banach spaces is bounded.
    \end{result}

    \begin{property}
        If $T$ is a compact self-adjoint operator on a Hilbert space, then $-\|T\|$ or $\|T\|$ are an eigenvalue of $T$. Furthermore, the set of nonzero eigenvalues is either finite or converges to 0.
    \end{property}

    \newdef{Calkin algebra}{\index{Calkin algebra}
        Consider the algebra $\mathcal{B}(V)$ of bounded linear operators on $V$ together with its two-sided ideal $\mathcal{B}_0(V)$ of compact operators. The quotient algebra $\mathcal{Q}(V)=\mathcal{B}(V)/\mathcal{B}_0(V)$ is called the Calkin algebra of $V$.
    }

    \newdef{Fredholm operator}{\index{Fredholm!operator}\label{functional:fredholm}
        A bounded linear operator for which the kernel and cokernel are finite-dimensional. The space of Fredholm operators on a space $V$ is denoted by $\mathfrak{F}(V)$.
    }

    By the following theorem, one can characterize Fredholm operators using the Calkin algebra.
    \begin{property}[Atkinson]\index{Atkinson}\index{parametrix}\label{functional:atkinson}
        A bounded linear operator $T:V\rightarrow W$ is a Fredholm operator if and only if it is invertible in the Calkin algebra, i.e.~there exists a bounded linear operator $S:W\rightarrow V$ and compact operators $C_1,C_2$ such that $\mathbbm{1}_W-TS=C_1$ and $\mathbbm{1}_V-ST=C_2$. $S$ is called the \textbf{parametrix} of $T$.
    \end{property}

    \newdef{Fredholm index}{\index{index!Fredholm}
        The index of a Fredholm operator $F$ is defined as follows:
        \begin{gather}
            \mathrm{ind}(F) := \dim\ker(F)-\dim\mathrm{coker}(F)\,.
        \end{gather}
    }

    \begin{property}
        The induced function
        \begin{gather}
            \mathrm{ind}:\pi_0\bigl(\mathfrak{F}(\mathcal{H})\bigr)\rightarrow\mathbb{Z}
        \end{gather}
        is a group isomorphism:
        \begin{itemize}
            \item $\mathrm{ind}(F^*)=-\mathrm{ind}(F)$, and
            \item $\mathrm{ind}(FG) = \mathrm{ind}(F)+\mathrm{ind}(G)$.
        \end{itemize}
        This theorem is generalized in \textit{$K$-theory} (see \cref{chapter:k}) by the \textit{Atiyah--J\"anich theorem}~\ref{k:atiyah_janich}.
    \end{property}

\subsection{Spectrum}\label{section:spectrum}

    \newdef{Resolvent operator}{\index{resolvent}
        Consider an operator $T\in\mathcal{B}(V)$ on a normed space $V$. The resolvent operator $T_\lambda$ for some $\lambda\in\mathbb{C}$ is defined as the linear operator $(T-\lambda\mathbbm{1}_V)^{-1}$.
    }

    \newdef{Resolvent set}{
        The resolvent set $\rho(T)$ consists of all scalars $\lambda\in\mathbb{C}$ for which the resolvent operator of A is a bounded linear operator on a dense subset of $V$. These scalars $\lambda$ are called \textbf{regular values} of $T$.
    }
    \newdef{Spectrum}{\index{spectrum}\label{functional:spectrum}
        The set of scalars $\mu\in\mathbb{C}\backslash\rho(T)$ is called the spectrum $\sigma(T)$.
    }

    From \cref{linalgebra:eigenvalue_remark}, it is clear that every eigenvalue of $T$ belongs to the spectrum of $T$. The converse, however, is not true. This is remedied by introducing the following concepts.
    \newdef{Point spectrum}{
        The set of scalars $\mu\in\mathbb{C}$ for which $T-\mu\mathbbm{1}_V$ fails to be injective is called the point spectrum $\sigma_p(T)$. This set coincides with the set of eigenvalues of $T$.
    }
    \newdef{Continuous spectrum}{
        The set of scalars $\mu\in\mathbb{C}$ for which $T-\mu\mathbbm{1}_V$ is injective with dense image but fails to be surjective is called the continuous spectrum of $T$.
    }
    \newdef{Residual spectrum}{
        The set of scalars $\mu\in\mathbb{C}$ for which $T-\mu\mathbbm{1}_V$ is injective but fails to have a dense image is called the residual spectrum $\sigma(T)$.
    }

    \newdef{Essential spectrum}{\index{essential!spectrum}
        The set of scalars $\mu\in\mathbb{C}$ for which $T-\mu\mathbbm{1}_V$ is not a Fredholm operator is called the essential spectrum $\sigma_{\text{ess}}(T)$.
    }

    From Atkinson's theorem~\ref{functional:atkinson}, one can derive the following result.\footnote{In fact, one could (equivalently) define the essential spectrum in terms of the Calkin algebra using Atkinson's theorem. Then, this property would be an obvious consequence.}
    \begin{property}
        Let $T$ be a bounded linear operator and let $C$ be a compact operator. The essential spectra of $T$ and $T+C$ coincide.
    \end{property}

    \begin{property}[Bounded spectrum]
        A self-adjoint operator is bounded if and only if its spectrum is bounded. Furthermore, it is positive if and only if its spectrum lies in $\mathbb{R}^+$.
    \end{property}

\subsection{Spectral theorem \& functional calculus}\label{section:PVM}

    This section focuses on the algebra of bounded operators $\mathcal{B}(\mathcal{H})$ on a (complex) Hilbert space $\mathcal{H}$.

    \begin{property}[Closed subspaces]
        There exists a bijection between the set of closed subspaces of $\mathcal{H}$ and the set of orthogonal projections in $\mathcal{B}(\mathcal{H})$. Furthermore, if the projection $p$ corresponds to a subspace $\mathcal{H}_p$, the projection $\mathbbm{1}_{\mathcal{H}}-p$ corresponds to the orthogonal complement $\mathcal{H}_p^\perp$.
    \end{property}

    \newdef{Projection-valued measure}{\index{measure!projection-valued}\index{spectral!measure}\label{operators:spectral_measure}
        Consider a measurable space $(X,\Sigma)$. A projection-valued measure (PVM) or \textbf{spectral measure}\footnote{Sometimes also called a \textbf{resolution of the identity}.} on $X$ is a map $P:\Sigma\rightarrow\mathcal{B}(\mathcal{H})$ satisfying the following conditions:\footnote{The third property can in fact be shown to follow from the others.}
        \begin{enumerate}
            \item $P_E$ is a projection for all $E\in\Sigma$,
            \item $P_X=\mathbbm{1}_{\mathcal{H}}$,
            \item $P_AP_B = P_{A\cap B}$, and
            \item for all disjoint $\seq{E}\subset\Sigma$:
                \begin{gather}
                    \sum_{n=0}^{+\infty}P_{E_n} = P_{\cup_{n\in\mathbb{N}}E_n}\,.
                \end{gather}
        \end{enumerate}
    }
    \begin{property}
        Let $P$ be a spectral measure on $(X,\Sigma)$. For every two elements $v,w\in\mathcal{H}$, the map
        \begin{gather}
            E\mapsto\mu^P_{v,w}(E):=\langle v\mid P_Ew \rangle
        \end{gather}
        defines a (complex) measure $\mu^P_{v,w}$ on $X$. The square of the norm of an element $v\in\mathcal{H}$ is then simply given by $\mu^P_{v,v}(X)$ due to the second condition above.
    \end{property}

    \begin{property}
        Let $f:X\rightarrow\mathbb{C}$ be a measurable function on a measurable space $(X,\Sigma)$. Given a spectral measure $P$, one defines $\Delta_f$ to be the set of all $v\in\mathcal{H}$ for which $f\in L^2(X,\mu^P_{v,v})$. This set is dense in $\mathcal{H}$. Moreover, the map
        \begin{gather}
            \Int_Xf(\lambda)\,dP(\lambda):\Delta_f\rightarrow\mathcal{H}
        \end{gather}
        defined by
        \begin{gather}
            \left\langle v\,\middle\vert\,\Int_Xf(\lambda)\,dP(\lambda)w\right\rangle := \Int_Xf(\lambda)\,d\mu^P_{v,w}(\lambda)
        \end{gather}
        is closed and normal. It also satisfies the following two equalities:
        \begin{align}
            \left(\Int_Xf(\lambda)\,dP(\lambda)\right)^\dag &= \Int_X\overline{f(\lambda)}\,dP(\lambda)\,,\\
            \left\|\Int_Xf(\lambda)\,dP(\lambda)v\right\|^2 &= \Int_X|f(\lambda)|^2\,d\mu^P_{v,v}(\lambda)\,.
        \end{align}
        If $f$ is bounded, the above operator is bounded by the supremum norm of $f$:
        \begin{gather}
            \left\|\Int_Xf(\lambda)\,dP(\lambda)\right\|\leq\|f\|_\infty\,.
        \end{gather}
    \end{property}

    \begin{theorem}[Spectral theorem]\index{spectral!theorem}\label{operators:spectral_theorem}
        Let $A$ be a self-adjoint operator on a Hilbert space $\mathcal{H}$. There exists a unique spectral measure $P_A:\Sigma_{\mathbb{R}}\rightarrow\mathcal{B}(\mathcal{H})$ on the Borel $\sigma$-algebra of the real line such that
        \begin{gather}
            A = \Int_{\mathbb{R}}\lambda\,dP_A(\lambda)\,.
        \end{gather}
        If $\mathbb{R}$ is replaced by $\mathbb{C}$, this theorem also holds for normal operators. If $A$ is compact, the measure becomes purely atomic, i.e.~there exists a countable orthonormal basis of $\mathcal{H}$ consisting of eigenvectors of $A$ such that $\lambda_i\longrightarrow0$:
        \begin{gather}
            A = \sum_{i=1}^{+\infty}\lambda_ie_i^*\otimes e_i\,.
        \end{gather}
    \end{theorem}
    \begin{property}[Spectrum and support]
        The spectrum of a self-adjoint operator $A$ coincides with the support of its associated spectral measure $P_A$. A number $\lambda\in\mathbb{R}$ belongs to the point spectrum of $A$ if and only if $P_A$ does not vanish on $\{\lambda\}$. A number $\lambda\in\mathbb{R}$ belongs to the continuous spectrum of $A$ if $P_A$ vanishes on $\{\lambda\}$ but is nonvanishing on any open set containing $\lambda$.
    \end{property}

    \newdef{Singular value}{\index{singular!value}
        Let $A$ be a compact operator. The singular values of $A$ are given by the square roots of the eigenvalues of the self-adjoint (and compact) operator $A^*A$. If $A$ is self-adjoint, its singular values and eigenvalues coincide.
    }

    \newdef{Measurable functional calculus}{\index{functional!calculus}
        Let $(X,\Sigma)$ be a measurable space and let $\mathcal{H}$ be a Hilbert space. A measurable functional calculus $(\Phi,\mathcal{H})$ on $(X,\Sigma)$ is an assignment
        \begin{gather}
            \Phi:\mathbf{Meas}(X,\mathbb{C})\rightarrow\mathcal{B}_0(\mathcal{H})
        \end{gather}
        satisfying the following conditions:
        \begin{enumerate}
            \item\textbf{Unitality}: $\Phi(1)=\mathbbm{1}_{\mathcal{H}}$.
            \item\textbf{Sublinearity}: $\Phi(\lambda f + g)\subseteq\lambda\Phi(f)+\Phi(g)$ and the equality holds if either operator is bounded.
            \item\textbf{Submultiplicativity}: $\Phi(f)\Phi(g)\subseteq\Phi(fg)$ and the equality holds if either operator is bounded. Moreover, the product is commutative if either operator is bounded.
            \item\textbf{Density}:\footnote{This condition is actually redundant.} $\Phi(f)$ is densely defined.
            \item\textbf{Involutivity}: $\Phi\bigl(\overline{f}\bigr)=\overline{\Phi(f)}$.
            \item\textbf{Boundedness}: $\Phi(f)$ is bounded if $f$ is bounded.
            \item\textbf{Convergence}: If the bounded sequence $f_n\longrightarrow f$ pointwise, then $\Phi(f_n)\longrightarrow\Phi(f)$ strongly.
        \end{enumerate}
    }

    \begin{property}
        If $(\Phi,\mathcal{H})$ is a measurable functional calculus on $(X,\Sigma)$, then
        \begin{gather}
            P_\Phi:\Sigma\rightarrow\mathcal{B}(\mathcal{H}):E\mapsto\Phi(\mathbbm{1}_E)
        \end{gather}
        is a projection-valued measure. Conversely, every projection-valued measure $P$ gives rise to a measurable functional calculus $\Phi$ such that $P=P_\Phi$.
    \end{property}

    \newdef{Borel functional calculus}{\index{Borel!functional calculus}
        A functional calculus on $(X,\Sigma)$ where $X$ is a topological space and $\Sigma$ is its Borel $\sigma$-algebra. If $X\subseteq\mathbb{C}$ and the identity function $\mathbbm{1}_{\mathbb{C}}:z\mapsto z$ is measurable, $(\Phi,\mathcal{H})$ is called a Borel functional calculus for the operator $A$ if $\Phi(\mathbbm{1}_{\mathbb{C}})=A$.
    }

    The above properties allow to compose self-adjoint operators with (measurable) functions similar to how one can compute $f(X)$ for finite-dimensional operators by applying $f$ to the eigenvalues of $X$.
    \begin{formula}[Borel functional calculus]
        Consider a bounded operator $A\in\mathcal{B}(\mathcal{H})$. Let $f:\sigma(A)\rightarrow\mathbb{C}$ be a measurable function (with respect to the restriction of the Borel algebra on $\mathbb{R}$) and let $g:\mathbb{R}\rightarrow\mathbb{C}$ be any other measurable function that coincides with $f$ on $\sigma(A)$.
        \begin{gather}
            f(A) := \Int_{\sigma(A)}f(\lambda)\,dP_A(\lambda) = \Int_{\mathbb{R}}g(\lambda)\,dP_A(\lambda) =: g(A)\,.
        \end{gather}
    \end{formula}

    The Spectral Theorem~\ref{operators:spectral_theorem} can also be stated in terms of functional calculus.
    \begin{theorem}[Spectral theorem]\index{spectral!theorem}
        A normal operator $A\in\mathcal{B}(\mathcal{H})$ on a Hilbert space $\mathcal{H}$ induces a unique Borel functional calculus on $\sigma(A)$.
    \end{theorem}

\subsection{Dixmier trace}

    \newdef{Dixmier ideal}{\index{Dixmier!ideal}
        For every compact operator, one can sort the singular values in decreasing order. The Dixmier ideal is defined as follows:
        \begin{gather}
            \mathfrak{D}(\mathcal{H}):=\bigl\{A\in\End_0(\mathcal{H})\bigm\vert\sigma_n(A)=O(\ln(n))\bigr\}\,,
        \end{gather}
        where
        \begin{gather}
            \sigma_n(A) := \sum_{i=1}^ns_i(A)\,.
        \end{gather}
    }
    The following functional gives an alternative for the ordinary trace on operators that are not trace-class.
    \newdef{Dixmier trace}{\index{Dixmier!trace}\index{measurable!operator}\label{operators:dixmier_trace}
        For every element in $A\in\mathfrak{D}(\mathcal{H})$, the Dixmier trace is defined as follows:
        \begin{gather}
            \tr_{\mathfrak{D}}(A) := \omega\left(\left\{\frac{\sigma_n(A)}{1+\ln(n)}\,\middle\vert\,n\in\mathbb{N}\right\}\right),
        \end{gather}
        where $\omega:\ell^\infty(\mathbb{N})\rightarrow\mathbb{R}$ is a linear functional satisfying the following conditions:
        \begin{enumerate}
            \item\textbf{Positivity}: $\omega(s)\geq0$ if $s_n\geq0$ for all $n\in\mathbb{N}$.
            \item\textbf{Convergence}: $\omega(s)=\lim_{n\rightarrow\infty}s_n$ if $s$ is convergent.
            \item\textbf{Dilation-invariant}: Let $D:\ell^\infty(\mathbb{N})\rightarrow\ell^\infty(\mathbb{N})$ be the following operator:
            \begin{gather}
                D:(s_1,s_2,\ldots)\mapsto(s_1,s_1,s_2,s_2,\ldots).
            \end{gather}
            The state $\omega$ is dilation-invariant if $\omega(s)=\omega\bigl(D(s)\bigr)$ for all $s\in\ell^\infty(\mathbb{N})$.
        \end{enumerate}
        If for an operator $A\in\mathfrak{D}(\mathcal{H})$ the Dixmier trace is independent of the chosen state, it is said to be \textbf{measurable}.
    }
    \begin{property}\index{singular!trace}
        The Dixmier trace has the following properties:
        \begin{itemize}
            \item $\tr_{\mathfrak{D}}(A+B)=\tr_{\mathfrak{D}}(A) + \tr_{\mathfrak{D}}(B)$ for all $A,B\in\mathfrak{D}(\mathcal{H})$.
            \item If $A$ is bounded and $B$ is an element of the Dixmier ideal, then $\tr_{\mathfrak{D}}(AB)=\tr_{\mathfrak{D}}(BA)$.
            \item The Dixmier ideal vanishes on trace-class operators (in particular, it is \textbf{singular}, i.e.~it vanishes on finite-rank operators):
                \begin{gather}
                    A\in\mathcal{B}_1(\mathcal{H})\implies\tr_{\mathfrak{D}}(A)=0\,.
                \end{gather}
        \end{itemize}
    \end{property}

\section{\texorpdfstring{$C^*$-}{C-star }algebras}\label{section:c_star_algebras}
\subsection{Involutive algebras}

    \newdef{Involutive algebra}{\index{algebra!involutive}\index{$*$-algebra|see{algebra, involutive}}
        An involutive algebra is an associative algebra $A$ over a commutative involutive ring $(R,\overline{\,\cdot\,\vphantom{a}}\,)$ together with an algebra involution $\cdot^*:A\rightarrow A$ such that:
        \begin{enumerate}
            \item $(a+b)^* = a^*+b^*$,
            \item $(ab)^* = b^*a^*$, and
            \item $(\lambda a)^* = \overline\lambda a^*$,
        \end{enumerate}
        for all $a,b\in A$ and $\lambda\in R$. These algebras are also sometimes called \textbf{$\ast$-algebras}.
    }

    \newdef{Isometry}{\index{isometry}
        An element $a$ of a $*$-algebra $A$ satisfying
        \begin{gather}
            a^*a = 1\,,
        \end{gather}
        where $1$ is the unit of $A$.
    }
    \newdef{Normal element}{\index{normal!element}
        An element of a $\ast$-algebra that commutes with its adjoint:
        \begin{gather}
            a^*a = aa^*\,.
        \end{gather}
    }

    \begin{property}
        Every element in a $\ast$-algebra can be decomposed as the sum of two normal elements:
        \begin{gather}
            a = \frac{1}{2}\bigl((a+a^*) + (a-a^*)\bigr)\,.
        \end{gather}
        This implies that a linear operator defined on normal elements extends uniquely to the whole algebra.
    \end{property}

    \newdef{Projection}{\index{projection}\label{operators:projection}
        An element $p$ of a $\ast$-algebra such that
        \begin{gather}
            p = p^2 = p^*\,.
        \end{gather}
        This terminology reflects the property that in an algebra of bounded operators on a Hilbert space, the projections are exactly the operators associated to a orthogonal projections (cf.~\cref{section:orthogonal_projections}).
    }

    \newdef{$C^*$-algebra}{\index{C$^*$-algebra}\label{operators:c_star}
        A $C^*$-algebra is an involutive Banach algebra $A$ (\cref{functional:banach_space}) such that the \textbf{$C^*$-identity}
        \begin{gather}
            \|a^*a\| = \|a\|\,\|a^*\|
        \end{gather}
        is satisfied for all $a\in A$.
    }

    The Artin--Wedderburn theorem~\ref{algebra:artin_wedderburn} implies the following decomposition.
    \begin{theorem}
        Let $A$ be a finite-dimensional $C^*$-algebra over the field $\mathfrak{K}$. There exist unique integers $N$ and $d_1,\ldots,d_N$ such that
        \begin{gather}
            A\cong\bigoplus_{i=1}^NM_{d_i}(\mathfrak{K})\,.
        \end{gather}
    \end{theorem}
    This implies that every $C^*$-algebra can be represented using block matrices.

    \begin{example}[Bounded operators]\label{operators:bounded_operators}
        Let $\mathcal{H}$ be a finite-dimensional Hilbert space. The space of bounded operators $\mathcal{B}(\mathcal{H})$ is a $C^*$-algebra.
    \end{example}

    \begin{property}
        Every norm-closed $\ast$-subalgebra of a $C^*$-algebra is a $C^*$-algebra.
    \end{property}

    \newdef{$H^*$-algebra}{\index{H-!$\ast$-algebra}
        A Hilbert space $\mathcal{H}$ equipped with a unital $\ast$-algebra structure such that $\ast$-adjoint and multiplicative adjoints coincide:
        \begin{enumerate}
            \item $\langle ab\mid c\rangle = \langle b\mid a^*c\rangle$, and
            \item $\langle ab\mid c\rangle = \langle a\mid cb^*\rangle$,
        \end{enumerate}
        for all $a,b,c\in\mathcal{H}$.
    }
    \begin{example}[Linear operators]\index{Hilbert--Schmidt!norm}\label{operators:hilbert_schmidt_inner_product}
        The canonical example of $H^*$-algebras is given by the algebra of linear operators on a Hilbert space $\mathcal{H}$, where the involution is given by taking adjoints and the inner product is the Hilbert--Schmidt inner product induced by the norm (\cref{linalgebra:hilbert_schmidt_norm}) (up to a factor $k>0$):
        \begin{gather}
            \langle f\mid g\rangle_{\text{HS}} := k\,\tr(f^*g)\,.
        \end{gather}
        The resulting space is denoted by $L^2(\mathcal{H},k)$. A result, analogous to the Artin--Wedderburn theorem~\ref{algebra:artin_wedderburn}, states that every $H^*$-algebra can be decomposed as an orthogonal direct sum of finitely many algebras of the form $L^2(\mathcal{H}_i,k_i)$.
    \end{example}

\subsection{Positive maps}

    \newdef{Positive element}{\index{positive}
        A self-adjoint element of a $C^*$-algebra whose spectrum is contained in $[0,+\infty[$. The cone of all positive elements in $A$ is often denoted by $A^+$.
    }
    \begin{property}[Positive decomposition]
        An element $a\in A$ is positive if and only if $a=b^*b$ for some element $b\in A$. A normal element is positive if and only if it is self-adjoint.
    \end{property}

    \newdef{Positive map}{\label{operators:positive_map}
        A morphism $T:A\rightarrow B$ of $C^*$-algebras such that $T(A^+)\subseteq B^+$.
    }
    \newdef{Completely positive map}{\index{positive!completely}\label{operators:cp_map}
        A morphism $T:A\rightarrow B$ of $C^*$-algebras such that the following map is positive for all $k\in\mathbb{N}$:
        \begin{gather}
            \mathbbm{1}_k\otimes T:\mathbb{C}^{k\times k}\otimes A\rightarrow\mathbb{C}^{k\times k}\otimes B\,.
        \end{gather}
        If $T$ satisfies this condition only up to an integer $n\in\mathbb{N}$, it is said to be \textbf{$n$-positive}.
    }

    \newdef{State}{\index{state}\label{operators:state}
        A positive linear functional of unit norm on a $C^*$-algebra.
    }
    \begin{property}[Convexity]\index{state!pure}
        The set of states is a convex set. The extreme points of this set are called \textbf{pure states}, all other elements are called \textbf{mixed states}.
    \end{property}

    \newdef{Positivity-improving map}{
        A positive map $T$ that satisfies
        \begin{gather}
            a\geq0\implies T(a)>0\,.
        \end{gather}
    }
    \newdef{Ergodic map}{\index{ergodic}
        A positive map $T$ that satisfies
        \begin{gather}
            \forall a>0:\exists t_a\in\mathbb{R}_0:\exp(t_aT)\,a>0\,.
        \end{gather}
    }

    The following theorem can be seen as a Jordan algebra-theoretic analogue of the Gel'fand--Naimark theorem~\ref{operators:gelfand_naimark}.
    \begin{theorem}[Alfsen--Shultz]\index{Alfsen--Shultz}
        The state spaces of two $C^*$-algebras are isomorphic if and only if the algebras are isomorphic as (special) Jordan algebras (\cref{linalgebra:jordan_algebra}).
    \end{theorem}

    \newdef{Cuntz algebra}{\index{Cuntz algebra}\index{isometry}
        The $n^{\text{th}}$ Cuntz algebra $\mathcal{O}_n$ is defined as the (universal) unital $C^*$-algebra generated by $n\in\mathbb{N}$ isometries $s_i$ under the additional relation
        \begin{gather}
            \sum_{i=1}^ns_is_i^* = 1\,,
        \end{gather}
        where $1$ is the unit element.
    }

\subsection{Traces}

    \newdef{Trace}{\index{trace}
        Let $A$ be a $C^*$-algebra. A trace on $A$ is a linear functional that satisfies the following conditions:
        \begin{enumerate}
            \item\textbf{Positivity}: $\tr(A^+)\geq 0$, and
            \item\textbf{Tracial}: $\tr(ab)=\tr(ba)$ for all $a,b\in A$.
        \end{enumerate}
    }
    \remark{The tracial property could have been replaced by the following equivalent, but seemingly weaker, condition: $\tr(a^*a)=\tr(aa^*)$ for all $a\in A$.}

    \newdef{Adjoint map}{\index{adjoint!operator}
        Assume that a trace functional $\tr$ is given on a $C^*$-algebra and consider a continuous linear operator $T$ defined on the Schatten class $\mathcal{I}_p$ (\cref{functional:schatten_class}). One can define the adjoint map $T^*$ on $\mathcal{I}_q$ whenever $p,q$ are H\"older conjugate (\cref{measure:holders_inequality}). This adjoint is given by the following equation:
        \begin{gather}
            \tr\bigl((T^*a)b\bigr) = \tr\bigl(a^*Tb\bigr)\,,
        \end{gather}
        where $a\in\mathcal{I}_q$ and $b\in\mathcal{I}_p$.
    }
    \newdef{Trace-preserving map}{\label{operators:trace_preserving}
        A linear operator $T:A\rightarrow B$ is said to be trace-preserving if it satisfies
        \begin{gather}
            \tr_B(Ta) = \tr_A(a)
        \end{gather}
        for all trace-class elements $a\in A$. Using the above definition, it is easy to see that, on a unital $C^*$-algebra, this is equivalent to
        \begin{gather}
            T^*1 = 1\,,
        \end{gather}
        i.e.~to its adjoint being unital.
    }

    \begin{property}
        A completely positive, trace-preserving map $T$ satisfies:
        \begin{gather}
            \|T\|_1 = 1\,,
        \end{gather}
        where the subscript $1$ indicates that this operator is defined on trace-class elements.
    \end{property}

    \begin{property}[States]\index{state}\label{operators:tracial_state}
        Whenever a $C^*$-algebra is commutative, every state defines a trace and, after a suitable normalization, every trace defines a state. However, for noncommutative $C^*$-algebras, only the latter implication holds.
    \end{property}

\subsection{Representations}

    \newdef{$C^*$-representation}{\index{representation!of a $C^*$-algebra}
        A representation of a $C^*$-algebra $A$ on a Hilbert space $\mathcal{H}$ is a unital $\ast$-morphism $\pi:A\rightarrow\mathcal{B}(\mathcal{H})$.
    }

    \newdef{Normal state}{\index{state!normal}\label{operators:normal_state}
        Consider a Hilbert space $\mathcal{H}$ and a $C^*$-representation $\pi:A\rightarrow\mathcal{B}(\mathcal{H})$. A normal state $\omega$ on $A$ is a state such that there exists a trace-class operator $\rho\in\mathcal{B}_1(\mathcal{H})$ with the following property:
        \begin{gather}
            \omega(a) = \frac{\tr\bigl(\rho\pi(a)\bigr)}{\tr(\rho)}\,.
        \end{gather}
        When $A=\mathcal{B}(\mathcal{H})$, the normal states are exactly the $\sigma$-weakly continuous states (\cref{functional:sigma_weak_topology}).
    }

    \newdef{Folium}{\index{folium}
        Let $\pi:A\rightarrow\mathcal{B}(\mathcal{H})$ be a $C^*$-representation. The space of normal states of this representation is called its folium.
    }

    \newdef{Cyclic vector}{\index{cyclic!vector}
        A cyclic vector for a $C^*$-algebra representation $\pi:A\rightarrow\mathcal{B}(\mathcal{H})$ is a vector $\xi\in\mathcal{H}$ such that $\{\pi(a)\xi\mid a\in A\}$ is (norm-)dense in $\mathcal{H}$.
    }

    The injective tensor product of Banach spaces admits an equivalent construction in the case of $C^*$-algebras.
    \newdef{Spatial tensor product}{\index{tensor product!spatial}\label{operator:spatial_tensor_product}
        Let $A,B$ be two $C^*$-algebras with faithful representations $\pi_A:A\rightarrow\mathcal{B}(\mathcal{H}_1)$ and $\pi_B:B\rightarrow\mathcal{B}(\mathcal{H}_2)$. The spatial tensor product $A\otimes_{\text{sp}}B$ is the norm closure of the algebraic tensor product $A\otimes B$. It can be shown that this definition does not depend on the choice of representation and, moreover, coincides with the injective tensor product of the underlying Banach spaces as induced by \cref{functional:injective_banach_product}.
    }

\subsection{Gel'fand duality}\index{Gel'fand!duality}

    \newdef{Gel'fand spectrum}{\index{Gel'fand!spectrum}\index{character}\label{operators:gelfand_spectrum}
        Consider a commutative $C^*$-algebra $A$ (in fact any involutive algebra suffices). Its set of \textbf{characters}, i.e.~the algebra morphisms $A\rightarrow\mathbb{C}$, can be equipped with a locally compact Hausdorff topology: the weak-* topology (\cref{functional:weak_star_topology}).
    }
    \begin{property}
        The Gel'fand spectrum of a $C^*$-algebra is compact if and only if the algebra is unital.
    \end{property}

    \newdef{Gel'fand representation}{\index{Gel'fand!representation}
        Consider a $C^*$-algebra $A$ and let $\Phi_A$ denote its Gel'fand spectrum. The \textbf{Gel'fand transformation} of an element $a\in A$ is defined as the morphism $\widehat{a}:\Phi_A\rightarrow\mathbb{C}$ given by the following formula:
        \begin{gather}
            \widehat{a}(\lambda) = \langle\lambda,a\rangle\,,
        \end{gather}
        where $\langle\cdot,\cdot\rangle$ denotes the pairing between $A$ and $\Phi_A$. By definition of the Gel'fand spectrum, the functional $\widehat{a}$ is continuous for all $a\in A$. The mapping $a\mapsto\widehat{a}$ is called the Gel'fand representation of $A$.
    }

    \begin{theorem}[Gel'fand--Naimark: Commutative case]\index{Gel'fand--Naimark}
        Let $A$ be a commutative $C^*$-algebra. The Gel'fand representation gives an isometric $\ast$-isomorphism between $A$ and the set $C_0(\Phi_A)$ of continuous complex-valued functions that vanish at infinity (\cref{distributions:riesz_markov}) on its Gel'fand spectrum.
    \end{theorem}
    \begin{remark}
        In fact, the Gel'fand--Naimark theorem gives an equivalence between the category of commutative nonunital $C^*$-algebras and the category of locally compact Hausdorff spaces (with continuous functions that vanish at infinity).
    \end{remark}

    \begin{property}[Connectedness]
        A compact Hausdorff space is connected if and only if its algebra of continuous functions has no nontrivial projections.
    \end{property}
    \begin{property}[Metrizability]
        A commutative $C^*$-algebra is separable if and only if its Gel'fand spectrum is metrizable.
    \end{property}

    \begin{formula}[Spectrum]
        Recall \cref{functional:spectrum} of the spectrum of an operator. This is related to the spectrum of a unital, commutative $C^*$-algebra as follows:
        \begin{gather}
            \sigma(a) = \{a(x)\mid x\in\Phi_A\}\,,
        \end{gather}
        for all $a\in A$.
    \end{formula}

    \begin{construct}[Gel'fand--Naimark--Segal]\index{Gel'fand--Naimark--Segal construction}\label{operators:gns}
        Let $A$ be a $C^*$-algebra. Given a state $\omega$ on $A$, there exists a $C^*$-representation $\pi:A\rightarrow\mathcal{B}(D)$, where $D\subset\mathcal{H}$ is a dense subspace of a Hilbert space $\mathcal{H}$, such that the following conditions are satisfied:
        \begin{itemize}
            \item There exists a cyclic (unit) vector $\xi\in D$.
            \item For all elements $a\in A$, the following equality holds:
                \begin{gather}
                    \omega(a) = \langle\pi(a)\xi\mid\xi\rangle\,.
                \end{gather}
        \end{itemize}

        \todo{COMPLETE CONSTRUCTION}
    \end{construct}

    \begin{theorem}[Gel'fand--Naimark: General case]\index{Gel'fand--Naimark}\label{operators:gelfand_naimark}
        Every $C^*$-algebra is isometrically $\ast$-isomorphic to a norm closed ($C^*$-)algebra of bounded operators on a Hilbert space $\mathcal{H}$.
    \end{theorem}

    The GNS construction can be generalized as follows.
    \begin{theorem}[Stinespring\footnotemark]\index{Stinespring}\label{operators:stinespring}
        Consider a linear operator $T:A\rightarrow\mathcal{B}(\mathcal{H})$ between $C^*$-algebras. It is completely positive if and only if there exists a $C^*$-representation $\pi:A\rightarrow\mathcal{B}(\mathcal{K})$ and a bounded operator $V:\mathcal{H}\rightarrow\mathcal{K}$ such that
        \begin{gather}
            Ta = V^\dag\pi(a)V
        \end{gather}
        or, equivalently,
        \begin{gather}
            T^\dagger b = \pi^\dagger(VbV^\dagger)\,.
        \end{gather}
        Moreover, $V$ is an isometry if and only if $T$ is unital and, hence, if $T^\dagger$ is trace-preserving by \cref{operators:trace_preserving}. The triple $(\mathcal{K},\pi,V)$ is often called a \textbf{Stinespring representation} of $T$.
        \footnotetext{Sometimes called the \textbf{Stinespring dilation theorem} or \textbf{Stinespring factorization theorem}.}
    \end{theorem}
    \begin{remark}
        Because the adjoint of a completely positive map is again completely positive, the above two characterizations can be used interchangeably. The latter is mostly used by mathematicians, whereas the former is better known in the physics literature.

        The most common situation is where $\mathcal{H}:=\mathcal{H}_1=\mathcal{H}_2$ and
        \begin{gather}
            \pi:\mathcal{B}(\mathcal{H})\rightarrow\mathcal{B}(\mathcal{H}\otimes\mathcal{K}):a\mapsto a\otimes b
        \end{gather}
        for some density operator $b\in\mathcal{B}(\mathcal{K})$. The adjoint to tensoring by a density operator is taking the partial trace $\tr_{\mathcal{K}}$. This way, one obtains the following expressions:
        \begin{gather}
            Ta = V^\dag(a\otimes b)V
        \end{gather}
        and, equivalently,
        \begin{gather}
            T^\dagger b = \tr_{\mathcal{K}}(VbV^\dagger)\,.
        \end{gather}
    \end{remark}

    \begin{result}[GNS construction]
        The GNS construction follows from the Stinespring theorem by taking $T$ to be a state (hence $\mathcal{H}=\mathbb{C}$).
    \end{result}

    \begin{result}[Choi]\index{Choi}\index{Kraus decomposition}\label{operators:kraus}
        Let $\mathcal{H}_1,\mathcal{H}_2$ be two finite-dimensional Hilbert spaces of dimensions $m,n\in\mathbb{N}$, respectively. A linear map $\Phi:\mathcal{B}(\mathcal{H}_1)\rightarrow\mathcal{B}(\mathcal{H}_2)$ is completely positive if and only if it can be expressed as
        \begin{gather}
            \Phi(a) = \sum_{i=1}^{mn}V^\dagger_iaV_i
        \end{gather}
        for some bounded operators $V_i:\mathcal{H}_2\rightarrow\mathcal{H}_1$. Furthermore, it is trace-preserving if and only if
        \begin{gather}
            \sum_{i=1}^{mn}A^\dag_iA_i = \mathbbm{1}\,.
        \end{gather}
        This decomposition is also often called an \textbf{operator-sum decomposition} and the operators $\{V_i\}_{i\leq mn}$ are called \textbf{Kraus operators} of $\Phi$.
    \end{result}

\subsection{\difficult{Hilbert modules}}

    \newdef{Hilbert $C^*$-module}{\index{Hilbert!module}
        Let $A$ be a $C^*$-algebra and $H$ a vector space. $H$ is a (right) pre-Hilbert $A$-module if there exists a map
        \begin{gather}
            \langle\cdot\mid\cdot\rangle_A:H\times H\rightarrow A\,,
        \end{gather}
        sometimes called the \textbf{$A$-inner product}, satisfying the following conditions:
        \begin{enumerate}
            \item\textbf{Conjugate linearity}: For all $x,y,z\in H$ and $\lambda\in\mathbb{C}$:
                \begin{align}
                    \langle x\mid \lambda y+z\rangle_A &= \langle x\mid y\rangle_A+\overline{\lambda}\langle x\mid z\rangle_A\,,\\
                    \langle x\mid y\rangle_A &= \langle y\mid x\rangle_A^*\,.
                \end{align}
            \item\textbf{Nondegeneracy}: For all $x\in H$:
                \begin{align}
                    &\langle x\mid x\rangle_A \geq 0\,,\\
                    &\langle x\mid x\rangle_A = 0 \iff x=0\,.
                \end{align}
            \item\textbf{Equivariance}: For all $x,y\in H$ and $a\in A$:
                \begin{gather}
                    \langle x\mid y\rangle_A\,a = \langle x\mid y\cdot a\rangle_A\,.
                \end{gather}
        \end{enumerate}
        A left module is obtained by requiring linearity and (left-)equivariance in the first argument. The completion of a pre-Hilbert module with respect to the norm $x\mapsto\sqrt{\|\langle x|x \rangle_A\|}$ is called a \textbf{Hilbert $A$-module}.
    }
    \newdef{Hilbert $C^*$-bimodule}{
        Let $A,B$ be $C^*$-algebras and consider a Hilbert $C^*$-module $H$ over $B$. $H$ is called a Hilbert $(A,B)$-bimodule if it admits a left $\ast$-representation of $A$ such that
        \begin{gather}
            \langle a^*\cdot x\mid y\rangle_B = \langle x\mid a\cdot y\rangle_A
        \end{gather}
        for all $a\in A$, i.e.~adjoints in $A$ correspond to adjoints with respect to the $B$-inner product.
    }

    \newdef{Fredholm module}{\index{Fredholm!module}\label{operators:fredholm_module}
        Let $\pi:A\rightarrow\End(\mathcal{H})$ be a $\ast$-representation of a unital $C^*$-algebra on a Hilbert space $\mathcal{H}$. When equipped with an operator $F\in\End(\mathcal{H})$, this gives an (\textbf{odd}) Fredholm module if
        \begin{gather}
            F=F^*\qquad\text{and}\qquad F^2=\mathbbm{1}_{\mathcal{H}}\qquad\text{and}\qquad[F,\pi(A)]\subseteq\mathcal{B}_0(\mathcal{H})\,.
        \end{gather}
        The Fredholm operator is said to be \textbf{even} if $\mathcal{H}$ is a super-Hilbert space with grading $\Gamma\in\End(\mathcal{H})$ satisyfing
        \begin{gather}
            \Gamma^*=\Gamma\qquad\text{and}\qquad\Gamma^2=\mathbbm{1}_{\mathcal{H}}\qquad\text{and}\qquad[\Gamma,\pi(A)]=0\qquad\text{and}\qquad\{\Gamma,F\}_+=0\,.
        \end{gather}
        $F$ has the form
        \begin{gather}
            \begin{pmatrix}
                0&F_+\\F_-&0
            \end{pmatrix}\,,
        \end{gather}
        where $F_+$ is Fredholm.
    }

    \begin{example}
        Consider $A=\mathbb{C}$ and let $\pi:\mathbb{C}\rightarrow\End(\mathcal{H})$ be the unique unital representation. Fredholm modules with this data correspond to (essentially self-adjoint) Fredholm operators on $\mathcal{H}$.
    \end{example}

    \begin{remark}[Kasparov $K$-theory]
        If $A$ is not unital, one should relax the first two relations up to compact operators, i.e.~$F=F^*$ and $F^2=\mathbbm{1}_{\mathcal{H}}$ in the Calkin algebra. This gives a clear relation to \textit{Kasparov's $K$-theory} (see below), where the homology complex $K\!K_\bullet$ is defined in terms of such generalized Fredholm modules. However, it can be shown that these induce the same $K\!K$-classes.
    \end{remark}

    \newdef{Kasparov bimodule}{\index{Kasparov bimodule}
        Let $A,B$ be two $C^*$-algebras and let $H$ be a super-Hilbert $(A,B)$-bimodule. $H$ is called a Kasparov $(A,B)$-bimodule if there exists an odd adjointable operator $F\in\mathcal{B}(H)$ satisfying the following properties for all $a\in A$:
        \begin{enumerate}
            \item $(F^2-\mathbbm{1}_H)\pi_A(a)$ is compact,
            \item $(F-F^*)\pi_A(a)$ is compact, and
            \item $[F,\pi_A(a)]$ is compact.
        \end{enumerate}
        If $A$ is unital (and the representation respects units), this implies that $F$ is a projection in the Calkin algebra. If $F$ is a projection on the nose, the bimodule is said to be \textbf{normalized}.
    }
    \newdef{Kasparov $K$-theory}{\index{K-theory!Kasparov}\label{operators:kk_theory}
        The set, in fact Abelian group, of homotopy classes of Kasparov bimodules $K\!K(A,B)$, where a homotopy between two Kasparov $(A,B)$-bimodules is a Kasparov $(A,C([0,1],B))$-bimodule that restricts to the given bimodules on the boundaries of the interval.
    }
    \begin{property}
        Every Kasparov bimodule is homotopic to a normalized bimodule. Hence, for Kasparov $K$-theory, one can restrict to normalized bimodules.
    \end{property}

\section{von Neumann algebras}

    \newdef{von Neumann algebra}{\index{von Neumann!algebra}
        A $*$-subalgebra of a $C^*$-algebra equal to its double commutant (\cref{algebra:centralizer}): $M=M''$.
    }
    \newdef{Concrete von Neumann algebra}{\index{concrete|seealso{von Neumann algebra}}
        A weakly closed unital $*$-algebra of bounded operators on some Hilbert space.
    }
    \begin{theorem}[Double Commutant theorem\footnotemark]
        \footnotetext{Often called \textbf{von Neumann's double commutant theorem}.}
        The above definitions are equivalent.
    \end{theorem}

    \begin{property}[Projections]
        Every von Neumann algebra is generated by its projections.
    \end{property}
    \begin{property}[Orthomodular lattices]
        The set of projections in a von Neumann algebra or, equivalently, the set of closed subspaces of a Hilbert space forms a complete orthomodular lattice (\cref{set:complemented_lattice}).

        Given two closed subspaces, the join and meet are constructed as follows:
        \begin{itemize}
            \item Meet: $A\land B=A\cap B$, and
            \item Join: $A\lor B=$ smallest closed subspace containing $A\cup B$.
        \end{itemize}
    \end{property}

    \newdef{Murray--von Neumann equivalence}{\index{equivalence!Murray--von Neumann}
        Closed subspaces of a Hilbert space are said to be Murray--von Neumann equivalent if they are isomorphic through a partial isometry. In terms of projections, this means that $p\sim q$ if and only if there exists a partial isometry $u$ such that $p=uu^*$ and $q=u^*u$.
    }

    \newprop{Partial order}{\index{projection!finite}
        The inclusion of subspaces induces a partial order on the set of projections. A projection $p$ is said to be finite if there exists no smaller projection $q$ that is equivalent to $p$.
    }

\subsection{Factors}

    \newdef{Factor}{\index{factor}
        Consider a von Neumann algebra $M$. A $*$-subalgebra $A\subseteq M$ is called a factor of $M$ if its center $Z(A)$ is given by the scalar multiples of the identity.
    }

    \newdef{Type-$\mathrm{I}$ factor}{
        A factor that contains a minimal projection.
    }
    \begin{property}[Type-$\mathrm{I}_n$ factors]
        Any type $\mathrm{I}$-factor is isomorphic to the algebra of all bounded operators on a Hilbert space. To indicate the dimension $n\in\overline{\mathbb{N}}$ of this Hilbert space (which may be $+\infty$), one sometimes uses the subclassification of type $\mathrm{I}_n$-factors.
    \end{property}

    \newdef{Powers index}{\index{index!Powers}
        Consider a Hilbert space $\mathcal{H}$ together with its von Neumann algebra of bounded operators $\mathcal{B}(\mathcal{H})$. A unital $*$-endomorphism $\alpha$ is said to have Powers index $n\in\mathbb{N}$ if the space $\alpha(\mathcal{B}(\mathcal{H}))$ is isomorphic to a type $\mathrm{I}_n$-factor.
    }

    \newdef{Type-$\mathrm{II}$ factor}{
        A factor that contains nonzero finite projections but no minimal ones. If the identity is finite, the factor is sometimes said to be of type $\mathrm{II}_1$, otherwise it is said to be of type $\mathrm{II}_\infty$.
    }

    \newdef{Type-$\mathrm{III}$ factor}{
        A factor that does not contain any nonzero finite projections.
    }
\chapter{Operator Theory}\label{chapter:operator_algebras}

    The main reference for this chapter is~\citet{blackadar_operator_2013}.

\section{\texorpdfstring{$C^*$-}{C-star }algebras}
\subsection{Involutive algebras}

    \newdef{Involutive algebra}{\index{algebra!involutive}\index{$*$-algebra|see{algebra, involutive}}
        An involutive algebra is an associative algebra $A$ over a commutative involutive ring $(R,\overline{\,\cdot\,}\,)$ together with an algebra involution $\cdot^*:A\rightarrow A$ such that:
        \begin{enumerate}
            \item $(a+b)^* = a^*+b^*$,
            \item $(ab)^* = b^*a^*$, and
            \item $(\lambda a)^* = \overline\lambda a^*$,
        \end{enumerate}
        for all $a,b\in A$ and $\lambda\in R$. These algebras are also sometimes called \textbf{$\ast$-algebras}.
    }

    \newdef{Normal element}{\index{normal!element}
        An element of a $\ast$-algebra that commutes with its adjoint:
        \begin{gather}
            a^*a = aa^*\,.
        \end{gather}
    }
    \begin{property}
        Every element in a $\ast$-algebra can be decomposed as the sum of two normal elements:
        \begin{gather}
            a = \frac{1}{2}\bigl((a+a^*) + (a-a^*)\bigr)\,.
        \end{gather}
        This implies that a linear morphism defined on normal elements extends uniquely to the whole algebra.
    \end{property}

    \newdef{$C^*$-algebra}{\index{C$^*$-algebra}\label{operators:c_star}
        A $C^*$-algebra is an involutive Banach algebra (\cref{functional:banach_space}) such that the \textbf{$C^*$-identity}
        \begin{gather}
            \|a^*a\| = \|a\|\|a^*\|
        \end{gather}
        is satisfied for all $a\in A$.
    }

    The Artin--Wedderburn theorem~\ref{algebra:artin_wedderburn} implies the following decomposition.
    \begin{theorem}
        Let $A$ be a finite-dimensional $C^*$-algebra. There exist unique integers $N$ and $d_1,\ldots,d_N$ such that
        \begin{gather}
            A\cong\bigoplus_{i=1}^NM_{d_i}(K)\,.
        \end{gather}
    \end{theorem}
    This implies that every $C^*$-algebra can be represented using block matrices.

    \begin{example}[Bounded operators]
        Let $\mathcal{H}$ be a finite-dimensional Hilbert space. The space of bounded operators $\mathcal{B}(\mathcal{H})$ is a $C^*$-algebra.
    \end{example}

    \begin{property}
        Every norm-closed $\ast$-subalgebra of a $C^*$-algebra is a $C^*$-algebra.
    \end{property}

    \newdef{$H^*$-algebra}{\index{H-!$\ast$-algebra}
        A Hilbert space $\mathcal{H}$ equipped with a unital $\ast$-algebra structure such that $\ast$-adjoint and multiplicative adjoints coincide:
        \begin{enumerate}
            \item $\langle ab\mid c\rangle = \langle b\mid a^*c\rangle$, and
            \item $\langle ab\mid c\rangle = \langle a\mid cb^*\rangle$,
        \end{enumerate}
        for all $a,b,c\in\mathcal{H}$.
    }
    \begin{example}[Linear operators]\index{Hilbert--Schmidt!norm}\label{operators:hilbert_schmidt_inner_product}
        The canonical example of $H^*$-algebras is given by the algebra of linear operators on a Hilbert space $\mathcal{H}$, where the involution is given by taking adjoints and the inner product is the Hilbert--Schmidt inner product induced by the norm (\cref{linalgebra:hilbert_schmidt_norm}) (up to a factor $k>0$):
        \begin{gather}
            \langle f\mid g\rangle_{\text{HS}} := k\,\tr(f^*g)\,.
        \end{gather}
        The resulting space is denoted by $L^2(\mathcal{H},k)$. A result, analogous to the Artin--Wedderburn theorem~\ref{algebra:artin_wedderburn}, states that every $H^*$-algebra can be decomposed as an orthogonal direct sum of finitely many algebras of the form $L^2(\mathcal{H}_i,k_i)$.
    \end{example}

\subsection{Positive maps}

    \newdef{Positive element}{\index{positive}
        A self-adjoint element of a $C^*$-algebra $A$ for which its spectrum is contained in $[0,+\infty[$. The cone of all positive elements in $A$ is often denoted by $A^+$.
    }
    \begin{property}[Positive decomposition]
        Every positive element $a$ can be written as $a=b^*b$ for some element $b$. Moreover, every positive element admits a positive square-root.
    \end{property}

    \newdef{Positive map}{\label{operators:positive_map}
        A morphism $T:A\rightarrow B$ of $C^*$-algebras such that $T(A^+)\subseteq B^+$.
    }
    \newdef{Completely positive map}{\index{positive!completely}\label{operators:cp_map}
        \nomenclature[A_CP]{CP}{completely positive}
        A morphism $T:A\rightarrow B$ of $C^*$-algebras such that the following map is positive for all $k\in\mathbb{N}$:
        \begin{gather}
            \mathbbm{1}_k\otimes T:\mathbb{C}^{k\times k}\otimes A\rightarrow\mathbb{C}^{k\times k}\otimes B\,.
        \end{gather}
        If $T$ satisfies this condition only up to an integer $n$, it is said to be \textbf{$n$-positive}.
    }

    \newdef{State}{\index{state}
        A positive linear functional of unit norm on a $C^*$-algebra.
    }
    \begin{property}[Convexity]\index{state!pure}
        The set of states is a convex set. The extreme points of this set are called \textbf{pure states}, all other elements are called \textbf{mixed states}.
    \end{property}

    \newdef{Positivity-improving map}{
        A positive map $T$ that satisfies
        \begin{gather}
            a\geq0\implies T(a)>0\,.
        \end{gather}
    }
    \newdef{Ergodic map}{\index{ergodic}
        A positive map $T$ that satisfies
        \begin{gather}
            \forall a>0:\exists t_a\in\mathbb{R}_0:\exp(t_aT)a>0\,.
        \end{gather}
    }

    \begin{example}[Function algebras]
        Let $X$ be a locally compact Hausdorff space and consider its algebra of functions with compact support $C_c(X)$. By the Riesz--Markov theorem~\ref{distributions:riesz_markov}, the positive linear functionals on this algebra correspond to Radon measures on $X$. The subspace of states then correspond exactly to the probability measures on $X$. Under this identification, the pure states correspond to Dirac measures (evaluation functionals).
    \end{example}

    The following is some kind of Jordan algebra-theoretic analogue of the Gel'fand--Naimark theorem~\ref{operators:gelfand_naimark}.
    \begin{theorem}[Alfsen--Schutz]\index{Alfsen--Schutz}
        The state spaces of two $C^*$-algebras are isomorphic if and only if the algebras are isomorphic as (special) Jordan algebras (\cref{linalgebra:jordan_algebra}).
    \end{theorem}

    \newdef{Cuntz algebra}{\index{Cuntz algebra}
        The $n^{th}$ Cuntz algebra $\mathcal{O}_n$ is defined as the (universal) unital $C^*$-algebra generated by $n$ isometric elements $s_i$ under the additional relation
        \begin{gather}
            \sum_{i=1}^ns_i^*s_i = 1\,,
        \end{gather}
        where 1 is the unit element.
    }

\subsection{Traces}

    \newdef{Trace}{\index{trace}
        Let $A$ be a $C^*$-algebra. A trace on $A$ is a linear functional that satisfies the following conditions:
        \begin{enumerate}
            \item\textbf{Positivity}: $\tr(A^+)\geq 0$, and
            \item\textbf{Tracial}: $\tr(ab)=\tr(ba)$ for all $a,b\in A$.
        \end{enumerate}
    }
    \remark{The tracial property could have been replaced by the following equivalent, but seemingly weaker, condition: $\tr(a^*a)=\tr(aa^*)$ for all $a\in A$.}

    \newdef{Adjoint map}{\index{adjoint!operator}
        Assume that a trace functional $\tr$ is given on a $C^*$-algebra and consider a continuous linear map $\phi$ defined on the Schatten class $\mathcal{I}_p$ (\cref{functional:schatten_class}). One can define the adjoint map $\phi^*$ on $\mathcal{I}_q$ whenever $p,q$ are H\"older conjugate. This adjoint is given by the following equation:
        \begin{gather}
            \tr\bigl((\phi^*(a))^*b\bigr) = \tr\bigl(a^*\phi(b)\bigr)\,,
        \end{gather}
        where $a\in\mathcal{I}_q$ and $b\in\mathcal{I}_p$.
    }
    \newdef{Trace-preserving map}{\label{operators:trace_preserving}
        A map $\phi:A\rightarrow\mathbb{R}$ is said to be trace-preserving if it satisfies
        \begin{gather}
            \tr\bigl(\phi(a)\bigr) = \tr(a)
        \end{gather}
        for all trace-class elements $a\in A$. Using the above definition it is easy to see that on a unital $C^*$-algebra this is equivalent to
        \begin{gather}
            \phi^*(1) = 1\,.
        \end{gather}
    }

    \begin{property}
        A completely positive, trace-preserving map $\phi$ satisfies:
        \begin{gather}
            \|\phi\|_1 = 1\,,
        \end{gather}
        where the subscript $1$ indicates that this operator is defined on trace-class elements.
    \end{property}

    \begin{property}[States]\index{state}
        Whenever a $C^*$-algebra is commutative, every state defines a trace and, possibly after a suitable normalization, every trace defines a state. However, for noncommutative $C^*$-algebras only the latter implication holds.
    \end{property}

\subsection{Representations}

    \newdef{$C^*$-algebra representation}{\index{representation!of an algebra}
        A representation of a $C^*$-algebra $A$ on a Hilbert space $\mathcal{H}$ is a unital $\ast$-morphism $A\rightarrow\mathcal{B}(\mathcal{H})$.
    }

    \newdef{Normal state}{\index{state!normal}\label{operators:normal_state}
        Consider a Hilbert space $\mathcal{H}$ and a $C^*$-representation $\pi:A\rightarrow\mathcal{B}(\mathcal{H})$. A normal state $\omega\in A$ is a state such that there exists a trace-class operator $\rho\in\mathcal{B}_1(\mathcal{H})$ with the following property:
        \begin{gather}
            \omega(a) = \frac{\tr\bigl(\rho\pi(a)\bigr)}{\tr(\rho)}\,.
        \end{gather}
        In case where $A=\mathcal{B}(\mathcal{H})$, the normal states are exactly the $\sigma$-weakly continuous states (\cref{functional:sigma_weak_topology}).
    }

    \newdef{Folium}{\index{folium}
        Let $\pi:A\rightarrow\mathcal{B}(\mathcal{H})$ be a $C^*$-representation. The space of normal states of this representation is called its folium.
    }

    \newdef{Cyclic vector}{\index{cyclic!vector}
        A cyclic vector for a $C^*$-algebra representation $\rho:A\rightarrow\mathcal{B}(\mathcal{H})$ is a vector $\xi\in\mathcal{H}$ such that $\{\rho(a)\xi\mid a\in A\}$ is (norm-)dense in $\mathcal{H}$.
    }

    \begin{theorem}[Stinespring]\index{Stinespring}\label{operators:stinespring}
        Consider a linear operator $\Phi:\mathcal{B}(\mathcal{H}_1)\rightarrow\mathcal{B}(\mathcal{H}_2)$. It is completely positive if and only if there exists a $C^*$-representation $\pi:\mathcal{B}(\mathcal{H}_2)\rightarrow\mathcal{B}(\mathcal{K})$ and a bounded linear operator $V:\mathcal{H}_1\rightarrow\mathcal{K}$ such that
        \begin{gather}
            \Phi^\dagger(a) = V^\dag\pi(a)V
        \end{gather}
        or, equivalently,
        \begin{gather}
            \Phi(\rho) = \pi^\dagger(V\rho V^\dagger)\,.
        \end{gather}
        Moreover, $V$ is an isometry if and only if $\Phi$ is trace-preserving.
    \end{theorem}
    \begin{remark}
        Because the adjoint of a completely positive map is again completely positive, the above two characterizations can be used interchangeably. The former is the mostly used by mathematicians, while the latter is better known in the physics literature.

        The most common situation is where $\mathcal{H}:=\mathcal{H}_1=\mathcal{H}_2$ and $\pi:\mathcal{B}(\mathcal{H})\rightarrow\mathcal{B}(\mathcal{H}\otimes\mathcal{K}):a\mapsto a\otimes b$, for some density operator $b\in\mathcal{B}(\mathcal{K})$. The adjoint to tensoring by a density operator is taking the partial trace $\tr_{\mathcal{K}}$. This way one obtains the following expressions:
        \begin{gather}
            \Phi^\dagger(a) = V^\dag(a\otimes b)V
        \end{gather}
        and, equivalently,
        \begin{gather}
            \Phi(\rho) = \tr_{\mathcal{K}}(V\rho V^\dagger)\,.
        \end{gather}
    \end{remark}

\subsection{Gel'fand duality}\index{Gel'fand!duality}

    \newdef{Gel'fand spectrum}{\index{Gel'fand!spectrum}\label{operators:gelfand_spectrum}
        Consider a commutative $C^*$-algebra $A$ (in fact any involutive algebra suffices). Its set of characters, i.e.~the algebra morphisms $A\rightarrow\mathbb{C}$, can be equipped with a locally compact Hausdorff topology: the weak-* topology (\cref{functional:weak_star_topology}). This space is compact if and only if the algebra is unital.
    }
    \newdef{Gel'fand representation}{
        Consider a $C^*$-algebra $A$ and let $\Phi_A$ denote its Gel'fand spectrum. The Gel'fand transformation of an element $a\in A$ is defined as the morphism $\hat{a}:\Phi_A\rightarrow\mathbb{C}$ given by the following formula:
        \begin{gather}
            \hat{a}(\lambda) = \langle\lambda,a\rangle\,,
        \end{gather}
        where $\langle\cdot,\cdot\rangle$ denotes the pairing between $A$ and $\Phi_A$. By definition of the topology on the Gel'fand spectrum the functional $\hat{a}$ is continuous for all $a\in A$. The mapping $a\mapsto\hat{a}$ is called the Gel'fand representation of $A$.
    }

    \begin{theorem}[Gel'fand--Naimark: Commutative case]\index{Gel'fand--Naimark}
        Let $A$ be a commutative $C^*$-algebra. The Gel'fand representation gives an isometric $\ast$-isomorphism between $A$ and the set $C_0(\Phi_A)$ of continuous complex-valued functions that vanish at infinity on its Gel'fand spectrum.
    \end{theorem}
    \begin{remark}
        In fact, the Gel'fand--Naimark theorem gives an equivalence between the category of commutative nonunital $C^*$-algebras and the category of locally compact Hausdorff spaces (with continuous functions that vanish at infinity).
    \end{remark}

    \begin{property}
        A compact Hausdorff space is connected if and only if its algebra of continuous functions has no nontrivial projections.
    \end{property}
    \begin{property}
        A commutative $C^*$-algebra is separable if and only if its Gel'fand spectrum is metrizable.
    \end{property}

    \begin{formula}[Spectrum]
        Recall \cref{functional:spectrum} of the spectrum of an operator. This is related to the spectrum of a unital, commutative $C^*$-algebra as follows:
        \begin{gather}
            \sigma(a) = \{a(x)\mid x\in\Phi_A\}\,,
        \end{gather}
        for all $a\in A$.
    \end{formula}

    \begin{construct}[Gel'fand--Naimark--Segal]\index{Gel'fand--Naimark--Segal construction}\label{operators:gns}
        \nomenclature[A_GNS]{GNS}{Gel'fand--Naimark--Segal}
        Let $A$ be a $C^*$-algebra. Given a state $\omega$ on $A$ there exists a $C^*$-representation $\rho:A\rightarrow\mathcal{B}(D)$, where $D\subset\mathcal{H}$ is a dense subspace of a Hilbert space $\mathcal{H}$, such that the following conditions are satisfied:
        \begin{itemize}
            \item There exists a distinguished cyclic unit vector $\xi$.
            \item For all elements $a\in A$ the following equality holds:
                \begin{gather}
                    \omega(a) = \langle\rho(a)\xi\mid\xi\rangle\,.
                \end{gather}
        \end{itemize}

        @@ COMPLETE CONSTRUCTION @@
    \end{construct}

    \begin{theorem}[Gel'fand--Naimark: General case]\index{Gel'fand--Naimark}\label{operators:gelfand_naimark}
        Every $C^*$-algebra is isometrically $\ast$-isomorphic to a norm closed ($C^*$-)algebra of bounded operators on a Hilbert space $\mathcal{H}$.
    \end{theorem}

\subsection{\difficult{Hilbert modules}}

    \newdef{Hilbert $C^*$-module}{\index{Hilbert!module}
        Let $A$ be a $C^*$-algebra and $H$ a vector space. $H$ is a (right) pre-Hilbert $A$-module if there exists a map
        \begin{gather}
            \langle\cdot\mid\cdot\rangle_A:H\times H\rightarrow A\,,
        \end{gather}
        sometimes called the \textbf{$A$-inner product}, satisfying the following conditions:
        \begin{enumerate}
            \item\textbf{Conjugate linearity}: For all $x,y,z\in H$:
                \begin{gather}
                    \langle x\mid y+z\rangle_A = \langle x\mid y\rangle_A+\langle x\mid z\rangle_A \qquad\qquad \langle x\mid y\rangle_A = \langle y\mid x\rangle_A^*\,.
                \end{gather}
            \item\textbf{Nondegeneracy}: For all $x\in H$:
                \begin{gather}
                    \langle x\mid x\rangle_A\geq0 \qquad\qquad \langle x\mid x\rangle_A = 0 \iff x=0\,.
                \end{gather}
            \item\textbf{Equivariance}: For all $x,y\in H$ and $a\in A$:
                \begin{gather}
                    \langle x\mid y\rangle_A\,a = \langle x\mid y\cdot a\rangle_A\,.
                \end{gather}
        \end{enumerate}
        A left module is obtained by requiring linearity and (left-)equivariance in the first argument. The completion of a pre-Hilbert module with respect to the norm $x\mapsto\sqrt{\|\langle x|x \rangle_A\|}$ is called a \textbf{Hilbert $A$-module}.
    }
    \newdef{Hilbert $C^*$-bimodule}{
        Let $A,B$ be $C^*$-algebras and consider a Hilbert $C^*$-module $H$ over $B$. $H$ is called a Hilbert $(A,B)$-bimodule if it admits a left $\ast$-representation of $A$ such that
        \begin{gather}
            \langle a^*\cdot x\mid y\rangle_B = \langle x\mid a\cdot y\rangle_A
        \end{gather}
        for all $a\in A$, i.e.~adjoints in $A$ correspond to adjoints with respect to the $B$-inner product.
    }

    \newdef{Fredholm module}{\index{Fredholm!module}\label{operators:fredholm_module}
        Let $\pi:A\rightarrow\End(\mathcal{H})$ be a $\ast$-representation of a unital $C^*$-algebra on a Hilbert space $\mathcal{H}$. When equipped with an operator $F\in\End(\mathcal{H})$ this gives an (\textbf{odd}) Fredholm module if
        \begin{gather}
            F=F^*\qquad\text{and}\qquad F^2=\mathbbm{1}_{\mathcal{H}}\qquad\text{and}\qquad[F,\pi(A)]\subseteq\mathcal{B}_0(\mathcal{H})\,.
        \end{gather}
        The Fredholm operator is said to be \textbf{even} if $\mathcal{H}$ is a super-Hilbert space with grading $\Gamma\in\End(\mathcal{H})$ satisyfing
        \begin{gather}
            \Gamma^*=\Gamma\qquad\text{and}\qquad\Gamma^2=\mathbbm{1}_{\mathcal{H}}\qquad\text{and}\qquad[\Gamma,\pi(A)]=0\qquad\text{and}\qquad\{\Gamma,F\}_+=0\,.
        \end{gather}
        $F$ has the form
        \begin{gather}
            \begin{pmatrix}
                0&F_+\\F_-&0
            \end{pmatrix},
        \end{gather}
        where $F_+$ is Fredholm.
    }

    \begin{example}
        Consider $A=\mathbb{C}$ and let $\pi:\mathbb{C}\rightarrow\End(\mathcal{H})$ be the unique unital representation. Fredholm modules with this data correspond to (essentially self-adjoint) Fredholm operators on $\mathcal{H}$.
    \end{example}

    \begin{remark}[Kasparov $K$-theory]
        If $A$ is not unital, one should relax the first two relations up to compact operators, i.e.~$F=F^*$ and $F^2=\mathbbm{1}_{\mathcal{H}}$ in the Calkin algebra. This gives a clear relation to \textit{Kasparov's $K$-theory} (see below), where the homology complex $K\!K_\bullet$ is defined in terms of such generalized Fredholm modules. However, it can be shown that these induce the same $K\!K$-classes.
    \end{remark}

    \newdef{Kasparov bimodule}{\index{Kasparov bimodule}
        Let $A,B$ be two $C^*$-algebras and let $H$ be a super-Hilbert $(A,B)$-bimodule. $H$ is called a Kasparov $(A,B)$-bimodule if there exists an odd adjointable operator $F\in\mathcal{B}(H)$ satisfying the following properties for all $a\in A$:
        \begin{enumerate}
            \item $(F^2-\mathbbm{1}_H)\pi_A(a)$ is compact,
            \item $(F-F^*)\pi_A(a)$ is compact, and
            \item $[F,\pi_A(a)]$ is compact.
        \end{enumerate}
        If $A$ is unital (and the representation respects units), this implies that $F$ is a projection in the Calkin algebra. If $F$ is a projection on the nose, the bimodule is said to be \textbf{normalized}.
    }
    \newdef{Kasparov $K$-theory}{\index{K-theory!Kasparov}\label{operators:kk_theory}
        The set, in fact Abelian group, of homotopy classes of Kasparov bimodules $K\!K(A,B)$, where a homotopy between two Kasparov $(A,B)$-bimodules is a Kasparov $(A,C([0,1],B))$-bimodule that restricts to the given bimodules on the boundaries of the interval.
    }
    \begin{property}
        Every Kasparov bimodule is homotopic to a normalized bimodule. Hence, for Kasparov $K$-theory, one can restrict to normalized bimodules.
    \end{property}

\section{von Neumann algebras}

    \newdef{von Neumann algebra}{\index{von Neumann!algebra}
        A $*$-subalgebra of a $C^*$-algebra equal to its double commutant: $M=M''$.
    }
    \newdef{Concrete von Neumann algebra}{
        A weakly closed unital $*$-algebra of bounded operators on some Hilbert space.
    }
    \begin{theorem}[Double Commutant theorem\footnotemark]
        \footnotetext{Often called \textbf{von Neumann's double commutant theorem}.}
        The above definitions are equivalent.
    \end{theorem}

    \newdef{Projection}{\index{projection}\label{operators:projection}
        An element $p$ of a von Neumann algebra such that
        \begin{gather}
            p = p^2 = p^*\,.
        \end{gather}
        This terminology reflects the property that if a von Neumann algebra is regarded as an algebra of bounded operators, the projections are exactly the operators associated to an orthogonal projection.
    }
    \begin{property}
        Any von Neumann algebra is generated by its projections.
    \end{property}

    \newdef{Murray--von Neumann equivalence}{\index{equivalence!Murray--von Neumann}
        Two closed subspaces are said to be Murray-von Neumann equivalent if one is mapped isomorphically onto the other by a partial isometry. In terms of projections this means that $p\sim q$ if and only if there exists a partial isometry $u$ such that $p=uu^*$ and $q=u^*u$.
    }

    \newprop{Partial order}{\index{projection!finite}
        The collection of projections inherits the structure of a partial order from the partial order on the corresponding subspaces. A projection $p$ is said to be finite if there exists no smaller projection $q$ that is equivalent to $p$.
    }

    \begin{theorem}[Gleason]\index{Gleason}\index{measure}
        Consider a \textbf{finitely additive measure}\footnote{If this also holds for countable sequences, it is called \textbf{$\sigma$-additive} in analogy with \cref{lebesgue:measure}.} on the von Neumann algebra $\mathcal{B}(\mathcal{H})$, i.e.~a function $\mu$ on the set of orthogonal projections in $\mathcal{B}(\mathcal{H})$ such that
        \begin{gather}
            \mu\left(\sum_{i=1}^np_i\right) = \sum_{i=1}^n\mu(p_i)
        \end{gather}
        for all $n\in\mathbb{N}$. If $\dim(\mathcal{H})\neq2$, each finitely additive measure can be uniquely extended to a state on $\mathcal{B}(\mathcal{H})$. Moreover, every $\sigma$-additive measure can be uniquely extended to a normal state.

        Conversely, the restriction of every state (resp.~normal state) to the set of orthogonal projections induces a finitely additive (resp.~$\sigma$-additive) measure.
    \end{theorem}

\subsection{Factors}

    \newdef{Factor}{\index{factor}
        Consider a von Neumann algebra $M$. A $*$-subalgebra $A$ is called a factor of $M$ if its center $Z(A)$ is given by the scalar multiples of the idenity.
    }

    \newdef{Type-$\mathrm{I}$ factor}{
        A factor is of type $\mathrm{I}$ if it contains a minimal projection.
    }
    \begin{property}[Type-$\mathrm{I}_n$ factors]
        Any type $\mathrm{I}$-factor is isomorphic to the algebra of all bounded operators on a Hilbert space. To indicate the dimension $n$ of this Hilbert space (which may be $\infty$) one sometimes uses the subclassification of type $\mathrm{I}_n$-factors.
    \end{property}

    \newdef{Powers index}{\index{index!Powers}
        Consider a Hilbert space $\mathcal{H}$ together with its von Neumann algebra of bounded operators $\mathcal{B}(\mathcal{H})$. A unital $*$-endomorphism $\alpha$ is said to have Powers index $n\in\mathbb{N}$ if the space $\alpha(\mathcal{B}(\mathcal{H}))$ is isomorphic to a type $\mathrm{I}_n$-factor.
    }

    \newdef{Type-$\mathrm{II}$ factor}{
        A factor is of type $\mathrm{II}$ if it contains nonzero finite projections but no minimal ones. If the identity is finite, the factor is sometimes said to be of type $\mathrm{II}_1$, otherwise it is of type $\mathrm{II}_\infty$.
    }

    \newdef{Type-$\mathrm{III}$ factor}{
        A factor is of type $\mathrm{III}$ if it does not contain any nonzero finite projections.
    }
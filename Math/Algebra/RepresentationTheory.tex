\chapter{Representation Theory}

    References for this chapter are~\citet{jeevanjee_introduction_2015,
    fulton_representation_2004}. \Cref{section:groups,section:group_actions} can be visited for an introduction to groups and group actions.

    \minitoc

\section{Group representations}

    Group actions on vector spaces are so important that they receive their own name.
    \newdef{Representation}{\index{representation!of a group}\label{rep:representation}
        A representation of a group $G$ on a vector space $V$ is a group morphism $\rho:G\rightarrow\GL(V)$ from $G$ to the automorphism group (\cref{linalgebra:automorphism}) of $V$:
        \begin{gather}
            \rho(g)\circ\rho(g') = \rho(gg')
        \end{gather}
        for all $g,g'\in G$.
    }
    \begin{property}[Freeness]
        Because every linear map takes the zero vector to itself, a representation can never be free.
    \end{property}

    \newdef{Subrepresentation}{
        A subrepresentation of a representation on $V$ is a subspace of $V$ invariant under the action of the group $G$ (together with the restricted action).
    }

    \begin{example}[Alternating representation]\index{representation!alternating}\label{rep:sign_representation}
        The alternating or \textbf{sign} representation of $S_n$ on an $\mathbb{R}$-vector space is defined by multiplication by the parity of the permutation.
    \end{example}

    \begin{example}[Permutation representation]\index{permutation!representation}\label{rep:permutation}
        Consider a vector space $V$ with basis $\{e_i\}_{i\leq n}$. Since the symmetric group $S_n$ acts on the index set $\{1,\ldots,n\}$, this induces a representation given by
        \begin{gather}
            \rho(g):\sum_{i=1}^nv_ie_i\mapsto\sum_{i=1}^nv_ie_{g\cdot i}\,.
        \end{gather}
    \end{example}

    \begin{example}[Contragredient representation]\index{dual!representation}\index{contra-!gredient|see{dual representation}}
        For every representation $\rho$ on $V$, there exists a natural representation on the dual space $V^*$:
        \begin{gather}
            \rho^*(g) := \rho^T(g^{-1}):V^*\rightarrow V^*\,,
        \end{gather}
        where $\rho^T$ is the transpose (\cref{linalgebra:transpose}). It is implicitly defined by requiring
        \begin{gather}
            \Bigl\langle\rho^*(g)(v^*),\rho(g)(v)\Bigr\rangle = \langle v^*,v \rangle
        \end{gather}
        for all $v\in V$ and $v^*\in V^*$, where $\langle\cdot,\cdot\rangle$ is the natural pairing.
    \end{example}

    \begin{example}[Tensor product representation]\index{tensor product!representation}
        A group $G$ that acts on vector spaces $V,W$ also has a representation on the tensor product $V\otimes W$ in the following way:
        \begin{gather}
            \rho_{V\otimes W}(g)(v\otimes w) := \rho_V(g)(v)\otimes\rho_W(g)(w)\,.
        \end{gather}
        More generally, consider two representations $\phi:G\rightarrow\GL(V)$ and $\psi:H\rightarrow\GL(W)$. A representation of the direct product $G\times H$ on $V\otimes W$ is given by the tensor product of representations:
        \begin{gather}
            \rho(g,h)(v\otimes w) := \phi(g)(v)\otimes\psi(h)(w)\,.
        \end{gather}
        The former case can be obtained as a subrepresentation induced by the diagonal subgroup inclusion $\Delta_G:G\hookrightarrow G\times G$.
    \end{example}

    \newdef{Intertwiner}{\index{intertwiner}
        If one interprets $G$-representations as $G$-modules, the natural morphisms are the intertwiners (\cref{group:equivariant}).
    }

\section{Irreducible representations}\label{section:irreducibility}

    \newdef{Irreducibility}{\index{irreducible!representation}\index{irrep}
        A representation is said to be irreducible if there exist no proper nonzero subrepresentations. Irreducible representations are also abbreviated as \textbf{irreps} (especially in physics).
    }

    \begin{example}[Standard representation]
        Consider the action of $S_n$ on a vector space $V$ with basis $\{e_i\}_{i\leq n}$. The line generated by $e_1+e_2+\cdots+e_n$ is invariant under the permutation action of $S_n$. It follows that the permutation representation (on finite-dimensional spaces) is never irreducible.

        The $(n-1)$-dimensional complementary subspace
        \begin{gather}
            W = \left\{\sum_{i=1}^n\lambda_ie_i\,\middle\vert\,\sum_{i=1}^n\lambda_i=0\right\}
        \end{gather}
        forms an irreducible representation. It is called the standard representation of $S_n$ on $V$.
    \end{example}

    Schur's lemma~\ref{cat:schur_lemma} and its corollary are usually found in the following form.
    \begin{theorem}[Schur's lemma]\index{Schur!lemma}\label{rep:schurs_lemma}
        Let $V,W$ be two finite-dimensional irreducible representations of a group $G$ and let $\varphi:V\rightarrow W$ be an intertwiner.
        \begin{itemize}
            \item Either $\varphi$ is an isomorphism or $\varphi=0$.
            \item If $V=W$, then $\varphi$ is constant, i.e. $\varphi$ is a scalar multiple of the identity map $\mathbbm{1}_V$.
        \end{itemize}
    \end{theorem}

    \begin{property}[Complementary representation]\index{complement!representation}\index{invariant!inner product}
        Let $V$ be a representation of a finite group $G$. For every subrepresentation $W$, there exists an invariant complementary subspace $W'$. This space can be found as follows. Choose an arbitrary complement $U$ such that $V=W\oplus U$ with the associated projection map $\pi_0:V \rightarrow W$. Averaging over $G$ gives the intertwiner
        \begin{gather}
            \pi(v) := \sum_{g\in G}g\circ\pi_0(g^{-1}v)\,.
        \end{gather}
        On $W$ it is given by multiplication by $|G|$. Its kernel is an invariant subspace of $V$, complementary to $W$.

        More generally, let $V$ be an inner product space with a representation of a compact group $G$. A $G$-invariant inproduct on $V$ can be obtained through the Haar measure (\cref{distributions:haar}):
        \begin{gather}
            \langle v\mid w\rangle_G := \Int_G \langle gv\mid gw\rangle\,d\mu(g)\,.
        \end{gather}
        The complementary subrepresentation of $W$ is then simply the orthogonal complement $W^\perp$ with respect to this invariant inner product.
    \end{property}
    \begin{theorem}[Maschke]\index{Maschke}
        Let $G$ be a finite group with a representation space $V$ such that the characteristic of the underlying field does not divide the order of $G$. This representation can be uniquely (up to isomorphism) decomposed as
        \begin{gather}
            V = V_1^{\oplus a_1}\oplus\cdots\oplus V_k^{\oplus a_k}\,,
        \end{gather}
        where all $V_k$'s are distinct irreducible representations.
    \end{theorem}

    \newdef{Matrix coefficient}{\index{matrix!coefficient}
        Consider a compact topological group $G$. A matrix coefficient is a character $\chi:G\rightarrow\mathbb{C}^\times$ (\cref{distribution:character}) of the form
        \begin{gather}
            \chi\equiv T\circ\rho\,,
        \end{gather}
        where $\rho:G\rightarrow\GL(V)$ is a finite-dimensional representation of $G$ and $T:\End(V)\rightarrow\mathbb{C}$ is a linear functional.\footnote{Note that $G$ need not be a matrix group, i.e.~none of the representations have to be faithful.}
    }

    \begin{theorem}[Peter--Weyl]\index{Peter--Weyl}\label{rep:peter_weyl}$ $
        \begin{enumerate}[I)]
            \item The space of matrix coefficients is dense in $C(G)$ under the $L^\infty$-norm and, accordingly, also in $L^2(G)$.\footnote{Square-integrability is well defined due to Haar's theorem~\ref{distribution:haar_theorem}.}
            \item Let $\rho:G\rightarrow\mathrm{U}(\mathcal{H})$ be a unitary representation on a (complex) Hilbert space $\mathcal{H}$ (see \cref{functional:hilbert_space}). There exists an orthogonal decomposition of $\mathcal{H}$ in finite-dimensional, unitary irreps of $G$.
            \item $L^2(G)$ can be decomposed as
            \begin{gather}
                L^2(G) = \widehat{\bigoplus_{[V_\rho]}}V_\rho^{\oplus\dim(V_\rho)}\,,
            \end{gather}
            where $[V_\rho]$ runs over (the underlying vector spaces of) all isomorphism classes of finite-dimensional unitary irreps of $G$. In particular, an orthonormal basis is given by
            \begin{gather}
                \left\{\sqrt{\dim(V_\rho)}u_{ij}^\rho\,\middle\vert\,[V_\rho],1\leq i,j\leq\dim(V_\rho)\right\}\,,
            \end{gather}
            where the $u_{ij}^\rho$ are matrix coefficients of the representation $\rho$ with respect to a chosen orthonormal basis of $V_\rho$.
        \end{enumerate}
    \end{theorem}

    \begin{result}[Class functions]\index{class!function}
        When restricted to the set of $\Ad_G$-invariant functions, the \textbf{class functions} play a special role, i.e.~functions $f$ such that $f(g^{-1}hg)=f(h)$ for all $g,h\in G$. A basis for the subspace of square-integrable class functions is given by the matrix coefficients for which the linear functional is given by the trace:
        \begin{gather}
            \tr_\rho = \sum_{i=1}^{\dim(V_\rho)}u^\rho_{ii}\,.
        \end{gather}
    \end{result}

\section{Classification}

    For an introduction to Young tableaux, see \cref{section:partition}.

    \newdef{Permutation module}{\index{permutation!module}
        Let $\lambda$ be a partition. The permutation module $M^\lambda$ is defined as the vector space generated by the Young tabloids of shape $\lambda$.
    }

    \newdef{Specht module}{\index{Specht module}\index{poly-!tabloid}
        Consider a permutation module $M^\lambda$ for some $\lambda$ with $|\lambda|=n$. Since the permutation group $S_n$ acts on Young tableaux by permuting the entries, it also has an induced action\footnote{This can be generalized to an action of the group algebra $\mathfrak{K}[S_n]$.} on $M^\lambda$. For every Young tableau $t$, it has a subrepresentation induced by the subgroup $Q_t$ of permutations that leave the columns invariant. The Specht module is the representation generated by the following elements (called \textbf{polytabloids}):
        \begin{gather}
            e_t := \sum_{\sigma\in Q_t}\sgn(\sigma)\sigma\cdot\{t\}\,,
        \end{gather}
        where $t$ ranges over the Young tableaux of shape $\lambda$ and $\{t\}$ denotes the Young tabloid associated to the tableau $t$. In fact, one can just take the standard Young tableaux as generators. This is sometimes called the \textbf{Specht basis}.
    }

    \begin{property}
        A representation (over $\mathbb{C}$) of $S_n$ is irreducible if and only if it is a Specht module $S^\lambda$ for some partition $\lambda$ of $n$.
    \end{property}

    One can restate the definition of a Specht module using the following operator.
    \newdef{Young symmetrizer}{\index{Young!symmetrizer}
        Given a Young tableau $t$ of shape $\lambda$, one can decompose the permutation group $S_{|\lambda|}$ as the union of two types of permutations. First, one has the permutations that preserve the rows, denote these by $P_\lambda$. Then, one has also the permutations that preserve the columns, denote these by $Q_\lambda$. These two subgroups induce elements in the group algebra $\mathbb{C}[S_{|\lambda|}]$ as follows:
        \begin{align}
            a_\lambda &:= \sum_{p\in P_\lambda}p\,,\\
            b_\lambda &:= \sum_{q\in Q_\lambda}\sgn(q)q\,.
        \end{align}
        The product $c_\lambda:=b_\lambda a_\lambda$ is called the Young symmetrizer of $\lambda$.
    }
    Comparing the above definition of the Specht module to the Young symmetrizers leads to the following alternative definition.
    \newadef{Specht module}{
        The space $\mathbb{C}[S_{|\lambda|}]c_\lambda$ is called the Specht module $S_\lambda$.
    }

    Consider a vector space $V$ together with its general linear group $\GL(V)$. For all $n\in\mathbb{N}$, there is an induced (diagonal) representation on $V^{\otimes n}$ by $\GL(V)$. There is also an action by the permutation group $S_n$ that permutes the elements in a monomial $v_1\otimes\cdots\otimes v_n\in V^{\otimes n}$.
    \begin{theorem}[Schur--Weyl]\index{Schur--Weyl duality}\index{Schur!functor}
        This representation of $\GL(V)\times S_n$ can be decomposed as follows:
        \begin{gather}
            V^{\otimes n}\cong\bigoplus_{|\lambda|=n}V_\lambda\otimes S_\lambda V\,,
        \end{gather}
        where
        \begin{itemize}
            \item the sum ranges over all partitions of $n$ or, equivalently, all Young diagrams with $n$ boxes,
            \item the $V_\lambda$ are Specht modules and, hence, irreducible representations of $S_n$, and
            \item the $S_\lambda V$ are (possibly zero) irreducible representations of $\GL(V)$ of the form $S_\lambda V \equiv\hom_{S_n}(V_\lambda,V^{\otimes n})$.
        \end{itemize}
    \end{theorem}
    The spaces $V_\lambda$ can be interpreted as multiplicity spaces. The spaces $S_\lambda V$ can be rewritten more explicitly as $V_\lambda\otimes_{S_n} V^{\otimes n}$ by self-duality of $S_n$-representations or, by using the explicit characterization of Specht modules given above, as $V^{\otimes n}c_\lambda$. Because of the functoriality of the involved operations, one can also see that $S_\lambda$ is in fact a functor, the \textbf{Schur functor}.

    \begin{example}[Algebraic curvature tensor]
        The above definition of the Schur spaces $S_\lambda V$ allows for an a concise description of representations in terms of Young diagrams (and tableaux). Consider for example the \textit{Riemann curvature tensor} $R_{ijkl}$ from \labelref{chapter:riemann}. This tensor has the following symmetries:
        \begin{itemize}
            \item $R_{ijkl} = -R_{jikl} = -R_{ijlk}$,
            \item $R_{ijkl} = R_{klij}$, and
            \item $R_{ijkl} + R_{iljk} + R_{iklj} = 0$.
        \end{itemize}
        By looking at the definition of the Young symmetrizer, one can see that these symmetries are exactly those of the irreducible component in $S_\lambda V$ associated to the partition $(2,2)$.
    \end{example}

\section{Generalized representations}
\subsection{Projective representations}\index{representation!projective}\label{section:projective_representation}

    \Cref{rep:representation} represented group elements by operators in such a way that the composition/multiplication is preserved. However, in some applications, such as quantum mechanics (see \labelref{chapter:qm}), operations are only determined up to a phase factor and, hence, it makes sense that this composition rule is also relaxed. This is formalized as follows.

    \newdef{Projective representation}{\index{projective!group}
        A projective representation of a group $G$ on a vector field $V$ is a group morphism $\rho:G\rightarrow\mathrm{PGL}(V)$ from $G$ to the \textbf{projective general linear group} of $V$:
        \begin{gather}
            \mathrm{PGL}(V) := \mathrm{GL}(V)/\mathfrak{K}^\times\,,
        \end{gather}
        where $\mathfrak{K}$ is the underlying field of $V$. This means that
        \begin{gather}
            \rho(g)\circ\rho(g')=\omega(g,g')\rho(gg')
        \end{gather}
        for all $g,g'\in G$ for some function $\omega:G\times G\rightarrow \mathfrak{K}^\times$.
    }

\subsection{Group cohomology}\index{cohomology!group}

    \begin{example}[Bilinear forms]
        Consider a vector space $V$ and a bilinear form $\omega:V\times V\rightarrow\mathbb{R}$ ($\mathbb{R}$ can be replaced by any other field). Then $\omega$ induces a class in $H^2(V;\mathbb{R})$ by linearity.
    \end{example}

\todo{COMPLETE (classification by cocycles, covering groups, generalized representations)}

\section{Tensor operators}

    \newdef{Representation operators}{\index{representation!operator}\index{tensor!operator|see{representation operator}}
        An intertwiner $\psi:(\rho,V_0)\rightarrow\End(V)$ between a $G$-representation on an auxiliary vector space $V_0$ to the space of linear maps on a $G$-vector space $V$ (equipped with the adjoint action).

        More explicitly, consider a set of operators $\{\widehat{O}_i\}_{i\in I}\subset\End(V)$ acting on a vector space $V$ equipped with a representation $\rho$ of the group $G$. This collection defines a representation operator with respect to $G$ if there exists a matrix representation $R$ of $G$ such that the following equation holds:
        \begin{gather}
            \rho(g)\widehat{O}_i\rho(g)^{-1} = \sum_{j\in I}R(g)_{ij}\widehat{O}_j\,.
        \end{gather}
    }
    \begin{example}[Tensor operators]
        Consider $G=\mathrm{SO}(3)$ and $V_0=\mathcal{T}^r_s(\mathbb{R}^3)$. This choice gives a set of operators that transform as tensors under rotations. By choosing $V_0=\mathbb{R}^3$ or $V_0=\mathcal{H}_l(\mathbb{R}^3)$, the space of spherical harmonics of degree $l$, one obtains the \textbf{vector} and \textbf{spherical operators}.
    \end{example}

    The following property is often used in quantum mechanics to quickly find forbidden transitions in atomic or molecular systems.
    \begin{property}[Selection rules]
        Let $G$ be a semisimple group and let $W_1,W_2$ be two inequivalent (finite-dimensional), irreducible, unitary subrepresentations of a Hilbert space $\mathcal{H}$. Let $\widehat{O}$ be a representation operator indexed by a vector space $V$. For all $v\in V,w_i\in W_i$ one has
        \begin{gather}
            \langle w_1\mid\widehat{O}(v)w_2 \rangle=0\,,
        \end{gather}
        unless $V\otimes W_2$ contains a subrepresentation equivalent to $W_1$.
    \end{property}

    \begin{theorem}[Wigner--Eckart]\index{Wigner--Eckart}
        Consider two irreducible $\mathrm{SU}(2)$-subrepresentations $W_j$ and $W_{j'}$ of some unitary representation $\mathcal{H}$, together with two degree-$q$ spherical tensors $\widehat{O},\widehat{O}':V_0\rightarrow\End(\mathcal{H})$. If there exists at least one index $k\leq q$ and one pair of vectors $(v,v')\in W_j\times W_{j'}$ such that \[\langle v'\mid\widehat{O}_kv \rangle\neq0\,,\] then for all indices $k\geq 0$ and pairs $(v,v')\in W_j,W_{j'}$ the following equality holds
        \begin{gather}
            \langle v'\mid\widehat{O}_k'v \rangle = C\langle v'\mid\widehat{O}_kv \rangle
        \end{gather}
        for some constant $C$ that only depends on $q,j$ and $j'$.
    \end{theorem}
    By noting that the \textit{Clebsch--Gordan coefficients} (see \cref{angular_momentum:clebsch-gordan}) are the components of the projection $W_q\otimes W_j\rightarrow W_{j'}$, which is itself an intertwiner, one can recast the Wigner--Eckart theorem as a statement about matrix elements.
    \begin{result}\index{momentum!angular}\index{Clebsch--Gordan coefficient}
        Consider an irreducible tensor operator $T_j^m$ (with respect to the rotation group). The matrix elements of this operator with respect to a symmetry-adapted basis (`angular momentum' basis) decompose as a product of a Clebsch--Gordan coefficient and a factor that only depends on the eigenvalues of the Casimir operator:
        \begin{gather}
            \langle j',m'\mid R^{(q)}\mid j,m \rangle = \langle j'\|R_k^{(q)}\|j \rangle\langle j',m'\mid q,j;k,m \rangle\,.
        \end{gather}
        The factor $\langle j'\|R^{(q)}\|j \rangle$ is sometimes called the \textbf{reduced matrix element}.
    \end{result}
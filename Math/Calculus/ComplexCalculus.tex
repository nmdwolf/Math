\chapter{Complex Analysis}\label{chapter:complexcalculus}

\section{Complex algebra}

    The set of complex numbers $\mathbb{C}$ forms a 2-dimensional vector space over the field of real numbers (\cref{chapter:linear_algebra}). At the same time, the operations of complex addition and complex multiplication also turn the complex numbers into a field.

    \newdef{Complex conjugate}{\index{complex conjugate}
        Complex conjugation
        \begin{gather}
            \overline{\,\cdot\,\vphantom{z}}:a+bi\mapsto a-bi
        \end{gather}
        is an involution (\cref{set:involution}). It is sometimes denoted by $z^*$ instead of $\overline{z}$, but, unless this would cause confusion, the former notation will be used throughout this compendium.
    }

    \newformula{Real/imaginary part}{
        A complex number can also be written as
        \begin{gather}
            \mathrm{Re}(z) + i\ \mathrm{Im}(z)\,,
        \end{gather}
        where
        \begin{align}
            \mathrm{Re}(z) &:= \frac{z + \overline{z}}{2}\,,\\
            \mathrm{Im}(z) &:= \frac{z - \overline{z}}{2i}\,.
        \end{align}
    }
    \newdef{Argument}{\index{argument}\index{polar!form}
        Consider a complex number expressed in \textit{polar form}:
        \begin{gather}
            z = re^{i\theta}
        \end{gather}
        The number $\theta$ is called the argument of $z$ and it is denoted by $\arg(z)$.

        \todo{DERIVE POLAR FORM}
    }

    \newdef{Riemann sphere}{\index{Riemann!sphere}
        Consider the one-point compactification $\overline{\mathbb{C}} = \mathbb{C}\cup\{\infty\}$ (\cref{topology:alexandrov_compactification}). This set is called the Riemann sphere or \textbf{extended complex plane}. The standard operations on $\mathbb{C}$ can be generalized to $\overline{\mathbb{C}}$ for all nonzero $z\in\mathbb{C}$ in the following way:
        \begin{align}
            z + \infty &:= \infty\nonumber\\
            z * \infty &:= \infty\\
            \frac{z}{\infty} &:= 0\,.\nonumber
        \end{align}
        Since there exists no multiplicative inverse for $\infty$, the Riemann sphere is not a field.
    }

\section{Holomorphic functions}

    \newdef{Holomorphic function}{\index{holomorphic!function}\label{complexcalculus:holomorphic}
        A function $f$ on an open set $U\subseteq\mathbb{C}$ that is complex differentiable at every point $z_0\in U$, i.e.~for every point $z_0\in U$ the following limit exists:
        \begin{gather}
            f'(z_0) := \lim_{z\rightarrow z_0}\frac{f(z) - f(z_0)}{z-z_0}\,.
        \end{gather}
    }
    \newdef{Biholomorphic function}{\index{bi-!holomorphic}
        A holomorphic function admitting a holomorphic inverse.
    }
    \newdef{Entire}{\index{entire}
        A function that is holomorphic on all of $\mathbb{C}$.
    }

    \begin{property}[Cauchy--Riemann conditions]\index{Cauchy--Riemann!conditions}\label{complex:cauchy_riemann}\index{Wirtinger derivative}
        A holomorphic function $f$ satisfies the following conditions:
        \begin{gather}
            \pderiv{u}{x} = \pderiv{v}{y} \text{\qquad and\qquad} \pderiv{u}{y} = -\pderiv{v}{x}\,.
        \end{gather}
        These conditions can be combined into one equation using the so-called \textbf{Wirtinger derivative}:
        \begin{gather}
            \label{complex:holomorphic_alternative_condition}
            \pderiv{f}{\overline{z}} = 0\,.
        \end{gather}
    \end{property}

    \begin{theorem}[Looman--Menchoff\footnotemark]\index{Looman--Menchoff}\index{Cauchy--Goursat|see{Looman--Menchoff}}
        \footnotetext{This is the most general theorem on the holomorphy of continuous functions. It generalizes the original results by \textit{Riemann} and \textit{Cauchy--Goursat}.}
        Let $f$ be a continuous function defined on a subset $U\in\mathbb{C}$. If the partial derivatives of the real and imaginary part exist and if $f$ satisfies the Cauchy--Riemann conditions, then $f$ is holomorphic on $U$.
    \end{theorem}

    \begin{property}[Laplace equation]\index{harmonic!function}
        The functions $u,v$ satisfying the Cauchy--Riemann conditions are harmonic functions, i.e.~they satisfy the \textit{Laplace equation} (see \cref{pde:laplace_equation}).
    \end{property}
    \begin{property}[Level sets]
        The functions $u,v$ satisfying the Cauchy--Riemann conditions have orthogonal level curves (\cref{set:level_set}).
    \end{property}

    \begin{property}[Real functions]
        Consider a real-valued function $f$ defined on the complex plane. If it is holomorphic, the Cauchy--Riemann conditions imply that $f$ is a constant.
    \end{property}

    \begin{theorem}[Identity theorem]\index{identity!theorem}
        If two holomorphic functions on a domain $D$ coincide on a set containing an accumulation point of $D$, they coincide on all of $D$.
    \end{theorem}

\section{Contour integrals}

    \sremark{Whenever contours are considered for integration purposes, they have been chosen to be evaluated counter-clockwise (by convention). To obtain results concerning clockwise evaluation, most of the time adding a minus sign is sufficient.}

    \newdef{Contour integral}{\index{integral!contour}\label{complex:contour_integral}
        The contour integral of a complex function
        \begin{gather}
            f(z)=u(z)+iv(z)
        \end{gather}
        is defined as the following line integral:
        \begin{gather}
            \Int_{z_1}^{z_2}f(z)\,dz = \Int_{(x_1,y_1)}^{(x_2,y_2)}\bigl(u(x,y) + iv(x,y)\bigr)(dx+idy)\,.
        \end{gather}
    }

    \begin{theorem}[Cauchy's integral theorem\footnotemark]\index{Cauchy!integral theorem}\index{rectifiable curve}\index{Cauchy--Goursat}\label{complex:cauchy_integral_theorem}
        \footnotetext{Also called the \textbf{Cauchy--Goursat theorem}.}
        Let $\Omega$ be a simply-connected subset of $\mathbb{C}$ and let $f:\mathbb{C}\rightarrow\mathbb{C}$ be a holomorphic function on $\Omega$. The contour integral around every closed, rectifiable (i.e.~of finite length) contour $C$ in $\Omega$ vanishes:
        \begin{gather}
            \Oint_Cf(z)\,dz = 0\,.
        \end{gather}
    \end{theorem}
    \begin{result}[Freedom of contour]
        The contour integral of a holomorphic function depends only on the limits of integration and not on the contour connecting them.
    \end{result}

    \begin{formula}[Cauchy's integral formula]\index{Cauchy!integral formula}\label{complex:cauchy_integral_formula}
        Let $\Omega$ be a connected subset of $\mathbb{C}$ and let $C$ be a closed contour in $\Omega$. If $f:\mathbb{C}\rightarrow\mathbb{C}$ is holomorphic on $\Omega$, then, at every point $z_0$ inside $C$, one can express $f$ as follows:
        \begin{gather}
            f(z_0) = \frac{1}{2\pi i}\Oint_C\,\frac{f(z)}{z-z_0}\,dz\,.
        \end{gather}
    \end{formula}

    \begin{result}[Analytic function]\index{analytic!function}\label{complex:cauchy_integral_formula_derivative}
        Let $\Omega$ be a connected subset of $\mathbb{C}$ and let $C$ be a closed contour in $\Omega$. If $f:\mathbb{C}\rightarrow\mathbb{C}$ is holomorphic on $\Omega$, then $f$ is analytic (\cref{calculus:analytic}) on $\Omega$ and
        \begin{gather}
            f^{(n)}(z_0) = \frac{1}{2\pi i}\Oint_Cf(z)\frac{n!}{(z-z_0)^{n+1}}\,dz\,.
        \end{gather}
        Furthermore, the derivatives are also holomorphic on $\Omega$.
    \end{result}

    \begin{theorem}[Morera]\index{Morera}
        If $f:\mathbb{C}\rightarrow\mathbb{C}$ is continuous on a connected open set $\Omega$ and
        \begin{gather}
            \Oint_Cf(z)\,dz=0
        \end{gather}
        for every closed contour $C$ in $\Omega$, then $f$ is holomorphic on $\Omega$.
    \end{theorem}

    \begin{theorem}[Liouville]\index{Liouville!theorem on entire functions}
        Every bounded, entire function is constant.
    \end{theorem}

    \begin{theorem}[Sokhotski--Plemelj]\index{Sokhotski--Plemelj}\label{complex:sokhotski_plemelj}
        Let $f:\mathbb{R}\rightarrow\mathbb{C}$ be continuous and consider $a<0<b$, then
        \begin{gather}
            \lim_{\varepsilon\rightarrow0^+}\Int_a^b\frac{f(x)}{x\pm i\varepsilon}\,dx = \mp i\pi f(0) + \mathcal{P}\Int_a^b\frac{f(x)}{x}\,dx\,,
        \end{gather}
        where $\mathcal{P}$ denotes the Cauchy principal value.

        \todo{ADD (Cauchy principal value)}
    \end{theorem}

\section{Laurent series}

    \begin{definition}[Laurent series]\index{Laurent!series}\index{annulus}\index{principal!part}\label{complex:laurent_series}
        If $f:\mathbb{C}\rightarrow\mathbb{C}$ is an analytic function defined on an \textbf{annulus}, i.e.~a ring-shaped region, then $f$ can be expanded as the following series:
        \begin{gather}
            f(z) = \sum^{+\infty}_{n=-N}a_n(z-z_0)^n \qquad\text{with}\qquad a_n = \frac{1}{2\pi i}\Oint\frac{f(z)}{(z-z_0)^{n+1}}\,dz\,,
        \end{gather}
        The subseries containing all terms of negative degree is called the \textbf{principal part} of the Laurent series. 
    \end{definition}
    \begin{notation}
        The ring of Laurent series in the indeterminate $z$ is given by the ring $\mathbb{C}[[z,z^{-1}]]$.
    \end{notation}

    \begin{remark}[Multiplication]
        This definition can easily be generalized from $\mathbb{C}$ to general rings $R$. However, to be able to multiply two Laurent series, one has to restrict to series with a finite number of negative powers (unless $R$ is topological such that convergence can be defined). This gives rise to the ring of \textbf{formal Laurent series} $R((z))$. This ring can be obtained as the fraction field of $R[[z]]$ (\cref{algebra:fraction_field}).
    \end{remark}

    \begin{property}[Convergence]
        The Laurent series of an analytic function $f:\mathbb{C}\rightarrow\mathbb{C}$ converges uniformly to $f$ on the annulus $R_1 < |z-z_0| < R_2$, with $R_1$ and $R_2$ the distances from $z_0$ to the two closest \textit{poles} (see \cref{complex:pole} further below).
    \end{property}

    \newdef{Analytic continuation}{\index{analytic!continuation}
        Consider an analytic function $f:\mathbb{C}\rightarrow\mathbb{C}$ defined on an open subset $U\subset\mathbb{C}$. If $V\subset\mathbb{C}$ is an open subset containing $U$ and if there exists an analytic function $F$ on $V$ such that $F(z)=f(z)$ for all $z\in U$, then $F$ is called the analytic continuation of $f$ to $V$. Using the identity theorem for holomorphic functions, one can prove that analytic continuations are unique (on connected domains).
    }

    \begin{theorem}[Schwarz's reflection principle]\index{Schwarz!reflection principle}
        Let $f:\mathbb{C}\rightarrow\mathbb{C}$ be analytic on the upper half plane. If $z\in\mathbb{R}\implies f(z)\in\mathbb{R}$, then
        \begin{gather}
            f\bigl(\overline{z}\bigr) = \overline{f(z)}\,.
        \end{gather}
    \end{theorem}

\section{Singularities}
\subsection{Poles}

    \newdef{Pole}{\index{pole}\label{complex:pole}
        A function $f:\mathbb{C}\rightarrow\mathbb{C}$ has a pole of order $m\in\mathbb{N}_0$ at a point $z_0\in\mathbb{C}$ if its Laurent series at $z_0$ satisfies $\forall n<-m:a_n = 0$ and $a_{-m}\neq0$.
    }

    \newdef{Meromorphic}{\index{meromorphic}
        A function $f:\mathbb{C}\rightarrow\mathbb{C}$ is said to be meromorphic if it is analytic on the whole complex plane with exception of isolated poles and removable singularities. Every meromorphic function can be written as a fraction of two holomorphic functions, where the poles coincide with the zeros of the denominator.
    }

    \newdef{Essential singularity}{\index{essential!singularity}
        A function $f$ has an essential singularity at a point $z_0$ if its Laurent series at $z_0$ satisfies $\forall n\in\mathbb{N}:a_{-n}\neq0$, i.e.~if its Laurent series has infinitely many negative degree terms.
    }

    \newmethod{Frobenius transformation}{\index{Frobenius!transformation}
        To study the behaviour of a function $f$ at $z\longrightarrow\infty$, one can apply the Frobenius transformation $h=1/z$ and study the limit $\lim_{h\rightarrow0}f(h)$. For example, a singularity at $\infty$ is defined as a singularity of $f(1/z)$ at 0.
    }
    \begin{property}[Polynomials]\index{poly-!nomial}
        An entire function $f:\mathbb{C}\rightarrow\mathbb{C}$ is polynomial if and only if it has a pole at $\infty$.
    \end{property}

    \begin{theorem}[Casorati--Weierstrass]\index{Casorati--Weierstrass}
        Let $f:\mathbb{C}\rightarrow\mathbb{C}$ be holomorphic on the punctured open set $U\backslash\{z_0\}$ with an essential singularity at $z_0$. For every neighbourhood $V$ of $z_0$ contained in $U$, the image $f(V\backslash\{z_0\})$ is dense in $\mathbb{C}$.
    \end{theorem}
    \begin{result}
        If $f:\mathbb{C}\rightarrow\mathbb{C}$ is a nonpolynomial, entire function, then, for every $c\in\mathbb{C}$, there exists a sequence $z_n\longrightarrow\infty$ such that $f(z_n)\longrightarrow c$.\footnote{Polynomials are excluded due to the property above.}
    \end{result}

    \begin{theorem}[Picard's little theorem]\index{Picard}
        The range of a nonconstant, entire function is the complex plane with at most a single exception.
    \end{theorem}
    \begin{theorem}[Picard's great theorem]
        Let $f:\mathbb{C}\rightarrow\mathbb{C}$ be an analytic function with an essential singularity at $z_0$. On every punctured neighbourhood of $z_0$, $f$ takes on all possible values, with at most a single exception, infinitely many times.
    \end{theorem}

\subsection{Branch cuts}

    \newformula{Roots}{\index{root}
        Let $z\in\mathbb{C}$. The $n^{\text{th}}$ roots\footnote{Also see the fundamental theorem of algebra (\cref{alggeom:fundamental_theorem_of_algebra}).} of $z = re^{i\theta}$ are given by
        \begin{gather}
            \left\{\sqrt[n]{r}\exp\left(\frac{\theta + 2\pi k}{n}i\right)\,\middle\vert\,k\in\{0,1,\ldots,n\}\right\}\,.
        \end{gather}
    }
    \newformula{Complex logarithm}{\index{logarithm}
        The natural logarithm can be continued to the complex plane (as a multi-valued function) as follows:
        \begin{gather}
            \mathrm{LN}(z) := \bigl\{\ln(r) + i(\theta + 2\pi k)\bigm\vert k\in\mathbb{Z}\bigr\}\,.
        \end{gather}
    }

    \newdef{Branch}{\index{branch}
        The problem with the previous two formulas is that they represent multi-valued functions. To get an unambiguous image it is necessary to fix a value of the parameter $k$. By doing so there will arise curves, called \textbf{branch cuts}, in the complex plane where the function becomes discontinuous. A \textbf{branch} is defined as a particular choice of the parameter $k$.

        For the logarithm, the choice for $\arg(\mathrm{LN})\in\ ]\alpha, \alpha + 2\pi]$ is often denoted by $\mathrm{LN}_\alpha$ or $\log_\alpha$.
    }
    \newdef{Principal value}{\index{principal!value}
        The principal value\footnote{Not to be confused with the Cauchy principal value.} of a multi-valued complex function is defined as the value associated with a choice of branch for which $\arg(f)\in\ ]\!-\pi,\pi]$.
    }

    \newdef{Branch point}{
        Let $f$ be a complex-valued function. A point $z_0$ for which there exists no neighbourhood on which $f$ is single-valued is called a branch point.
    }
    \newdef{Branch cut}{
        A line connecting exactly two branch points, one possibly being $\infty$, is called a branch cut. In case there exist multiple branch cuts, they are required to never cross.
    }

    \begin{example}
        Consider the complex function
        \begin{gather}
            f(z) = \frac{1}{\sqrt{(z-z_1)\cdots(z-z_n)}}\,.
        \end{gather}
        This function has singularities at $z_1,\ldots,z_n$. If $n$ is even, this function will have $n$ (finite) branch points. This implies that the points can be grouped in pairs connected by non-intersecting branch cuts. If $n$ is odd, this function will have $n$ (finite) branch points and one branch point at infinity. The finite branch points will be grouped in pairs connected by non-intersecting branch cuts and the remaining branch point will be joined to infinity by a branch cut that does not intersect the others.
    \end{example}

\subsection{Residue theorem}\index{residue}

    \newdef{Residue}{\label{complex:residue_def}
        By applying \cref{complex:contour_integral} to a polynomial function, one finds
        \begin{gather}
            \Oint_C(z-z_0)^n\,dz = 2\pi i\delta_{n,-1}\,,
        \end{gather}
        where $C$ is a contour around the pole $z=z_0$. This means that integrating a Laurent series around a pole isolates the coefficient $a_{-1}$. This coefficient is, therefore, called the residue of the function at the given pole.
    }
    \begin{notation}
        The residue of a complex function $f:\mathbb{C}\rightarrow\mathbb{C}$ at a pole $z_0$ is denoted by
        \begin{gather}
            \mathrm{Res}[f(z)]_{z=z_0}\,.
        \end{gather}
    \end{notation}

    \begin{formula}
        For a pole of order $m\in\mathbb{N}_0$, the residue is calculated as follows:
        \begin{gather}
            \label{complex:residue}
            \mathrm{Res}\left[f(z)\right]_{z=z_j} = \lim_{z\rightarrow z_0}\frac{1}{(m - 1)!} \left(\pderiv{}{z}\right)^{m-1}\left(f(z)(z-z_0)\right)\,.
        \end{gather}
        For essential singularities, the residue can be found by writing out the Laurent series explicitly.
    \end{formula}

    \begin{theorem}[Residue theorem]\label{complex:residue_theorem}
        If $f:\mathbb{C}\rightarrow\mathbb{C}$ is meromorphic on $\Omega\subseteq\mathbb{C}$ and if $C$ is a closed contour in $\Omega$ that contains the poles $z_j$ of $f$, then
        \begin{gather}
            \Oint_Cf(z)\,dz = 2\pi i\sum_j\mathrm{Res}\left[f(z)\right]_{z=z_j}\,.
        \end{gather}
        For poles on the contour $C$, only half of the residue contributes to the integral.
    \end{theorem}

    \begin{formula}[Argument principle]\index{argument!principle}
        Let $f:\mathbb{C}\rightarrow\mathbb{C}$ be meromorphic and denote the number of zeros and poles of $f$ inside the contour $C$ by $Z_f(C)$ and $P_f(C)$, respectively. From the residue theorem, one can derive the following formula:
        \begin{gather}
            \frac{1}{2\pi i}\Oint_C\frac{f(z)}{f'(z)}\,dz = Z_f(C) - P_f(C)\,.
        \end{gather}
    \end{formula}
    \begin{definition}[Winding number]\index{winding number}\index{index!of map}
        Let $f:\mathbb{C}\rightarrow\mathbb{C}$ be meromorphic and let $C$ be a simple closed contour. For all $a\not\in f(C)$, the winding number, also called the \textbf{index}, of $a$ with respect to the function $f$ is defined as follows:
        \begin{gather}
            \mathrm{Ind}_f(a) := \frac{1}{2\pi i}\Oint_C\frac{f'(z)}{f(z) - a}\,dz\,.
        \end{gather}
        This number is always an integer.
    \end{definition}

\section{Limit theorems}

    \begin{theorem}[Small limit theorem]\index{limit!theorem}\label{complex:small_limit}
        Let $f:\mathbb{C}\rightarrow\mathbb{C}$ be a function that is holomorphic almost everywhere and let the contour $C$ be a circular segment with radius $\varepsilon$ and central angle $\alpha$. If $z$ is parametrized as $z=\varepsilon e^{i\theta}$, then
        \begin{gather}
            \Oint_Cf(z)\,dz = i\alpha A
        \end{gather}
        with
        \begin{gather}
            A = \lim_{\varepsilon\rightarrow0}f(z)\,\,.
        \end{gather}
    \end{theorem}

    \begin{theorem}[Great limit theorem]\label{complex:great_limit}
        Let $f:\mathbb{C}\rightarrow\mathbb{C}$ be a function that is holomorphic almost everywhere and let the contour $C$ be a circular segment with radius $R$ and central angle $\alpha$. If $z$ is parametrized as $z=Re^{i\theta}$, then
        \begin{gather}
            \Oint_Cf(z)\,dz = i\alpha B
        \end{gather}
        with
        \begin{gather}
            B = \lim_{R\rightarrow\infty}f(z)\,.
        \end{gather}
    \end{theorem}

    \begin{theorem}[Jordan's lemma]\index{Jordan}\label{complex:jordan}
        Let $g:\mathbb{C}\rightarrow\mathbb{C}$ be a continuous function that can be written as $g(z) = f(z)e^{bz}$ and let the contour $C$ be a semicircle lying in the half-plane bounded by the real axis and oriented away of the point $i\overline{b}$. If $z$ is parametrized as $z=Re^{i\theta}$ and
        \begin{gather}
            \lim_{R\rightarrow\infty}f(z) = 0\,,
        \end{gather}
        then
        \begin{gather}
            \Oint_Cg(z)\,dz = 0\,.
        \end{gather}
    \end{theorem}
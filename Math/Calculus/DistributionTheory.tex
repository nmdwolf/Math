\chapter{Distributions}\label{chapter:distributions}

	The main references for this chapter are~\cite{georgiev_theory_2015,choquet-bruhat_analysis_1991,choquet-bruhat_analysis_2000}. Although this chapter is technically part of functional analysis and, hence, uses the language of normed spaces (see \cref{chapter:functional}), it is located in the part on calculus due to its strong relation to measure and integration theory. (Terminology from the chapter on functional analysis will not be indicated in italic for simplicity.)

    \minitoc

\section{Distributions}

	\newdef{Distribution}{\index{distribution}\index{generalized!function}\index{bump function}\index{test!function|see{bump function}}\label{distribution:distribution}
		The space of distributions or \textbf{generalized functions} on an open set $U\subset\mathbb{R}^n$ is defined as the set of continuous linear functionals on $\mathcal{D}(U):=C^\infty_c(U)$, the space of smooth functions with compact support (also called \textbf{bump functions} or \textbf{test functions}).

        First, $\mathcal{D}(U)$ has to be endowed with a topology. For every compact set $K\subset U$ and every $m\in\mathbb{N}$, a locally convex topology (\ref{functional:locally_convex_seminorm}) on $\mathcal{D}^m_K(U):=C^m_K(U)$ is constructed using the following family of seminorms:
        \begin{gather}
            \mathcal{P}=\left\{\sup_{x\in K}\|f^{(i)}(x)\|\,\middle\vert\,|i|\leq m\right\}\,.
        \end{gather}
        A topology on all of $\mathcal{D}^m(U)$ is then defined as the inductive limit (\cref{topology:final_topology}) over all compact subsets $K\subset U$, i.e.~a subset of $\mathcal{D}^m(U)$ is open if and only if its intersection with all $\mathcal{D}^m_K(U)$ is open. All of these topologies are Fr\'echet (see \cref{functional:frechet_space}). A topology on $\mathcal{D}(U)$ is obtained by taking a further inductive limit of the $\mathcal{D}^m(U)$ over $m\in\mathbb{N}$.

		The dual space $\mathcal{D}'(U)$ is equipped with the weak-* topology (see \cref{functional:weak_star_topology}) and, accordingly, a sequence of distributions $\seq{\phi}$ converges to a distribution $\phi$ if and only if $\langle\phi_n,f\rangle\longrightarrow\langle\phi,f\rangle$ for all $f\in\mathcal{D}(U)$. This definition immediately implies that two distributions $\phi,\psi$ are equal if and only if $\langle\phi,f\rangle = \langle\psi,f\rangle$ for all $f\in\mathcal{D}(U)$. Note that $\mathcal{D}(U)$ is not Fr\'echet in contrast to $C^\infty(U)$, where a countable family of seminorms is obtained by taking $K$ to be closed balls of radius $k$.
	}

    \begin{property}[Equivalent seminorms]
        The seminorms used in the definition of the locally convex topology on $\mathcal{D}(U)$ can be replaced by the following equivalent ones:
        \begin{gather}
            \label{distribution:D_seminorm}
            \begin{cases}
                &\sup_{|i|\leq m}\sup_{x\in K}\|f^{(i)}(x)\|\,,\\
                &\sup_{x\in K}\sum_{|i|\leq m}\|f^{(i)}(x)\|\,,\\
                &\sum_{|i|\leq m}\sup_{x\in K}\|f^{(i)}(x)\|\,.
            \end{cases}
        \end{gather}
    \end{property}

	\begin{property}
		A linear functional $\phi$ on $\mathcal{D}(U)$ is a distribution if and only if it satisfies one of the following equivalent statements:
		\begin{itemize}
            \item It is continuous when restricted to every $\mathcal{D}_K(U)$ for $K\subset U$ compact.
			\item If the sequence $\seq{f}$ converges to 0 in $\mathcal{D}(U)$, then $\langle\phi,f_n\rangle\longrightarrow0$.
			\item For every compact subset $K\subset U$, there exist a constant $C_K>0$ and an integer $m_K\geq0$ such that
			\begin{gather}
				|\langle\phi,f\rangle|\leq C_K\,p_{K,m_K}(f)
			\end{gather}
			for all $f\in\mathcal{D}_K(U)$.
		\end{itemize}
	\end{property}

	\newdef{Order}{\index{order!of a distribution}
		The order of a distribution $\phi$ is the smallest integer $m$ such that
		\begin{gather}
			|\langle\phi,f\rangle|\leq C_K\,p_{K,m}(f)
		\end{gather}
		for all $f\in\mathcal{D}_K(U)$ and all compact subsets $K\subset U$. Note that the integer $m$ is independent of the compact set $K$.
	}

	\begin{property}
		A distribution is of order $k$ if and only if it can be (uniquely) extended to a continuous linear functional on $\mathcal{D}^k(U)\equiv C^k_c(U)$.
	\end{property}

    \begin{theorem}[Riesz--Markov--Kakutani]\index{Riesz--Markov--Kakutani}\index{vanish at infinity}\label{distributions:riesz_markov}
        The space of positive continuous functionals on $C_c(X)$, i.e.~the space of continuous functions with compact support on a locally compact Hausdorff space $X$, is homeomorphic to the space of Radon measures (\cref{measure:radon_measure}) on $X$. Every such functional $\Lambda$ can be represented as
        \begin{gather}
            \Lambda(f) = \Int_Xf\,d\mu
        \end{gather}
        for some Radon measure $\mu$.

        The topological dual of $C_0(X)\equiv C(\widehat{X})$, the continuous functions on the one-point compactification (\cref{topology:alexandrov_compactification}), i.e.~those functions that \textbf{vanish at infinity}, is isometrically isomorphic to the space of finite signed Radon measures equipped with the total variation norm (\cref{measure:total_variation}).
    \end{theorem}

    \begin{example}[Ordinary function as generalized function]\label{distributions:ordinary_function}
       	By \cref{measure:measure_by_integral}, every locally integrable function $f\in L^1_{\text{loc}}$ gives rise to a distribution:
       	\begin{gather}
   	    	\langle f,g \rangle = \Int_{\mathbb{R}}f(x)g(x)\,dx\,.
       	\end{gather}
       Distributions of this form are also said to be \textbf{regular}. These distributions are of order 0.
   	\end{example}

    \begin{property}
        The space $\mathcal{D}$ is dense in $\mathcal{D}'$.
    \end{property}

    The density of regular distributions can be used to generalize operations to all of $\mathcal{D}$.
    \begin{property}[Product with smooth functions]
        For every smooth function $f$ and every distribution $\phi$, the product $f\phi$ is defined as
        \begin{gather}
            \langle f\phi,g \rangle := \langle\phi,fg\rangle\,.
        \end{gather}
        This turns $\mathcal{D}'$ into a $C^\infty$-module.
    \end{property}

\subsection{Derivatives}

	\newdef{Derivative of a distribution}{\index{derivative!of distributions}\index{derivative!weak}\label{distributions:weak_derivative}
		The derivative of a distribution $\phi$ is defined by duality:
		\begin{gather}
			\left\langle\pderiv{\phi}{x},f\right\rangle := -\left\langle\phi,\pderiv{f}{x}\right\rangle\,.
		\end{gather}
		This formula is a reasonable definition, since if $\phi$ is regular, the above formula is the one obtained through integration by parts.

        In general, a function $g\in L^1_{\text{loc}}$ is said to be a \textbf{weak derivative} of a function $f\in L^1_{\text{loc}}$ if it satisfies the following equation for all $h\in\mathcal{D}$:
        \begin{gather}
            \langle f,h' \rangle = -\langle g,h \rangle\,.
        \end{gather}
	}

	\begin{property}[Smoothness]\index{smooth!distribution}
		Every distribution is smooth, i.e.~it is infinitely differentiable. Furthermore, it satisfies the conclusion of Schwarz's theorem~\ref{calculus:schwarz_theorem}.
	\end{property}
    \begin{property}[Constant distributions]
        If a distribution $T$ satisfies $T'=0$, it is a regular distribution induced by a constant function.
    \end{property}

    \newdef{Fundamental solution}{\index{fundamental!solution}\label{distributions:fundamental_solution}
        Let $D$ be a differential operator. A fundamental solution for $D$ is a distribution $\phi$ such that
        \begin{gather}
            D\phi = \delta\,.
        \end{gather}-

        \todo{DEFINE differential operator IN CHAPTER ON PDEs}
    }

\subsection{Support}

	\newdef{Support}{\index{support}
		The support of a distribution is defined as the smallest closed set on which it does not vanish.
	}

	\begin{property}
		A distribution has compact support if and only if it can be extended to a continuous linear functional on $C^\infty(U)$. This gives a nice duality: distributions act on compactly supported functions and compactly supported distributions act on functions.
	\end{property}

	\begin{property}[Order]
		Distributions with compact support have finite order.
	\end{property}

	\begin{property}
		A distribution that is supported only at 0 can be written as a linear combination of derivatives of the Dirac distribution. More generally, a distribution with support at a finite set of points can be written as a linear combination of (shifted) Dirac distributions.
	\end{property}

    \newdef{Singular support}{\index{support!singular}\label{distributions:singular_support}
        The complement of the largest open set on which a distribution is regular. The singular support of $\phi$ is denoted by $\mathrm{sing\ supp}(\phi)$.
    }

\subsection{Examples}

	\newdef{Heaviside distribution}{\index{Heaviside!function}\label{distribution:heaviside_function}
    	The Heaviside function is defined as follows:\footnote{The case $x=0$ is often left undefined, but since this function will always enter formulas inside an integral this does not matter.}
    	\begin{gather}
			H(x) :=
			\begin{cases}
				0&\cif x<0\\
				1&\cif x>0
			\end{cases}
		\end{gather}
        From this definition, it follows that for every $f\in\mathcal{D}(U)$:
       	\begin{gather}
       		\label{distribution:heaviside_function_integral}
			\langle H,f \rangle = \Int_0^{+\infty}f(x)\,dx\,.
		\end{gather}
	}

	\newdef{Dirac delta distribution}{\index{Dirac!delta function}\label{distribution:dirac_delta}
    	The Dirac delta distribution is defined as the weak derivative of the Heaviside function:
        \begin{align*}
			\langle \delta,f \rangle&:=\langle H',f \rangle\\
       		&=-\langle H,f' \rangle\\
			&=-\ds\Int_0^{+\infty}f'(x)\,dx\\
       		&=f(0)\,.
		\end{align*}
	}

	\begin{property}[Sampling property]\label{distribution:sieving_dirac_delta}
    	The previous definition can be generalized in the following way (whenever $x_0\in U$):
    	\begin{gather}
			f(x_0) = \Int_Uf(x)\delta(x - x_0)\,dx=\Int_Uf\,d\delta_{x_0}\,,
		\end{gather}
		where the suggestive notation\footnote{See the section on kernels further on.} $\delta(x-x_0)$ was used to denote the Dirac delta distribution with support at $x_0$.
	\end{property}

	\newdef{Dirac comb}{\index{Dirac!comb}\label{distribution:dirac_comb}
    	\begin{gather}
			\mathrm{III}_b(x) := \sum_{n=-\infty}^{+\infty}\delta(x-nb)
		\end{gather}
    }

	\begin{property}[Transformation]\label{distribution:delta_of_function}
		Let $f(x)\in C^1(\mathbb{R})$ be a function with $n\in\mathbb{N}$ roots $x_i$ such that $f'(x_i)\neq0$. The Dirac delta distribution has the following property:
		\begin{gather}
			\delta\bigl(f(x)\bigr) = \sum_{i=1}^n\frac{\delta(x-x_i)}{|f'(x_i)|}\,.
		\end{gather}
	\end{property}

	\newformula{Differentiation across discontinuities}{\index{derivative}\index{jump}
    	Let $f$ be a piecewise continuous function with discontinuities at $x_1,\ldots,x_n$. Define the \textbf{jumps} of $f$ at its discontinuities by $\sigma_i := f^+(x_i) - f^-(x_i)$. Next, define the (continuous) function
        \begin{gather}
            f_c(x) := f(x) - \sum_{i=1}^n\sigma_iH(x-x_i)\,.
        \end{gather}
        Differentiation of this formula gives
        \begin{gather}
            f'(x) = f'_c(x) + \sum_{i=1}^n\sigma_i\delta(x-x_i)\,.
        \end{gather}
        It follows that the derivative, in the generalized sense, of a piecewise continuous function equals the derivative in the classical sense plus a sum of delta functions at the jump discontinuities.
	}

    \begin{example}[Principal value]\index{principal!value}
        The function $\frac{1}{x}$ is clearly not integrable on $\mathbb{R}$. However, its Cauchy principal value exists. This procedure also defines a distribution:
        \begin{gather}
            \left\langle\mathcal{P}\frac{1}{x},f\right\rangle := \lim_{\varepsilon\downarrow0}\Int_\varepsilon^{+\infty}\frac{f(x)-f(x^-)}{x}\,dx.
        \end{gather}
        Moreover, this is the distributional derivative of $\ln|x|$.
    \end{example}

    \begin{formula}\label{distribution:delta_equation}
        Consider a polynomial $P(x)\in\mathbb{C}[x]$. Over $L^1$ (or any ordinary function space), the following equation would not have any solutions:
        \begin{gather}
            P(x)f(x) = 0\,,
        \end{gather}
        i.e.~the kernel of the multiplication operator by $P(x)$ is trivial. However, when passing to distributions $\phi\in\mathcal{D}'$, the situation changes. Over $\mathbb{C}$, $P$ splits and, hence, one obtains the following equation:
        \begin{gather}
            (x-\lambda_i)^{k_i}\cdots(x-\lambda_n)^{k_n}\phi = 0\,.
        \end{gather}
        for some $n,k_1,\ldots,k_n\in\mathbb{N}$ and $\lambda_1,\ldots,\lambda_n\in\mathbb{C}$. The solution space is spanned by the following distributions:
        \begin{gather}
            \phi\in\mathrm{span}_{\mathbb{C}}\left\{\delta(x-\lambda_1),\ldots,\delta^{(k_1)}(x-\lambda_1),\ldots,\delta(x-\lambda_n),\ldots,\delta^{(k_n)}(x-\lambda_n)\right\}\,,
        \end{gather}
        i.e.~it is spanned by delta distributions at the roots of the polynomial (and their derivatives up to order one less than the mulitplicity of the roots).
    \end{formula}

\subsection{Growth rates}

	\newdef{Schwartz space}{\index{Schwartz space}\label{distribution:schwartz_space}
		The Schwartz space of \textbf{rapidly decreasing functions} $\mathscr{S}(\mathbb{R}^n)$ is defined as follows:
		\begin{gather}
    		\mathscr{S}(\mathbb{R}^n) := \bigl\{f\in C^\infty(\mathbb{R}^n)\bigm\vert\forall i,j\in\mathbb{N}^n,\forall x\in\mathbb{R}^n:|x^if^{(j)}(x)|<+\infty\bigr\}\,,
		\end{gather}
        where, for every multi-index $i$, the symbol $x^i$ denotes the monomial $x_1^{i_1}x_2^{i_2}\cdots$. An equivalent condition is the following. For every $p\in\mathbb{N}$ and $j\in\mathbb{N}^n$, there exists a constant $M_{p,j}(f)$ such that
        \begin{gather}
            \sup_{x\in\mathbb{R}^n}\bigl(1+\|x\|\bigr)^p|f^{(j)}(x)|\leq M_{p,j}(f)\,.
        \end{gather}
        This space has the structure of a Fr\'echet space under the family of seminorms
        \begin{gather}
            s_{p,N}(f) := \sup_{x\in\mathbb{R}^n}\sup_{|j|\leq N}\bigl(1+\|x\|\bigr)^p|f^{(j)}(x)|\,.
        \end{gather}
	}
    \remark{These functions are said to be rapidly decreasing because every derivative $f^{(j)}(x)$ decays faster than any inverse power $x^i$ for $\|x\|\longrightarrow+\infty$.}

	\newdef{Functions of slow growth}{
		The space of functions of slow growth $\mathcal{N}(\mathbb{R}^n)$ is defined as follows:
		\begin{gather}
            \mathcal{N}(\mathbb{R}^n) := \bigl\{f\in C^\infty(\mathbb{R}^n)\bigm\vert\forall i\in\mathbb{N},\exists M_i>0:\bigl|f^{(i)}(x)\bigr|=O(\|x\|^i)\text{ for }\|x\|\longrightarrow+\infty\bigr\}\,.
		\end{gather}
	}

	\begin{property}
		If $f\in\mathscr{S}(\mathbb{R})$ and $f\in\mathcal{N}(\mathbb{R})$, then $fg\in\mathscr{S}(\mathbb{R})$.
	\end{property}

\section{Convolutions and kernels}

	\newdef{Direct product}{\index{direct product!of distributions}
		Consider two distributions $\phi\in\mathcal{D}'(U)$ and $\psi\in\mathcal{D}'(V)$. The direct product distribution $\phi\times\psi\in\mathcal{D}'(U\times V)$ is defined by one of the following two equivalent formulas:
		\begin{gather}
			\langle\phi\times\psi,f\rangle := \langle\phi,\langle\psi,f\rangle\rangle
		\end{gather}
		or
		\begin{gather}
            \langle\phi\times\psi,f\rangle := \langle\psi,\langle\phi,f\rangle\rangle\,.
		\end{gather}
	}

	\newdef{Convolution}{\index{convolution}
		The convolution of two distributions is defined as follows (if it exists):
		\begin{gather}
			\langle\phi\ast\psi,f\rangle := \langle\phi\times\psi,g\rangle\,,
		\end{gather}
		where $g(x,y) := f(x+y)$. It should be noted that the convolution is commutative.
	}
	\begin{example}[Convolution with delta distribution]
		For every distribution $\phi$, one has the following property:
		\begin{gather}
			\delta\ast\phi = \phi\,.
		\end{gather}
	\end{example}

    \begin{formula}[Convolution of functions]\label{distributions:function_convolution}
        The convolution $f\ast g$ of two (locally integrable) functions can be defined through \cref{distributions:ordinary_function}:
        \begin{gather}
            (f\ast g)(x) := \Int_{\mathbb{R}}f(y)g(x-y)\,dy\,.
        \end{gather}
    \end{formula}

    \begin{property}[Young inequality]\index{Young!inequality}
        If $f,g\in L^1$, then $f\ast g$ exists a.e. and
        \begin{gather}
            \|f\ast g\|_1\leq \|f\|_1\,\|g\|_1\,.
        \end{gather}
        This also implies that $f\ast g\in L^1$. Furthermore, consider $p,q$ and $r\in\ ]0,\infty]$ such that
        \begin{gather}
            \frac{1}{p}+\frac{1}{q} = \frac{1}{r}+1\,.
        \end{gather}
        If $f\in L^p$ and $g\in L^q$, then
        \begin{gather}
            \|f\ast g\|_r\leq\|f\|_p\,\|g\|_q\,.
        \end{gather}
        This also implies that $f\ast g\in L^r$. A result similar to \cref{measure:holders_inequality} holds for H\"older conjugates ($r=+\infty$). Their convolution is an element of $L^\infty$. Furthermore, the convolution is uniformly continuous on all of $\mathbb{R}^n$ and, if either $p>1$ or $q>1$, the convolution vanishes at infinity.
    \end{property}

    \begin{theorem}[Schwartz's kernel theorem]\index{Schwartz!kernel theorem}\label{distribution:kernel_theorem}
        There exists an isomorphism
        \begin{gather}
            D'(U\times V)\rightarrow D'(V,D'(U))
        \end{gather}
        given by
        \begin{gather}
            \langle\phi,f\times g\rangle = \langle T_\phi g,f \rangle\,.
        \end{gather}
    \end{theorem}

    \todo{COMPLETE (kernels, ...)}

\section{Transformations}
\subsection{Fourier series}

	\newdef{Dirichlet kernel}{\index{Dirichlet!kernel}\label{distributions:dirichlet_kernel}
   		The Dirichlet kernel is the collection of functions of the form:
        \begin{gather}
            D_n(x) := \frac{1}{2\pi}\sum_{k=-n}^ne^{ikx}
        \end{gather}
        for $n\in\mathbb{N}$.
	}
    \newformula{Sieve property}{
    	If $f\in C^1([-\pi,\pi])$, then
        \begin{gather}
        	\lim_{n\rightarrow\infty}\Int_{-\pi}^\pi f(x)D_n(x)\,dx = 0\,.
        \end{gather}
    }

	\newformula{Generalized Fourier series}{\index{Fourier!series}\label{distributions:fourier_series}
    	Let $f\in L^2([-l,l])$ be a $2l$-periodic function. This function can be approximated by the following series:
        \begin{gather}
            f(x) = \sum_{n=-\infty}^{+\infty}\,\left(\frac{1}{2l}\Int_{-l}^le^{-i\frac{n\pi x'}{l}}f(x')\,dx'\right)\,e^{i\frac{n\pi x}{l}}\,.
        \end{gather}
	}

    \begin{formula}[Fourier coefficients]\label{distributions:fourier_coefficients}
		As seen in the above formula, the Fourier coefficients can be calculated through the inner product~\eqref{functional:inner_product_L2} on $L^2$:
		\begin{gather}
       		\widetilde{f}(k) = \Int_{-l}^le_k^*(x)f(x)\,dx, \qquad\text{where}\qquad e_k := \sqrt\frac{1}{2l}e^{i\frac{k\pi x}{l}}\,.
		\end{gather}
	\end{formula}

    \begin{formula}
       	For $2\pi$-periodic functions, the order-$n$ Fourier approximation is given by the following convolution:
       	\begin{gather}
       		s_n(x) = \sum_{k=-n}^n\widetilde{f}(k)e^{ikx} = (D_n \ast f)(x)\,.
       	\end{gather}
    \end{formula}

    \begin{property}[Convergence of the Fourier series]
       	Let $f:\mathbb{R}\rightarrow\mathbb{R}$ be a $2\pi$-periodic function. If $f$ is piecewise $C^1$ on $[-\pi,\pi]$, then
        \begin{gather}
            (D_n\ast f)(x)\xrightarrow{\ n\longrightarrow\infty\ }\frac{f(x+)+f(x-)}{2}\,.
        \end{gather}
    \end{property}

	\newdef{Periodic extension}{\index{periodic!extension}
    	Let $f$ be piecewise $C^1$ on $[-L,L]$. The periodic extension $f^L$ is defined by gluing `copies' of $f$ together. The \textbf{normalized periodic extension} is defined as follows:
        \begin{gather}
        	f^{L,\nu}(x) := \frac{f^L(x^+) + f^L(x^-)}{2}\,.
        \end{gather}
    }
    \begin{property}
    	If $f$ is piecewise $C^1$ on $[-L,L]$, the Fourier series approximation of $f$ converges to $f^{L,\nu}$ on all of $\mathbb{R}$.
    \end{property}

\subsection{Fourier transform}\index{Fourier!transform}\label{section:fourier_transform}

    The Fourier series can be used to expand a $2l$-periodic function as an infinite series of exponentials. However, to expand a nonperiodic function $f\in L^1(\mathbb{R})$, one needs the integral Fourier transform.\footnote{All functions are required to be Lebesgue-integrable to make the integrals converge. Weaker conditions are possible (see the literature).}
    \begin{gather}
        \label{distributions:fourier}
        \mathcal{F}\!f(k) := \frac{1}{\sqrt{2\pi}}\Int_{\mathbb{R}}f(x)e^{-ikx}\,dx\,.
    \end{gather}
    The inverse transform, if it exists, is given by
    \begin{gather}
        \label{distributions:inverse_fourier}
        f(x) = \mathcal{F}^{-1}(\mathcal{F}\!f)(x) = \frac{1}{\sqrt{2\pi}}\ \mathcal{P}\Int_{\mathbb{R}}\mathcal{F}\!f(k)e^{ikx}\,dk\,.
    \end{gather}
    \Cref{distributions:fourier} is called the (forward) Fourier transform of $f$ and \cref{distributions:inverse_fourier} is called the inverse Fourier transform. The pair $(f,\mathcal{F}\!f)$ is called a \textbf{Fourier transform pair}.
    \begin{notation}
        The Fourier transform of a function $f$ is often denoted by $\widetilde{f}$ or $\widehat{f}$.
    \end{notation}

    \begin{property}
        From the Riemann--Lebesgue lemma~\ref{measure:riemann_lebesue_lemma}, it follows that
        \begin{gather}
            \mathcal{F}\!f(\omega)\longrightarrow0\qquad\text{if}\qquad |\omega|\longrightarrow0\,.
        \end{gather}
    \end{property}

    \begin{theorem}[Parceval]\index{Parceval}\label{distributions:parcevals_theorem}
        Let $(f,\widetilde{f})$ and $(g,\widetilde{g})$ be two Fourier transform pairs.
        \begin{gather}
            \Int_{\mathbb{R}} f(x)g(x)\,dx = \Int_{\mathbb{R}}\widetilde{f}(k)\widetilde{g}(k)\,dk
        \end{gather}
    \end{theorem}
    \begin{result}[Plancherel]\index{Plancherel}\label{distributions:plancherel_theorem}
        The integral of the square (of the modulus) of a Fourier transform is equal to the integral of the square (of the modulus) of the original function:
        \begin{gather}
            \Int_{\mathbb{R}}|f(x)|^2\,dx = \Int_{\mathbb{R}}|\widetilde{f}(k)|^2\,dk\,.
        \end{gather}
        This implies that the Fourier transform defines an isometry on $L^2$. In this case, it is often called the \textbf{Fourier--Plancherel transform}.
    \end{result}

    Now, one can wonder why the Fourier transform is introduced in this chapter. The reason is that this transformation can be generalized to distributions in a convenient way. Naively, one could try to extend the definition through duality, but for an arbitrary $\phi\in\mathcal{D}'$ it is not guaranteed that $\mathcal{F}\phi\in\mathcal{D}'$. This is where the Schwartz spaces come in.
    \begin{property}
        The Fourier transform defines an isomorphism on $\mathscr{S}$.
    \end{property}

    \newdef{Tempered distribution}{\index{distribution!tempered}
        The space of tempered distributions is the topological dual of the Schwartz space (with its Fr\'echet topology). It comes equipped with the weak-* topology.
    }

    This space has the following important property.
    \begin{property}
        $\mathcal{D}$ is dense in $\mathscr{S}$ and, hence, tempered distributions are determined by their values on $\mathcal{D}$.
    \end{property}

    \begin{property}\index{convolution!theorem}
        The Fourier transform of tempered distributions has some nice additional properties:
        \begin{itemize}
            \item The Fourier transform defines an isomorphism on $\mathscr{S}^*$.
            \item The Fourier transform of a compactly supported function is of slow growth.
            \item The Fourier transform of a convolution is equal to the product of the individual Fourier transforms. (Here, one should restrict to the case of a compactly supported and a tempered distribution such that the convolution is also tempered.) This is the \textbf{convolution theorem}:
            \begin{gather}
                \label{distributions:convolution_theorem}
                \mathcal{F}(\phi\ast\psi) = \mathcal{F}\phi\ast\mathcal{F}\psi\,.
            \end{gather}
        \end{itemize}
    \end{property}

    The Fourier transform also induces the following isomorphisms.
    \begin{theorem}[Paley--Wiener\footnotemark]\index{Paley--Wiener}\label{distributions:paley_wiener}
        \footnotetext{This version, stated in terms of distributions, is actually due to \textit{Schwartz}.}
        The space of compactly supported distributions of order $N$ is isomorphic to the space of entire functions satisfying
        \begin{gather}
            |F(z)|\leq C(1+|z|)^Ne^{b\,|\!\im(z)|}
        \end{gather}
        for some constants $b,C\in\mathbb{R}^+$ such that the distributions have support on $\overline{B}(0,b)$.

        The space of regular, compactly supported distributions is isomorphic to the space of entire functions satisfying
        \begin{gather}
            \forall N\in\mathbb{N},\exists C_N\in\mathbb{R}^+:|F(z)|\leq C_N(1+|z|)^{-N}e^{b\,|\!\im(z)|}\,,
        \end{gather}
        for some constant $b\in\mathbb{R}^+$ such that the distributions have support on $\overline{B}(0,b)$.
    \end{theorem}

    The Fourier transform~\ref{distributions:fourier} can be generalized to finite measures.
    \newdef{Fourier--Stieltjes transform}{\index{Fourier--Stieltjes transform}\label{distributions:fourier_stieltjes}
        Let $\mu$ be a finite Borel measure on $\mathbb{R}^n$. The Fourier(--Stieltjes) transform of $\mu$ is defined as follows:
        \begin{gather}
            \widehat{\mu}(\omega) := \Int_Xe^{-2\pi ix\cdot\omega}\,d\mu(x)\,.
        \end{gather}
    }

\subsection{Wave front sets}

    By the Paley--Wiener--Schwartz theorem~\ref{distributions:paley_wiener}, the regularity of a distribution can be characterized in terms of a growth condition on its Fourier transform. This approach can also be used to characterize in which directions one can move without losing regularity of a distribution.

    \newdef{Singular fibre}{\index{cutoff function}\label{distributions:cutoff_function}
        Consider a distribution $\phi$ on $\mathbb{R}^n$. Its singular fibre $\Sigma_x(\phi)$ at a point $x\in\mathbb{R}^n$ is the complement of the set of vectors $v\in\mathbb{R}^n$ for which its Fourier transform satisfies the Palais--Wiener estimate
        \begin{gather}
            |\mathcal{F}(\eta\phi)(w)|\leq C_N\bigl(1+\|w\|\bigr)^{-N}
        \end{gather}
        for all vectors $w\in\Gamma$ in a conical neighbourhood (see \cref{functional:locally_convex}) of $v$ and all \textbf{cut-off functions} $\eta$, i.e.~functions that are equal to 1 on a compact neighbourhood of $x$ and equal to 0 outside of some larger compact set.

        This definition says that the singular fibre of a distribution at some point contains those directions in which the Fourier distribution (localized around the point) becomes singular and, hence, indicates that the function is not smooth along these directions.
    }
    \begin{property}
        The singular fibre is itself a conical subset.
    \end{property}

    \newdef{Wave front set}{\index{wave!front}\label{distributions:wave_front_set}
        Consider a distribution $\phi$ on $\mathbb{R}^n$. The wave front set of $\phi$ is defined as follows:
        \begin{gather}
            \mathrm{WF}(\phi) := \bigl\{(x,v)\in\mathbb{R}^n\times\mathbb{R}^n\backslash{0}\bigm\vert v\in \Sigma_x(f)\bigr\}\,.
        \end{gather}
        The projection of the wave front set onto its first argument recovers the singular support (\cref{distributions:singular_support}).
    }

    The following property establishes the result hinted at above.
    \begin{property}[Regularity]
        A compactly supported distribution is regular if and only if its wave front set is empty.
    \end{property}

    \begin{property}[Derivatives]
        The wave front set of the derivative of a distribution is contained in the wave front set of the distribution.
    \end{property}

\subsection{Laplace transform}\label{section:laplace_transform}

    \newformula{Laplace transform}{\index{Laplace!transform}\label{distributions:laplace}
        \begin{gather}
            \mathcal{L}\{f\}(s) := \Int_{0}^{+\infty}f(t)e^{-st}\,dt
        \end{gather}
	}

	\newformula{Bromwich integral}{\index{Bromwich integral}\label{distributions:inverse_laplace}
        \begin{gather}
            f(t) = \frac{1}{2\pi i}\Int_{\gamma-i\infty}^{\gamma+i\infty}\mathcal{L}\{f\}(s)e^{st}\,ds
        \end{gather}
	}

    Just as in the case of the Fourier transform (\cref{distributions:fourier_stieltjes}), the Laplace transform can be generalized to finite measures.
    \newdef{Laplace--Stieltjes transform}{\index{Laplace--Stieltjes transform}\label{distributions:laplace_stieltjes}
        Let $\mu$ be a finite Borel measure on $\mathbb{R}$. The Laplace(--Stieltjes) transform of $\mu$ is defined as follows:
        \begin{gather}
            \mathcal{L}\{\mu\}(s) := \Int_{\mathbb{R}}e^{-st}\,d\mu(t)\,.
        \end{gather}
    }

\subsection{Integral representations}

    \todo{ADD LINK WITH Gelfan'd transform}

	\newformula{Mellin transform}{\index{Mellin}\label{distributions:mellin}
    	\begin{gather}
    		\mathcal{M}\{f\}(s) := \Int_0^{+\infty}x^{s-1}f(x)\,dx
    	\end{gather}
	}
	\newformula{Inverse Mellin transform}{\label{distributions:inverse_mellin}
		\begin{gather}
			f(x) = \frac{1}{2\pi i}\Int_{\gamma-i\infty}^{\gamma+i\infty}\mathcal{M}\{f\}_{(s)}x^{-s}\,ds
		\end{gather}
	}

	\newformula{Heaviside step function}{\index{Heaviside!step function}
		\begin{gather}
			\theta(x) = \frac{1}{2\pi i}\Int_{\mathbb{R}}\frac{e^{ikx}}{k-i\varepsilon}\,dk
		\end{gather}
	}
	\newformula{Dirac delta distribution}{\index{Dirac!delta function}
		\begin{gather}
			\delta(x) = \frac{1}{2\pi}\Int_{\mathbb{R}} e^{ikx}\,dk
		\end{gather}
	}

\section{\difficult{Harmonic analysis}}\index{harmonic!analysis}

    \newdef{Haar measure}{\index{Haar measure}\label{distributions:haar}
        A left (resp.~right) Haar measure on a topological group is a regular Borel measure (\cref{measure:regular_measure}) that is finite on compact subsets and invariant under the left (resp.~right) group action. For locally compact groups, this is a left-invariant (resp.~right-invariant) Radon measure (\cref{measure:radon_measure}).
    }
    \begin{example}[Lebesgue measure]
        Consider $\mathbb{R}^n$ as an additive group. \Cref{measure:translation_invariant} implies that the Lebesgue measure is a left (and right) Haar measure.
    \end{example}

    \begin{theorem}[Haar\footnotemark]\index{Haar}\label{distribution:haar_theorem}
        \footnotetext{A similar theorem holds for right Haar measures.}
        If $G$ is locally compact, there exists a left Haar measure that is unique up to a scalar factor. Moreover, if $G$ is compact, this constant can be fixed by requiring the normalization condition $\mu(G) = 1$.
    \end{theorem}

    \newdef{Pontryagin dual}{\index{Pontryagin!dual}\index{character}\label{distribution:character}
        Let $G$ be a locally compact Abelian group. Its (Pontryagin) dual is defined as the group of continuous homomorphisms from $G$ to the circle group:
        \begin{gather}
            G^\vee := \hom(G,S^1)\,.
        \end{gather}
        This group is endowed with the compact-open topology (\cref{topology:compact_open_topology}). Elements of this group are called \textbf{group characters} of $G$.
    }
    \begin{theorem}[Pontryagin duality]
        There exists a natural isomorphism $G\cong G^{\vee\vee}$.
    \end{theorem}

    \begin{construct}[Fourier transform]\index{Fourier!transform}
        Consider a locally compact Abelian group $G$ together with its canonical Haar measure $\mu$. For every $f\in L^1(G,\mu)$, one defines the Fourier transform as follows for all $\chi\in G^\vee$:
        \begin{gather}
            \widehat{f}(\chi) := \Int_G f(g)\overline{\chi(g)}\,d\mu(g)\,,
        \end{gather}
        where the identification $S^1\cong\mathrm{U}(1)$ is used.
    \end{construct}

    \begin{theorem}[Bochner]\index{Bochner}
        Consider a locally compact Abelian group $G$. There is a bijective correspondence between normalized, positive-definite, continuous functions on $G$ and probability measures on $G^\vee$, where
        \begin{gather}
            f(g) = \Int_{G^\vee}\chi\,d\nu\,.
        \end{gather}
    \end{theorem}

    \todo{COMPLETE}
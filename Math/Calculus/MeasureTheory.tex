\chapter{Measure \& Integration Theory}\label{chapter:measure}

    The main references for this chapter are~\citet{capinski_measure_2013,choquet-bruhat_analysis_1991}.

    \minitoc

\section{Measure theory}
\subsection{General definitions}

    \newdef{Measure}{\index{measure}\index{outer!measure}\index{$\sigma$!additivity}\label{measure:measure}
        Recall \cref{set:sigma_algebra} and consider a measurable space $(X,\Sigma)$. A function $\mu:\Sigma\rightarrow\overline{\mathbb{R}}$ is called a measure if it satisfies the following conditions:
        \begin{enumerate}
            \item\textbf{Nonnegativity}: $\forall E\in\Sigma:\mu(E)\geq0$,
            \item\textbf{Empty set is null}: $\mu(\emptyset)=0$, and
            \item\textbf{$\sigma$-additivity}: $\forall i\neq j:E_i\cap E_j=\emptyset\implies\mu\left(\bigcup_{n=1}^\infty E_n\right) = \sum_{i=n}^{+\infty}\mu(E_n)$.
        \end{enumerate}
        When $\mu$ is defined on all of $P(X)$ and only satisfies countable subadditivity, i.e.~the equality in the last condition becomes an inequality $\leq$, it is called an \textbf{outer measure}.
    }
    \begin{remark}
        To show that two measures coincide on a $\sigma$-algebra, it suffices to show that they coincide on the generating sets and apply the Monotone Class Theorem~\ref{set:monotone_class}.
    \end{remark}

    \newdef{Measure space}{\label{measure:measure_space}
        A triple $(X,\Sigma,\mu)$, where the pair $(X,\Sigma)$ is a measurable space and $\mu$ is a measure thereon, is called a measure space.
    }

    \newdef{Null set}{\index{null!set}
        A set $A\subset\mathbb{R}$ is said to be null if $\mu(A)=0$.
    }

    \newdef{Almost everywhere\footnotemark}{\index{almost everywhere}\label{measure:almost_everywhere}
        \footnotetext{In probability theory this is often called \textbf{almost surely}.}
        Let $(X,\Sigma,\mu)$ be a measure space. A property $P$ is said to hold on $X$ almost everywhere (abbreviated as \textbf{a.e.}) if it satisfies the following equation:
        \begin{gather}
            \mu\bigl(\{x\in X\bigm\vert\neg P(x)\}\bigr) = 0\,,
        \end{gather}
        i.e.~it holds everywhere except for a null set.
    }

    \newdef{Complete measure space}{\index{complete!measure space}
        A measure space $(X,\Sigma,\mu)$ is said to be complete if for every $E\in\Sigma$ with $\mu(E)=0$ the implication $A\subset E\implies A\in\Sigma$ holds. Additivity then necessarily implies that $\mu(A)=0$.
    }
    \newdef{Completion}{
        Let $\Sigma\subseteq\overline{\Sigma}$ be $\sigma$-algebras on a set $X$. $(X,\overline{\Sigma},\overline{\mu})$ is called the completion of $(X,\Sigma,\mu)$ if:
        \begin{enumerate}
            \item $\forall A\in\Sigma:\overline{\mu}(A)=\mu(A)$,
            \item $(X,\overline{\Sigma},\overline{\mu})$ is complete, and
            \item $\overline{\Sigma}$ is the smallest $\sigma$-algebra with these properties.
        \end{enumerate}
    }

    \newdef{\texorpdfstring{$\sigma$-}{sigma-}finite measure}{\index{$\sigma$!finite}\label{measure:sigma_finite_measure}
        Let $(X,\Sigma,\mu)$ be a measure space. The measure $\mu$ is said to be $\sigma$-finite if there exists a sequence $\seq{A}$ of measurable sets such that $\bigcup_{n=1}^{+\infty}A_n=X$ with $\forall n\in\mathbb{N}:\mu(A_n)<+\infty$.
    }

    \newdef{Borel measure}{\index{Borel!measure}
        Consider a topological space together with its Borel $\sigma$-algebra (\cref{topology:borel_set}). Any measure defined on this measurable space is called a Borel measure.
    }

    \newdef{Locally finite measure}{
        A measure on a Hausdorff space whose measurable sets contain the Borel sets such that every point has an open neighbourhood of finite measure. On locally compact spaces, this is equivalent to requiring that every compact subset has finite measure.
    }

    Given a Borel measure on a topological space, there are different ways of how one can approximate the measure of Borel sets by those of open or closed sets. Sadly, different distinct definitions can be found in the literature all under the name of `regular' measure. The next definition gives the most widely used ones.
    \newdef{Regular measure}{\label{measure:regular_measure}
        An outer measure $\mu$ on a topological space $X$ whose completion contains the Borel sets is said to be \textbf{Borel regular} if for every set $A\subset X$, there exists a Borel set $B\subseteq A$ such that $\mu(A)=\mu(B)$.

        Let $\mu$ be a measure on $(X,\Sigma)$, where $X$ is a topological space. It is said to be \textbf{regular} if if it satisfies the following conditions for every measurable set $B$:
        \begin{enumerate}
            \item\textbf{Outer regularity}: $\mu(B)=\inf\bigl\{\mu(O)\bigm\vert O\text{ open and measurable},O\supset B\bigr\}$, and
            \item\textbf{Inner regularity}: $\mu(B)=\sup\bigl\{\mu(K)\bigm\vert K\text{ closed and measurable},K\subset B\bigr\}$.
        \end{enumerate}
        If $X$ is Hausdorff, the measure is said to be \textbf{regular} if it satisfies the following conditions for every measurable set $B$:
        \begin{enumerate}
            \item\textbf{Outer regularity}: $\mu(B)=\inf\bigl\{\mu(O)\bigm\vert O\text{ open and measurable},O\supset B\bigr\}$, and
            \item\textbf{Inner regularity} or \textbf{tightness}: $\mu(B)=\sup\bigl\{\mu(K)\bigm\vert K\text{ compact and measurable},K\subset B\bigr\}$.
        \end{enumerate}
        Note that since compact subsets of Hausdorff spaces are closed, tight measures are in particular regular.
    }

    By slightly modifying the definition of tight Borel measures, one can obtain yet another type of regularity.
    \newdef{Radon measure}{\index{Radon!measure}\index{locally!finite}\label{measure:radon_measure}
        A Borel measure on a Hausdorff space that is outer regular on Borel sets, inner regular on open sets and locally finite. (For locally compact spaces, outer regularity is superfluous.)
    }

    \newdef{Radon space}{\index{Radon!space}
        A topological space on which every finite Borel measure is Radon. It is sometimes called \textbf{strongly Radon} if every locally finite Borel measure is Radon.
    }
    \begin{example}
        Locally compact, separable metric space, Polish spaces and even Suslin spaces (\cref{metric:suslin}) are Radon. Slightly weaker, every finite Borel measure on a metric space is regular.
    \end{example}

\subsection{Lebesgue measure}

    \newdef{Lebesgue outer measure}{\index{Lebesgue!outer measure}\label{measure:outer_measure}
        Let $X\subseteq\mathbb{R}$ be a set. The (Lebesgue) outer measure of $X$ is defined as follows:
        \begin{gather}
            \lambda^*(X) := \inf\left\{\sum_{n=1}^{+\infty}l(I_n)\,\middle\vert\,\seq{I}\text{ a sequence of open intervals that covers }X\right\}\,.
        \end{gather}
    }

    \begin{property}[Intervals]
        The outer measure of an interval $I$ equals its length: $\lambda^*(I)=l(I)$.
    \end{property}
    \begin{property}[Translation-invariance]\label{measure:translation_invariant}
        The outer measure is translation-invariant:
        \begin{gather}
            \lambda^*(A+t) = \lambda^*(A)
        \end{gather}
        for all $A\subset\mathbb{R}$ and $t\in\mathbb{R}$.
    \end{property}

    \begin{theorem}[Carath\'eodory's criterion]\index{Carath\'eodory!criterion}\index{Lebesgue!measure}\index{measurable!set}\label{measure:lebesgue_measure}
        Let $X$ be a subset of $\mathbb{R}$. If $X$ satisfies the following equation, it is said to be \textbf{Lebesgue measurable}:
        \begin{gather}
            \forall A\subseteq\mathbb{R}:\lambda^*(A) = \lambda^*(A\cap X) + \lambda^*(A\cap X^c)\,.
        \end{gather}
        The collection of all Lebesgue-measurable sets is denoted by $\mathcal{M}$ and the outer measure $\lambda^*(X)$, now denoted by $\lambda$, is called the \textbf{Lebesgue measure} of $X$.
    \end{theorem}
    \begin{result}\label{measure:completion_remark}
        The Lebesgue $\sigma$-algebra $\mathcal{M}$ is the completion of the Borel $\sigma$-algebra $\mathcal{B}$. (This is how the Lebesgue $\sigma$-algebra was introduced historically.)
    \end{result}

    \begin{property}\label{measure:countable_set_is_null}
        Any countable set is null with respect to the Lebesgue outer measure.
    \end{property}

    \begin{property}[Regularity]
        The Lebesgue measure is a regular Borel measure. For every $A\subseteq\mathbb{R}$, there exists a sequence $\seq{O}$ of open sets such that
        \begin{gather}
            \label{measure:open_cover_existence}
            A\subset\bigcap_{n=1}^{+\infty} O_n\qquad\text{and}\qquad\lambda\left(\bigcap_{n=1}^{+\infty} O_n\right) = \lambda^*(A)\,,
        \end{gather}
        and for every $E\in\mathcal{M}$, there exists a sequence $\seq{F}$ of closed sets such that
        \begin{gather}
            \label{measure:closed_cover_existence}
            \bigcup_{n=1}^{+\infty} F_n\subset E\qquad\text{and}\qquad\lambda\left(\bigcup_{n=1}^{+\infty} F_n\right) = \lambda(E)\,.
        \end{gather}
    \end{property}

    \begin{property}
        Consider a set $A\subset\mathbb{R}$. $A\in\mathcal{M}$ if and only if for every $\varepsilon>0$ there exist an open set $O\supset A$ and a closed set $F\subset A$ such that $\lambda^*(O\backslash A) < \varepsilon$ and $\lambda^*(A\backslash F)<\varepsilon$.
    \end{property}

    \begin{property}
        Let $\seq{A}$ be a sequence of sets in $\mathcal{M}$. The following two properties apply:
        \begin{align}
            \forall i\in\mathbb{N}:A_i\subseteq A_{i+1} &\implies \lambda\left(\bigcup_{n=1}^{+\infty} A_n\right) = \lim_{n\rightarrow\infty}\lambda(A_n)\,,\\
            \forall i\in\mathbb{N}:A_i\supseteq A_{i+1}\land\lambda(A_1)<+\infty &\implies\lambda\left(\bigcap_{i=n}^{+\infty} A_n\right) = \lim_{n\rightarrow\infty}\lambda(A_n)\,.
        \end{align}
    \end{property}
    \remark{This property is valid for every $\sigma$-additive set function.}

    \begin{construct}[Restriction]\index{Lebesgue!restricted measure}\label{measure:restricted_lebesgue_measure}
        Let $A\in\mathcal{M}$ have nonzero measure. The restriction of the Lebesgue measure to the set $B$ is defined as follows:
        \begin{gather}
            \mathcal{M}_A := \{A\cap B\mid B\in\mathcal{M}\} \qquad\text{and}\qquad \forall E\in\mathcal{M}_A:\lambda_A(E) := \lambda(E)\,.
        \end{gather}
        It can be shown that the measure space $(A,\mathcal{M}_A,\lambda_A)$ is complete.
    \end{construct}

    The construction of Lebesgue measurable sets from the Lebesgue outer measure can be generalized to arbitrary sets and outer measures.
    \begin{construct}[Carath\'eodory's extension theorem\footnotemark]\index{pre-!measure}\index{Hahn--Kolmogorov}\label{measure:caratheodory}
        \footnotetext{Also called the \textbf{Hahn--Kolmogorov theorem}.}
        Every outer measure $\mu^*$ gives rise to a $\sigma$-algebra consisting of those sets that satisfy Carath\'eodory's criterion (\cref{measure:lebesgue_measure}) with respect to $\mu^*$. Furthermore, consider a \textbf{premeasure} $\mu_0$, i.e.~a $\sigma$-additive function defined on an algebra of sets (\cref{set:algebra_of_sets}) such that $\mu_0(\emptyset) = 0$. \Cref{measure:outer_measure} can be used to define an outer measure $\mu^*$ in terms of the premeasure $\mu_0$ by replacing intervals with elements from the given algebra of sets. The $\sigma$-algebra generated by this outer measure contains the given algebra of sets and $\mu^*$ restricts to $\mu_0$. This shows that any premeasure can be extended to a genuine measure, uniquely if $\mu_0$ is $\sigma$-finite. Moreover, it can be shown that this measure is complete.
    \end{construct}

\subsection{Measurable functions}

    \newdef{Measurable function}{\index{measurable!function}
        Consider two measurable spaces $(X,\Sigma_X)$ and $(Y,\Sigma_Y)$. A function $f:X\rightarrow Y$ is said to be measurable if for every measurable set $A\in\Sigma_Y$ the preimage $f^{-1}(A)$ is also measurable. Equivalently, the $\sigma$-algebra generated by the preimages of measurable sets in $\Sigma_Y$ should be a sub-$\sigma$-algebra of $\Sigma_X$.
    }
    \begin{property}
        Measurability is closed under composition. This turns the collection of measurable spaces and measurable functions into a category $\mathbf{Meas}$. (This notation is also used for the subcategory on \textit{measure spaces} (\cref{measure:measure_preserving}).)
    \end{property}

    \newdef{Measurable}{\index{measurable}
        Sometimes an equivalence class of real- or complex-valued measurable functions up to functions that are zero a.e.~is called a measurable.
    }

    Two important examples are given below:
    \begin{example}[Borel-measurable function]\label{measure:borel_measurable_function}
        A continuous function $f:X\rightarrow Y$ such that for every open set $O\in\mathcal{B}_Y:f^{-1}(O)\in\mathcal{B}_X$.
    \end{example}
    \begin{example}[Lebesgue-measurable function]\label{measure:measurable_function}
        A function $f:\mathbb{R}\rightarrow\mathbb{R}$ such that for every interval $I\subset\mathbb{R}:f^{-1}(I)\in\mathcal{M}$.
    \end{example}
    \remark{The inclusion $\mathcal{B}\subset\mathcal{M}$ implies that every Borel-measurable function is also Lebesgue-measurable.}

    \begin{property}
        The class of Borel/Lebesgue-measurable functions defined on $E\in\mathcal{M}$ forms an algebra.
    \end{property}

    \begin{example}
        The following types of functions are Lebesgue-measurable:
        \begin{itemize}
            \item monotonic functions,
            \item continuous functions, and
            \item indicator functions.
        \end{itemize}
    \end{example}
    \begin{result}
        Let $f,g$ be Lebesgue-measurable functions and let $F:\mathbb{R}\times\mathbb{R}\rightarrow\mathbb{R}$ be a continuous function. The composition $F\bigl(f(x),g(x)\bigr)$ is also measurable.
    \end{result}

    \begin{property}
        Let $f$ be a Lebesgue-measurable function. The level set $\{x\mid f(x)=a\}$ is measurable for all $a\in\mathbb{R}$.
    \end{property}

    \begin{property}
        Define the following functions (which are measurable if $f$ is measurable as a result of the previous properties):
        \begin{align}
            \label{measure:positive_part}
            f^+(x) &:= \max(f,0) =
            \begin{cases}
                f(x)&\cif f(x)>0\\
                0&\cif f(x)\leq0,
            \end{cases}\\\nonumber\\
            \label{measure:negative_part}
            f^-(x) &:= \max(-f,0) =
            \begin{cases}
                0&\cif f(x)>0\\
                -f(x)&\cif f(x)\leq0\,.
            \end{cases}
        \end{align}
        The function $f:\mathbb{R}\rightarrow\mathbb{R}$ is measurable if and only if both $f^+$ and $f^-$ are measurable. Furthermore, $f$ is measurable if $|f|$ is measurable (the converse is false in general).
    \end{property}

    \newdef{Pushforward}{\index{pushforward!of a measure}
        Consider two measurable spaces $(X_1,\Sigma_1)$ and $(X_2,\Sigma_2)$ together with a measurable function $f:X_1\rightarrow X_2$. For every measure $\mu$ on $X_1$ one can define the pushforward measure $f_*\mu$ on $X_2$ as follows:
        \begin{gather}
            f_*\mu(A) := \mu\bigl(f^{-1}(A)\bigr)\,.
        \end{gather}
    }

    \newdef{Measure-preserving function}{\label{measure:measure_preserving}
        Let $(X,\Sigma,\mu)$ be a measure space and consider a measurable function $T:X\rightarrow X$. $T$ is said to be measure-preserving if
        \begin{gather}
            T_\ast\mu = \mu\,.
        \end{gather}
        These functions form the morphisms in the category $\mathbf{Meas}$ of measure spaces.
    }

    \newdef{Ergodic function}{\index{ergodic}
        Let $(X,\Sigma,\mu)$ be a measure space and consider a measure-preserving function $T:X\rightarrow X$. It is said to be ergodic if the following condition is satisfied:
        \begin{gather}
            T(A) = A\implies\mu(A) = 0\lor\mu(X\backslash A) = 0\,.
        \end{gather}
        This is equivalent to stating that for every set $A\in\Sigma$ with positive measure the following condition holds:
        \begin{gather}
            \mu\left(\bigcup_{n=1}^{+\infty} T^{-n}(A)\right) = 1\,.
        \end{gather}
    }

    \begin{property}
        Consider a topological space $X$ with Borel $\sigma$-algebra $\mathcal{B}$ and let $T$ be an ergodic function. Almost every $T$-orbit is dense in the support of $\mu$.
    \end{property}

    \newdef{Mixing}{\index{mixing}
        An endomorphism of a measure spaces $(X,\Sigma,\mu)$ is said to be mixing if for all measurable spaces $A,B$ the following equality holds:
        \begin{gather}
            \lim_{n\rightarrow\infty}\mu\left(T^{-n}(A)\cap B\right) = \mu(A)\mu(B)\,
        \end{gather}
    }
    \begin{property}
        All mixing transformations are ergodic.
    \end{property}

    \begin{property}[Additivity]\index{additive!function}
        Every measurable, additive function $f:\mathbb{R}\rightarrow\mathbb{R}$ is linear.
    \end{property}
    \begin{result}
        From the basic properties of exponential and logarithmic functions, the following results can be obtained:
        \begin{itemize}
            \item Let $f:\mathbb{R}\rightarrow\mathbb{R}$ be a measurable function. If $f(x+y) = f(x)f(y)$, then $f(x)=e^{\lambda x}$ for some $\lambda\in\mathbb{R}$.
            \item Let $f:[0,+\infty]\rightarrow\mathbb{R}$ be a measurable function. If $f(xy) = f(x)+f(y)$, then $f(x)=\lambda\log(x)$ for some $\lambda\in\mathbb{R}$.
            \item Let $f:[0,+\infty]\rightarrow[0,+\infty]$ be a measurable function. If $f(xy) = f(x)f(y)$, then $f(x)=x^\lambda$ for some $\lambda\in\mathbb{R}$.
        \end{itemize}
    \end{result}

\subsection{Limit operations}

    \begin{property}
        Let $\seq{f}$ be a sequence of measurable functions. The following functions are also measurable:
        \begin{itemize}
            \item $\ds\min_{i\leq k}(f_i)$ and $\ds\max_{i\leq k}(f_i)$,
            \item $\ds\inf_{n\in\mathbb{N}}(f_n)$ and $\ds\sup_{n\in\mathbb{N}}(f_n)$, and
            \item $\ds\liminf_{n\rightarrow\infty}(f_n)$ and $\ds\limsup_{n\rightarrow\infty}(f_n)$.
        \end{itemize}
    \end{property}

    \begin{property}
        If $f$ is a measurable function and $g$ is a function such that $f=g$ almost everywhere, then $g$ is measurable as well.
    \end{property}
    \result{As a result of the previous two properties, if a sequence of measurable functions converges pointwise a.e., the limit is also a measurable function.}

    \newdef{Essential supremum}{\index{essential!supremum}\label{measure:essential_supremum}
        \begin{gather}
            \esssup(f) := \inf\{z\in\mathbb{R}\mid f\leq z\text{ a.e.}\}
        \end{gather}
    }
    \newdef{Essential infimum}{\index{essential!infimum}\label{measure:essential_infimum}
        \begin{gather}
            \essinf(f) := \sup\{z\in\mathbb{R}\mid f\geq z\text{ a.e.}\}
        \end{gather}
    }
    \begin{property}
        Every measurable function $f$ satisfies the following inequalities:
        \begin{itemize}
            \item $f\leq\esssup(f)\text{ a.e.}$ and $f\geq\essinf(f)\text{ a.e.}$, and
            \item $\esssup(f)\leq\sup(f)$ and $\essinf(f)\geq\inf(f)$.
        \end{itemize}
        The latter pair of inequalities becomes a pair of equalities if $f$ is continuous.
    \end{property}
    \begin{property}
        If $f,g$ are measurable functions, then $\esssup(f+g)\leq\esssup(f)+\esssup(g)$. An analogous inequality holds for the essential infimum.
    \end{property}

    \newdef{Weak convergence}{\index{convergence!weak}\index{continuity!set}\label{measure:weak_convergence}
        A sequence of measures $\seq{\mu}$ is said to converge weakly to a measure $\mu$ on a metrizable space $X$ if any of the following conditions is satisfied:
        \begin{enumerate}
            \item $\int_Xf\,d\mu_n\longrightarrow\int_Xf\,d\mu$ for all bounded, continuous functions $f$.
            \item $\mu_n(A)\longrightarrow\mu(A)$ for all \textbf{continuity sets} $A$ of $\mu$, i.e.\ for all Borel sets $A$ such that $\mu(\partial A)=0$.
            \item $\lim\inf\mu_n(U)\geq\mu(U)$ for all open sets $U$.
            \item $\lim\sup\mu_n(V)\leq\mu(V)$ for all closed sets $V$.
        \end{enumerate}
        If $X=\mathbb{R}$ with its standard topology, the sequence $\seq{\mu}$ converges weakly to $\mu$ if and only if $\mu_n(\{x\in\mathbb{R}:x\leq y\})\longrightarrow\mu(\{x\in\mathbb{R}:x\leq y\})$ for all points $y\in\mathbb{R}$ where these functions are continuous.
    }

    \begin{remark}[\difficult{Relation to functional weak convergence}]\index{Stone--$\check{C}$ech compactification}
        Consider \cref{functional:weak_topology} of weak convergence in functional analysis. By the Riesz--Markov theorem~\ref{distributions:riesz_markov} one can identify measures with functionals on functions vanishing at infinity. By noting that bounded continuous functions on $X$ are equivalent to functions of compact support on the \textit{Stone--$\check{C}$ech compactification} $\beta X$, one obtains that a sequence of measures converges weakly if and only if the associated sequence of functionals converges weakly.

        \todo{CHECK THIS STATEMENT}
    \end{remark}

\subsection{Polish spaces}

    The content of this section heavily relies on that of \cref{chapter:topology} and \cref{chapter:metric}.

    \newdef{Polish space}{\index{Polish space}\label{measure:polish_space}
        A separable, completely metrizable space.
    }
    \newdef{Standard Borel space}{\index{Borel!space}
        A Borel space associated to a Polish space.
    }

    \begin{theorem}[Kuratowski]\index{Kuratowski}
        Every standard Borel space is (Borel) isomorphic to either $\mathbb{R}$, $\mathbb{Z}$ or a finite set. Hence, it is completely characterized by its cardinality.
    \end{theorem}

    \begin{property}
        All open and closed subsets of a Polish space are Polish. So are $G_\delta$-subsets (\cref{top:g_delta}).
    \end{property}

    \newdef{Lusin space}{\index{Lusin space}
        A topological space that is homeomorphic to a Borel subset of a compact metric space. Equivalently, a topological space that admits a stronger topology that is Polish or, in other words, it is the image of a Polish space under a continuous bijection. In particular, every Polish space is Lusin.
    }
    \newdef{Suslin space\footnotemark}{\index{Suslin space}\label{metric:suslin}
        \footnotetext{Sometimes written as `Souslin'.}
        The image of a Polish space under a continuous function. In particular, every Lusin space is Suslin.
    }

    \begin{theorem}[Lusin--Suslin]
        A subset of a Polish space is Lusin if and only if it is Borel.
    \end{theorem}

\section{Lebesgue integral}
\subsection{Simple functions}

    \newdef{Indicator function}{\index{indicator function}\label{measure:indicator_function}
        \begin{gather}
    	    \mathbbm{1}_A(x) :=
            \begin{cases}
	            1&\cif x\in A\,,\\
                0&\cif x\not\in A\,.
	        \end{cases}
	    \end{gather}
    }

    \newdef{Simple function}{\index{simple!function}\label{measure:simple_function}
        A function $f:X\rightarrow\mathbb{R}$ on a measurable space $(X,\Sigma)$ that can be expressed as
        \begin{gather}
            f(x) = \sum_{i=1}^na_i\mathbbm{1}_{A_i}(x)
        \end{gather}
        for some $\{a_i\geq0\}_{i\leq n},\{A_i\}_{i\leq n}\subset\Sigma$ and $n\in\mathbb{N}$.
    }

    \newdef{Step function}{\index{step function}\label{measure:step_function}
        If $(X,\Sigma)=(\mathbb{R},\mathcal{M})$ and the sets $A_i$ are intervals, the above function is often called a step function.
    }

    \newdef{Lebesgue integral of simple functions}{\index{Lebesgue!integral}\label{measure:integral_simple_function}
        Consider a simple function $\varphi$ on a measure space $(X,\Sigma,\mu)$. The Lebesgue integral of $\varphi$ over a measurable set $A\in\Sigma$ with respect to $\mu$ is given by
        \begin{gather}
            \Int_A\varphi\,d\mu := \sum_{i=1}^na_i\mu(A\cap A_i)\,.
        \end{gather}
        As usual, if the domain of integration is not mentioned explicitly, an integral over the whole space $X$ is implied.
    }

    \begin{example}
        Let $\mathbbm{1}_{\mathbb{Q}}$ be the indicator function of the rational numbers. Contrary to the case of Riemann integrals, the above definition makes it possible to integrate the rational indicator function over the real line:
        \begin{gather}
            \Int_{\mathbb{R}}\mathbbm{1}_{\mathbb{Q}}\,d\lambda = 1\times\lambda(\mathbb{Q}) + 0\times\lambda(\mathbb{R}\backslash\mathbb{Q}) = 0\,,
        \end{gather}
        where the measure of the rational numbers is 0 because it is a countable set (\cref{measure:countable_set_is_null}).
    \end{example}

\subsection{Measurable functions}

    \newdef{Integral for nonnegative functions}{\index{Lebesgue!integral}\label{measure:integral}
        The definition for simple functions can be generalized to nonnegative measurable functions $f$ as follows:
        \begin{gather}
            \Int_Af\,d\mu := \sup\left\{\Int_A\varphi\,d\mu\,\middle\vert\,\varphi\text{ a simple function such that }\varphi\leq f\right\}\,.
        \end{gather}
        This integral is always nonnegative.
    }

    \begin{formula}\label{measure:domain_change}
        The following equality allows to change the domain of integrals:
        \begin{gather}
            \Int_Af\,d\mu = \int_Xf\mathbbm{1}_A\,d\mu\,.
        \end{gather}
    \end{formula}

    \begin{property}
        The Lebesgue integral over a null set is 0.
    \end{property}

    \begin{theorem}[Mean value theorem]\index{mean!value theorem}
        If $a\leq f(x)\leq b$, then $a\lambda(A)\leq\int_Af\,d\lambda\leq b\lambda(A)$.
    \end{theorem}

    \begin{property}[Simple approximation]
        Let $f$ be a nonnegative measurable function. There exists an increasing sequence $\seq{\varphi}$ of simple functions such that $\varphi_n\nearrow f$. Moreover, if $f$ is bounded on $A\in\Sigma$, the sequence can be chosen to be uniformly convergent on $A$.
    \end{property}

\subsection{Integrable functions}

    \newdef{Integrable function}{\index{integrable}\label{measure:integrable_function}
        Let $A$ be a measurable subset of a measure space $(X,\Sigma,\mu)$. A measurable function $f$ is said to be integrable over $A$ if both $\int_Af^+\,d\mu$ and $\int_Af^-\,d\mu$ are finite. The Lebesgue integral of $f$ over $A$ is then defined as
        \begin{gather}
            \Int_Af\,d\mu := \Int_Af^+\,d\mu - \Int_Af^-\,d\mu\,.
        \end{gather}
        If only one of the functions $f^+,f^-$ is finite, $f$ is said to be \textbf{quasi-integrable}.
    }

    \begin{property}[Absolute integrability]\label{measure:absolute_integrability}
        $f$ is integrable if and only if $|f|$ is integrable. Furthermore,
        \begin{gather}
            \Int_A|f|\,d\mu = \Int_Af^+\,d\mu + \Int_Af^-\,d\mu\,.
        \end{gather}
    \end{property}
    \begin{property}
        Let $f,g$ be integrable functions on a measure space $(X,\Sigma,\mu)$. The following important properties hold:
        \begin{itemize}
            \item\textbf{Linearity}: $\int_A(f+\lambda g)d\mu = \int_Af\,d\mu+\lambda\int_Ag\,d\mu$ for all $\lambda\in\mathbb{R}$
            \item\textbf{Monotonicity}: $f\leq g$ a.e. implies $\int_Af\,d\mu\leq\int_Ag\,d\mu$ and $\forall A\in\Sigma:\int_Af\,d\mu\leq\int_Ag\,d\mu\implies f\leq g$ a.e.
            \item\textbf{Finiteness}: $f$ is finite a.e.
            \item $|\int_Af\,d\mu|\leq\int_A|f|\,d\mu$.
            \item $\int_Af\,d\mu=0,\forall A\in\Sigma\implies f=0$ a.e.
        \end{itemize}
    \end{property}

    \newdef{Integrable functions}{
        The set of integrable functions over a set $A\in\mathcal{M}$ forms the vector space $\mathcal{L}^1(A)$.
    }

    \begin{property}[Continuous approximation]
        Let $f\in\mathcal{L}^1$ and $\varepsilon>0$. There exists a continuous (or step or even simple) function $g$, vanishing outside a finite (or even compact) set, such that $\int|f-g|\,d\mu<\varepsilon$.
    \end{property}

    \newdef{Locally integrable function}{\index{locally!integrable}\label{measure:locally_integrable}
        A measurable function is said to be locally integrable if it is integrable on every compact subset of its domain. The space of locally integrable functions is denoted by $\mathcal{L}^1_{\text{loc}}$.
    }
    \begin{example}
        All continuous functions are locally integrable.
    \end{example}

    \begin{property}[Absolute continuity]\index{continuity!absolute}\label{measure:measure_by_integral}
        Let $f\geq0$ be a measurable function. The mapping $A\mapsto\int_Af\,d\mu$ defines a measure that is $\sigma$-finite if $f$ is locally integrable and finite if $f$ is integrable. Furthermore, this measure is said to be absolutely continuous (with respect to $\mu$). See \cref{section:Radon-Nikodym} for a generalization to arbitrary measures.
    \end{property}

\subsection{Convergence theorems}

    \begin{theorem}[Fatou's lemma]\index{Fatou}\label{measure:fatous_lemma}
        Let $\seq{f}$ be a sequence of nonnegative measurable functions.
        \begin{gather}
            \Int_A\left(\liminf_{n\rightarrow\infty}f_n\right)\,d\mu \leq \liminf_{n\rightarrow\infty}\Int_Af_n\,d\mu
        \end{gather}
    \end{theorem}
    \begin{theorem}[Monotone convergence]\index{convergence!monotone}\label{measure:monotone_convergence_theorem}
        Let $A$ be measurable and let $\seq{f}$ be an increasing sequence of nonnegative measurable functions such that $f_n\nearrow f$ pointwise a.e.
        \begin{gather}
            \Int_Af\,d\mu = \lim_{n\rightarrow\infty}\Int_Af_n\,d\mu\,.
        \end{gather}
    \end{theorem}

    \begin{method}\label{measure:linear_proofs}
        To prove results concerning integrable functions in spaces such as $\mathcal{L}^1$ it is often useful to proceed as follows:
        \begin{enumerate}
            \item Verify that the property holds for indicator functions. (This often follows by definition.)
            \item Use linearity to extend the property to simple functions.
            \item Apply the monotone convergence theorem to show that the property holds for all nonnegative measurable functions.
            \item Extend the property to all integrable functions by decomposing $f=f^+-f^-$ and applying linearity again.
        \end{enumerate}
    \end{method}

    \begin{theorem}[Dominated convergence]\index{convergence!dominated}\label{measure:dominated_convergence_theorem}
        Let $A$ be measurable set and consider a sequence of measurable functions $\seq{f}$ such that $\forall n\in\mathbb{N}:|f_n|\leq g$ a.e. for some function $g\in\mathcal{L}^1(A)$. If $f_n\longrightarrow f$ pointwise a.e., then $f$ is integrable over $A$ and
        \begin{gather}
            \Int_Af\,d\mu = \lim_{n\rightarrow\infty}\Int_Af_n\,d\mu\,.
        \end{gather}
    \end{theorem}

    \begin{property}
        Let $\seq{f}$ be a sequence of nonnegative measurable functions
        \begin{gather}
            \Int_A\sum_{n=1}^{+\infty}f_n\,d\mu = \sum_{n=1}^{+\infty}\Int_Af_n\,d\mu\,.
        \end{gather}
        One cannot conclude that the right-hand side is finite a.e., so the series on the left-hand side need not be integrable.
    \end{property}

    \begin{theorem}[Beppo Levi\footnotemark]\index{Beppo Levi}\label{measure:beppo_levi}
        \footnotetext{Various other theorems and variants of this theorem can be found in the literature under the same name.}
        Suppose that \[\sum_{i=1}^{+\infty}\int_A|f_n|\,d\mu\] is finite. The series $\sum_{i=1}^{+\infty}f_n(x)$ converges a.e.~Furthermore, the series is integrable and
        \begin{gather}
            \Int_A\sum_{i=1}^{+\infty}f_n\,d\mu = \sum_{i=1}^{+\infty}\Int_Af_n\,d\mu\,.
        \end{gather}
    \end{theorem}

    \begin{theorem}[Riemann--Lebesgue lemma]\index{Riemann--Lebesgue lemma}\label{measure:riemann_lebesue_lemma}
        Let $f\in\mathcal{L}^1(\mathbb{R})$. The sequences
        \begin{gather}
            s_k=\Int_{\mathbb{R}}f(x)\sin(kx)\,dx
        \end{gather}
        and
        \begin{gather}
            c_k=\Int_{\mathbb{R}}f(x)\cos(kx)\,dx
        \end{gather}
        both converge to 0.
    \end{theorem}

    \begin{theorem}[Birkhoff ergodicity]\index{ergodic!theorem}\index{Birkhoff|seealso{ergodic}}\label{measure:ergodic}
        Let $(X,\Sigma,\mu)$ be a measure space and let $T$ be a $\mu$-ergodic map. For every measurable function $f$ and for $\mu$-almost every element $x\in X$ the integral of $f$ can be computed as an average over the orbit of $x$:
        \begin{gather}
            \lim_{n\rightarrow\infty}\frac{1}{n+1}\sum_{t=0}^nf\bigl(T^n(x)\bigr) = \Int_Xf\,d\mu\,.
        \end{gather}
    \end{theorem}

\subsection{Relation to the Riemann integral}

    \begin{property}
        Let $f:[a,b]\rightarrow\mathbb{R}$ be a bounded function.
        \begin{itemize}
            \item $f$ is Riemann-integrable if and only if $f$ is continuous a.e.~with respect to the Lebesgue measure on $[a,b]$, i.e.~the set of discontinuities of $f$ has measure zero.
            \item Riemann-integrable functions on $[a,b]$ are integrable with respect to the Lebesgue measure on $[a,b]$ and the integrals coincide.
        \end{itemize}
    \end{property}

    \begin{property}
        If $f\geq0$ and the improper Riemann integral (\cref{calculus:improper_integral}) exists, the Lebesgue integral exists and the two integrals coincide. Note that positivity is necessary here. Because the Lebesgue integral is absolute (\cref{measure:absolute_integrability}), positive and negative parts cannot cancel (Lebesgue integrals can never be conditionally convergent).
    \end{property}

    The following definition should be compared to \cref{measure:indicator_function} and \cref{distribution:dirac_delta}.
    \newdef{Dirac measure}{\index{Dirac}\label{measure:dirac_measure}
        Define the Dirac (delta) measure as follows:
        \begin{gather}
            \delta_a(A) :=
            \begin{cases}
                1&\cif a\in A\,,\\
                0&\cif a\not\in A\,.
            \end{cases}
        \end{gather}
        Integration with respect to the Dirac measure has the following important property:
        \begin{gather}
            \Int_Xf\,d\delta_a = f(a)\,.
        \end{gather}
    }

\section{Space of integrable functions}
\subsection{Distance}\index{distance}

    To define a distance between functions, a notion of the length of a function is introduced first. Normally, this would not be a problem, one could use the integral of a function to define a norm. However, the fact that two functions differing on a null set have the same integral carries problems with it: a nonzero function could have a zero length. To avoid this issue, these degenerate functions are quotiented out.

    \newdef{$L^1$-space}{
        Define the set of equivalence classes $L^1=\mathcal{L}^1_{/\equiv}$ by introducing the following equivalence relation: $f\equiv g$ if and only if $f=g$ a.e.
    }
    \begin{property}
        $L^1$ is a \textit{Banach space} (see \cref{functional:banach_space}). The norm on $L^1$ is given by
        \begin{gather}
            \label{measure:L1_norm}
            \|f\|_1 := \Int_X|f|\,d\mu\,.
        \end{gather}
        In particular,
        \begin{gather}
            \Int_X|f|\,d\mu=0\implies f=0\text{ a.e.}
        \end{gather}
    \end{property}

\subsection{Hilbert space \texorpdfstring{$L^2$}{L2}}\label{section:hilbert_space}

    \begin{property}\label{measure:L2_hilbert_space}
        $L^2$ is a \textit{Hilbert space} (see \cref{functional:hilbert_space}). The norm on $L^2$ is given by
        \begin{gather}
            \label{measure:L2_norm}
            \|f\|_2 := \left(\Int_X|f|^2\,d\mu\right)^{1/2}\,.
        \end{gather}
        This norm is induced by the following inner product:
        \begin{gather}
            \label{measure:L2_inner_product}
            \braket{f}{g} := \Int_X\overline{f}g\,d\mu\,.
        \end{gather}
    \end{property}

    \begin{formula}[Cauchy--Schwarz inequality]\index{Cauchy--Schwarz inequality}\label{measure:schwarz_inequality}
        Let $f,g\in L^2(X,\mathbb{C})$. \Cref{measure:holders_inequality} implies that $fg\in L^1(X,\mathbb{C})$ and
        \begin{gather}
            \left|\Int\overline{f}g\,d\mu\,\right|\leq\|fg\|_1\leq\|f\|_2\|g\|_2\,.
        \end{gather}
    \end{formula}

\subsection{\texorpdfstring{$L^p$}{Lp}-spaces}

    Generalizing the previous two function classes leads to the notion of $L^p$-spaces with the following norm.
    \begin{formula}
        For all $1\leq p\leq\infty$, $L^p(X)$ is a \textit{Banach space} (see \cref{functional:banach_space}) when equipped with the following norm:
        \begin{gather}
            \label{measure:Lp_norm}
            \|f\|_p := \left(\Int_X|f|^p\,d\mu\right)^{1/p}\,.
        \end{gather}
    \end{formula}
    \remark{Note that $L^2$ is the only $L^p$-space that is also a \textit{Hilbert space} (see \ref{functional:hilbert_space}). The other $L^p$-spaces do not have a norm induced by an inner product.}

    \newformula{H\"{o}lder's inequality}{\index{H\"older!inequality}\index{H\"older!conjugates}\label{measure:holders_inequality}
        Let $\frac{1}{p}+\frac{1}{q} = 1$ with $p\geq1$ (numbers satisfying this equality are called \textbf{H\"older conjugates}). For every $f\in L^p$ and $g\in L^q$ one has that
        \begin{gather}
            \|fg\|_1\leq\|f\|_p\|g\|_q\,.
        \end{gather}
        This also implies that $fg\in L^1$.
    }
    \newformula{Minkowski's inequality}{\index{Minkowski!inequality}\label{measure:minkowskis_inequality}
        For every $p\geq1$ and $f,g\in L^p$ one has that
        \begin{gather}
            \|f+g\|_p\leq\|f\|_p + \|g\|_p\,.
        \end{gather}
        This also implies that $f+g\in L^p$.
    }

    \begin{property}[Inclusions]
        $L^1(X)\cap L^\infty(X)\subset L^2(X)$. Moreover, if $X$ has finite measure, then $L^q(X)\subset L^p(X)$ whenever $1\leq p\leq q<+\infty$.
    \end{property}

    Using the H\"older inequality one can prove the following property.
    \begin{property}\label{measure:Lp_duals}
        Let $p,q$ be H\"older conjugates. The spaces $L^p$ and $L^q$ are topological duals, i.e.~every function $f\in L^p$ can be identified (one-to-one) with a continuous functional on $L^q$.
    \end{property}

    \newdef{Essentially bounded function}{
        Let $f$ be a measurable function satisfying $\esssup|f|<+\infty$. The function $f$ is said to be essentially bounded and the set of all such functions is denoted by $L^\infty$ (again after quotienting out all functions that are equal a.e.).
    }

    \begin{formula}\index{supremum!norm}
        A norm on $L^\infty$ is given by
        \begin{gather}
            \|f\|_\infty := \esssup|f|\,.
        \end{gather}
        This norm is called the \textbf{supremum norm} and it induces the supremum metric (\cref{metric:supremum_distance}).
    \end{formula}
    \begin{property}
        Equipped with the above norm the space $L^\infty$ becomes a Banach space.
    \end{property}

\section{Product measures}
\subsection{Construction}

    The general condition for product measures is given by the following equation that should hold for all $A_1\in\Sigma_1$ and $A_2\in\Sigma_2$:
    \begin{gather}
        \label{measure:general_condition}
        \mu(A_1\times A_2) = \mu_1(A_1)\mu_2(A_2)\,.
    \end{gather}

    \newdef{Section}{\index{section!of a product set}
        Let $A=A_1\times A_2$. The following two sets are called sections:
        \begin{align*}
            A_{x_1} &:= \{x_2\in X_2\mid(x_1,x_2)\in A\}\subset\Sigma_2\,,\\
            A_{x_2} &:= \{x_1\in X_1\mid(x_1,x_2)\in A\}\subset\Sigma_1\,.
        \end{align*}
    }
    The following property follows immediately from the definition of product $\sigma$-algebras (\cref{set:product_of_sigma_algebras}).
    \begin{property}
        Let $\Sigma := \Sigma_1\times\Sigma_2$ be the product $\sigma$-algebra. If $A\in\Sigma$, then $A_{x_1}\in\Sigma_2$ for each $x_1\in X_1$ and $A_{x_2}\in\Sigma_1$ for each $x_2\in X_2$. Equivalently, the sets $\mathcal{G}_1 = \{A\in\Sigma\mid\forall x_1\in X_1:A_{x_1}\in\Sigma_2\}$ and $\mathcal{G}_2 = \{A\in\Sigma\mid\forall x_2\in X_2: A_{x_2}\in\Sigma_1\}$ coincide with $\Sigma$.
    \end{property}

    \begin{property}
        The function $A_{x_2}\mapsto\mu(A_{x_2})$ is a step function:
        \begin{gather}
            \mu(A_{x_2}) =
            \begin{cases}
                \mu_1(A_1)&\cif x_2\in A_2\,,\\
                0&\cif x_2\not\in A_2\,.
            \end{cases}
        \end{gather}
    \end{property}

    \begin{formula}[Product measure]\index{measure}
        From the previous property it follows that the product measure $\mu(A)$ can be written in the following way:
        \begin{gather}
            \mu(A) = \Int_{X_2}\mu_1(A_{x_2})\,d\mu_2(x_2)\,.
        \end{gather}
    \end{formula}
    \begin{property}
        Let $\mu_1,\mu_2$ be $\sigma$-finite measures. If $A\in\Sigma$, the functions
        \[x_1\mapsto\mu_2(A_{x_1}) \qquad\text{and}\qquad x_2\mapsto\mu_1(A_{x_2})\]
        are measurable with respect to $\Sigma_1$ and $\Sigma_2$, respectively. Moreover,
        \begin{gather}
            \Int_{X_2}\mu_1(A_{x_2})\,d\mu_2(x_2) = \Int_{X_1}\mu_2(A_{x_1})\,d\mu_1(x_1)\,.
        \end{gather}
        Furthermore, the set function $\mu$ is countably additive and if any other product measure coincides with $\mu$ on all product sets, it coincides with $\mu$ on the whole product $\sigma$-algebra.
    \end{property}

\subsection{Fubini's theorem}\label{section:fubini}

    \begin{property}
        Let $f:X_1\times X_2\rightarrow\mathbb{R}$ be a nonnegative function. If $f$ is measurable with respect to $\Sigma_1\times\Sigma_2$, then for each $x_1\in X$ the function $x_2\mapsto f(x_1,x_2)$ is measurable with respect to $\Sigma_2$ (and vice versa). Their integrals with respect to $\mu_1$ and $\mu_2$ respectively are also measurable.
    \end{property}
    \newdef{Section}{\index{section}
        The functions $x_1\mapsto f(x_1,x_2)$ and $x_2\mapsto f(x_1,x_2)$ are called sections of $f$.
    }

    \begin{theorem}[Tonelli]\index{Tonelli}\label{measure:tonelli_theorem}
        Let $f:X_1\times X_2\rightarrow\mathbb{R}$ be a nonnegative function. The following equalities hold:
        \begin{align}
            \Int_{X_1\times X_2}f\,d\mu &= \Int_{X_1}\left(\Int_{X_2}f(x_1,x_2)d\mu_2(x_2)\right)d\mu_1(x_1)\nonumber\\
            &= \Int_{X_2}\left(\Int_{X_1}f(x_1,x_2)d\mu_1(x_1)\right)d\mu_2(x_2)\,.
        \end{align}
    \end{theorem}

    \begin{result}[Fubini]\index{Fubini}
        Let $f\in L^1(X_1\times X_2)$. The sections of $f$ are integrable in the appropriate spaces. Furthermore, the functions
        \begin{gather}
            x_1\mapsto\Int_{X_2}f(x_1,x_2)\,d\mu_2(x_2)
        \end{gather}
        and
        \begin{gather}
            x_2\mapsto\Int_{X_1}f(x_1,x_2)\,d\mu_1(x_1)
        \end{gather}
        are in $L^1(X_1)$ and $L^1(X_2)$, respectively, and Tonelli's theorem holds.
    \end{result}
    \remark{The previous construction and theorems also apply to higher-dimensional product spaces. These theorems provide a way to construct higher-dimensional measures by defining them (as the completion of) the product of measures.}

\section{Radon--Nikodym theorem}\index{Radon--Nikodym}\label{section:Radon-Nikodym}

    \newdef{Absolute continuity}{\index{continuity!absolute}\index{equivalent!measures}\label{measure:absolute_continuity}
        Let $(X,\Sigma)$ be a measurable space and let $\mu,\nu$ be two measures defined on this space. Then $\nu$ is said to be absolutely continuous with respect to $\mu$ if
        \begin{gather}
            \forall A\in\Sigma:\mu(A) = 0\implies\nu(A) = 0\,.
        \end{gather}
        This relation is often denoted by $\nu\ll\mu$. If both $\mu\ll\nu$ and $\nu\ll\mu$, the measures are said to be equivalent. This is denoted by $\mu\sim\nu$.
    }

    The following property relates the notion of absolute continuity above with that of \cref{calculus:absolute_continuity}.
    \begin{property}[Absolute continuity]
        Let $\mu,\nu$ be finite measures on a measurable space $(X,\Sigma)$. Then $\nu\ll\mu$ if and only if
        \begin{gather}
            \forall\varepsilon>0:\exists\delta>0:\forall A\in\Sigma:\mu(A)<\delta\implies\nu(A)<\varepsilon\,.
        \end{gather}
    \end{property}

    \newdef{Singular measures}{\index{measure!singular}\index{orthogonal!measure|see{measure, singular}}
        Consider two measures $\mu,\nu$. If there exists a set $A$ such that $\mu(A)=0=\nu(A^c)$, they are said to be singular (or \textbf{orthogonal}). This is denoted by $\mu\perp\nu$.
    }
    \begin{theorem}[Lebesgue's decomposition theorem]
        Let $\mu,\nu$ be two $\sigma$-finite measures. There exist two other $\sigma$-finite measures $\nu_a,\nu_s$ such that $\nu=\nu_a+\nu_s$, where $\nu_a\ll\mu$ and $\nu_s\perp\mu$.
    \end{theorem}

    \newdef{Dominated measure}{\index{measure!dominated}
        Let $\mu,\nu$ be two measures defined on a measurable space $(X,\Sigma)$. Then $\mu$ is said to \textbf{dominate} $\nu$ if $0\leq\nu(F)\leq\mu(F)$ for every $F\in\Sigma$.
    }

    \begin{theorem}[Radon--Nikodym theorem]\label{measure:radon_nikodym}
        Let $(X,\Sigma)$ be a measurable space and let $\mu,\nu$ be two $\sigma$-finite measures defined on $\Sigma$ such that $\nu\ll\mu$. There exists a nonnegative, measurable function $f:X\rightarrow\mathbb{R}$ such that
        \begin{gather}
            \nu(A) = \Int_Af\,d\mu
        \end{gather}
        for all $A\in\Sigma$.
    \end{theorem}
    \newdef{Radon--Nikodym derivative}{\index{derivative|seealso{Radon--Nikodym}}
        The function $f$ in the previous theorem is called the Radon-Nikodym derivative of $\nu$ with respect to $\mu$. It is generally denoted by $\deriv{\nu}{\mu}$.
    }

    \remark{The function $f$ in this theorem is unique up to a $\mu$-null (and thus $\nu$-null) set.}
    \begin{property}[Integrability]
        In general the Radon--Nikodym derivative is not integrable (unless the measures are finite). However, it is always locally integrable (\cref{measure:locally_integrable}). Together with \cref{measure:measure_by_integral}, this implies that (densities of) absolutely continuous measures are in bijection with locally integrable functions.
    \end{property}

    \begin{property}[Change of variables]
        Let $\mu,\nu$ be finite measures such that $\nu\ll\mu$ and let $\deriv{\nu}{\mu}$ be the associated Radon--Nikodym derivative. For every $\nu$-integrable function $f$ the following equality holds
        \begin{gather}
            \Int_A f\,d\nu = \Int_Af\deriv{\nu}{\mu}\,d\mu
        \end{gather}
        for all $A\in\Sigma$.
    \end{property}

    \begin{property}\index{chain!rule}
        Let $\lambda,\nu$ and $\mu$ be $\sigma$-finite measures. If $\lambda\ll\mu$ and $\nu\ll\mu$, the following two properties hold:
        \begin{itemize}
            \item\textbf{Linearity}: $\ds\deriv{(\lambda+\nu)}{\mu} = \deriv{\lambda}{\mu} + \deriv{\lambda}{\mu}$.
            \item\textbf{Chain rule}: If $\ds\lambda\ll\nu$, then $\ds\deriv{\lambda}{\mu} = \deriv{\lambda}{\nu}\deriv{\nu}{\mu}$ a.e.
        \end{itemize}
    \end{property}

\section{Lebesgue--Stieltjes integral}\index{integral!Lebesgue--Stieltjes}

    Aside from the Lebesgue measure, one can construct some other important measures (and their associated integrals) on the Borel $\sigma$-algebra of the real line $\mathbb{R}$. These constructions will be important in the study of density functions in probability theory and the extension of calculus to stochastic processes. To this end, consider a function $F$ that is right-continuous, i.e.~$F(x^+)=F(x)$, and increasing. The length of an interval can be generalized in the following way.
    \newdef{$F$-length}{\index{length}
        The $F$-length of an interval $]a,b]$ is defined as follows:
        \begin{gather}
            l_F\bigl(]a,b]\bigr) := F(b) - F(a)\,.
        \end{gather}
        The restriction to half-open intervals assures that this function is additive when taking unions of intervals. The footnote in \cref{topology:borel_set} also assures that the $\sigma$-algebra generated by these intervals is the Borel $\sigma$-algebra on $\mathbb{R}$.
    }

    An immediate extension of \cref{measure:outer_measure} gives the outer measure associated to $F$.
    \newdef{$F$-outer measure}{\index{outer!measure}\index{measure!Lebesgue--Stieltjes}\label{measure:lebesgue_stieltjes_measure}
        Let $X\subseteq\mathbb{R}$ be a set. The (Lebesgue--Stieltjes) $F$-outer measure of $X$ is defined as follows:
        \begin{gather}
            \mu_F^*(X) := \inf\left\{\sum_{n=1}^{+\infty}l_F(I_n)\,\middle\vert\,\seq{I}\text{ a sequence of half-open intervals that cover }X\right\}\,.
        \end{gather}
    }

    Using this outer measure, one can define the $\mu_F$-measurable sets as those sets satisfying Carath\'eodory's criterion (with respect to $\mu_F^*$). The main difference with the Lebesgue measure is that $\mu_F$ is not necessarily translation invariant and that singletons are not necessarily null.
    \begin{property}[Singletons]\index{atom}
        The $F$-measure of a singleton $\{x\}$ is equal to the jump of $F$ at $x$:
        \begin{gather}
            \mu_F\bigl(\{x\}\bigr) = F(x) - F(x^-)\,.
        \end{gather}
        Such elements are examples of \textbf{atoms}, sets of positive measure for which every proper measurable subset is null. Note that by right-continuity of $F$, the number of discontinuities is countable and the only atoms of $\mu_F$ are exactly these discontinuities.
    \end{property}
    \begin{result}
        It follows that the Lebesgue--Stieltjes measures having null singletons are exactly those for which $F$ is continuous.
    \end{result}

    \begin{property}[Regularity]\index{Borel!measure}
        The Lebesgue--Stieltjes measure is a Radon measure. Furthermore, every Radon measure $\mu$ on $\mathbb{R}$ is equal to a Lebesgue--Stieltjes measure induced by the \textit{cumulative distribution function} (see \cref{prob:cdf})
        \begin{gather}
            F_\mu(x) = \mu\bigl(]-\!\infty,x]\bigr)\,.
        \end{gather}
    \end{property}

    \begin{example}[Lebesgue measure]\index{Lebesgue!measure}
        The Lebesgue measure is the Lebesgue--Stieltjes measure associated to $F(x)=x$.
    \end{example}
    \begin{example}[Dirac measure]\index{Dirac!measure}
        The Dirac measure at $x\in\mathbb{R}$ can be obtained as the Lebesgue--Stieltjes measure for $F(x)=\mathbbm{1}_{[x,\infty[}$.
    \end{example}

    \begin{property}
        If the Lebesgue--Stieltjes measure $\mu_F$ is nonatomic, then
        \begin{gather}
            F_\ast\mu_F = \lambda\,,
        \end{gather}
        for $\lambda$ the Lebesgue measure.
    \end{property}

    \begin{property}
        Let $\mu,\nu$ be two absolutely continuous measures\footnote{In fact, one can relax this to merely being nonatomic.} on $\mathbb{R}^n$ (with respect to the Lebesgue measure). There exists a unique increasing triangular Borel function $T:\mathbb{R}^n\rightarrow\mathbb{R}^n$ such that
        \begin{gather}
            T_\ast\mu = \nu\,.
        \end{gather}
        Triangular means that $Tx\equiv\bigl(f_1(x_1),f_2(x_1,x_2),\ldots,f_n(x_1,\ldots,x_n)\bigr)$ for some Borel functions $f_i:\mathbb{R}^i\rightarrow\mathbb{R}$ and increasing means that each $f_i$ is an increasing function of $x_i$.
    \end{property}
    \begin{remark}\index{quantile}
        For $\mathbb{R}$, the Borel function $T$ is obtained by first mapping $\mu$ to the Lebesgue measure on $[0,1]$ and then mapping it to $\nu$. Explicitly, let $F_\mu$ be the \textit{cumulative distribution function} (see \cref{prob:cdf}) of $\mu$:
        \begin{gather}
            F_\mu(x) := \mu\bigl(]-\infty,x]\bigr)\,,
        \end{gather}
        and let $G_\nu$ be the \textbf{quantile function} of $\nu$:
        \begin{gather}
            G_\nu(x) := \inf\{t\in\mathbb{R}\mid F_\nu(t)\geq x\}\,.
        \end{gather}
        The required transformation is then given by the composition:
        \begin{gather}
            T = G_\nu\circ F_\mu\,.
        \end{gather}

        \todo{ADD quantile function}
    \end{remark}

    \begin{remark}[Bounded variation]\label{measure:bounded_variation_integral}
        Note that the Lebesgue--Stieltjes integral can be generalized to signed measures, through functions of (locally) bounded variation, since these are exactly the functions that can be decomposed as the difference of two increasing functions on any interval.
    \end{remark}
    
    \begin{formula}[Integration by parts]\index{integration!by parts}\index{jump}\index{Froda}\label{measure:integration_by_parts}
        Let $F,G:\mathbb{R}\rightarrow\mathbb{R}$ be two right-continuous functions of lcoally bounded variation. Then
        \begin{gather}
            \begin{aligned}
                F(x)G(x) &= F(0)G(0) + \Int_0^xF(s)\,dG(s) + \Int_0^xG(s^-)\,dF(s)\\
                &= F(0)G(0) + \Int_0^xF(s^-)\,dG(s) + \Int_0^xG(s^-)\,dF(s) + \sum_{s\leq x}\Delta F(s)\Delta G(s)\,,
            \end{aligned}
        \end{gather}
        where $\Delta F(s):=F(s)-F(s^-)$ denotes the \textbf{jump} at $F$, i.e.~the Lebesgue--Stieltjes measure $\mu_F(\{s\})$. Note that for the sum on the second line, there are only countably many terms by \textit{Froda's theorem}.
    \end{formula}

    \begin{formula}[Chain rule]\index{chain rule}
        Let $f:\mathbb{R}\rightarrow\mathbb{R}$ be continuously differentiable and let $F:\mathbb{R}\rightarrow\mathbb{R}$ be right-continuous and of locally bounded variation. Then
        \begin{gather}
            f(F(x)) = f(F(0)) + \Int_0^xf'(F(s^-))\,dF(s) + \sum_{s\leq x}\bigl(f(F(s)) - f(F(s^-)) - f'(F(s^-))\Delta F(s)\bigr)\,.
        \end{gather}
    \end{formula}

\section{Signed measures}

    \newdef{Signed measure}{\index{measure!signed}
        Consider a measurable space $(X,\Sigma)$. A function $\mu:\Sigma\rightarrow\overline{\mathbb{R}}$ is called a signed measure if it satisfies the following conditions:
        \begin{enumerate}
            \item\textbf{Measure zero}: $\mu(\emptyset)=0$, and
            \item\textbf{$\sigma$-additivity}: $\forall i\neq j:E_i\cap E_j=\emptyset\implies\mu\left(\bigcup_{n=1}^{+\infty}E_n\right) = \sum_{i=n}^{+\infty}\mu(E_n)$.
        \end{enumerate}
        Note that these requirements are the same as for an ordinary measure (\cref{measure:measure}), except that now the function is allowed to become negative. The function is, however, not allowed to attain $-\infty$ to exclude undefined expressions such as $\infty-\infty$.
    }
    \begin{remark}
        An important consequence of this generalization is that signed measures are not necessarily monotonic, i.e.~$A\subseteq B\slashed{\implies}\mu(A)\leq\mu(B)$. In fact, this is a strict relation. A signed measure is monotonic if and only if it is a genuine measure.
    \end{remark}

    \newdef{Total variation}{\index{variation}\index{norm!total variation}\label{measure:total_variation}
        Consider a signed measure $\mu$ on a measurable space $(X,\Sigma)$. The total variation $|\mu|$ is the measure defined as follows:
        \begin{gather}
            |\mu|(A) := \sup\left\{\sum_{P\in\mathcal{P}}|\mu(P)|\,\middle\vert\,\mathcal{P}\subset\Sigma,\mathcal{P}\text{ partitions }A\right\}\,.
        \end{gather}
        If one chooses $A=X$, the total variation norm is obtained: $\|\mu\|_{\text{TV}} := |\mu|(X)$.
        
        Using this measure, one can decompose the signed measure $\mu$ as a difference of two genuine measures:
        \begin{align}
            \mu &= \mu^+-\mu^-\nonumber\\
            &= \frac{1}{2}(|\mu|+\mu)+\frac{1}{2}(|\mu|-\mu)\,.
        \end{align}
        Furthermore, this decomposition is minimal in the sense that if $\mu=\lambda_1-\lambda_2$ for any two measures, then $\mu^+\leq\lambda_1$ and $\mu^-\leq\lambda_2$.
    }

    The following theorem generalizes both the Radon--Nikodym and Lebesgue decomposition theorems to the case of signed measures.
    \begin{theorem}\index{Radon--Nikodym}\label{measure:signed_radon_nikodym}
        Consider a $\sigma$-finite signed measure $\mu$ and a $\sigma$-finite measure $\nu$ on a measurable space $(X,\Sigma)$. There exists a $\nu$-a.e.~unique integrable function $f\in L^1(\nu)$ and a $\sigma$-finite measure $\mu_s\perp\nu$ such that for all $A\in\Sigma$:
        \begin{gather}
            \mu(A) = \Int_Af\,d\nu + \mu_s(A)\,.
        \end{gather}
    \end{theorem}
    As before, the function $f$ in the preceding is called the Radon--Nikodym derivative of $\mu$.

    \begin{theorem}[Hahn--Jordan]\index{Hahn--Jordan}
        Consider a signed measure $\mu$ on a measurable space $(X,\Sigma)$. There exists a set $A\in\Sigma$ such that the minimal decomposition $\mu=\mu^+-\mu^-$ in terms of two measures $\mu^\pm$ is given by
        \begin{gather}
            \mu^+(B) = \mu(A\cap B)\qquad\qquad\mu^-(B)=\mu(A^c\cap B)\,.
        \end{gather}
    \end{theorem}

    \newdef{Integral with respect to a signed measure}{\index{integral!signed measure}
        Let $\mu$ be a signed measure on a measurable space $(X,\Sigma)$ and consider a measurable function $f$ on $A\in\Sigma$. The integral of $f$ with respect to $\mu$ is defined as follows:
        \begin{gather}
            \Int_Af\,d\mu := \Int_Af\,d\mu^+ - \Int_Af\,d\mu^-\,.
        \end{gather}
    }

    \newdef{Lebesgue--Stieltjes signed measure}{\index{measure!Lebesgue--Stieltjes}
        Let $F$ be a function of bounded variation. According to \cref{calculus:bounded_variation_decomposition}, it can be written as $F=F_1-F_2$, where $F_1,F_2$ are monotonically increasing, absolutely continuous functions. The \textbf{Lebesgue--Stieltjes (signed) measure} associated to $F$ is defined as $\mu_F := \mu_{F_1}-\mu_{F_2}$.
    }

    \begin{theorem}[Fundamental theorem of calculus]\index{fundamental theorem!of calculus}
        Let $F$ be an absolutely continuous function on the closed interval $[a,b]$. Then $F$ is differentiable $\lambda$-a.e.~($\lambda$ being the Lebesgue measure) and its associated Lebesgue--Stieltjes measure $\mu_F$ has Radon--Nikodym derivative $\deriv{\mu_F}{\lambda}=F'$ $\lambda$-a.e. Furthermore, for all $x\in[a,b]$, one has
        \begin{gather}
            F(x) - F(a) = \mu_F([a,x]) = \Int_a^xF'(t)\,dt\,.
        \end{gather}
    \end{theorem}
    \begin{result}
        If $F$ is absolutely continuous and $F'=0$ $\lambda$-a.e., then $F$ is constant.
    \end{result}

\section{Bochner integral}\index{Bochner!integral}\label{section:bochner_integral}

    In this section, the Lebesgue integral is generalized to functions taking values in general \textit{Banach spaces} (see \cref{section:banach}). Since addition still makes sense in these spaces, \cref{measure:integral_simple_function} can still be used.
    \newdef{Bochner integral of simple functions}{\index{Lebesgue!integral}\label{measure:bochner_integral_simple_function}
        Consider a simple function $\varphi$ on a measure space $(X,\Sigma,\mu)$ taking values in a Banach space $V$. The Lebesgue integral of $\varphi$ over a measurable set $A\in\Sigma$ with respect to $\mu$ is given by
        \begin{gather}
            \Int_A\varphi\,d\mu := \sum_{i=1}^na_i\mu(A\cap A_i)\,.
        \end{gather}
        As usual, if the domain of integration is not mentioned explicitly, an integral over the whole space $X$ is implied. If the above integral is finite, the simple function is said to be integrable.
    }

    \newdef{Bochner integrability}{
        A Banach space-valued measurable function $f:(X,\Sigma)\rightarrow(V,\Omega)$ is said to be Bochner integrable if there exists a sequence $\seq{\varphi}$ of integrable simple functions such that
        \begin{gather}
            \lim_{n\rightarrow\infty}\Int_X\|f-\varphi_n\|_V\,d\mu=0\,.
        \end{gather}
        The Bochner integral of such a function is given by
        \begin{gather}
            \Int_Xf\,d\mu:=\lim_{n\rightarrow\infty}\Int_X\varphi_n\,d\mu\,.
        \end{gather}
    }

    \todo{COMPLETE}
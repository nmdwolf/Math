\chapter{Symplectic Geometry}\label{chapter:symplectic}

    References for this chapter are \cite{mcduff, symplectic}.

\section{Symplectic manifolds}

    \newdef{Symplectic form}{
        Le $M$ be a smooth manifold. A two-form $\omega\in\Omega^2(M)$ is said to be symplectic if it satisfies the following properties:
        \begin{enumerate}
            \item\textbf{Closedness}: $d\omega = 0$, and
            \item\textbf{Nondegeneracy}: $\iota_X\omega = 0\implies X=0$.
        \end{enumerate}
        If the closedness condition is dropped, one obtains an \textbf{almost symplectic form}.
    }
    \newdef{Symplectic manifold}{\index{symplectic!manifold}
        A manifold equipped with a symplectic form.
    }

    \begin{property}[Dimension]
        From the antisymmetry and the nondegeneracy of the symplectic form it follows that the manifold is necessarily even-dimensional.
    \end{property}

    \begin{theorem}[Darboux]\index{Darboux}
        Let $(M,\omega)$ be a symplectic manifold. For every neighbourhood $\Omega$ in $T^*M$ there exists an adapted coordinate system $(q^i,p^i)$ such that
        \begin{gather}
            \left.\omega\right|_\Omega = \sum_idp^i\wedge dq^i.
        \end{gather}
        \emph{The adapted charts in this theorem are called \textbf{Darboux charts}. Any symplectic manifold admits a covering by Darboux charts.}
    \end{theorem}
    \begin{remark}
        The Darboux theorem shows that all symplectic manifolds of the same dimension are locally isomorphic and, therefore, admit no local invariants. This is in stark contrast to for example Riemannian manifolds.
    \end{remark}

    \begin{formula}
        In Darboux coordinates the components of the symplectic form $\omega$ are given by
        \begin{gather}
            \omega =
            \begin{pmatrix}
                0&-\mathbbm{1}\\
                \mathbbm{1}&0
            \end{pmatrix}.
        \end{gather}
        Using the nondegeneracy condition one can define the ``dual'' or inverse $\omega^\sharp$ as
        \begin{gather}
            \omega^\sharp =
            \begin{pmatrix}
                0&\mathbbm{1}\\
                -\mathbbm{1}&0
            \end{pmatrix}.
        \end{gather}
        Note that the literature is very divided on what convention to use. Some authors use the opposite of the above convention.
    \end{formula}

    \begin{property}\label{symplectic:symplectic_G_structure}
        In the language of $G$-structures (Section \ref{section:G-structure}) one can restate the definition of symplectic manifolds. A smooth $2n$-dimensional manifold is almost symplectic exactly if it admits an $\mathrm{Sp}(2n)$-structure. It is symplectic exactly if the $\mathrm{Sp}(2n)$-structure is integrable, which by Darboux's theorem is equivalent to first-order integrability, i.e. $d\omega=0$.
    \end{property}

    \newdef{Symplectic potential}{\index{symplectic!potential}
        By the Poincaré lemma \ref{bundle:poincare} the symplectic form $\omega$ locally defines a one-form:
        \begin{gather}
            \omega = d\theta.
        \end{gather}
        This one-form is sometimes called the sympletic potential of $(M,\omega)$.
    }
    \begin{construct}[Liouville one-form]\index{Liouville!form}\index{tautological!form|see{Liouville}}
        Let $M$ be a smooth manifold. The cotangent bundle $\pi:T^*M\rightarrow M$ comes equipped with a canonical symplectic form that can be derived from a \textbf{tautological one-form}. In local coordinates on $T^*M$ it is given by \[\alpha := p_idq^i.\] In coordinate-free notation this can be written as follows:
        \begin{gather}
            \label{symplectic:liouville}
            \alpha|_y(X) = y(\pi_*X),
        \end{gather}
        where $X\in T_yT^*M$. This one-form serves as a symplectic potential for the cotangent bundle: \[\omega = d\alpha.\]
    \end{construct}

    \newdef{Liouville vector field}{\index{Liouville!vector field}
        Consider a symplectic manifold $(M,\omega)$. A vector field $X$ is called a Liouville vector field if it preserves the symplectic form:
        \begin{gather}
            \mathcal{L}_X\omega = \omega.
        \end{gather}
        By Cartan's magic formula a global symplectic potential is then given by
        \begin{gather}
            \theta := X\intmul\omega.
        \end{gather}
        It follows that the existence of a Liouville vector field implies the exactness of $M$.
    }

    \newdef{Multisymplectic structure}{\index{multi-!plectic manifold}
        The definition of a symplectic structure can be generalized to multisymplectic or \textbf{$n$-plectic} structures as $(n+1)$-forms $\omega$ that satisfy the conditions:
        \begin{enumerate}
            \item\textbf{Closedness}: $d\omega = 0$, and
            \item\textbf{Nondegeneracy}: $\iota_X\omega = 0\implies X=0$.
        \end{enumerate}
    }
    \begin{construct}[Tautological form]
        The construction of the tautological one-form on cotangent bundles $T^*M$ can be generalized to exterior powers of the cotangent bundle through Equation \eqref{symplectic:liouville}:
        \begin{gather}
            \alpha|_y(X_1,\ldots,X_n) := y\big(\pi_*X_1,\ldots,\pi_*X_n\big),
        \end{gather}
        where $X_1,\ldots,X_n\in T_y\Lambda^nT^*M$.
    \end{construct}
    \begin{example}[Killing form]\index{Killing!form}
        The transgression 3-cocycle \ref{lie:killing_transgression} induced by the Killing form of a compact simple Lie group turns this group into a 2-plectic manifold.
    \end{example}

    \newdef{Symplectomorphism}{\index{symplectomorphism}
        A symplectomorphism is an isomorphism of symplectic manifolds, i.e. a diffeomorphism $f:(M,\omega_M)\rightarrow(N,\omega_N)$ satisfying
        \begin{gather}
            f^*\omega_N = \omega_M.
        \end{gather}
        These functions form an infinite-dimensional Lie group called the \textbf{symplectomorphism group}. This should not be confused with the symplectic group $\mathrm{Sp}(n)$.
    }
    \newdef{Symplectic vector field}{\index{vector field!symplectic}\label{symplectic:symplectic_vector_field}
        A vector field is said to be symplectic if its flow preserves the symplectic form $\omega$:
        \begin{gather}
            \mathcal{L}_X\omega = 0.
        \end{gather}
        Equivalently, a vector field is symplectic if its flow is a symplectomorphism. These vector fields form a Lie subalgebra of $\mathfrak{X}(M)$.
    }

    \begin{theorem}[Gromov's nonsqueezing theorem\footnotemark]
        \footnotetext{Also called the theorem of the \textbf{symplectic camel} (after a paper by \textit{I. Stewart}).}
        Consider the Euclidean ball and cylinder $B^{2n}(r)$ and $Z^{2n}(R)$ in $\mathbb{R}^{2n}$ with their standard (induced) symplectic forms. If there exists a symplectic embedding $B^{2n}(r)\hookrightarrow Z^{2n}(R)$, then $r\leq R$.
    \end{theorem}
    This theorem implies that one cannot arbitrarily squeeze a symplectic manifold while preserving its (symplectic) volume. This hints towards the existence of certain invariants:
    \newdef{Symplectic capacity}{\index{capacity}
        Consider the category of symplectic manifolds with symplectic embeddings as morphisms. A symplectic capacity is a functor $C$ from this category to the poset of (nonnegative) real numbers (this just means that it is a monotonic function on symplectic manifolds) satisfying the following conditions:
        \begin{enumerate}
            \item\textbf{Homogeneity}: If $\alpha\neq0$, $C(M,\alpha\omega)=|\alpha|C(M,\omega)$.
            \item\textbf{Normalization}: $C(B^{2n}(r))=C(Z^{2n}(r))=\pi r^2$.
        \end{enumerate}
    }
    \begin{example}
        Inspired by \textit{Gromov}'s theorem, one can obtain capacities by taking the largest ball that embeds into the manifold and the smallest cylinder such that the manifolds embeds in it. These define the minimal and maximal capacities, respectively.
    \end{example}

\section{Hamiltonian vector fields}\label{section:hamiltonian_vector_fields}

    \newdef{Hamiltonian vector field}{\index{Hamilton!vector field}\label{symplectic:hamilton_vectorfield}
        Let $(M,\omega)$ be a symplectic manifold. For every function $f\in C^\infty(M)$ the associated Hamiltonian vector field $X_f$ is defined by the following equation\footnote{A lot of different conventions exist in the literature. Here the one that is compatible with the Hamiltonian equations \ref{lagrange:hamilton_equations} (which are universally accepted) is used.}:
        \begin{gather}
            \label{symplectic:hamilton_vectorfield1}
            \omega(X_f,\cdot) = -df(\cdot).
        \end{gather}
        This can be rewritten using $\omega^\sharp$ as
        \begin{gather}
            X_f(\cdot) = \omega^\sharp(-df,\cdot).
        \end{gather}
        Hamiltonian vector fields form a Lie subalgebra of the Lie algebra of symplectic vector fields. The flow associated to a Hamiltonian vector field is sometimes called a \textbf{Hamiltonian symplectomorphism}.\footnote{The fact that the Hamiltonian flow indeed preserves the symplectic form follows from the closedness of $\omega$.}
    }

    \newdef{Poisson bracket}{\index{Poisson!bracket}\index{Lagrange!bracket}\label{symplectic:poisson}
        Let $(M,\omega)$ be a symplectic manifold. The Poisson bracket of two functions $f,g\in C^\infty(M)$ is defined as
        \begin{gather}
            \{f,g\} := X_f(g)
        \end{gather}
        or, equivalently, as
        \begin{gather}
            X_{\{f,g\}} := [X_f,X_g],
        \end{gather}
        where $X_f,X_g$ are the Hamiltonian vector fields associated to $f$ and $g$. In Darboux coordinates the Poisson bracket of the coordinates is represented by the dual matrix $\omega^\sharp$. The bracket operation represented by the symplectic form itself is often called the \textbf{Lagrange bracket}.
    }
    \begin{property}\label{symplectic:poisson_hamiltonian_surjection}
        The Poisson bracket induced by the symplectic form turns the structure $(C^\infty(M),\{\cdot,\cdot\})$ into a Lie algebra and the second equation above gives a (surjective\footnote{The kernel is given by the constant functions and, hence, it is not a bijection.}) Lie algebra morphism \[(C^\infty(M),\{\cdot,\cdot\})\rightarrow(\{X\mid X\text{ is a HVF on M}\},[\cdot,\cdot]).\] Furthermore, together with the pointwise multiplication the structure becomes a Poisson algebra \ref{lie:poisson_algebra}.
    \end{property}

    \newdef{Poisson manifold}{\index{Poisson!manifold}\label{symplectic:poisson_manifold}
        A smooth manifold on which the algebra of smooth functions can be equipped with a Poisson algebra structure. This is equivalently encoded in a bivector field $\Pi\in\Gamma(\Lambda^2TM)$ such that
        \begin{gather}
            \{f,g\} = df\wedge dg(\Pi).
        \end{gather}
    }
    \begin{property}
        Every symplectic manifold is a Poisson manifold. The converse, however, is not true.
    \end{property}

    By generalizing this structure even more, one obtains Jacobi manifolds:
    \newdef{Jacobi manifold}{\index{Jacobi!manifold}
        A smooth manifold on which the algebra of smooth functions can be equipped with a Lie algebra structure $\{\cdot,\cdot\}$ such that
        \begin{gather}
            \supp(\{f,g\})\subseteq\supp(f)\cap\supp(g).
        \end{gather}
        This is equivalent to the existence of a vector field $v\in\Gamma(TM)$ and a bivector field $\Pi\in\Gamma(\Lambda^2TM)$ such that
        \begin{gather}
            \{f,g\} = (fdg-gdf)(v) + df\wedge dg(\Pi).
        \end{gather}
    }

    \begin{remark}[Multisymplectic geometry]\label{symplectic:hamiltonian_forms}
        Most of this section can be generalized to multisymplectic manifolds. For example, the definitions of Hamiltonian vector fields and the induced Lie algebra structure remain virtually the same. However, given a $(n-1)$-form $\zeta$ there might not exist a vector field $X_\zeta$ that satisfies Equation \eqref{symplectic:hamilton_vectorfield1}. If one restricts to the subspace $\mathrm{Ham}(M)$ of the \textbf{Hamiltonian forms} that do induce a Hamiltonian vector field, one can define a generalized Poisson structure as follows:
        \begin{gather}
            \{\zeta,\xi\} := \mathcal{L}_{X_\xi}\zeta.
        \end{gather}
        Since the Lie derivative of a function along a vector field is equal to the action of the vector field on the function, it can be seen that this definition reduces to the ordinary definition in the case of $n=1$. For $n>1$ the structure is not exactly Poisson because the bracket is only antisymmetric up to an exact form. Following \cite{category_symplectic} this will be called the \textbf{hemi-bracket}.

        For $n=1$ one can equivalently define the Poisson (semi)bracket as
        \begin{gather}
            \{f,g\}_s := \iota_{X_g}\iota_{X_f}\omega.
        \end{gather}
        For $n>1$, however, this structure differs from the hemi-bracket by an exact form:
        \begin{gather}
            \{\zeta,\xi\} = \{\zeta,\xi\}_s + d\iota_{X_\zeta}\xi.
        \end{gather}
        The \textbf{semi-bracket} $\{\cdot,\cdot\}_s$ now satisfies antisymmetry, but it fails to satisfy the Jacobi identity. Although these brackets do not define a Lie algebra, they do define a Lie 2-algebra \ref{hda:2-algebra} where $L_1:=C^\infty(M)$ and $L_0:=\mathrm{Ham}(M)$. They are isomorphic as Lie 2-algebras.
    \end{remark}

    \begin{remark}[Presymplectic manifolds]
        A manifold equipped with a closed two-form (sometimes required to satisfy certain conditions on its rank). In contrast to symplectic manifolds, which also satisfy a nondegeneracy condition, there is no surjective correspondence between smooth functions and Hamiltonian vector fields as given in Property \ref{symplectic:poisson_hamiltonian_surjection}. In this setting a Hamiltonian vector field should really be viewed as a pair $(X,h)$ of a smooth function $h$ and a vector field $X$ such that $X=X_h$.
    \end{remark}

\section{Lagrangian submanifolds}

    \newdef{Symplectic complement}{\index{complement!symplectic}
        Let $(M,\omega)$ be a symplectic manifold and let $S\subset M$ be an embedded submanifold $\iota:S\hookrightarrow M$. The symplectic orthogonal complement $T^\bot_pS$ (sometimes denoted by $T^\omega_pS$) at the point $p\in S$ is defined as the subspace
        \begin{gather}
            T^\bot_pS := \{v\in T_pM\mid\omega(v,\iota_*w) = 0,\forall w\in T_pS\}.
        \end{gather}
    }

    \newdef{Isotropic submanifold}{\index{iso-!tropic manifold}
        Let $(M,\omega)$ be a symplectic manifold. An embedded submanifold $\iota:S\hookrightarrow M$ is said to be isotropic if $\iota_*T_pS\subset T^\bot_pS$ or, equivalently, if $\omega|_S\equiv0$. It is said to be coisotropic if $T^\bot_pS\subset\iota_*T_pS$.
    }
    \begin{property}[Characteristic distribution]\label{symplectic:characteristic_distribution}
        Consider a symplectic manifold $M$ and let $\iota:S\hookrightarrow M$ be a coisotropic submanifold. The orthogonal complement $T^\perp S$ defines an integrable distribution where $S$ is foliated by isotropic leaves. More generally, if $\iota:S\hookrightarrow M$ is an embedded submanifold, the intersection $TS\cap T^\perp S$ defines an isotropic foliation of $S$ (if the intersection is of constant rank).
    \end{property}

    \newdef{Lagrangian submanifold}{\index{Lagrange!submanifold}
        Let $(M,\omega)$ be a symplectic manifold. An embedded submanifold $\iota:S\hookrightarrow M$ is said to be Lagrangian if $\iota_*T_pS = T^\bot_pS$. This is equivalent to $S$ being isotropic and satisfying $\dim(S)=\frac{1}{2}\dim(M)$. Therefore, they are sometimes called maximal isotropic submanifolds.
    }

    \begin{example}[Closed sections]\label{symplectic:closed_section_submanifold}
        Consider a closed section of a cotangent bundle $T^*M$, i.e. a map $\sigma:M\rightarrow T^*M$ such that $d\sigma=0$. The graph of $\sigma$ is a Lagrangian submanifold.
    \end{example}
    \begin{example}[Conormal bundle]
        Consider a submanifold $\iota:S\hookrightarrow M$ of a symplectic manifold. The conormal bundle
        \begin{gather}
            N^*S := \{(p,\alpha)\mid\forall v\in T_pS:\alpha(v)=0\}
        \end{gather}
        is a Lagrangian submanifold of $T^*M$.
    \end{example}

    \begin{theorem}[Maslov \& H\"ormander]\index{Maslov-H\"ormander}\index{Morse!family}\index{generating!function}\label{symplectic:maslow_hormander}
        Let $M$ be a smooth manifold and consider a smooth function $W:M\times\mathbb{R}^k\rightarrow\mathbb{R}$ with $k\geq 0$. If 0 is a regular value of the map \[\pderiv{W}{q}:M\times\mathbb{R}^k\rightarrow\mathbb{R}^k,\] the subset $\Lambda\subset T^*M$, locally defined by the equations
        \begin{gather}
            \pderiv{W}{u^i} = 0 \qquad\qquad p_i=\pderiv{W}{q^i},
        \end{gather}
        is a Lagrangian submanifold. Conversely, if $\Lambda\xhookrightarrow{\iota}T^*M$ is a Lagrangian submanifold, then at every $\lambda_0\in\Lambda$ there exists an integer \[k_0\geq\dim(M) - \rk\left(D(\pi\circ\iota)|_{\lambda_0}\right)\] such that locally around $\lambda_0$ the submanifold $\Lambda$ is described by some function $W:M\times\mathbb{R}^{k_0}$ satisfying the above equations.
    \end{theorem}
    Any function $W$ generating a Lagrangian submanifold through the above equations will be called a \textbf{generating function}. Functions satisfying both the equations and the regularity condition are called \textbf{Morse families}.

    \newdef{Liouville class}{\index{Liouville!class}
        Consider an exact symplectic manifold $(M,\omega)$ and a Lagrangian submanifold $L$. Because $\omega|_L=0$, every (global) symplectic potential $\theta$ defines a cohomology class $[\theta]\in H^1_\mathrm{dR}(L)$.\footnote{This class depends on the choice of potential.} For $M$ a cotangent bundle and $\theta$ the Liouville one-form, this is called the Liouville class of $L$.
    }

    \newdef{Real polarization}{\index{polarization}
        A (real) polarization of a symplectic manifold $(M,\omega)$ is a foliation by Lagrangian submanifolds, i.e. a subbundle $P\subset TM$ such that the following conditions are satisfied:
        \begin{enumerate}
            \item\textbf{Maximality}: $\dim(TM)=2\dim(P)$,
            \item\textbf{Isotropy}: $\iota_X\omega = 0$ for all $X\in P$, and
            \item\textbf{Involutivity}: $[X,Y]=0$ for all $X,Y\in P$.
        \end{enumerate}
        The last condition characterizes $P$ as an integrable subbundle by Frobenius's theorem. In fact, this implies that $TM$ is locally spanned by Hamiltonian vector fields.
    }
    More generally one can define a (complex) polarization:
    \newdef{Polarization}{
        An integrable Lagrangian subbundle $\mathcal{P}$ of the complexified tangent bundle $TM^\mathbb{C}$ with the additional property that $\dim(\mathcal{P}\cap\overline{\mathcal{P}}\cap TM)$ is constant throughout the entire manifold.

        A real polarization is (after complexifying it) the same as a complex polarization for which $\mathcal{P}=\overline{\mathcal{P}}$.
    }
    \begin{remark}
        The constant rank condition implies that there exists a real subbundle $D\subset TM$ such that $D\otimes\mathbb{C}\cong\mathcal{P}\cap\overline{\mathcal{P}}$.
    \end{remark}

    \begin{example}[Vertical polarization]\index{vertical!polarization}
        Consider the cotangent bundle $T^*M$ of a smooth manifold $M$. Define the bundle $\mathcal{P}$ at every point $\alpha\in T^*M^\mathbb{C}$ as \[\ker(\pi_*)=T_\alpha T^*_{\pi(\alpha)}M\otimes\mathbb{C}=\mathrm{span}_\mathbb{C}\left\{\pderiv{}{p_i}\,\middle\vert\,p_i\text{ is a Darboux coordinate on }T^*M\right\},\] where $\pi:T^*M^\mathbb{C}\rightarrow M$ is the (complexified) cotangent bundle projection. It can be shown that this polarization is real.
    \end{example}

    \newdef{Admissible polarization}{
        Let $\mathcal{P}$ be a polarization of a manifold $M$. This defines two new subbundles $D:=\mathcal{P}\cap\overline{\mathcal{P}}\cap TM$ and $E:=(\mathcal{P}+\overline{\mathcal{P}})\cap TM$ which are each others symplectic complement. These are sometimes called the \textbf{isotropic} and \textbf{coisotropic} distributions, respectively. Since $\mathcal{P}$ is integrable, $D$ is too. A polarization is said to be (strongly) admissible if $E$ is also integrable and the leaf sets $M/D$ and $M/E$ are smooth manifolds. (Sometimes the projection $M/D\rightarrow M/E$ is required to be a submersion.)
    }
    \newdef{K\"ahler polarization}{\index{K\"ahler!polarization}
        A polarization $\mathcal{P}$ such that $\mathcal{P}\cap\overline{\mathcal{P}}=\emptyset$. In this case $\mathcal{P}$ is always admissible and $E=TM$. Every \textit{K\"ahler manifold} (see Chapter \ref{chapter:complex_geometry}) admits such a polarization, namely its (anti)holomorphic tangent bundle. Conversely, the existence of a K\"ahler polarization implies that the manifold is (pseudo)K\"ahler, where the (anti)holomorphic bundle is given by the polarization. More generally, the (anti)holomorphic tangent bundles of any complex bundle are called the (anti)holomorphic polarization.
    }

    \newdef{Lagrangian Grassmannian}{\index{Grassmannian!Lagrangian}
        The set of isomorphism classes of Lagrangian subspaces of $\mathbb{R}^{2n}$. It is often denoted by $\mathrm{LGr}(n)$. It can be shown to be isomorphic to the homogeneous space $\mathrm{U}(n)/\mathrm{O}(n)$.
    }
    \begin{property}[Maslov class]\index{Maslov!index}\index{Maslov!cycle}
        The square of the determinant function gives an isomorphism $\pi_1(\mathrm{LGr}(n))\cong\pi_1(S^1)$, where the generator of $\pi_1(S^1)$ is identified with the path of unitary matrices $\mathrm{diag}(e^{i\pi t},1,1,\ldots)$. By the Hurewicz theorem and the definition of (singular) cohomology, one obtains $H^1(\mathrm{LGr}(n))\cong\mathbb{Z}$, which is just the degree of the induced element in $\pi_1(S^1)$. This class is called the \textbf{Maslov class} or Maslov index.

        The Maslov index (for loops) of a Lagrangian submanifold $L\subset M$ is defined analogously. Every loop in $L$ defines a loop in $\mathrm{LGr}(n)$, since the tangent spaces are Lagrangian subspaces. The Maslov index $\mu(L):\pi_1(L)\rightarrow\mathbb{Z}$ is defined as the Maslov index of this induced loop.

        This definition can be extended to arbitrary paths of Lagrangians. For any Lagrangian $L_0\in\mathrm{LGr}(n)$, one can stratify the Grassmannian by the sets
        \begin{gather}
            \Lambda_k := \{L\in\mathrm{LGr}(n)\mid\dim(L\cap L_0) = k\}.
        \end{gather}
        The union $\Sigma(L_0):=\bigcup_{i=1}^n\Lambda_i\cong\overline{\Lambda_1}$ is called the \textbf{Maslov cycle} (with respect to $L_0$). It consists of all subspaces that intersect nontrivially with $L_0$. The Maslov index $\mu$ is the Poincar\'e dual to this codimension-1 cycle, it counts the number of intersections of the given path with the Maslov cycle.

        Another approach passes through relative homology. Because the stratum $\Lambda_0$ is contractible, the long exact sequence in homology implies the existence of an isomorphism $H_1(\mathrm{LGr}(n),\Lambda_0)\cong H_1(\mathrm{LGr}(n))\cong\mathbb{Z}$. Combined with the Hurewicz isomorphism and degree map, one again gets a Maslov index for paths with endpoints in $\Lambda_0$. It immediately follows that a path entirely in $\Lambda_0$ has vanishing Maslov index.
    \end{property}
    \begin{remark}
        There exists a generalization \cite{path_maslov}, based on intersection theory, for paths with endpoints in nonzero strata, i.e. where the endpoints satisfy $\gamma(0)\in\Lambda_{k_0}$ and $\gamma(1)\in\Lambda_{k_1}$ for $k_0,k_1\geq1$. The important difference is that for these paths the Maslov index can be a half-integer:
        \begin{gather}
            \mu + \frac{k_0-k_1}{2}\in\mathbb{Z}.
        \end{gather}
    \end{remark}

    \begin{property}[Homotopy invariance]
        The Maslov index is invariant under homotopies rel endpoints.
    \end{property}
    \begin{property}[Additivity]
        If $\gamma_1$ and $\gamma_2$ are two paths that can be concatenated, then
        \begin{gather}
            \mu(\gamma_2\ast\gamma_1) = \mu(\gamma_1) + \mu(\gamma_2).
        \end{gather}
    \end{property}

    \newadef{Maslov index}{
        The above two properties give rise to an axiomatic characterization of the Maslov index:\footnote{Some authors introduce additional factors of 2 in these conditions to account for nonorientablity.}
        \begin{enumerate}
            \item\textbf{Homotopy invariance}: $\gamma_1\sim\gamma_2\iff\mu(\gamma_1)=\mu(\gamma_2)$,
            \item\textbf{Additivity}: $\mu(\gamma_1\ast\gamma_2)=\mu(\gamma_1)+\mu(\gamma_2)$,
            \item\textbf{Direct sum}: $\mu(\gamma_1\oplus\gamma_2)=\mu(\gamma_1)+\mu(\gamma_2)$, and
            \item\textbf{Normalization}: $\mu(t\mapsto e^{2\pi it})=1$.
        \end{enumerate}
    }

    \begin{formula}
        Consider a loop $\gamma:[0,1]\rightarrow\mathrm{LGr}(n)$. An explicit formula for its Maslov index is given by
        \begin{gather}
            \mu(L) = \frac{\alpha(1)-\alpha(0)}{\pi},
        \end{gather}
        where $\exp(i\alpha(t))$ is the determinant of any unitary lift of $\gamma$.
    \end{formula}

    \newdef{Pairwise Maslov index}{
        One can also extend the definition of the Maslov index to a pair of paths. To this end one should note that the diagonal map $\Delta:M\rightarrow M\times M$ gives a Lagrangian embedding. The pairwise Maslov index is then defined as follows:
        \begin{gather}
            \mu(\gamma_1,\gamma_2) := \mu(\gamma_1\times\gamma_2;\Delta),
        \end{gather}
        where the right-hand side denotes the Maslov index with respect to the $\Delta$-stratification of $\mathrm{LGr}(2n)$. However, this index is only well-defined if the pair $(\gamma_1,\gamma_2)$ is \textbf{admissible}, i.e. if $\gamma_1(0)\cap\gamma_2(0)=\emptyset=\gamma_1(1)\cap\gamma_2(1)$. For pairs that do not satisfy this condition, one should choose a \textit{compatible almost complex structure} $J$ (Chapter \ref{chapter:complex_geometry}). Then, the Maslov index is defined as follows
        \begin{gather}
            \mu(\gamma_1,\gamma_2) := \mu(\gamma_1,e^{-\theta J}\gamma_2),
        \end{gather}
        where $\theta$ is chosen such that $(\gamma_1,e^{-\theta' J}\gamma_2)$ is admissible for all $0<|\theta'|\leq\theta$. Homotopy invariance then implies that the definition does not depend on the precise choice of $\theta$.
    }

\section{Hamiltonian dynamics}\label{section:hamiltonian_dynamics}
\subsection{Dynamical systems}

    \newdef{Dynamical system}{\index{Hamilton!function}\index{dynamical!system}
        Let $(M,\omega)$ be a symplectic manifold and let $H\in C^\infty(M)$ be a distinguished ``observable''. The triple $(M,\omega,H)$ is called a dynamical system with \textbf{Hamiltonian} $H$. The time derivative of any function $F\in C^\infty(M)$ is defined by\footnote{Note that this construction can in fact be generalized to Poisson manifolds.}
        \begin{gather}
            \dot{F} := \{H,F\},
        \end{gather}
        where $\{\cdot,\cdot\}$ is the Poisson bracket on $M$. The time evolution is completely governed by the Hamiltonian flow $\exp(tX_H)$.
    }
    \newdef{Conserved quantity}{
        Let $(M,\omega,H)$ be a dynamical system. A function $F\in C^\infty(M)$ is said to be conserved if it satisfies $\dot{F}\equiv\{H,F\} = 0$.
    }

    \begin{property}[Noether's theorem]\index{Noether!theorem}
        Every function whose Poisson bracket leaves the Hamiltonian invariant is a conserved quantity:
        \begin{gather}
            \{H,Q\} = 0\iff\{Q,H\} = 0.
        \end{gather}
    \end{property}

    From here on a specific type of Hamiltonian function, called a \textbf{mechanical Hamiltonian}, is considered. Let $(Q,g)$ be a Riemannian manifold and equip the cotangent bundle $M:=T^*Q\overset{\pi}{\rightarrow}Q$ with its canonical symplectic structure. The Hamiltonians that will be considered are of the form (in local Darboux coordinates)
    \begin{gather}
        H(q,p) = \frac{1}{2}g(p,p) + V(q),
    \end{gather}
    where $V$ is smooth. These Hamiltonians have two types of symmetries (conserved quantities):
    \newdef{Kinematical symmetry}{\index{symmetry!kinematical}\label{symplectic:kinematical_symmetry}
        Consider a conserved quantity $C$. The symmetry is said to be kinematical if $\pi_*(X_C)\in\Gamma(TQ)$ exists and satisfies $\mathcal{L}_{\pi_*(X_C)}g = 0$.
    }
    \sremark{The second condition says that $\pi_*(X_C)$ is a Killing vector \ref{riemann:killing_vector}.}
    \newdef{Dynamical symmetry}{\index{symmetry!dynamical}\label{symplectic:dynamical_symmetry}
        Any symmetry that is not a kinematical symmetry.
    }

    The following algorithm gives a way to find conditions to see whether a given observable is conserved:
    \begin{method}[Van Holten's algorithm]\index{Van Holten}
        Let the conserved quantity be analytic, i.e. \[C(q,p) = \sum_{k=0}^N\frac{1}{k!}a^{(n_1\ldots n_k)}(q)p_{n_1}\ldots p_{n_k}\] for some $N\in\mathbb{N}$, where the brackets around indices denote symmetrization. For a manifold where the metric $g$ does not depend on $q$, one can rewrite $\{C,T+V\} = 0$ as \[\sum_{n=1}^N\left[\frac{1}{(k-1)!}a^{n_1\ldots n_{k-1}i}p_{n_1}\ldots p_{n_{k-1}}\pderiv{V}{q^i} - \frac{2}{k!}\pderiv{}{q^i}a^{n_1\ldots n_k}p_{n_1}\cdots p_{n_k}g^{im}p_m\right] = 0.\] Because two polynomials are equal if and only if their corresponding coefficients are equal, one obtains the following equations:
        \begin{itemize}
            \item $0^{th}$ order:
                \begin{gather}
                    a^k\pderiv{V}{q^k} = 0,
                \end{gather}
            \item $1^{st}$ order:
                \begin{gather}
                    a^{(n_1i)}\pderiv{V}{q^i} - 2\pderiv{a}{q^i}g^{in_1} = 0,
                \end{gather}
            \item $\cdots$, and
            \item $N^{th}$ order:
                \begin{gather}
                    \frac{1}{N!}a^{(n_1\ldots n_Ni)}\pderiv{V}{q^i} - \frac{2}{(N-1)!}\pderiv{}{q^i}a^{(n_1\ldots n_{N-1}}g^{i)n_N} = 0,
                \end{gather}
        \end{itemize}
        where one should pay attention to the symmetrization brackets in the second term of the last equation. Pulling down the indices by multiplying with the metric gives
        \begin{gather}
            a_{(m_1\ldots m_N)}^{\phantom{(m_1\ldots m_N)}i}\partial_iV - 2N\partial_{(m_N}a_{m_1\ldots m_{N-1})} = 0.
        \end{gather}
        The upper bound $N$ in the series expansion is determined by the generalized Killing condition \eqref{riemann:killing_tensor}:
        \begin{gather}
            \partial_{(m_{N+1}}a_{m_1\ldots m_N)} = 0\implies a_{(m_1\ldots m_{N+1})} = 0.
        \end{gather}
    \end{method}
    \begin{remark}
        The above algorithm still holds for curved manifolds upon replacing all partial derivatives $\partial_i$ by (Levi-Civita) covariant derivatives $\nabla_i$.
    \end{remark}

\subsection{Hamilton-Jacobi equation}\index{Hamilton-Jacobi equation}\label{section:hamilton_jacobi}

    \newdef{Hamilton-Jacobi equation}{
        Consider a smooth manifold $M$ such that its cotangent bundle comes equipped with a Hamiltonian function $H:T^*M\rightarrow\mathbb{R}$. The Hamilton-Jacobi equation for $H$ is the differential equation for a (smooth) function $S:M\rightarrow\mathbb{R}$ of the following form:
        \begin{gather}
            H\circ dS = 0.
        \end{gather}
        This can be rewritten as
        \begin{gather}
            \label{symplectic:hamilton_jacobi}
            H\left(q,\pderiv{S}{q}\right) = 0.
        \end{gather}
    }

    Because of Property \ref{symplectic:closed_section_submanifold} one immediately sees that the solutions of the Hamilton-Jacobi equation define a Lagrangian submanifold of the cotangent bundle. Furthermore, they have the property that they are transversal to the fibres of the projection $\pi:T^*M\rightarrow M$. By relaxing this transversality condition one obtains the following more general notion:
    \newdef{Geometric solution}{
        A Lagrangian submanifold of the level set $H^{-1}(0)$ of a smooth function $H:T^*M\rightarrow\mathbb{R}$.
    }
    \remark{It can be proven that geometric solutions can locally be described by a solution of the Hamilton-Jacobi equation.}

\subsection{Integrability}

    \newdef{Integrable system}{\index{integrability}\label{symplectic:CIS}
        \nomenclature[A_CIS]{CIS}{completely integrable system}
        Consider a smooth vector-valued function \[F\equiv(F_1,\ldots,F_n):M\rightarrow\mathbb{R}^n\] on a symplectic manifold $(M,\omega)$. This map defines a \textbf{completely integrable system} (CIS) if it satisfies the following conditions:
        \begin{enumerate}
            \item The dimension is maximal, i.e. $\dim(M) = 2n$.
            \item The Hamiltonian vector fields $\{X_{F_i}\}_{i\leq n}$ are almost everywhere linearly independent or, equivalently, the Jacobian $DF$ has full rank almost everywhere.
            \item For all $i,j\leq n:\{F_i,F_j\} = 0$.
        \end{enumerate}
    }

    \begin{property}
        Because the Poisson brackets \ref{symplectic:poisson} are related to the commutator of Hamiltonian vector fields, a CIS gives rise to a maximal set of mutually commuting vector fields. Frobenius's theorem \ref{bundle:frobenius} then says that a CIS also gives rise to a maximally integrable distribution and, hence, an $n$-dimensional regular \textit{foliation}.
    \end{property}

\section{Symplectic reduction}
\subsection{Hamiltonian actions}

    \newadef{Hamiltonian torus action}{\index{Hamilton!action}
        Let $(M,\omega)$ be a symplectic manifold. First, consider the case of an action of $G=\mathbb{R}$ or $G=S^1$ on $M$. If $G$ acts by Hamiltonian symplectomorphisms, the action is said to be \textbf{Hamiltonian}. For $G=\mathbb{R}^n$ or $G=T^n$, the action is said to be Hamiltonian, if the restriction to every factor $\mathbb{R}$ or $S^1$ is Hamiltonian.
    }

    For general Lie groups one needs the concept of a moment map:
    \newdef{Moment map}{\index{moment!map}
        Consider a Lie group $G$ with associated Lie algebra $\mathfrak{g}$ acting on a symplectic manifold $M$. The $G$-action on $M$ is said to be \textbf{Hamiltonian} with moment map $\mu:M\rightarrow\mathfrak{g}^*$ if the following conditions are satisfied:
        \begin{enumerate}
            \item For every $\xi\in\mathfrak{g}$, the map
            \begin{gather}
                \mu^\xi:M\rightarrow\mathbb{R}:p\mapsto\langle\mu(p),\xi\rangle
            \end{gather}
            is the Hamiltonian function for the fundamental vector field \ref{bundle:fundamental_vector_field} associated to $\xi$, i.e. the vector field $X_\xi$ generated by one-parameter group $e^{t\xi}$.
            \item $\mu$ is an intertwiner between the $G$-action on $M$ and the coadjoint representation \ref{lie:coadjoint_representation} on $\mathfrak{g}^*$.
            \item[2*.] Equivalently, this says that the assignment of Hamiltonian vector fields $\xi\mapsto X_\xi$ is $G$-equivariant, i.e. $X_{g^{-1}\xi g} = g^*X_\xi$.
        \end{enumerate}
        If only the first condition holds, the action is said to be \textbf{weakly Hamiltonian}.
    }

    \begin{property}
        A symplectic $S^1$-action on a symplectic manifold for which $[\omega]\in H^2(M)$ is a positive multiple of the first Chern class is Hamiltonian.
    \end{property}

    \begin{property}[Obstruction]
        If $G$ is compact and connected, $G$-equivariance of a weakly Hamiltonian action is equivalent to the assignment $\xi\mapsto H_\xi$ being a Lie algebra morphism with respect to the Poisson algebra structure on $C^\infty(M)$. This also explains the terminology ``moment map'', an action is Hamiltonian if the moment map is constant along the Hamiltonian flow, just like ordinary linear and angular momenta.

        The obstruction to a weakly Hamiltonian action being Hamiltonian is given by a class in Lie algebra cohomology (Section \ref{section:lie_algebra_cohomology}). Let $H:\mathfrak{g}\rightarrow C^\infty(M):\xi\mapsto H_\xi$ be a map assigning Hamiltonian functions to infinitesimal generators. The obstruction
        \begin{gather}
            \tau(\xi,\zeta) := \{H_\xi,H_\zeta\} - H_{[\xi,\zeta]}
        \end{gather}
        satisfies the 2-cocycle condition
        \begin{gather}
            \tau([\xi,\zeta],\theta) + \tau([\zeta,\theta],\xi) + \tau([\theta,\xi],\zeta) = 0,
        \end{gather}
        i.e. this obstruction determines a class $[\tau]\in H^2(\mathfrak{g})$. Furthermore, this class vanishes if and only if the action is Hamiltonian.
    \end{property}

    \newdef{Toric manifold}{\index{manifold!toric}
        A closed, connected, symplectic $2n$-manifold equipped with a faithful, Hamiltonian torus action of $\mathbb{T}^n$.
    }
    \begin{theorem}[Atiyah-Guillemin-Sternberg]\index{Atiyah-Guillemin-Sternberg}
        Let $(M,\omega)$ be a toric manifold with moment map $\mu:M\rightarrow\mathbb{R}^n$. The image $\mu(M)$ is a convex polytope. More precisely, the fixed point set of the action is a finite disjoint union of connected, symplectic submanifolds $C_i$, the moment map is constant on every component and the image $\mu(M)$ is the convex hull of the points $\mu(C_i)$.
    \end{theorem}

    \newdef{Delzant polytope}{\index{Delzant polytope}
        A polytope $\Delta\subset\mathbb{R}^n$ satisfying the following conditions:
        \begin{enumerate}
            \item\textbf{Symplicity}: There are $n$ edges meeting at every vertex.
            \item\textbf{Rationality}: The edges meeting at a given vertex $p$ are of the form $p+\lambda v_i$ with $v_i\in\mathbb{Z}^n$.
            \item\textbf{Smoothness}: For every vertex $p$, the slopes $\{v_i\}_{i\leq n}$ form a basis of $\mathbb{Z}^n$.
        \end{enumerate}
    }
    \begin{theorem}[Delzant]
        The symplectomorphism classes of toric manifolds are in bijection with the isomorphism classes of Delzant polytopes.
    \end{theorem}

    \begin{theorem}[Duistermaat-Heckman]\index{Duitstermaat-Heckman localization}
        Consider a symplectic manifold $(M,\omega)$ and a Hamiltonian $G$-action with moment map $\mu:M\rightarrow\mathfrak{g}^*$. The pushforward of the Liouville measure along $\mu$ is a piecewise polynomial measure:
        \begin{gather}
            \int_M (f\circ\mu)\frac{\omega^n}{n!} = \int_{\mathfrak{g}^*}fPd\lambda
        \end{gather}
        for all $f\in L^1(\mathfrak{g}^*)$, where $P$ is piecewise polynomial and $\lambda$ is the Lebesgue measure on $\mathfrak{g}^*$.
    \end{theorem}
    \begin{result}[Localization]
        Consider a circle action on a closed, symplectic $n$-manifold $(M,\omega)$, generated by a Morse function \ref{manifold:morse_function}. The Fourier transform of the Liouville measure is a piecewise polynomial function:
        \begin{gather}
            \int_Me^{-\lambda f}\frac{\omega^n}{n!} = \sum_{p\in\mathrm{Crit}(f)}\frac{e^{-\lambda f(p)}}{\lambda^n e(p)},
        \end{gather}
        where $e:M\rightarrow\mathbb{Z}$ is the product of the weights obtained by considering the tangent space $T_pM$ as an $S^1$-representation.
    \end{result}

    ?? COMPLETE ??

\subsection{Reduction}

    \begin{property}[Coisotropic reduction]
        If the leaf space of the characteristic distribution \ref{symplectic:characteristic_distribution} of a coisotropic submanifold $\iota:S\hookrightarrow(M,\omega)$ is itself a smooth manifold, it admits a symplectic form that pulls back to $\iota_*\omega$. This is for example the case when the coisotropic submanifold is \textbf{regular}, i.e. when there exists a submanifold $W\subset S$ such that $TS=TW\oplus T^\perp S$ and $W$ intersects every leaf of $S$ only once.
    \end{property}

    \newdef{Reduced vector field}{
        Let $M$ be a smooth manifold and $G$ a Lie group that acts freely and properly on $M$. This implies that the quotient space $M/G$ is a smooth manifold by Property \ref{bundle:quotient_manifold_theorem}. Now, suppose that $G$ acts as a symmetry group on some vector field $X\in\mathfrak{X}(M)$, i.e. $\Phi_g^*X=X$ for all $g\in G$. The reduced vector field $\overline{X}\in\mathfrak{X}(M/G)$ is defined through the following equation:
        \begin{gather}
            \overline{X}(\pi(m)) := \pi_*X(m),
        \end{gather}
        where $\pi:M\rightarrow M/G$ is the quotient projection.

        ?? CHECK (cursus Antwerpen) ??
    }

    \begin{theorem}[Marsden-Weinstein \& Meyer]\index{Marsden-Weinstein!reduction}
        Consider a Hamiltonian action of a connected Lie group $G$ on a symplectic manifold $(M,\omega)$ with moment map $\mu:M\rightarrow\mathfrak{g}^*$. Let $M_0:=\mu^{-1}(0)$ and consider the quotient space $\overline{M}:=M_0/G$. If $G$ acts freely and properly on $M_0$, $\overline{M}$ is a smooth manifold and, moreover, it is symplectic with the symplectic form $\overline{\omega}$ defined by the following equation:
        \begin{gather}
            \iota^*\omega = \pi^*\overline{\omega},
        \end{gather}
        where $\iota:M_0\rightarrow M$ is the canonical inclusion and $\pi:M_0\rightarrow\overline{M_0}$ the quotient projection.
    \end{theorem}
    \remark{The quotient $M/\!\!/G:=\mu^{-1}(0)/G$ is called the \textbf{Marsden-Weinstein reduction} of $M$ by $G$. There exists a more general construction, where instead of the level set of 0, the inverse image of a coadjoint orbit is considered. Furthermore, if one only requires 0 to be a regular value, the reduction process still applies, but the result is only a \textit{symplectic orbifold}.}

\subsection{Poisson reduction}

    \newdef{Poisson map}{\index{Poisson!map}
        Let $(M,\{\cdot,\cdot\})$ and $(N,[\cdot,\cdot])$ be two Poisson manifolds. A Poisson map $\Phi:M\rightarrow N$ is a smooth function satisfying the following equality for all $f,g\in C^\infty(N)$:
        \begin{gather}
            \Phi^*[f,g] = \{\Phi^*f,\Phi^*g\}.
        \end{gather}
    }
    \newdef{Poisson action}{
        Let $G$ be a Lie group and let $(M,\{\cdot,\cdot\})$ be a Poisson manifold. A $G$-action on $M$ is called a Poisson action or \textbf{canonical action} if every $g\in G$ acts by a Poisson map.
    }

    \begin{theorem}[Poisson reduction]\index{Poisson!reduction}
        Let $G$ be a Lie group that acts freely and properly on a Poisson manifold $(M,\{\cdot,\cdot\})$. If the action is canonical, the Poisson bracket on $M$ descends (uniquely) to a Poisson bracket on the quotient manifold $M/G$. Furthermore, the projection $\pi:M\rightarrow M/G$ is a Poisson map with respect to this structure.
    \end{theorem}
    \begin{property}
        Let $H:M\rightarrow\mathbb{R}$ be a $G$-invariant Hamiltonian function. Its Hamiltonian vector field $X_H$ is also $G$-invariant and the Hamiltonian vector field of the reduced Hamiltonian $h:M/G\rightarrow\mathbb{R}$, defined by $H:=h\circ\pi$, is given by the reduced vector field of $X_H$.
    \end{property}

\subsection{Lie-Poisson reduction}

    In the case of Lie-Poisson reductions one considers the cotangent bundle $T^*G$ of a Lie group $G$ as the configuration manifold. It is not too hard to show that $T^*G/G\cong\mathfrak{g}^*$.

    \newformula{Lie-Poisson equations}{\index{Lie-Poisson equations}
        First, assign to any vector field $X:Q\rightarrow TQ$ a linear function $\mu_X:T^*Q\rightarrow\mathbb{R}$ by the following formula:
        \begin{gather}
            \mu_X\left(\alpha|_q\right) := \alpha(X)|_q.
        \end{gather}
        For these functions one has $\{\mu_X,\mu_Y\} = -\mu_{[X,Y]}$.

        Now, choose a basis $\{E^i\}_{i\leq\dim(\mathfrak{g})}$ for $\mathfrak{g}^*$. This induces a basis $\{(E^i)_L\}$ of left-invariant one-forms on $G$. The projection of a one-form $\alpha\in T^*G$ onto its component associated to the basis element $(E^i)_L$ defines a linear map $\mu_i:\mathfrak{g}^*\rightarrow\mathbb{R}$ by the following formula:
        \begin{gather}
            \mu_i\circ\pi:\alpha_k(E^k)_L\mapsto\alpha_i,
        \end{gather}
        where $\pi:T^*G\rightarrow T^*G/G\cong\mathfrak{g}$ is the quotient map $(E^i)_L\mapsto E^i$. It can be shown that $\mu_i\circ\pi$ is exactly the linear map associated to the corresponding left-invariant vector field $(E_i)_L$.

        The Lie-Poisson equations for $G$ are the following set of equations:
        \begin{gather}
            \dot{\mu}_i = \{\mu_i,h\}_{\mathfrak{g}^*} = -C^k_{ij}\mu_k\pderiv{h}{\mu_j},
        \end{gather}
        where the Poisson bracket on $\mathfrak{g}^*$ is defined by applying the Poisson reduction theorem to $T^*G$.
    }

\section{Metaplectic structures}

    \newdef{Metaplectic group}{\index{metaplectic!group}
        Consider the symplectic group $\mathrm{Sp}(2n,\mathbb{R})$ as defined in \ref{linalgebra:symplectic_group}. This group admits a double covering called the metaplectic group $\mathrm{Mp}(2n,\mathbb{R})$.
    }
    \remark{In contrast to the $\mathrm{Spin}$-groups that are the double covers of $\mathrm{SO}(n)$ by Property \ref{clifford:pin_group}, the metaplectic groups are not matrix groups, i.e. they do not admit a faithful finite-dimensional representation.}

    \newdef{Metaplectic structure}{\index{metaplectic!structure}\label{symplectic:metaplectic_structure}
        Consider a symplectic $2n$-manifold $(M,\omega)$. By Property \ref{symplectic:symplectic_G_structure} the frame bundle $FM$ can be reduced to a $\mathrm{Sp}(2n)$-bundle $\pi_\mathrm{Sp}:F_\mathrm{Sp}M\rightarrow M$.

        The smooth manifold $M$ is said to have a metaplectic structure if there exists a principal $\mathrm{Mp}(2n)$-bundle $\pi_\mathrm{Meta}:P_\mathrm{Meta}\rightarrow M$ and an equivariant 2-fold lifting of $F_\mathrm{Sp}$ to $P_\mathrm{Meta}$, i.e. a morphism $\xi:P_\mathrm{Meta}\rightarrow F_\mathrm{Sp}M$ together with a 2-fold covering map $\rho:\mathrm{Mp}(2n)\rightarrow\mathrm{Sp}(2n)$ that satisfies:
        \begin{itemize}
            \item $\pi_\mathrm{Sp}\circ\xi = \pi_\mathrm{Meta}$, and
            \item $\xi(p\vartriangleleft g) = \xi(p)\cdot\rho(g)$
        \end{itemize}
        for all $g\in\mathrm{Mp}(2n)$, where $\vartriangleleft$ and $\cdot$ denote the right actions of the respective structure groups.
    }

    ?? COMPLETE ??
\chapter{\difficult{Synthetic Differential Geometry}}

\section{Neighbourhoods}

    \newdef{Neighbourhood relation}{\index{neighbourhood}
        A reflexive and symmetric relation with the additional property that the morphisms in the category under consideration preserve this relationship.
    }
    \begin{example}[Monad]\index{monad}
        Let $M$ be a set. Given a neighbourhood relation $\sim$ on $M$, the (first order) monad around $x\in M$ is defined as
        \begin{gather}
            \underline{\mathfrak{M}}(x) := \{y\in M\mid y\sim x\}.
        \end{gather}
    \end{example}

    \newdef{Infinitesimal simplex}{\index{simplex!infinitesimal}
        An infinitesimal $k$-simplex with respect to a neighbour relation $\sim$ is a collection of $k+1$ points $\{x_i\}_{i\leq k}$ such that $x_i\sim x_j$ for every $i,j\leq k$.
    }

    \newdef{Geometric distribution}{\index{distribution!involutive}
        Let $M$ be a set equipped with a neighbourhood relation $\sim$. A (geometric) distribution on $M$ is a reflexive symmetric refinement $\approx$ of $\sim$. A distribution is said to be \textbf{involutive} if
        \begin{gather}
            (x\approx y)\land(y\approx z)\land(x\sim z)\implies x\approx z
        \end{gather}
        for all $x,y,z\in M$.
    }
    \newdef{Integral subset}{\index{integral!subset}
        Let $M$ be a set equipped with a neighbourhood relation $\sim$ and an associated distribution $\approx$. A subset $N\subseteq M$ is said to be integral with respect to $\approx$ if $\approx$ and $\sim$ coincide on $N$.
    }
    \begin{theorem}[Frobenius' theorem]\index{Frobenius!integrability theorem}\index{leaf}
        An involutive distribution admits maximal connected integral subsets. These subsets are called the \textbf{leaves} of the distribution.
    \end{theorem}

\section{Affine connections}

    \newdef{Affine connection}{\index{connection!affine}
        An affine connection is a map $\lambda:M\times M\times M\rightarrow M$ that for every three points $x,y,z\in M$ such that $y\sim x$ and $z\sim x$ gives a point $w\in M$ such that $w\sim z$ and $w\sim y$. Graphically this is given by the completion of a span as shown in Diagram \ref{fig:synth_connection_complete}.
        \begin{figure}[ht!]
            \centering
            \begin{tikzpicture}
                \node (X) at (0, 0) {$x$};
                \node (Y) at (4, 1) {$y$};
                \node (Z) at (0, 1) {$z$};
                \node (W) at (4, 2) {$w$};
                \draw[->] (X) -- (Y);
                \draw[->] (X) -- (Z);
                \draw[dashed, ->] (Y) -- (W);
                \draw[dashed, ->] (Z) -- (W);
            \end{tikzpicture}
            \caption{Connection in synthetic theories.}
            \label{fig:synth_connection_complete}
        \end{figure}
    }
    \sremark{By looking at these diagrams the concept of parallel transport can be made a lot more intuitive than in classic differential geometry, e.g.~Diagram \ref{fig:synth_connection_complete} shows the parallel transport of the point $z$ along $xy$.}

    \newdef{Symmetric connection}{\index{torsion}
        An affine connection $\lambda$ is said to be symmetric or \textbf{torsion-free} if $\lambda(x,y,z) = \lambda(x,z,y)$.
    }

    \newdef{Flat connection}{\index{connection!flat}
        An affine connection $\lambda$ is said to be flat or \textbf{curvature-free} if parallel transporting a point around an infinitesimal 2-simplex leaves the point invariant.
    }
    \newdef{Curvature}{\index{curvature}
        Let $\lambda$ be an affine connection on $M$. The curvature of $\lambda$ is the map $\mathcal{R}$ that assigns to every infinitesimal 2-simplex $\{x_0,x_1,x_2\}$ the automorphism \[\underline{\mathfrak{M}}(x_0)\rightarrow\underline{\mathfrak{M}}(x_0):z\mapsto\text{result of parallel transporting z around }\{x_0,x_1,x_2\}.\]
    }

    \newdef{Geodesic}{\index{geodesic}
        A subset $S\subseteq M$ stable under the affine connection $\lambda$.
    }

\section{Euclidean geometry}
\subsection{Infinitesimal elements}

    \newdef{Infinitesimal line}{\index{infinitesimal}\label{synth:infinitesimal_line}
        Consider the line $R$. By choosing two points, 0 and 1, on $R$ it can be given the structure of a commutative ring. The infinitesimal line is defined as the following set:
        \begin{gather}
            \Delta := \{x\in R\mid x^2 = 0\}.
        \end{gather}
        A neighbourhood relation on $R$ is induced by setting $\underline{\mathfrak{M}}(0)=\Delta$.
    }
    \remark{If the Euclidean doctrine would be followed where $R\equiv\mathbb{R}$, this set would be $\{0\}$. However, by not requiring $R$ to be a field, a larger set is obtained.}

    \begin{axiom}[Kock-Lawvere]\index{Kock-Lawvere axiom}\index{slope}\label{synth:kock_lawvere_axiom}
        For every map $f:\Delta\rightarrow R$ there exists a unique element $b\in R$, called the \textbf{slope} of $f$, such that
        \begin{gather}
            f(d) = f(0) + d\cdot b
        \end{gather}
        for all $d\in\Delta$.
    \end{axiom}
    \begin{result}
        The function $\alpha:R\times R\rightarrow  R^\Delta:(a,b)\mapsto(f:d\mapsto a+d\cdot b)$ is invertible and, hence, is an isomorphism\footnote{If one equips the set $R\times R$ with the multiplication rule $(a,b)\cdot(a',b') = (a\cdot a',a\cdot b' + a'\cdot b)$, this becomes an $R$-algebra isomorphism.}.
    \end{result}
    \result{Let $a,b\in R$. If $d\cdot a=d\cdot b$ for all $d\in\Delta$, then $a=b$.}

    \begin{notation}
        Analogous to the infinitesimal line one can define the sets $\mathbb{D}^k$ in the following way:
        \begin{gather}
            \mathbb{D}^k := \{x\in R\mid x^{k+1}=0\}.
        \end{gather}
        With this notation one has $\Delta\equiv\mathbb{D}^1$.
    \end{notation}

\subsection{Calculus}

    \begin{formula}[Taylor expansion]\index{Taylor!expansion}
        The Kock-Lawvere axiom implies the following exact Taylor expansion:
        \begin{gather}
            f(x+d) = f(x) + d\cdot f'(x)\,,
        \end{gather}
        where $f'(x)$ can be interpreted as the derivative of $f$ at the point $x\in R$.
    \end{formula}
    \sremark{If $f$ also depends on additional parameters in $R$, one can define the partial derivatives in a similar fashion.}
    \begin{property}
        The Kock-Lawvere axiom implies that the derivative is linear and satisfies Leibniz's rule.
    \end{property}

    Using the sets $\mathbb{D}^k$ one can derive higher-order expansions. To this end, Axiom \ref{synth:kock_lawvere_axiom} is generalized:
    \begin{axiom}\label{synth:axiom1b}
        For every $g:\mathbb{D}^k\rightarrow R$ there exist unique elements $\{b_1,\ldots,b_k\}$ in $R$ such that
        \begin{gather}
            f(d) = f(0) + \sum_jd^j\cdot b_j
        \end{gather}
        for all $d\in\mathbb{D}^k$.
    \end{axiom}
    \begin{result}
        Let $f:R\rightarrow R$ and $d\in D_k$.
        \begin{gather}
            f(x+d) = f(x) + d\cdot f'(x) + \cdots + \frac{d^k}{k!}\cdot f^{(k)}(x)
        \end{gather}
    \end{result}
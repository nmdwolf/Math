\chapter{Lie groups and Lie algebras}\label{chapter:lie}

    References for this chapter are~\citet{jeevanjee_introduction_2015,fulton_representation_2004}. The main reference for the section on coadjoint orbits is~\citet{witten_coadjoint_1988}. For some concepts such as vector fields and pushforwards, the reader is referred to \cref{chapter:vector_bundles}.

    \minitoc

\section{Lie groups}\label{section:lie_groups}

    \newdef{Lie group}{\index{Lie!group}\label{lie:lie_group}
        A group that is also a differentiable manifold such that both the multiplication and inversion maps are smooth. For complex Lie groups, the definition of a \textit{complex manifold} is required (see \cref{chapter:complex_geometry}).
    }
    \newdef{Lie subgroup}{
        A subset of a Lie group that is both a subgroup and an immersed submanifold. If it is a regular submanifold, it is sometimes called a \textbf{regular Lie subgroup}.
    }

    \begin{theorem}[Closed subgroup theorem\footnotemark]\index{Cartan}
        \footnotetext{Sometimes simply called \textbf{Cartan's theorem}.}
        A closed subgroup of a Lie group is a regular Lie subgroup.
    \end{theorem}

    \begin{property}[Generating neighbourhoods]\label{lie:prop_connected}
        Let $G$ be a connected Lie group. Every neighbourhood $U_e$ of the identity generates $G$, i.e.~every element $g\in G$ can be written as a word in $U_e$.
    \end{property}

    \newdef{Isogeny}{\index{iso-!geny}
        Two Lie groups are said to be isogenous if one is a covering space (\cref{topology:covering_space}) of the other. The covering map is then called an isogeny between the groups.
    }

\subsection{Left invariant vector fields}

    \newdef{Left-invariant vector field}{\index{vector field!left-invariant}
        Let $G$ be a Lie group and let $X$ be a vector field on $G$. $X$ is said to be left-invariant if the following equivariance relation holds for all $g\in G$:
        \begin{gather}
            L_{g,\ast}X(h) = X(gh)\,,
        \end{gather}
        where $L$ denotes the regular (left) action on $G$. The term  `left-invariant vector field' is often abbreviated as \textbf{LIVF}.
    }
    \begin{property}
        The set $\mathfrak{X}^L(G)$ of LIVFs on a (real) Lie group $G$ is a vector space over $\mathbb{R}$.
    \end{property}
    \begin{property}[Tangent space]\label{lie:livf_prop}
        The map $L_{g,\ast}$ is an isomorphism for every $g\in G$. It follows that a LIVF is uniquely determined by its value at the identity $e\in G$. Furthermore, for every $v\in T_eG$, there exists a LIVF $X\in\mathfrak{X}^L(G)$ such that $X(e)=v$. This mapping is an isomorphism from $T_eG$ to $\mathfrak{X}^L(G)$.
    \end{property}

\subsection{One-parameter subgroups}

    \newdef{One-parameter subgroup}{\index{one-parameter subgroup}\label{lie:one_parameter_subgroup}
        A Lie group morphism $\Phi:\mathbb{R}\rightarrow G$ from the additive group of real numbers to $G$.
    }
    \begin{property}\label{lie:OPS_composition}
        Let $\Phi:\mathbb{R}\rightarrow G$ be a one-parameter subgroup and let $\Psi:G\rightarrow H$ be a Lie group morphism, then $\Psi\circ\Phi:\mathbb{R}\rightarrow H$ is a one-parameter subgroup of $H$.
    \end{property}
    \remark{The above definition and property can be generalized to the topological setting if Lie groups and Lie group morphisms are replaced by topological groups and continuous group morphisms.}

    \begin{property}[Left-invariant vector fields]\label{lie:livf_subgroup}
        All LIVFs $X$ are \textit{complete} (see \cref{bundle:complete_vector_field} for a general statement), i.e.~for every LIVF $X$ one can find an integral curve $\gamma_X$ with initial condition $\gamma_X(0)=e$ for which the maximal \textit{flow domain} (\cref{bundle:flow}) $D(X)$ is $\mathbb{R}$. This implies that the associated flow $\sigma_t$ determines a one-parameter subgroup of $G$. Conversely, for every one-parameter subgroup $\phi(t)$, one can construct a LIVF by taking $X:=\phi'(0)$. This correspondence is in fact a bijection.
    \end{property}

\section{Lie algebras}

    There are two ways to define a Lie algebra. The first one is purely algebraic and consists of a vector space equipped with a multiplication operation satisfying certain conditions. The second one establishes a direct correspondence between Lie groups and Lie algebras.

\subsection{Definitions}

    \newdef{Pre-Lie algebra}{\index{pre-!Lie algebra}\label{lie:pre_lie_algebra}
        Let $V$ be a $k$-vector space equipped with a binary operation $(\cdot,\cdot):V\times V\rightarrow V$. $(V,(\cdot,\cdot))$ is a pre-Lie algebra if the bracket satisfies the following conditions:
        \begin{enumerate}
            \item\textbf{Bilinearity}: $(\lambda x+y,z) = \lambda(x,z) + (y,z)$ for all $\lambda\in k$,
            \item\textbf{Invariance}: $((x,y),z) - (x,(y,z)) = ((x,z),y) - (x,(z,y))$.
        \end{enumerate}
    }

    \newdef{Lie algebra}{\index{Lie!algebra}\index{Lie!bracket}\index{Jacobi!identity}\label{lie:lie_algebra}
        Let $V$ be a $k$-vector space equipped with a binary operation $[\cdot,\cdot]:V\times V\rightarrow V$, called the \textbf{Lie bracket}. $(V,[\cdot,\cdot])$ is a Lie algebra if the Lie bracket satisfies the following conditions:
        \begin{enumerate}
            \item\textbf{Bilinearity}: $[\lambda x+y,z] = \lambda[x,z] + [y,z]$ for all $\lambda\in k$,
            \item\textbf{Alternativity}: $[y,y] = 0$, and
            \item\textbf{Jacobi identity}: $[x,[y,z]] + [y,[z,x]] + [z,[x,y]] = 0$.
        \end{enumerate}
    }
    \begin{remark}
        Note that, often, the alternativity condition is replaced by an antisymmetry condition. However, this is only equivalent over fields of characteristic $\mathrm{char}(k)\neq2$. For $\mathrm{char}(k)=2$, one only has that alternativity implies antisymmetry. Since the fields of interest will almost exclusively be $\mathbb{R}$ or $\mathbb{C}$, this will not pose any problems and, hence, the antisymmetry condition will always be used.
    \end{remark}

    \begin{example}
        Every pre-Lie algebra $V$ induces a Lie algebra through the commutator
        \begin{gather}
            [x,y] := (x,y) - (y,x)\,.
        \end{gather}
        Note that this does not assume that the pre-Lie structure on $V$ comes from an associative algebra operation. Only the invariance condition is required for the Jacobi identity to hold.
    \end{example}

    \newdef{Poisson algebra}{\index{algebra!Poisson}\index{Poisson|seealso{algebra}}\index{Poisson!bracket}\label{lie:poisson_algebra}
        A vector space $V$ equipped with two bilinear operations $\star$ and $\{\cdot,\cdot\}$ that satisfy the following conditions:
        \begin{enumerate}
            \item The couple $(V,\star)$ is an associative algebra.
            \item The couple $(V,\{\cdot,\cdot\})$ is a Lie algebra.
            \item The \textbf{Poisson bracket} $\{\cdot,\cdot\}$ acts as a derivation \ref{manifold:derivation} with respect to the operation $\star$, i.e.~\[\{x,y\star z\} = \{x,y\}\star z + y\star\{x,z\}\] for all $x,y$ and $z\in V$.
        \end{enumerate}
    }

    \newdef{Structure constants}{\index{structure!constants}
        Because Lie algebras are closed under the Lie bracket, the Lie bracket of two basis elements can again be expressed in terms of the same basis $\{e_k\}_{k\in I}$:
        \begin{gather}
            [e_i,e_j] = \sum_{k\in I}c_{ij}^{\ \ k}e_k\,.
        \end{gather}
        The coefficients $c_{ij}^{\ \ k}$ are called the structure constants of the Lie algebra. (Note that these constants are basis-dependent.)
    }
    \begin{property}[Isomorphism]
        Two Lie algebras are isomorphic if one can find bases for these Lie algebras such that the associated structure constants are equal.
    \end{property}

    \begin{example}[Lie algebra of LIVFs]
        Consider the vector space $\mathfrak{X}^L(G)$ of LIVFs on a Lie group $G$. Using \cref{bundle:lie_bracket}, one can show that the commutator also defines a LIVF on $G$. It follows that $\mathfrak{X}^L(G)$ is closed under Lie brackets and, hence, is a Lie algebra.
    \end{example}

    For the following alternative definition, \cref{lie:livf_prop} is used to relate the above Lie algebra of left-invariant vector fields to the tangent space at the identity.
    \newadef{Lie algebra of Lie group}{
        Let $G$ be a Lie group. The tangent space $\mathfrak{g}:=T_eG$ has the structure of a Lie algebra, where the Lie bracket is induced by the \textit{commutator of vector fields} (see \cref{bundle:lie_bracket}) in the following way:
        \begin{gather}
            [x,y]_{\mathfrak{g}} := L_{g^{-1},*}\left[L_{g,*}x,L_{g,*}y\right]\,,
        \end{gather}
        where $x,y\in T_eG$ and where $[\cdot,\cdot]$ denotes the Lie bracket on $\mathfrak{X}^L(G)$. This induces an isomorphism of Lie algebras $\mathfrak{g}\cong\mathfrak{X}^L(G)$. This isomorphism will be freely used throughout this text.
    }
    \begin{notation}
        Lie algebras are generally denoted by fraktur symbols. For example, as in the previous definition, the Lie algebra associated with the Lie group $G$ is often denoted by $\mathfrak{g}$.
    \end{notation}

    \begin{theorem}[Ado]\index{Ado}\label{lie:ado}
        Every finite-dimensional Lie algebra can be embedded as a subalgebra of $\mathfrak{gl}_n\cong M_n$.
    \end{theorem}
    \begin{theorem}[Lie's third theorem]\index{Lie!third theorem}
        Every finite-dimensional Lie algebra $\mathfrak{g}$ is the Lie algebra of a unique simply-connected Lie group $G$.
    \end{theorem}

    \newdef{Lie algebra morphism}{\index{morphism!of Lie algebras}
        A map $\Phi:\mathfrak{g}\rightarrow\mathfrak{h}$ that satisfies the equation
        \begin{gather}
            \Phi\bigl([x,y]\bigr) = \bigl[\Phi(x),\Phi(y)\bigr]
        \end{gather}
        for all $x,y\in\mathfrak{g}$.
    }
    \newformula{Induced morphism}{\index{morphism!induced}\index{differential}\label{lie:induced_homomorphism}
        Let $\phi:G\rightarrow H$ be a Lie group morphism. This morphism induces a Lie algebra morphism $\Phi:\mathfrak{g}\rightarrow\mathfrak{h}$ given by
        \begin{gather}
            \Phi(x) := \left.\deriv{}{t}\phi(e^{tx})\right|_{t=0}
        \end{gather}
        or, equivalently,
        \begin{gather}
            \phi(e^{tx}) = e^{t\Phi(x)}\,.
        \end{gather}
        The morphism $\Phi$ is sometimes called the \textbf{differential} of $\phi$.
    }

    \begin{property}[Homomorphism theorem\footnotemark]\label{lie:prop_hom}
        \footnotetext{Also called \textbf{Lie's second theorem}.}
        Let $G,H$ be two Lie groups and assume that $G$ is simply-connected. Every Lie algebra morphism $\Phi:\mathfrak{g}\rightarrow\mathfrak{h}$ corresponds to a unique Lie group morphism $\phi:G\rightarrow H$ such that $\Phi=\phi_*$.
    \end{property}

\subsection{Examples}

    \begin{example}[Cross product]
        The cross product $\times:\mathbb{R}^3\times\mathbb{R}^3\rightarrow\mathbb{R}^3$ turns $\mathbb{R}^3$ into a Lie algebra.
    \end{example}
    \begin{example}[General linear group]
        An interesting example is the Lie algebra associated to the Lie group of invertible complex\footnote{This result is also valid for real matrices.} matrices $\GL_n(\mathbb{C})$. This Lie group is a subset of its own Lie algebra $\mathfrak{gl}_n(\mathbb{C}) = M_n(\mathbb{C})$. It follows that, for every $A\in\GL_n(\mathbb{C})$ and every $B\in\mathfrak{gl}_n(\mathbb{C})$, the following equality holds:
        \begin{gather}
            L_{A,*}(B)=L_A(B)\,.
        \end{gather}
    \end{example}
    \begin{result}\label{lie:end_as_lie_algebra}
        By noting that the endomorphism ring $\End(V)$ of an $n$-dimensional vector space $V$ is given by the matrix ring $M_n(K)$, it can be seen that $\End(V)$ also forms a Lie algebra when equipped with the commutator of linear maps.
    \end{result}

    \newdef{Derivation}{\index{derivation}
        Given a Lie algebra $\mathfrak{g}$, the space of derivations $\mathfrak{Der}(\mathfrak{g})$ is defined as the space of linear maps $\dr:\mathfrak{g}\rightarrow\mathfrak{g}$ such that
        \begin{gather}
            \dr([x,y]) = [\dr x,y] + [x,\dr y]
        \end{gather}
        for all $x,y\in\mathfrak{g}$. This vector space inherits the Lie algebra structure on endomorphisms.
    }

    \begin{example}[Isometries]\index{iso-!metry}\label{lie:lie_isometry}
        The Lie algebra $\mathfrak{isom}(V)$ associated with the group of isometries $\mathrm{Isom}(V)$ of a nondegenerate Hermitian form satisfies the condition
        \begin{gather}
            \langle Xv,w \rangle = -\langle v,Xw \rangle
        \end{gather}
        for all elements $X\in\mathfrak{isom}(V)$. It follows that the Lie algebra of isometries consists of all skew-Hermitian operators.
    \end{example}

    Two explicit examples for common matrix Lie groups are given.
    \begin{example}[Lie algebra of $\mathrm{O}(3)$]\label{lie:so3}
        This Lie algebra is isomorphic to the set of $3\times3$ skew-symmetric matrices. It is important to note that $\mathfrak{o}(3)=\mathfrak{so}(3)$. The structure constants of this Lie algebra are given by the Levi-Civita symbol (\cref{vector:levi_civita_symbol}): $c_{ijk}=\varepsilon_{ijk}$.
    \end{example}
    \begin{example}[Lie algebra of $\mathrm{SU}(2)$]
        This Lie algebra is isomorphic to the set of $2\times2$ traceless skew-Hermitian matrices. This result can be generalized to arbitrary $n\in\mathbb{N}$.
    \end{example}

    Another important example is obtained by considering the subset of $\GL(2,\mathbb{C})$ consisting of matrices with unit determinant.
    \begin{example}[Lie algebra of $\mathrm{SL}(2,\mathbb{C})$]
        To compute the Lie bracket of the Lie algebra $\mathfrak{sl}(2,\mathbb{C})$, one needs to find the action of $L_{g,*}$ on any element of $\mathfrak{sl}(2,\mathbb{C})$. This is given by:
        \begin{gather}
            L_{\left(\begin{smallmatrix}a&b\\c&d\end{smallmatrix}\right),*}\left(\left.\pderiv{}{x^i}\right|_e\right)
            = \begin{pmatrix}a&0&b\\-b&a&0\\c&0&\frac{1+bc}{a}\end{pmatrix}^m_i\left.\pderiv{}{x^m}\right|_{\left(\begin{smallmatrix}a&b\\c&d\end{smallmatrix}\right)}\,,
        \end{gather}
        where the coordinate chart $(U,\phi)$ defined by
        \begin{gather}
            U = \left\{\begin{pmatrix}a&b\\c&d\end{pmatrix}\in\mathrm{SL}(2,\mathbb{C})\,\middle\vert\,a\neq0\right\}
        \end{gather}
        and
        \begin{gather}
            \phi:U\rightarrow\mathbb{C}^3:\begin{pmatrix}a&b\\c&d\end{pmatrix}\mapsto(a,b,c)
        \end{gather}
        was used. One can then use this formula to work out the Lie bracket of the basis vectors $\partial_i$ to obtain the following structure relations:
        \begin{gather}
            \label{lie:sl2c_lie_brackets}
            \begin{aligned}
                [\partial_1,\partial_2] &= 2\partial_2\,,\\
                [\partial_1,\partial_3] &= -2\partial_3\,,\\
                [\partial_2,\partial_3] &= \partial_1\,.
            \end{aligned}
        \end{gather}
    \end{example}

\subsection{Exponential map}

    \newformula{Exponential map}{\index{exponential!map}
        Let $X$ be a LIVF on $G$. The exponential map is defined as
        \begin{gather}
            \exp:\mathfrak{g}\rightarrow G:X\mapsto\gamma_X(1)\,,
        \end{gather}
        where $\gamma_X$ is the associated one-parameter subgroup from \cref{lie:livf_subgroup}.
    }

    \begin{property}[Uniqueness]
        The exponential map is the unique map $\mathfrak{g}\rightarrow G$ such that $\exp(0) = e$ and for which the restrictions to the lines through the origin in $\mathfrak{g}$ are one-parameter subgroups of $G$.
    \end{property}
    \begin{result}\label{lie:exp_result}
        Because the identity $\mathbbm{1}_{\mathfrak{g}}=\exp_\ast$ is an isomorphism, the \textit{inverse function theorem} (see \cref{bundle:inverse_function_theorem}) implies that the image of $\exp$ will contain a neighbourhood of the identity $e\in G$. If $G$ is connected, \cref{lie:prop_connected} implies that the exponential map generates all of $G$.

        Together with the property that $\psi\circ\exp = \exp\circ\,\psi_\ast$ for every Lie group morphism $\psi:G\rightarrow H$, one can conclude that if $G$ is connected, a Lie group morphism $\psi:G\rightarrow H$ is completely determined by its differential $\psi_\ast$ at the identity $e\in G$.
    \end{result}

    \begin{example}[Matrix Lie groups]\index{matrix!exponentiation}
        For matrix Lie groups, one can define the classic matrix exponential:
        \begin{gather}
            e^{tA} := \sum_{k=0}^{+\infty}\frac{(tA)^k}{k!}\,.
        \end{gather}
        This operation defines a one-parameter subgroup of $G$ and, from the uniqueness property above, it follows that this matrix exponential is in fact the exponential map for $G$. It should be noted that this formula converges for every $A\in M_{m,n}(\mathbb{R})$ and that it is invertible with inverse given by $\exp(-A)$. Using Ado's theorem~\ref{lie:ado}, one can then use this matrix exponential to represent the exponential map for any (finite-dimensional) Lie algebra.
    \end{example}
    \begin{remark}
        If $G$ is a (connected) compact Lie group, the exponential map is surjective. However, because the associated Lie algebra $\mathfrak{g}$ is clearly noncompact, the exponential map cannot be homeomorphic and, hence, cannot be injective.
    \end{remark}

    \newformula{Baker--Campbell--Hausdorff formula}{\index{Baker--Campbell--Hausdorff formula}\label{lie:bch_formula}
        Consider the equation
        \begin{gather}
            z = \ln\bigl(\exp(x)\exp(y)\bigr)\,,
        \end{gather}
        where $x,y\in\mathfrak{g}$. The solution is given by the following formula:
        \begin{gather}
            z = x + y + \frac{1}{2}[x,y] + \frac{1}{12}[x,[x,y]] - \frac{1}{12}[y,[x,y]] + \cdots\,.
        \end{gather}
        One should note that this formula will only converge if $x,y$ are sufficiently `small'. For matrix Lie algebras, this means that
        \begin{gather}
            \|x\| + \|y\| < \frac{\ln(2)}{2}
        \end{gather}
        in the Hilbert--Schmidt norm (\cref{linalgebra:hilbert_schmidt_norm}). Due to the closure under the Lie bracket, the exponent in the BCH formula is also an element of the Lie algebra. So, this formula gives an expression for Lie group multiplication in terms of (Lie brackets of) Lie algebra elements (whenever the formula converges).
    }
    \begin{result}[Lie product formula\footnotemark]\index{Trotter expansion|see{Lie--Trotter formula}}\index{Lie--Trotter formula}\label{lie:lie_product_formula}
        \footnotetext{Also called the \textbf{Lie--Trotter formula}. Later, extensions where given by \textit{Kato} and \textit{Suzuki} for certain unbounded operators.}
        Let $\mathfrak{g}$ be a Lie algebra. The following formula applies to any $x,y\in\mathfrak{g}$:
        \begin{gather}
            e^{x+y} = \lim_{n\rightarrow\infty}\left(e^{\nicefrac{x}{n}}e^{\nicefrac{y}{n}}\right)^n\,.
        \end{gather}
    \end{result}

\subsection{Solvable Lie algebras}

    \newdef{Normalizer}{\index{normalizer}
        The normalizer of a subset of a Lie algebra $S\subset\mathfrak{g}$ is the space of elements $x\in\mathfrak{g}$ that satisfy $[x,S]\subseteq S$.
    }
    \newdef{Centralizer}{\index{centralizer}
        The centralizer of a subset of a Lie algebra $S\subset\mathfrak{g}$ is the space of elements $x\in\mathfrak{g}$ that satisfy $[x,S]=0$.
    }

    \newdef{Derived algebra}{\index{derived!algebra}
        Let $\mathfrak{g}$ be a Lie algebra. The derived Lie algebra is defined as follows:
        \begin{gather}
            [\mathfrak{g},\mathfrak{g}] := \bigl\{[x,y]\bigm\vert x,y\in\mathfrak{g}\bigr\}\,.
        \end{gather}
    }
    \newdef{Solvable Lie algebra}{\index{solvable}\label{lie:solvable}
        Consider the sequence of derived subalgebras
        \begin{gather}
            \mathfrak{g}\geq[\mathfrak{g},\mathfrak{g}] \geq \bigl[[\mathfrak{g},\mathfrak{g}],[\mathfrak{g},\mathfrak{g}]\bigr]\geq\cdots\,.
        \end{gather}
        If this sequence ends in the zero space, the Lie algebra $\mathfrak{g}$ is said to be solvable.
    }
    \remark{In general, one can define the derived series for any ideal and, accordingly, define solvability for ideals.}

    \newdef{Radical}{\index{radical}
        The largest solvable ideal of a Lie algebra.
    }

\subsection{Simple Lie algebras}

    \newdef{Direct sum}{\index{direct sum}
        The direct sum $\mathfrak{g}\oplus\mathfrak{h}$ of two Lie algebras $\mathfrak{g},\mathfrak{h}$ is defined as the direct sum (\cref{linalgebra:direct_sum}) of the underlying vector spaces where the Lie bracket is extended using the relation
        \begin{gather}
            [x,y] = 0
        \end{gather}
        for all $x\in\mathfrak{g}$ and $y\in\mathfrak{h}$.
    }

    \newdef{Semidirect product}{\index{semi-!direct product}
        The semidirect product (or sum) $\mathfrak{g}\ltimes\mathfrak{h}$ of two Lie algebras $\mathfrak{g},\mathfrak{h}$ with respect to a Lie algebra morphism $\rho:\mathfrak{g}\rightarrow\mathrm{Der}(\mathfrak{h})$ is defined as the direct sum of the underlying vector spaces where the Lie bracket is extended using the relation
        \begin{gather}
            [g,h] := \rho(g)(h)
        \end{gather}
        for all $g\in\mathfrak{g},h\in\mathfrak{h}$. This also turns $\mathfrak{h}$ into an ideal.
    }

    \newdef{Simple Lie algebra}{\index{simple!Lie algebra}
        A Lie algebra is said to be simple if it is non-Abelian and if it has no nontrivial ideals.
    }
    \newdef{Semisimple Lie algebra}{
        A Lie algebra is said to be semisimple if it is the direct sum of simple Lie algebras.
    }

    \begin{theorem}[Levi decomposition]\index{Levi!decomposition}
        Let $\mathfrak{g}$ be a finite-dimensional Lie algebra. It can be decomposed as follows:
        \begin{gather}
            \mathfrak{g} = \mathfrak{R}\ltimes(\mathfrak{L}_1\oplus\cdots\oplus\mathfrak{L}_n)\,,
        \end{gather}
        where $\mathfrak{R}$ is the radical of $\mathfrak{g}$ and the algebras $\mathfrak{L}_i$ are simple subalgebras.
    \end{theorem}
    \newdef{Levi factor}{\index{Levi!subalgebra}\index{factor|seealso{Levi subalgebra}}
        The semisimple subalgebra $\mathfrak{L}_1\oplus\cdots\oplus\mathfrak{L}_n$ in the Levi decomposition of $\mathfrak{g}$ is called the \textbf{Levi subalgebra} or \textbf{Levi factor} of $\mathfrak{g}$.
    }

\subsection{Central extensions}\label{section:central_extension_algebra}

    \newdef{Central extension}{\index{central extension!Lie algebra}
        A central extension of a Lie algebra $\mathfrak{g}$ by an Abelian Lie algebra $\mathfrak{a}$ is an exact sequence (\cref{homalg:short_exact_sequence}) of Lie algebras of the form
        \begin{gather}
            0\longrightarrow\mathfrak{a}\longrightarrow\mathfrak{h}\longrightarrow\mathfrak{g}\longrightarrow0\,,
        \end{gather}
        where the image of $\mathfrak{a}$ lies in the center of $\mathfrak{h}$.
    }

    \begin{construct}[Extension by cocycles]\index{cocycle!Lie algebra}\label{lie:cocycle}
        Consider a Lie algebra morphism $\Theta:\mathfrak{g}\times\mathfrak{g}\rightarrow\mathfrak{a}$ with the following properties:
        \begin{enumerate}
            \item $\Theta$ is bilinear,
            \item $\Theta$ is antisymmetric, and
            \item $\Theta\bigl([u,v],w\bigr) + \Theta\bigl([v,w],u\bigr) + \Theta\bigl([w,u],v\bigr) = 0$ for all $u,v,w\in\mathfrak{g}$.
        \end{enumerate}
        Such a morphism is called a \textbf{Lie algebra 2-cocycle} (see \cref{section:lie_algebra_cohomology} further below for more information). Now, every 2-cocycle $\Theta:\mathfrak{g}\times\mathfrak{g}\rightarrow\mathfrak{a}$ induces a central extension of $\mathfrak{g}$ by $\mathfrak{a}$ in the following way. Because the exact sequences characterizing central extensions (of Lie algebras) are, in particular, short exact sequences of vector spaces, they always split and, hence, one can always choose $\mathfrak{h}=\mathfrak{g}\oplus\mathfrak{a}$ to be the underlying vector space. The Lie bracket on this space is then defined as follows:
        \begin{gather}
            [v\oplus\lambda,w\oplus\mu] := [v,w]_{\mathfrak{g}}\oplus\Theta(v,w)\,.
        \end{gather}
    \end{construct}

\section{Representation theory}
\subsection{Lie groups}

    \newdef{Representation of Lie groups}{\index{representation!of a Lie group}
        A representation of a Lie group $G$ on a vector space $V$ is a Lie group morphism $\rho:G\rightarrow\GL(V)$.
    }

    \begin{example}[Adjoint representation of Lie groups]\index{adjoint!representation}\label{lie:adjoint_representation}
        Let $G$ be a Lie group and consider the adjoint action
        \begin{gather}
            \mathrm{Ad}_g:h\mapsto ghg^{-1}\,.
        \end{gather}
        The adjoint representation of $G$ on $\mathfrak{g}$ is defined as the differential $\mathrm{Ad}_{g,*}$ at the identity element. For matrix Lie groups, this becomes
        \begin{gather}
            \mathrm{Ad}_g:T_eG\rightarrow T_eG:x\mapsto gxg^{-1}\,.
        \end{gather}
    \end{example}

\subsection{Lie algebras}

    \newdef{Representation of Lie algebras}{\index{representation!of a Lie algebra}
        A representation of a Lie algebra $\mathfrak{g}$ on a vector space $V$ is a Lie algebra morphism $\rho:\mathfrak{g}\rightarrow\End(V)$.
    }

    The morphism induced by $\mathrm{Ad}:G\rightarrow H$ is precisely $\mathrm{ad}:\mathfrak{g}\rightarrow\mathfrak{h}$. Informally, one can thus say that the infinitesimal version of the similarity transformation is given by the commutator.
    \begin{example}[Adjoint representation of Lie algebras]
        Using the fact that the adjoint representation of Lie groups is smooth, one obtains:
        \begin{gather}
           \mathrm{ad}_x = \mathrm{Ad}_{g,*}\,,
        \end{gather}
        where $g=e^{tx}$. More explicitly, the adjoint map is given by
        \begin{gather}
            \label{lie:bracket_as_adjoint_rep}
            \mathrm{ad}_x(y) = [x,y]\,,
        \end{gather}
        for all $x,y\in\mathfrak{g}$.
    \end{example}
    \begin{property}[Faithfulness]
        The adjoint representation $\mathrm{ad}_x$ is faithful for all $x\in\mathfrak{g}$.
    \end{property}

    \begin{property}[Jacobi identity]
        Given the antisymmetry of the Lie bracket, the Jacobi identity is equivalent to $\mathrm{ad}:\mathfrak{g}\rightarrow\End(\mathfrak{g})$ being a Lie algebra morphism, i.e.
        \begin{gather}
            \mathrm{ad}_{[x,y]} = [\mathrm{ad}_x,\mathrm{ad}_y]\,.
        \end{gather}
    \end{property}

    \begin{formula}[Structure constants]\label{lie:ad_structure_constant}
        Let $\{e_i\}_{i\leq n}$ be a basis of the Lie algebra $\mathfrak{g}$. The structure constants are related to the adjoint representation as follows:
        \begin{gather}
            (\mathrm{ad}_{e_i})^j_k = c_{ik}^{\ \ j}\,.
        \end{gather}
    \end{formula}

    \begin{result}[Commutator]\index{commutator}
        For the algebra of the general linear group $\GL_n$, the Lie bracket is given by the commutator:
        \begin{gather}
            [A,B] = AB-BA\,.
        \end{gather}
    \end{result}

\subsection{Coadjoint orbits}\index{co-!adjoint}\label{section:coadjoint_orbits}

    \newdef{Coadjoint representation}{\label{lie:coadjoint_representation}
        The representation of a Lie group $G$ on the dual space $\mathfrak{g}^*$ defined by
        \begin{gather}
            \langle\mathrm{Ad}_g^*(\alpha),v\rangle := \langle\alpha,\mathrm{Ad}_g^{-1}(v)\rangle\,.
        \end{gather}
        Infinitesimally, this induces a representation of $\mathfrak{g}$ on its linear dual:
        \begin{gather}
            \mathrm{ad}^*_x:\alpha\mapsto-\alpha\circ\mathrm{ad}_x\,.
        \end{gather}
    }

    \newdef{Coadjoint orbit}{
        Given an element $\alpha\in\mathfrak{g}^*$, the coadjoint orbit $\Omega_\alpha$ is defined as the orbit of $\alpha$ under the action of $G$. This orbit can also be defined as the homogeneous space $G/G_\alpha$.
    }

    The following important construction shows that every coadjoint orbit is in fact canonically a \textit{symplectic manifold} (see \cref{chapter:symplectic}).
    \newdef{Kirillov--Kostant--Souriau form}{\index{Kirillov--Kostant--Souriau form}
        Consider a coadjoint orbit $\Omega_\alpha$. Because $\alpha\in\mathfrak{g}^*$ is an element of the coadjoint representation of $G$, the tangent vectors to $\Omega_\alpha$ (at $\alpha$) are elements of the induced representation of $\mathfrak{g}$, i.e.~for any tangent vector $v$ one can write $v=\mathrm{ad}^*_x(\alpha)$ for some $x\in\mathfrak{g}$. Now, a symplectic form on $\Omega_\alpha$ is defined as follows:
        \begin{gather}
            \omega_\alpha\bigl(\mathrm{ad}^*_x(\alpha),\mathrm{ad}^*_y(\alpha)\bigr) = \langle\alpha,[x,y]\rangle\,.
        \end{gather}
    }

    \todo{COMPLETE (perhaps move to chapter on symplectic geometry)}

\section{Structure}
\subsection{Killing form}

    \newdef{Killing form}{\index{Killing!form}\index{Cartan--Killing|see{Killing form}}\label{lie:killing_form}
        Let $\mathfrak{g}$ be a finite-dimensional Lie algebra over the field $k$. The Killing form (also called the \textbf{Cartan--Killing} form) on $\mathfrak{g}$ is defined as the following symmetric bilinear form:
        \begin{gather}
            K:\mathfrak{g}\otimes\mathfrak{g}\rightarrow k:x\otimes y\mapsto\tr(\mathrm{ad}_x\circ\mathrm{ad}_y)\,.
        \end{gather}
        The trace can be calculated by choosing a (finite-dimensional) representation of the Lie algebra using Ado's theorem~\ref{lie:ado}.

        From \cref{lie:ad_structure_constant}, one can derive the value of the Killing form on the basis $\{e_i\}_{i\leq n}$:
        \begin{gather}
            K_{ij} = c_{ik}^{\ \ l}c_{jl}^{\ \ k}\,,
        \end{gather}
        where $c_{ij}^{\ \ k}$ are the structure constants.
    }

    \begin{theorem}[Cartan's criterion]\index{Cartan!criterion}
        A Lie algebra is semisimple if and only if its Killing form is nondegenerate.
    \end{theorem}

    \begin{property}
        If a Lie group $G$ is compact, the Killing form of its Lie algebra $\mathfrak{g}$ is negative-definite.
    \end{property}
    \begin{result}
        Let $G$ be a compact Lie group. If its Lie algebra is semisimple, the Killing form $K$ induces a metric
        \begin{gather}
            g:x\otimes y\mapsto-\tr(\mathrm{ad}_x\circ\mathrm{ad}_y) = -K(x,y)
        \end{gather}
        which turns the corresponding Lie group $G$ into a \textit{Riemannian manifold} (see \cref{chapter:riemann}).
    \end{result}

    \begin{property}[Killing form is invariant]
        The Killing-form is $\mathrm{Ad}$-invariant:
        \begin{gather}
            K\bigl(\mathrm{Ad}_g(x),\mathrm{Ad}_g(y)\bigr) = K(x,y)
        \end{gather}
        for all $g\in G$. More generally, the Killing form is invariant under all automorphisms of $\mathfrak{g}$.
    \end{property}
    \begin{result}\label{lie:ad_killing_form}
        The adjoint map $\mathrm{ad}_z$ is skewsymmetric with respect to the Killing form:
        \begin{gather}
            K(\mathrm{ad}_zx,y) = -K(x,\mathrm{ad}_zy)\,.
        \end{gather}
    \end{result}

    \begin{property}[Invariant forms]\label{lie:killing_trace}
        For a simple Lie algebra, every ($\mathrm{ad}$-)invariant symmetric bilinear form is a scalar multiple of the Killing form.
    \end{property}
    \begin{example}
        For $\mathfrak{su}(n)$, the trace can easily be seen to be $\mathrm{ad}$-invariant and, hence, satisfy the above property. The exact relation is given by
        \begin{gather}
            \tr(AB) = 2nK(A,B)\,.
        \end{gather}
    \end{example}

    \begin{property}[Antisymmetric structure constants]\index{structure!constants}
        When the Lie algebra $\mathfrak{g}$ is compact and semisimple, i.e.~when the Killing form induces a metric, one can find a basis of $\mathfrak{g}$, constructed by orthonormalizing a given basis with respect to the Killing metric, such that the structure constants are invariant under cyclic permutation of the indices:
        \begin{gather}
            c_{ijk} = c_{jki}\,.
        \end{gather}
        A corollary of this property is also that the structure constants become totally antisymmetric.
    \end{property}

    \begin{construct}[Induced Killing form]\label{lie:rho_killing_form}
        Let $\mathfrak{g}$ be a Lie algebra and let $V$ be a vector space equipped with a Lie algebra representation $\rho:\mathfrak{g}\rightarrow\End(V)$. One can define a Killing form associated with $\rho$ in the following way:
        \begin{gather}
            K_\rho(x,y) := \tr\bigl(\rho(x)\circ\rho(y)\bigr)\,.
        \end{gather}
        This is a generalization of \cref{lie:killing_form} which reduces to the Killing form $K$ when choosing $V$ to be $\mathfrak{g}$ in the adjoint representation.
    \end{construct}

\subsection{Weights, roots and Dynkin diagrams}\label{section:weights_roots}

    From here on, the base field is assumed to be algebraically closed (take $\mathbb{C}$ for simplicity).

    \newdef{Cartan subalgebra}{\index{Cartan!subalgebra}
        Let $\mathfrak{g}$ be a Lie algebra. A subalgebra $\mathfrak{h}$ is called a Cartan subalgebra if it satisfies the following two conditions:
        \begin{enumerate}
            \item\textbf{Nilpotency}: Its lower central series terminates:
                \begin{gather}
                    \exists n\in\mathbb{N}:\underbrace{[\mathfrak{h},[\mathfrak{h},[\mathfrak{h},\ldots]]]}_{n\text{ times}} = 0\,.
                \end{gather}
            \item\textbf{Self-normalizing}:
                \begin{gather}
                    \forall x\in\mathfrak{h}:[x,y]\in\mathfrak{h}\implies y\in\mathfrak{h}\,.
                \end{gather}
        \end{enumerate}
    }

    From here on, it will also be assumed that all Lie algebras are finite-dimensional. This assumption is motivated by the following property.
    \begin{property}
        Every finite-dimensional Lie algebra contains a Cartan subalgebra.
    \end{property}
    \begin{property}
       If $\mathfrak{g}$ is semisimple, its Cartan subalgebra is Abelian.
    \end{property}

    \begin{construct}
        Let $\mathfrak{g}$ be a semisimple Lie algebra. A Cartan subalgebra $\mathfrak{h}$ can be constructed as follows. For some integer $k\leq\dim(\mathfrak{g})$, choose $k$ linearly independent, commuting vectors $\{h_i\}_{i\leq k}\subset\mathfrak{g}$. (The existence of such a choice, which is equivalent to requiring simultaneous diagonalization, is only guaranteed for semisimple Lie algebras.) If this set can be extended to a basis $\{h_i\}_{i\leq k}\cup\{g_j\}_{j\leq \dim(\mathfrak{g})-k}$ of $\mathfrak{g}$ such that every $g_j$ is a nontrivial eigenvector of the adjoint map $\mathrm{ad}_{h_i}$ for all $i\in I$, the algebra $\mathfrak{h} = \mathrm{span}\{h_i\}_{i\leq k}$ is a Cartan subalgebra.
    \end{construct}

    \newdef{Weight space}{\index{weight}\index{module!weight}\label{lie:weight_space}
        Let $V$ be a representation of a Lie algebra $\mathfrak{g}$ with Cartan subalgebra $\mathfrak{h}$. For every linear functional $\lambda$ on $\mathfrak{h}$, the weight space $V_\lambda$ with \textbf{weight} $\lambda$ is defined as follows:
        \begin{gather}
            V_\lambda := \{v\in V\mid\forall h\in\mathfrak{h}:h\cdot v=\lambda(H)v\}\,.
        \end{gather}
        The nonzero elements of a weight space are called \textbf{weight vectors}. If the representation $V$ can be decomposed as a direct sum of weight spaces, it is called a \textbf{weight module}:
        \begin{gather}
            V = \bigoplus_{\lambda\in\mathfrak{h}^*}V_\lambda\,.
        \end{gather}
    }

    In the case where $V$ is the adjoint representation, the (nonzero) weights are called \textbf{roots}.
    \newdef{Root}{\index{root}
        Let $\mathfrak{g}$ be a Lie algebra with Cartan subalgebra $\mathfrak{h}$. From the definition of a Cartan subalgebra, it follows that for all $h\in\mathfrak{h}$:
        \begin{gather}
            [h,g_j]=\alpha_j(h)g_j\,,
        \end{gather}
        where $\{g_j\}_{j\in J}$ is the basis extension of $\mathfrak{g}$ with respect to $\mathfrak{h}$. The eigenvalues have been written suggestively as maps $\alpha_j(h)$ acting on $\mathfrak{h}$, which is justified because, due to the definition of the $g_j$'s and the structure of the above formula, the $\alpha_j$'s for different $h$ piece together to give linear maps. By comparing the formula to the definition of weights above, it can be seen that the $\alpha_j$'s are the weights of the adjoint representation. The nonzero weights are called the roots of $\mathfrak{g}$ and form the so-called \textbf{root system} $\Phi$.

        It follows that there exists a weight space decomposition of $\mathfrak{g}$:
        \begin{gather}
            \mathfrak{g} = \mathfrak{h} \oplus \bigoplus_{\lambda\in\Phi}\mathfrak{g}_\lambda\,,
        \end{gather}
        where the one-dimensional spaces $\mathfrak{g}_\lambda$ are the weight spaces associated to the roots $\lambda$ ($\mathfrak{h}$ is equal to $\mathfrak{g}_0$ in this notation).
    }

    \begin{property}
        If $\alpha\in\Phi$, then $-\alpha\in\Phi$. Furthermore, if $\alpha\in\Phi$ and $c\alpha\in\Phi$, then $c=\pm1$.
    \end{property}
    This property says that the root system $\Phi$ is not linearly independent. To introduce some kind of basis, the following notion is introduced.
    \newdef{Simple root}{
        The set of simple roots $\Delta$ is a linearly independent subset of $\Phi$ such that every element $\lambda\in\Phi$ can be written as
        \begin{gather}
            \lambda = \pm\sum_i^na_i\lambda_i\,,
        \end{gather}
        where $a_i\in\mathbb{N}$ and $\lambda_i\in\Delta$. (Such a set always exists.) This definition enforces the expansion coefficients $a_i$ of a root $\lambda$ to be either all positive or all negative.
    }

    More generally, one can define the following equivalence relation on a root system.
    \newdef{Positive roots}{
        Let $\Phi$ be the root system of a given Lie algebra $\mathfrak{g}$. Because the only scalar multiples of a root $\lambda\in\Phi$ in the root system are $\pm\lambda$, one can define a set of positive roots $\Phi^+$ as follows:
        \begin{itemize}
            \item $\lambda\in\Phi^+\implies-\lambda\not\in\Phi^+$, and
            \item $\alpha, \beta\in\Phi^+\land\alpha+\beta\in\Phi\implies\alpha+\beta\in\Phi^+$.
        \end{itemize}
        The simple roots are then exactly the elements in $\Phi^+$ that cannot be written as a sum of other elements in $\Phi^+$.
    }

    \newdef{Triangular decomposition}{\index{Borel!subalgebra}\index{triangular decomposition}
        Given a choice of positive roots $\Phi^+$, one can decompose the Lie algebra $\mathfrak{g}$ as follows:
        \begin{gather}
            \mathfrak{g} = \mathfrak{n}_-\oplus\mathfrak{h}\oplus\mathfrak{n}_+\,,
        \end{gather}
        where $\mathfrak{n}_\pm=\bigoplus_{\alpha\in\Phi^\pm}\mathfrak{g}_\alpha$. The subalgebra $\mathfrak{h}\oplus\mathfrak{n}_+$ is called the \textbf{Borel subalgebra}. It is the maximal solvable subalgebra of $\mathfrak{g}$ (\cref{lie:solvable}).
    }

    \begin{property}[Rank]\index{rank!of a Lie algebra}
        Let $\mathfrak{h}$ be a Cartan subalgebra. The set of simple roots $\Delta$ forms a basis for the dual space $\mathfrak{h}^*$ (over $\mathbb{C}$). It follows that the cardinality of $\Delta$ is equal to the dimension of the Cartan subalgebra. This dimension is called the \textbf{rank} of the Lie algebra.
    \end{property}

    \newdef{Weyl group}{\index{Weyl!group}\label{lie:weyl_group}
        For every simple root $\lambda$, one can construct a Householder transformation (\cref{linalgebra:householder_transformation}) as follows:
        \begin{gather}
            \sigma_\lambda:\mathrm{span}_{\mathbb{R}}(\Delta)\rightarrow\mathrm{span}_{\mathbb{R}}(\Delta):\mu\mapsto\mu-2\frac{\langle\mu\mid\lambda\rangle}{\langle\lambda\mid\lambda\rangle}\lambda\,,
        \end{gather}
        where the inner product $\langle\cdot\mid\cdot\rangle$ is the dual Killing form. The Weyl group $W$ is defined as the group generated by these transformations.
    }
    \begin{property}[Weyl group symmetries]
        Every root $\alpha\in\Phi$ can be written as $\alpha = \sigma(\mu)$ for some $\mu\in\Delta$ and $\sigma\in W$. Furthermore, the root system $\Phi$ is closed under the action of $W$. In particular, it can be shown that the Weyl group $W$ is precisely the symmetry group of the root system $\Phi$ and the isometry group of the Killing form (and its dual).
    \end{property}

    \newdef{Coroot}{\index{co-!root}
        Consider the \textit{sharp map} (see \cref{riemann:sharp_map}), where the metric $g$ is given by the Killing form $K$. The dual Killing form $K^*$ is then a proper inner product (when restricted to the real span of $\Delta$) defined as
        \begin{gather}
            K^*(\cdot,\cdot) = K(\cdot^\sharp,\cdot^\sharp)\,.
        \end{gather}
        The restriction of the dual Killing form to the real span $\mathfrak{h}_0^*$ of the roots of $\mathfrak{g}$ induces a dual space $\mathfrak{h}_0\subset\mathfrak{h}$. The coroot $\alpha^\vee\in\mathfrak{h}_0$ associated to a root $\alpha$ is then defined by the following formula (where the metric isomorphism $\mathfrak{h}_0\cong\mathfrak{h}_0^{**}$ is used):
        \begin{gather}
            \alpha^\vee := 2\frac{\langle\alpha\mid\cdot\rangle}{\langle\alpha\mid\alpha\rangle}\equiv 2\frac{\alpha}{\langle\alpha\mid\alpha\rangle}\,.
        \end{gather}
        With this definition the Weyl transformations can be rewritten as follows:
        \begin{gather}
            \sigma_\lambda:\mathrm{span}_{\mathbb{R}}(\Delta)\rightarrow\mathrm{span}_{\mathbb{R}}(\Delta):\mu\mapsto\mu - \mu(\lambda^\vee)\lambda\,.
        \end{gather}
    }
    \begin{notation}[Coroot]
        Sometimes it is more favourable to denote the coroot associated to $\alpha$ by $H^\alpha$. This convention will be adopted in the remainder of this chapter.
    \end{notation}

       \newdef{Weyl chamber}{\index{Weyl!chamber}\index{weight!dominant}
        Given a choice of positive roots $\Phi^+$, the closed (fundamental) Weyl chamber associated to this ordering is defined as the subset of $\mathfrak{h}_0$ that contains the elements $w$ satisfying the following equation for all $\gamma\in\Phi^+$:
        \begin{gather}
            w(H^\gamma)\geq0\,.
        \end{gather}
        Elements of a Weyl chamber are called \textbf{dominant weights}.
    }
    \begin{property}[Weyl group]
        The Weyl group acts transitively on the set of Weyl chambers and, accordingly, on the orderings of the root system.
    \end{property}

    \begin{property}
        Let $\alpha\in\Phi$ be a root. Choose a generating element $E^\alpha$ of the weight space $\mathfrak{g}_\alpha$ associated to $\alpha$ and let $F^\alpha$ be the generator of the weight space $\mathfrak{g}_{-\alpha}$ such that $\mathrm{span}\{E^\alpha,F^\alpha,[E^\alpha,F^\alpha]\}$ is a one-dimensional simple Lie algebra. The following relations hold (for $\beta\neq\pm\alpha$):
        \begin{itemize}
            \item $\beta(H^\alpha) = 2\frac{\langle\alpha|\beta\rangle}{\langle\alpha|\alpha\rangle}\in\mathbb{Z}$,
            \item $[H^\alpha,E^\alpha] = \alpha(H^\alpha)E^\alpha = 2E^\alpha$, and
            \item $[H^\alpha,F^\alpha] = -\alpha(H^\alpha)F^\alpha = -2F^\alpha$.
        \end{itemize}
    \end{property}

    \newdef{Cartan matrix}{\index{Cartan!matrix}
        Let $\lambda_i,\lambda_j\in\Delta$ be simple roots. Because the Weyl group is the symmetry group of the root system, \[\sigma_{\lambda_i}(\lambda_j) = \lambda_j - 2\frac{\langle\lambda_i|\lambda_j\rangle}{\langle\lambda_i\mid\lambda_i\rangle}\lambda_i\] is a root. From the properties above, it then follows that the quantity
        \begin{gather}
            C_{ij} := 2\frac{\langle\lambda_i\mid\lambda_j\rangle}{\langle\lambda_i\mid\lambda_i\rangle} = \lambda_j(H^{\lambda_i})
        \end{gather}
        is an integer. The matrix formed by these numbers is called the Cartan matrix.
    }
    \begin{property}[Properties of Cartan matrix]\label{lie:cartan_prop}
        The Cartan matrix $C$ satisfies the following properties:
        \begin{itemize}
            \item $C_{ii}=2$,
            \item $C_{ij}\in\mathbb{Z}_{\leq0}$ if $i\neq j$, and
            \item $C_{ij}=0\iff C_{ji}=0$.
        \end{itemize}
        This last property does not imply that the Cartan matrix is symmetric. The fact that it is not symmetric can immediately be seen from its definition. However,
        \begin{itemize}
            \item it is \textit{symmetrizable}, i.e.~there exist a positive diagonal matrix $D$ and a symmetric matrix $S$ such that $C=DS$.
            \item it is positive definite.
        \end{itemize}
    \end{property}

    \newdef{Bond number}{\index{bond!number}
        For all indices $i\neq j$, the bond number $n_{ij}$ is defined as follows:
        \begin{gather}
            n_{ij} := C_{ij}C_{ji}\,.
        \end{gather}
        Using the definition of the coefficients $C_{ij}$, it can be seen that $n_{ij}$ is an integer equal to $4\cos^2\sphericalangle(\lambda_i,\lambda_j)$. This implies that $n_{ij}$ can only take on the values $0,1,2,3$. The value 4 would only be possible if the angle between $\lambda_i$ and $\lambda_j$ is 0, but this can only occur in the case where $i=j$ (which was excluded by definition).
    }
    \begin{remark}
        In the case of $n_{ij} = 2$ or $n_{ij} = 3$, two possibilities arise: $C_{ij}>C_{ji}$ or $C_{ij}<C_{ji}$ for $i<j$. From the definition of the Cartan integers and the symmetry of the dual Killing form, these cases correspond to $\langle\lambda_i\mid\lambda_i\rangle<\langle\lambda_j\mid\lambda_j\rangle$ and $\langle\lambda_i\mid\lambda_i\rangle>\langle\lambda_j\mid\lambda_j\rangle$.
    \end{remark}

    \begin{construct}[Dynkin diagram]\index{Dynkin diagram}\label{lie:construct_dynkin}
        For a semisimple Lie algebra $\mathfrak{g}$ with simple roots $\Delta$, one can draw a diagram using the following rules:
        \begin{enumerate}
            \item For every simple root $\lambda\in\Delta$, draw a circle $\bigcirc$.
            \item Draw $n_{ij}$ lines between the circles associated to $\lambda_i$ and $\lambda_j$.
            \item If $n_{ij}=2$ or 3, add a $<$ or $>$ sign based on the relation between their lengths (see previous remark).
        \end{enumerate}
    \end{construct}

    \begin{property}[Classification]\index{exceptional!lie algebra}
        The Dynkin diagrams can be classified as follows (for every type, the first three examples are given):
        \begin{itemize}
            \item $A_n$: \begin{center}\dynk \dynkA{2} \dynkA{3} \end{center}
        \item $B_n,n\geq2$: \begin{center}\dynkB{2} \dynkB{3} \dynkB{4} \end{center}
        \item $C_n,n\geq2$: \begin{center}\dynkC{2} \dynkC{3} \dynkC{4} \end{center}
        \item $D_n,n\geq4$: \begin{center}\dynkD{4} \dynkD{5} \dynkD{6} \end{center}
        \end{itemize}
        These are the only possible diagrams for simple Lie algebras with exception of $E_6,E_7,E_8,F_4$ and $G_2$, the so-called \textit{exceptional Lie algebras}.
    \end{property}

    \begin{example}[Special linear group]
        By looking at the Lie brackets in \cref{lie:sl2c_lie_brackets}, one can see that the one-element set $\{X_1\}$ forms a Cartan subalgebra of $\mathfrak{sl}(2,\mathbb{C})$. From the same equation, it is also immediately clear that the simple root set $\Delta$ is given by the one-element set $\{\lambda\in\mathfrak{sl}^*(2,\mathbb{C})\mid\lambda(X_1)\mapsto 2\}$. Hence, the Dynkin diagram for $\mathfrak{sl}(2,\mathbb{C})$ is $A_1$.
    \end{example}

    \begin{theorem}[Cartan--Killing]\index{Cartan--Killing}
        Every finite-dimensional simple Lie algebra over $\mathbb{C}$ can be reconstructed from its set of simple roots $\Delta$.
    \end{theorem}
    \begin{construct}[Chevalley--Serre]\index{Chevalley--Serre!relations}\label{lie:reconstruction}
        Given a Dynkin diagram of a simple Lie algebra, one can reconstruct the original Lie algebra $\mathfrak{g}$ over $\mathbb{C}$ up to isomorphism.

        The number of nodes is equal to the number of simple roots and, hence, determines the rank $n$ of $\mathfrak{g}$. First, construct the free Lie algebra on $3n$ generators $\{E_i,F_i,H_i\}_{i\leq n}$. The Cartan subalgebra $\mathfrak{h}\leq\mathfrak{g}$ is constructed from the generators $H_i$ by imposing the following relations:
        \begin{itemize}
            \item $[H_i,H_j] = 0$,
            \item $[H_i,E_j] = a_{ij}E_j$,
            \item $[H_i,F_j] = -a_{ij}F_j$, and
            \item $[E_i,F_j] = \delta_{ij}H_j$,
        \end{itemize}
        where the numbers $(a_{ij})$ form the Cartan matrix obtained by reversing \cref{lie:construct_dynkin}. To complete the reconstruction, one imposes the following additional relations:
        \begin{itemize}
            \item $\mathrm{ad}_{E_i}^{|a_{ij}|+1}(E_j) = 0$, and
            \item $\mathrm{ad}_{F_i}^{|a_{ij}|+1}(F_j) = 0$.
        \end{itemize}
        The first set of relations are called the \textbf{Chevalley relations} and the last two are called the \textbf{Serre relations}. The full construction is called the \textbf{Chevalley--Serre presentation}.
    \end{construct}
    \remark{For composite diagrams that correspond to semisimple Lie algebras, one first has to construct the Lie algebras corresponding to every simple diagram and then take the direct sum.}

    \begin{property}[$\mathfrak{sl}_2$]
        For every Cartan matrix of rank $n\in\mathbb{N}$, the triplets $\{E_i,F_i,H_i\}$ with $i\leq n$ generate $\mathfrak{sl}_2$-algebras. All (semi)simple Lie algebras are thus in some sense built up from $\mathfrak{sl}_2$-algebras, similar to how in algebraic topology simplicial complexes are constructed by gluing simplices together.
    \end{property}

\subsection{Highest weight theory}

    \newdef{Algebraically integral}{\index{weight!lattice}
        An element $H\in\mathfrak{h}_0$ is said to be algebraically integral if its value on every root is an integer. The set of all algebraically integral elements is called the \textbf{weight lattice}.
    }
    \newdef{Fundamental weight}{\index{weight!fundamental}
        Let $\Delta=\{\alpha_i\}_{i\leq\rk(\mathfrak{g})}$ be the set of simple roots. The fundamental weights $\{\omega_i\}_{i\leq\rk(\mathfrak{g})}$ are defined as the elements of $\mathfrak{h}_0^*$ for which the following formula is satisfied for all $i,j\leq|\Delta|$:
        \begin{gather}
            \omega_i(H^{\alpha_j})=\delta_{ij}\,.
        \end{gather}
        This implies that an element $\lambda\in\mathfrak{h}_0$ is algebraically integral if it is an integral combination of fundamental weights.
    }

    \newdef{Ordering of weights}{
        Let $\Phi^+$ be a choice of positive roots. One can define a partial ordering on the set of weights $\mathfrak{h}^*_0$ in the following way:
        \begin{gather}
            \mu\leq\lambda\iff\lambda-\mu\in\mathrm{span}_{\mathbb{N}}\left(\Phi^+\right)\,.
        \end{gather}
    }

    \newdef{Highest weight vector}{
        Consider a representation $V$ of a Lie algebra $\mathfrak{g}$. An element $v\in V$ is said to be a highest weight vector if it is a weight vector that is annihilated by all positive roots. A highest weight module is a weight module that is generated by a highest weight vector.
    }

    \begin{theorem}[Highest weight theorem]
        Consider a finite-dimensional Lie algebra. The following statements hold:
        \begin{itemize}
            \item Every finite-dimensional irreducible representation has a unique dominant integral highest weight.
            \item If two irreducible representations have the same highest weight, they are isomorphic.
            \item Every dominant integral weight is the highest weight of an irreducible finite-dimensional representation.
        \end{itemize}
    \end{theorem}

\subsection{Coxeter groups}

    The properties about root systems in \cref{section:weights_roots} can be abstracted into the following definition.
    \newdef{Root system}{\index{root!system}
        Let $(E,\langle\cdot\mid\cdot\rangle)$ denote a finite-dimensional vector space equipped with a positive-definite, symmetric bilinear form. A root system on $E$ is a finite set $\Phi$ of nonzero vectors satisfying the following conditions:
        \begin{enumerate}
            \item\textbf{Basis}: $E=\mathrm{span}_{\mathbb{R}}(\Phi)$.
            \item\textbf{Scalar closure}: $\alpha\in\Phi\land\lambda\alpha\in\Phi\implies\lambda=\pm1$.
            \item\textbf{Reflection closure}: $\sigma_\alpha\Phi=\Phi$, where $\sigma_\alpha$ is the Householder transformation induced by $\alpha\in\Phi$.
            \item\textbf{Integrality}: $\alpha,\beta\in\Phi\implies2\frac{\langle\alpha\mid\beta\rangle}{\langle\alpha\mid\alpha\rangle}\in\mathbb{Z}$.
        \end{enumerate}
        Many of the other constructions, such as the definition of positive and simple roots, also carry over from the case of Lie algebras.
    }
    \begin{remark}[Lie algebras]
        The root system of a Lie algebra $\mathfrak{g}$ relative to a Cartan subalgebra $\mathfrak{h}$ is actually a root system for $\mathfrak{h}^*$ in the sense of the foregoing definition.
    \end{remark}

    \begin{property}[Finite reflections]
        Every finite reflection group, i.e.~a finite group generated by Householder transformations, admits a root system. Conversely, every root system generates a finite reflection group given by its Weyl group (\cref{lie:weyl_group}).
    \end{property}

    The Weyl group induced by a root system is a specific instance of the following notion.
    \newdef{Coxeter group}{\index{Coxeter group}\label{lie:coxeter_group}
        A group admitting a presentation of the form
        \begin{gather}
            \langle r_1,\ldots,r_n\mid(r_ir_j)^{m_{ij}}=e \rangle\,,
        \end{gather}
        where $m_{ii}=1$ and $m_{ij}\geq2$ for $i\neq j$. If $m_{ij}=\infty$, no relation is imposed.
    }

\subsection{\difficult{Kac--Moody algebras}}

    \begin{construct}[Kac--Moody algebra]\index{Kac--Moody algebra}\index{Cartan!matrix}\index{realization}\label{lie:kac_moody}
        Consider the Cartan matrix $A$ associated to a finite-dimensional (semi)simple Lie algebra. This matrix has the properties listed in \cref{lie:cartan_prop}. If one drops the positivity condition, the definition of a \textbf{generalized Cartan matrix} is obtained. For such matrices, one can construct a (possibly infinite-dimensional) Lie algebra using an analogue of the Chevalley--Serre presentation.

        However, since a generalized Cartan matrix might have a vanishing determinant, one cannot use the Chevalley--Serre presentation as given in \cref{lie:reconstruction} as the constructed roots might be linearly dependent. Luckily, this problem can easily be solved. Let $A$ be an $n\times n$ (generalized) Cartan matrix. First, choose a complex vector space $\mathfrak{h}$ equipped with, for every $i\leq n$, a simple root $\alpha_i$ (resp.~coroot $H^{\alpha_i}$) in $\mathfrak{h}^*$ (resp.~$\mathfrak{h}$) such that these are linearly independent and satisfy the condition $\alpha_i(H^{\alpha_j})=A_{ij}$. (Such a \textbf{realization} always exists.) It can be shown that $\mathfrak{h}$ satisfies $\dim(\mathfrak{h})\geq2n-\rk(A)$. A minimal realization, i.e.~one that satisfies $\dim(\mathfrak{h})=2n-\rk(A)$, is unique up to isomorphism.

        Then, construct the direct sum $\mathfrak{g}$ of the free Lie algebra on $2n$ generators $\{E_i,F_i\}_{i\leq n}$ with $\mathfrak{h}$ and take the quotient by the following relations:
        \begin{itemize}
            \item $[E_i,F_j] = \delta_{ij}H^{\alpha_i}$,
            \item $[H,H']=0$ for $H,H'\in\mathfrak{h}$,
            \item $[H,E_i]=\alpha_i(H)E_i$ for $H\in\mathfrak{h}$, and
            \item $[H,F_i]=-\alpha_i(H)F_i$ for $H\in\mathfrak{h}$.
        \end{itemize}
        Given this Lie algebra, one can find the unique maximal ideal $\mathfrak{m}\leq\mathfrak{g}$ for which $\mathfrak{m}\cap\mathfrak{h}=\{0\}$. The quotient algebra $\mathfrak{g}/\mathfrak{m}$ is called the Kac--Moody algebra associated to $A$.\footnote{\textit{Kac} proved that for symmetrizable matrices this construction is equivalent to the Chevalley--Serre presentation with the generators as given here.}

        Although an analogue of the Serre relations is not included, it can be shown that these relations still hold (see for example~\citet{amini_infinite-dimensional_2014}). In fact, when $A$ is symmetrizable, the ideal $\mathfrak{m}$ is exactly generated by the Serre relations.
    \end{construct}
    \begin{remark}[Classification]
        There exist three distinct classes of Kac--Moody algebras:
        \begin{itemize}
            \item If $A$ is positive-definite, one obtains a finite-dimensional (semi)simple Lie algebra. $A$ is also said to be of \textbf{finite type}.
            \item If $A$ is positive-semidefinite, one obtains a Kac--Moody algebra of \textbf{affine type}. In fact, it can be shown for generalized Cartan matrices that affinity is equivalent to the existence of a unique (up to scaling) real vector $v$ such that $Av=0$.
            \item If $A$ is indefinite, one obtains a Kac--Moody of \textbf{indefinite type}.
        \end{itemize}
    \end{remark}

    \begin{definition}[Loop algebra]\index{loop!algebra}
        Consider a finite-dimensional Lie algebra $\mathfrak{g}$. Its associated \textbf{loop algebra} $L\mathfrak{g}$ is the vector space $\mathfrak{g}\otimes\mathbb{C}[t,t^{-1}]$ equipped with the following Lie bracket:
        \begin{gather}
            [x\otimes t^k,y\otimes t^l] := [x,y]_{\mathfrak{g}}\otimes t^{k+l}\,.
        \end{gather}
        Because this strongly resembles the definition of the ring of Laurent polynomials $\mathfrak{K}[t,t^{-1}]$ over a field $\mathfrak{K}$, the loop algebra is sometimes denoted by $\mathfrak{g}[t,t^{-1}]$.

        Equivalently, one can obtain the loop algebra as the space of polynomial maps from $S^1$ to $\mathfrak{g}$ (hence the name). Furthermore, if $G$ is the Lie group associated to $\mathfrak{g}$ and $LG$ denotes its (free) loop group (\cref{topology:loop_group}), then $LG$ has the natural structure of a (infinite-dimensional) Lie group and its Lie algebra is precisely $L\mathfrak{g}$.
    \end{definition}

    \begin{definition}[Affine Lie algebra]\index{affine!algebra}
        Given a simple Lie algebra $\mathfrak{g}$, one can define the affine Lie algebra $\widehat{\mathfrak{g}}$ as the central extension of the loop algebra $L\mathfrak{g}$ by $\mathbb{C}$ associated to the cocycle
        \begin{gather}
            \Theta:(x\otimes t^k,y\otimes t^l)\mapsto kK(x,y)\delta_{k+l,0}\,,
        \end{gather}
        where $K$ is the Killing form on $\mathfrak{g}$. The generator $c$ of $\mathbb{C}$ is often called the \textbf{central element} since it is mapped to an element in the center of $\widehat{\mathfrak{g}}$. This cocycle can also be defined using the residue (\cref{complex:residue_def}) of a Laurent polynomial:
        \begin{gather}
            \Theta(f,g) := \mathrm{Res}\left[K(f,g)\right]_t\,,
        \end{gather}
        where the Killing form is extended from $\mathfrak{g}$ to $L\mathfrak{g}$ by $K(x\otimes t^k,y\otimes t^l):=K(x,y)t^{k+l}$.

        However, to obtain a well-behaved affine Kac--Moody algebra, one also needs to extend this affine Lie algebra by a derivation. Observe that the loop algebra and, accordingly, the affine Lie algebra is $\mathbb{Z}$-graded. A well-defined derivation is then obtained through multiplication by the grading. To this end, add a formal generator $d$ together with the following relations (the central element from the previous step is still denoted by $c$):
        \begin{gather}
            \begin{aligned}
                [d,x\otimes P(t)] &= x\otimes t\deriv{}{t}P(t)\\
                [d,c] &= 0\,,
            \end{aligned}
        \end{gather}
        where $P(t)\in\mathbb{C}[t,t^{-1}]$.
    \end{definition}

    \newdef{Extended Cartan matrix}{\index{Cartan!matrix}
        Consider a simple Lie algebra $\mathfrak{g}$ with Cartan matrix $A$ and denote the associated Chevalley generators by $\{E_i,F_i,H_i\}_{i\leq n}$. There exist unique (up to scaling) nonzero elements $E_0,F_0$ such that
        \begin{gather}
            \begin{aligned}
                [E_0,F_i] &= 0\,,\\
                [F_0,E_i] &= 0
            \end{aligned}
        \end{gather}
        for all $1\leq i\leq n$. This also implies that $H_0:=[E_0,F_0]$ is a linear combination of $\{H_i\}_{i\leq n}$. The elements $E_0,F_0$ can be normalized by enforcing the following Chevalley-type relations:
        \begin{gather}
            \begin{aligned}
                [H_0,E_0] &= 2E_0\,,\\
                [H_0,F_0] &= -2F_0\,.
            \end{aligned}
        \end{gather}
        The extended Cartan matrix $\widehat{A}$ is defined by adjoining a new row and column to $A$. These new entries are defined by
        \begin{gather}
            \begin{aligned}
                [H_0,E_i] &=: a_{0i}E_i\,,\\
                [H_i,E_0] &=: a_{i0}E_0\,.
            \end{aligned}
        \end{gather}
    }

    \newdef{Twisted Kac--Moody algebra}{\index{Kac--Moody algebra}\index{twisted|see{Kac--Moody algebra}}
        Consider an indecomposable generalized Cartan matrix. Such a matrix is of affine type if and only if all its proper principal minors are positive-definite. Hence, by deleting the first row and column, one obtains the Cartan matrix $B$ for a finite-dimenionsal simple Lie algebra $\mathfrak{g}(B)$. It can be shown that the affine Kac--Moody algebra $\mathfrak{g}(A)$, as defined in \cref{lie:kac_moody}, is isomorphic to the affine Kac--Moody algebra as constructed above, starting from the simple Lie algebra $\mathfrak{g}(B)$, if and only if $A=\widehat{B}$.

        All affine Kac--Moody algebras that are isomorphic to Lie algebras defined in this way are said to be \textbf{untwisted}. All other affine Kac--Moody algebras are said to be \textbf{twisted}.
    }

    \begin{property}
        Let $\mathfrak{g}$ be a simple Lie algebra with Cartan matrix $A$. The affine Lie algebra $\widehat{\mathfrak{g}}$ is isomorphic to the derived subalgebra $[\mathfrak{g}(\widehat{A}),\mathfrak{g}(\widehat{A})]$ of the Kac--Moody algebra $\mathfrak{g}(\widehat{A})$.
    \end{property}

\subsection{Universal enveloping algebra}\label{section:universal_enveloping_algebra}

    \newdef{Universal enveloping algebra}{\index{universal!enveloping algebra}
        Let $\mathfrak{g}$ be a Lie algebra and consider its tensor algebra $T(\mathfrak{g})$. The universal enveloping algebra $U(\mathfrak{g})$ is defined as the quotient of $T(\mathfrak{g})$ by the two-sided ideal generated by $\{g\otimes h - h\otimes g - [g,h]\mid g,h\in\mathfrak{g}\}$.
    }

    \begin{construct}\label{lie:enveloping_algebra_construct}
        If the Chevalley--Serre presentation from \cref{lie:reconstruction} is regarded as a presentation for a unital associative algebra instead of as a Lie algebra presentation (by replacing the Lie bracket by the commutator constructed from the algebra multiplication), the universal enveloping algebra $U(\mathfrak{g})$ of $\mathfrak{g}$ is obtained.
    \end{construct}

    \begin{theorem}[Poincar\'e--Birkhoff--Witt]\index{Poincar\'e--Birkhoff--Witt}\label{lie:pbw}
        Let $\mathfrak{g}$ be a Lie algebra with a totally ordered basis $\{g_i\}_{i\leq\dim(\mathfrak{g})}$. The monomials of the form $g_1^{m_1}g_2^{m_2}\cdots g_N^{m_N}$ constitute a basis for $U(\mathfrak{g})$.
    \end{theorem}

    \newdef{Casimir invariant\footnotemark}{\index{Casimir!invariant}\label{lie:casimir_invariant}
        \footnotetext{Also known as a \textbf{Casimir operator} or \textbf{Casimir element}.}
        Let $\mathfrak{g}$ be a Lie algebra. A Casimir invariant is an element in the center of $U(\mathfrak{g})$.
    }
    \newformula{Quadratic Casimir invariant}{
        Consider a Lie algebra representation $\rho:\mathfrak{g}\rightarrow\End(V)$ and let $\{X_i\}_{i\leq\dim(\mathfrak{g})}$ be a basis for $\mathfrak{g}$. The (quadratic) Casimir invariant associated with $\rho$ is given by
        \begin{gather}
            \Omega_\rho := \sum_{i=1}^{\dim(\mathfrak{g})}\rho(X_i)\circ\rho(\xi_i)\,,
        \end{gather}
        where the set $\{\xi_i\}_{i\leq n}$ is defined by the relation $K_\rho(X_i,\xi_j)=\delta_{ij}$ using the Killing form (\cref{lie:rho_killing_form}).
    }
    \begin{property}[Casimir invariants of irreducible representations]
        When the representation $\rho:\mathfrak{g}\rightarrow\End(V)$ is irreducible, Schur's lemma~\ref{rep:schurs_lemma} says that
        \begin{gather}
            \Omega_\rho = c_\rho\mathbbm{1}_V\,.
        \end{gather}
        By taking the trace of this formula and using \cref{lie:rho_killing_form}, it can be seen that
        \begin{gather}
            c_\rho = \frac{\dim(\mathfrak{g})}{\dim(V)}\,.
        \end{gather}
    \end{property}

    \newdef{Verma module}{\index{Verma module}\label{lie:verma_module}
        Consider a finite-dimensional Lie algebra $\mathfrak{g}$ with Borel subalgebra $\mathfrak{b}$. The Verma module with highest weight $\lambda$ is defined as follows\footnote{This can be seen as an `extension of scalars'-procedure where a $U(\mathfrak{b})$-module is turned into a $U(\mathfrak{g})$-module.}:
        \begin{gather}
            V(\lambda) := U(\mathfrak{g})\otimes_{U(\mathfrak{b})}\mathbb{C}_\lambda\,,
        \end{gather}
        where $\mathbb{C}_\lambda$ is the one-dimensional left $\mathfrak{b}$-module on which the Cartan subalgebra acts by weight $\lambda$ and $\mathfrak{n}_+\subset\mathfrak{b}$ acts trivially. $U(\mathfrak{g})$ contains $U(\mathfrak{b})$ as a subalgebra by the Poincar\'e--Birkhoff--Witt theorem~\ref{lie:pbw} and, hence, becomes a right $U(\mathfrak{b})$-module through right multiplication. Since $U(\mathfrak{g})$ is trivially a left module over itself, the Verma module also becomes a left $U(\mathfrak{g})$-module.
    }
    The Verma module with highest weight $\lambda$ can also be defined using a quotient construction.
    \begin{adefinition}
        Let $I_\lambda\subset U(\mathfrak{g})$ be the left ideal generated by the following elements (these relations precisely give the conditions for a highest weight vector):
        \begin{itemize}
            \item $x_\alpha\in\mathfrak{g_\alpha}$ for all positive roots $\alpha$, and
            \item $h-\lambda(h)1$ for all $h\in\mathfrak{h}$.
        \end{itemize}
        The Verma module $V(\lambda)$ is isomorphic to the quotient $U(\mathfrak{g})/I_\lambda$.
    \end{adefinition}

    The importance of Verma modules is given by the following property.
    \begin{property}[Highest weight modules]
        The Verma module $V(\lambda)$ is a highest weight module with highest weight vector $1\otimes1$, where the former $1$ indicates the unit of $U(\mathfrak{g})$ and the latter indicates the unit of $\mathbb{C}$. Furthermore, every highest weight module with highest weight $\lambda$ is a quotient of the Verma module $V(\lambda)$.
    \end{property}

    \begin{property}[Basis of Verma module]
         A basis for $V(\lambda)$ is given by the monomials
         \begin{gather}
             F_{\alpha_1}^{m_1}F_{\alpha_2}^{m_2}\cdots F_{\alpha_N}^{m_N}v_\lambda\,,
         \end{gather}
         where $v_\lambda$ is the highest weight vector, $\alpha_i$ are negative roots, $m_i\in\mathbb{N}$ and $F_{\alpha_i}\in\mathfrak{g}_{\alpha_i}$.
    \end{property}

\section{Group contractions}

    \newdef{In\"on\"u--Wigner contraction}{\index{In\"on\"u--Wigner contraction}\label{lie:inonu_wigner}
        Consider an $n$-dimensional Lie group $G$ with Lie algebra $\mathfrak{g}$ and choose a basis $\{e_i\}_{i\leq n}$ for $\mathfrak{g}$. A nonsingular transformation of the basis would leave the structure of the group unchanged. However, this nonsingular transformation can be rewritten in terms of a singular transformation:
        \begin{gather}
            U = u + \varepsilon w\,.
        \end{gather}
        The group contraction is obtained by taking the limit $\varepsilon\longrightarrow0$. In terms of the structure constants, this is equivalent to setting some of the structure constants to zero, thereby obtaining a subalgebra (and its associated subgroup). It can be shown that there exists a bijection between continuous subgroups and group contractions. The Lie algebra elements belonging to the contracted subalgebra form an Abelian invariant subalgebra and, hence, generate an Abelian invariant subgroup. The group contraction $\widetilde{G}$ is obtained as the quotient group of $G$ with respect to this Abelian subgroup.
    }

    \begin{example}[Galilei group]\index{Galilei!group}
        The \textit{Galilei group} in $d\in\mathbb{N}$ dimensions can be obtained as a group contraction of the \textit{inhomogeneous Lorentz group} in $d+1$ (spacetime) dimensions with respect to time displacements and spatial rotations.
    \end{example}

\section{\difficult{Lie algebra cohomology}}\index{cohomology!Lie algebra}\label{section:lie_algebra_cohomology}

    Although the construction of Lie algebra-cohomology can be generalized almost verbatim to the infinite-dimensional case, it is only stated for finite dimensions (the following definition, for example, is only valid for finite-dimensional algebras).
    \newdef{Chevalley--Eilenberg algebra}{\index{Chevalley--Eilenberg algebra}\label{lie:chevalley_eilenberg_algebra}
        Let $\mathfrak{g}$ be a finite-dimensional Lie algebra. Consider a basis $\{t_a\}_{a\leq\dim(\mathfrak{g})}$ of $\mathfrak{g}$ and its linear dual $\{t^a\}_{a\leq\dim(\mathfrak{g})}$. The Chevalley--Eilenberg algebra $\mathrm{CE}(\mathfrak{g})$ is defined as the Grassmann algebra $\Lambda^\bullet\mathfrak{g}^*$ with a dg-algebra structure induced by the differential\footnote{In \cref{section:higher_lie_structures} it was explained how this differential can be obtained as the dual of the Lie bracket.}
        \begin{gather}
            \dr t^a:=-\frac{1}{2}c_{bc}^{\ \ a}t^b\wedge t^c\,,
        \end{gather}
        where $c_{bc}^{\ \ a}$ are the structure constants of $\mathfrak{g}$.
    }

    By analogy with the case of group (co)homology as in \cref{section:group_cohomology}, the (co)homology of a Lie algebra is defined through the Tor- and Ext-functors. The natural choice of ring in the case of Lie algebras is the universal enveloping algebra $U(\mathfrak{g})$. The tensor and hom-operations underlying the construction are defined with respect to the trivial $U(\mathfrak{g})$-module $\mathfrak{K}$ (the underlying field of the Lie algebra). This gives
    \begin{gather}
        \begin{aligned}
            H^i_{\text{Lie}}(\mathfrak{g};M) &:= \mathrm{Ext}^i_{U(\mathfrak{g})}(\mathfrak{K},M)\,,\\
            H_i^{\text{Lie}}(\mathfrak{g};M) &:= \mathrm{Tor}_i^{U(\mathfrak{g})}(\mathfrak{K},M)\,,
        \end{aligned}
    \end{gather}
    where $M$ is a $\mathfrak{g}$-module and, by extension, a $U(\mathfrak{g})$-module.

    For simplicity, only cohomology will be considered here. The chapter on homological algebra, in particular \cref{section:tor_ext}, says that one has to find a projective resolution of $\mathfrak{K}$ to determine the Ext-functor in terms of hom-sets $\hom(\cdot,M)$. It can be shown that such a resolution is given by the tensor product $U(\mathfrak{g})\otimes\Lambda^\bullet\mathfrak{g}$:
    \begin{gather}
        \mathrm{Ext}^i_{U(\mathfrak{g})}(\mathfrak{K},M) = H^i\bigl(\hom_{\mathfrak{g}}(U(\mathfrak{g})\otimes_{\mathfrak{K}}\Lambda^\bullet\mathfrak{g},M)\bigr)\cong H^i\bigl(\hom_{\mathfrak{K}}(\Lambda^\bullet\mathfrak{g},M)\bigr)\,,
    \end{gather}
    where the differential of the (middle) complex is given by
    \begin{align}
        \dr(u\otimes g_1\wedge\cdots\wedge g_n) := &\sum_i(-1)^{i+1}ug_i\otimes(g_1\wedge\cdots\wedge\widehat{g}_i\wedge\cdots\wedge g_n)\\
        &+\sum_{i<j}(-1)^{i+j}u\otimes[g_i,g_j]\wedge\cdots\wedge\widehat{g}_i\wedge\cdots\wedge\widehat{g}_j\wedge\cdots\wedge g_n\,,\nonumber
    \end{align}
    where as usual the caret $\widehat\cdot$ indicates the omission of a factor. In the case $M=\mathfrak{K}$, the hom-complex can easily be seen to be the Chevalley--Eilenberg algebra $\mathrm{CE}(\mathfrak{g})\cong\Lambda^\bullet\mathfrak{g}^*$ (as stated above, this latter identification is only valid for finite-dimensional algebras). By a change of coefficients, the general case can be shown to be isomorphic to $\Lambda^\bullet\mathfrak{g}^*\otimes M$, where the usual differential on $\Lambda^\bullet\mathfrak{g}^*$ gets extended by an additional term $(-1)^n\dr m\otimes\omega$ with $\dr m(g):=g\cdot m$.

    \begin{property}[$H^0$ and $H^1$]
        The zeroth cohomology group $H^0(\mathfrak{g};M)$ is equal to the algebra of $\mathfrak{g}$-invariants in $M$:
        \begin{gather}
            \label{lie:zeroth_cohomology}
            H^0(\mathfrak{g};M) = \{m\in M\mid\mathfrak{g}\cdot m = m\}\,.
        \end{gather}
        The first cohomology group with coefficients in $\mathfrak{K}$ is isomorphic to the quotient of $\mathfrak{g}$ by its first derived ideal:
        \begin{gather}
            H^1(\mathfrak{g})\cong\mathfrak{g}/[\mathfrak{g},\mathfrak{g}]\,.
        \end{gather}
    \end{property}

    \begin{property}[Whitehead lemma]\index{Whitehead!lemma}
        If $\mathfrak{g}$ is semisimple, the cohomology groups $H^1(\mathfrak{g})$ and $H^2(\mathfrak{g})$ vanish. Conversely, a Lie algebra $\mathfrak{g}$ is semisimple if and only if $H^1(\mathfrak{g};M)$ vanishes for all finite-dimensional $\mathfrak{g}$-modules $M$.
    \end{property}

    \begin{property}[Classification of central extensions]\index{central extension!Lie algebra}
        The 2-cocycles (with values in $\mathfrak{K}$) from \cref{lie:cocycle} define classes in $H^2(\mathfrak{g})$. Furthermore, a central extension is trivial if and only if its associated cocycle is a coboundary. This says that the central extensions of $\mathfrak{g}$ by $\mathfrak{K}$ are classified by $H^2(\mathfrak{g})$.
    \end{property}
    \begin{result}
        Semisimple Lie algebras do not admit nontrivial central extensions.
    \end{result}

    \newdef{Weil algebra}{\index{Weil!algebra}\label{lie:weil_algebra}
        Consider a Lie algebra $\mathfrak{g}$. Its Weil algebra is defined as the semifree dg-algebra $\Lambda^\bullet(\mathfrak{g}^*\oplus\mathfrak{g}^*[1])$ with differential $\dr_W:=\dr_{\text{CE}}+\Pi$, where $\dr_{\text{CE}}$ is the differential on the Chevalley--Eilenberg subalgebra $\mathrm{CE}(\mathfrak{g})\subset W(\mathfrak{g})$ and $\Pi$ shifts the degree by $1$. The action of $\dr_{\text{CE}}$ on shifted generators is defined through the relation $[\Pi,\dr_{\text{CE}}]=0$.
    }
    \newdef{Horizontal elements}{\index{horizontal!elements}
        The elements of the subalgebra $\Lambda^\bullet\mathfrak{g}^*[1]$ are sometimes called the horizontal elements.
    }

    From here on, the subscript will be dropped and the differential of the Weil algebra will be denoted by $\dr$. It is clear that the above constructions fit in a short exact sequence.
    \begin{gather}
        \label{lie:weil_algebra_sequence}
        0\rightarrow\ker(p)\rightarrow W(\mathfrak{g})\overset{p}{\twoheadrightarrow}\mathrm{CE}(\mathfrak{g})\rightarrow0\,,
    \end{gather}
    where $p$ is the obvious projection map. An important subpace of $\ker(p)$ is given by the algebra of invariant polynomials $\mathrm{inv}(\mathfrak{g})$.
    \newdef{Invariant polynomial}{\index{invariant!polynomial}
        A horizontal element $\omega$ for which $\dr\omega$ is also horizontal. (Sometimes, the horizontality condition is replaced by $\dr\omega=0$.)
    }

    It should be noted that, although this definition might seem complicated, it is (for ordinary Lie algebras\footnote{The above definition leads to a straightforward generalization in the context of $L_\infty$-algebras.}) equivalent to the definition in terms of $\mathrm{Ad}$-invariant polynomials.
    \newadef{Invariant polynomial}{\index{invariant!polynomial}
        Let $G$ be a Lie group with Lie algebra $\mathfrak{g}$. A polynomial $P\in\mathfrak{K}[\mathfrak{g}]$ is said to be invariant (or $\mathrm{Ad}$-invariant) if
        \begin{gather}
            P(x) = P(gxg^{-1})
        \end{gather}
        for all $x\in\mathfrak{g}$ and $g\in G$. This subalgebra of $\mathfrak{K}[\mathfrak{g}]$ is denoted by $\mathfrak{K}[\mathfrak{g}]^G$.
    }

    A concept that will be important later on in the study of characteristic classes on fibre bundles is the transgression map.
    \newdef{Transgression}{\index{transgression}
        The exact sequence~\eqref{lie:weil_algebra_sequence} induces a long exact sequence in cohomology. An invariant polynomial is said to be in transgression with a cocycle in $\mathrm{CE}(\mathfrak{g})$ if their cohomology classes are related by the connecting morphism. More explicitly, by definition of invariant polynomials, one has $\dr\omega=0$ and, since $W(\mathfrak{g})$ has vanishing cohomology, there exists an element $c_\omega$ such that $\omega=\dr c_\omega$. By restricting $c_\omega$ to $\mathrm{CE}(\mathfrak{g})$, one obtains a $\mathfrak{g}$-cocycle, since $\dr_{\text{CE}}c_\omega=0$.
    }
    \begin{example}[Killing form]\label{lie:killing_transgression}
        Consider the invariant polynomial $\langle\cdot,\cdot\rangle$ induced by the Killing form on a semisimple Lie algebra. By transgression one obtains the canonical 3-cocycle $\langle\cdot,[\cdot,\cdot]\rangle$.
    \end{example}
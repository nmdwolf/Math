\chapter{Elementary Geometry}

\minitoc

\section{Straight lines}

    \begin{theorem}[Thales]
        
    \end{theorem}

\section{Trigonometry}

    \begin{figure}[ht!]
        \centering
        \begin{tikzpicture}
            \node (A) at (0, 0) {};
            \node (B) at (5, 0) {};
            \node (C) at (4, 3) {};
            \draw (A.center) -- node[below]{$b$} (B.center) -- node[right]{$a$} (C.center) -- node[above]{$c$} (A.center) pic["$\alpha$", draw, angle radius = 1cm] {angle = B--A--C} pic["$\gamma$", draw, angle radius = 1cm] {angle = C--B--A} pic["$\beta$", draw, angle radius = 1cm] {angle = A--C--B};
        \end{tikzpicture}
        \caption{General triangle.}
        \label{fig:general_triangle}
    \end{figure}

    \newdef{Trigonometric functions}{\index{sine}\index{cosine}\index{tangent}\index{hypotenuse}
        Consider a triangle as in \cref{fig:general_triangle} with $\gamma=\tfrac{\pi}{2}$, i.e.~a right-angled triangle.
        \begin{align}
            \sin(\alpha) &:= \frac{a}{c}\\
            \cos(\alpha) &:= \frac{b}{c}\\
            \tan(\alpha) &:= \frac{a}{b}
        \end{align}
        The longest side, $c$ in this case, is called the \textbf{hypotenuse}.
    }
    \begin{result}
        Since the adjacent side for one angle is the \todo{overstaande??} for the other, the formulas above give rise to the following relations:
        \begin{align}
            \sin(\alpha) &= \cos(\beta)\,,\\
            \cos(\alpha) &= \sin(\beta)\,,\\
            \tan(\alpha) &= \frac{1}{\tan(\beta)}\,.
        \end{align}
    \end{result}
    \begin{notation}[Cotangent]\index{cotangent}
        Because it occurs relatively often, the expression in the last line above received a distinct notation:
        \begin{gather}
            \cot(\alpha) := \frac{1}{\tan(\alpha)}\,.
        \end{gather}
    \end{notation}

    \begin{formula}
        \begin{gather}
            \tan(\theta)=\frac{\sin(\theta)}{\cos(\theta)}
        \end{gather}
        for any angle $\theta\in[0,2\pi[$.
    \end{formula}

    \begin{theorem}[Pythagoras]\index{Pythagoras}
        In any right-angled triangle, where $c$ denotes the hypotenuse, the following relation between the sides holds:
        \begin{gather*}
            a^2 + b^2 = c^2\,.
        \end{gather*}
    \end{theorem}
    Using the definitions of the trigonometric functions, this can also be rewritten.
    \begin{theorem}[Fundamental theorem of trigonometry]\index{fundamental theorem!of trigonometry}
        \begin{gather}
            \sin^2(\theta)+\cos^2(\theta)=1
        \end{gather}
        for any angle $\theta\in[0,2\pi[$.
    \end{theorem}

    The Pythagorean theorem can be extended as follows.
    \begin{formula}[Cosine law]
        \begin{gather}
            a^2+b^2-2ab\cos(\gamma) = c^2
        \end{gather}
    \end{formula}

    \begin{formula}[Sine law]
        \begin{gather}
            \frac{\sin(\alpha)}{a} = \frac{\sin(\beta)}{b} = \frac{\sin(\gamma)}{c}
        \end{gather}
    \end{formula}

    \begin{formula}[Tangent law]
        \begin{gather}
            \frac{a-b}{a+b} = \frac{\tan\bigl(\tfrac{1}{2}(\alpha-\beta)\bigr)}{\tan\bigl(\tfrac{1}{2}(\alpha+\beta)\bigr)}
        \end{gather}
    \end{formula}

    \begin{formula}[Addition formulas]
        \begin{align}
            \sin(\alpha+\beta) &= \sin(\alpha)\cos(\beta)+\cos(\alpha)\sin(\beta)\\
            \cos(a+b) &= \cos(\alpha)\cos(\beta)-\sin(\alpha)\sin(\beta)\\
            \tan(\alpha+\beta) &= \frac{\tan(\alpha)+\tan(\beta)}{1+\tan(\alpha)\tan(\beta)}
        \end{align}
    \end{formula}

    Besides the addition formulas, there are also product formulas. (Only one is given, since the others can be derived from it.)
    \begin{result}[Werner--Simpson]\index{Werner--Simpson formulas}
        \begin{gather}
            2\sin(\alpha)\cos(\beta) = \sin(\alpha-\beta)+\sin(\alpha+\beta)
        \end{gather}
    \end{result}

    Lesser known trigonometric functions are the cotangent-like versions of the sine and cosine.
    \newdef{Secant}{\index{secant}
        \begin{align}
            \sec(\theta) &:= \frac{1}{\sin(\theta)}\\
            \csc(\theta) &:= \frac{1}{\cos(\theta)}
        \end{align}
    }
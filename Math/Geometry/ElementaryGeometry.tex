\chapter{Elementary Geometry}

\minitoc

\section{Euclidean geometry}
\subsection{Postulates}

    \begin{axiom}
        Between every two points, one can draw a straight line.
    \end{axiom}

    \begin{axiom}
        Every (finite) line segment can be extended to a straight line.
    \end{axiom}

    \begin{axiom}
        Given any point and any (nonnegative) distance, one can draw a circle with the given point as centre and distance as radius.
    \end{axiom}

    \begin{axiom}
        All right angles are congruent.
    \end{axiom}

    \begin{axiom}[Parallel postulate]
        Consider a straight line intersecting two other straight lines as in \cref{fig:parallel_postulate}. If the sum of the interior angles ($\hat{A}$ and $\hat{B}$) on one side is less than $180^\circ$ (2 right angles), the two straight lines will intersect on that side (if extended indefinitely).
    \end{axiom}

    \begin{figure}[t!]
        \centering
        \begin{tikzpicture}
            \draw (0, 5) -- (10, 2);
            \draw (0, -3) -- (10, 1);
            \draw (3, 5) -- (2, -3);
            \node at (3.2, 3.7) {$\hat{A}$};
            \node at (2.5, -1.6) {$\hat{B}$};
        \end{tikzpicture}
        \caption{Parallel postulate.}
        \label{fig:parallel_postulate}
    \end{figure}

    \begin{result}
        If the sum of the interior angles on side (and, hence, on both sides) is exactly $180^\circ$, i.e.~they are supplementary, the two straight lines are parallel. (This is the reason for the name of the axiom.)
    \end{result}

    \begin{property}[Alternate interior angles]\index{interior!angles}
        Consider a straight line intersecting two parallel lines as follows:
        \begin{gather*}
            \begin{tikzpicture}
                \node (A) at (0, 0) {};
                \node (D) at (8, -3) {};
                \node (P) at (3, -3) {};
                \node (Q) at (5, 0) {};
                \draw (A) -- (8, 0);
                \draw (0, -3) -- (D);
                \draw (2, -4.5) -- (6, 1.5);
                \draw pic[draw, angle radius = .5cm] {angle =D--P--Q};
                \draw pic[draw, angle radius = .5cm] {angle =A--Q--P};
                \node at (4, -2.4) {$\alpha$};
                \node at (4, -.5) {$\beta$};
            \end{tikzpicture}
        \end{gather*}
        The parallel postulate implies that the angles $\alpha$ and $\beta$, called alternate interior angles, are equal.
    \end{property}

\subsection{Triangles and quadrangles}

    \begin{method}[Congruent triangles]
        Two triangles are congruent when they satisfy one and, hence, all of the following equivalent conditions:
        \begin{enumerate}[\quad a)]
            \item SSS (side--side--side): The triangles have corresponding sides of equal length.
            \item SAS (side--angle-side): The triangles have two corresponding pairs of sides of equal length and the included angles are of equal size.
            \item ASA (angle--side--angle): The triangles have two corresponding pairs of equal angles and the included sides are of equal length.
        \end{enumerate}
    \end{method}
    \begin{result}
        In right triangles, by \textit{Pythagoras' theorem} (see \cref{geom:pythagoras} below), the SSS condition gives rise to another possibility. The RHS (right-angle--hypotenuse--side) condition says that two right triangles are congruent if and only if
        \begin{enumerate}
            \item their \textbf{hypotenuses}, i.e.~the sides opposite to the right angle, are of equal length, and
            \item they have a pair of corresponding sides of equal length.
        \end{enumerate}
    \end{result}

    \begin{theorem}[Thales]\index{Thales}
        Consider \cref{fig:thales}. The ratios of sides in the triangles $\bigtriangleup ABC$ and $\bigtriangleup PQC$ are equal:
        \begin{gather}
            \frac{|PQ|}{|AB|} = \frac{|PC|}{|AC|} = \frac{|QC|}{|BC|}\,.
        \end{gather}
    \end{theorem}

    \begin{figure}
        \centering
        \begin{tikzpicture}
            \node (A) at (0, 0) {};
            \node (B) at (10, 2) {};
            \node (C) at (5, 8) {};
            \node (P) at (2, 3.2) {};
            \node (Q) at (8, 4.4) {};
            \draw (A.center) node[left]{$A$} -- (B.center) node[right]{$B$} -- (C.center) node[above]{$C$}-- (A.center);
            \draw (P.center) node[left]{$P$} -- (Q.center) node[right]{$Q$};
            \draw pic[draw, angle radius = 1cm] {angle =B--A--C};
            \draw pic[draw, angle radius = .9cm] {angle =B--A--C};
            \draw pic[draw, angle radius = 1cm] {angle =C--B--A};
            \draw pic[draw, angle radius = 1cm] {angle =Q--P--C};
            \draw pic[draw, angle radius = .9cm] {angle =Q--P--C};
            \draw pic[draw, angle radius = 1cm] {angle =C--Q--P};
        \end{tikzpicture}
        \caption{Thales' theorem.}
        \label{fig:thales}
    \end{figure}

    \begin{theorem}[Characteristics of parallelograms]\index{parallelogram}
        A quadrangle is a parallellogram if and only if one and, hence, all of the following equivalent conditions hold:\footnote{The first condition is usually taken as the definition.}
        \begin{enumerate}
            \item opposite sides are parallel,
            \item opposite sides are of equal length,
            \item opposite corners are of equal size, or
            \item diagonals bisect.
        \end{enumerate}
    \end{theorem}

\section{Trigonometry}

    \begin{figure}[ht!]
        \centering
        \begin{tikzpicture}
            \node (A) at (0, 0) {};
            \node (B) at (10, 0) {};
            \node (C) at (10, 5) {};
            \draw (A.center) -- node[below]{$b$} (B.center) -- node[right]{$a$} (C.center) -- node[above]{$c$} (A.center) pic["$\alpha$", draw, angle radius = 1.5cm] {angle = B--A--C} pic["$\gamma$", draw, angle radius = 1cm] {angle = C--B--A} pic["$\beta$", draw, angle radius = 1cm] {angle = A--C--B};
        \end{tikzpicture}
        \caption{Right-angled triangle.}
        \label{fig:general_triangle}
    \end{figure}

    \newdef{Trigonometric functions}{\index{sine}\index{cosine}\index{tangent}\index{hypotenuse}
        Consider a triangle as in \cref{fig:general_triangle}, where $\gamma=\tfrac{\pi}{2}$.
        \begin{align}
            \sin(\alpha) &:= \frac{a}{c}\\
            \cos(\alpha) &:= \frac{b}{c}\\
            \tan(\alpha) &:= \frac{a}{b}
        \end{align}
        The longest side, $c$ in this case, always corresponds to the hypotenuse.
    }
    \begin{result}
        Since the adjacent side for one angle is the opposite side for the other, the formulas above give rise to the following relations:
        \begin{align}
            \sin(\alpha) &= \cos(\tfrac{\pi}{2}-\alpha)\,,\\
            \cos(\alpha) &= \sin(\tfrac{\pi}{2}-\alpha)\,,\\
            \tan(\alpha) &= \frac{1}{\tan(\tfrac{\pi}{2}-\alpha)}\,.
        \end{align}
    \end{result}
    
    \begin{notation}[Cotangent]\index{cotangent}
        Because it occurs relatively often, the last expression above receives a distinct notation:
        \begin{gather}
            \cot(\alpha) := \frac{1}{\tan(\alpha)}\,.
        \end{gather}
    \end{notation}
    Lesser known trigonometric functions are the cotangent-like versions of the sine and cosine.
    \newnot{Secant}{\index{secant}
        \begin{align}
            \sec(\theta) &:= \frac{1}{\sin(\theta)}\\
            \csc(\theta) &:= \frac{1}{\cos(\theta)}
        \end{align}
    }

    \begin{formula}
        \begin{gather}
            \tan(\theta)=\frac{\sin(\theta)}{\cos(\theta)}
        \end{gather}
        for any angle $\theta\in[0,2\pi[$.
    \end{formula}

    \begin{theorem}[Pythagoras]\index{Pythagoras}\label{geom:pythagoras}
        In any right-angled triangle, where $c$ denotes the hypotenuse, the following relation between the sides holds:
        \begin{gather*}
            a^2 + b^2 = c^2\,.
        \end{gather*}
    \end{theorem}
    Using the definitions of the trigonometric functions, this can also be rewritten.
    \begin{theorem}[Fundamental theorem of trigonometry]\index{fundamental theorem!of trigonometry}
        \begin{gather}
            \sin^2(\theta)+\cos^2(\theta)=1
        \end{gather}
        for any angle $\theta\in[0,2\pi[$.
    \end{theorem}

    The Pythagorean theorem can be extended as follows.
    \begin{formula}[Cosine law]
        In a general triangle, sides and angles are related as follows:
        \begin{gather}
            a^2+b^2-2ab\cos(\gamma) = c^2\,.
        \end{gather}
    \end{formula}

    \begin{formula}[Sine law]
        In a general triangle, sides and sines of the opposite angles are related as follows:
        \begin{gather}
            \frac{\sin(\alpha)}{a} = \frac{\sin(\beta)}{b} = \frac{\sin(\gamma)}{c}
        \end{gather}
    \end{formula}

    \begin{formula}[Tangent law]
        \begin{gather}
            \frac{a+b}{a-b} = \frac{\tan\bigl(\tfrac{1}{2}(\alpha+\beta)\bigr)}{\tan\bigl(\tfrac{1}{2}(\alpha-\beta)\bigr)}
        \end{gather}
    \end{formula}

    \begin{formula}[Addition formulas]
        \begin{align}
            \sin(\alpha+\beta) &= \sin(\alpha)\cos(\beta)+\cos(\alpha)\sin(\beta)\\
            \cos(a+b) &= \cos(\alpha)\cos(\beta)-\sin(\alpha)\sin(\beta)\\
            \tan(\alpha+\beta) &= \frac{\tan(\alpha)+\tan(\beta)}{1+\tan(\alpha)\tan(\beta)}
        \end{align}
    \end{formula}

    Besides the addition formulas, there are also product formulas. (Only one is given, since the others can be derived from it.)
    \begin{result}[Werner--Simpson]\index{Werner--Simpson formulas}
        \begin{gather}
            2\sin(\alpha)\cos(\beta) = \sin(\alpha-\beta)+\sin(\alpha+\beta)
        \end{gather}
    \end{result}
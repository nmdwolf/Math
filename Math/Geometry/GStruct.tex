\chapter{\texorpdfstring{$G$-Structures}{G-Structures}}
\setcounter{section}{1}

    In the following table, an overview of the most common $G$-structures on a smooth (simply-connected) manifold $M^n$ is given.
    \begin{center}
        \begin{tabularx}{\textwidth}{|l|c|X|}
             \hline
                 Geometric structure&Structure group&Remarks\\
             \hline
                 Orientation&$\mathrm{SL}(n,\mathbb{R})$&$\GL^+(n,\mathbb{R})$ is sufficient for orientability. The special linear group gives rise to a volume form.\\
                 Riemannian metric&$\mathrm{O}(n)$&\\&&\\
                 Almost-symplectic structure*&$\mathrm{Sp}(n,\mathbb{R})$&Integrability (in the form of a closed form) gives a symplectic manifold.\\&&\\
                 Almost-complex structure*&$\GL(k,\mathbb{C})$&Integrability (in the sense of Newlander--Nirenberg) gives a complex manifold.\\&&\\
                 Almost-Hermitian structure*&$\mathrm{U}(k)$&Integrability gives a K\"ahler manifold.\\&&\\
                 Calabi--Yau*&$\mathrm{SU}(k)$&\\&&\\
                 Hyper-K\"ahler**&$\mathrm{Sp}(k)$&Hyper-K\"ahler implies Calabi--Yau.\\&&\\
                 Almost quaternionic**&$(\GL(k,\mathbb{H})\times\mathbb{H}^\times)/\mathbb{R}^\times$&Integrability gives a quaternionic manifold. $k\geq2$ is required because for $k=1$ one would obtain that every orientable 4-manifold is quaternionic (amongst other things).\\&&\\
                 Quaternionic K\"ahler**&$(\mathrm{Sp}(k)\times\mathrm{Sp}(1))/\mathbb{Z}_2$&These manifolds are not strictly K\"ahler since the structure group is not a subgroup of $\mathrm{U}(2k)$.\\
             \hline
        \end{tabularx}
    \end{center}
    Structures marked with $\ast$ require the real dimension $n=2k$ to be even. Structures marked with $\ast\ast$ require the real dimension $n=4k$ to be a multiple of 4.

    \begin{remark*}
        This table is strongly related to the classification of (\textit{irreducible}, simply-connected and \textit{nonsymmetric}) Riemannian manifolds by \textit{Berger}. (The $\mathrm{SL}(n,\mathbb{R})$-structure is technically not part of the original classification since it is not a subgroup of $\mathrm{O}(n)$ and, hence, the manifold is not necessarily Riemannian.) A more general classification for manifolds that are not necessarily Riemannian was initiated by \textit{Berger} and finished by others. This extension will only be mentioned here. For references, see~\citet{rudolph_differential_2017}.

        Since not all concepts from this classification were defined throughout the compendium they are explained here:
        \begin{itemize}
            \item\textbf{Irreducible}: A Riemannian manifold is said to be irreducible if it is not locally isomorphic to a product of Riemannian manifolds.
            \item\textbf{Symmetric}: A smooth manifold, locally modelled on $V\cong\mathbb{R}^n$, is said to be symmetric (or \textbf{locally symmetric}) if the curvature mapping $FM\rightarrow\Lambda^2V^*\otimes\mathfrak{g}$ is covariantly constant.
        \end{itemize}
    \end{remark*}

    \begin{remark}
        Although most manifolds from the above list admit an explicit definition, the quaternionic K\"ahler manifolds are exactly defined by their structure group/holonomy group.

        It is also clear that hyper-K\"ahler manifolds are a specific class of quaternionic K\"ahler manifolds since $\mathrm{Sp}(k)$ can be embedded in $\mathrm{Sp}(k)\cdot\mathrm{Sp}(1)$. To exclude this class, one can just require the holonomy group to be all of $\mathrm{Sp}(k)\cdot\mathrm{Sp}(1)$. This is equivalent to requiring that quaternionic K\"ahler manifolds should have a nonvanishing scalar curvature.
    \end{remark}

    This remark is related to the following property.
    \begin{property}[Einstein]\index{Einstein!manifold}
        Every quaternionic K\"ahler manifold is also Einstein (\cref{riemann:einstein_manifold}). The hyper-K\"ahler manifolds are exactly the quaternionic K\"ahler manifolds with vanishing scalar curvature (which is constant since the manifold is Einstein).
    \end{property}
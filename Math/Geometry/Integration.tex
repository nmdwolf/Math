\section{Integration Theory}\label{section:integration_manifolds}

    For the theory of measure spaces and Lebesgue integration, see \cref{chapter:measure}. Further below, integration theory will be generalized from orientable manifolds to nonorientable manifolds. A good introduction for this is~\citet{tiee_contravariance_2006}.

\subsection{Orientation and densities}\label{section:orientability}

    One can define an orientation on manifolds by generalizing the situation for vector spaces (\cref{vector:orientation}).
    \newdef{Orientable manifold}{\index{orientation}\index{orientable!manifold}\index{volume!form}\label{bundle:orientability}
        First, the definition of the volume element needs to be slightly modified. A \textbf{volume form} on $M$ is a nowhere-vanishing top-dimensional differential form $\vol\in\Omega^{\dim(M)}(M)$. The definition of an orientation is now virtually the same as for vector spaces.

        An \textbf{oriented atlas} is given by all charts of $M$ for which the pullback of the Euclidean volume form is a positive multiple of $\vol$. This also implies that the transition functions have a positive Jacobian determinant. A differentiable manifold that can equipped with such a volume form (and corresponding orientable atlas) is called an orientable manifold. If such a choice has been made, it is said to be \textbf{oriented}. Moreover, on an oriented manifold, a basis $\{v_1,\ldots,v_n\}$ for a tangent space $T_pM$ is said to be \textbf{positively oriented} if
        \begin{gather}
            \vol_p(v_1,\ldots,v_n)>0\,.
        \end{gather}
    }

    \begin{example}[Determinant]\index{determinant}
        Let $M=\mathbb{R}^n$. The canonical Euclidean volume form is given by the determinant map
        \begin{gather}
            \det:(v_1,\ldots,v_n)\mapsto\det(v_1,\ldots,v_n)\,,
        \end{gather}
        where the $v_n$'s are expanded in the standard basis $\{e_1,\ldots,e_n\}$ to calculate the determinant. The terminology of `volume forms' is justified by noting that the determinant map gives the signed volume of the $n$-dimensional parallelotope spanned by the vectors $\{v_1,\ldots,v_n\}$.
    \end{example}

    \begin{property}
        Let $\omega_1,\omega_2$ be two volume forms on $M$. Because the space of top-degree forms is one-dimensional, there exists a smooth function $f$ such that \[\omega_1 = f\omega_2\,.\] Furthermore, the sign of this function is constant on every connected component of $M$.
    \end{property}

    One can also rephrase orientability of manifolds in terms of bundles.
    \newdef{Orientation bundle}{\index{orientation}\label{bundle:orientation_bundle}
        Consider a manifold $M$. The transition function $A$ of $TM$ is given by the Jacobian of the transitions functions on $M$. The associated line bundle with transition function $\sgn\det(A)$ is called the orientation bundle $o(M)$.

        In general, one can define the orientation bundle $o(E)$ for any vector bundle $E$, where one replaces the Jacobian in the above construction by the transition maps of $E$. From this, it is clear that the orientation bundle $o(M)$ is the same as $o(TM)$.
    }
    \newadef{Orientable manifold}{
        A smooth manifold for which the orientation bundle is trivial.
    }

    \begin{remark}
        By definition of the orientation bundle, the transition functions are those that have a positive determinant. This gives the equivalence with \cref{bundle:orientability}. In the next chapter on principal bundles, yet another (equivalent) definition of orientability in terms of \textit{$G$-structures} will be given (see \cref{bundle:orientable_structure}).
    \end{remark}

    A last definition, strongly related to the previous one, is the following.
    \begin{adefinition}\label{bundle:orientable_skeleton}
        Consider a manifold $M$ and choose a triangulation $\mathcal{K}$ (\cref{homalg:triangulation}). $M$ is orientable if the pullback of the tangent bundle $TM$ to the $1$-skeleton $\mathcal{K}^1$ is trivial.
    \end{adefinition}

    \newdef{Tensor density}{\index{tensor!density}\label{bundle:density}
        Consider a vector bundle $E\rightarrow M$ defined by transition functions $A$. The associated bundle of (tensor) $s$-densities is obtained by using the representation
        \begin{gather}
            \rho:A\mapsto\det(A)^{-s}\,.
        \end{gather}
        The number $s\in\mathbb{R}$ is called the \textbf{weight} of the density. For $E\equiv TM$, one obtains the (tensor) $s$-densities on $M$, which, in the case of $s=1$, are equivalent to top-dimensional forms on $M$. When twisting a vector bundle by an $s$-density bundle, the prefix `$s$-weighted' is often added.
    }

    \begin{example}[Pseudoscalars]\index{pseudo-!scalar}
        Let $G$ be a Lie group and consider a group morphism $\phi:G\rightarrow\mathrm{O}(p,q)$ for some $p,q\in\mathbb{N}$. The pseudoscalar representation of $G$, induced by $\phi$, is defined as the one-dimensional representation given by
        \begin{gather}
            \symbf{1}_{\sgn}:g\mapsto\det\bigl(\phi(g)\bigr)\,.
        \end{gather}
        The notation $\symbf{1}_{\sgn}$ refers to the fact that this is a generalization of the alternating representation (\cref{rep:sign_representation}) of the permutation groups $S_n$.

        Sections of a vector bundle with transition functions defined by $\symbf{1}_{\sgn}$ are generally called \textbf{pseudoscalar fields}. When using the pseudoscalar representation of the transition functions of the tangent bundle $TM$ to construct an associated bundle, one obtains the pseudoscalar bundle $\Psi_M$. A vector bundle twisted by the pseudoscalar bundle $\Psi$ often receives the prefix `pseudo', e.g.~the $\Psi$-twisted $k$-form bundle is called the bundle of $k$-\textbf{pseudoforms}.
    \end{example}
    \begin{example}[Pseudovectors]\index{pseudo-!vector}
        The representation
        \begin{gather}
            \rho:A\mapsto\det(A)A
        \end{gather}
        gives rise to a bundle similar to the tangent bundle, where the sign of the cocycles now has an influence on the fibres. Sections of such bundles are called \textbf{pseudovector fields}. This construction is equivalent to twisting the tangent bundle by the pseudoscalar bundle, hence its name.
    \end{example}

    \begin{remark}[Honest densities]\index{density}\label{bundle:honest_density}
        One should pay incredible attention to the definition of a \textbf{density} (i.e.~without the prefix `tensor'). A density is defined as an $n$-pseudoform, i.e.~a section of the \textbf{density bundle} $|\Omega|(M):=\Omega^n(M)\otimes o(M)$. Here, the transition function is $|\!\det(A)|$, where $A$ is the transition function of $T^*M$.\footnote{Note that, in the literature, one often uses the transition functions of $TM$, which are the inverses $A^{-1}$. This leads to expressions involving negative exponents.} These are the objects one can integrate over any manifold, even the nonorientable ones. They are essentially maps $\Gamma(\det(T^*M))\rightarrow C^\infty(M)$. A naive way to construct a density on a manifold $M$ is by choosing a volume form $\vol(M)$ and taking the absolute value $|\!\vol(M)|$.

        One can also define \textbf{honest $s$-densities} $|\Omega|^s(M)$ by tensoring tensor densities with the orientation bundle to obtain transition maps $|\!\det(A)|^s$, where $A$ is again the cotangent transition function. This is also the only possible way to generalize the (tensor) $s$-densities to real $s$.
    \end{remark}

    \begin{property}[Orientability]
        A smooth manifold is orientable if and only if its canonical line bundle (\cref{bundle:canonical_bundle}) is trivial. Furthermore, for orientable manifolds, there exists an isomorphism $\Gamma(\det(T^*M))\cong\Gamma(|\Omega|(M))$.
    \end{property}

\subsection{Orientation in homology}

    In this section, a characterization of orientability in terms of homology is given. As such, smoothness is not required. See \cref{section:homology} and \cref{section:singular_homology} for an introduction to homology.

    First, consider the canonical example $\mathbb{R}^n$. Intuitively one would expect an orientation on an Euclidean space to be a property that is preserved under rotations and reversed by reflections. On the sphere these operations have degree $1$ and $-1$ respectively, so the perfect choice for an orientation would be the generator of $H_n(S^n)\cong\mathbb{Z}$. Luckily, there exists an isomorphism $H_n(S^n)\cong H_n(\mathbb{R}^n,\mathbb{R}^n\backslash\{\ast\})$. So, for every point $x\in\mathbb{R}^n$, one can define a local orientation as a choice of generator of the local homology group $H_n(\mathbb{R}^n,\mathbb{R}^n\backslash\{x\})$.

    \newadef{Orientation}{\index{orientation}
        An orientation on a manifold $M$ is a choice of local orientation for every point $p\in M$ such that every two points admitting a common covering chart have consistent local orientations.
    }

    \begin{property}[Orientability]\index{fundamental!class}\index{orientation!class}\label{bundle:orientation_class}
        If a closed, connected manifold is ($\mathbb{Z}\,$-)orientable, there exists an isomorphism
        \begin{gather}
            H_n(M)\cong H_n(M,M\backslash\{p\})\cong\mathbb{Z}
        \end{gather}
        for all points $p\in M$. A choice of class in $H_n(M)$ that maps to a generator of $H_n(M,M\backslash\{p\})$ for all $p\in M$ is called a \textbf{fundamental class} or \textbf{orientation class}.

        When $M$ is not connected, the fundamental class equals the direct sum of the generators of the connected components (following the idea of the additivity axiom~\ref{topology:eilenberg_steenrod_axioms}).
    \end{property}
    The above definition and property can be generalized to arbitrary unital rings $R$.
    \newdef{$R$-orientability}{\index{orientable!manifold}
        A manifold is $R$-orientable if a consistent choice of local $R$-orientation exists or, equivalently, if $H_n(M;R)\cong R$.
    }

    \begin{property}[Nonorientable manifolds]
        If $M$ is not $R$-orientable, the map \[H_n(M;R)\rightarrow H_n(M,M\backslash\{p\};R)\] is still injective with image $\{r\in R\mid 2r=0\}$. In particular, every closed manifold is $\mathbb{Z}_2$-orientable.
    \end{property}

    \begin{property}[Orientability implies $R$-orientability]
        By the \textit{universal coefficient theorem}, it follows that a $\mathbb{Z}$-orientable manifold is also $R$-orientable for all unital rings $R$. Conversely, a manifold is $\mathbb{Z}$-orientable if it is $R$-orientable for all unital rings $R$.
    \end{property}

\subsection{Integration}\index{Lebesgue!integral}\index{measure}

    \newdef{Measure zero}{\index{null!set}
        A subset $U\subseteq M$ of an orientable manifold is said to be of measure zero (or \textbf{null}) if it is the countable union of inverse images (with respect to the chart maps on $M$) of null sets in $\mathbb{R}^n$.
    }

    \newdef{Integrable form}{\index{integrable}
        A differential form for which its components with respect to any basis of $\Omega^{\dim(M)}(M)$ are Lebesgue integrable on $\mathbb{R}^{\dim(M)}$.
    }

    \begin{formula}[Integration with compact support]\label{bundle:integration_compact_support}
        Consider a top-dimensional form $\omega\in\Omega^{\dim(M)}$ on $M$ with compact support on a coordinate patch $U\subseteq M$.
        \begin{gather}
            \Int_M\omega = \Int_U\omega := \Int_{-\infty}^{+\infty}\cdots\Int_{-\infty}^{+\infty}\omega_{12\ldots n}(x)\,dx^1\,dx^2\cdots dx^n\,.
        \end{gather}
        This integral is well defined because, under an orientation-preserving coordinate transformation, the component $\omega_{1\ldots n}$ transforms as $\omega'_{1\ldots n} = \det(J)\omega_{1\ldots n}$, where $J$ is the Jacobian of the coordinate transformation. Inserting this in the integral and replacing $dx_i$ by $dx'_i$ gives the well-known change-of-variables formula from (Lebesgue) integration theory.

        If one requires the manifold $M$ to be paracompact, such that every open cover $\{U_i\subseteq M\}_{i\in I}$ admits a subordinate partition of unity $\{\phi_i\}_{i\in I}$, one can define the integral of a general compactly supported form $\omega\in\Omega^n(M)$ as follows:
        \begin{gather}
            \Int_M\omega := \sum_{i\in I}\Int_{U_i}\rho_i\omega\,.
        \end{gather}
    \end{formula}
    \begin{remark}
        Although integration was only defined for compactly supported forms, the general formula can also be applied to general forms. It is well defined whenever the forms $\rho_i\omega$ are integrable and the sum in the definition converges.
    \end{remark}

    \newprop{Compact manifolds}{
        Let $M$ be a smooth compact manifold. Because every form on $M$ is automatically compactly supported, all forms are integrable on $M$.
    }

    \newprop{Invariance under pullbacks}{
        Consider an orientation-preserving diffeomorphism $f:M\rightarrow N$.
        \begin{gather}
            \Int_Mf^*\omega = \Int_N\omega
        \end{gather}
    }

    \begin{notation}
        Because the integral of differential forms is linear in the integrand and additive over disjoint unions, it can be interpreted as a bilinear pairing. This motivates the following notation:
        \begin{gather}
            \langle M,\omega \rangle := \Int_M\omega\,.
        \end{gather}
    \end{notation}

\subsection{Stokes' theorem}

    \begin{theorem}[Stokes' theorem]\index{Stokes}\label{bundle:stokes_theorem}
        Let $M$ be an orientable manifold with boundary $\partial M$ and let $\omega$ be a differential $k$-form on $M$.
        \begin{gather}
            \Int_{\partial M}\omega = \Int_M\dr\omega
        \end{gather}
    \end{theorem}
    \begin{result}
        The Kelvin--Stokes theorem~\ref{vector:kelvin_stokes_theorem}, the divergence theorem~\ref{vector:divergence_theorem} and Green's identity~\ref{vector:green_indentity} are immediate results of Stokes' theorem.
    \end{result}

    \newdef{Calibration}{\index{calibration}
        A degree-$p$ calibration on a smooth manifold $M$ with volume form $\mathrm{Vol}$ is a differential form $\omega\in\Omega^p(M)$ satisfying the following conditions:
        \begin{enumerate}
            \item\textbf{Closedness}: $\dr\omega=0$.
            \item\textbf{Volume}: Over any dimension-$p$ submanifold, the integral of $\omega$ is smaller than its volume, with at least one submanifold saturating the inequality.
        \end{enumerate}
        A submanifold is said to be \textbf{calibrated} if the restriction of the calibration to this submanifold coincides with the induced volume form.
    }
    \begin{property}
        Calibrated submanifolds minimize the volume within their homology class.
    \end{property}
\section{Morse theory}\label{section:morse}
\subsection{Morse functions}

    \newdef{Nondegeneracy}{\index{critical!point}\index{nondegeneracy}\label{manifold:nondegeneracy}
        At a critical point $p\in M$, the Hessian of a smooth function $f:M\rightarrow\mathbb{R}$, in local coordinates, gives a well-defined quadratic form. A critical point is said to be nondegenerate if the Hessian is nonsingular there.
    }

    \newdef{Morse function}{\index{Morse!function}\label{manifold:morse_function}
        Let $M$ be a smooth manifold. A smooth function is called a Morse function if it has no degenerate critical points.
    }

    \begin{property}[Density]
        The set of Morse functions is open and dense in the $C^2$-topology (see \cref{section:jet_bundles} on \textit{jet spaces}).
    \end{property}

    \newdef{Palais--Smale condition}{\index{Palais--Smale condition}
        A smooth function $f\in C^1(M)$ is said to satisfy the Palais--Smale condition if every sequence $\seq{x}\subset M$ with
        \begin{enumerate}
            \item $|f(x_n)|$ bounded for all $n\in\mathbb{N}$, and
            \item $\|Df(x_n)\|\longrightarrow0$
        \end{enumerate}
        contains a convergent subsequence. It is clear that every smooth function on a compact manifold and every proper function (\cref{topology:proper_function}) satisfies this condition.
    }
    \begin{result}
        If $f\in C^1(M)$ is Morse and satisfies the Palais--Smale condition, it has only finitely many critical points in every bounded subset or in any set where $f$ is bounded.
    \end{result}

    \newdef{Morse index}{\index{index}
        Consider a Morse function $f\in C^\infty(M)$. The number of negative eigenvalues at a critical point $p\in M$ is called the (Morse) index of $f$ at $p$. This is often denoted by $\lambda_p(f)$.

        To any Morse function one can associate a series called the \textbf{Morse counting-series}:
        \begin{gather}
            M_t(f) := \sum_{p\in\mathrm{crit}(f)}t^{\lambda_p(f)}\,.
        \end{gather}
        If $M$ is compact, the nondegeneracy condition implies that the above sum only has a finite number of terms.
    }

    \begin{property}[Morse lemma]\index{Morse!lemma}
        Consider a Morse function $f:M\rightarrow\mathbb{R}$ and let $p\in M$ be a nondegenerate critical point of $f$. There exists a chart $(U,x_1,\ldots,x_n)$ around $p$ such that $x_i(p)=0$ and
        \begin{gather}
            f|_U(x) = f(p) - x_1^2-\cdots + x_k^2+\cdots\,,
        \end{gather}
        where $k=\lambda_p(f)$ is the Morse index of $f$.
    \end{property}
    \begin{result}
        The critical points of a Morse function are isolated.
    \end{result}
    \begin{remark}[Morse--Palais lemma]\index{Morse--Palais}
        The Morse lemma can be generalized to open subsets of Banach spaces (and thus to infinite-dimensional manifolds).
    \end{remark}

    \newdef{Self-indexing function}{
        A Morse function whose value at every critical points is equal to its index.
    }

\subsection{Morse--Bott functions}

    By the Morse lemma, the critical points of a Morse function are isolated. When this condition is relaxed, a more general class of functions is obtained. Here, it is assumed that $M$ comes equipped with a covariant derivative (\cref{section:linear_connections}).

    \newdef{Morse--Bott function}{\index{Morse--Bott function}
        A smooth function $f:M\rightarrow\mathbb{R}$ for which the critical set $\mathrm{Crit}(f)$ is a submanifold of $M$ and at every point $p\in\mathrm{Crit}(f)$ the tangent space is the kernel of the Hessian of $f$, i.e.~its Hessian is nondegenerate in the normal directions at every critical point.
    }

\subsection{Morse homology}\label{section:morse_homology}

    \newdef{Gradientlike vector field}{\index{vector field!gradientlike}
        Consider a Morse function $f\in C^\infty(M)$. A vector field $X$ is said to be gradientlike with respect to $f$ if is satisfies the following conditions:
        \begin{enumerate}
            \item For all $p\not\in\mathrm{Crit}(f):X|_p(f)>0$.
            \item For all $p\in\mathrm{Crit}(f)$, there exists a Morse chart containing $p$ such that
            \begin{gather}
                X = -2\sum_{i=1}^{\lambda_p(f)}x^i\partial_i+2\sum_{i=\lambda_p(f)+1}^{\dim(M)}x^i\partial_i\,.
            \end{gather}

        \end{enumerate}
        Its flow lines have the same orientation away from critical points and it coincides with the gradient at critical points. Furthermore, such vector fields always exist.
    }
    \begin{property}
        Let $f\in C^\infty(M)$ be a Morse function on a compact\footnote{This property also holds more generally, but then one has to restrict to integral curves $\gamma$ whose image $f(\gamma)$ is bounded.} manifold and consider a gradientlike vector field $X$ with respect to $f$. The integral curves of $X$ are complete and the limits are critical points of $f$.
    \end{property}

    \newdef{Stable and unstable manifold}{\index{stable!manifold}
        Let $f\in C^\infty(M)$ be a Morse function and consider a gradientlike vector field $X$ (with respect to $f$). For every critical point $p$ of $f$, the stable and unstable manifold of $X$ are defined as follows:
        \begin{gather}
            W^\pm_p(X) := \bigl\{x\in M\bigm\vert\lim_{t\rightarrow\pm\infty}\Phi_t(x)=p\bigr\}\,,
        \end{gather}
        where $\Phi_t$ denotes the flow of $-X$. These sets carry the structure of smooth manifolds that are locally diffeomorphic to $\mathbb{R}^{\dim(M)-\lambda_p(f)}$ and $\mathbb{R}^{\lambda_p(f)}$, respectively.
    }

    \newdef{Morse--Smale pair}{\index{Morse--Smale pair}
        Let $f\in C^\infty(M)$ be a Morse function and consider a gradientlike vector field $X$ with respect to $f$. If, for all critical points $p,q\in\mathrm{Crit}(f)$, one has that
        \begin{gather}
            W^+_p(X)\pitchfork W^-_q(X)\,,
        \end{gather}
        the pair $(f,X)$ is called a Morse--Smale pair.
    }
    \begin{property}
        If $M$ is compact, there exists a self-indexing Morse--Smale pair.
    \end{property}

    \begin{property}
        For every Morse function on a compact manifold, there exists a generic metric such that $(f,\nabla f)$ is Morse--Smale.
    \end{property}

    From here on, it will be assumed that, given a Morse function $f\in C^\infty(M)$, the pair $(f,\nabla f)$ is Morse--Smale. Let $\mathcal{M}(p,q)$ denote the set of integral curves of $-\nabla f$ that start at $p$ and end at $q$, i.e.~the integral curves $\gamma$ that satisfy
    \begin{gather}
        \gamma([0,1])\subset W^-_p(\nabla f)\cap W^+_q(\nabla f)\,.
    \end{gather}
    By the structure of the stable and unstable manifolds, this solution space has dimension $\lambda_p(f)-\lambda_q(f)$. Integral curves can be arbitrarily reparametrized. To obtain a well-defined moduli space $\overline{\mathcal{M}}(p,q)$, this $\mathbb{R}$-action is quotiented out (it is free and proper, so the resulting space is again a smooth manifold).

    \newdef{Morse homology}{\index{Morse|seealso{homology}}\index{homology!Morse}
        The chain groups are defined as follows
        \begin{gather}
            CM_k(M,f) := \bigoplus_{\substack{p\in\mathrm{Crit}(f)\\\lambda_p(f)=k}}\mathbb{Z}\langle p \rangle\,.
        \end{gather}
        For critical points $p,q\in\mathrm{Crit}(f)$ such that $\lambda_p(f)=\lambda_q(f)+1$, the moduli space is a discrete, compact set. This allows to define the boundary operator as follows:
        \begin{gather}
            \partial p := \sum_{\substack{q\in\mathrm{Crit}(f)\\\lambda_q(f)=\lambda_p(f)-1}}\left|\,\overline{\mathcal{M}}(p,q)\right|\langle q \rangle\,.
        \end{gather}
        One can show that $\partial^2=0$. Morse homology is defined as the homology of this complex:
        \begin{gather}
            HM_\bullet(M,f) := \frac{\ker(\partial)}{\im(\partial)}\,.
        \end{gather}
    }
\chapter{Computer Science}

    Although slightly off-topic compared to the other chapters in this compendium, many concepts in (quantum) information theory and quantum computing (\cref{chapter:quantum_computing}) are strongly related to computer science. Moreover, an interest in theoretical computer science grew even more during the PhD years.

    \minitoc

\section{Computability}\label{section:computability}
\subsection{Turing machines}

    \newdef{Turing machine}{\index{Turing machine}
        A (\textbf{deterministic}) Turing machine consists of 4 components:
        \begin{itemize}
            \item\textbf{Tape}: This consists of sequential cells, each containing one symbol from an alphabet $\mathcal{A}$ (\cref{section:logic}).
            \item\textbf{Head}: This can change the symbol of the current cell and change the current cell index.
            \item\textbf{State register}: Stores the state of the Turing machine.
            \item\textbf{Table of instruction}: A finite list of instructions that, given the current state and cell symbol, consisting of three steps:
            \begin{enumerate}
                \item Erase or modify the symbol of the current cell.
                \item Move the head one cell to the left or right.
                \item Update the current state.
            \end{enumerate}
        \end{itemize}
        If the instruction table prescribes multiple possible actions for some combinations of state and symbol, the Turing machine is said to be \textbf{nondeterministic}.
    }

\subsection{Functions}

    \newdef{Recursively enumerable set}{\index{set!enumerable}
        A set $S$ of natural numbers is said to be recursively (or \textbf{computably}) enumerable if there exists a partial recursive function $f$ such that $\dom(f)=S$.
    }

    \newdef{Uniformly recursively enumerable}{
        A sequence $\seq{S}$ of sets of natural numbers is said to be uniformly recursively enumerable if there exists a sequence $\seq{f}$ of uniformly partial recursive functions such that $\dom(f_n)=S_n$ for all $n\in\mathbb{N}$.
    }

    \todo{ADD (e.g.~computability, decidability)}

\section{Quantum computing}

    \todo{ADD (Shor, Grover, speed ups, ...)}

\section{Complexity theory}

    \newdef{Complexity class}{\index{complexity!class}\index{Savitch}
        The main complexity classes, together with their (partial) order relation are shown in the top row of \cref{fig:complexity_order}. These are defined as follows:
        \begin{itemize}
            \item$\mathbf{P}$: Problems that can be solved by a deterministic Turing machine in polynomial time.
            \item$\mathbf{NP}$: Problems for which the verification of their solutions lies in $\mathrm{P}$.
            \item$\mathbf{PSPACE}$: Problems that can be solved by a deterministic Turing machine in polynomial space.\footnote{The space equivalent of $\mathrm{NP}$ does not have to be included since $\mathrm{PSPACE}=\mathrm{NPSPACE}$ by Savitch's theorem.\index[author]{Savitch}}
        \end{itemize}
        Some of the inclusions are straightforward, e.g.~$\mathrm{P}\subseteq\mathrm{NP}$ and $\mathrm{P}\subseteq\mathrm{EXP}$. Moreover, the inclusion $\mathrm{P}\subseteq\mathrm{EXP}$ is strict. The inclusion $\mathrm{P}\subseteq\mathrm{PSPACE}$ can be understood by noting that a Turing machine cannot consume space more rapidly than the number of time steps it runs.
    }

    \begin{figure}[ht!]
        \centering
        \begin{tikzpicture}
            \node (P) at (0, 0) {$\mathrm{P}$};
            \node (NP) at (3, 0) {$\mathrm{NP}$};
            \node (PSPACE) at (6, 0) {$\mathrm{PSPACE}$};
            \node (EXP) at (9, 0) {$\mathrm{EXP}$\footnotemark};
            \node (BQP) at (0, -3) {$\mathrm{BQP}$};
            \draw[right hook->] (P) -- (BQP);
            \draw[right hook->] (P) -- (NP);
            \draw[right hook->] (NP) -- (PSPACE);
            \draw[right hook->] (PSPACE) -- (EXP);
            \draw[right hook->] (BQP) -- (PSPACE);
            \draw[->] (-1, 1) -- node[above, rotate = 90]{quantum} (-1, -4);
            \draw[->] (-1, 1) -- node[above]{complexity} (10, 1);
        \end{tikzpicture}
        \caption{Complexity order.}
        \label{fig:complexity_order}
    \end{figure}

    \footnotetext{$\mathrm{EXP}$ should be read as a further partial order consisting of $\mathrm{EXPTIME}\subseteq\mathrm{EXPSPACE}$.}
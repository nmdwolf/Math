\chapter{\difficult{Higher topos theory}}

    \minitoc

\section{Higher category theory}\label{cat:higher_category_theory}

    This section gives an introduction to the theory of higher categories, in particular $n$-categories for finite $n>0$. For the notion of $\infty$-categories, see \cref{section:infinity_categories}.

\subsection{\texorpdfstring{$n$-categories}{n-categories}}

    \newdef{$n$-category}{\index{n-category}
        A (strict) $n$-category consists of:
        \begin{itemize}
            \item objects (0-morphisms),
            \item 1-morphisms going between 0-morphisms,
            \item ...
            \item $n$-morphisms going between $(n-1)$-morphisms,
        \end{itemize}
        such that the composition of $k$-morphisms ($k\leq n$) is associative and satisfies the unit laws as required in an ordinary category. By generalizing this definition to arbitrary $n\in\mathbb{N}$, one can define the notion of a (strict) $\infty$-category.

        If one relaxes the associativity and unit laws up to higher coherent morphisms, one obtains the notion of a \textbf{weak $n$-category}. Explicit definitions for such categories have been constructed up to \textbf{tetracategories} $(n=4)$. However, this construction by \textit{Trimble} takes about 50 pages of diagrams.
    }
    \sremark{$n$-morphisms are also called \textbf{$n$-cells}. This makes their relation to topological spaces (and, in particular, simplicial spaces) more apparent.}

    \begin{example}
        The classical examples of a 1-category and 2-category are $\mathbf{Set}$ and $\mathbf{Cat}$, respectively.
    \end{example}

    \begin{property}[Composition in 2-categories]\index{Godement product}\index{interchange law}
        2-morphisms can be composed in two different ways:
        \begin{itemize}
            \item\textbf{Horizontal composition}:
                Consider two 2-morphisms $\alpha:f\Rightarrow g$ and $\beta:f'\Rightarrow g'$, where $f'\circ f$ and $g'\circ g$ are well-defined. These 2-morphisms can be composed as
                \begin{gather}
                    \beta\circ\alpha: f'\circ f\Rightarrow g'\circ g\,.
                \end{gather}
                This is sometimes called the \textbf{Godement product}.
            \item\textbf{Vertical composition}:
                Consider two 2-morphisms $\alpha:f\Rightarrow g$ and $\beta:g\Rightarrow h$, where $f,g$ and $h$ have the same domain and codomain. These 2-morphisms can be composed as
                \begin{gather}
                    \beta\cdot\alpha:f\Rightarrow h\,.
                \end{gather}
        \end{itemize}
        As a consistency condition, the horizontal and vertical composition are required to satisfy the following \textbf{interchange law}:
        \begin{gather}
            (\alpha\cdot\beta)\circ(\gamma\cdot\delta) = (\alpha\circ\gamma)\cdot(\beta\circ\delta).
        \end{gather}
    \end{property}
    \newdef{\texorpdfstring{$(n,r)$-category}{(n,r)-Category}}{
        A higher ($\infty$-)category for which
        \begin{itemize}
            \item all parallel $k$-morphisms with $k>n$ are equivalent and, hence, trivial.
            \item all $k$-morphisms with $k>r$ are invertible (or equivalences in the fully weak $\infty$-sense).
        \end{itemize}
    }

    \newdef{Weak inverse}{\index{weak!inverse}
        Let $\mathbf{C}$ be a 2-category. A 1-morphism $f:x\rightarrow y$ is weakly invertible if there exist a 1-morphism $g:y\rightarrow x$ and 2-isomorphisms $g\circ f\Rightarrow\mathbbm{1}_x$ and $f\circ g\Rightarrow\mathbbm{1}_y$.
    }

    At this point, it should be obvious that the definition of a unit-counit adjunction (\cref{cat:unit_counit_adjunction}) can be generalized to general 2-categories:
    \newdef{Adjunction in 2-category}{
        Let $\mathbf{C}$ be a 2-category. An adjunction in $\mathbf{C}$ is a pair of 1-morphisms $F:x\rightarrow y$ and $G:y\rightarrow x$ together with 2-morphisms $\varepsilon:F\circ G\Rightarrow\mathbbm{1}_y$ and $\eta:\mathbbm{1}_x\Rightarrow G\circ F$ that satisfy the zig-zag identities.
    }

    By looking at the defining relations of duals in a rigid monoidal category (\cref{section:duality}), it should be clear that these are in fact the same as the defining relations of the unit and counit of an adjunction. This is a consequence of the fact that a 2-category with a single object can be regarded as a (strict) monoidal category where the composition in the 2-category becomes the tensor product in the monoidal category. Similarly, adjoint 1-morphisms in the 2-category become duals in the monoidal category. This is formalized as follows.
    \begin{property}[Monoidal categories]\index{monoidal!category}\index{delooping!of monoidal categories}\label{cat:monoidal_or_2}
        Consider a monoidal category $(\mathbf{C},\otimes,\mathbf{1})$. From this monoidal category, one can construct the so-called \textbf{delooping} (bicategory) $\mathbf{BC}$ in the following way:
        \begin{itemize}
            \item There is a single object $\ast$.
            \item The 1-morphisms in $\mathbf{BC}$ are the objects in $\mathbf{C}$.
            \item The 2-morphisms in $\mathbf{BC}$ are the morphisms in $\mathbf{C}$.
            \item Horizontal composition in $\mathbf{BC}$ is the tensor product in $\mathbf{C}$.
            \item Vertical composition in $\mathbf{BC}$ is composition in $\mathbf{C}$.
        \end{itemize}
        Conversely, every 2-category with a single object comes from a monoidal category. Hence, the 2-category of (pointed) 2-categories with a single object and the 2-category of monoidal categories are equivalent. (This property and its generalizations are the content of the \textit{delooping hypothesis}.)

        In the same way one can deloop a braided monoidal category twice and find an identification with a one-object tricategory with one 1-morphism. However, this identification is not a trivial one as it makes use of the Eckmann--Hilton argument (\cref{cat:eckmann_hilton}) to identify different monoidal structures on this tricategory. (See also \cref{section:monoidal_n_cat}.)
    \end{property}

\subsection{\texorpdfstring{$n$-functors}{n-functors}}

    \newdef{2-functor}{\index{pseudo-!functor}\label{cat:pseudofunctor}
        A 2-functor $\func{F}{C}{D}$ (often called a \textbf{pseudofunctor}) is a morphism between bicategories. It consists of the following data:
        \begin{itemize}
            \item a function $F_0:\ob{C}\rightarrow\ob{D}$, and
            \item for every two objects $x,y\in\ob{C}$, a functor $F_{x,y}:\mathbf{C}(x,y)\rightarrow\mathbf{D}(Fx,Fy)$.
        \end{itemize}
        The function $F_0$ and the functors $F_{x,y}$ are also often denoted by $F$ by abuse of notation. This data is required to satisfy some coherence conditions. These are specified by the following data:
        \begin{enumerate}
            \item\textbf{Associator}: For every pair of composable 1-morphisms $f\circ g$ in $\mathrm{hom}(\mathbf{C})$, a 2-isomorphism $\gamma_{f,g}:Ff\circ Fg\Rightarrow F(f\circ g)$ such that for every triple of composable morphisms $f\circ g\circ h$ in $\mathrm{hom}(\mathbf{C})$, the following identity holds:
            \begin{gather}
                \gamma_{f\circ g,h}\circ(\gamma_{f,g}\cdot\mathbbm{1}_{Fh}) = \gamma_{f,g\circ h}\circ(\mathbbm{1}_{Ff}\cdot\gamma_{g,h}).
            \end{gather}
            \item\textbf{Unitor}: For every object $x\in\ob{C}$, a 2-isomorphism $\iota_x:\mathbbm{1}_{Fx}\Rightarrow F\mathbbm{1}_x$ such that for every morphism $f:x\rightarrow y$ in $\mathrm{hom}(\mathbf{C})$, the following identities hold:
            \begin{gather}
                \begin{aligned}
                    \iota_y\cdot\mathbbm{1}_{Ff} &= \gamma_{\mathbbm{1}_y,f}\\
                \mathbbm{1}_{Ff}\cdot\iota_x &= \gamma_{f,\mathbbm{1}_x}\,.
                \end{aligned}
            \end{gather}
        \end{enumerate}
        Note that to be completely formal, one should have inserted the unitors and associators of the bicategories $\mathbf{C},\mathbf{D}$.
    }
    \newdef{Lax natural transformation}{
        Consider two 2-functors $\func{F,G}{C}{D}$. A lax natural transformation $\eta:F\Rightarrow G$ consists of the following data:
        \begin{enumerate}
            \item For every object $x\in\ob{C}$, a 1-morphism $\eta_x:Fx\rightarrow Gx$.
            \item For every 1-morphism $f:x\rightarrow y$ in $\mathrm{hom}(\mathbf{C})$, a 2-morphism $\eta_f:Gf\circ\eta_x\Rightarrow\eta_y\circ Ff$ such that the $\eta_f$ are the components of a natural transformation $(\eta_x)^*\circ G\Rightarrow(\eta_y)_*\circ F$ and such that the assignment $f\mapsto\eta_f$ satisfies the `obvious' identity and composition axioms.
        \end{enumerate}
    }
    \begin{remark}\index{pseudo-!natural transformation}
        As usual in the context of higher category theory, one can speak of lax 2-functors if the associator and unitors are merely required to be 2-morphisms, and of strict 2-functors if these morphisms are required to be identities. If the natural transformations between morphism categories in the definition of a lax natural transformation are all isomorphisms, this is called a \textbf{pseudonatural transformation}. If the 1-morphisms $\eta$ are equivalences, they are called lax natural equivalences.
    \end{remark}

    \newdef{Modification}{\index{modification}\label{cat:modification}
        Consider two 2-functors $\func{F,G}{C}{D}$ and two parallel (lax) natural transformations $\alpha,\beta:F\Rightarrow G$. A modification $\mathfrak{m}:\alpha\Rrightarrow\beta$ maps every object $x\in\ob{C}$ to a 2-morphism $\mathfrak{m}_x:\alpha_x\Rightarrow\beta_x$ such that $\beta_f\circ(\mathbbm{1}_{Gf}\cdot\mathfrak{m}_x) = (\mathfrak{m}_y\cdot\mathbbm{1}_{Ff})\circ\alpha_f$.
    }
    This is generalized as follows.
    \newdef{Transfor}{\index{transfor}\label{cat:transfor}
        A $k$-transfor\footnote{This name was first introduced by~\citet{crans_localizations_1998}. A different name that is sometimes used is \textbf{$(n,k)$-transformation}, but this should not be confused with the natural transformations in the context of $(n,r)$-categories.} between two $n$-categories maps $j$-morphisms to $(j+k)$-morphisms (in a coherent way).
    }
    \begin{example}\index{perturbation}
        The definitions for operations in bicategories above lead us to the following `explicit' expressions for $k$-transfors (for small $k$):
        \begin{itemize}
            \item $k=0$: $n$-functors,
            \item $k=1$: ($n$-)natural transformations,
            \item $k=2$: modifications, and
            \item $k=3$: \textbf{perturbations}.
        \end{itemize}
    \end{example}

    The following definition generalizes the notion of essential surjectivity (\cref{cat:essentialy_surjective}) to higher category theory.
    \newdef{$n$-surjective functor}{\index{surjective}
        An $\infty$-functor $\func{F}{C}{D}$ is said to be $n$-surjective if for any two parallel $(n-1)$-morphisms $f,g$ in $\mathbf{C}$ and $n$-morphism $\alpha:Ff\rightarrow Fg$ in $\mathbf{D}$, there exists an $n$-morphism $\widetilde{\alpha}$ in $\mathbf{C}$ such that $F\widetilde{\alpha}\cong\alpha$.
    }

    \newdef{Indexed category}{\index{category!indexed}\label{cat:indexed_category}
        Consider a category $\mathbf{I}$. An $\mathbf{I}$-indexed category is a pseudofunctor $\mathbf{C}:\mathbf{I}^{\text{op}}\rightarrow\mathbf{Cat}$, i.e.~a 2-presheaf on $\mathbf{I}$. Indexed functors and natural transformations are defined analogously.
    }

    \Cref{cat:universal_morphism} can be generalized as follows.
    \newdef{Universal morphism}{\index{universal!morphism}\label{cat:higher_universal_morphism}
        Consider an $n$-functor $\func{F}{C}{D}$ and an object $x\in\ob{D}$. A universal morphism from $x$ to $F$ is a pair $(d,f:x\rightarrow Fd)$ such that:
        \begin{enumerate}
            \item Every morphism $x\rightarrow Fd'$ factors through $f$, and
            \item The precomposition $f^*:\mathbf{D}(Fd,Fd')\rightarrow\mathbf{D}(x,Fd')$ is fully faithful.
        \end{enumerate}
    }

\subsection{Higher (co)limits}\label{section:higher_limits}

    \newdef{Weighted 2-limit}{\index{limit!weighted}
        Consider 2-categories $\mathbf{I,C}$ together with 2-functors $\func{W}{I}{Cat}$ and $\func{F}{I}{C}$. By direct generalization of the ordinary definition of weighted limits (\cref{section:weighted_limits}), one says that $\wlim{W}F$ is the $W$-weighted (2-)limit of $F$ if there exists a pseudonatural equivalence
        \begin{gather}
            \mathbf{C}(x,\wlim{W}F)\cong\funccat{I}{Cat}\bigl(W,\mathbf{C}(x,F-)\bigr).
        \end{gather}
        By restricting to the 2-category of strict 2-categories, strict 2-functors and strict natural transformations the resulting notion of a weighted 2-limit coincides with that of an ordinary weighted limit enriched in $\mathbf{Cat}$ (since strict 2-categories are simply $\mathbf{Cat}$-enriched 1-categories.)
    }

    \todo{COMPLETE}

\section{\texorpdfstring{$\infty$-Categories}{Infinity-categories}}\label{section:infinity_categories}
\subsection{Simplicial approach}

    The first approach to $\infty$-category theory is the simplicial one. The motivation is \cref{model:horn_filler}, which relates the categorical structure to the existence of certain horn fillers. The generalization is then given by the notion of quasicategories (\cref{model:quasicategory}).

    \begin{theorem}[Lurie]\index{Lurie}\label{model:lurie_presentation}
        An $\infty$-category is presentable if and only if it is equivalent to  the coherent nerve of the fibrant-cofibrant subcategory of a combinatorial model category and, hence by Dugger's theorem~\ref{model:dugger}, can be presented by simplicial presheaves.
     \end{theorem}

\subsection{Quasicategories}

    \newdef{Overcategory}{
        Consider a simplicial morphism $F:X\rightarrow\mathbf{C}$ where $\mathbf{C}$ is a quasicategory. The overcategory $\mathbf{C}_{/F}$, generalizing the comma category $\Delta\downarrow F$, is characterized by a natural isomorphism
        \begin{gather}
            \hom_\mathbf{sSet}(Y,\mathbf{C}_{/F})\cong\hom_{X_{/\mathbf{sSet}}}(\iota_Y,F)\,,
        \end{gather}
        where $\iota_Y:X\hookrightarrow Y\star X$ is the join inclusion.
    }
    \begin{remark}[1-categorical interpretation]
        Consider the case where $X$ is the simplicial nerve of the one-object groupoid $\{\ast\}$. The functor $F$ then singles out an object $c\in\ob{C}$. Since the image of $F$ is 0-truncated, morphisms on the right-hand side are also 0-truncated.
        
        \todo{COMPLETE}
    \end{remark}

    \newdef{Terminal object}{
        An object $1\in\ob{C}$ such that
        \begin{enumerate}
            \item The projection $\mathbf{C}_{/1}\rightarrow\mathbf{C}$ is an acyclic fibration, and
            \item for every object $d\in\ob{C}$ the right hom-Kan complex $\hom^R(d,1):=\mathbf{C}_{/1}\times_\mathbf{C}\{d\}$ is contractible.
        \end{enumerate}
    }

    \Cref{cat:limit} can be generalized as follows.
    \newdef{Limit}{
        Consider an $(\infty,1)$-functor $\func{F}{C}{D}$ of quasicategories. The limit of $F$ is defined as the terminal object in the overcategory $\mathbf{C}_{/F}$.
    }

\subsection{Categorical notions}

    \newdef{Truncated object}{
        An $\infty$-groupoid is said to be $n$-truncated for $n\in\mathbb{N}$ if it is an $n$-groupoid. An object $x\in\ob{C}$ of an $(\infty,1)$-category is said to be $n$-truncated if for all objects $y\in\ob{C}$, the hom-groupoid $\mathbf{C}(y,x)$ is $n$-truncated.
    }

    \newdef{Monomorphism}{\index{monomorphism}
        A morphism $f:x\rightarrow y$ in an $(\infty,1)$-category $\mathbf{C}$ that is $(-1)$-truncated as an object of the slice category $\mathbf{C}_{/y}$. Equivalently, the projection $\mathbf{C}_{/f}\rightarrow\mathbf{C}_{/y}$ is fully faithful.
    }

    \newdef{Epimorphism}{\index{epimorphism}
        A morphism $f:x\rightarrow y$ in an $(\infty,1)$-category $\mathbf{C}$ such that the induced morphism $\mathbf{C}(f,z):\mathbf{C}(y,z)\rightarrow(x,z)$ is monic for all objects $z\in\ob{C}$.
    }

    \section{Stacks}\index{stack}\label{section:stacks}
    \subsection{2-sheaves}
    
        An important subject, especially in the context of gauge theories in physics, is that of groupoid-valued (pre)sheaves. To this end, sites are generalized to higher category theory.
        \newdef{2-presheaf}{\index{presheaf}\index{pre-!stack}
            Consider a 2-category $\mathbf{C}$. A 2-presheaf on $\mathbf{C}$ is a pseudofunctor $\cfunc{F}{C}{Cat}$. When $\mathbf{C}$ is the categorification of a 1-category, i.e.~when it has discrete hom-categories, 2-presheaves are better known as \textbf{prestacks}.
        }
        \newdef{2-coverage}{\index{coverage}\index{site}
            This is virtually the same as an ordinary coverage (\cref{topos:coverage}), but factorization is only required to exist up to an isomorphism. A 2-category equipped with a 2-coverage is called a \textbf{2-site}.
    
            As for 1-sites, every coverage generates a unique sieve. It is the full subcategory on those morphisms that factor through a covering map in the given coverage (again up to isomorphism).
        }
    
        As in the case of ordinary categories (\cref{topos:local_object_sheaf}), one can define 2-sheaves through a descent condition.
        \newdef{2-sheaf}{\label{topos:2_sheaf}
            A 2-presheaf $\cfunc{F}{C}{Cat}$ on a 2-site $(\mathbf{C},J)$ is said to be a 2-sheaf with respect to $J$ if for all sieves $S\in J$ the following functor is an equivalence:
            \begin{gather}
                Fc\cong\mathbf{Psh}_2(\mathcal{Y}c,F)\rightarrow\mathbf{Psh}_2(S,F)\,,
            \end{gather}
            where the fist equivalence is just the 2-Yoneda lemma.
        }
        \begin{remark}
            It should be noted that 2-(pre)sheaves can also be defined on ordinary (1-)sites. Sieves, regarded as subfunctors of the Yoneda embedding, take values in $\mathbf{Set}$. By composing these with the embedding $\mathbf{Set}\hookrightarrow\mathbf{Cat}$ of sets as (discrete) categories, one obtains 2-subfunctors of the 2-Yoneda embedding. Often 2-sheaves over 1-sites are called \textbf{stacks} (although this terminology is also used for general 2-sites).
        \end{remark}
    
        \newdef{Prestack of groupoids}{
            Consider a category $\mathbf{C}$. A prestack of groupoids on $\mathbf{C}$ is a $\mathbf{Grpd}$-valued prestack on $\mathbf{C}$.
    
            The category of (groupoid-valued) prestacks becomes $\mathbf{Grpd}$-enriched if one takes the hom-category between two prestacks $F,G$ to consist of the following data:
            \begin{itemize}
                \item\textbf{Objects}: Natural transformations $\alpha:F\Rightarrow G$ (note that the components are themselves functors).
                \item\textbf{Morphisms}: `strict modifications' in the sense that they map objects in $\mathbf{C}$ to natural transformations satisfying the whiskering condition (see also Definition \ref{cat:modification})
                \begin{gather}
                    \mathbbm{1}_{Ff}\cdot\mathfrak{m}_b = \mathfrak{m}_a\cdot\mathbbm{1}_{Gf}\,.
                \end{gather}
            \end{itemize}
        }
    
        For ordinary sites and presheaves, descent was defined in terms of matching families. Since presheaves are now taking values in a 2-category, the matching families are a bit more complex. However, this structure is already familiar from differential geometry and algebraic topology, where it is known under the name of the \textit{\v{C}ech nerve}.
        \newdef{\v{C}ech groupoid}{\index{Cech!groupoid}
            Consider a site $(\mathbf{C},J)$. To every covering family $\mathcal{U}=\{f_i:x_i\rightarrow x\}_{i\in I}$, one can assign an internal groupoid in presheaves $C(\mathcal{U})$ consisting of the following data:
            \begin{itemize}
                \item\textbf{Objects}: $\bigsqcup_i\mathcal{Y}x_i$, and
                \item\textbf{Morphisms}: $\bigsqcup_{i,j}\mathcal{Y}x_i\times_{\mathcal{Y}x}\mathcal{Y}x_j$.
            \end{itemize}
            This is equivalent to the ($\mathbf{Grpd}$-valued) presheaf that assigns to every object $y\in\ob{C}$ the groupoid consisting of the following data:
            \begin{itemize}
                \item\textbf{Objects}: Pairs $(i,g_i:y\rightarrow x_i)$ where $x_i\in\mathcal{U}$, and
                \item\textbf{Morphisms}: A unique arrow $(i,g_i)\rightarrow(j,g_j)$ if and only if $f_i\circ g_i = f_j\circ g_j$.
            \end{itemize}
        }
        Comparing the definition of morphisms in the \v{C}ech groupoid to the condition for matching families in \cref{topos:matching_family}, shows that one could presume that the \v{C}ech groupoid is related to the matching families. This intuition is indeed correct as explained by the following property.
        \begin{property}[Matching families]\label{topos:cech_matching_families}
            Any ordinary presheaf $F$ can be considered to be $\mathbf{Grpd}$-valued by postcomposing with the embedding $\mathbf{Set}\hookrightarrow\mathbf{Grpd}$. For any covering family $\mathcal{U}$, there exists an isomorphism
            \begin{gather}
                \cfunccat{C}{Grpd}\bigl(C(\mathcal{U}),F\bigr)\cong\mathbf{Psh}_2(\mathcal{U},F)\,.
            \end{gather}
            Because the \v{C}ech groupoid (co)represents a descent object, it is sometimes called a \textbf{codescent object}.
        \end{property}
        It is exactly this (co)descent property of the \v{C}ech groupoid that will be used in \cref{chapter:hdg} to define (higher) smooth groupoids. Readers with some experience in algebraic topology will also notice that the \v{C}ech groupoid only contains the first degrees of the \v{C}ech complex. The full \v{C}ech complex can be obtained from the following construction.
        \newdef{\v{C}ech nerve}{\index{Cech!nerve}
            Consider a morphism $f:y\rightarrow x$ in a category $\mathbf{C}$. The \v{C}ech nerve $C_\bullet(f)$ is the simplicial object (\cref{model:simplicial_object}) that contains, in degree $k\in\mathbb{N}$, the $(k+1)$-fold pullback of $f$ along itself. For a covering family $\mathcal{U}\equiv\{f_i:x_i\rightarrow x\}$, the \v{C}ech nerve is defined as
            \begin{gather}
                C_\bullet(\mathcal{U}):=C_\bullet\left(\bigsqcup_ix_i\rightarrow x\right)\,.
            \end{gather}
        }
        For $\infty$-sheaves, the full \v{C}ech nerve will be used. However, for 2-sheaves and, in particular, stacks, only its 3-coskeleton is necessary. The extra information will encode the \textit{cocycle condition}~\eqref{bundle:G_cocycle_condition}, well-known from the study of \textit{fibre bundles}.
    
    \subsection{Stacks on a 1-site}
    
        For the definition of stacks, one needs the notions of fibred categories or, equivalently, pseudofunctors as defined in \cref{section:fibred_categories}. The definitions are recalled here.
        \begin{quote}
            Consider a functor $\func{\Pi}{A}{B}$. A morphism $f$ in $\mathbf{A}$ is said to be $\Pi$-Cartesian if, for every morphism $\varphi$ in $\mathbf{A}$ and factorization of $\Pi\varphi$ through $\Pi f$ in $\mathbf{B}$, there exists a unique factorization of $\varphi$ through $f$. $f$ is called the inverse image of $\Pi f$.
    
            A fibred category consists of a functor $\func{\Pi}{A}{B}$ such that for each morphism $f:c\rightarrow d$ in $\mathbf{B}$ with $d\in\im(\Pi)$ and each lift $y\in\mathbf{A}_d$ there exists at least one inverse image in $(\widetilde{f}:x\rightarrow y)\in\mathbf{A}$ of $f$. By the Grothendieck construction every fibred category gives rise to a pseudofunctor $\cfunc{F}{B}{Cat}$ by sending objects to their fibres under $\Pi$ and sending morphisms $f$ to their pullback functors $f^*$.
        \end{quote}
    
        \newdef{Descent datum}{\index{descent}
            Consider a category $\mathbf{C}$ with a covering family $\mathcal{U}\equiv\{f_i:x_i\rightarrow x\}$ and a pseudofunctor $\cfunc{F}{C}{Cat}$. The projections associated to the pullback $x_i\cap x_j:=x_i\times_xx_j$ will be denoted by $\pi_i$ and $\pi_j$, respectively (and analogously for iterated pullbacks). A descent datum for $\mathcal{U}$ with respect to $F$ is a pair of families $(\{g_i\},\{f_{ij}\})_{i,j\in I}$, where $\{g_i\}$ is a matching family for $\mathcal{U}$ with respect to $F$ and every $f_{ij}$ is an isomorphism $\pi_i^*x_i\cong \pi_j^*x_j$. This data is required to satisify the following \textbf{cocycle condition}:
            \begin{gather}
                \pi_{ik}^*f_{ik} = \pi_{ij}^*f_{ij}\circ\pi_{jk}^*f_{jk}\,.
            \end{gather}
            Morphisms $(\{g_i\},\{f_{ij}\})\rightarrow(\{g'_i\},\{f'_{ij}\})$ between descent data are families of morphisms $\{\phi_i:g_i\rightarrow g'_i\}$ that satisfy
            \begin{gather}
                \pi_i^*\phi_i\circ f_{ij} = f'_{ij}\circ\pi_j^*\phi_j\,.
            \end{gather}
            The category of descent data for $\mathcal{U}$ with respect to $F$ will be denoted by $\mathrm{Descent}(\mathcal{U},F)$.
        }
        \begin{construct}
            Consider an object $\xi$ in $Fx$. From this object, one can construct a descent datum as follows. The morphisms $g_i$ are the pullbacks $f_i^*\xi$ and the isomorphisms $f_{ij}:\pi_i^*f_i^*\xi\cong\pi_j^*f_j^*\xi$ are obtained from the fact that both these objects are (Cartesian) pullbacks of the same morphisms. Arrows in $Fx$ induce morphisms of descent data by (Cartesian) pullbacks along the covering maps. This construction defines a functor $Fx\rightarrow\mathrm{Descent}(\mathcal{U},F)$. It can be shown that this construction is independent of a choice of cleavage up to equivalence.
        \end{construct}
    
        \newdef{Stack}{\index{pre-!stack}\index{stack}
            Consider a fibred category $F$ over a site $(\mathbf{C},J)$.
            \begin{itemize}
                \item $F$ is called a \textbf{separated prestack} if for each covering family $\mathcal{U}$ on $x\in\ob{C}$, the functor $Fx\rightarrow\mathrm{Descent}(\mathcal{U},F)$ is fully faithful.
                \item $F$ is called a \textbf{stack} if for each covering family $\mathcal{U}$ on $x\in\ob{C}$ the functor $Fx\rightarrow\mathrm{Descent}(\mathcal{U},F)$ is an equivalence.
            \end{itemize}
            This is a generalization of the descent condition in \cref{topos:local_object_sheaf}. This can be seen by observing that $\mathrm{Descent}(\mathcal{U},F)\cong\mathbf{Psh}_2(S(\mathcal{U}),F)$, where $S(\mathcal{U})$ is the sieve generated by $\mathcal{U}$ regarded as a fibred category. When $F$ is fibred over groupoids, it is called a \textbf{stack of groupoids}. This forms the category $\mathbf{Sh}_{(2,1)}(\mathbf{C})$ of $(2,1)$-sheaves. In fact, it is this subcategory that is usually meant when considering stacks.
        }
    
        A more conceptual (although completely equivalent) generalization from (1-)sheaves to 2-sheaves can be obtained by starting from \cref{topos:cech_matching_families}. There, it was shown that matching families for (1-)presheaves can be obtained as natural transformations from the \v{C}ech groupoid.
        \begin{property}[Descent data and \v{C}ech nerve]
            Let $C(\mathcal{U})$ denote the 3-coskeleton of the \v{C}ech nerve $C_\bullet(\mathcal{U})$. Pseudonatural transformations $C(\mathcal{U})\Rightarrow F$ can be shown to be equivalent to tuples $(c,\{c_i\},\{c_{ij}\},\{c_{ijk}\})$, with $c_i\in Fx_i$, that fit into cubes lying in the image of $C_2(\mathcal{U})$ in which all edges consist of Cartesian morphisms. Arrows between such cubes are given by arrows between the vertices that make the `obvious' diagrams commute.
    
            By comparing these cubes to the previous definition of descent data, one obtains the following equivalence:
            \begin{gather}
                \mathrm{Descent}(\mathcal{U},F)\cong\cfunccat{C}{Cat}\bigl(C(\mathcal{U}),F\bigr)\,.
            \end{gather}
            \todo{FINISH THIS}
        \end{property}
    
        \begin{remark}[1-sheaves]
            Although most of the above seems very abstract and complex compared to ordinary sheaves, it is not quite so. In fact, when restricting to pseudofunctors of the form $\op{\mathbf{C}}\rightarrow\mathbf{Set}$, where the embedding $\mathbf{Set}\hookrightarrow\mathbf{Cat}$ sends sets to discrete categories, one obtains ordinary sheaves as a subcategory of stacks. For example, by the equivalence between pseudofunctors and Grothendieck fibrations, it is known that the Cartesian pullbacks $f^*$ are in fact just the images of $f$ under the pseudofunctor $F$. This way, the condition $\pi_1^*c_i\cong\pi^*_2c_j$ can be rewritten as $Ff'_i(c_i)=Ff'_j(c_j)$, which is nothing but the matching family condition~\eqref{topos:matching_family_condition}.
        \end{remark}
    
    \section{Higher topos theory}
    
        In this section, the notion of topos is generalized from ordinary category theory to higher category theory. In particular, $\infty$-sheaves will be defined. This will require a suitable foundation for $\infty$-category theory. To this end, the language of (simplicial) model categories as introduced in \cref{chapter:model_theory} will be used.
    
        \newdef{\texorpdfstring{$\infty$-groupoid}{Infinity-groupoid}}{\index{groupoid}\label{topos:infty_groupoid}
            Objects of the full simplicial subcategory of $\mathbf{sSet}_\text{Quillen}$ on Kan complexes. From \cref{model:horn_filler}, it is immediately clear how this generalizes the definition of ordinary groupoids. For groupoids one needs unique horn fillers (composition in ordinary categories is unique), while for $\infty$-groupoids this is allowed to be unique up to higher coherence.
        }
        \newdef{\texorpdfstring{$(\infty,1)$-category}{(Infinity,1)-category}}{\index{category}
            An $\infty\mathbf{Grpd}$-enriched category or, equivalently, a simplicially enriched category for which all hom-objects are Kan complexes. The functor category between $(\infty,1)$-categories is defined through the (simplicial) nerve and realization functors (\cref{model:nerve}):
            \begin{gather}
                \funccat{C}{D} := |\mathbf{sSet}(N\mathbf{C},N\mathbf{D})|\,.
            \end{gather}
        }
    
        \begin{property}[\v{C}ech model structure]\index{Cech!model structure}\label{topos:cech_model_structure}
            For any small category $\mathbf{C}$, the $\infty$-category of $\infty\mathbf{Grpd}$-valued $\infty$-sheaves can be represented by the category $\cfunccat{C}{sSet}$ of simplicial presheaves on $\mathbf{C}$ by a theorem of \textit{Lurie} (\cref{model:lurie_presentation}), i.e.~there exists an $\infty$-equivalence between $\mathbf{Sh}_{(\infty,1)}(\mathbf{C})$ and the full subcategory on fibrant-cofibrant objects of the (left Bousfield) localization of $\cfunccat{C}{sSet}$ at the \v{C}ech nerve projections. The resulting model structure is called the \textbf{\v{C}ech model structure}.
    
            A presheaf $X$ is fibrant in this model structure if the map
            \begin{gather}
                \hom(M,X)\rightarrow\hom\bigl(\mathcal{C}(\mathcal{U}),X\bigr)
            \end{gather}
            is a weak equivalence for all open covers $\mathcal{U}$, i.e.~exactly if $X$ satisfies the descent condition and, hence, is an $\infty$-stack.
        \end{property}
    
        The most straightforward definition of an $\infty$-sheaf generalizes \cref{topos:local_object_sheaf}.
        \newdef{\texorpdfstring{$\infty$-sheaf}{Infinity-sheaf}}{\index{sheaf}
            Consider an $\infty$-site $(\mathbf{C},J)$ and let $S$ denote the collection of monomorphisms in $\mathbf{Psh}_\infty(\mathbf{C})$ induced by the covering sieves. An $\infty$-presheaf on $\mathbf{C}$ is called a $J$-sheaf if it is $S$-local. A presheaf with values in an $\infty$-category $\mathbf{D}$ is called a sheaf if the representable presheaf $\mathbf{D}(x,F-)$ is a $J$-sheaf for all $x\in\ob{D}$.
    
            In terms of the \v{C}ech nerve $\mathcal{C}$, the descent condition can be written as follows:
            \begin{gather}
                Fx\simeq\mathbf{Psh}_\infty\bigl(\mathcal{C}(\mathcal{U}),F\bigr)
            \end{gather}
            for all covers $\mathcal{U}$ of $x$, where $\simeq$ denotes a weak equivalence.
        }
        \newdef{\texorpdfstring{$\infty$-stack}{Infinity-stack}}{\index{stack}
            An $(\infty,1)$-sheaf taking values in $\infty\mathbf{Grpd}$.
        }
    
        \Cref{topos:global_sections} can be generalized as follows.
        \begin{property}
            For every $\infty$-topos $\mathbf{H}$, there exists a geometric morphism $(\mathrm{Disc}\dashv\Gamma):\mathbf{H}\leftrightarrows\infty\mathbf{Grpd}$. Any morphism into a discrete object $\mathrm{Disc}(X)$ is constant.
    
            The left adjoint is sometimes called the \textbf{discrete object functor}. This terminology stems from the case of the forgetful functor $\func{\Gamma}{Top}{Set}$, where the (fully faithful) left adjoint equips a set with the discrete topology.
        \end{property}
        \begin{example}[Sheaves on manifolds]
            One of the archetypal examples of $\infty$-topoi is the topos of sheaves over smooth manifolds. By the Yoneda embedding, one can regard a manifold as a sheaf and the global sections functor maps this representable sheaf to the manifold itself: $\Gamma(M)=M$. For a Lie group, one can construct the classifying stack $\mathcal{B}G$. The global sections functor maps this stack to the delooping groupoid $\mathbf{B}G$.
        \end{example}
    
        \newdef{Mapping stack}{\label{topos:mapping_stack}
            Consider two $\infty$-stacks $X,Y\in\mathbf{Sh}_{(\infty,1)}(\mathbf{C})$. The mapping stack is defined as follows:
            \begin{gather}
                [X,Y](U):=\mathbf{Sh}_{(\infty,1)}(\mathbf{C})(X\times U,Y)\,,
            \end{gather}
            where on the right-hand side, $U$ denotes the representable $\infty$-stack.
        }
    
        \todo{FINISH (PERHAPS MOVE infinity-CATEGORY STUFF TO CHAPTER ON MODEL THEORY)}
    
    \section{Cohomology}\index{cohomology}
    
        In this section, cohomology will be generalized to the $\infty$-categorical setting.
    
        First, take a topological space $X$ and an $\infty$-groupoid $G$. Geometric realization (\cref{model:geometric_realization}) gives an equivalence $\infty\mathbf{Grpd}\cong\mathbf{Top}$ and, therefore, one can define the intrinsic cohomology of $X$ with coefficients in $G$ as follows:
        \begin{gather}
            H(X;G) := \pi_0\mathbf{Top}(X,|G|)\,.
        \end{gather}
        $X$ can also be identified with its petit ($\infty$-)topos $\mathbf{Sh}_{(\infty,1)}(X)$, in which $X$ sits as the terminal object. From this point of view, the intrinsic cohomology of $X$ with coefficients in $G$ is
        \begin{gather}
            \overline{H}(X;G) := \pi_0\mathbf{Sh}_{(\infty,1)}(X)(X,\mathrm{LConst}\,G)\cong\pi_0\circ\Gamma\circ\mathrm{LConst}(G)\,.
        \end{gather}
        This is the \textbf{cohomology with constant coefficients} of $X$ with coefficients in $G$. If $X$ is paracompact, the two cohomologies coincide: $H(X;G)\cong\overline{H}(X;G)$.
    
        Now, it is time to pass to general cohomology.
        \newdef{Intrinsic cohomology}{\index{cohomology}\label{topos:cohomology}
            Consider a $(\infty,1)$-category $\mathbf{H}$. For every two objects $X,A\in\mathbf{H}$, the hom-space $\mathbf{H}(X,A)$ is an $\infty$-groupoid. Define the following notions:
            \begin{itemize}
                \item The objects in $\mathbf{H}(X,A)$ are called \textbf{cocycles}.
                \item The morphism in $\mathbf{H}(X,A)$ are called \textbf{coboundaries}.
                \item The set of connected components
                \begin{gather}
                    H(X;A):=\pi_0\mathbf{H}(X,A)=\hom_\mathbf{Ho(H)}(X,A)\,,
                \end{gather}
                where $\mathbf{Ho(H)}$ is the homotopy category \ref{model:homotopy_category_2} of $\mathbf{H}$, is called the intrinsic cohomology of $X$ with coefficients in $A$.
            \end{itemize}
            If the object $A$ admits an $n$-delooping $\mathbf{B}^nA$, the $n^{\text{th}}$ cohomology group of $X$ is defined as
            \begin{gather}
                H^n(X;A):=H(X;\mathbf{B}^nA)\,.
            \end{gather}
        }
    
        \begin{example}[Singular cohomology]
            Consider a topological space $X$. For every group $G$ one can define the first delooping (\cref{cat:group_delooping}), so one can also define the zeroth and first cohomology groups $H^{0,1}(X;G)$. Only when $G$ is Abelian do higher deloopings exists (in fact, if $G$ is Abelian all higher deloopings exist), and so in this case higher cohomology groups $H^{\geq 2}(X;G)$ can be defined. It can be shown that these coincide with the singular cohomology groups of $X$.
        \end{example}
    
        \begin{example}[Group cohomology]
            Consider a (discrete) group $G$ together with its delooping groupoid $\mathbf{B}G$. The cohomology of a group with coefficients in an Abelian group $A$ (\cref{section:group_cohomology}) is given by the intrinsic cohomology of $\infty\mathbf{Grpd}$ of the delooping groupoids:
            \begin{gather}
                H(G;A)\cong\pi_0\infty\mathbf{Grpd}(\mathbf{B}G,\mathbf{B}A)\,.
            \end{gather}
        \end{example}
    
        By replacing the topos $\mathbf{H}$ by a slice topos $\mathbf{H}_{/X}$, one obtains twisted cohomology.
        \newdef{Twisted cohomology}{\index{cohomology!twisted}
            Consider a $(\infty,1)$-topos $\mathbf{H}$ with some object $X\in\ob{H}$. The mapping space $\mathbf{H}(X,A)$, the cocycles of $X$ with coefficients in $A$, is easily seen to be isomorphic to the mapping space $\mathbf{H}_{/X}(X,X\times A)$, where the second argument is equipped with the canonical projection morphism. Morphisms in this space are just sections of the trivial $A$-$\infty$-bundle over $X$. General twisted cohomology can then be defined as the space of sections of an arbitrary $A$-$\infty$-bundle over $X$.
    
            By passing to classifying morphisms of bundles, one obtains the twist $\chi:X\rightarrow\mathbf{BAut}(A)$ and the universal bundle $\rho_A:A/\!\!/\mathbf{Aut}(A)\rightarrow\mathbf{BAut}(A)$. $\chi$-twisted cohomology is then given by (the connected components of) the following mapping space:
            \begin{gather}
                \mathbf{H}_{/\mathbf{BAut}(A)}(\chi,\rho_A)\,.
            \end{gather}
        }
\chapter{Number Theory}

    The section on modular forms and $L$-functions is based on~\citet{sutherland_modular_2017,mustata_zeta_2011}.

    \minitoc

\section{Adic numbers}

    This section will make use of the content of \cref{section:stone_spaces}.

\subsection{Finite fields}

    \newdef{Finite field}{\index{field!finite}\index{Galois|seealso{field, finite}}\index{order!of finite field}\label{number:finite_field}
        The finite field (or \textbf{Galois field}) $\mathbb{F}_p$ is the field with a finite number of elements $p\in\mathbb{N}$. The number $p$ is called the \textbf{order} of $\mathbb{F}_p$.
    }
    \begin{property}[Prime order]
        If $p\in\mathbb{N}$ is prime, then $\mathbb{F}_p\cong\mathbb{Z}/p\mathbb{Z}$.
    \end{property}

\subsection{Rational numbers}

    \newdef{$p$-adic numbers}{\index{adic numbers}
        Consider a prime number $p\in\mathbb{N}$. For every $m,n\in\mathbb{N}$, one has that $\mathbb{Z}/p^n\mathbb{Z}\subseteq\mathbb{Z}/p^m\mathbb{Z}$ whenever $m\leq n$. The group $\mathbb{Z}_p$ of \textbf{$p$-adic integers} is given by the profinite group (\cref{topology:profinite_group}) obtained from this system of inclusions. The $p$-adic integers are of the form
        \begin{gather}
            \sum_{n=0}^{+\infty}a_np^n\,,
        \end{gather}
        with integers $0\leq a_n<p$ for all $n\in\mathbb{N}$.

        Just like $\mathbb{Q}$ is the field of fractions (\cref{algebra:fraction_field}) of $\mathbb{Z}$ as in \cref{algebra:integers_rationals}, the $p$-adic numbers $\mathbb{Q}_p$ can be obtained as the field of fractions of $\mathbb{Z}_p$. The $p$-adic numbers are of the form
        \begin{gather}
            \sum_{n=-N}^{+\infty}a_np^n\,,
        \end{gather}
        with $N\in\mathbb{N}$ and integers $0\leq a_n<p$ for all $n\in\mathbb{N}$.
    }

    An alternative definition makes use of the notion of valuations (\cref{algebra:valuation}).
    \newadef{$p$-adic numbers}{\index{absolute value}
        Consider a prime number $p\in\mathbb{N}$. The $p$-adic valuation of an integer $z\in\mathbb{Z}$ is defined as follows:
        \begin{gather}
            \nu_p(z) :=
            \begin{cases}
                \max\{n\in\mathbb{N}:p^n\mid z\}&\cif z\neq0\,,\\
                +\infty&\cif z=0\,.
            \end{cases}
        \end{gather}
        This valuation extends to the rational numbers by taking
        \begin{gather}
            \nu_p\left(\frac{a}{b}\right) := \nu_p(a)-\nu_p(b)\,.
        \end{gather}
        In turn, the $p$-adic valuation also induces an absolute value on $\mathbb{Z}$ (and on $\mathbb{Q}$) as in \cref{algebra:nonarchimedean_valuation}, given by
        \begin{gather}
            |z|_p := p^{-\nu_p(z)}\,.
        \end{gather}
        The metric completions of $\mathbb{Z}$ and $\mathbb{Q}$ by these absolute values are the rings of $p$-adic integers and numbers, respectively.
    }
    \begin{remark}
        The $p$-adic valuation of a rational number $q\in\mathbb{Q}$ can also be defined as the (unique) integer $k\in\mathbb{Z}$ such that
        \begin{gather}
            q = p^k\frac{m}{n}
        \end{gather}
        for some integers $m,n\in\mathbb{Z}$ such that $p^k$, $m$ and $n$ are all coprime.
    \end{remark}

    \begin{property}\label{number:rational_adic_integer}
        If $b\bmod p\neq0$, then $\tfrac{a}{b}\in\mathbb{Z}_p$ for all $a,b\in\mathbb{Z}$ and $p\in\mathbb{N}$.
        
        The construction of the $p$-adic representation of a rational number $q\in\mathbb{Q}$ proceeds similar to the construction of the usual decimal representation, i.e.~through long division:
        \begin{enumerate}
            \item Find the unique coprime integers satisfying $q=\frac{c}{d}p^k$ and, hence, $\nu_p(q)=k$.
            \item $k-1$ zeros have to be added to the $p$-adic representation.
            \item Then, find the unique integers satisfying $q=ap^k+r$ with $0\leq a<p$ and $\nu_p(r)>k$.
            \item Then the next coefficients is $a$ and the algorithm can be restarted with $q:=r$.
        \end{enumerate}
    \end{property}

    \begin{theorem}[Ostrowski]\index{Ostrowski}
        The only nontrivial absolute values on $\mathbb{Q}$ are either the ordinary absolute value $|\cdot|$ or the $p$-adic absolute values $|\cdot|_p$.

        More generally, consider a complete field $\mathfrak{K}$ with respect to an absolute value. Then, either the absolute value is Archimedean and $\mathfrak{K}\cong\mathbb{R},\mathbb{C}$, or the absolute value is non-Archimedean.
    \end{theorem}

    \newdef{Profinite integers}{\index{integer!profinite}
        Similar to the construction of the $p$-adic integers, one can also construct the profinite completion (\cref{topology:profinite_completion}) of $\mathbb{Z}$. Instead of taking the inverse limit over the integers modulo a prime power, one simply takes the inverse limit over all finite cyclic groups:
        \begin{gather}
            \widehat{\mathbb{Z}} := \varprojlim_{n\in\mathbb{N}}\mathbb{Z}/n\mathbb{Z}\,.
        \end{gather}
        It can be shown that this is equivalent to taking the product of all $p$-adic integers:
        \begin{gather}
            \widehat{\mathbb{Z}}\cong\prod_{p\text{ prime}}\mathbb{Z}_p\,.
        \end{gather}
    }

    \newdef{Ad\`eles}{\index{ad\`ele}
        The ring of integral ad\`eles is defined as the product
        \begin{gather}
            \label{number:integral_adeles}
            \mathbb{A}_{\mathbb{R}} := \mathbb{R}\times\widehat{\mathbb{Z}}\,.
        \end{gather}
        To obtain the proper ring of ad\`eles, the above ring is rationalized:
        \begin{gather}
            \mathbb{A}_{\mathbb{Q}} := \mathbb{Q}\otimes_{\mathbb{Z}}\mathbb{A}_{\mathbb{R}}\,.
        \end{gather}
    }

    \newdef{Id\`eles}{\index{id\`ele}
        The group of units $\mathbb{I}_{\mathbb{Q}}:=\mathbb{A}_{\mathbb{Q}}^\times$.
    }

\subsection{Number and global fields}

    The product expression from the previous section can be reinterpreted by noting that the rationalization of the integral ad\`eles can be written as a \textbf{restricted product}.\index{product!restricted} Recall \cref{cat:product} of products. Given an indexing set $S$ and a collection of morphisms $\{\iota_s:X_s\rightarrow Y_s\}_{s\in S}$, the restricted product $\prod'_{s\in S}Y_s$ is given by the subset of the ordinary product $\prod_{s\in S}Y_s$ where all but a finite number of elements lie in the image of the morphisms $\iota_s$.

    Now, by the \textbf{prime factorization theorem} and \cref{number:rational_adic_integer}, every element of the form
    \begin{gather}
        \frac{a}{b}\otimes c_p
    \end{gather}
    with $a,b\in\mathbb{Z}$ and $c_p\in\mathbb{Z}_p$ will either be an element of $\mathbb{Z}_p$ if $b\bmod p\neq0$ or an element of $\mathbb{Q}_p$ if $b\bmod p=0$. Since there are only a finite number of prime factors of $b$, this shows that
    \begin{gather}
        \mathbb{A}_{\mathbb{Q}}\cong\mathbb{R}\times\prod'_{p\text{ prime}}\mathbb{Q}_p\,.
    \end{gather}
    Moreover, taking $\mathbb{Q}_\infty:=\mathbb{R}$ and recalling Ostrowski's theorem, one has
    \begin{gather}
        \mathbb{A}_{\mathbb{Q}}\cong\prod'_{p\in\mathrm{places}(\mathbb{Z})}\mathbb{Q}_p\,.
    \end{gather}

    Now, consider a number field $\mathfrak{K}$ (\cref{algebra:number_field}). Consider the following notations:
    \begin{itemize}
        \item $\mathcal{O}$ denotes the ring of algebraic integers (\cref{algebra:algebraic_integer}) of $\mathfrak{K}$.
        \item $P$ denotes the set of places of $\mathfrak{K}$ with $S\subset P$ the Archimedean (or infinite) places.
        \item For every place $\nu\in P$, $\mathfrak{K}_\nu$ denotes the completion at $\nu$.
    \end{itemize}
    By Ostrowski's theorem, the completions at the infinite place are isomorphic to $\mathbb{R}$ or $\mathbb{C}$. For each (\textit{local}) field $\mathfrak{K}_\nu$, one can also construct the \textbf{valuation ring} as the subring $\mathcal{O}_\nu$ of elements of norm at most 1.\index{valuation!ring}\index{field!local}

    The \textbf{integral ad\`eles} are defined as follows, in analogy to \cref{number:integral_adeles}:\index{ad\`ele}
    \begin{gather}
        \mathbb{A}_{\mathcal{O}} := \widehat{\mathcal{O}}\times\prod_{\nu\in S}\mathfrak{K}_\nu
    \end{gather}
    and the \textbf{ad\`ele ring} is defined as its `rationalization':
    \begin{gather}
        \mathbb{A}_{\mathfrak{K}} := \mathfrak{K}\otimes_{\mathcal{O}}\mathbb{A}_{\mathcal{O}}\,.
    \end{gather}
    Note that the profinite completion $\widehat{\mathcal{O}}$ again admits a factorization over non-Archimedean places:
    \begin{gather}
        \widehat{\mathcal{O}}\cong\prod_{\nu\in P\backslash S}\mathcal{O}_\nu\,.
    \end{gather}
    Combining these facts, one obtains the following expression:
    \begin{gather}
        \mathbb{A}_{\mathfrak{K}}\cong\prod_{\nu\in S}\mathfrak{K}_\nu\times\prod'_{\nu\in P\backslash S}\mathfrak{K}_\nu\cong\prod'_{\nu\in P}\mathfrak{K}_\nu\,.
    \end{gather}

    This definition can now be generalized to global fields as is.
    \newdef{Global field}{\index{field!global}
        Either a number field or a function field (\cref{alggeom:coordinate_ring}) over a finite field. Equivalently, a finite extension of either the rational numbers $\mathbb{Q}$ or the rational functions with coefficients in a finite field.
    }

\section{\difficult{Algebraic geometry}}

    This section gives a relation between number theory and (algebraic) geometry (\cref{chapter:alggeom}). The content of \namecrefs{chapter:algebra}~\nameref{chapter:algebra} and~\nameref{chapter:complexcalculus} will also be used throughout this section.

\subsection{Modular forms}\label{section:modular_forms}

    \newdef{Modular group}{\index{modular!group}\index{M\"obius transformation}\label{alggeom:modular_group}
        In the setting of number theory, the projective special linear group $\mathrm{PSL}(2,\mathbb{Z})$ is often called the modular group.\footnote{This name is also used for the special linear group $\SL(2,\mathbb{Z})$ since M\"obius transformations are invariant under rescaling.} The modular group acts on the complex plane by \textbf{M\"obius transformations}:
        \begin{gather}
            \begin{pmatrix}
                a&b\\
                c&d
            \end{pmatrix}
            \cdot z := \frac{az+b}{cz+d}\,.
        \end{gather}
        For this reason, $\mathrm{PSL}(2,\mathbb{C})$ is sometimes also called the \textbf{M\"obius group}. The full modular group $\SL(2,\mathbb{Z})$ will be denoted by $\Gamma$.
    }

    \newdef{Modular form}{\index{modular!form}\index{cusp form}
        A modular form of weight $k\in\mathbb{R}$ is a holomorphic function on the upper-half plane $f:\mathcal{H}\rightarrow\mathbb{C}$ satisfying the following two conditions:
        \begin{enumerate}
            \item\textbf{Automorphicity}: For $g\equiv\begin{pmatrix}a&b\\c&d\end{pmatrix}\in\SL(2,\mathbb{Z})$, one has
            \begin{gather}
                f\bigl(g\cdot z\bigr)=(cz+d)^kf(z)\,,
            \end{gather}
            and
            \item\textbf{Bounded growth}: $f(z)$ is bounded for $z\longrightarrow i\infty$.
        \end{enumerate}
        If the modular form satisfies the stronger condition $f(z)\longrightarrow0$ when $z\longrightarrow i\infty$, it is said to be \textbf{cuspidal} or it is simply called a \textbf{cusp form}. The spaces of modular forms and cusp forms of weight $k\in\mathbb{R}$ are, respectively, denoted by $M_k(\Gamma)$ and $S_k(\Gamma)$.

        \todo{MAYBE ADD automorphicity in full generality}
    }
    \begin{remark}[Arithmetic group]\index{congruence!subgroup}\index{group!arithmetic}
        Modular forms can also be defined for subgroups of $\mathrm{SL}(2,\mathbb{Z})$ of finite index, such as the \textit{congruence groups} (or their generalizations, the \textit{arithmetic groups}).
    \end{remark}

    \begin{property}
        The generators of the modular group are given by
        \begin{gather}
            z\mapsto-\frac{1}{z}\qquad\text{and}\qquad z\mapsto z+1\,.
        \end{gather}
        Invariance under the second generator shows that modular forms are, in particular, periodic and, hence, admit a Fourier expansion. Cusp forms are exactly those modular forms with vanishing constant Fourier coefficient.
    \end{property}

    \begin{example}[Eisenstein series]\index{Eisenstein series}
        Consider a complex number $\tau\in\mathcal{H}$. The Eisenstein series of weight $2k\in2\mathbb{Z}$, with $k>2$, is defined as follows:
        \begin{gather}
            G_{2k}(\tau) := \sum_{(m,n)\in\mathbb{Z}^2\backslash(0,0)}\frac{1}{(m+n\tau)^{2k}}\,.
        \end{gather}
    \end{example}

    \begin{remark}[Fourier expansion]
        Since $\begin{pmatrix}1&1\\0&1\end{pmatrix}$ is an element of $\mathrm{PSL}(2,\mathbb{2})$, every modular form is periodic and, hence, admits a Fourier expansion (\cref{distributions:fourier_series}):
        \begin{gather}
            f(z) = \sum_{n=-\infty}^{+\infty}a_ne^{2\pi iz}\,.
        \end{gather}
        $f$ can then be interpreted as a function on the punctured disk by making the transformation $z\longrightarrow q:=e^{2\pi iz}$\,. By Riemann's theorem on removable singularities (\cref{complex:riemann_theorem_singularities}), the growth condition on modular forms is equivalent to $f(q)$ admitting a holomorphic continuation to the origin. Moreover, it is a cusp form exactly if $a_0=0$. The growth condition also implies that the Fourier series can be truncated to positive integers:
        \begin{gather}
            f(z) = \sum_{n=0}^{+\infty}a_ne^{2\pi iz}\,.
        \end{gather}
    \end{remark}

    \newdef{Hecke operator}{\index{Hecke operator}\index{inner product!Petersson}
        For every $n\in\mathbb{N}_0$, let $\Gamma_n$ denote the subgroup of $\Gamma$ on matrices with determinant $n$. On $M_k(\Gamma)$, one can define the following averaging operators:
        \begin{gather}
            T_nf(z) := n^{k-1}\sum_{g\in\Gamma\backslash\Gamma_n}(cz+d)^{-k}f(g\cdot z)\,,
        \end{gather}
        where $g\equiv\begin{pmatrix}a&b\\c&d\end{pmatrix}$. One can easily see that $[T_m,T_n]=0$.

        The Hecke operators also have an important relation to cusp forms. The subspace $S_k(\Gamma)$ is preserved under the action of the Hecke operators and, moreover the joint eigenforms of the Hecke operators satisfy the following relations on their Fourier coefficients:
        \begin{gather}
            a_n = \lambda_na_1\,,
        \end{gather}
        where $\lambda_n$ is the $n^{\text{th}}$ eigenvalue. $S_k(\Gamma)$ admits an inner product, the \textbf{Petersson inner product}, defined as follows:
        \begin{gather}
            \braket{f}{g} := \Int_{\mathcal{F}}f(z)\overline{g}(z)y^{k-2}\,dx\,dy\,,
        \end{gather}
        where $\mathcal{F}\subset\mathcal{H}$ is any fundamental domain for $\Gamma$. With respect to this inner product, there exists a normalized eigenbasis of the Hecke operators for $S_k(\Gamma)$, where normalized means that the first Fourier coefficient is equal to 1 (and, hence, the Fourier coefficients correspond to the Hecke eigenvalues).
    }

\subsection{Algebraic functions}

    \Cref{algebra:algebraic_element} can be generalized to the functional setting.
    \newdef{Algebraic function}{\index{algebraic!function}\index{transcendental!function}
        Let $R$ be a commutative ring. A function $f:R^n\rightarrow R$ is said to be algebraic if it is the solution of a polynomial equation with coefficients in $R[x_1,\ldots,x_n]$.\footnote{Often, the polynomial is required to be irreducible.} If $f$ is not algebraic, it is said to be \textbf{transcendental}.
    }

\section{\texorpdfstring{$L$-functions}{L-functions}}

    \newdef{Dirichlet series}{\index{Dirichlet!series}
        A series of the form
        \begin{gather}
            \sum_{n=1}^{+\infty}\frac{a_n}{n^s}\,,
        \end{gather}
        where $s\in\mathbb{C}$ and $\seq{a}\subset\mathbb{C}$. If the coefficients satisfy the growth bound $|a_n|=O(n^\sigma)$, the series converges (absolutely) to a holomorphic function whenever $\Re(s)>\sigma$.
    }

    The Dirichlet series for unit coefficients is of major interest in mathematics.
    \newdef{Riemann $\zeta$-function}{\index{Riemann!$\zeta$-function}
        \begin{gather}
            \zeta(s) := \sum_{n=1}^{+\infty}\frac{1}{n^s}
        \end{gather}
        for all $s\in\mathbb{C}$ for which $\Re(s)>1$. For general $s$, an \textit{analytic continuation} is used (see \cref{complex:analytic_continuation}).
    }
    \begin{formula}[Euler product]\index{Euler product}\label{number:riemann_euler_product}
        \begin{gather}
            \zeta(s) = \prod_{p\text{ prime}}\left(1 - \frac{1}{p^s}\right)^{-1}
        \end{gather}
        for all $s\in\mathbb{C}$ for which $\Re(s)>1$. This is an example of an Euler product, a product over all prime numbers.
    \end{formula}

    \newdef{Selberg $\zeta$-function}{\index{$\zeta$-function}\index{Selberg|see{$\zeta$-function}}\index{Ramanujan!conjecture}
        A function $\zeta:\mathbb{C}\rightarrow\mathbb{C}$ satisfying the following conditions:
        \begin{enumerate}
            \item There exists a $\eta\in\mathbb{R}^+$ (usually taken to be 1) such that $\zeta$ admits a convergent series expansion of the form
            \begin{gather}
                \zeta(s) = \sum_{n=1}^{+\infty}\lambda^{-s}\,,
            \end{gather}
            for some sequence of real numbers $\seq{\lambda}\subset\mathbb{R}$, or admits a Euler product expansion.
            \item $\zeta$ analytic continuation (\cref{complex:analytic_continuation}) of the above series on all of $\mathbb{C}$.
            \item There exists a `completion' $\widehat{\zeta}$ of $\zeta$ that admits a functional relation of the form
            \begin{gather}
                \label{number:functional_equation}
                \widehat{\zeta}(1-s) = \widehat{\zeta}(s)\,.
            \end{gather}
            \item[4*.] \textit{Ramanajuan conjecture}: $a_1=1$ and $a_n=O(n^\varepsilon)$ for any $\varepsilon>0$.
        \end{enumerate}
        \todo{CHECK AND COMPLETE}
    }

    \begin{example}[Differential operators]
        Consider a self-adjoint \textit{elliptic differential operator} (see \cref{bundle:differential_operator}) $D$ whose spectrum lies in the positive real numbers $\mathbb{R}^+_0$.
        \begin{gather}
            \zeta_D(s) := \tr\bigl(D^{-s}\bigr)\,.
        \end{gather}
        It can be shown that this is equal to the \textit{Mellin transform} (see \cref{distributions:mellin}) of the \textit{partition function} of $D$:
        \begin{gather}
            \zeta_D(s) = \Intt{0}{+\infty}t^{s-1}\left(\sum_{n=1}^{+\infty}\exp(-t\lambda_n)\right)\,dt\,.
        \end{gather}
    \end{example}
    \begin{remark}[Bost--Connes system]\index{Bost--Connes system}\index{Riemann!hypothesis}
        The \textit{Bost--Connes system}, a \textit{$C^*$-algebra} (see \cref{operators:c_star}) of the form $C^*\bigl(\mathbb{Q}[\mathbb{Q}/\mathbb{Z}]\rtimes\mathbb{N}^\times\bigr)$, admits a Hamiltonian whose associated $\zeta$-function is the Riemann $\zeta$-function. A physical realization of this system would be able to shed light on the \textit{Riemann hypothesis}.
    \end{remark}

    \newdef{Hasse--Weil $\zeta$-function\footnotemark}{\index{Hasse--Weil!$\zeta$-function}
        \footnotetext{Also called the \textbf{local Hasse--Weil function} or \textbf{local $L$-function}.}
        Let $V$ be a nonsingular projective variety over the finite field $\mathbb{F}_p$ and, for all $k\in\mathbb{N}$, let $N_K:=\left|\mathbb{F}_{p^k}(V)\right|$ be the number of rational points over $\mathbb{F}_{p^k}$.
        \begin{gather}
            Z(V,t) := \exp\left(\sum_{k=1}^{+\infty}N_k\frac{t^k}{k}\right)\in\mathbb{Q}[[t]]\,.
        \end{gather}
    }
    \begin{remark}[Weil conjectures]\index{Weil!conjectures}\index{Riemann!hypothesis}\index{\'etale!cohomology}
        The Weil conjectures are three related statements of the Hasse--Weil function of profound importance. The conjectures were:
        \begin{enumerate}
            \item $Z(V,t)$ is a rational function of $t$.
            \item $Z(V,t)$ admits a functional relation as in \cref{number:functional_equation}.
            \item Discrete \textit{Riemann hypothesis}: The zeroes of $P_k(t)$, for all $k\in\mathbb{N}$, lie on the critical line $\Re(z)=k/2$, where the $P_k(t)$ are the integral polynomials in the rational form of $Z(V,t)$:
            \begin{gather}
                Z(V,t) = \frac{P_1(t)\cdots P_{2\dim(V)-1}(t)}{P_0(t)\cdots P_{2\dim(V)}(t)}
            \end{gather}
            with $P_0(t)=1-t$ and $P_{2\dim(V)}(t)=1-p^nt$.
            \item If $X$ is a \textit{good reduction} of nonsingular projective variety over a number field in the complex numbers, then $\deg(P_k)$ equals the $k^{\text{th}}$ Betti number (\cref{topology:betti_number}) of the set of complex points of $V$.
        \end{enumerate}
        Items 1, 2 and 4 were proven by \indexauthor{Grothendieck} using \textit{\'etale cohomology}, whereas the discrete Riemann hypothesis was proven by \indexauthor{Deligne} using \textit{$l$-adic cohomology}.
    \end{remark}

    \newdef{$L$-function of varieties}{\index{L-!function of variety}
        Consider a nonsingular projective variety $V$ over $\mathbb{Q}$. For almost all primes $p$, one obtains a good reduction $V_p$. The Dirichlet series
        \begin{gather}
            Z_{\mathbb{Q},V}(s) := \prod_{p\text{ good prime}}Z(V_p,p^{-s})
        \end{gather}
        is called the $L$-function of $V$ (or sometimes also the Hasse--Weil function of $V$).
    }

    \begin{example}[Riemann $\zeta$-function]\index{Riemann!$\zeta$-function}
        Consider $V=\spec{\mathbb{Q}}$ with $V_p\cong\spec(\mathbb{F}_p)$ for prime $p\in\mathbb{N}$. In this case,
        \begin{gather}
            Z(V_p,t) = \exp\left(\sum_{n=1}^{+\infty}\frac{t^n}{n}\right) = (1-t)^{-1}
        \end{gather}
        and, hence,
        \begin{gather}
            Z_{\mathbb{Q},V}(s) = \prod_{p\text{ prime}}(1-p^{-s})^{-1} = \zeta(s)
        \end{gather}
        by \cref{number:riemann_euler_product}.
    \end{example}

    Recall the cusp forms of \cref{section:modular_forms}.
    \newdef{$L$-function of modular forms}{\index{L-!function of modular form}
        Let $f:\mathcal{H}\rightarrow\mathbb{C}$ be a cusp form of weight $k\in\mathbb{R}$ and consider its Fourier series
        \begin{gather}
            f(z) = \sum_{n=1}^{+\infty}a_ne^{2\pi iz}\,.
        \end{gather}
        The associated $L$-function is given by the following Dirichlet series:
        \begin{gather}
            L_f(s) := \sum_{n=1}^{+\infty}\frac{a_n}{n^s}\,.
        \end{gather}
        The domain of convergence is bounded by $\Re(s)>1+\frac{k}{2}$.
    }

    \begin{theorem}[Modularity theorem\footnotemark]\index{Tayinama--Shimura}\index{modularity theorem}
        \footnotetext{Also called the \textbf{Taniyama--Shimura(--Weil) theorem/conjecture} (or any of its permutations).}
        Every Hasse--Weil $L$-function of an elliptic curve can be obtained as the $L$-function of a modular form.
    \end{theorem}
\chapter{Finance}

    The content of \cref{chapter:measure} and \cref{chapter:probability} is essential for this chapter.

\section{Securities}

    \newdef{Security}{\index{security}
        A tradable financial asset.
    }

    \newdef{Bond}{\index{bond}
        A security for which the issuer (debtor) owes the holder (creditor) a debt and has to provide a cash flow to the holder/creditor.

        When the interest rate is set at 0, i.e.~only the face value is repaid at time of maturity, the bond is called a \textbf{zero-coupon bond} (or \textbf{discount bond}).
    }

\section{Derivatives}

    \newdef{IR derivatives}{\index{cap}\index{floor}
        Caps and floors are derivatives for which the buyer receives a payment whenever the interest rate exceeds or stays under a certain threshold.
    }

    \newdef{Option}{\index{option}\index{call}\index{put}\index{swap}\index{forward}
        A derivative that allows (but not obliges) the holder to buy/sell an asset at a specified strike price on or before a specified date. Some examples are:
        \begin{itemize}
            \item\textbf{Call option}: An option to buy an asset.
            \item\textbf{Put option}: An option to sell an asset.
            \item\textbf{Swaption} (or swap option): An option for a \textbf{swap}, i.e.~the exchange of assets. 
        \end{itemize}
        Options are very similar to \textbf{futures} (future contracts) and \textbf{forwards}\footnote{The main difference between futures and forwards is that the latter are barely regulated.} (forward contracts), although the latter two oblige the holder to perform the specified transaction.

        There are various families of options, depending on the specific ways the payoff is calculated. The main families are the \textbf{European} and \textbf{American options}. The former can only be exercised at expiration date, while the latter can be exercised at any moment before the expiration date.
    }
    \begin{remark}[Spots]\index{spot}
        When the settlement date (also called the \textbf{spot data}) of a trade is immediate, i.e.~usually two days after the trade, the contract is called a \textbf{spot} (contract).
    \end{remark}

    \section{Markets}

    \newdef{Arbitrage}{\index{arbitrage}
        Taking advantage of the difference in price on two markets.
    }

    \newdef{Completeness}{\index{complete!market}
        A market is said to be \textbf{complete} (or called an \textbf{Arrow-Debreu market}) if it satisfies the following two conditions:
        \begin{enumerate}
            \item Transactions are negligible and, hence, information is perfect.
            \item Every asset has a price.
        \end{enumerate}
    }

    \newdef{Risk-neutral measure}{\index{risk}\index{discount}
        Consider an asset with price $\tseq{S}$ on a probability space $(\Omega,\Sigma,P)$ and assume that an investor has access to a bond with interest rate $R:T\rightarrow\mathbb{R}^+$. A risk-neutral measure for the given stock is a measure $\widetilde{P}$ that is equivalent to $P$ and such that the discounted process $\{D_tS_t\}_{t\in T}$ is a martingale relative to $\widetilde{P}$, where the \textbf{discount process} $\tseq{D}$ is defined by
        \begin{gather}
            D_t := \exp\left(-\Int_0^tR(s)\,ds\right)\,.
        \end{gather}
    }
    \begin{example}\index{Sharpe ratio}\index{price}
        Assume that the stock price $\tseq{S}$ is a generalized Brownian motion:
        \begin{gather}
            dX_t = \alpha(t)S_t\,dt + \sigma(t)S_t\,dW_t\,.
        \end{gather}
        The discounted process can then be rewritten as
        \begin{gather}
            D_tS_t = \sigma(t)D_tS_t\bigl(\theta(t)\,dt + dW_t\bigr)\,,
        \end{gather}
        where
        \begin{gather}
            \theta(t) := \frac{\alpha(t) - R(t)}{\sigma(t)}
        \end{gather}
        is called the \textbf{market price of risk} or \textbf{Sharpe ratio}. By the Girsanov theorem~\ref{measure:girsanov}, a risk-neutral measure for this process is given by
        \begin{gather}
            \deriv{\widetilde{P}}{P} = \exp\left(-\Int_0^t\theta(s)\,dW_s - \frac{1}{2}\Int_0^t\theta(s)^2\,ds\right)\,.
        \end{gather}
    \end{example}

    \begin{theorem}[Fundamental theorem of asset pricing for discrete markets]\index{fundamental theorem!of asset pricing}
        \begin{enumerate}
            \item A discrete market on a discrete probability space ($\Omega,\Sigma,P)$ is arbitrage-free if and only if there exists a risk-neutral measure that is equivalent to $P$.
            \item An arbitrage-free market on a discrete probability space ($\Omega,\Sigma,P)$ with stocks $S$ and a risk-free bond $B$ is complete if and only if there exists a unique risk-neutral measure with numeraire $B$ that is equivalent to $P$.
        \end{enumerate}
    \end{theorem}

    \begin{theorem}[Fundamental theorem of asset pricing]\index{fundamental theorem!of asset pricing}
        
    \end{theorem}

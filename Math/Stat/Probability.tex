\chapter{Probability}\label{chapter:probability}

    The majority of this chapter uses the language of measure theory. For an introduction see Chapter \ref{chapter:measure}. The section on \textit{imprecise probabilities} is mainly based on \cite{imprecise}.

\section{Probability}

    The Kolmogorov axioms of probability state when a set admits the definition of a probability theory:
    \newdef{Kolmogorov axioms}{\index{Kolmogorov!axioms}\index{probability}\index{sample space}
        A probability space $(\Omega,\Sigma,P)$ is a measure space \ref{lebesgue:measure_space} with finite measure $P(X)=1$. The set $\Omega$ is called the \textbf{sample space}.
    }

    \newdef{Random variable}{\index{random variable}
        Let $(\Omega,\Sigma,P)$ be a probability space. A function $X:\Omega\rightarrow\mathbb{R}$ is called a random variable if $\forall a\in\mathbb{R}:X^{-1}\big([a,\infty[\big)=\{\omega\in\Omega\mid X(\omega)\geq a\}\in\Sigma$.
    }

    \newdef{$\sigma$-algebra of a random variable}{\index{$\sigma$!algebra}
        Let $X$ be a random variable defined on a probability space $(\Omega,\Sigma,P)$ and denote the Borel $\sigma$-algebra of $\mathbb{R}$ by $\mathcal{B}$. The following family of sets is a $\sigma$-algebra:
        \begin{gather}
            \label{prob:sigma_algebra_generated_random_variable}
            X^{-1}(\mathcal{B}) := \{S\in\Sigma\mid\exists B\in\mathcal{B}:S = X^{-1}(B)\}.
        \end{gather}
    }
    \begin{notation}
        The $\sigma$-algebra generated by the random variable $X$ is often denoted by $\mathcal{F}_X$, analogous to \ref{set:notation:generated_sigma_algebra}.
    \end{notation}

    \newdef{Event}{\index{event}
        Let $(\Omega,\Sigma,P)$ be a probability space. An element $S$ of the $\sigma$-algebra $\Sigma$ is called an event.

        From this definition it is clear that a single possible outcome of a measurement can be a part of multiple events. So, although only one outcome can occur at the same time, multiple events can occur simultaneously.
    }
    \begin{remark*}
        The Kolmogorov axioms use the $\sigma$-algebra \ref{set:sigma_algebra} of events instead of the power set \ref{set:power_set} of all events. Intuitively this seems to mean that some possible outcomes are not treated as events. However, one can make sure that the $\sigma$-algebra still contains all ``useful'' events by using a ``nice'' definition of probability spaces.
    \end{remark*}

    \begin{formula}[Union]\label{prob:union}
        Let $A,B$ be two events. The probability that at least one of them occurs is given by the following formula:
        \begin{gather}
            P(A\cup B) = P(A) + P(B) + P(A\cap B).
        \end{gather}
    \end{formula}

    \newdef{Disjoint events}{
        Two events $A$ and $B$ are said to be disjoint if they cannot happen at the same time:
        \begin{gather}
            P(A\cap B) = 0.
        \end{gather}
    }
    \result{If $A$ and $B$ are disjoint, the probability that both $A$ and $B$ occur is just the sum of their individual probabilities.}

    \newformula{Complement}{\index{complement}\label{prob:complement}
        Let $A$ be an event. The probability of $A$ being false is denoted as $P\left(\overline{A}\right)$ and is given by
        \begin{gather}
            P\left(\overline{A}\right) = 1 - P(A).
        \end{gather}
    }
    \begin{result}
        From the previous equation and de Morgan's laws \eqref{set:de_morgan_union} and \eqref{set:de_morgan_intersection}, one can derive the following formula:
        \begin{gather}
            P\left(\overline{A}\cap\overline{B}\right) = 1 - P(A\cup B).
        \end{gather}
    \end{result}

\section{Conditional probability}

    \newdef{Conditional probability}{\index{probability!conditional}\label{prob:conditional_probability}
        Let $A,B$ be two events. The probability of $A$ given that $B$ is true is denoted as $P(A|B)$:
        \begin{gather}
            P(A|B) = \frac{P(A\cap B)}{P(B)}.
        \end{gather}
    }
    By interchanging $A$ and $B$ in previous equation and by observing that this has no effect on the quantity $P(A\cap B)$ the following important result can be derived:
    \begin{theorem}[Bayes]\index{Bayes}\label{prob:bayes}
        Let $A,B$ be two events.
        \begin{gather}
            P(A|B) = \frac{P(B|A)P(A)}{P(B)}.
        \end{gather}
    \end{theorem}

    \begin{formula}
        Let $\seq{B}$ be a sequence of pairwise disjoint events. If $\bigsqcup_{n=1}^\infty B_n = \Omega$, the total probability of a given event $A$ can be calculated as follows:
        \begin{gather}
            \label{probability:total_probability_conditional}
            P(A) = \sum_{n=1}^\infty P(A|B_n)P(B_n).
        \end{gather}
    \end{formula}

    \newdef{Independent events}{\index{independence}
        Let $A,B$ be two events. $A$ and $B$ are said to be independent if they satisfy the following relation:
        \begin{gather}
            P(A\cap B) = P(A)P(B).
        \end{gather}
    }
    \begin{result}
        If $A$ and $B$ are two independent events, Bayes's theorem simplifies to
        \begin{gather}
            P(A|B) = P(A).
        \end{gather}
    \end{result}
    The above definition can be generalized to multiple events:
    \begin{definition}
        The events $A_1,\ldots,A_n$ are said to be independent if for each choice of $k$ events the probability of their intersection is equal to the product of their individual probabilities.
    \end{definition}
    This definition can be stated in terms of $\sigma$-algebras:
    \begin{definition}[Independence]\index{independence}
        The $\sigma$-algebras $\mathcal{F}_1,\ldots,\mathcal{F}_n$ defined on a probability space $(\Omega,\mathcal{F},P)$ are said to be independent if for all choices of distinct indices $i_1,\ldots,i_k$ and for all choices of sets $F_{i_n}\in\mathcal{F}_{i_n}$ the following equation holds:
        \begin{gather}
            \label{prob:independent_sigma_algebras}
            P(F_{i_1}\cap\cdots\cap F_{i_k}) = P(F_{i_1})\cdots P(F_{i_k}).
        \end{gather}
    \end{definition}
    \begin{result}
        Let $X,Y$ be two random variables. $X$ and $Y$ are independent if the $\sigma$-algebras generated by them are independent.
    \end{result}

\section{Probability distribution}

    \newdef{Probability distribution}{\index{probability!distribution}\label{prob:probability_distribution}
        Let $X$ be a random variable defined on a probability space $(\Omega,\Sigma,P)$. The following function is a measure on the Borel $\sigma$-algebra of $\mathbb{R}$:
        \begin{gather}
            P_X(B) = P(X^{-1}(B)).
        \end{gather}
        This measure is called the probability distribution of $X$.
    }

    \begin{example}[Rademacher variable]\index{Rademacher!variable}\label{prob:rademacher}
        A random variable on $\Omega=\{-1,1\}$ with probability distribution $P_X(-1)=P_X(1)=\frac{1}{2}$.
    \end{example}

    \newdef{Density}{\index{density}\index{mass}
        Let $f\geq0$ be an integrable function and recall Property \ref{lebesgue:measure_by_integral}. The function $f$ is called the density of the measure $P(A):=\int_Af\,d\lambda$ (with respect to the Lebesgue measure $\lambda$). If the measure is a probability measure, i.e. is normalized to 1, $f$ is called a \textbf{probability density function}.

        More generally, by the Radon-Nikodym theorem \ref{lebesgue:radon_nikodym}, every absolutely continuous probability distribution $P$ is of the form
        \begin{gather}
            P(A) = \int_Af\,d\lambda
        \end{gather}
        for some integrable function $f$.

        In the case where $P$ is discrete, i.e. one works with respect to the counting measure, the Radon-Nikodym derivative is called the \textbf{probability mass function}. (In this compendium this function will also often be called the density function.)
    }

    \newdef{Cumulative distribution function}{\index{cumulative distribution function}\label{prob:cdf}
        \nomenclature[A_CDF]{CDF}{cumulative distribution function}
        Consider a random variable $X$ and its associated distribution $P_X$. The cumulative distribution function $F_X:\mathbb{R}\rightarrow[0,1]$ is defined as follows:
        \begin{gather}
            F_X(a) := P_X(\{x\in\mathbb{R}\mid x\leq a\}).
        \end{gather}
    }

    \begin{theorem}[Skorokhod's representation theorem]\index{Skorokhod}
        Let $F:\mathbb{R}\rightarrow[0,1]$ be a function that satisfies the following three properties:
        \begin{itemize}
            \item $F$ is nondecreasing.
            \item $\ds\lim_{x\rightarrow-\infty}F(x) = 0$ and $\ds\lim_{x\rightarrow\infty}F(x) = 1$.
            \item $F$ is right-continuous, i.e. $\ds\lim_{y\nearrow y_0}F(y)=F(y_0)$.
        \end{itemize}
        There exists a random variable $X:[0,1]\rightarrow\mathbb{R}$ defined on the probability space $([0,1],\mathcal{B}_{[0,1]},\lambda_{[0,1]})$ such that $F=F_X$, where $\mathcal{B}_{[0,1]}$ is the Borel $\sigma$-algebra of $[0,1]$ with its Euclidean topology.
    \end{theorem}

    The following theorem is a specific instance of the more general change-of-variables formula:
    \begin{theorem}[Theorem of the unconscious statistician]\label{prob:unconscious_statistician}
        Consider a random variable $X$ on a probability space $(\Omega,\Sigma,P)$. The following equality holds for every integrable function $g\in L^1(\mathbb{R})$:
        \begin{gather}
            \int_\Omega g\circ X\,dP = \int_\mathbb{R}g\,dP_X.
        \end{gather}
    \end{theorem}
    \begin{remark}
        The name of this theorem stems from the fact that many scientists take this equality to be a definition of the expectation value $\expect{g(X)}$. However, this equality should be proven since the measure on the right-hand side is the one belonging to the random variable $X$ and not $g(X)$.
    \end{remark}

    \begin{formula}
        Consider an absolutely continuous probability function $P$ defined on $\mathbb{R}^n$ and let $f$ be the associated density. Let $g:\mathbb{R}^n\rightarrow\mathbb{R}$ be integrable with respect to $P$.
        \begin{gather}
            \int_{\mathbb{R}^n}g\,dP = \int_{\mathbb{R}^n}f(x)g(x)\,dx
        \end{gather}
    \end{formula}
    \begin{result}
        The previous formula together with Theorem \ref{prob:unconscious_statistician} gives rise to
        \begin{gather}
            \label{prob:omega_int_to_real_int}
            \int_\Omega g\circ X\,dP = \int_{\mathbb{R}^n}f_X(x)g(x)\,dx.
        \end{gather}
    \end{result}

    \begin{formula}
        Let $X$ be a random variable with density function $f_X$ and let $g:\mathbb{R}\rightarrow\mathbb{R}$ be smooth and strictly monotone. The random variable $g\circ X$ has an associated density $f_g$ given by
        \begin{gather}
            \label{prob:function_of_random_variable}
            f_g(y) = f(g^{-1}(y))\left|\deriv{g^{-1}}{y}(y)\right|.
        \end{gather}
    \end{formula}

    Weak convergence of measures \ref{lebesgue:weak_convergence} induces a notion for convergence of random variables:
    \newdef{Convergence in distribution}{\index{convergence!in distribution}\label{prob:convergence_in_distribution}
        A sequence $\seq{X}$ of random variables is said to converge in distribution to a random variable $Y$ if the associated cumulative distribution functions $F_{X_n}$ converge pointwise to $F_Y$, i.e. $\lim_{n\rightarrow\infty}F_{X_n}(x)=F_Y(x)$ for all $x\in\mathbb{R}$, where $F$ is continuous. This is equivalent to requiring that the associated probability measures $P_{X_n}$ converge weakly to $P_X$ (Definition \ref{lebesgue:weak_convergence}).
    }
    \begin{notation}
        If a sequence $\seq{X}$ converges in distribution to a random variable $Y$, this is often denoted by $X_n\overset{d}{\longrightarrow}Y$. Sometimes the $d$ (for ``distribution'') is replaced by the $\mathcal{L}$ (for ``law'').
    \end{notation}

    \begin{theorem}[Slutsky]\index{Slutsky}
        Let $\seq{X},\seq{Y}$ be two sequences of random variables converging in probability to a random variable $X$ and a constant $c$, respectively. The following statements hold:
        \begin{itemize}
            \item $X_n+Y_n\overset{d}{\longrightarrow}X+c$,
            \item $X_nY_n\overset{d}{\longrightarrow}cX$, and
            \item $X_n/Y_n\overset{d}{\longrightarrow}X/c$.
        \end{itemize}
    \end{theorem}

    \newdef{Convergence in probability}{\index{convergence!in probability}\label{prob:convergence_in_probability}
        A sequence $\seq{X}$ of random variables on a metric space $(\Omega,d)$ is said to converge in probability to a random variable $Y$ if for all $\varepsilon>0$ the following statement holds:
        \begin{gather}
            \lim_{n\rightarrow\infty}\mathrm{Pr}(d(X_n,X)>\varepsilon)=0.
        \end{gather}
        Convergence in probability implies convergence in distribution.
    }

    \newdef{\difficult{Giry monad}}{\index{Giry monad}
        Consider the category $\mathbf{Meas}$ of measurable spaces. On this space one can define a monad \ref{cat:monad} that sends a set $X$ to its collection of probability distributions equipped with the $\sigma$-algebra generated by all evaluation maps $\mathrm{ev}_U$, where $U$ runs over the measurable subsets of $X$.

        The unit of the Giry monad $\mathbb{P}$ is defined by assigning Dirac measures:
        \begin{gather}
            \eta_X(x) := \delta_x.
        \end{gather}
        The multiplication map is defined as follows:
        \begin{gather}
            \mu_X(Q)(U) := \int_{P\in\mathbb{P}X}\mathrm{ev}_U(P)\,dQ.
        \end{gather}
    }

\section{Moments}
\subsection{Expectation value}

    \newdef{Expectation value}{\index{expectation}\label{prob:expectation_value}
        Let $X$ be random variable defined on a probability space $(\Omega,\Sigma,P)$.
        \begin{gather}
            \expect{X} := \int_\Omega X\,dP
        \end{gather}
    }
    \begin{notation}
        Other common notations are $\langle X \rangle$ and $\mu_X$. However, the latter might be confused with a general measure on the space $X$ and will, therefore, not be used here.
    \end{notation}

    \begin{property}[Markov's inequality]\index{Markov!inequality}
        Let $X$ be a random variable. For every constant $a>0$ the following inequality holds:
        \begin{gather}
            \mathrm{Pr}(X\geq a)\leq\frac{\expect{X}}{a}.
        \end{gather}
    \end{property}

    \newdef{Moment of order \texorpdfstring{$r$}{r}}{\index{moment}\label{prob:moment}
        The moment of order $r$ is defined as the expectation value of the $r^{th}$ power of $X$. By Equation \eqref{prob:omega_int_to_real_int} this becomes
        \begin{gather}
            \expect{X^r} = \int_\mathbb{R}x^rf_X(x)\,dx.
        \end{gather}
    }
    \newdef{Central moment of order \texorpdfstring{$r$}{r}}{\index{central!moment}\label{prob:central_moment}
        \begin{gather}
            \expect{(X-\mu)^r} = \int_\mathbb{R}(x-\mu)^rf_X(x)\,dx
        \end{gather}
    }
    \begin{remark}
        Moments of order $n$ are determined by central moments of order $k\leq n$ and, conversely, central moments of order $n$ are determined by moments of order $k\leq n$.
    \end{remark}
    \newdef{Variance}{\index{variance}
        The central moment of order 2 is called the variance:
        \begin{gather}
            \variance{X} := \expect{(X-\mu)^2}.
        \end{gather}
    }
    \newdef{Standard deviation}{\index{standard!deviation}
        \begin{gather}
            \sigma_X := \sqrt{V[X]}
        \end{gather}
    }

    \begin{property}
        If $\expect{|X|^n}$ is finite for some $n>0$, then $\expect{X^k}$ exists and is finite for all $k\leq n$.
    \end{property}

    \begin{property}[Chebyshev's inequality]\index{Chebyshev!inequality}
        Let $X$ be a nonnegative random variable. For every constant $a>0$ the following inequality holds:
        \begin{gather}
            \mathrm{Pr}(|X-\expect{X}|\geq a)\leq\frac{\variance{X}}{a^2}.
        \end{gather}
    \end{property}

    \newdef{Moment generating function}{\index{moment!generating function}\label{prob:moment_generating_function}
        \begin{gather}
            M_X(t) := \expect{e^{tX}} = \int_\mathbb{R}e^{tx}f_X(x)\,dx
        \end{gather}
    }
    \begin{property}
        If the moment generating function exists, the moments $\expect{X^n}$ can be expressed in terms of $M_X$:
        \begin{gather}
            \label{prob:moment_generating}
            \expect{X^n} = \left.\mderiv{n}{M_X(t)}{t}\right|_{t=0}.
        \end{gather}
    \end{property}

    \begin{method}[Chernoff bound]\index{Chernoff bound}
        The Chernoff bound for a random variable gives a bound on the tail probabilities. For all constants $\lambda>0$, the Markov inequality implies the following statement:
        \begin{gather}
            \mathrm{Pr}(X\geq a)=\mathrm{Pr}(e^{\lambda X}\geq e^{\lambda a})\leq\frac{\expect{e^{\lambda X}}}{e^{\lambda a}}.
        \end{gather}
        If one has more information about the moment generating function, the Chernoff bound can be used to obtain improved concentration inequalities by optimizing over $\lambda$.
    \end{method}
    \begin{property}[Hoeffding's inequalities]\index{Hoeffding!inequality}\label{prob:hoeffding_inequality}
        Consider a collection of bounded, independent random variables $X_1,\ldots,X_n$. Without loss of generality one can assume that they are bounded by the unit intevral, i.e. $0\leq X_i\leq 1$. For every constant $\lambda\geq0$ the following inequality holds:
        \begin{gather}
            \mathrm{Pr}\left(\overline{X}-\expect{\overline{X}}\geq\lambda\right)\leq\exp(-2n\lambda^2).
        \end{gather}
        If one can sharpen the bounds for the variables such that $X_i\in[a_i,b_i]$, then
        \begin{gather}
            \mathrm{Pr}\left(\overline{X}-\expect{\overline{X}}\geq\lambda\right)\leq\exp\left(-\frac{2n^2\lambda^2}{\sum_{i=1}^n(b_i-a_i)^2}\right).
        \end{gather}
    \end{property}

    \newdef{Characteristic function}{\index{characteristic!function}\label{prob:characteristic_function}
        \begin{gather}
            \varphi_X(t) := \expect{e^{itX}}
        \end{gather}
    }
    \begin{property}\label{prob:characteristic_function_properties}
        The characteristic function has the following properties:
        \begin{itemize}
            \item $\varphi_X(0) = 1$,
            \item $|\varphi_X(t)| \leq 1$, and
            \item $\varphi_{aX+b}(t) = e^{itb}\varphi_X(at)$ for all $a,b\in\mathbb{R}$.
        \end{itemize}
    \end{property}

    \begin{formula}
        If $\varphi_X(t)$ is $k$ times continuously differentiable, then $X$ has a finite $k^{th}$ moment and
        \begin{gather}
            \label{prob:characteristic_function_as_moment_generator}
            \expect{X^k} = \frac{1}{i^k}\mderiv{k}{}{t}\varphi_X(0).
        \end{gather}
        Conversely, if $X$ has a finite $k^{th}$ moment, then $\varphi_X(t)$ is $k$ times continuously differentiable and the above formula holds.
    \end{formula}

    \newformula{Inversion formula}{\index{inversion!formula}
        Let $X$ be a random variable. If the CDF of $X$ is continuous at $a,b\in\mathbb{R}$, then
        \begin{gather}
            \label{prob:inversion_formula}
            F_X(b) - F_X(a) = \lim_{c\rightarrow\infty}\frac{1}{2\pi}\int_{-c}^c\frac{e^{-ita} - e^{-itb}}{it}\varphi_X(t)\,dt.
        \end{gather}
    }
    \begin{formula}
        If $\varphi_X(t)$ is integrable, the CDF is given by:
        \begin{gather}
            f_X(x) = \frac{1}{2\pi}\int_\mathbb{R}e^{-itx}\varphi_X(t)\,dt.
        \end{gather}
    \end{formula}
    \remark{This formula implies that the density function and the characteristic function form a Fourier transform pair.}

\subsection{Correlation}

    \begin{property}\index{independence}\label{prob:independence_expectation_values}
        Two random variables $X,Y$ are independent if and only if $\expect{f(X)g(Y)} = \expect{f(X)}\expect{g(Y)}$ holds for all measurable bounded functions $f,g$.
    \end{property}

    The value $\expect{XY}$ is equal to the inner product $\langle X|Y \rangle$ as defined in \eqref{lebesgue:L2_inner_product}. It follows that independence of random variables implies orthogonality. To generalize this concept, the following notions are introduced:
    \newdef{Centred random variable}{\index{random variable}
        Let $X$ be a random variable with finite expectation value $\expect{X}$. The centred random variable $X_c$ is defined as $X_c = X-\expect{X}$.
    }
    \newdef{Covariance}{\index{covariance}
        The covariance of two random variables $X,Y$ is defined as follows:
        \begin{gather}
            \label{prob:covariance}
            \mathrm{cov}(X,Y) := \langle X_c\mid Y_c \rangle = \expect{(X-\expect{X})(Y-\expect{Y})}.
        \end{gather}
        Some basic math gives
        \begin{gather}
            \mathrm{cov}(X,Y) = \expect{XY} - \expect{X}\expect{Y}.
        \end{gather}
    }
    \newdef{Correlation}{\index{correlation}\label{prob:correlation}
        The correlation of two random variables $X,Y$ is defined as the cosine of the angle between $X_c$ and $Y_c$:
        \begin{gather}
            \rho_{XY} := \frac{\mathrm{cov}(X,Y)}{\sigma_X\sigma_Y}.
        \end{gather}
    }
    \result{From Theorem \ref{prob:independence_expectation_values} it follows that independent random variables are uncorrelated.}
    \result{If the random variables $X$ and $Y$ are uncorrelated, they satisfy $\expect{XY} = \expect{X}\expect{Y}$.}

    \begin{formula}[Bienaym\'e formula]\index{Bienaym\'e}\label{prob:bienayme}
        Let $\seq{X}$ be a sequence of independent (or uncorrelated) random variables. Their variances satisfy the following equation:
        \begin{gather}
            \label{prob:variance_of_sum}
            \variance{\sum_{i=1}^\infty X_i} = \sum_{i=1}^\infty\variance{X_i}.
        \end{gather}
    \end{formula}

\subsection{Conditional expectation}

    Let $(\Omega,\Sigma,P)$ be a probability space. Consider a random variable $X\in L^2(\Omega,\Sigma,P)$ and a sub-$\sigma$-algebra $\mathcal{G}\subset\Sigma$. Property \ref{lebesgue:L2_hilbert_space} implies that the spaces $L^2(\Sigma)$ and $L^2(\mathcal{G})$ are complete and, hence, the projection theorem \ref{functional:projection_theorem} can be applied. For every $X\in L^2(\Sigma)$ there exists a random variable $Y\in L^2(\mathcal{G})$ such that $X-Y$ is orthogonal to $L^2(\mathcal{G})$. This has the following result:
    \begin{gather}
        \forall Z\in L^2(\mathcal{G}):\langle X-Y\mid Z \rangle\equiv\int_\Omega(X-Y)Z\,dP = 0.
    \end{gather}
    Since $\mathbbm{1}_G\in L^2(\mathcal{G})$ for every $G\in\mathcal{G}$, Equation \eqref{lebesgue:domain_change} can be rewritten as
    \begin{gather}
        \label{prob:conditional_expectation_condition}
        \int_GX\,dP = \int_GY\,dP
    \end{gather}
    for all $G\in\mathcal{G}$. This leads to the following definition:
    \newdef{Conditional expectation}{\index{expectation!conditional}\label{prob:conditional_expectation}
        Let $(\Omega,\Sigma,P)$ be a probability space and let $\mathcal{G}$ be a sub-$\sigma$-algebra of $\Sigma$. For every $\Sigma$-measurable random variable $X\in L^2(\Sigma)$ there exists a unique (up to a null set) random variable $Y\in L^2(\mathcal{G})$ that satisfies Equation \eqref{prob:conditional_expectation_condition} for every $G\in\mathcal{G}$. This variable $Y$ is called the conditional expectation of $X$ given $\mathcal{G}$ and it is denoted by $\expect{X\mid\mathcal{G}}$:
        \begin{gather}
            \int_G\expect{X\mid\mathcal{G}}\,dP = \int_GX\,dP.
        \end{gather}
    }
    \begin{remark}
        Although this construction was based on orthogonal projections, one could as well have used the (signed) Radon-Nikodym theorem \ref{lebesgue:signed_radon_nikodym} since $G\mapsto\int_GX\,dP$ is absolutely continuous with respect to $P|_{\mathcal{G}}$.
    \end{remark}

    \begin{property}\label{prob:conditional_expectation_props}
        Let $(\Omega,\Sigma,P)$ be a probability space and consider a sub-$\sigma$-algebra $\mathcal{G}\subset\Sigma$. If the random variable $X$ is $\mathcal{G}$-measurable, then
        \begin{gather}
            \expect{X\mid\mathcal{G}} = X\text{ a.s.}
        \end{gather}
        On the other hand, if $X$ is independent of $\mathcal{G}$, then
        \begin{gather}
            \expect{X\mid\mathcal{G}} = \expect{X}\text{ a.s.}
        \end{gather}
    \end{property}

\section{Joint distributions}

    \newdef{Joint distribution}{\index{distribution!joint}
        Let $X,Y$ be two random variables defined on the same probability space $(\Omega,\Sigma,P)$ and consider the vector random variable $(X,Y):\Omega\rightarrow\mathbb{R}^2$. The distribution of $(X,Y)$ isa probability measure defined on the Borel algebra of $\mathbb{R}^2$ defined by
        \begin{gather}
            P_{(X,Y)}(B) = P((X,Y)^{-1}(B)).
        \end{gather}
    }
    \newdef{Joint density}{
        If the probability measure from the previous definition can be written as
        \begin{gather}
            P_{(X,Y)}(B) = \int_Bf_{(X,Y)}(x,y)\,dxdy
        \end{gather}
        for some integrable $f_{(X,Y)}$, it is said that $X$ and $Y$ have a joint density.
    }

    \newdef{Marginal distribution}{\index{distribution!marginal}
        The distributions of the one-dimensional random variables is determined by the joint distribution:
        \begin{gather}
            P_X(A) = P_{(X,Y)}(A\times\mathbb{R})\\
            P_Y(A) = P_{(X,Y)}(\mathbb{R}\times A).
        \end{gather}
    }
    \begin{result}
        If the joint density exists, the marginal distributions are absolutely continuous and the associated density functions are given by
        \begin{gather}
            f_X(x) = \int_\mathbb{R}f_{(X,Y)}(x,y)\,dy\\
            f_Y(y) = \int_\mathbb{R}f_{(X,Y)}(x,y)\,dx.
        \end{gather}
        The converse, however, is not always true. The one-dimensional distributions can be absolutely continuous without the existence of a joint density.
    \end{result}

    \begin{property}[Independence]\index{independence}\label{prob:independent_densities}
        Let $X,Y$ be two random variables with joint distribution $P_{(X,Y)}$. $X$ and $Y$ are independent if and only if the joint distribution coincides with the product measure:
        \begin{gather}
            P_{(X,Y)} = P_X\otimes P_Y.
        \end{gather}
        If $X$ and $Y$ are absolutely continuous, the previous properties also applies to the densities instead of the distributions.
    \end{property}

    \begin{formula}[Sum of random variables]
        Consider two independent random variables $X,Y$ and let $Z=X+Y$ denote their sum. The density $f_Z$ is given by the following convolution:
        \begin{gather}
            f_Z(z) := f\ast g(z) = \int_\mathbb{R}g(x)h(z-x)\,dx = \int_\mathbb{R}g(z-y)h(y)\,dy,
        \end{gather}
        where $g,h$ denote the densities of $X,Y$ respectively.
    \end{formula}
    \begin{formula}[Product of random variables]
        Consider two independent random variables $X,Y$ and let $Z=XY$ denote their product. The density $f_Z$ is given by
        \begin{gather}
            f_Z(z) = \int_\mathbb{R}g(x)h(z/x)\,\frac{dx}{|x|} = \int_\mathbb{R}g(z/y)h(y)\,\frac{dy}{|y|},
        \end{gather}
        where $g,h$ denote the densities of $X,Y$ respectively.
    \end{formula}
    \begin{result}
        Taking the Mellin transform \ref{distributions:mellin} of both the positive and negative part of the above integrand (to be able to handle the absolute value) gives the following relation:
        \begin{gather}
            \mathcal{M}\{f\} = \mathcal{M}\{g\}\mathcal{M}\{h\}.
        \end{gather}
    \end{result}

    \newformula{Conditional density}{\index{conditional density}
        Let $X,Y$ be two random variables with joint density $f_{(X,Y)}$. The conditional density of $Y$ given $X\in A$ is
        \begin{gather}
            \label{prob:conditional_distribution}
            h(y\mid X\in A) = \frac{\int_Af_{(X,Y)}(x,y)\,dx}{\int_Af_X(x)\,dx}.
        \end{gather}
        For $X=\{a\}$ this equation is ill-defined since the denominator would become 0. However, it is possible to avoid this problem by formally setting
        \begin{gather}
            \label{prob:formal_conditional}
            h(y\mid A=a) := \frac{f_{(X,Y)}(a,y)}{f_X(a)},
        \end{gather}
        where $f_X(a)\neq0$. This last condition is nonrestrictive\marginpar{\dbend} because the probability of having a measurement $(X,Y)\in\{(x,y)\mid f_X(x) = 0\}$ is 0 (for nonsingular measures). One can thus define the conditional probability of $Y$ given $X=a$ as follows:
        \begin{gather}
            P(Y\in B\mid X=a) := \int_B h(y\mid X=a)\,dy.
        \end{gather}
    }

    \newformula{Conditional expectation}{\index{expectation!conditional}
        \begin{gather}
            \expect{Y\mid X}(\omega) = \int_\mathbb{R}yh(y\mid X(\omega))\,dy
        \end{gather}
        Let $\mathcal{F}_X$ denote the $\sigma$-algebra generated by the random variable $X$ as before. Using Fubini's theorem one can prove that for all sets $A\in\mathcal{F}_X$ the following equality holds:
        \begin{gather}
            \int_A\expect{Y\mid X}\,dP = \int_AY\,dP.
        \end{gather}
        This implies that the conditional expectation $\expect{Y\mid X}$ on $\mathcal{F}_X$ coincides with Definition \ref{prob:conditional_expectation}.
    }
    Applying Property \ref{prob:conditional_expectation_props} to the case $\mathcal{G}=\mathcal{F}_X$ gives the law of total expectation:
    \begin{property}[Law of total expectation\footnotemark]
        \footnotetext{Also called the \textbf{tower property}.}
        \begin{gather}
            \expect{\expect{Y\mid X}} = \expect{Y}
        \end{gather}
    \end{property}

    \begin{theorem}[Bayes's theorem]\index{Bayes}\label{prob:bayes_density}
        The conditional density can be computed without prior knowledge of the joint density:
        \begin{gather}
            g(x\mid y) = \frac{h(y\mid x)f_X(x)}{f_Y(y)}.
        \end{gather}
    \end{theorem}

\section{Stochastic calculus}

    \newdef{Stochastic process}{\index{stochastic!process}
        A sequence of random variables $\tseq{X}$ for some index set $T$. In practice $T$ will often be a totally ordered set, e.g. $(\mathbb{R},\leq)$ in the case of a time series. This will be assumed from here on.
    }

    \newdef{Filtered probability space}{\index{probability!space}
        Consider a probability space $(\Omega,\Sigma,P)$ together with a filtration \ref{set:filtration} of $\Sigma$, i.e. a collection of $\sigma$-algebras $\mathbb{F}\equiv\tseq{\mathbb{F}}$, such that $i\leq j\implies\mathbb{F}_i\subseteq\mathbb{F}_j$. The quadruple $(\Omega,\Sigma,\mathbb{F},P)$ is called a filtered probability space.

        Often the filtration is required to be exhaustive and separated (where $\emptyset$ is replaced by $\mathbb{F}_0=\{\emptyset,\Omega\}$ since any $\sigma$-algebra has to contain the total space).
    }

    \newdef{Adapted process}{\index{adapted!process}
        A stochastic process $\tseq{X}$ on a filtered probability space $(\Omega,\Sigma,\mathbb{F},P)$ is said to be adapted to the filtration $\mathbb{F}$ if $X_t$ is $\mathbb{F}_t$-measurable for all $t\in T$.
    }
    \newdef{Predictable process}{\index{predictable}
        A stochastic process $\tseq{X}$ on a filtered probability space $(\Omega,\Sigma,\mathbb{F},P)$ is said to be predictable if $X_{t+1}$ is $\mathbb{F}_t$-measurable for all $t\in T$.
    }

    \newdef{Stopping time}{\index{stopping time}
        Consider a random variable $\tau$ on filtered probability space $(\Omega,\Sigma,\mathbb{F},P)$ where the codomain of $\tau$ coincides with the index set of $\mathbb{F}$. This variable is called a stopping time for $\mathbb{F}$ if
        \begin{gather}
            \{\tau\leq t\}\in\mathbb{F}_t
        \end{gather}
        for all $t$. The stopping time is a ``time indicator'' that only depends on the knowledge of the process up to time $t\in T$.
    }

\subsection{Martingales}

    From here on the index set $T$ will be $\mathbb{R}_+\equiv[0,\infty[$ so that the index $t$ can be interpreted as a true time parameter. The discrete case $T=\mathbb{N}$ can be obtained as the restriction of most definitions or properties and, if necessary, this will be made explicit.

    \newdef{Martingale}{\index{martingale}\label{prob:martingale}
        Consider a filtered probability space $(\Omega,\Sigma,\mathbb{F},P)$. A stochastic process $\tseq{X}$ is called a martingale relative to $\mathbb{F}$ if it satisfies the following conditions:
        \begin{enumerate}
            \item $\tseq{X}$ is adapted to $\mathbb{F}$.
            \item Each random variable $X_t$ is integrable, i.e. $X_t\in L^1(P)$ for all $t\geq0$.
            \item For all $t>s\geq0:\expect{X_{t}\mid\mathbb{F}_s}=X_s$.
        \end{enumerate}
        If the equality in the last condition is replaced by the inequality $\leq$ (resp. $\geq$), the stochastic process is called a \textbf{supermartingale} (resp. \textbf{submartingale}).
    }
    \begin{example}[Doob martingale]\index{Doob!martingale}
        Consider an integrable random variable $X$ and a filtration $\mathbb{F}$. The associated Doob martingale (a martingale with respect to $\mathbb{F}$) is given by
        \begin{gather}
            Y_t := \expect{X\mid\mathbb{F}_t}.
        \end{gather}
    \end{example}

    \begin{property}[Doob-Ville inequality]\index{Doob!inequality}\index{Ville}\label{prob:doob_inequality}
        Consider a c\`adl\`ag submartinagle $\tseq{X}$.
        \begin{gather}
            \mathrm{Pr}\left(\sup_{t\leq\tau}X_t\geq C\right)\leq\frac{\expect{\max(0,X_\tau)}}{C}
        \end{gather}
        for all $C\geq1$ and $\tau\in T$.
    \end{property}

    The following property generalizes the Hoeffding inequalities \ref{prob:hoeffding_inequality}:
    \begin{property}[Hoeffding-Azuma inequality]\index{Hoeffding-Azuma inequality}\index{McDiarmid inequality}\label{prob:hoeffding_azuma}
        Let $\seq{X}$ be a (super)martingale with bounded differences, i.e. there exist constants $c_k>0$ such that
        \begin{gather}
            |X_k-X_{k-1}|\leq c_k.
        \end{gather}
        The following inequality holds for all $\lambda\geq0$:
        \begin{gather}
            \mathrm{Pr}(X_N-X_0\geq\lambda)\leq\exp\left(-\frac{\lambda^2}{2\sum_{i=1}^Nc_i^2}\right).
        \end{gather}
        A symmetric result for the lower tail holds for (sub)martingales. Moreover, if there exist predictable processes $\seq{A},\seq{B}$ such that
        \begin{gather}
            A_k\leq X_k-X_{k-1}\leq B_k
        \end{gather}
        and
        \begin{gather}
            B_k-A_k\leq c_k
        \end{gather}
        for all $k\in\mathbb{N}$, the inequality can be sharpened:
        \begin{gather}
            \mathrm{Pr}(X_N-X_0\geq\lambda)\leq\exp\left(-\frac{2\lambda^2}{\sum_{i=1}^Nc_i^2}\right).
        \end{gather}
        Now, consider a function $f:\Omega^n\rightarrow\mathbb{R}$ such that
        \begin{gather}
            \sup_{x_1,\ldots,x_n,x'_k}|f(x_1,\ldots,x_k,\ldots,x_n)-f(x_1,\ldots,x'_k,\ldots,x_n)|\leq c_k
        \end{gather}
        for all $k\in\mathbb{N}$. By applying the above inequalities to the Doob martingale
        \begin{gather}
            Z_m:=\expect{f(X_1,\ldots,X_n)\mid X_1,\ldots,X_m},
        \end{gather}
        one obtains the following inequality:
        \begin{gather}
            \mathrm{Pr}(f(X_1,\ldots,X_n)-\expect{f}\geq\lambda)\leq\exp\left(-\frac{2\lambda^2}{\sum_{i=1}^nc_i^2}\right).
        \end{gather}
        This inequality is sometimes called the \textbf{McDiarmid inequality}.
    \end{property}

    \begin{theorem}[Doob decomposition]\index{Doob!decomposition}
        Any integrable adapted process $\tseq{X}$ can be decomposed as $X_t=X_0+M_t+A_t$, where $\tseq{M}$ is a martingale and $\tseq{A}$ is a predictable process. These two processes are constructed iteratively as follows:
        \begin{align}
            A_0 = 0\qquad&\qquad M_0 = 0\\
            \Delta A_t = \expect{\Delta X_t\mid\mathbb{F}_{t-1}}\qquad&\qquad\Delta M_t = \Delta X_t - \Delta A_t.
        \end{align}
        Furthermore, $\tseq{X}$ is a submartingale if and only if $\tseq{A}$ is (almost surely) increasing.
    \end{theorem}
    \begin{result}\index{variation!quadratic}
        Consider the special case $X=Y^2$ for some martingale $Y$. One can show the following property:
        \begin{gather}
            \Delta A_t = \expect{(\Delta Y_t)^2\mid\mathbb{F}_{t-1}}\qquad\forall t\in\mathbb{R}_+.
        \end{gather}
        The process $\tseq{A}$ is often called the \textbf{quadratic variation process} of $\tseq{X}$ and is denoted by $\tseq{[X]}$.
    \end{result}

    \newdef{Discrete stochastic integral\footnotemark}{\index{integral!stochastic}\index{martingale!transform|see{integral, stochastic}}
        \footnotetext{Sometimes called the \textbf{martingale transform}.}
        Let $\seq{M}$ be a martingale on a filtered probability space $(\Omega,\Sigma,\mathbb{F},P)$ and let $\seq{X}$ be a predictable stochastic process with respect to $\mathbb{F}$. The (discrete) stochastic integral of $X$ with respect to $M$ is defined as follows:
        \begin{gather}
            (X\cdot M)_t(\omega) := \sum_{i=1}^tX(\omega)_i\Delta M_i(\omega),
        \end{gather}
        where $\omega\in\Omega$. For $t=0$ the convention $(X\cdot M)_0=0$ is used.
    }
    \begin{property}
        If the process $\seq{X}$ is bounded, the stochastic integral defines a martingale.
    \end{property}

    \begin{property}[It\^o isometry]\index{It\^o!isometry}
        Consider a martingale $\seq{M}$ and a predictable process $\seq{X}$. Using the Doob decomposition theorem one can show the following equality for all $n\geq0$:
        \begin{gather}
            \expect{\left(X\cdot M\right)_n^2} = \expect{(X^2\cdot[M])_n}.
        \end{gather}
    \end{property}
    It is this property that allows for the definition of integrals with respect to continuous martingales, since although the martingales are not in general of bounded variation (and hence do not induce a well-defined Lebesgue-Stieltjes integral), their quadratic variations are (e.g. the Wiener process).

\subsection{Markov processes}

    \newdef{Markov process}{\index{Markov!process}
        A Markov process (or chain) is a stochastic process $\tseq{X}$ adapted to a filtration $\tseq{\mathbb{F}}$ such that
        \begin{gather}
            P(X_t\mid\mathbb{F}_s) = P(X_t\mid X_s)
        \end{gather}
        for all $t,s\in T$. For discrete processes, the first-order Markov chains are the most common. These satisfy
        \begin{gather}
            P(X_t\mid X_{t-1},\ldots,X_{t-r}) = P(X_t\mid X_{t-1})
        \end{gather}
        for all $t,r\in\mathbb{N}$.
    }

\section{Information theory}

    \newdef{Self-information}{\index{information}
        The self-information of an event $x$ described by a distribution $P$ is defined as follows:
        \begin{gather}
            I(x) := -\ln P(x).
        \end{gather}
        This definition is modeled on the following (reasonable) requirements:
        \begin{itemize}
            \item Events that are almost surely going to happen, i.e. events $x$ such that $P(x)=1$, contain only little information: $I(x)=0$.\footnote{And by extension $P(x)\approx1\implies I(x)\approx0$.}
            \item Events that are very rare contain a lot of information.
            \item Independent events contribute additively to the information.
        \end{itemize}
    }
    \newdef{Shannon entropy}{\index{entropy!Shannon}\label{prob:shannon_entropy}
        The amount of uncertainty in a discrete distribution $P$ is characterized by its (Shannon) entropy
        \begin{gather}
            H(P) := \expect{I(X)} = -\sum_iP_i\ln(P_i).
        \end{gather}
    }

    \newdef{Kullback-Leibler divergence}{\index{Kullback-Leibler divergence}\index{entropy!relative}\label{prob:kullback_leibler}
        Let $P,Q$ be two probability distributions. The Kullback-Leibler divergence (or \textbf{relative entropy}) of $P$ with respect to $Q$ is defined as follows:
        \begin{gather}
            D_\mathrm{KL}(P\|Q) := \int_\Omega\log\left(\frac{P}{Q}\right)\,dP.
        \end{gather}
        This quantity can be interpreted as the information gained when using the distribution $P$ instead of $Q$. Instead of a base-10 logarithm, any other logarithm can be used since this simply changes the result by a (positive) scaling constant.
    }

    \begin{property}[Gibbs's inequality]
        By noting that the logarithm is a concave function and applying Jensen's equality \ref{calculus:jensen_inequality}, one can prove that the Kullback-Leibler divergence is nonnegative:
        \begin{gather}
            D_\mathrm{KL}(P\|Q)\geq0.
        \end{gather}
        Furthermore, the Kullback-Leibler divergence is zero if and only if $P$ and $Q$ are equal almost everywhere.
    \end{property}

\section{Extreme value theory}

    \newdef{Conditional excess}{
        Consider a random variable $X$ with distribution $P$. The conditional probability that $X$ is larger than a given threshold is given by the conditional excess distribution:
        \begin{gather}
            F_u(y) = \mathrm{Pr}(X-u\leq y\mid X>u) = \frac{P(u+y)-P(u)}{1-P(u)}.
        \end{gather}
    }

    \newdef{Extreme value distribution}{
        The extreme value distribution is given by the following formula:
        \begin{gather}
            F(x;\xi) = \exp\left(-(1+x\xi)^{-1/\xi}\right).
        \end{gather}
        In the case that $\xi=0$, one can use the definition of the Euler number to rewrite the definition as
        \begin{gather}
            F(x;0)=\exp(-e^{-x}).
        \end{gather}
        The number $\xi$ is called the \textbf{extreme value index}.
    }

    \newdef{Maximum domain of attraction}{
        The (maximum) domain of attraction of a distribution function $H$ consist of all distribution functions $F$ for which there exist sequences $(a_n>0)_{n\in\mathbb{N}}$ and $\seq{b}$ such that $F^n(a_nx+b_n)\longrightarrow H(x)$.
    }

    \begin{theorem}[Fischer, Tippett \& Gnedenko]
        Consider a sequence of i.i.d. random variables with distribution $F$. If $F$ lies in the domain of attraction of $G$, then $G$ has the form of an extreme value distribution.
    \end{theorem}

    \begin{theorem}[Pickands, Balkema \& de Haan]
        Consider a sequence of i.i.d. random variables with conditional excess distribution $F_u$. If the distribution $F$ lies in the domain of attraction of the extreme value distribution, the conditional excess distribution $F_u$ converges to the generalised Pareto distribution when $u\longrightarrow\infty$.
    \end{theorem}

\section{Copulas}

    \begin{property}[Uniformization transform]
        Consider a continuous random variable $X$ and let $U$ be the result of the probability integral transformation, i.e. $U:=F_X(X)$. This transformed random variable has a uniform cumulative distribution, i.e. $F_U(u)=u$.
    \end{property}

    \newdef{Copula}{\index{copula}
        The joint cumulative distribution function of a random variable with uniform marginal distributions.
    }
    The following alternative definition is more analytic in nature:
    \newadef{Copula}{
        A function $C:[0,1]^d\rightarrow[0,1]$ satisfying the following properties:
        \begin{enumerate}
            \item\textbf{Normalization} $C(x_1,\ldots,x_d)=0$ if any of the $x_i$ is zero.
            \item\textbf{Uniformity:} $C(1,1,\ldots,x_i,1,\ldots)=x_i$ for all $1\leq i\leq d$.
            \item\textbf{$d$-nondecreasing:} For every box $B=\prod_{1\leq i\leq d}[a_i,b_i]\subseteq[0,1]^d$ the $C$-volume is nonnegative:
            \begin{gather}
                \int_BdC := \sum_{\mathbf{z}\in\prod_i\{a_i,b_i\}}(-1)^{N_b(\mathbf{z})}C(\mathbf{z})\geq0,
            \end{gather}
            where $N_B(\mathbf{z}) = \mathrm{Card}(\{i\mid a_i=z_i\})$.
        \end{enumerate}
    }

    \begin{theorem}[Sklar]\index{Sklar}
        For every joint distribution function $H$ with marginals $F_i$ there exists a unique copula $C$ such that
        \begin{gather}
            H(x_1,\ldots,x_d) = C(F_1(x_1),\ldots,F_d(x_d)).
        \end{gather}
    \end{theorem}

    \begin{property}[Fr\'echet-Hoeffding bounds]\index{Fr\'echet-Hoeffding bounds}
        Every copula $C:[0,1]^d\rightarrow[0,1]$ is bounded in the following way:
        \begin{gather}
            \max\left(\sum_{i=1}^du_i-d+1,0\right)\leq C(u_1,\ldots,u_d)\leq \min_iu_i
        \end{gather}
        for all $(u_1,\ldots,u_d)\in[0,1]^d$. Furthermore, the upper bound is sharp, i.e.\ $\min_iu_i$ is itself a copula.\footnote{The lower bound is only a copula for $d=2$. In general this bound is only pointwise sharp.}
    \end{property}

    \newdef{Extreme value copula}{
        A copula $C$ for which there exists a copula $\widetilde{C}$ such that
        \begin{gather}
            \left[\widetilde{C}(u_1^{1/n},\ldots,u_d^{1/n})\right]^n\longrightarrow C(u_1,\ldots,u_d)
        \end{gather}
        for all $(u_1,\ldots,u_d)\in[0,1]^d$.
    }
    \begin{property}
        A copula $C$ is an extreme value copula if and only if it is stable in the following sense:
        \begin{gather}
            C(u_1,\ldots,u_d) = \left[C(u_1^{1/n},\ldots,u_d^{1/n})\right]^n
        \end{gather}
        for all $n\geq1$.
    \end{property}

\section{\difficult{Randomness}}

    This section is strongly related to Section \ref{section:turing} on computability theory.

    \newdef{Kolmogorov randomness}{\index{Kolmogorov!randomness}
        Consider a \textit{universal Turing machine} $U$. The \textbf{Kolmogorov complexity} $C(\kappa)$ of a finite bit string $\kappa$ (with respect to $U$) is defined as
        \begin{gather}
            C(\kappa) := \min\{|\sigma|\mid\sigma\text{ is finite}\land U(\sigma)=\kappa\}.
        \end{gather}
        A finite bit string is said to be Kolmogorov random (with respect to $U$) if there exists an integer $n\in\mathbb{N}$ such that $C(\kappa)\geq|\sigma|-n$.
    }

    \begin{property}
        For every universal Turing machine there exists at least one Kolmogorov random string. This easily follows from the pigeonhole principle since for every $n\in\mathbb{N}$ there are $2^n$ strings of length $n$ but only $2^n-1$ programs of length less than $n$.
    \end{property}
    \remark{Note that, although universal Turing machines can emulate each other, the randomness of a string is not absolute. Its randomness depends on the chosen machine.}

    It would be pleasing if this notion of randomness could easily be extended to infinite bit strings, for example by giving such a string the label random if there exists a uniform choice of constant $k$ such that all initial segments of the string are $k$-random. However, by a result of \textit{Martin-L\"of}, there does not exist any string satisfying this condition.

\section{Optimal transport}\label{section:optimal_transport}

    In this section a new notion of atomicity of measures will be used:
    \begin{definition}
        A measure on $\mathbb{R}^n$ is said to \textbf{give mass to small sets} if there exists a subset of \textit{Hausdorff dimension} $n-1$ (or smaller) that has nonzero measure.
    \end{definition}

\subsection{Kantorovich duality}

    The problem of optimal transport constitutes the search of the most cost efficient transportation scheme that connects a set of producers to a set of consumers. Assume that these are described by the probability spaces $(X,\Sigma_X,\mu_X)$ and $(Y,\Sigma_Y,\mu_Y)$, respectively.

    \newdef{Cost function}{\index{cost}
        A measurable function $X\times Y\rightarrow\overline{\mathbb{R}}$.
    }
    \newdef{Transportation scheme}{
        A transportation scheme or \textbf{transference plan} is a joint distribution $\pi\in\mathbb{P}(X\times Y)$ whose marginals coincide with $\mu_X$ and $\mu_Y$.
    }

    \newdef{Monge-Kantorovich problem}{\index{Monge-Kantorovich problem}
        The optimal transportation scheme for a given cost function according to \textit{Kantorovich} is the solution of the following optimization problem:
        \begin{gather}
            \inf_{\pi\in\mathbb{P}(X\times Y)}\mathrm{E}_\pi[c] = \inf_{\pi\in\mathbb{P}(X\times Y)}\int_{X\times Y}c(x,y)\,d\pi(x,y).
        \end{gather}
        The original problem of optimal transportation was considered by \textit{Monge}. However, he studied a restricted problem, where every producer only delivers to a unique consumer. In this case the joint distributions have a specific form, namely
        \begin{gather}
            \int_{X\times Y}c(x,y)\,d\pi(x,y) = \int_Xc(x,T(x))\,d\mu_X(x)
        \end{gather}
        for some measurable function $T:X\rightarrow Y$ such that $T_\ast\mu_X=\mu_Y$.
    }

    \begin{example}[Finite state spaces]
        Consider the case where both $X$ and $Y$ are finite of the same size and are both equipped with the uniform distribution. In this case the joint distributions $\pi$ can be represented by \textit{bistochastic matrices}, i.e. matrices with nonnegative entries such that every column and every row sums to one. This also implies that the optimization problem reduces to a linear problem on a convex, compact subset. This allows one to use Choquet's theorem \ref{functional:choquet} to restrict the attention to the extremal points, which in this case are given by permutation matrices. So, the optimal solution is given by the optimal one-to-one pairing of producers and consumers.
    \end{example}

    \begin{property}[Kantorovich duality]\index{Kantorovich!duality}\index{Kantorovich|seealso{Monge}}
        Let $X,Y$ be Polish spaces \ref{metric:polish_space} and consider a lower semicontinuous function $c:X\times Y\rightarrow\overline{\mathbb{R}^+}$ (Definition \ref{topology:semicontinuity}). Denote by $\mathbb{P}_\text{Borel}(\mu,\nu)$ the space of Borel measures on $X\times Y$ whose marginals are given by $\mu_X$ and $\mu_Y$. Moreover, denote by $\Phi_c\subseteq\mathcal{L}^1(X)\times\mathcal{L}^1(Y)$ the space of pairs of integrable functions satisfying
        \begin{gather}
            c_X(x)+c_Y(y)\leq c(x,y)
        \end{gather}
        for $\mu_X$-almost all $x\in X$ and $\mu_Y$-almost all $y\in Y$. Then
        \begin{gather}
            \inf_{\pi\in\mathbb{P}_\text{Borel}(\mu,\nu)}\int_{X\times Y}c\,d\pi = \sup_{(c_X,c_Y)\in\Phi_c}\int_Xc_X\,d\mu_X+\int_Yc_Y\,d\mu_Y
        \end{gather}
        and the this problem admits a solution. Moreover, one can restrict the space of would-be solutions on the right-hand side to those that are also bounded and continuous without changing the solution.
    \end{property}

    \newdef{Kantorovich distance}{\index{Kantorovich!distance}\index{Wasserstein metric}
        Let $X$ be a Polish space and consider a lower semicontinuous metric $d$ on $X$. The Kantorovich(-Rubinstein) distance $\mathcal{T}_d$ between two Borel probability measures $\mu,\nu$ on $X$ is defined as the optimal transport cost between them:
        \begin{gather}
            \mathcal{T}_d(\mu,\nu) := \inf_{\pi\in\mathbb{P}_\text{Borel}(X\times X)}\int_{X\times X}d(x,x')\,d\pi(x,x').
        \end{gather}
        If the metric $d$ is the one inducing the topology on $X$, one obtains the definition of the \textbf{Wasserstein 1-metric}.
    }

    \begin{theorem}[Kantorovich-Rubinstein]\index{Kantorovich-Rubinstein}\index{Lipschitz!norm}
        If $X=Y$ and $c$ is equal to some metric $d$ on $X$, the Kantorovich distance is given by
        \begin{gather}
            \mathcal{T}_d(\mu,\nu) = \sup\left\{\int_X\varphi\,d\mu-\int_X\varphi\,d\nu\,\middle\vert\,\varphi\in\mathrm{Lip}(X,d)\cap\mathcal{L}^1(\mu)\cap\mathcal{L}^1(\nu)\land\|\varphi\|_\mathrm{Lip}\leq1\right\},
        \end{gather}
        where
        \begin{gather}
            \|\varphi\|_\mathrm{Lip} := \sup_{x\neq x'\in X}\frac{|\varphi(x)-\varphi(x')|}{d(x,x')}
        \end{gather}
        is the \textbf{Lipschitz norm}.
    \end{theorem}

    \begin{property}[Translation invariance]
        The Kantorovich distance is invariant under translations by finite measures.
    \end{property}

    \begin{property}
        When $X=Y=\mathbb{R}^n$ with $d$ the Euclidean metric, the Kantorovich distance admits yet another description. In this case the Lipschitz norm is equal to the supremum norm of the gradient. This gives
        \begin{gather}
            \mathcal{T}_d(\mu,\nu) = \inf\left\{\|\sigma\|_1\,\middle\vert\,\nabla\cdot\sigma=\mu-\nu\right\},
        \end{gather}
        where the condition on $\sigma$ makes sense by the Riesz-Markov theorem \ref{distributions:riesz_markov}.
    \end{property}

\subsection{Convex costs}

    In this section cost functions of the form
    \begin{gather}
        c(x,y) = h(x-y)
    \end{gather}
    for some convex function $h:\mathbb{R}^n\rightarrow\mathbb{R}$ are considered. Moreover, the function $h$ will be assumed to be at least differentiable with locally Lipschitz gradient.

    \newdef{$c$-concave function}{\index{concave}
        A function $f:\mathbb{R}^n\rightarrow[-\infty,+\infty[$, not identically $-\infty$, is said to be $c$-concave if there exists a set $A\subset\mathbb{R}^n\times\mathbb{R}$ such that
        \begin{gather}
            f(x) = \inf_{(x',\lambda)\in A}c(x,x')+\lambda.
        \end{gather}
    }

    \begin{theorem}[Gangbo-McCann]\index{Gangbo-McCann}
        If $c$ is strictly convex and $\mu$ does not give mass to small sets, the Monge-Kantorovich problem has a a.s.~unique minimizer $\pi = (\mathbbm{1}\times T)_\ast\mu$ with
        \begin{gather}
            T(x) = x - (\nabla h)^{-1}(\nabla\psi(x))
        \end{gather}
        for some $h$-concave function $\psi:\mathbb{R}^n\rightarrow\overline{\mathbb{R}}$.
    \end{theorem}
    \begin{remark}
        If $h$ is a strictly convex function of the distance $\|x-y\|$, the theorem has to be modified:
        \begin{itemize}
            \item If $\mu\perp\nu$, the theorem still holds.
            \item If the measures are not singular, one has to restrict to transportation schemes that fix the shared mass. In effect, one removes the shared mass from the problem to recover the previous case.
        \end{itemize}
    \end{remark}
    Note that if $h$ is sufficiently differentiable, the inverse $\nabla h^{-1}$ is equal to the gradient of the Legendre transform by Property \ref{calculus:legendre_condition}.

\subsection{Concave costs}

    In this section cost functions of the form
    \begin{gather}
        c(x,y) = g(\|x-y\|)
    \end{gather}
    for some concave function $g:\mathbb{R}\rightarrow\mathbb{R}$ are considered.

    \begin{property}
        Let $c$ be strictly concave. If the transportation cost is not everywhere infinite and if $\mu$ does not give mass to small sets, then:
        \begin{itemize}
            \item If $\mu\perp\nu$, there exists a unique optimal transport scheme such that $\nu = T_\ast\mu$ with
            \begin{gather}
                T(x) = x - (\nabla g)^{-1}\nabla\varphi(x)
            \end{gather}
            for some \textit{$c$-concave function} $\varphi$.
            \item If the measures are not singular, there still exists a unique optimum by restricting to those schemes that fix shared mass.
        \end{itemize}
    \end{property}

\subsection{Densities}

    \begin{property}[Continuity equation]\index{continuity!equation}
        Let $X$ be a complete smooth manifold and consider a family $(T_t)_{0\leq t\leq1}$ of locally Lipschitz diffeomorphism on $X$ such that $T_0=\mathbbm{1}_X$ with associated vector fields $v_t$. If $\mu$ is a probability measure on $X$, the family $(\mu_t:=T_{t,\ast}\mu)_{0\leq t\leq 1}$ uniquely satisfies the \textbf{continuity equation}:
        \begin{gather}
            \pderiv{\mu_t}{t} + \nabla\cdot(\mu_tv_t) = 0,
        \end{gather}
        where the divergence of a measure is defined by duality.

        Let $v:\mathbb{R}^n\rightarrow\mathbb{R}^n$ be an almost everywhere smooth vector field. This induces a linear, constant velocity flow as follows:
        \begin{gather}
            T_t(x) := x - tv(x).
        \end{gather}
        If all $T_t$ are diffeomorphisms, the Eulerian velocity field $v_t(x):=T_t^{-1}\big(v(x)\big)$ satisfies the Eulerian continuity equation:
        \begin{gather}
            \pderiv{v_t}{t} + (v_t\cdot\nabla)v_t = 0.
        \end{gather}
    \end{property}
    \begin{formula}
        Given a solution of the continuity equation, the associated flow determines an optimal transport scheme for a cost function $c$ if and only if
        \begin{gather}
            v_0 = -(\nabla c)^{-1}\nabla\psi
        \end{gather}
        for some $c$-concave function $\psi$. Moreover, if $v_t = (\nabla c)^{-1}\nabla u$ for some function $u(t,x)$, then $u$ satisfies the \textbf{Hamilton-Jacobi equation} with Hamiltonian $c^*$:
        \begin{gather}
            \pderiv{u}{t} + c^*(\nabla u) = 0.
        \end{gather}
    \end{formula}

    In this section one considers absolutely continuous measures with respect to the Lebesgue measure on $\mathbb{R}^n$:
    \begin{gather}
        d\mu_X = \rho_0dx\qquad\qquad d\mu_Y = \rho_1dx.
    \end{gather}
    The transport cost in the Monge problem can then be rewritten as
    \begin{gather}
        \int_{\mathbb{R}^n}c(x,T(x))\rho_0(x)\,dx
    \end{gather}
    with
    \begin{gather}
        \int_{T^{-1}(A)}\rho_0(x)\,dx = \int_A\rho_1(x)\,dx
    \end{gather}
    for all measurable $A\subset\mathbb{R}^n$. By the change-of-variables formula this (weak) integral equation is equivalent to the Jacobian equation for
    \begin{gather}
        \det(DT(x))\rho_1(T(x))=\rho_0(x).
    \end{gather}

    \begin{example}[Euclidean metric]\index{Brenier map}
        If the cost function $c$ is the square of the Euclidean distance, the optimal transport mapping $T$, called the \textbf{Brenier map}, is given by the gradient of a convex potential:
        \begin{gather}
            T(x) = \nabla\varphi(x),
        \end{gather}
        and the optimal cost is equal to the square of the \textbf{Wasserstein 2-metric}:
        \begin{gather}
            \mathcal{T}_{\|\cdot\|_2^2}(\rho_0,\rho_1) = \inf_{\pi\in\mathbb{P}_\text{Borel}(\rho_0,\rho_1)}\int_{\mathbb{R}^n}\|x-x'\|\,d\pi(x,x')=W^2_2(\rho_0,\rho_1).
        \end{gather}
        Moreover, this minimum is unique a.e.

        It can also be shown that the flow acts affinely:
        \begin{gather}
            \sigma_t(x) = t\nabla\Phi(x) + (1-t)x.
        \end{gather}
    \end{example}

    In fact, the affinity of the flow can be shown more generally:
    \begin{property}
        Consider the time-dependent Monge-Kantorovich problem. If the differential cost $c:\mathbb{R}^n\rightarrow\mathbb{R}$ is strictly convex, the flows are given by straight lines:
        \begin{gather}
            x_t = x + t(x'-x).
        \end{gather}
        This situation can be generalized to (complete) smooth manifolds, where the minimizers of $\ell^p$-costs are geodesics with arc length parametrization.
    \end{property}

    It is possible to relate optimal transport to mechanics (Section \ref{section:analytical_mechanics}) in the following way:
    \begin{method}[Benamou-Brenier formulation]\index{Benamou-Brenier}
        Let $\rho_0$ and $\rho_1$ describe the density of particles in a system at time steps $t=0$ and $t=1$. Assume that there exists a time-dependent velocity field $v:\mathbb{R}\times\mathbb{R}^n\rightarrow\mathbb{R}^n$. These are related by the \textit{continuity equation} \ref{lagrange:eulerian_continuity_gather}:
        \begin{gather}
            \pderiv{\rho}{t} + \nabla\cdot(\rho v) = 0.
        \end{gather}
        The optimization problem now becomes minimizing the \textit{action} or \textit{kinetic energy}:
        \begin{gather}
            K(\rho,v) := \frac{1}{2}\int_{\mathbb{R}^n}\int_0^T\rho(t,x)\|v(t,x)\|^2\,dtdx.
        \end{gather}
        By making the change of variables $(\rho,v)\longrightarrow(\rho,m:=\rho v)$, one obtains a convex problem with a linear constraint (the continuity equation).
    \end{method}
    \begin{property}
        The infimum of the Benamou-Brenier action is equal (up to constant factors) to the square of the Wasserstein 2-metric and, hence, gives an equivalent characterization of the Monge-Kantorovich problem for the Euclidean distance.
    \end{property}
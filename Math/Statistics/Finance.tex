\chapter{Finance}

    The content of \crefrange{chapter:measure}{chapter:stochastic_calculus} is essential for this chapter. References for this chapter are~\citet{elliott_mathematics_2005,jeanblanc_mathematical_2009}

    \minitoc

\section{Securities}

    \newdef{Security}{\index{security}
        A tradeable financial asset.
    }

    \newdef{Bond}{\index{bond}\index{face!value}
        A security for which the issuer (debtor) owes the holder (creditor) a debt and has to provide a cash flow to the holder/creditor.

        When the average interest rate is set at 0, i.e.~only the \textbf{face value} is repaid at time of maturity, the bond is called a \textbf{zero-coupon bond} (or \textbf{discount bond}).
    }

    \newdef{Discounting}{\index{discounting}\index{risk!free}\label{finance:discounting}
        Consider an initial asset price $S_0$ and let $r:\mathbb{R}^+\rightarrow\mathbb{R}$ denote the continuous-time interest rate of the asset. The associated asset price $S:\mathbb{R}^+\rightarrow\mathbb{R}$ is given by
        \begin{gather}
            \label{finance:riskless_asset}
            S(t) := S_0\exp\left(\Int_0^tr(s)\,ds\right)\,.
        \end{gather}
        Any asset satisfying this equation or, equivalently, satisfying the SDE
        \begin{gather}
            \dr S = r_tS_t\dr t\,,
        \end{gather}
        is said to be \textbf{riskfree} or \textbf{riskless}. For discrete accruals, e.g.~annual interest rates, the correct formula is
        \begin{gather}
            S_n = S_0\prod_{i=1}^n(1+r_i)\,.
        \end{gather}
        For fixed interest rates $r\in\mathbb{R}$, this gives the classical formula
        \begin{gather}
            S_n = S_0(1+r)^n\,.
        \end{gather}
        The proportionality factor is often denoted by $B_t$, while its (multiplicative) inverse is denoted by $D_t:=B_t^{-1}$. The reason for this terminology is that $D_t$ is a \textbf{discount factor} and $B_t$ represents the value of a bond with initial value 1.
    }

    \newdef{Payoff}{\index{payoff}
        Consider an asset with initial price $P_0$ and interest rate process $\rseq{r}$. The payoff at time $t\in\mathbb{R}^+$, i.e.~the market value at time $t$, is given by
        \begin{gather}
            P_t = P_0D_t^{-1}\,.
        \end{gather}
        Again, for fixed (discrete) interest rates, this gives the classical formula
        \begin{gather}
            P_n = P_0(1+r)^n\,.
        \end{gather}
    }
    \begin{result}[Zero-coupon bond]
        Consider a zero-coupon bond with deterministic interest rate $\rseq{r}$ of maturity $T\in\mathbb{R}_+$. Its price at time $t\leq T$ is given by
        \begin{gather}
            P(t,T) = \exp\left(-\Intt{t}{T}r(s)\,ds\right)\,.
        \end{gather}
    \end{result}

    \newdef{At par}{\index{par}
        A coupon bond is said to be at par if its face value is equal to its initial price. This happens whenever the coupon rate is equal to the interest rate (when expressed in terms of the coupon frequency).
    }

\section{Instruments and derivatives}

    \newdef{Contingent claim}{\index{claim}\index{Arrow--Debreu!asset}
        A contingent claim (with time of maturity $T\in\mathbb{R}^+$) on a filtered probability space $(\Omega,\Sigma,\mathbb{F},P)$ is an $\Sigma$-measurable function that depends on the stochastic process $(S_t)_{t\in[0,T]}$. (Often, this function is also required to be square integrable.)
        
        \textbf{European claims} only depend on the asset value $S$ at maturity $T$, i.e.
        \begin{gather}
            X = f(S_T)
        \end{gather}
        for some measurable $f:\mathbb{R}^m\rightarrow\mathbb{R}^n$, i.e.~they are $\mathbb{F}_T$-measurable functions. Claims that are not European are said to be \textbf{American}.
        
        \finance{A (European) contingent claim, with maturity $T\in\mathbb{R}^+$, corresponds to a contract where the seller promises a random payoff at time $T$.}
    }

    \newdef{Forward}{\index{forward}
        A forward with forward price $K\in\mathbb{R}$ for an asset $\rseq{S}$ at maturity $T\in\mathbb{R}^+$ has payoff
        \begin{gather}
            \label{finance:forward}
            \Phi_T = S_T - K\,.
        \end{gather}
        I.e.~one buys one unit of the asset for a fixed price at time $T$.
    }

    \newdef{IR derivatives}{\index{cap}\index{floor}
        Caps and floors are derivatives for which the buyer receives a payment whenever the interest rate exceeds or stays under a certain threshold.
    }

    \newdef{Option}{\index{option}\index{call}\index{put}\index{swap}\index{forward}
        A derivative that allows (but not obliges) the holder to buy/sell an asset at a specified strike price on or before a specified date. Some examples are:
        \begin{itemize}
            \item\textbf{Call option}: An option to buy an asset.
            \item\textbf{Put option}: An option to sell an asset.
            \item\textbf{Swaption} (or swap option): An option for a \textbf{swap}, i.e.~the exchange of assets. 
        \end{itemize}
        Options are very similar to \textit{futures} and forwards\footnote{The main difference between futures and forwards is that the latter are less regulated, they are \textbf{over-the-counter} (OTC) contracts.}, although the latter two oblige the holder to perform the specified transaction.

        There are various families of options, depending on the specific ways the payoff is calculated. The main families are the \textbf{European} and \textbf{American options}. The former, as remarked above, can only be exercised at expiration date, while the latter can be exercised at any moment before the expiration date.

        Using \cref{finance:forward}, the payoff for a (European) call option is
        \begin{gather}
            \Phi_T =
            \begin{cases}
                S_T - K&\cif S_T\geq K\,,\\
                0&\text{otherwise}\,.
            \end{cases}
            = \max(S_T-K,0)\,.
        \end{gather}
    }
    \begin{remark}[Spots]\index{spot}
        When the settlement date (also called the \textbf{spot date}) of a trade is immediate, i.e.~usually two days after the trade, the contract is called a \textbf{spot} (contract).
    \end{remark}

    \newdef{Hedge}{\index{hedge}\label{finance:hedge}
        An investment position that tries to minimize potential losses. An example would be where an asset is acquired for the value $x\in\mathbb{R}^+$. If the value would go down, this would result in a loss. However, if at the same time, the investor also buys a put option for the assets, this loss can be offset.
    }

\section{Markets}
\subsection{Introduction}

    The predictable processes (\cref{stoch:predictable_process}) represent the trade strategies, i.e.~the amount of assets a trader has at a point $t\in\mathbb{R}^+$ in time, and the semimartingales (\cref{stoch:semimartingale}) represent asset prices (a combination of these two gives a \textbf{portfolio}). The value (or \textbf{wealth}) of a portfolio is given by the stochastic integral (\cref{section:stochastic_integral}):\index{portfolio}
    \begin{gather}
        \mathrm{Val}_t(\pi;S) := \pi\cdot S_t = \Intt{0}{t}\pi_s\,dS_s\,.
    \end{gather}
    An investor is said to be \textbf{short} if $\pi^i_t<0$ for some $i\leq d$ and $t\in\mathbb{R}_+$.

    \newdef{Self-financing portfolio}{\index{self-financing}
        A $d$-dimensional portfolio $(S,\pi)$ is said to be self-financing if the total wealth
        \begin{gather}
            \mathrm{Val}_t(\pi;S) := \sum_{i=1}^d\pi_t^iS_t^i
        \end{gather}
        satisfies the following condition
        \begin{gather}
            \mathrm{Val}_t(\pi;S) = V_0 + \sum_{i=1}^d\pi^i\cdot S_t^i
        \end{gather}
        or, in differential form,
        \begin{gather}
            \dr\mathrm{Val}_t(\pi;S) = \sum_{i=1}^d\pi^i_t\dr S^i_t\,.
        \end{gather}
        Essentially, this means that both ways of writing the dot product $\pi\cdot S$, one being the inner product and the other being the stochastic integral, coincide. Practically, this means that the value is completely driven by price fluctuations and that no external funding or withdrawal is allowed.
    }
    \begin{formula}
        For discrete-time portfolios, the self-financing condition can also be rewritten as follows:
        \begin{gather}
            \sum_{i=1}^d\pi_t^iS_t^i = \sum_{i=1}^d\pi_{t-1}^iS_t^i
        \end{gather}
        for all $t\in\mathbb{N}_0$.
    \end{formula}
    \begin{property}[Self-financing discount process]
        If $(S,\pi)$ is self-financing, so is the discounted portfolio $(D^0S,\pi)$. Note that discounting happens with respect to the first asset in $S$, which is assumed to be the `riskless' asset, i.e.~have a nonnegative, deterministic interest rate.

        Conversely, if
        \begin{gather}
            \dr\mathrm{Val}_t(\pi;S) = r_t\mathrm{Val}_t(\pi;S)\dr t + \pi_t(\dr S_t - r_tS_t\dr t)
        \end{gather}
        with $\mathrm{Val}_0(\pi;S)=\lambda$ for some $\lambda\in\mathbb{R}^+$, then $\bigl((1,S),(\mathrm{Val}(\pi;S)-\pi S,\pi)\bigr)$ is a self-financing portfolio.
    \end{property}

    \newdef{Replicating portfolio}{\index{replicating}\index{Greek}
        Two types of replication exist:
        \begin{itemize}
            \item A portfolio $(\pi,S)$ is said to be statically replicating for a set of assets $T$ when it has the same properties (e.g.~cash flow).
            \item A portfolio $(\pi,S)$ is said to be dynamically replicating for a set of assets $T$ when it has the same increments (`Greeks').
        \end{itemize}
    }

    \newdef{Num\'eraire}{\index{num\'eraire}\index{price}
        A nonnegative, adapted stochastic process $\rseq{\rho}$. The \textbf{relative price process} of any asset $\rseq{S}$ with respect to $\rho$ is defined as
        \begin{gather}
            S^\rho_t := \frac{S_t}{\rho_t}\,.
        \end{gather}
    }

    \newdef{Equivalent martingale measure}{\index{measure!equivalent martingale}\index{risk}\index{measure!risk-neutral}
        Consider an asset $\rseq{S}$ and a num\'eraire $\rseq{\rho}$ on a filtered probability space $(\Omega,\Sigma,\mathbb{F},P)$. An equivalent martingale measure (EMM) for $(S,\rho)$ is an equivalent measure $Q\sim P$ such that $\rseq{S^\rho}$ is an $(\mathbb{F},Q)$-(local) martingale.

        In case where $\rho_t=D_t$ is the discount process of some bond, the EMM is often called the \textbf{risk-neutral measure}. This measure turns the discounted process $\{D_tS_t\}_{t\in T}$ into a martingale.
    }
    \newdef{Equivalent separating measure}{\index{measure!equivalent separating}
        A set of admissible portfolios $\mathcal{X}$ (\cref{stoch:admissible_process}) on a probability space $(\Omega,\Sigma,P)$ is said to satisfy the equivalent separating measure (ESM) property when there exists an equivalent measure $Q\sim P$ such that
        \begin{gather}
            \mathrm{E}_Q[S_1]\leq 0
        \end{gather}
        for all $S\in\mathcal{X}$.
    }

    \begin{property}[Change of num\'eraire]
        Consider two num\'eraires $\rseq{N},\rseq{N'}$ and assume that they admit corresponding risk-neutral measures $P,Q$.
        \begin{gather}
            \left.\deriv{Q}{P}\right|_t = \frac{N_0}{N_t}\frac{N'_t}{N'_0}\,.
        \end{gather}
    \end{property}

    \begin{example}[Spot measure]\index{spot!measure}
        Consider the discounting factor in \cref{finance:discounting}. The \textbf{banking account num\'eraire} is given by accruing interests with a starting value of 1:
        \begin{gather}
            N_t = \exp\left(\Intt{0}{t}\rho(s)\,ds\right)\,.
        \end{gather}
        The associated EMM is called the spot measure.
    \end{example}
    \begin{example}[Forward measure]
        Consider a zero-coupon bond with maturity $T\in\mathbb{R}_+$ and a risk-neutral measure $\mu_*$. The price of the bond at time $t\leq T$ is given by
        \begin{gather}
            N_t = \expect{\exp\left(-\Intt{t}{T}\rho_s\,ds\right)\,\middle\vert\,\mathbb{F}_t} = P(t,T)\,.
        \end{gather}
        The associated EMM, after a change of num\'eraire, is called the $T$-forward measure:
        \begin{gather}
            \deriv{Q_T}{P} = \frac{D_T}{\mathrm{E}_{\mu_*}\left[D_T\right]}\,.
        \end{gather}
        If the interest rate $\rseq{\rho}$ is deterministic, this implies that the forward measure is simply the risk-free measure.

        Now, also consider some stopping tim $\tau$. The risky bond measure is obtained by taking the num\'eraire of the forward measure, but instead of taking the conditional expectation with respect to the given filtration, the expectation is taken with respect to the indicator function $\mathbbm{1}_{\tau>T}$:
        \begin{gather}
            \mathcal{N}_t = \expect{\exp\left(-\Intt{t}{T}\rho_s\,ds\right)\,\middle\vert\,\mathbbm{1}_{\tau>T}} = \mathcal{P}(t,T)\,.
        \end{gather}
    \end{example}
    \begin{example}\index{Sharpe ratio}\index{price}
        Assume that the stock price $\tseq{S}$ is a generalized Brownian motion:
        \begin{gather}
            \dr X_t = \alpha(t)S_t\,\dr t + \sigma(t)S_t\,\dr W_t\,.
        \end{gather}
        The discounted process can then be rewritten as
        \begin{gather}
            D_tS_t = \sigma(t)D_tS_t\bigl(\theta(t)\,\dr t + \dr W_t\bigr)\,,
        \end{gather}
        where
        \begin{gather}
            \theta(t) := \frac{\alpha(t) - R(t)}{\sigma(t)}
        \end{gather}
        is called the \textbf{market price of risk} or \textbf{Sharpe ratio}. By Girsanov's theorem~\ref{stoch:girsanov}, an EMM for this process is given by
        \begin{gather}
            \deriv{Q}{P} = \exp\left(-\Int_0^t\theta(s)\,dW_s - \frac{1}{2}\Int_0^t\theta^2(s)\,ds\right)\,.
        \end{gather}
    \end{example}

    \newdef{Superhedge}{\index{hedge}\label{stoch:superhedge}
        Recall \cref{finance:hedge}. A superhedge or $(x,H)$-hedge of a European claim $H$ is an admissible strategy $\rseq{\pi}$ such that, at maturity $T\in\mathbb{R}^+$,
        \begin{gather}
            \mathrm{Val}_T(\pi)\geq H
        \end{gather}
        with initial value
        \begin{gather}
            \mathrm{Val}_0(\pi) = x\,.
        \end{gather}
        A hedge for which $\mathrm{Val}_T(\pi)\geq H$ is called a \textbf{minimal hedge}. An example would be a replicating portfolio.

        For American claims, a hedge is called \textbf{minimal} if there exists a stopping time $\tau$ such that
        \begin{gather}
            \mathrm{Val}_\tau(\pi)\geq H_\tau\,.
        \end{gather}
    }

\subsection{Arbitrage}

    An \textit{arbitrage (opportunity)}, to be formally defined below, is the possibility of taking advantage of the difference in price on two markets, thereby obtaining a profit (without the need of having initial assets). To exclude arbitrage opportunities, it will be necessary to put constraints on the allowed portfolios. Consider for example a constant $\omega\in\mathbb{R}^+$. By Dudley's theorem~\ref{stoch:dudleys_theorem}, there exists a process $\rseq{\phi}$ such that
    \begin{gather}
        \omega = \Intt{0}{+\infty}\phi_s\,dW_s
    \end{gather}
    for a standard Brownian motion $\rseq{W}$. However, this integral exactly represents the final value of the portfolio $(W,\phi)$ and, hence, an arbitrage opportunity is obtained. Examples like these are called \textit{doubling strategies} (or \textit{martingale strategies}). Since these lead to unbounded portfolios, an admissibility criterion is needed to exclude them. The parameter $\lambda\in\mathbb{R}^+$ in the definition of admissible processes (\cref{stoch:admissible_process}) can be interpreted as a `credit line'.\index{credit}

    \newdef{Arbitrage}{\index{arbitrage}
        In the language of \cref{chapter:stochastic_calculus}, an arbitrage (opportunity) can be characterized as follows. Consider a semimartingale $\rseq{S}$ representing the asset prices. An arbitrage is an admissible strategy $\rseq{H}$ with $H_0=0$ a.s.~such that
        \begin{gather}
            H\cdot S_t\geq0\text{ a.s.}\qquad\text{and}\qquad\Prob\left(H\cdot S_t>0\right)>0\,.
        \end{gather}
    }

    \newdef{Viability}{\index{market!viable}
        A market in which no arbitrage opportunities arise is said to be \textbf{viable}.
    }

    \begin{remark}[Admissibility]
        One could wonder whether the requirement of being admissible is necessary. When relaxing this requirement and only asking the final payoff value $H\cdot S_T$ to be nonnegative, one obtains so-called \textbf{weak arbitrage opportunities}. It can be shown that every weak arbitrage can be modified to give an admissible strategy and, hence, an arbitrage. Restricting to admissible strategies is, accordingly, sufficient. However, if one would consider strategies with negative starting values, an additional condition to rule out arbitrage opportunities is necessary.
    \end{remark}

    No-arbitrage arguments are very common in (mathematical) finance. However, for the purpose of many theorems, the preceding definition is slightly too weak.
    \newdef{No free lunch}{\index{no free lunch}
        Consider an \'Emery-closed convex set $\mathcal{X}$ of semimartingales that start at 0 and are bounded from below. The most common choice here, would be the set of stochastic integrals $\rseq{H\cdot S}$ for some semimartingale $\rseq{S}$, where $H$ ranges over the admissible strategies. $\mathcal{X}$ is called the set of `admissible portfolios'. Denote the set of evaluations of $\mathcal{X}$ at the final time ($T=1$ by convention) by $\mathcal{K}$. Assume, moreover, that, if the `concatenation property'
        \begin{gather}
            Z_t := H\cdot S_t+G\cdot T_t \geq -1
        \end{gather}
        is satisfied for all nonnegative, bounded predictable strategies with $(GH)_t=0$, then $\rseq{Z}\in\mathcal{X}$.

        The \textbf{No Arbitrage} (NA) criterium can be restated as saying that $\mathcal{K}\cap L_{\geq0}^0(\Omega,P)=\{0\}$ or, if $\mathcal{C}$ denotes the linear difference $\bigl(\mathcal{K}\ominus L^0_{\geq0}(\Omega,P)\bigr)\cap L^\infty(\Omega,P)$, that
        \begin{gather}
            \mathcal{C}\cap L^\infty_{\geq0}(\Omega,P)=\{0\}\,.
        \end{gather}
        More generally, the set $\mathcal{X}$ is said to admit \textbf{No Free Lunch} (NFL) if
        \begin{gather}
            \overline{\mathcal{C}}\vphantom{\mathcal{C}}^*\cap L_{\geq0}^\infty(\Omega,P)\,,
        \end{gather}
        where $\overline{\mathcal{C}}\vphantom{\mathcal{C}}^*$ denotes the closure in the weak-$\ast$ topology (\cref{functional:weak_star_topology}). If this condition holds by passing to the norm closure (where convergence is determined by sequence instead of nets), hence in $L^\infty$, the set $\mathcal{X}$ is said to admit \textbf{No Free Lunch with Vanishing Risk} (NFLVR).

        Another notion is that of \textbf{No Unbounded Profit with Bounded Risk} (NUPBR), which says that $\mathcal{K}^1$ is bounded in $L^0(\Omega,P)$, where $\mathcal{K}^1$ is the subset of final asset prices that are generated by portfolios bounded from below by $-1$.
    }
    \begin{property}
        NFLVR is equivalent to NA and NUPBR combined.
    \end{property}

\subsection{Completeness}

    \newdef{Completeness}{\index{complete!market}\index{!Arrow--Debreu!market}\index{hedging}
        A market is said to be \textbf{complete} or to be an \textbf{Arrow--Debreu market} if every (European) contingent claim admits a replicating, self-financing portfolio. This portfolio is also called a \textbf{hedging strategy} and assets for which such a strategy exists, are said to be \textbf{hedgeable}. The hedging assets are $\bigl(S,(\mathrm{Val}_t(\pi;S)-\pi_tS_t,\pi)\bigr)$.
        
        On a finite state space $\Omega$, this corresponds to there being $|\Omega|$ linearly independent assets. These can be, for example, \textbf{Arrow--Debreu assets}\footnote{Also called \textbf{primitive} or \textbf{pure securities}}, i.e.~assets that pays of one unit of wealth for a given state and zero for all others.
    }

    \begin{theorem}[Fundamental theorems of asset pricing: Discrete markets]\index{fundamental theorem!of asset pricing}
        The following two statements relate EMMs and no-arbitrage criteria.
        \begin{enumerate}
            \item A discrete market on a discrete probability space ($\Omega,\Sigma,P)$ is arbitrage free if and only if there exists a risk-neutral measure $Q$. Moreover, $Q$ can be taken to have a bounded Radon--Nikodym derivative with respect to $P$.
            \item A market on a discrete probability space is complete and, in particular, arbitrage free if and only if there exists a unique risk-neutral measure.
        \end{enumerate}
    \end{theorem}

    \begin{theorem}[Fundamental theorems of asset pricing: Continuous markets]\index{fundamental theorem!of asset pricing}
        The following two statements relate EMMs and no-arbitrage criteria. Consider a market consisting of asset prices $\rseq{S}$.
        \begin{enumerate}
            \item Let $S$ be a locally bounded semimartingale. There exists a risk-neutral measure if and only if $\mathcal{X}=\{S\}$ satisfies NFLVR. Moreover, if $S$ is adapted and \cdlg and satisfies NFLVR, then it is a semimartingale.
            \item A market is complete and, in particular, arbitrage free if and only if there exists a unique risk-neutral measure.
        \end{enumerate}
    \end{theorem}

    \begin{property}[Arbitrage pricing\footnotemark]\index{price}
        \footnotetext{These formulas hold in any viable market. Completeness is only required to price contingent claims in terms of replicating portfolios.}
        Let $\rseq{N}$ be a num\'eraire as before and consider an asset price $\rseq{S}$ in a complete and arbitrage free market. If $Q$ is the unique risk-neutral measure with respect to $N$, the (relative) price of $S$ satisfies
        \begin{gather}
            S_s^N = \mathrm{E}_Q[S_t^N\mid\mathbb{F}_s]
        \end{gather}
        for all $s<t$ and, hence, the relative price process $\rseq{S^N}$ is a $Q$-martingale. In particular, the price of a (bounded) contingent claim $H$ at maturity $t\in\mathbb{R}^+$ is given by:
        \begin{gather}
            V_t^H = B_t\mathrm{E}_Q[D_TH\mid\mathbb{F}_t]\,,
        \end{gather}
        A market is arbitrage free if and only if this formula holds for all hedgeable contingent claims. If a contingent claim $H$ is not hedgeable, its set of \textbf{arbitrage-free prices}, i.e.~all initial values that do not give rise to an arbitrage opportunity when augmenting the assets $S$ with $H$, is given by the set
        \begin{gather}
            \Pi(H;S) = \{\mathrm{E}_Q[D_TH]\mid Q\in\mathcal{P}(S)\land \mathrm{E}_Q[H]<+\infty\}\,,
        \end{gather}
        where $\mathcal{P}(S)$ denotes the set of risk-free measures for $S$. It is not hard to show that this is either the empty set or an interval.
    \end{property}

    \begin{result}[Put-call parity]\index{parity!put-call}
        Consider (European) put and call options on some underlying asset with costs $P\in\mathbb{R}$ and $C\in\mathbb{R}$, respectively. If $S_0\in\mathbb{R}$ is the initial asset price, the interest rate is constant and the strike price $K\in\mathbb{R}$ and maturity time $T\in\mathbb{N}$ of the options are the same, the no-arbitrage condition implies that
        \begin{gather}
            P_t+S_t = C_t + K\beta^{-(T-t)}\,,
        \end{gather}
        where $\beta\in\mathbb{R}^+$ is the compounding factor (i.e.~$1+r$ for discrete compounding and $e^r$ for continuous compounding).\footnote{This also holds for risk-free interest rates after replacing the factor $\beta^{-(T-t)}$ by a factor $\beta_{t,T}$ that takes into account the changing interest rate.}
    \end{result}
    \begin{remark}[Equity and debt]\index{Modigliani--Miller}
        This result can also be interpreted in terms of firm values, where it is equivalent to the \textit{Modigliani--Miller theorem}.
    \end{remark}

    \begin{property}[Minimal hedge]
        In a complete market, the buy and sell prices of European claims coincide and, moreover, the replicating strategy gives a minimal hedge (\cref{stoch:superhedge}).
    \end{property}

    \begin{remark}[Martingale representation]
        Recall \cref{stoch:prp}. Comparing this to the definition of complete markets, it can be seen that market completeness implies that the normalized asset prices $\{S^1/S^0,\ldots,S^d/S^0\}$ satisfy the predictable representation property and form a \textit{martingale basis}:
        \begin{gather}
            H/S^0_t = H_0 + \sum_{i=1}^d\pi^i\cdot\left(\frac{S^i}{S^0}\right)_t\,.
        \end{gather}
    \end{remark}

    \begin{remark}[Adding information]
        A corollary of \cref{stoch:prp_refinement} is that, if a market $S$ is complete (i.e.~satisfies the PRP) and $\mathbb{F}_t\neq\mathbb{G}_t\cap\mathbb{F}_T$ for some $t\leq T$ or $\mathbb{F}_T=\mathbb{G}_T$, for a refinement $\mathbb{F}\hookrightarrow\mathbb{G}$ and maturity $T\in\mathbb{R}^+$, then there are no equivalent martingale laws for $S$ with respect to $\mathbb{G}$.
        
        \finance{Refinement of filtrations corresponds to adding information. By the aforementioned result, this means that, if one adds information without changing the payoff, arbitrage opportunities arise.}
    \end{remark}

    \newdef{Perfect market}{\index{market!perfect}
        A market is said to be perfect if it satisfies the following two conditions:
        \begin{enumerate}
            \item\textbf{Frictionless}: Transactions and costs are negligible and and information is perfect.
            \item\textbf{Liquidity}: Every asset has a price and amounts are unbounded.
            \item 
        \end{enumerate}

        \todo{COMPLETE (add other conditions (e.g. completeness))}
    }

\section{Models}
\subsection{Va\u{s}\'i\u{c}ek model}\index{Va\u{s}\'i\u{c}ek model}

    The \textit{Merton model} models defaults by comparing the `firm value' to a threshold induced by the probability of default. Assume there are $n\in\mathbb{N}$ obligors and let $PD_i$ denote the probability of default (either modelled or known) for the $i^{\text{th}}$ obligor. The firm value for this obligor is given by
    \begin{gather}
        V_i = \sqrt{\rho_i}Y + \sqrt{1-\rho}\varepsilon_i\,,
    \end{gather}
    where:
    \begin{itemize}
        \item $Y$ is a standard normally distributed \textbf{systematic factor}, which models the common contributions of all obligors\footnote{For example, macro-economic variables.},
        \item $\rho$, which models the asset correlation, and
        \item $\varepsilon_i$ is a standard normally distributed \textbf{idiosyncratic factor} (i.i.d.~with respect to the other $\varepsilon_j$), which models the intrinsic contributions of obligor $i$.
    \end{itemize}
    The relevance of the distinction between systematic and idiosyncratic contributions is that, conditional on $Y$, the firm values of the different obligors are independent. Under the Va\u{s}\'i\u{c}ek model, the probability of default $i$, conditional on the systemic factor $Y=y$, is given by
    \begin{gather}
        p_i(y) = \Phi\left(\frac{\Phi^{-1}(PD_i)-\sqrt{\rho}y}{\sqrt{1-\rho}}\right)\,.
    \end{gather}
    In practice, one samples a standard normally distributed random variable for each obligor and compares the value to the threshold in between parentheses.

\subsection{Black--Scholes model}\index{Black--Scholes model}

    In this section, all price processes will be assumed to be It\^o processes (\cref{stoch:ito_process}). Moreover, the filtration on $(\Omega,\Sigma,P)$ will be assumed to be generated by the $d$-dimensional Brownian motion $\rseq{W}$ underlying the price process:
    \begin{gather}
        \dr S^i = \Xi^i(S)\dr t + \sum_{j=1}^dC^{ij}(S)\dr W^j\,.
    \end{gather}

    One component of the assets in the Black--Scholes model is given by bank account $S^1\equiv B$, which satisfies
    \begin{gather}
        \dr B = r B\dr t
    \end{gather}
    and $B_0=1$. The (constant) parameter $r>0$ is called the \textbf{(risk-free) interest rate}.\index{rate!interest} The second component $S^2\equiv S$ is given by a risky asset, where the drift and diffusion functions ($\Xi$ and $C$) are linear in the asset price:
    \begin{gather}
        \dr S = \mu S\dr t + \sigma S\dr W\,.
    \end{gather}
    This has the form of the SDE defining the stochastic exponential (\cref{stoch:stochastic_exponential}) and, hence, \cref{stoch:continuous_stochastic_exponential} gives an explicit expression for the stock price:
    \begin{gather}
        S_t = S_0\exp\left(\sigma W_t + \left(\mu-\frac{\sigma^2}{2}\right)t\right)\,.
    \end{gather}
    Solutions of the SDE, which represents an It\^o process with linear coefficients, are said to follow a \textbf{geometric Brownian motion}.\index{Brownian motion!geometric} 

    \begin{formula}[Black--Scholes--Merton]
        \begin{gather}
            \pderiv{f}{t} + rS\pderiv{f}{S} + \frac{1}{2}\sigma^2S^2\mpderiv{2}{f}{S} = rf\,.
        \end{gather}
    \end{formula}
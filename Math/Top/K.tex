\chapter{\texorpdfstring{\difficult{$K$-theory}}{K-theory}}\index{K-theory}\label{chapter:k}

    The main reference for this chapter is~\citet{karoubi_k-theory_1978}.

    In this chapter all topological (base) spaces are supposed to be both compact and Hausdorff (unless stated otherwise). This ensures that the complex of $K$-theories satisfies the Eilenberg--Steenrod axioms (\cref{topology:eilenberg_steenrod_axioms}). In general all statements will given for an arbitrary base field $k$. When it is necessary to restrict to the specific cases $k=\mathbb{R},\mathbb{C}$, this will be explicitly stated.

\section{Preliminaries}

    \newdef{Stable general linear group}{\index{stable!group}\label{k:stable_group}
        Let $R$ be a (unital) ring. For every two integers $m<n$ there exists a canonical inclusion $\GL(m,R)\hookrightarrow\GL(n,R)$ through extension (direct sum) by $\mathbbm{1}_{n-m}$. This allows to define the stable general linear group as the direct limit
        \begin{gather}
            \GL(R) := \varinjlim_{n\in\mathbb{N}}\GL(n,R)\,.
        \end{gather}
    }
    \remark{A similar construction leads to the stable orthogonal and stable unitary groups $\mathrm{O}$ and $\mathrm{U}$ when $R=\mathbb{R}$ and $R=\mathbb{C}$, respectively. Sometimes the notations $\mathrm{O}(\infty)$ and $\mathrm{U}(\infty)$ are also used.

\section{\texorpdfstring{Topological $K$-theory}{Topological K-theory}}
\subsection{Introduction}

    \newdef{$K$-theory}{
        Let $\text{Vect}(X)/\sim$ be the set of isomorphism classes of finite-dimensional vector bundles over a base space $X$. Because this set is well-behaved with respect to Whitney sums, the pair $(\mathrm{Vect}(X)/\sim,\oplus)$ is an Abelian monoid. The Grothendieck completion (\cref{group:grothendieck_completion}) of this monoid is called the (real) $K$-theory of $X$.
    }
    \begin{notation}
        \nomenclature[S_K]{$K^0(X)$}{$K$-theory over a (compact Hausdorff) space $X$}
        The $K$-theory of a space $X$ is denoted by $K^0(X)$. If one wants to emphasize the base field $k$, the notation $K^0_k(X)$ is sometimes used, e.g.~for complex vector bundles one writes $K^0_{\mathbb{C}}(X)$. However, because the real and complex numbers are the most important choices, these two $K$-theories are simply denoted by $K\!O(X)$ and $K(X)$, respectively.
    \end{notation}

    \begin{example}[Point]\label{k:point}
        Let $\ast$ be the one-point space. The $K$-theory $K^0(\ast)$ is isomorphic to the additive group of integers $(\mathbb{Z},+)$.
    \end{example}

    \newdef{Virtual vector bundle}{\index{virtual!vector bundle}\label{k:virtual_bundle}
        The elements of $K^0(X)$ are pairs $([E],[E'])$ that can formally be written as a difference $[E]-[E']$ of (isomorphism classes of) vector bundles. Such pairs are called virtual (vector) bundles.
    }
    \newdef{Virtual rank}{\index{virtual!rank}\index{rank!of a vector bundle}
        The virtual rank of the virtual bundle $([E],[E'])$ is defined as follows:
        \begin{gather}
            \rk([E],[E']) := \rk(E) - \rk(E')\,.
        \end{gather}
    }

    \begin{property}
        \Cref{bundle:hausdorff} implies that every virtual bundle has a representant of the form $[E]-[X\times\mathbb{R}^n]$ for some vector bundle $E$ and integer $n\in\mathbb{N}$.
    \end{property}

    \newdef{Reduced $K$-theory}{
        Let $(X,x_0)$ be a pointed space. The inclusion $\{x_0\}\hookrightarrow X$ induces a group morphism $\rho:K^0(X)\rightarrow K^0(x_0)$ given by the restriction of virtual bundles to the basepoint $x_0$. The reduced $K$-theory $\widetilde{K}^0(X)$ is defined as $\ker(\rho)$.
    }

    \begin{adefinition}
        The reduced $K$-theory $\widetilde{K}^0(X)$ can equivalently be defined as follows. Consider the stable isomorphism classes (\cref{bundle:stable_isomorphism}) of vector bundles over $X$. Under Whitney sums these form a commutative group $(\mathrm{Vect}(X)/\sim_\text{stable},\oplus)$, which is (naturally) isomorphic to $\widetilde{K}^0(X)$.
    \end{adefinition}

    The following construction is very similar to \cref{clifford:functor_k_group}. In fact, it is the one obtained for the Banach functor that restricts vector bundles to subspaces.
    \newdef{Relative $K$-theory}{\label{k:relative}
        Consider a space $X$ and a closed subspace $Y$. Let $\mathscr{V}(X,Y)$ denote the set of triples $(E,E',f)$ where $E,E'$ are vector bundles over $X$ and $f$ is an isomorphism between the restrictions $E|_Y$ and $E'|_Y$. Elements in $\mathscr{V}(X,Y)$ are deemed isomorphic if there exist isomorphisms of vector bundles that make the ``obvious'' diagram commute. The sum of such triples is defined elementwise. Let $\mathscr{E}(X,Y)$ denote the subset of $\mathscr{V}(X,Y)$ consisting of triples $(F,F',g)$ where $F=F'$ and $g$ is homotopic to $\mathbbm{1}_{F|_Y}$ in $\Aut(F|_Y)$.

        The relative $K$-theory $K^0(X,Y)$ is defined as the quotient of $\mathscr{V}(X,Y)$ by the following equivalence relation:
        \begin{gather}
            x\sim x' \iff \exists e,e'\in\mathscr{E}(X,Y):x+e\cong_{\mathscr{V}}x'+e'\,.
        \end{gather}
        Elements of relative $K$-theory are pairs of vector bundles over $X$ that coincide on the subspace $Y$ modulo a relation akin to that in the Grothendieck construction. It should also be clear that choosing $Y=\emptyset$ gives exactly the Grothendieck construction and, hence, $K^0(X,\emptyset)\equiv K^0(X)$.
    }
    \begin{property}
        Consider a space $X$ with a closed subspace $Y$.
        \begin{gather}
            K^0(X,Y)\cong\ker\left(K^0(X)\rightarrow K^0(Y)\right)
        \end{gather}
    \end{property}
    \begin{property}[Excision]\index{excision}\label{k:excision}
        Consider a space $X$ together with a closed subspace $Y$. The relative $K$-theory is related to the reduced $K$-theory as follows:
        \begin{gather}
            K^0(X,Y)\cong\widetilde{K}^0(X/Y)\,.
        \end{gather}
    \end{property}

\subsection{Classification}

    \begin{property}[Orthogonal groups]\index{classifying!space}
        Consider the classifying space (\cref{bundle:classifying_space}) of the orthogonal group $\mathrm{O}(n)$ and recall the Grassmannian $\mathrm{Gr}(n,\mathbb{R}^N)$ of $n$-dimensional subspaces in $\mathbb{R}^k$. There exists a canonical inclusion of Grassmannians:
        \begin{gather}
            \iota_k:\mathrm{Gr}(n,\mathbb{R}^k)\hookrightarrow\mathrm{Gr}(n,\mathbb{R}^{k+1}):W\mapsto W\,.
        \end{gather}
        By taking the direct limit of these inclusions, one obtains the infinite Grassmannian:
        \begin{gather}
            B\mathrm{O}(n) := \varinjlim_{k\in\mathbb{N}}\mathrm{Gr}(n,\mathbb{R}^k)\,.
        \end{gather}
        As the notation implies, it can be shown that this is the classifying space of $\mathrm{O}(n)$.

        There also exists a canonical inclusion of Grassmannians
        \begin{gather}
            \iota_{n,k}:\mathrm{Gr}(n,\mathbb{R}^k)\hookrightarrow \mathrm{Gr}(n+1,\mathbb{R}^{k+1}):W\mapsto W\oplus\mathrm{span}\{e_{k+1}\}\,.
        \end{gather}
        This in turn induces an inclusion $B\mathrm{O}(n)\hookrightarrow B\mathrm{O}(n+1)$ of classifying spaces. The direct limit of this system of inclusions is denoted by $B\mathrm{O}$. It is the classifying space of the stable orthogonal group $\mathrm{O}$.
    \end{property}
    \begin{remark}
        A similar construction exists for the classifying spaces $B\mathrm{U}(n)$ and $B\mathrm{U}$.
    \end{remark}
    \begin{remark}[Kuiper's theorem]\index{Kuiper}\label{k:kuiper_remark}
        It should be noted that neither $B\mathrm{O}$ nor $B\mathrm{U}$ can be expressed as classifying spaces of automorphism groups of infinite-dimensional Hilbert spaces. This follows from \textit{Kuiper's theorem}, which states that such groups are contractible and, hence, have vanishing homotopy groups (this clearly does not hold for the classifying spaces in this section).
    \end{remark}

    \begin{property}[Homotopy classification]
        For all spaces $X$ the $K$-theory can be represented as follows:
        \begin{gather}
            K^0_k(X) = [X,B\!\GL(k)\times\mathbb{Z}]\,.
        \end{gather}
        When specializing to $k=\mathbb{R},\mathbb{C}$ this gives
        \begin{gather}
            KO^0(X) = [X,B\mathrm{O}\times\mathbb{Z}]
        \end{gather}
        and
        \begin{gather}
            K^0(X) = [X,B\mathrm{U}\times\mathbb{Z}]
        \end{gather}
        due to homotopy invariance. Reduced $K$-theory can be obtained by considering basepoint-preserving homotopies, which for connected spaces gives $[X,B\!\GL(k)]$.
    \end{property}

    \begin{remark}[Noncompact spaces]
        When considering noncompact spaces one can still use either the Grothendieck construction or the representable definition for topological $K$-theory. However, these will not coincide anymore even though there does exist an injection from Grothendieck $K$-theory to representable $K$-theory.
    \end{remark}

    The following theorem should be compared to \cref{k:kuiper_remark} above (in fact this theorem can be proven through \textit{Kuiper's theorem}).
    \begin{theorem}[Atiyah--J\"anich]\index{Atiyah--J\"anich}\label{k:atiyah_janich}
        The space of Fredholm operators (\cref{functional:fredholm}) on a separable, infinite-dimensional Hilbert space forms a classifying space for $K$-theory.
    \end{theorem}

\subsection{Negative degree}

    \newdef{$K^{-1}$}{
        For every field $k$ the functor $K^{-1}$ is defined as follows (this can again be obtained as a property when using a different definition):
        \begin{gather}
            K^{-1}(X,Y) := [X/Y,\GL(k)]_*\,,
        \end{gather}
        where the asterisk denotes the fact that only basepoint-preserving homotopies are considered. $K^{-1}(X)$ can be obtained by considering $Y=\emptyset$ and recalling \cref{topology:empty_quotient}:
        \begin{gather}
            K^{-1}_k(X) := [X,\GL(k)]\,.
        \end{gather}
        For $k=\mathbb{R},\mathbb{C}$ one can use homotopy invariance to obtain
        \begin{align}
            K\!O^{-1}(X) &:= [X,\mathrm{O}]\,,\\
            K^{-1}(X) &:= [X,\mathrm{U}]\,.
        \end{align}
    }

    To define lower degree groups, it will be useful to extend $K$-theory to locally compact spaces.
    \newdef{$K$-theory of locally compact spaces}{\label{k:locally_compact}
        Let $X$ be a locally compact space and denote its one-point compactification (\cref{topology:alexandrov_compactification}) by $\widehat{X}$. The groups $K^0(X)$ and $K^{-1}(X)$ are defined as follows:
        \begin{align}
            K^0(X) &:= \ker\left(K^0(\widehat{X})\rightarrow K^0(\{\infty\})\right)\label{k:locally_compact_formula}\,,\\
            K^{-1}(X) &:= \ker\left(K^{-1}(\widehat{X})\rightarrow K^{-1}(\{\infty\})\right)\,.
        \end{align}
        So, the $K$-theory of a locally compact space is defined as the reduced $K$-theory of its one-point compactification.
    }
    \begin{result}[Relative $K$-theory and complements]
        Consider a space $X$ with a closed subspace $Y$. One can identify $(X/Y)\backslash\{y_0\}$ with $X\backslash Y$ and, hence, one obtains
        \begin{gather}
            K^0(X\backslash Y) = \widetilde{K}^0(X/Y)\,.
        \end{gather}
        When combined with the excision property this gives a result akin to ordinary (singular) cohomology, where the relative cocycles were those defined on the complement $X\backslash Y$:
        \begin{gather}
            \label{k:relative_and_complement}
            K^0(X,Y)\cong K^0(X\backslash Y)\,.
        \end{gather}
    \end{result}

    \newdef{$K^{-n}$}{\label{k:lower_groups}
        Lower-degree relative $K$-groups are defined as follows:
        \begin{gather}
            \label{k:lower_relative}
            K^{-n}(X,Y) := K^0\bigl((X\backslash Y)\times\mathbb{R}^n\bigr)\,.
        \end{gather}
        By taking $Y=\emptyset$ one obtains the groups $K^{-n}(X)$ as $K^0$-groups of trivial line bundles:
        \begin{gather}
            K^{-n}(X) := K^0(X\times\mathbb{R}^n)\,.
        \end{gather}
        Before relating this to reduced $K$-theory, the reader should be aware of a possible confusion. Homotopy invariance of $K$-theory would seem to imply that the above definition is senseless, since $X\cong X\times\mathbb{R}^n$ in the homotopy category. However, $X\times\mathbb{R}^n$ is not compact (even if $X$ is) and, hence, one should work with \cref{k:locally_compact}.

        It can be shown through a series of homeomorphisms that the above definition is equivalent to the following one:
        \begin{gather}
            \label{k:relative_reduced}
            K^{-n}(X,Y)\cong\widetilde{K}^0\bigl(\Sigma^n(X/Y)\bigr)\,,
        \end{gather}
        where $\Sigma$ denotes the reduced suspension functor \ref{topology:suspension}. As such, the reduced suspension functor gives a way to move down in the tower of $K$-groups.
    }

\subsection{Bott periodicity}\index{Bott!periodicity}

    \newdef{Cup product}{\index{cup product}
        First, generalize the Whitney sum and tensor product constructions to vector bundles over different base spaces. Let $E,E'$ be vector bundles over the base spaces $B,B'$ and consider the projection maps $\pi:B\times B'\rightarrow B$ and $\pi':B\times B'\rightarrow B'$. The exterior sum $E\oplus E'\rightarrow B\times B'$ is defined as the Whitney sum $\pi^*(E)\oplus\pi'^*(E')$. Analogously, the exterior product bundle is defined as the tensor product $\pi^*(E)\otimes\pi'^*(E')$. Fibrewise, this is just the ordinary direct sum and tensor product of vector spaces.

        The exterior product induces a bilinear map on $K$-theory as follows. From \cref{k:virtual_bundle}, it follows that every element $x\in K^0(X)$ can be written as formal difference $[E]-[E']$ of vector bundles over $X$. Using this decomposition one can define the cup product $x\cup y$ through the following formula:
        \begin{gather}
            ([E]-[E'])\cup([F]-[F']) := [E\otimes F] + [E'\otimes F'] - [E\otimes F'] - [E'\otimes F]\,.
        \end{gather}
        This definition can now be extended to locally compact spaces. Every element of $K^0(Y)$, for $Y$ locally compact, defines an element in $K^0(\widehat{Y})$ and for such elements the cup product was just defined. By restricting to $Y\times Y'$ one can then obtain an element of $K^0(Y\times Y')$:
        \begin{gather}
            K^0(Y)\times K^0(Y')\overset{\iota\times\iota'}{\longrightarrow}K^0(\widehat{Y})\times K^0(\widehat{Y'})\overset{\cup}{\longrightarrow}K^0(\widehat{Y}\times\widehat{Y'})\overset{\mathrm{res}}{\longrightarrow}K^0(Y\times Y')\,,
        \end{gather}
        where $\iota,\iota'$ are the inclusions induced by \cref{k:locally_compact_formula}.
    }
    \begin{property}[Ring structure]
        By precomposing with the diagonal morphism $K^0(X\times X)\rightarrow K^0(X)$, the set $K^0(X)$ can be endowed with a commutative ring structure (at least over commutative fields such as $\mathbb{R},\mathbb{C}$).
    \end{property}

    \begin{property}
        It can immediately be seen that the cup product on $K^0$ also defines a bilinear operation $K^{-m}(X)\times K^{-n}(X)\rightarrow K^{-m-n}(X)$. Furthermore, as above, this induces a multiplicative structure on the complex $K^\bullet(X):=\bigoplus_{n=0}^\infty K^{-n}(X)$. This multiplication can be shown to endow the $K$-complex with the structure of a graded-commutative algebra (\cref{hda:graded_commutative}).

        By using the isomorphism~\eqref{k:relative_and_complement} one can also extend the cup product to an operation on relative $K$-theory:
        \begin{gather}
            K^{-m}(X,Y)\rightarrow K^{-n}(X',Y')\rightarrow K^{-m-n}(X\times X',X\times Y'\cup X'\times Y)\,.
        \end{gather}
    \end{property}

    \begin{notation}[Bott element]\index{Bott!element}
        Consider the complex relative $K$-group \[K^0_{\mathbb{C}}(D^2,S^1)\cong\widetilde{K}^0_{\mathbb{C}}(S^2)\cong K^0_{\mathbb{C}}(\mathbb{R}^2)\] of the unit disk with respect to its boundary. The Bott element represented by the triple $\left(D^2\times\mathbb{C}^2,D^2\times\mathbb{C}^2,\alpha:(x,v)\mapsto xv\right)$ will be denoted by $\beta$. In terms of virtual vector bundles it is represented by the difference of the trivial line bundle and the tautological line bundle on the 2-sphere.
    \end{notation}
    \begin{theorem}[Complex Bott periodicity]
        The cup product with the Bott element $\beta$ gives an isomorphism $K^{-n}(X,Y)\cong K^{-n}(X\times D^2,X\times S^1\cup Y\times D^2)\cong K^{-n-2}(X,Y)$. This also implies that cupping with $\beta$ gives an isomorphism $K^0(X)\cong K^0(X\times\mathbb{R}^2)$.
    \end{theorem}
    \begin{result}
        Applying Bott periodicity to the case $X=\ast,Y=\emptyset$ and comparing to Example \ref{k:point}, one obtains $K^0(D^2,S^1)\cong\mathbb{Z}$. One can also conclude that $\beta$ is a generator of $K^0(D^2,S^1)$.
    \end{result}
    \begin{result}[Spheres]
        Bott periodicity and \cref{k:relative_reduced} also allow to compute the reduced $K$-theory of spheres:
        \begin{gather}
            \widetilde{K}^0(S^n) =
            \begin{cases}
                0&n\text{ odd}\,,\\
                \mathbb{Z}&n\text{ even}\,.
            \end{cases}
        \end{gather}
        For even $n$, one can see that the generator is given by $\beta^{n/2}$.
    \end{result}

    \begin{property}[Homotopy groups of unitary group]\label{k:homotopy_group_U}
        \Cref{bundle:vector_bundles_over_sphere} can be generalized to the stable linear group to obtain an isomorphism $\widetilde{K}^0_k(S^n)\cong\pi_{n-1}\bigl(\GL(k)\bigr)$. By specializing to $k=\mathbb{C}$, recalling that $\GL(m,\mathbb{C})$ deformation retracts onto $\mathrm{U}(m)$ and applying Bott periodicity, one obtains that the homotopy groups of the stable unitary group are mod 2-periodic. Furthermore, by using the long exact sequence induced by the fibration
        \begin{gather}
            \mathrm{U}(n)\rightarrow\mathrm{U}(n+1)\rightarrow S^{2n+1}\,,
        \end{gather}
        one can show that for $n>i/2$ the homotopy groups $\pi_i\big(\mathrm{U}(n)\big)$ satisfy the same periodicity relation.
    \end{property}

    \begin{theorem}[Real Bott periodicity]
        The cup product with the real Bott element, i.e.~the generator of $K\!O^{-8}(\ast)\cong\mathbb{Z}$, gives an isomorphism
        \begin{gather}
            K\!O^{-n}(X,Y)\rightarrow K\!O^{-n-8}(X,Y)\,.
        \end{gather}
    \end{theorem}

    \begin{theorem}[Weak Bott periodicity]
        The following spaces are homotopy equivalent:\footnote{For an extensive list see~\citet{karoubi_k-theory_1978}.}
        \begin{align}
            \GL(\mathbb{R})&\sim\Omega^8\GL(\mathbb{R})\\
            \mathrm{O}&\sim\Omega^8\mathrm{O}\\
            \mathrm{U}&\sim\Omega(\mathbb{Z}\times B\mathrm{U})\\
            \mathbb{Z}\times B\mathrm{U}&\sim\Omega\mathrm{U}\,.
        \end{align}
    \end{theorem}
    \begin{result}
        Through Eckmann--Hilton duality, this implies the periodicity of the homotopy groups of the stable orthogonal and unitary groups (cf.~\cref{k:homotopy_group_U}).
    \end{result}
    \begin{property}
        One can also relate real and quaternionic $K$-theory through the following homotopy equivalences:
        \begin{align}
            \mathbb{Z}\times B\!\GL(\mathbb{R})&\sim\Omega^4(B\!\GL(\mathbb{H}))\,,\\
            \mathbb{Z}\times B\!\GL(\mathbb{H})&\sim\Omega^4(B\!\GL(\mathbb{R}))\,.
        \end{align}
    \end{property}

    \begin{remark}[Positive degree]
        Bott periodicity allows to define $K$-groups in positive degree.
    \end{remark}

\subsection{Clifford modules}\index{Clifford!module}

    One can restate the previous sections in terms of Clifford modules (\cref{riemann:clifford_module}), i.e.~vector bundles for which the fibres carry a representation of a Clifford algebra. The content of \cref{section:clifford_bott} will be used here.

    From \cref{clifford:functor_k_group} and \cref{clifford:k00}, it should be clear that what was called $K^0(X)$ is in fact equivalent to $K^{0,0}\bigl(\mathbf{Vect}(X)\bigr)$. In a similar vein one can prove that $K^{-1}(X)$ is equivalent to $K^{1,0}\bigl(\mathbf{Vect}(X)\bigr)$. This relation is in fact generalizable to all values of $p,q$.
    \begin{property}
        The $K^0(X)$-modules $K^{p,q}\bigl(\mathbf{Vect}(X)\bigr)$ and $K^{q-p}(X)$ are isomorphic.
    \end{property}

    \Cref{clifford:bott_periodicity_category} then implies the Bott periodicity for the groups $K^{q-p}(X)$. For complex $K$-theory one can use \cref{clifford:complex_bott_periodicity}. The most important takeaway for this section is that one can rephrase $K$-theory in terms of Clifford modules and canonically induced functors between them.

\subsection{Cohomology theory}

    \begin{property}[Excision]\index{excision}
        It can be shown that the excision property~\ref{k:excision} holds at every degree $n\in\mathbb{Z}$:
        \begin{gather}
            K^n(X/Y,\{y_0\})\cong K^n(X,Y)\,.
        \end{gather}
        More generally this can be stated as
        \begin{gather}
            K^n(X\backslash U,Y\backslash U)\cong K^n(X,Y)\,,
        \end{gather}
        where $\overline{U}\subset Y^\circ$.
    \end{property}
    \begin{property}[Homotopy invariance]
        Homotopic maps induce equal morphisms in $K$-theory at every degree. In particular this implies that homotopy equivalences induce isomorphisms in $K$-theory.
    \end{property}

    \begin{remark}[Generalized cohomology]
        The above properties imply that the complex of $K$-groups satisfies the Eilenberg--Steenrod axioms (\cref{topology:eilenberg_steenrod_axioms}) except for the dimension axiom. As such, it is a generalized cohomology theory.
    \end{remark}

\subsection{Applications}

    @@ REFER TO PHYSICS @@

    \begin{property}[Degree]\index{degree!of map}
        Recall \cref{topology:degree} of the degree of a function in algebraic topology. In a similar way one can assign to every continuous function $f:S^n\rightarrow S^n$ a degree through its induced action on $K$-theory. It can be shown that the topological degree and the $K$-theoretic degree coincide.

        One can also extend this to \textbf{multidegrees}. For example in the case of bidegree one considers continuous functions $\mu:S^n\times S^n\rightarrow S^n$. The bidegree $(p,q)$ of $\mu$ is defined as the pair of degrees of the maps $x\mapsto\mu(x,x_0)$ and $y\mapsto(x_0,y)$ for a fixed basepoint $x_0$.
    \end{property}

    \newdef{$H$-space}{\index{H-!space}
        A sphere $S^n$ is said to be an $H$-space if it admits a continuous function $\mu:S^n\times S^n\rightarrow S^n$ of bidegree $(1,1)$, i.e.~it admits such a function for which the pointwise maps are homotopic to the identity. (This second formulation can also be used for other topological spaces, e.g.~the $H$-structure on loop groups (\cref{topology:h_structure}).)
    }

    \begin{property}[Puppe sequence]\index{Puppe sequence}
        The mapping cone $C_f$ from \cref{topology:mapping_cylinder} fits in an exact sequence:
        \begin{gather}
            X\longrightarrow Y\longrightarrow C_f\longrightarrow\Sigma X\longrightarrow\Sigma Y\,.
        \end{gather}
        Reduced $K$-theory maps this to an induced exact sequence (whenever the reduced $K$-theory is defined for $X,Y$):
        \begin{gather}
            \widetilde{K}^0(\Sigma X)\longrightarrow\widetilde{K}^0(\Sigma Y)\longrightarrow \widetilde{K}^0(C_f)\longrightarrow\widetilde{K}^0(X)\longrightarrow\widetilde{K}^0(Y)\,.
        \end{gather}
    \end{property}
    The Puppe sequence in $K$-theory allows to prove an important theorem.
    \newdef{Hopf invariant}{\index{Hopf!invariant}
         The Puppe sequence for $f:S^{2n-1}\rightarrow S^n$ with $n$ even implies that $\widetilde{K}^0(C_f)\cong\mathbb{Z}\oplus\mathbb{Z}$, where the generators are induced by the Bott elements of the spheres. The relation $\beta_{2n}=\beta_n\cup\beta_n$ of Bott elements gives an induced relation $a^2=\lambda b$ of generators in $\widetilde{K}^0(C_f)$. The integer $\lambda$ is called the Hopf invariant of $f$.\footnote{The choice of generator $a$ corresponding to $\beta_n$ is not relevant due to the relations $b^2=ab=0$ obtained by dimensional arguments.}
     }
     One can show that every `multiplication' map $\mu:S^{n-1}\times S^{n-1}\rightarrow S^{n-1}$ of bidegree $(p,q)$ induces a map $S^{2n-1}\rightarrow S^n$ of Hopf invariant $pq$.

     \begin{theorem}[Atiyah--Adams]\index{Atiyah--Adams}
         Let $n$ be even. If a continuous function $S^{2n-1}\rightarrow S^n$ has odd Hopf invariant, then $n=2,4$ or 8.
     \end{theorem}
     \begin{result}
         A sphere $S^{n-1}$ can only admit a multiplication map of bidegree $(1,1)$ or, equivalently, admit an $H$-structure, if $n=2,4$ or 8. (The case $S^0$ can be proven in a different way.)
     \end{result}

    Using Chern--Weil theory (\cref{section:chern_weil}) one can define a morphism to ordinary cohomology.
    \begin{construct}[Chern character]\index{Chern!character}\label{k:chern_character}
        The Chern character of vector bundles (\cref{bundle:chern_character}) induces a Chern character on $K$-theory:
        \begin{gather}
            \mathrm{ch}:K(X)\mapsto H^\bullet(M):[E]-[F]\mapsto \mathrm{ch}(\nabla^E)-\mathrm{ch}(\nabla^F)\,,
        \end{gather}
        where $\nabla^E,\nabla^F$ are connections on $E$ and $F$, respectively. This morphism is a ring morphism (with respect to Whitney sums and tensor products). For relative $K$-theory classes $[E,F,\alpha]\in K(M,N)$ one should pick connections that are related by $\nabla^E|_N=\alpha^*(\nabla^F|_N)$.
    \end{construct}
    \begin{theorem}[Atiyah--Hirzebruch]\index{Atiyah--Hirzebruch}
        Consider a base space $X$. The Chern character induces the following isomorphisms:
        \begin{align}
            K^0_{\mathbb{C}}(X)\otimes\mathbb{Q}&\cong\bigoplus_{i\in\mathbb{Z}}H^{2i}(X;\mathbb{Q})\,,\\
            K^1_{\mathbb{C}}(X)\otimes\mathbb{Q}&\cong\bigoplus_{i\in\mathbb{Z}}H^{2i+1}(X;\mathbb{Q})\,.
        \end{align}
        For noncompact spaces, one needs to work with rational $K$-theory.
    \end{theorem}

    \begin{property}[$K\!K$-theory]\label{k:topological_kk_theory}
        Topological $K$-theory and Kasparov $K$-theory (\cref{operators:kk_theory}) are related as follows:
        \begin{align}
            K\!K\bigl(\mathbb{C},C(X)\bigr)&\cong K^0(X)\,,\\
            K\!K\bigl(C(\mathbb{R}),C(X)\bigr)&\cong K^1(X)\,.
        \end{align}
    \end{property}

\section{\texorpdfstring{Twisted $K$-theory}{Twisted K-theory}}\index{K-theory!twisted}\label{section:twisted_k_theory}

    The most straightforward way to introduce twisted $K$-theory is similar to that of twisted de Rham theory (\cref{bundle:twisted_de_rham}). Namely by introducing a degree-3 integral cohomology class or, equivalently, by twisting by a $\mathrm{U}(1)$-\textit{bundle gerbe} (see \cref{chapter:hdg}).

    One way to obtain such a twist is the following approach. By the Atiyah--J\"anich theorem, $K$-theory is classified by Fredholm operators on a separable infinite-dimensional Hilbert space $\mathcal{H}$. Now, consider a fibre bundle $\pi:P\rightarrow M$ with typical fibre the projective space $\mathcal{H}\mathbb{P}$. Locally, such a bundle can be obtained as the (fibrewise) projectivization of a Hilbert bundle. By lifting the $\mathrm{PGL}(\mathbb{C})$-valued transition functions to $\mathrm{GL}(\mathbb{C})$-valued functions, one obtains a $\mathrm{U}(1)$-valued 3-cocycle $f_{ijk}:=\widetilde{g}_{ij}\widetilde{g}_{jk}\widetilde{g}_{ki}$. This defines a class in $\check{H}^2(M;\mathrm{U}(1))\cong H^3(M;\mathbb{Z})$. Isomorphism classes of bundles of projective spaces are in bijection with such cohomology classes and if the class vanishes, the bundle can be globally obtained as the fibrewise (projectivization) of a Hilbert bundle. Given that the projective unitary group also acts on Fredholm operators (by conjugation), one can for every class $\alpha\in H^3(M;\mathbb{Z})$ construct a bundle of Fredholm operators $\mathrm{Fred}(\mathcal{H})\hookrightarrow\mathrm{Fred}(P)\rightarrow M$.

    \newdef{Twisted $K$-theory}{
        Consider a cohomology class $\alpha\in H^3(M;\mathbb{Z})$ together with its associated bundle of Fredholm operators $\mathrm{Fred}(P)\rightarrow M$. The $\alpha$-twisted $K$-theory on $M$ is defined as the set of homotopy classes of sections of this bundle:
        \begin{gather}
            K^0_\alpha(M) := \Gamma\bigl(\mathrm{Fred}(P)\bigr)/\sim.
        \end{gather}
    }

\section{\texorpdfstring{Differential $K$-theory}{Differential K-theory}}

    Just as in the case of ordinary cohomology (\cref{section:differential_cohomology}), one can also generalize ordinary $K$-theory to a differential cohomology theory by incorporating connection data. One of the definitions of $K$-theory makes use of (virtual) vector bundles. Therefore, the most natural way to define differential $K$-theory would be through (virtual) vector bundles with connection. This is the approach of \textit{Simons} and \textit{Sullivan}.

    \newdef{Simons--Sullivan model}{\index{Simons--Sullivan!model}
        A vector bundle equipped with an equivalence class of connections, where two connections are equivalent if there Chern-Simons form is exact, is called a \textbf{structured bundle}. Two structured bundles are isomorphic if there exists a vector bundle isomorphism that identifies the connections.

        Consider the set of $\mathrm{Struct}(M)$ of isomorphism classes of structured bundles over $M$. This set becomes a commutative monoid under Whitney sums. The Grothendieck completion (\cref{group:grothendieck_completion}) of $\mathrm{Struct}(M)$ gives a model for differential $K$-theory:
        \begin{gather}
            \widehat{K}^0(M) := G\bigl(\mathrm{Struct}(M)\bigr)\,.
        \end{gather}
    }

    @@ COMPLETE @@

\section{\texorpdfstring{Algebraic $K$-theory}{Algebraic K-theory}}
\subsection{Determinant}

    Over noncommutative rings $R$ the determinant of a matrix is not as easily defined as over commutative rings. For example, in the $2\times2$-case one could choose either $ad-bc$ or $da-bc$ (or any other permutation). There exists no canonical choice. To fix this issue one can take a look at the most important properties of the determinant map:
    \begin{itemize}
        \item It is invariant under elementary row/column operations (item 3 of \cref{linalgebra:determinant_properties}).
        \item It is invariant under augmentation by the identity, i.e.~under the transformation $A\mapsto A\oplus\mathbbm{1}$.
    \end{itemize}
    To implement the second property it will be necessary to move from the finite-dimensional general linear group $\GL(n,R)$ to its stable version (\cref{k:stable_group}). On this group one can define an equivalence relation by saying that two matrices are equivalent if they belong to the same coset with respect to the subgroup $E(R)$ generated by the elementary matrices \ref{linalgebra:elementary_matrix}. It can also be shown that $E(R)$ is equal to the commutator subgroup $[\GL(R),\GL(R)]$.

    The determinant map is then abstractly defined as the quotient map from the following definition.
    \newdef{$K_1$}{\index{determinant}
        The first algebraic $K$-group of a ring $R$ is defined as the Abelianization of its stable linear group:
        \begin{gather}
            K_1(R) := \GL(R)/[\GL(R),\GL(R)]\,.
        \end{gather}
        The quotient map $\pi:\GL(R)\rightarrow K_1(R)$ is called the \textbf{determinant map}.
    }

    To obtain lower $K$-groups a `suspension functor' needs to be defined.
    \newdef{Suspension}{\index{suspension}
        Let $R$ be a ring. Denote the infinite matrix ring over $R$ by $M(R)$, i.e.~the set of matrices with a finite number of nonzero entries in each row and column. This ring contains an ideal $M_{\text{fin}}(R)$ generated by all matrices that are zero outside a block of finite size. The suspension of $R$ is then defined as follows:
        \begin{gather}
            \Sigma R := M(R)/M_{\text{fin}}(R)\,.
        \end{gather}
    }
    \newdef{Lower $K$-groups}{
        For all integers $n\geq1$ one defines the $K$-groups as follows:
        \begin{gather}
            K_n(R) := K_{1-n}(\Sigma^nR)\,.
        \end{gather}
    }

    \begin{example}[$K_0$]
        It can be shown that $K_0(R)$ corresponds to the Grothendieck group associated to the monoid of finitely-generated projective $R$-modules. The relation to topological $K$-theory is then given by the Serre--Swan theorem~\ref{bundle:serre_swan}:
        \begin{gather}
            K_0\bigl(C(X)\bigr)\cong K^0(X).\,
        \end{gather}
    \end{example}

\section{\texorpdfstring{Operator $K$-theory}{Operator K-theory}}\index{K-theory!operator}

    \newdef{$K_0$}{
        Let $A$ be a unital $C^*$-algebra. The group $K_0(A)$ is defined as the Grothendieck completion of the monoid with the projections in $M_n(A)$ for all $n\in\mathbb{N}$ as generators and the relations
        \begin{enumerate}
            \item $[p\oplus q]=[p]+[q]$.
            \item $[0]=0$.
            \item If $p,q$ are connected by a continuous path of projections, then $[p]=[q]$.
        \end{enumerate}
        The first two conditions imply in particular that the embedding of a matrix as the upper-left part of a larger matrix leaves the class unchanged.
    }
    \begin{remark}
        One could replace homotopy equivalence by Murray-Von Neumann equivalence in this definition, since two projections $p$ and $q$ are Murray-Von Neumann-equivalent if and only if the matrices
        \begin{gather}
            \begin{pmatrix}
                p&0\\0&0
            \end{pmatrix}
            \qquad\text{and}\qquad
            \begin{pmatrix}
                q&0\\0&0
            \end{pmatrix}
        \end{gather}
        are homotopy-equivalent.
    \end{remark}

    \begin{property}[Topological $K$-theory]\label{k:serre_swan}
        If $X$ is a compact Hausdorff space, then
        \begin{gather}
            K^0(X)\cong K_0\bigl(C(X)\bigr)\,.
        \end{gather}
        This isomorphism is induced by the Serre--Swan theorem~\ref{bundle:serre_swan}.
    \end{property}

    \newdef{$K_1$}{
        Let $A$ be a unital $C^*$-algebra. The group $K_1(A)$ is defined as the Grothendieck completion of the monoid with the unitaries in $M_n(A)$ for all $n\in\mathbb{N}$ as generators and the relations
        \begin{enumerate}
            \item $[p\oplus q]=[p]+[q]$.
            \item $[\mathbbm{1}]=0$.
            \item If $p,q$ are connected by a continuous path of unitaries, then $[p]=[q]$.
        \end{enumerate}
        The first two conditions imply in particular that the embedding of a matrix as the upper-left part of a larger matrix leaves the class unchanged.
    }

    \begin{property}[Bott periodicity]
        First, define the suspension of a $C^*$-algebra $A$ as follows:
        \begin{gather}
            \Sigma A := \{f\in C([0,1],A)\mid f(0)=f(1)=0\}\,.
        \end{gather}
        Then, $K_1(A)\cong K_0(\Sigma A)$. Moreover, one defines all higher homology groups recursively as\footnote{There exists a more general definition of operator $K$-theory for which this relation is a property.}
        \begin{gather}
            K_n(A)\cong K_{n-1}(\Sigma A)\,.
        \end{gather}
    \end{property}

    \begin{property}[$K\!K$-theory]
        If $A$ is a $C^*$-algebra, then operator $K$-theory and Kasparov $K$-theory (\cref{operators:kk_theory}) are related as follows:
        \begin{align}
            K\!K(\mathbb{C},A)&\cong K_0(A)\,,\\
            K\!K\bigl(C(\mathbb{R}),A\bigr)&\cong K_1(A)\,.
        \end{align}
        \Cref{k:topological_kk_theory} is then simply a consequence of this isomorphism by \cref{k:serre_swan}.
    \end{property}
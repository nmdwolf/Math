\chapter{Sheaf Theory}\label{chapter:sheaf}

    A reference aimed towards the study of differential geometry is~\citet{brylinski_loop_1993}. For the concept of a sheaf in category theory, see \cref{section:grothendieck_topos}.

\section{Presheaves}

    \newdef{Presheaf}{\index{presheaf}\index{section!of a (pre)sheaf}
        Let $(X,\tau)$ be a topological space. A presheaf on $X$ consists of a choice of algebraic structure $\mathcal{F}(U)$ for every open set $U\in\tau$ and a morphism $\Phi^U_V:\mathcal{F}(U)\rightarrow \mathcal{F}(V)$ for every two open sets $U,V\in\tau$ with $V\subseteq U$ such that the following conditions are satisfied:
        \begin{enumerate}
            \item $\Phi^U_U = \mathbbm{1}_{\mathcal{F}(U)}$, and
            \item $W\subseteq V\subseteq U\implies\Phi^U_W = \Phi^V_W\circ\Phi^U_V$.
        \end{enumerate}
        The set $\mathcal{F}(U)$ is called the set of \textbf{sections} over $U$ and the morphisms $\Phi^U_V$ are called the \textbf{restriction maps}.
    }
    \newdef{Morphism of presheaves}{\index{morphism!of presheaves}\index{germ}
        Let $\mathcal{F},\mathcal{F}'$ be two presheaves on a topological space $X$. A morphism $\mathcal{F}\rightarrow\mathcal{F}'$ is a set of morphisms $\Psi_U:\mathcal{F}(U)\rightarrow\mathcal{F}'(U)$ that commute with the restriction maps $\Phi^U_V$.
    }

    \begin{adefinition}[Categorical definition]
        Using the language of category theory one can give a more concise definition. Let $\mathbf{C}$ be a category and let $X$ be a topological space. A $\mathbf{C}$-valued presheaf on $X$ is a contravariant functor $\mathcal{F}:\mathbf{Open}(X)\rightarrow\mathbf{C}$. A morphism of presheaves is a natural transformation between such functors. As such the category of presheaves on a topological space $X$ is in fact the presheaf topos on $\mathbf{Open}(X)$.
    \end{adefinition}

    \begin{example}[Constant presheaf]\label{sheaf:constant_presheaf}
        Let $S$ be any set. The constant presheaf on $X$ with target $S$ is defined by
        \begin{gather}
            \mathcal{F}(U) := S
        \end{gather}
        for every open set $U\subseteq X$.
    \end{example}

\section{Sheaves}

    \newdef{Sheaf}{\index{sheaf}\label{sheaf:def}
        Let $(X,\tau)$ be a topological space. A sheaf on $X$ is a presheaf $\mathcal{F}$ satisfying the following conditions:
        \begin{enumerate}
            \item\textbf{Locality (or separation)}: Let $\{U_i\}_{i\in I}\subset\tau$ be an open cover of $U\subseteq X$ and consider sections $s,t\in\mathcal{F}(U)$. If $\forall i\in I:s|_{U_i}=t|_{U_i}$, then $s=t$. This is equivalent to saying that that $\mathcal{F}(U)$ injects into $\prod_i\mathcal{F}(U_i)$ for all open covers of $U$.
            \item\textbf{Gluing}: Let $\{U_i\}_{i\in I}\subset\tau$ be an open cover of $U\subseteq X$ and let $\{s_i\in\mathcal{F}(U_i)\}_{i\in I}$ be a collection of local sections. If $\forall i,j\in I:s_i|_{U_i\cap U_j} = s_j|_{U_i\cap U_j}$, there exists a section $s\in\mathcal{F}(U)$ such that $\forall i\in I:s|_{U_i} = s_i$.
        \end{enumerate}
    }
    \begin{remark}[Separated presheaves]
        If a global section exists by the gluing condition, it is automatically unique by the separation axiom. In some texts these two conditions are combined in a single gluing condition that requires a unique global section. Presheaves satisfying only the first condition are said to be \textbf{separated}. (See also the footnote in \cref{topos:sheaf}.)
    \end{remark}

    \begin{notation}[Category of sheaves]
        \nomenclature[S_Sheaf]{$\mathbf{Sh}(X)$}{category of sheaves on a topological space $X$}
        Similar to the case of presheaves, one can define a morphism of sheaves as a collection of morphisms that commute with the restriction maps. The sheaves and sheaf morphisms on a space $X$ form a full subcategory of the category of presheaves, denoted by $\mathbf{Sh}(X)$.
    \end{notation}
    \begin{property}[\difficult{Sheaf topos}]
        The category of sheaves $\mathbf{Sh}(X)$ is in fact an elementary topos, called the sheaf topos on $X$. (See \cref{topos:sheaf_topos}.)
    \end{property}

    \begin{property}
        Let $X$ be a topological space and let $\mathcal{F}$ be a presheaf on $X$. $\mathcal{F}$ is a sheaf on $X$ if for any open $U\subseteq X$ and every open cover $\{U_i\}_{i\in I}$ of $U$ the following diagram is an equalizer diagram:
        \begin{gather}
            \label{sheaf:equalizer}
            \mathcal{F}(U)\rightarrow\prod_{i\in I}\mathcal{F}(U_i)\rightrightarrows\prod_{i, j\in I}\mathcal{F}(U_i\cap U_j)\,.
        \end{gather}
        The two morphisms on the right are induced by the restriction morphisms $\Phi^{U_i}_{U_i\cap U_j}$ and $\Phi^{U_j}_{U_i\cap U_j}$.
    \end{property}

    \newdef{Stalk}{\index{stalk}\index{germ}
        Consider a point $x\in X$ together with the set of all neighbourhoods of $x$. This set can be turned into a directed set (\cref{set:directed_set}) by equipping it with the (partial) order relation
        \begin{gather}
            U\geq V\implies U\subseteq V\,.
        \end{gather}
        This turns the sheaf $\mathcal{F}$ on $X$ into a directed system. The stalk at $x$ is defined as the following direct limit (\cref{set:direct_limit}):
        \begin{gather}
            \mathcal{F}_x := \varinjlim_{U\ni x} \mathcal{F}(U)\,.
        \end{gather}
        The equivalence class of a section $s\in\mathcal{F}(U)$ in $\mathcal{F}_x$ is called the \textbf{germ} of $s$ at $x$. Two sections belong to the same germ at $x$ if there exists a neighbourhood of $x$ on which they coincide.
    }
    \begin{notation}
        Similar to the notation of the restriction morphisms, the morphism that maps every section to its germ at $x$ is denoted by $\Phi^U_x$.
    \end{notation}

    \begin{property}
        Two subsheaves of a sheaf $\mathcal{F}$ on $X$ are equal if and only if their stalks are equal as subsets of $\mathcal{F}_x$ for all points $x\in X$. However, this does not imply that two sheaves with isomorphic stalks are equal (or even isomorphic)!
    \end{property}

    \begin{example}[Global sections functor]\index{global!sections}\label{sheaf:global_sections_functor}
        Let $X$ be a topological space. The global sections functor $\Gamma(X,-)$ is defined as the functor $\Gamma(X,-):\mathbf{Sh}(X)\rightarrow\mathbf{Set}:\mathcal{F}\rightarrow\mathcal{F}(X)$.
    \end{example}
    \begin{example}[Sheaf of sections]\index{sheaf!of sections}
        Consider a continuous function $f:X\rightarrow Y$. This function induces a sheaf on $Y$ in the following way:
        \begin{gather}
            \mathcal{F}(U) := \{s:U\rightarrow X\mid f\circ s=\mathbbm{1}_U\}\,.
        \end{gather}
        It is the sheaf that assigns to every open set the local sections of $f$ in the sense of \cref{cat:retract}.
    \end{example}

    \begin{construct}[Associated sheaf]\index{sheafification}\index{section!of a (pre)sheaf}\index{continuous}
        Consider a presheaf $\mathcal{F}$ on a topological space $X$. From this presheaf one can construct a sheaf $\overline{\mathcal{F}}$, called the \textbf{sheafification} or associated sheaf of $\mathcal{F}$, in the following way. First, define $\overline{\mathcal{F}}$ as the presheaf
        \begin{gather}
            \overline{\mathcal{F}}(U) := \left\{(s_x)_{x\in U}\in\prod_{x\in U}\mathcal{F}_x\,\middle\vert\,\forall x\in U:\exists\text{ open }V\ni x,t\in\mathcal{F}(V):\forall v\in V:s_v = \Phi^V_v(t)\right\}\,.
        \end{gather}
        Sections of this sheaf are said to be \textbf{continuous}. This statement can be made formal using the concept of an \'etal\'e space (see \cref{sheaf:etale_construction} further on). The restriction maps $\rho^U_V$ are defined as follows:
        \begin{gather}
            \rho^U_V:(s_x)_{x\in U}\mapsto(s_x)_{x\in V}\,.
        \end{gather}
        It is easily proven that this presheaf is in fact a sheaf, the sheafification of $\mathcal{F}$. This construction also gives a canonical morphism $\varphi:\mathcal{F}\rightarrow\overline{\mathcal{F}}$ since the canonical injection
        \begin{gather}
            \varphi(s):U\rightarrow\prod_{x\in U}\mathcal{F}_x:x \mapsto s_x = \Phi^U_x(s)\,,
        \end{gather}
        where $s\in\mathcal{F}(U)$ and $x\in U$, takes image in $\overline{\mathcal{F}}(U)$.
    \end{construct}
    \begin{uproperty}
        Let $\mathcal{F}$ be a presheaf on $X$ with associated sheaf $\overline{\mathcal{F}}$. Every sheaf morphism $\mathcal{F}\rightarrow \mathcal{G}$ factors uniquely through the canonical morphism $\mathcal{F}\rightarrow\overline{\mathcal{F}}$.
    \end{uproperty}

    \begin{property}[Stalks]
        Let $\mathcal{F}$ be a presheaf on $X$ with associated sheaf $\overline{\mathcal{F}}$. The morphism $\varphi:\mathcal{F}\rightarrow\overline{\mathcal{F}}$ induces an isomorphism $\varphi_x:\mathcal{F}_x\rightarrow\overline{\mathcal{F}}_x$ for all $x\in X$.
    \end{property}
    \begin{property}
        Let $\mathcal{F}$ be a sheaf on $X$ with associated sheaf $\overline{\mathcal{F}}$ (obtained by regarding $\mathcal{F}$ as a presheaf). The morphism $\varphi:\mathcal{F}\rightarrow\overline{\mathcal{F}}$ is an isomorphism.
    \end{property}

    There exists another, more topological, construction of the associated sheaf:
    \begin{construct}[\'Etal\'e spaces]\label{sheaf:etale_construction}
        Let $\mathcal{F}$ be a presheaf on $X$ and consider the disjoint union
        \begin{gather}
            \mathcal{F}^* := \bigsqcup_{x\in X}\mathcal{F}_x\,.
        \end{gather}
        Define for every local section $s\in\mathcal{F}(U)$ a function $\overline{s}:U\rightarrow\mathcal{F}^*:x\mapsto s_x\in\mathcal{F}_x$. The union $\mathcal{F}^*$ can be turned into an \'etal\'e space (\cref{topology:etale_space}) over $X$ by equipping it with the topology with basis
        \begin{gather}
            \bigl\{\overline{s}(U)\mid U\text{ open in }X, s\in\mathcal{F}(U)\bigr\}\,.
        \end{gather}
        The projection map $\pi$ is given by $\pi:s_x\mapsto x$. The sheafification $\overline{\mathcal{F}}$ is isomorphic to the sheaf of sections of $\mathcal{F}^*$.
    \end{construct}

    \begin{property}[Paracompact spaces]
        Let $X$ be a paracompact space (\cref{topology:paracompact}) and consider a sheaf $F$ of Abelian groups on $X$. For every closed subset $V\subset X$ there exists an isomorphism
        \begin{gather}
            \varinjlim_{U\supset K}F(U)\cong F(K)\,,
        \end{gather}
        where $F(K)$ is defined as the set of sections of the restriction of the \'etal\'e space $F^*$ to $K$. By definition this means that every section over a closed subset can be extended to a local section over some open neighbourhood.
    \end{property}

    \newdef{Flabby sheaf}{\index{sheaf!flabby}\index{sheaf!flasque|see{flabby}}
        A sheaf $F$ on a topological space $X$ such that for every two open subsets $U\subseteq V\subseteq X$ the restriction morphism $F(V)\rightarrow F(U)$ is surjective.
    }
    \newdef{Soft sheaf}{\index{sheaf!soft}\label{sheaf:soft}
        A sheaf on a topological space (often required to be paracompact Hausdorff) such that every section over a closed subset can be extended to a global section.

        From the previous property and definition it is clear that (on a paracompact Hausdorff space) every flabby sheaf is soft.
    }

    The sheafification can even be constructed in a third way.
    \begin{construct}[Abstract nonsense]\label{sheaf:colimit_construction}
        Consider the equalizer \cref{sheaf:equalizer}. To every presheaf $\mathcal{F}$, one can assign a separated presheaf $\mathcal{F}^\#$ by defining $\mathcal{F}^\#(U)$ as the direct limit (\cref{set:direct_limit}) of the equalizers over all open covers of $U$ ordered by refinement. A sheaf is obtained by applying this construction a second time.
    \end{construct}

    \begin{example}[Constant sheaf]\label{sheaf:constant_sheaf}
        Consider the constant presheaf on $X$ with target $S$ (\cref{sheaf:constant_presheaf}). The constant sheaf, denoted by $\underline{S}$ or $\flat S$, is defined as the associated sheaf of this presheaf. The stalks at every point $x\in X$ can be identified with $S$. The continuous sections $\underline{S}(U)$ are the locally constant functions $f:U\rightarrow S$.
    \end{example}

\section{Abelian sheaf cohomology}

    In this section only Abelian sheaves will be considered, i.e.~e. sheaves with values in $\mathbf{Ab}$ (unless stated otherwise).

    \begin{property}\label{sheaf:left_exact_functor}
        The global sections functor is only left exact.
    \end{property}

    Because of \cref{chapter:hom_alg} one can now construct derived functors. These give rise to a new cohomology theory. Although it will appear a lot more abstract than the (co)homology theories from \cref{chapter:algtop}, these theories are in fact specific instances.

    \newdef{Local system}{\index{local!system}\label{sheaf:local_system}
        A locally constant sheaf of Abelian groups. The cohomology of a topological space with coefficients in a local system is then simply the sheaf cohomology of this sheaf.
    }

\subsection{Derived cohomology}

    \begin{property}[Injective resolutions]
        Every Abelian sheaf admits an injective resolution or, equivalently, the category $\mathcal{AB}(X)$ of Abelian sheaves has enough injectives.
    \end{property}

    Because of \cref{sheaf:left_exact_functor}, one can construct nontrivial (right) derived functors of $\Gamma(X,-)$.
    \newdef{Sheaf cohomology group}{\index{cohomology!sheaf}
        Let $\mathcal{F}$ be a sheaf on $X$. Given an injective resolution $I$ of $\mathcal{F}$ (as usual the result will be independent of the chosen resolution), the sheaf cohomology groups of $\mathcal{F}$ on $X$ are defined as the cohomology groups of the complex
        \begin{gather}
            \cdots\longrightarrow\Gamma(X,I^i)\longrightarrow\Gamma(X,I^{i+1})\longrightarrow\cdots\,.
        \end{gather}
        The cohomology group $H^0(X;\mathcal{F})$ is equal to $\Gamma(X,\mathcal{F})$.
    }

    \newdef{Acyclic sheaf}{\index{sheaf!acyclic}
        A sheaf is said to be acyclic if its higher cohomology groups vanish (cf.~\cref{homalg:acyclic_object}).
    }
    \begin{example}[Soft sheaves]\label{sheaf:example_soft}
        Soft sheaves (\cref{sheaf:soft}) are acylic. Let $(M,\mathcal{O}_M)$ be a smooth manifold with its sheaf of smooth functions (resp.~complex manifold with its sheaf of holomorphic functions). All sheaves of $\mathcal{O}_M$-modules are soft and, hence, acyclic.
    \end{example}

    The following theorem is a specific instance of \cref{homalg:acyclic_derived_functors}.
    \begin{theorem}[de Rham \& Weil]\index{de Rham--Weil}
        There exists an isomorphism between the sheaf cohomology groups obtained using injective resolutions and the ones obtained using an acyclic resolution.
    \end{theorem}

    \newdef{Image and kernel}{
        Given a morphism of sheaves $\phi:\mathcal{F}\rightarrow\mathcal{G}$ on a space $X$ one can define the kernel/image presheaves that assign to every open subset $U\subseteq X$ the image/kernel of $\phi_U$.

        The kernel presheaf is already a sheaf and will be denoted by $\ker(\phi)$. The sheafification of the image presheaf will be denoted by $\im(\phi)$. In a similar way one can also define cokernels or any other notion that makes sense in Abelian categories.
    }

    \newdef{Cohomology sheaves}{
        Let $\mathcal{F}^\bullet$ be a cochain complex of sheaves on $X$. The cohomology sheaves $\underline{H}^i(X;\mathcal{F}^\bullet)$ are obtained by sheafifying the presheaves that assign to every open subset $U\subseteq X$ the quotient group $\ker(d^i_U)/\im(d^{i-1}_U)$.
    }

\subsection{\v{C}ech cohomology}\label{section:cech}

    Consider a chain complex $(A_\bullet,\partial_\bullet)\in\mathbf{Ch}(\mathcal{AB}(X))$ of Abelian sheaves on a topological space $X$ (for simplicity assume that the complex is connective).

    \newdef{\v{C}ech cohomology}{\index{Cech!cohomology}\label{sheaf:cech}
        For an open cover $\mathcal{U}=\{U_i\subseteq X\}_{i\in I}$, denote the intersection $U_{i_0}\cap\cdots\cap U_{i_k}$ by  $U_{i_0\ldots i_k}$. The cochain groups are defined for all $p\in\mathbb{N}$ as:
        \begin{gather}
            C^p(\mathcal{U};A_\bullet) := \bigoplus_{\substack{p=k-n\\i_0<\cdots<i_k}}A_n(U_{i_0\ldots i_k})\,.
        \end{gather}
        Since the (pre)sheaf takes values in Abelian groups, one can define the subtraction of elements and, hence, the following definition of the differential makes sense:
        \begin{align}
            (d\omega)_{i_0\ldots i_{k+1}} :=&\ \left(\partial\omega + (-1)^k\sum_{i=0}^k(-1)^iA_\bullet(\iota_i)\omega\right)_{i_0\ldots i_{k+1}}\label{sheaf:total_cech_differential}\\
            =&\ \partial\omega_{i_0\ldots i_{k+1}}+(-1)^k\sum_{j=0}^{k+1}(-1)^j\left.\omega_{i_0\ldots i_{j-1}i_{j+1}\ldots i_{k+1}}\right|_{U_{i_0\ldots i_{k+1}}}\,,\nonumber
        \end{align}
        where $\iota_i$ are the inclusion maps of the cover. The cohomology $\check{H}^\bullet(\mathcal{U};A_\bullet)$ of this complex is called the \v{C}ech cohomology of $\mathcal{U}$ with values in $A_\bullet$.

        By taking the direct limit over the direct system of open covers (the partial ordering is given by refinement of covers) one can define the \v{C}ech cohomology $\check{H}^\bullet(X;A_\bullet)$ of $X$ with values in $A_\bullet$.
    }
    \begin{remark}[Hypercohomology]\index{hyper-!cohomology}
        Two remarks should be made here. The above construction is essentially building a double complex and calculating the cohomology of the total complex. The definition \cref{sheaf:total_cech_differential} of the total differential might differ from others in the literature by a factor $(-1)^k$ as is often the case, since vertical and horizontal directions can be interchanged. Furthermore, sometimes the definition of \v{C}ech cohomology is only given with values in an Abelian group, such that the first term in \cref{sheaf:total_cech_differential} also vanishes. This case can be recovered from the above construction by considering chain complexes concentrated in a single degree. The more general case is sometimes called \textbf{hypercohomology}.
    \end{remark}

    The following two properties characterize when \v{C}ech cohomology calculates the (derived) cohomology of sheaves:
    \begin{property}
        In degrees 0 and 1 one always has
        \begin{gather}
            \check{H}^{0,1}(X;\mathcal{F})\cong H^{0,1}(X;\mathcal{F})\,.
        \end{gather}
        For a paracompact Hausdorff space, the \v{C}ech cohomology and (derived) sheaf cohomology coincide in all degrees.
    \end{property}
    \begin{property}[Leray]\index{Leray}\index{acyclic!cover}
        An open cover $\mathcal{U}=\{U_i\subset X\}_{i\in I}$ of a topological space is said to be \textbf{acyclic} with respect to a sheaf $\mathcal{F}$ if $\mathcal{F}$ is acyclic with respect to any finite subcover of $\mathcal{U}$:
        \begin{gather}
            H^{\bullet\geq1}(U_{i_1}\cap\ldots\cap U_{i_k};\mathcal{F})=0
        \end{gather}
        for all $i_1,\ldots,i_k\in I$. If $\mathcal{U}$ is acylic with respect to $\mathcal{F}$, then
        \begin{gather}
            \check{H}^\bullet(\mathcal{U};\mathcal{F})\cong\check{H}^\bullet(X;\mathcal{F})\cong H^\bullet(X;\mathcal{F})\,.
        \end{gather}
    \end{property}
    \begin{example}[Good covers]\index{cover!good}
        Consider a topological space admitting a good cover (\cref{manifold:good_cover}). The Leray theorem applies since all intersections are contractible and higher \v{C}ech cohomology vanishes on contractible spaces. Accordingly, for all topological spaces admitting a good cover (e.g. finite CW complexes or Riemannian manifolds), the \v{C}ech and singular cohomologies coincide.
    \end{example}

\section{Non-Abelian sheaf cohomology}\index{cohomology!non-Abelian}
\subsection{\v{C}ech cohomology}

    The issue with extending \v{Cech} cohomology to the non-Abelian context is that the whole definition made heavy use of operations that only exist in Abelian categories, e.g. images, kernels and addition. However, this problem can solved.

    As a first step, the case of a single group $G$ will be considered. The (would-be) differential $d$ is defined as before (now with a multiplicative convention):
    \begin{enumerate}
        \item For every 0-cochain $\{\phi_i:U_i\rightarrow G\}_{i\in I}$: $(d\phi)_{ij}:=\phi_j\phi^{-1}_i$.
        \item For every 1-cochain $\{\psi_{ij}:U_i\cap U_j\rightarrow G\}_{i,j\in I}$: $(d\psi)_{ijk} := \psi_{ij}\psi^{-1}_{ik}\psi_{jk}$.
        \item ...
    \end{enumerate}
    But now a major issue ariseq. With this definition, the maps $d_k$ are not group morphisms and the sets $\ker(d_k),\im(d_k)$ are not groups. To make matters worse, from $k=2$ onwards, $d$ stops being a differential altogether.

    For $k=0,1$ the situation can be saved. If $d\phi=e$ for some 0-cochain, $e$ being the identity element in $G$, then $\phi_i=\phi_j$ on $U_i\cap U_j$. So by the sheaf condition one obtains a genuine function $\phi:X\rightarrow G$, i.e.~$\check{H}^0(X;G)\cong C(X,G)$. For $k=1$ neither the cocycles nor coboundaries form a group, so forming a quotient is out of the question, but the 0-cochains do act on the 1-cocycles by conjugation
    \begin{gather}
        (\phi\cdot\psi)_{ij} := \phi_j\psi_{ij}\phi^{-1}_i\,.
    \end{gather}
    So one can take the quotient $\check{H}^1(X;G)$, as a set, of the 1-cocycles by this action and define this to be the first cohomology set. In the Abelian case, this recovers the usual construction of $\check{H}^1(X;G)$.

    \begin{remark*}
        The reason why higher cohomology $\check{H}^{\geq2}(X;G)$ can only be defined for Abelian groups, is not completely obvious. However, when reformulating cohomology in terms of mapping spaces with values in deloopings (see \cref{chapter:topos}), it becomes clear why this is the case.
    \end{remark*}

    @@ COMPLETE @@

\section{Ringed spaces}

    \newdef{Ringed space}{\index{ringed space}\label{sheaf:ringed_space}
        A topological space equipped with a sheaf of rings.
    }
    \newdef{Locally ringed space}{\label{sheaf:locally_ringed_space}
        A ringed space for which the stalk at every point is a local ring (\cref{algebra:local_ring}).
    }

    \newdef{Quasicoherent sheaf}{\index{sheaf!coherent}\label{sheaf:coherent}
        Let $(X,\mathcal{O}_X)$ be a ringed space. A quasicoherent sheaf $\mathcal{F}$ on $X$ is a $\mathcal{O}_X$-module that is locally the cokernel of a morphism of free modules.

        This means that $\mathcal{F}$ is locally presentable in the sense that it fits into an exact sequence of the form
        \begin{gather}
            \mathcal{O}^{k_i}_X|_{U_i}\longrightarrow\mathcal{O}^{l_i}_X|_{U_i}\longrightarrow\mathcal{F}|_{U_i}\longrightarrow0
        \end{gather}
        for some open cover $\{U_i\}_{i\in I}$ of $X$.

        A sheaf is said to be \textbf{coherent} if it is of \textbf{finite type} (or \textbf{finitely generated}), i.e.~every point has an open neighbourhood $U$ admitting a surjective morphism
        \begin{gather}
            \mathcal{O}^m_X|_U\longrightarrow\mathcal{F}|_U
        \end{gather}
        for $m\in\mathbb{N}$ and, for every open neighbourhood $V$ and integer $n\in\mathbb{N}$, any morphism
        \begin{gather}
            \mathcal{O}^n_X|_V\longrightarrow\mathcal{F}|_V
        \end{gather}
        has a kernel of finite type. All coherent sheaves are, in particular, quasicoherent.
    }

    \begin{property}
        The coherent sheaves over a ringed space $(X,\mathcal{O}_X)$ form an Abelian category (\cref{section:abelian_categories}).
    \end{property}

    \newdef{Locally free sheaf}{\index{free}\index{rank}\label{sheaf:locally_free}
        Let $(X,\mathcal{O}_X)$ be a ringed space. An $\mathcal{O}_X$-module is said to be locally free if for every point $x\in X$ there exists an open neighbourhood $U\subseteq X$ such that $\mathcal{F}|_U$ is free. If they are free of rank $n\in\mathbb{N}$, the sheaf is also said to be locally free of \textbf{rank} $n$.
    }
    \begin{property}\label{sheaf:locally_free_coherent}
        Locally free sheaves are quasicoherent. Moreover, locally free sheaves of finite rank are coherent if and only if the structure sheaf $\mathcal{O}_X$ is coherent. (In algebraic geometry, this will for example be the case for \textit{locally Noetherian varieties} and, in differential geometry, this is the case for \textit{complex manifolds}.)
    \end{property}

    \newdef{Invertible sheaf}{\index{sheaf!invertible}
        A locally free sheaf $\mathcal{F}$ over a ringed space $(X,\mathcal{O}_X)$ such that there exists another locally free sheaf $\mathcal{G}$ satisfying
        \begin{gather}
            \mathcal{F}\otimes\mathcal{G}\cong\mathcal{O}_X\,.
        \end{gather}
    }
    
    \begin{property}
        Over suitable spaces, such as \textit{schemes} (see \cref{chapter:alggeom}), locally free sheaves are invertible if and only if they are of rank 1.
    \end{property}
    \begin{remark}
        Because of the previous property, many authors simply use locally free of rank 1 as the definition of invertibility. Moreover, some authors even use the stronger condition of coherency instead of local freeness. By \cref{sheaf:locally_free_coherent} above, this is only equivalent in the case of \textit{locally Noetherian schemes} or \textit{complex manifolds}.
    \end{remark}

    \newdef{Picard group}{\index{Picard group}\label{sheaf:picard_group}
        The Abelian group of invertible sheaves over a ringed space. More generally, one can define the Picard group in any symmetric monoidal category (\cref{cat:symmetric}).
    }
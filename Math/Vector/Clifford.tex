\chapter{Clifford Algebra}\label{chapter:clifford}

    The main references for this chapter are~\citet{gallier_clifford_2008,choquet-bruhat_analysis_1991,choquet-bruhat_analysis_2000}. One should note that there are various conventions for the different structures that arise in the study of Clifford algebras and their representations. Even the references given here do not agree on the chosen conventions.

    In general, all metrics (and quadratic forms) wil be assumed to be nondegenerate. A part of the theory can also be extended to the degenerate case, but this will not be considered here. See~\citet{gallier_clifford_2008} for more information.

    \minitoc

\section{Clifford algebra}

    \newdef{Clifford algebra}{\index{Clifford!algebra}\label{clifford:clifford_algebra}
        Consider a unital associative $k$-algebra $A$ together with a quadratic form $Q:A\rightarrow k$. The Clifford algebra over $A$ associated to $Q$ is defined as the free algebra generated by $A$ under the following relation:
        \begin{gather}
            \label{clifford:condition}
            a^2 = Q(a)1\,,
        \end{gather}
        where $1$ is the unit element in $A$. This condition implies that the square of a vector is a scalar.
    }
    \begin{notation}
        The Clifford algebra corresponding to $A$ and $Q$ is often denoted by $C\ell(A,Q)$.
    \end{notation}

    \begin{construct}
        The previous definition can be given an explicit construction. First, construct the tensor algebra of $A$:
        \begin{gather}
            T(A) = \bigoplus_{n\in\mathbb{N}}A^{\otimes n}\,.
        \end{gather}
        Then, construct a two-sided ideal $I$ of $A$ generated by
        \begin{gather}
            \{a\otimes a - Q(a)1\mid a\in A\}
        \end{gather}
        as defined in \cref{algebra:generating_set_ideal}. The Clifford algebra $C\ell(A,Q)$ can then be constructed as the quotient algebra $T(A)/I$.
    \end{construct}

    \begin{remark}
        Looking at \cref{vector:adef_exterior_algebra}, it can be seen that the exterior algebra $\Lambda^\bullet(A)$ coincides with the Clifford algebra $C\ell(A,0)$. Note that if $Q\neq0$, the two algebras are still isomorphic as vector spaces (at least if\footnote{This condition will often come back in this chapter.} $\mathrm{char}(A)\neq2$), but not as algebras.
    \end{remark}

    \begin{property}[Dimension]
        If $A$ has dimension $n$, then $C\ell(A,Q)$ has dimension $2^n$.
    \end{property}

    \begin{example}[Inner product spaces]
        The classical example of a Clifford algebra is given by an inner product space with signature $(p,q)$. The Clifford algebra $C\ell_{p,q}(V)$ or $V_{p,q}$ is then defined as the Clifford algebra generated under the relation
        \begin{gather}
            v^2=-g(v,v)1\,.
        \end{gather}
        In physics, this convention would correspond to the `mostly pluses'-convention, which is mainly adopted in \textit{general relativity} (see \cref{chapter:GR}).
    \end{example}
    \begin{notation}
        The Clifford algebra over $\mathbb{R}^{p+q}$ with respect to the standard Hermitian form of signature $(p,q)$ is often denoted by $\mathbb{R}_{p,q}$.
    \end{notation}

    \begin{formula}[Dimensional reduction]
        \begin{gather}
            \mathbb{R}_{p+1,q+1}\cong\mathbb{R}_{p,q}\otimes M_2(\mathbb{R})
        \end{gather}
    \end{formula}
    \begin{formula}
        \begin{gather}
            \mathbb{R}_{p+1,q}\cong\mathbb{R}_{q+1,p}
        \end{gather}
    \end{formula}
    \begin{formula}
        \begin{gather}
            \mathbb{R}_{p,q+2}\cong\mathbb{R}_{q,p}\otimes\mathbb{H}
        \end{gather}
    \end{formula}

\section{Geometric algebra}

    \newdef{Geometric algebra}{\index{geometric!algebra}
        Let $V$ be a vector space equipped with a symmetric bilinear form $g:V\times V\rightarrow k$. The geometric algebra (GA) over $V$ is defined as the Clifford algebra $C\ell(V,g)$. Here, the classic relation $Q(v):=g(v,v)$ is implicitly used since quadratic forms are required in \cref{clifford:clifford_algebra}. This identification is unique as long as $\mathrm{char}(V)\neq2$.
    }

    \newdef{Inner and exterior product}{\index{inner!product}\index{exterior!product}
        Analogous to the inner product in linear algebra and the wedge product in exterior algebra, one can define (a)symmetric products on a geometric algebra. First of all, it should be noted that the product $ab$ of two vectors $a$ and $b$ can be written as the sum of a symmetric and an antisymmetric part:
        \begin{gather}
            \label{clifford:geometric_product}
            ab = \frac{1}{2}(ab + ba) + \frac{1}{2}(ab - ba)\,.
        \end{gather}
        One can then define the inner product as the symmetric part:
        \begin{gather}
            \label{clifford:inner_product}
            a\cdot b := \frac{1}{2}(ab + ba) = \frac{1}{2}\left((a+b)^2 - a^2 - b^2\right) = g(a,b)\,.
        \end{gather}
        Analogously, one can define the exterior (outer) product as the antisymmetric part:
        \begin{gather}
            \label{clifford:exterior_product}
            a\wedge b := \frac{1}{2}(ab - ba)\,.
        \end{gather}
        These definitions allow to rewrite the geometric product as follows:
        \begin{gather}
            ab = a\cdot b + a\wedge b\,.
        \end{gather}
    }
    \begin{remark*}
        Looking at the last equality in the definition of the inner product~\eqref{clifford:inner_product}, it can be seen that condition~\eqref{clifford:condition} is indeed satisfied when $a=b$.
    \end{remark*}

    \newdef{Multivector}{\index{multi-!vector}\index{blade}\index{pseudo-!scalar}\index{pseudo-!vector}
        Any element of a geometric algebra over $V$ is called a multivector. The simple multivectors of grade $k$, i.e.~elements of the form $v_1v_2\ldots v_k$ with $v_i\in V$ for all $i\leq k$, are called \textbf{$k$-blades}. (This should again remind the reader of the content of \cref{section:wedge_product}.) Sums of multivectors of different grades are called \textbf{mixed} multivectors (even though these elements do not represent a geometric structure). Taking $\dim(V)=n$, multivectors of grade $n$ and $n-1$ are also called \textbf{pseudoscalars} and \textbf{pseudovectors}, respectively.
    }

    \newdef{Grade projection operator}{\index{projection!function}
        Let $a\in\mathcal{G}(V)$ be a general multivector and consider $k\in\mathbb{N}$. The grade (projection) operator $\langle\cdot\rangle_k:\mathcal{G}(V)\rightarrow\mathcal{G}(V)_k$ is defined as the projection of $a$ onto the grade-$k$ part of $a$.
    }

    Using these projection operators, one can extend the inner and exterior product to the complete GA as follows.
    \begin{formula}
        Let $a,b$ be two multivectors of grades $m$ and $n$, respectively. Their inner product is given by
        \begin{gather}
            a\cdot b := \langle ab \rangle_{|m-n|}
        \end{gather}
        and their exterior product is given by
        \begin{gather}
            a\wedge b := \langle ab \rangle_{m+n}\,.
        \end{gather}
        An explicit calculation for $a\in\mathcal{G}(V)_1, b\in\mathcal{G}(V)_k$ gives
        \begin{gather}
            \begin{aligned}
                a\cdot b &= \frac{1}{2}\left(ab - (-1)^kba\right)\,,\\
                a\wedge b &= \frac{1}{2}\left(ab + (-1)^kba\right)\,.
            \end{aligned}
        \end{gather}
    \end{formula}

\section{\difficult{Bott periodicity}}\label{section:clifford_bott}

    The following theorem has deep implications in \textit{$K$-theory} (see  \cref{chapter:k}). It is also (through $K$-theory) related to the \textit{tenfold way} of \textit{Altland} \& \textit{Zirnbauer} in condensed matter physics.\index{tenfold way}
    \begin{theorem}[Bott periodicity]\index{Bott!periodicity}\label{clifford:bott_periodicity}
        The classification of (real) Clifford algebras is periodic modulo 8:
        \begin{gather}
            \mathbb{R}_{p,q+8}\cong\mathbb{R}_{p+8,q}\cong\mathbb{R}_{p,q}\otimes M_{16}(\mathbb{R})\,.
        \end{gather}
        For complex Clifford algebras, a similar statement exists, but with periodicity 2.
    \end{theorem}

    Bott periodicity also has some implications in category theory (\cref{chapter:cat}). Here, the language of \cref{section:abelian_categories} will be used.
    \newdef{Banach category}{\index{Banach!category}
        An additive category enriched over Banach spaces.

        For every Banach category $\mathbf{C}$ and finite-dimensional $\mathbb{R}$-algebra $A$, the category of $A$-representations $\mathbf{C}^A$ is defined as follows:
        \begin{enumerate}
            \item\textbf{Objects}: $A$-representations in $\mathbf{C}$, i.e.~pairs $(X,\rho)$ where $X\in\ob{C}$ and $\rho:A\rightarrow\End(X)$ is a Banach algebra morphism.
            \item\textbf{Morphism}: The $A$-equivariant morphisms/intertwiners.
        \end{enumerate}
    }
    \begin{notation}
        For brevity, some specific notation for the cases $A=\mathbb{R}_{p,q}$ and $A=M_n(\mathbb{R})$ are introduced, respectively: $\mathbf{C}^{p,q}$ and $\mathbf{C}(n)$.
    \end{notation}

    \begin{property}[Morita equivalence]\index{equivalence!Morita}
        Every pseudo-Abelian Banach category $\mathbf{C}$ is equivalent to $\mathbf{C}(n)$ for some $n\in\mathbb{N}$.
    \end{property}

    \begin{property}
        For $A,B$ two finite-dimensional $\mathbb{R}$-algebras and $\mathbf{C}$ a pseudo-Abelian Banach category, the following equivalences exist:
        \begin{gather}
            \begin{aligned}
                \mathbf{C}^{A\oplus B}&\cong\mathbf{C}^A\times\mathbf{C}^B\,,\\
                \mathbf{C}^{A\otimes B}&\cong\left(\mathbf{C}^A\right)^B\,.
            \end{aligned}
        \end{gather}
    \end{property}
    \begin{property}[Bott periodicity]\label{clifford:bott_periodicity_category}
        For $\mathbf{C}$ a pseudo-Abelian, real Banach category, the equivalence classes of categories $\mathbf{C}^{p,q}$ are determined by $p-q\bmod8$.
    \end{property}

    \begin{construct}[Grothendieck group]\index{Grothendieck!group}\index{Banach!functor}\label{clifford:functor_k_group}
        Consider a \textbf{Banach functor} $\func{\varphi}{C}{D}$ between Banach categories, i.e.~a functor that acts linearly and continuously on hom-spaces. Furthermore, assume that $\varphi$ is \textbf{quasi-surjective}, i.e.~every object in $\mathbf{D}$ is a direct summand of an object in the image of $\varphi$.

        To this functor, one can assign an Abelian group $K(\varphi)$ as follows. Let $\mathscr{V}(\varphi)$ denote the set of triples $(X,Y,f)$ where $X,Y\in\ob{C}$ and $f:\varphi(X)\rightarrow\varphi(Y)$ is an isomorphism. Elements in $\mathscr{V}(\varphi)$ are said to be isomorphic if there exist isomorphisms in $\mathbf{C}$ that make the `obvious' diagram in $\mathbf{D}$ commute. The sum of such triples is defined elementwise. Let $\mathscr{E}(\varphi)$ denote the subset of $\mathscr{V}(\varphi)$ consisting of triples $(X,Y,f)$ where $X=Y$ and $f$ is homotopic to $\mathbbm{1}_{\varphi(X)}$ in $\Aut(\varphi(X))$. The group $K(\varphi)$ is defined as the quotient of $\mathscr{V}(\varphi)$ by the following equivalence relation:
        \begin{gather}
            v\sim v'\iff\exists e,e'\in\mathscr{E}(\varphi):v+e\cong_{\mathscr{V}}v'+e'\,.
        \end{gather}
    \end{construct}

    \newdef{Restriction of scalars}{
        Consider a pseudo-Abelian Banach category $\mathbf{C}$. The group $K^{p,q}(\mathbf{C})$ is defined as the Grothendieck group of the canonical functor $\mathbf{C}^{p+1,q}\rightarrow\mathbf{C}^{p,q}$ (this functor is sometimes called the `restriction of scalars'-functor since it is contravariantly induced by the inclusion $\mathbb{R}_{p,q}\hookrightarrow\mathbb{R}_{p+1,q}$). Bott periodicity implies that these groups only depend on $p-q\bmod8$.
    }
    \begin{example}[$K^{0,0}$]\label{clifford:k00}
        The functor $\mathbf{C}^{1,0}\rightarrow\mathbf{C}^{0,0}$ is (up to equivalence) the direct sum functor $\mathbf{C}\times\mathbf{C}\rightarrow\mathbf{C}:(X,Y)\mapsto X\oplus Y$.
    \end{example}

    \begin{remark}[Complex spaces]\label{clifford:complex_bott_periodicity}
        The above constructions can be generalized to the setting of complex Banach spaces and complex algebras. However, Bott periodicity will then give a $p-q\bmod2$ classification.
    \end{remark}

\section{Pin group}
\subsection{Clifford group}

    In this section, it is assumed that $V$ is finite-dimensional and $Q$ is nondegenerate.

    \newdef{Transposition}{\index{transposition}\label{clifford:transposition}
        Let $\{e_i\}_{i\leq n}$ be a basis for $V$. On the tensor algebra $T(V)$, there exists an anti-automorphism $v^t$ that reverses the order of the basis vectors:
        \begin{gather}
            \cdot^t:e_{i_1}\otimes e_{i_2}\otimes\cdots\otimes e_{i_k}\mapsto e_{i_k}\otimes\cdots\otimes e_{i_2}\otimes e_{i_1}\,.
        \end{gather}
        Because the ideal in the definition of a Clifford algebra is invariant under this map, it induces an anti-automorphism, called the transposition or \textbf{reversal}, on $C\ell(V,Q)$.
    }

    \newdef{Main involution}{\index{involution}\index{inversion}
        Let $V_0,V_1$ be the grade-0 and grade-1 components of the Clifford algebra $C\ell(V,Q)$, respectively. Consider the following operator:
        \begin{gather}
            \widehat{v} :=
            \begin{cases}
                v&\cif v\in V_0\,,\\
                -v&\cif v\in V_1\,.
            \end{cases}
        \end{gather}
        This operator can be generalized to all of $C\ell(V,Q)$ by linearity. The resulting operator is called the main involution or \textbf{inversion} on $C\ell(V,Q)$. It turns the Clifford algebra into a \textit{superalgebra} (see \cref{hda:superalgebra}).
    }

    \newdef{Twisted conjugation}{\index{twisted conjugation}
        Let $C\ell(V,Q)$ be a Clifford algebra and consider its group of units $C\ell^*(V,Q)$. Note that this group consists exactly of the nonzero homogeneous elements of $C\ell(V,Q)$. Twisted conjugation by $s\in C\ell^*(V,Q)$ is defined as follows:
        \begin{gather}
            \chi:C\ell^*(V,Q)\rightarrow\Aut(V)\qquad\text{with}\qquad\chi(s)v = sv\widehat{s}^{-1}\,.
        \end{gather}
    }
    \newdef{Clifford group\footnotemark}{\index{Clifford!group}\index{Lipschitz!group}
        \footnotetext{Sometimes called the \textbf{Lipschitz group}.}
        The Clifford group $\Gamma(V,Q)$ is defined as follows:
        \begin{gather}
            \Gamma(V,Q) := \left\{s\in C\ell^*(V,Q)\,\middle\vert\,v\in V\implies sv\widehat{s}^{-1}\in V\right\}
        \end{gather}
        $\Gamma(V,Q)$ inherits the group structure from $C\ell^*(V,Q)$.
    }
    \begin{property}
        On $C\ell^*(V,Q)\cap V$, twisted conjugation is given by a Householder transformation (\cref{linalgebra:householder_transformation}).
    \end{property}

    \begin{property}\index{Cartan--Dieudonn\'e}
        If one interprets the condition $\chi(s)v\in V$ as stating the existence of a linear transformation $L_s\in\End(V)$ such that
        \begin{gather}
            se_i\widehat{s}^{-1} = (L_s)^j_ie_j\,,
        \end{gather}
        it can be seen that $L_s$ preserves the norm on $V$ and, accordingly, that the map $s\mapsto L_s$ defines a surjective homomorphism\footnote{In $\mathrm{char}(K)\neq2$, the surjectivity of the map $\chi$ follows from the \textit{Cartan--Dieudonn\'e theorem}. For characteristic 2, one can prove that the surjectivity holds using different methods.}
        \begin{gather}
            \rho:\Gamma(V,Q)\rightarrow\mathrm{O}(V,Q):s\mapsto L_s\,.
        \end{gather}
        Being a group morphism to a matrix group acting on $V$, it defines a representation called the \textbf{vector(ial) representation}. Furthermore, from the first isomorphism theorem~\ref{group:first_isomorphism_theorem} it follows that $\mathrm{O}(V,Q)$ is isomorphic to $\Gamma(V,Q)/\ker(\chi)$, where $\ker(\chi)=\mathbb{R}_0$. This isomorphism also implies\footnote{Again using the \textit{Cartan--Dieudonn\'e theorem}, valid only when $\mathrm{char}(K)\neq2$. In fact this statement is more or less the \textit{Cartan--Dieudonne\'e} theorem in terms of geometric algebra.} that the Clifford group is given by the set of finite products of invertible elements in $V$:
        \begin{gather}
            \Gamma(V,Q) = \left\{\prod_i^nv_i\,\middle\vert\,v_i\text{ invertible in }V\land n\in\mathbb{N}\right\}\,.
        \end{gather}
    \end{property}
    \begin{result}
        By noting that pure rotations can be decomposed as an even number of reflections, one finds that
        \begin{gather}
            \Gamma^+(V,Q)/\mathbb{R}_0\cong\mathrm{SO}(V,Q)\,,
        \end{gather}
        where $\Gamma^+:=C\ell_{\text{even}}(V,Q)\cap\Gamma(V,Q)$ is the intersection of the even Clifford algebra and the Clifford group.
    \end{result}

    \begin{remark}
        As was noted in the beginning of this chapter, there is a variety of different conventions in use. One of the important distinctions is the definition (or choice) of conjugation map $\chi$. \textit{Atiyah, Bott} and \textit{Shapiro} have introduced the twisted conjugation map that was used for the definition of the Clifford group. Before them, the usual choice was the ordinary conjugation map\footnote{The notation $\mathrm{ad}_s$ comes from the fact that this map resembles the adjoint action of a group.}
        \begin{gather}
            \mathrm{ad}_s:v\mapsto svs^{-1}\,.
        \end{gather}
        Although the difference between these maps seems to be rather subtle, the implications are important. If the ordinary conjugation would have been chosen for the definition of the Clifford group, only a surjective homomorphism would have been found in the case of $\dim(V)$ being odd. Moreover, the action by a degree-1 element would not be given by a Householder transformation anymore. Instead it would be the negative of this operation. This distinction is, in particular, important for the next section.
    \end{remark}

\subsection{Pin and Spin}\index{spin}\index{spinor}\label{section:spin}

    \newformula{Spinor norm}{\index{norm!spinor}
        On $\Gamma(V,Q)$, one can define the spinor norm\footnote{This map can be generalized to the full Clifford algebra, but then the image will not just be the underlying field anymore.}
        \begin{gather}
            \mathcal{N}(x):\Gamma(V,Q)\rightarrow K^\times:x\mapsto x^tx\,,
        \end{gather}
        where $x^t$ is the Clifford transposition (\cref{clifford:transposition}). On $V$, $\mathcal{N}$ coincides with the norm induced by $Q$.
    }
    \newdef{Pin and spin groups}{\index{pin}
        Using the spinor norm $\mathcal{N}$, one can now define the pin and spins groups as follows:
        \begin{gather}
            \mathrm{Pin}(V) := \{s\in\Gamma(V,Q)\mid\mathcal{N}(s)=\pm1\}
        \end{gather}
        and
        \begin{gather}
            \mathrm{Spin}(V) := \mathrm{Pin}(V)\cap\Gamma^+(V,Q)\,.
        \end{gather}
    }
    \begin{remark}
        In the literature, one can sometimes find the following alternative definition of the spinor norm:
        \begin{gather}
            \mathcal{N}(x) := \widehat{x}^tx\,.
        \end{gather}
    \end{remark}

    \begin{adefinition}
        The Pin group can also be defined as the set of elements in $\Gamma(V,Q)$ that can be written as a product of unit Clifford vectors (here, unit indicates unit norm and not just invertible as before). The Spin group is then defined as the elements that can be written as the product of an even number of unit Clifford vectors.
    \end{adefinition}

    \begin{property}\label{clifford:pin_group}
        The Pin group satisfies the following isomorphism:
        \begin{gather}
            \mathrm{Pin}(V,Q)/\mathbb{Z}_2\cong\mathrm{O}(V,Q)\,.
        \end{gather}
        An analogous relation holds for the Spin group and $\mathrm{SO}(V,Q)$. These relations imply that the Pin and Spin groups form a double cover (\cref{topology:covering_space}) of, respectively, the orthogonal and special orthogonal groups.
    \end{property}

    \begin{notation}
        The $\mathrm{Pin}$ groups associated to $\mathrm{O}(n,0)$ and $\mathrm{O}(0,n)$ are often denoted by $\mathrm{Pin}^+(n)$ and $\mathrm{Pin}^-(n)$, where the signs refer to the sign of squares in the Clifford condition~\eqref{clifford:condition}.
    \end{notation}

    \newdef{Spinor}{\index{Weyl!spinor}\index{Dirac!spinor}\index{gamma!matrix}\label{clifford:spinor}
        Consider a vector space $V$ equipped with a (faithful) representation of the group $\mathrm{Spin}(m,n)$. This representation is called the \textbf{spin(or) representation}. Elements of $V$ are called spinors.

        More precisely, if one considers the complex Clifford algebra $C\ell_{m,n}(\mathbb{C})$, two possibilities exist: either $m+n$ is even or $m+n$ is odd. In the even case ($m+n=2k$), one can prove (using the Artin--Wedderburn theorem~\ref{algebra:artin_wedderburn}) that the algebra is isomorphic to the matrix algebra $M(2^k,\mathbb{C})$. In the odd case ($m+n=2k+1$), the algebra is isomorphic to the direct sum $M(2^k,\mathbb{C})\oplus M(2^k,\mathbb{C})$.

        Inside these matrix algebras, one can find a set of elements satisfying the Clifford condition~\eqref{clifford:condition} and, thereby, generating the Clifford algebra. These are the so-called \textbf{gamma matrices}. The real algebra generated by these elements is isomorphic to the real Clifford algebra $C\ell_{m,n}(\mathbb{R})$.\footnote{Note, however, that the matrices themselves are, in general, still complex-valued.} The fundamental representation of this real algebra is often called the \textbf{Dirac representation}. If $m+n$ is even, the representation decomposes as the direct sum of two irreducible representations called the \textbf{Weyl} or \textbf{half-spin(or) representations}.
    }

    \begin{example}
        The following table gives some group isomorphisms for the Spin group in low dimensions $n\in\mathbb{N}$:
        \begin{gather*}
            \begin{array}{c|c}
                n&\mathrm{Spin}(n)\\
                \hline
                1&\mathrm{O}(1)\\
                2&\mathrm{U}(1)\\
                3&\mathrm{SU}(2)\\
                4&\mathrm{SU}(2)\times\mathrm{SU}(2)\,.
            \end{array}
        \end{gather*}
        For quadratic forms of signature $(p,q)$, the following table is found:
        \begin{gather*}
            \begin{array}{c|c}
                (1, n)&\mathrm{Spin}(1,n)\\
                \hline
                (1,1)&\GL(1,\mathbb{R})\\
                (1,2)&\mathrm{SL}(2,\mathbb{R})\\
                (1,3)&\mathrm{SL}(2,\mathbb{C})\\
                (1,5)&\mathrm{SL}(2,\mathbb{H})\\
                (1,9)&\mathrm{SL}(2,\mathbb{O})\,.\footnotemark
            \end{array}
        \end{gather*}
        \footnotetext{This last isomorphism should not exactly be understood in the sense of matrix algebras, since Spin acts associatively, but the octonions do not. Moreover, the dimensions do not even agree as vector spaces. A suitable definition of $\mathrm{SL}(2,\mathbb{O})$ falls outside the scope of this compendium (see e.g.~a paper by \textit{Hitchin}.)}
    \end{example}

    \begin{formula}\index{Pauli!matrix}
        Consider the basis of $\mathfrak{su}(2)$ given by the \textit{Pauli matrices} (see \cref{angular_momentum:pauli_matrices}). An explicit (double) covering map $\rho:\mathrm{Spin}(3)\cong\mathrm{SU}(2)\rightarrow\mathrm{SO}(3)$ is given by:
        \begin{gather}
            \rho:U\mapsto\frac{1}{2}\tr(U\sigma_i U^\dag\sigma^j)\,.
        \end{gather}
    \end{formula}

    \begin{property}
        For all $m,n\in\mathbb{N}$, the following isomorphism exists:
        \begin{gather}
            \mathrm{Spin}(m,n)\cong\mathrm{Spin}(n,m)\,.
        \end{gather}
    \end{property}
    \begin{remark}[Physical implications]
        Note that the above isomorphism only holds for the Spin groups and not for the associated Pin groups. This could have consequences in physics! In general, physicists freely switch between a $(1,3)$- and $(3,1)$-signature because all particles are assumed to be proper spinors. However, some pinors can only occur for a specific signature and this way it might be possible to detect the signature of the universe (see~\cite{berg_pin_2001}).
    \end{remark}

    \newdef{Semispin group}{\index{semi!spin}
        By definition, every Spin group has a canonical $\mathbb{Z}_2$-subgroup. However, for $n\in4\mathbb{N}_0$, $\mathrm{Spin}(n)$ also contains a noncanonical central $\mathbb{Z}_2$-subgroup $\iota:\mathbb{Z}_2\hookrightarrow\mathrm{Spin}(n)$. The quotient $\mathrm{Spin}(n)/\iota$ is called the semispin group $\mathrm{SemiSpin}(n)$.
    }
    \begin{property}
        For $n=8$, the $\mathbb{Z}_2$-subgroups are isomorphic and, as a consequence \begin{gather}
            \mathrm{SemiSpin}(8)\cong\mathrm{SO}(8)\,.
        \end{gather}
    \end{property}
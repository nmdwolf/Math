\section{Inner product}\label{section:innerproduct}

    In this section all vector spaces $V$ will be defined over $\mathbb{R}$ or $\mathbb{C}$.

\subsection{Inner product space}

    \newdef{Inner product}{\index{inner!product}\label{linalgebra:innerproduct}
        A function $\langle\cdot\mid\cdot\rangle:V\times V\rightarrow\mathbb{C}$ is called an inner product on $V$ if it satisfies the following properties for all $u,v,w\in V$ and $\lambda\in\mathbb{C}$:
        \begin{enumerate}
            \item\textbf{Conjugate symmetry}: $\langle v\mid w \rangle = \langle w\mid v \rangle^*$,
            \item\textbf{Linearity in the second argument}: $\langle u\mid \lambda v+w \rangle = \lambda\langle u\mid v \rangle + \langle u\mid w \rangle$,
            \item\textbf{Nondegeneracy}: $\langle v\mid v \rangle = 0 \iff v = 0$, and
            \item\textbf{Positive-definiteness}: $\langle v\mid v \rangle \geq 0$.
        \end{enumerate}
    }
    \begin{result}\index{sesquilinear}
        The first two properties have the result of conjugate linearity in the first argument:
        \begin{gather}
            \langle \lambda f + \mu g\mid h \rangle = \overline{\lambda}\langle f\mid h \rangle + \overline{\mu}\langle g\mid h \rangle\,.
        \end{gather}
        Therefore, these two properties together are often combined into a \textbf{sesquilinearity} axiom. When the underlying field is restricted to $\mathbb{R}$, such that the conjugate symmetry property is replaced by proper symmetry, the inner product becomes a bilinear form.
    \end{result}

    \newdef{Inner product space}{\index{Hilbert!space}
        A vector space equipped with an inner product $\langle\cdot\mid\cdot\rangle$. This is sometimes called a \textbf{pre-Hilbert space}.
    }

    \newdef{Metric dual}{\label{linalgebra:metric_dual}
        Using the inner product (or any other nondegenerate Hermitian form) one can define the metric dual of a vector by the following map:
        \begin{gather}
            L:V\rightarrow V^*:v\mapsto\langle v\mid\cdot \rangle\,.
        \end{gather}
        (See \cref{riemann:flat_map} for a generalization.) If the sesquilinearity condition would have been stated in the reversed convention, i.e.~conjugate linearity in the second argument, metric duals would be conjugate linear and, hence, would not be proper elements of the dual space.
    }
    \newdef{Adjoint map}{\index{adjoint!Hermitian}\index{Hermitian}\index{self-adjoint}\label{linalgebra:adjoint_operator}
        Let $A$ be a linear map on $V$. The (\textbf{Hermitian}) adjoint of $A$ is defined as the linear map $A^\dag$ that satisfies
        \begin{gather}
            \langle A^\dag v\mid w\rangle = \langle v\mid Aw\rangle
        \end{gather}
        for all $v,w\in V$. Alternatively, one can define the adjoint using the transpose and metric dual as follows:
        \begin{gather}
            A^\dag = L^{-1}\circ A^*\circ L\,.
        \end{gather}
        If $A=A^\dag$, $A$ is said to be \textbf{Hermitian} or \textbf{self-adjoint}. (In \cref{chapter:functional}, a distinction will be made between these two notions.)
    }

\subsection{Orthogonality}\label{section:orthogonality}

    \newdef{Orthogonal}{\index{orthogonal}\label{linalgebra:orthogonal}
        Consider two vectors $v,w\in V$ in an inner product space. These vectors are said to be orthogonal, denoted by $v\perp w$, if they obey the following relation:
        \begin{gather}
            \langle v\mid w \rangle = 0\,.
        \end{gather}
        An \textbf{orthogonal system} is a collection of vectors, none of them equal to 0, that are mutually orthogonal.
    }
    \begin{property}
        Orthogonal systems are linearly independent.
    \end{property}

    \newdef{Orthonormal}{\index{orthonormal}\label{linalgebra:orthonormal}
        A set of vectors $S$ is said to be orthonormal if it forms an orthogonal system and if all the elements $v\in S$ obey the following relation:
        \begin{gather}
            \langle v\mid v \rangle = 1\,.
        \end{gather}
    }
    \newdef{Orthogonal complement}{\index{complement!vector space}\label{linalgebra:orthogonal_complement}
        Let $W$ be a subspace of an inner product space $V$. The orthogonal complement of $W$ is defined as the following subspace:
        \begin{gather}
            W^\perp := \bigl\{v\in V\mid\forall w\in W:\langle v\mid w\rangle = 0\bigr\}\,.
        \end{gather}
    }
    \sremark{$W^\perp$ is pronounced as `W-perp'.}

    \begin{property}[Complements]
        Let $V$ be a finite-dimensional inner product space. The orthogonal complement $W^\perp$ is a complementary subspace to $W$, i.e.~$W\oplus W^\perp=V$.
    \end{property}
    \begin{result}\label{linalgebra:perp_of_perp}
        Let $W\leq V$ with $V$ a finite-dimensional inner product vector space. Forming orthogonal complements defines an involution:
        \begin{gather}
            (W^\perp)^\perp = W\,.
        \end{gather}
    \end{result}

    \newdef{Orthogonal projection}{\index{projection!orthogonal}\label{linalgebra:orthogonal_projection}
        Let $V$ be a finite-dimensional inner product vector space and consider a subspace $W\leq V$. Consider a vector $w\in W$ and let $\{w_1,\ldots,w_k\}$ be an orthonormal basis of $W$. The projections of $v\in V$ on $W$ and $w\in W$ are defined as follows:
        \begin{align}
            \proj_W(v) &:= \sum_{i=1}^k\langle v\mid w_i \rangle w_i\,,\\
            \proj_w(v) &:= \frac{\langle v\mid w \rangle}{\langle w\mid w \rangle}w\,.
        \end{align}
    }
    \begin{property}
        Orthogonal projections satisfy the following conditions:
        \begin{gather}
            \forall w\in W:\proj_W(w) = w \qquad\text{and}\qquad \forall u\in W^\perp:\proj_W(u) = 0\,.
        \end{gather}
    \end{property}

    \newmethod{Gram-Schmidt orthonormalization}{\label{linalgebra:gram_schmidt}
        Let $\{u_i\}_{i\leq n}$ be a set of linearly independent vectors. An orthonormal set $\{e_i\}_{i\leq n}$ can be constructed out of $\{u_i\}_{i\leq n}$ using the following procedure:
        \begin{enumerate}
            \item Orthogonalization:
                \begin{gather}
                    \begin{aligned}
                        w_1& = u_1&\\
                        w_2& = u_2 - \frac{\langle u_2\mid w_1\rangle}{\|u_2\|^2}w_1&\\
                        &\vdots&\\
                        w_n& = u_n - \sum_{i=1}^{n-1}\frac{\langle u_n\mid w_i\rangle}{\|u_n\|^2}w_i\,.&
                    \end{aligned}
                \end{gather}
            \item Normalization:
                \begin{gather}
                    \begin{aligned}
                        e_1& = \frac{w_1}{\|w_1\|}&\\
                        e_2& = \frac{w_2}{\|w_2\|}&\\
                        &\vdots&\\
                        e_n& = \frac{w_n}{\|w_n\|}\,.&
                    \end{aligned}
                \end{gather}
        \end{enumerate}
    }
    \begin{remark}\index{Hermitian!form}\index{signature}\index{Lorentz!metric}\label{linalgebra:NDH_form}
        Inner products can be generalized to \textbf{nondegenerate Hermitian forms} which do not satisfy the positive-definiteness property. For any vector space with a nondegenerate Hermitian form, one can find a basis such $\{e_1,\ldots,e_{p+q}\}$ such that
        \begin{gather}
            \langle e_i\mid e_i \rangle=
            \begin{cases}
                1&i\leq p\,,\\
                -1&p<i\leq p+q\,.
            \end{cases}
        \end{gather}
        These spaces are said to have (metric) \textbf{signature} $(p,q)$. A vector space with signature $(1,k)$ or $(k,1)$ is said to be \textbf{Lorentzian}. Note that the signature is basis-independent due to \textit{Sylvester's theorem}~\ref{linalgebra:sylvester}.
    \end{remark}

    \newdef{Householder transformation}{\index{Householder transformation}\label{linalgebra:householder_transformation}
        Let $v$ be an element of an inner product space $V$. The Householder transformation generated by $v$ is defined as the linear map
        \begin{gather}
            \sigma_v:V\rightarrow V:w\mapsto w - 2\frac{\langle w\mid v \rangle}{\langle v\mid v \rangle}v\,.
        \end{gather}
        This transformation amounts to a reflection in the hyperplane orthogonal to $v$.
    }

    \newdef{Angle}{\index{angle}\label{linalgebra:angle}
        \nomenclature[O_zsymangle]{$\sphericalangle(\cdot,\cdot)$}{angle between two vectors}
        Let $v,w$ be elements of an inner product space $V$. The angle $\theta$ between $v$ and $w$ is defined by the following formula:
        \begin{gather}
            \cos\theta := \frac{\langle v\mid w \rangle}{\|v\|\|w\|}\,.
        \end{gather}
        The angle between two vectors $v,w$ is sometimes denoted by $\sphericalangle(v,w)$.
    }
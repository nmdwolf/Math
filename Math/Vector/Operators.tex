\chapter{Operator Algebras}\label{chapter:operator_algebras}

    The main reference for this chapter is \cite{blackadar}.

\section{\texorpdfstring{$C^*$-}{C-star }algebras}
\subsection{Involutive algebras}

    \newdef{Involutive algebra\footnotemark}{\index{algebra!involutive}\index{$*$-algebra|see{algebra, involutive}}
        \footnotetext{Also called a $^*$\textbf{-algebra}.}
        An involutive algebra is an associative algebra $A$ over a commutative involutive ring $(R,\overline{\,\cdot\,}\,)$ together with an algebra involution $\cdot^*:A\rightarrow A$ such that:
        \begin{enumerate}
            \item $(a + b)^* = a^* + b^*$,
            \item $(ab)^* = b^*a^*$, and
            \item $(\lambda a)^* = \overline\lambda a^*$,
        \end{enumerate}
        for all $a,b\in A$ and $\lambda\in R$.
    }

    \newdef{$C^*$-algebra}{\index{C$^*$-algebra}
        A $C^*$-algebra is an involutive Banach algebra \ref{functional:banach_space} such that the \textbf{$C^*$-identity}
        \begin{gather}
            \|a^*a\| = \|a\|\|a^*\|
        \end{gather}
        is satisfied.
    }

    The Artin-Wedderburn theorem \ref{algebra:artin_wedderburn} implies the following decomposition theorem:
    \begin{theorem}
        Let $C$ be a finite-dimensional $C^*$-algebra. There exist unique integers $N,d_1,\ldots,d_N$ such that
        \begin{gather}
            C\cong\bigoplus_{i=1}^NM_{d_i}(K).
        \end{gather}
    \end{theorem}
    This implies that every $C^*$-algebra can be represented using block matrices.

\subsection{Positive maps}

    \newdef{Positive element}{\index{positive}
        An element of a $C^*$-algebra is called positive if its spectrum is contained in $[0,\infty[$.
    }
    \begin{property}
        Every positive element $a$ can be written as $a=b^*b$ for some element $b$. Hence, every positive element is self-adjoint.
    \end{property}

    \newdef{Cuntz algebra}{\index{Cuntz algebra}
        The $n^{th}$ Cuntz algebra $\mathcal{O}_n$ is defined as the (universal) unital $C^*$-algebra generated by $n$ isometric elements $s_i$ under the additional relation
        \begin{gather}
            \sum_{i=1}^ns_i^*s_i = 1,
        \end{gather}
        where 1 is the unit element.
    }

    \newdef{Positive map}{
        A morphism of $C^*$-algebras is called positive if every positive element is mapped to a positive element.
    }
    \newdef{Completely positive map}{\index{positive!completely}\index{CP map|see{positive, completely}}\label{operators:cp_map}
        \nomenclature[A_CP]{CP}{completely positive}
        A morphism of $C^*$-algebras $T:A\rightarrow B$ is called completely positive if for all $k\in\mathbb{N}$ the following map is positive:
        \begin{gather}
            \mathbbm{1}_k\otimes T:\mathbb{C}^{k\times k}\otimes A\rightarrow \mathbb{C}^{k\times k}\otimes B.
        \end{gather}
        If $T$ satisfies this condition only up to an integer $n$, it is said to be \textbf{$n$-positive}.
    }

    \newdef{State}{\index{state}
        Let $A$ be a $C^*$-algebra. A state $\psi$ on $A$ is a positive linear functional of unit norm.
    }

    \newdef{Adjoint map}{\index{adjoint}
        Consider a continuous linear map $\phi$ defined on the Schatten class $\mathcal{I}_p$. Given a trace functional $\mathrm{tr}$ on the $C^*$-algebra one can define the adjoint map $\phi^*$ defined on $\mathcal{I}_q$ whenever $p,q$ are H\"older conjugate. This adjoint is given by the following equation:
        \begin{gather}
            \mathrm{tr}\Big((\phi^*(A))^*B\Big) = \mathrm{tr}\Big(A^*\phi(B)\Big),
        \end{gather}
        where $A\in\mathcal{I}_q, B\in\mathcal{I}_p$.
    }
    \newdef{Trace-preserving map}{
        A map $\phi$ is said to be trace-preserving if it satisfies
        \begin{gather}
            \mathrm{tr}(\phi(A)) = \mathrm{tr}(A)
        \end{gather}
        for all trace class elements $A$. Using the above definition it is easily seen that on a unital $C^*$-algebra this is equivalent to
        \begin{gather}
            \phi^*(1) = 1.
        \end{gather}
    }

    \begin{property}
        \nomenclature[A_CPTP]{CPTP}{completely positive trace-preserving}
        A completely positive, trace preserving map $\phi$ satisfies:
        \begin{gather}
            \|\phi\|_1 = 1,
        \end{gather}
        where the subscript $1$ indicates that this operator is defined on trace class elements.
    \end{property}

    \newdef{Positivity-improving map}{
        A positive map $\phi$ that satisfies
        \begin{gather}
            A\geq0,A\neq0\implies\phi(A)>0.
        \end{gather}
    }
    \newdef{Ergodic map}{\index{ergodic}
        A positive map $\phi$ that satisfies
        \begin{gather}
            \forall A>0:\exists t_A>0:\exp(t_A\phi)A>0.
        \end{gather}
    }

\subsection{Representations}\index{representation}

    \newdef{$C^*$-algebra representation}{
        A representation of a $C^*$-algebra $\mathcal{C}$ is a unital $\ast$-morphism $\mathcal{C}\rightarrow\mathcal{B}(\mathcal{H})$.
    }
    \newdef{Cyclic vector}{\index{cyclic!vector}
        A cyclic vector for a $C^*$-algebra representation $\rho:\mathcal{C}\rightarrow\mathcal{B}(\mathcal{H})$ is a vector $\xi\in\mathcal{H}$ such that $\{\rho(c)\xi\mid c\in\mathcal{C}\}$ is (norm) dense in $\mathcal{H}$.
    }

    \begin{construct}[GNS\footnotemark\ construction]\index{GNS construction}\index{Gel'fand|seealso{GNS}}\label{operators:gns}
        \footnotetext{Gel'fand-Naimark-Segal}
        \nomenclature[A_GNS]{GNS}{Gel'fand-Naimark-Segal}
        Let $\mathcal{C}$ be a $C^*$-algebra. Given a state $\omega$ on $\mathcal{C}$ there exists a $C^*$-representation $\rho:\mathcal{C}\rightarrow\mathcal{B}(D)$ where $D\subset\mathcal{H}$ is a dense subspace of a Hilbert space $\mathcal{H}$ such that the following conditions are satisfied:
        \begin{itemize}
            \item There exists a distinguished cyclic unit vector $\xi$ such that $D = \{\rho(c)\xi\mid c\in\mathcal{C}\}$.
            \item For all elements $c\in\mathcal{C}$ the following equality holds:
                \begin{gather}
                    \omega(c) = \langle\rho(c)\xi,\xi\rangle.
                \end{gather}
        \end{itemize}

        ?? COMPLETE CONSTRUCTION ??
    \end{construct}

    \begin{theorem}[Gel'fand-Naimark]\index{Gel'fand-Naimark}
        Every $C^*$-algebra is isometrically $\ast$-isomorphic to a norm closed ($C^*$-)algebra of bounded operators on a Hilbert space $\mathcal{H}$.
    \end{theorem}

\subsection{Gel'fand duality}\index{Gel'fand!duality}

    \newdef{Gel'fand spectrum}{\index{Gel'fand!spectrum}
        Consider a unital $C^*$-algebra $A$. Its set of characters, i.e. the algebra morphisms $A\rightarrow\mathbb{C}$, can be equipped with a compact\footnote{Locally compact if the algebra is non-unital.} Hausdorff topology (the weak-* topology \ref{functional:weak_star_topology}).
    }
    \newdef{Gel'fand representation}{
        Consider a $C^*$-algebra $A$ and let $\Phi_A$ denote its Gel'fand spectrum. The Gel'fand transformation of an element $a\in A$ is defined as the morphism $\hat{a}:\Phi_A\rightarrow\mathbb{C}$ given by the following formula:
        \begin{gather}
            \hat{a}(\lambda) = \langle\lambda,a\rangle
        \end{gather}
        where $\langle\cdot,\cdot\rangle$ denotes the pairing between $A$ and $\Phi_A$. By definition of the topology on the Gel'fand spectrum the functional $\hat{a}$ is continuous for all $a\in A$. The mapping $a\mapsto\hat{a}$ is called the Gel'fand representation of $A$.
    }

    \begin{theorem}[Gel'fand-Naimark]\index{Gel'fand-Naimark}
        Let $A$ be a commutative $C^*$-algebra. The Gel'fand representation gives an isometric $\ast$-isomorphism between $A$ and the set of continuous functionals which vanish at infinity $C_0(\Phi_A)$ on its Gel'fand spectrum.
    \end{theorem}

\section{von Neumann algebras}

    \newdef{von Neumann algebra}{\index{von Neumann!algebra}
        A $*$-subalgebra of a $C^*$-algebra equal to its double commutant: $M'' = M$.
    }
    \newdef{Concrete von Neumann algebra}{
        A weakly closed unital $*$-algebra of bounded operators on some Hilbert space.
    }
    \begin{theorem}[Double Commutant theorem\footnotemark]
        \footnotetext{Often called \textbf{von Neumann's double commutant theorem}.}
        The above definitions are equivalent.
    \end{theorem}

    \newdef{Projection}{\index{projection}\label{operators:projection}
        An element $p$ of a von Neumann algebra is called a projection if it satisfies
        \begin{gather}
            p = p^2 = p^*.
        \end{gather}
        This terminology reflects the property that if a von Neumann algebra is regarded as an algebra of bounded operators, the projections are exactly the operators associated to an orthogonal projection.
    }
    \begin{property}
        Any von Neumann algebra is generated by its projections.
    \end{property}

    \newdef{Murray-von Neumann equivalence}{\index{equivalence!Murray-von Neumann}
        Two closed subspaces are said to be Murray-von Neumann equivalent if one is mapped isomorphically onto the other by a partial isometry. In terms of projections this means that $p\sim q$ if and only if there exists a partial isometry $u$ such that $p=uu^*$ and $q=u^*u$.
    }

    \newdef{Finite projection}{\index{projection!finite}
        The collection of projections inherits the structure of a partial order from the partial order on the corresponding subspaces. A projection $p$ is said to be finite if there exists no smaller projection $q$ that is equivalent to $p$.
    }

\subsection{Factors}

    \newdef{Factor}{\index{factor}
        Consider a von Neumann algebra $M$. A $*$-subalgebra $A$ is called a factor of $M$ if its center $Z(A)$ is given by the scalar multiples of the idenity.
    }

    \newdef{Type $\mathrm{I}$ factor}{
        A factor is of type $\mathrm{I}$ if it contains a \textit{minimal projection}.
    }
    \begin{property}[Type $\mathrm{I}_n$ factors]
        Any type $\mathrm{I}$ factor is isomorphic to the algebra of all bounded operators on a Hilbert space. To indicate the dimension $n$ of this Hilbert space (which may be $\infty$) one sometimes uses the subclassification of type $\mathrm{I}_n$ factors.
    \end{property}

    \newdef{Powers index}{\index{index!Powers}
        Consider a Hilbert space $\mathcal{H}$ together with its von Neumann algebra of bounded operators $\mathcal{B}(\mathcal{H})$. A unital $*$-endomorphism $\alpha$ has Powers index $n\in\mathbb{N}$ if the space $\alpha(\mathcal{B}(\mathcal{H}))$ is isomorphic to a type $\mathrm{I}_n$ factor.
    }

    \newdef{Type $\mathrm{II}$ factor}{
        A factor is of type $\mathrm{II}$ if it contains nonzero \textit{finite projections} but no minimal ones. If the identity is finite, the factor is sometimes said to be of type $\mathrm{II}_1$, otherwise it is of type $\mathrm{II}_\infty$.
    }

    \newdef{Type $\mathrm{III}$ factor}{
        A factor is of type $\mathrm{III}$ if it does not contain any nonzero finite projections.
    }

\subsection{Projection-valued measures}

    This section focuses on the algebra of bounded operators $\mathcal{B}(\mathcal{H})$ on a (complex) Hilbert space $\mathcal{H}$.

    \begin{property}[Closed subspaces]
        There exists a bijection between the set of closed subspaces of $\mathcal{H}$ and the set of projections in $\mathcal{B}(\mathcal{H})$. Furthermore, if the projection $p$ corresponds to a subspace $\mathcal{H}_p$, the projection $\mathbbm{1}_{\mathcal{H}}-p$ corresponds to the orthogonal complement $\mathcal{H}_p^\perp$.
    \end{property}

    \newdef{Projection-valued measure\footnotemark}{\index{measure!projection-valued}\index{spectral measure}
        \footnotetext{Also called a \textbf{spectral measure}.}
        \nomenclature[A_PVM]{PVM}{projection-valued measure}
        Consider a topological space $X$ and let $\Sigma$ be a $\sigma$-algebra \ref{set:sigma_algebra} on $X$. A projection-valued measure (PVM) on $X$ is a map $P_-:\Sigma\rightarrow\mathcal{B}(\mathcal{H})$ satisfying the following conditions:
        \begin{enumerate}
            \item $P_E$ is a projection for all $E\in\Sigma$,
            \item $P_X = \mathbbm{1}_\mathcal{H}$,
            \item $P_AP_B = P_{A\cap B}$, and
            \item for all $n\in\mathbb{N}$ and disjoint $\{E_i\}_{i\leq n}\subset\Sigma$:
                \begin{gather}
                    \sum_{i\leq n}P_{E_i} = P_{\cup_{i\leq n}E_i}.
                \end{gather}
                In fact one should also allow the sum on the left-hand side to run over all of $\mathbb{N}$.
        \end{enumerate}
    }
    \begin{property}
        For every two elements $v,w\in\mathcal{H}$ the map $E\mapsto\mu^P_{v,w}(E):=\langle v|P_Ew \rangle$ defines a (complex) measure $\mu^P_{v,w}$ on $X$. The square of the norm of an element $v\in\mathcal{H}$ is then simply given by $\mu^P_{v,v}(X)$ for any PVM $P$ due to the second condition above.
    \end{property}

    \begin{property}
        Let $f:X\rightarrow\mathbb{C}$ be a measurable function on a measurable space $(X,\Sigma)$. Given a PVM $P$ on $X$, one defines $\Delta_f$ to be the set of all $v\in\mathcal{H}$ for which $f\in L^2(X, \mu^P_{v,v})$. The operator $\int_Xf(\lambda)dP(\lambda):\Delta_f\rightarrow\mathcal{H}$ defined by
        \begin{gather}
            \left\langle v\left|\int_Xf(\lambda)dP(\lambda)w\right.\right\rangle = \int_Xf(\lambda)d\mu^P_{v,w}(\lambda)
        \end{gather}
        is closed and normal. Furthermore, it satisfies the following two equalities:
        \begin{align}
            \left(\int_Xf(\lambda)dP(\lambda)\right)^* &= \int_X\overline{f(\lambda)}dP(\lambda)\\
            \left|\left|\int_Xf(\lambda)dP(\lambda)v\right|\right|^2 &= \int_X|f(\lambda)|^2d\mu^P_{v,v}(\lambda).
        \end{align}
        If $f$ is bounded, the above operator is bounded by the supremum norm of $f$:
        \begin{gather}
            \left|\left|\int_Xf(\lambda)dP(\lambda)\right|\right|\leq\|f\|_\infty.
        \end{gather}
    \end{property}

    \begin{theorem}[Spectral decomposition]
        Let $A$ be a self-adjoint operator on a Hilbert space $\mathcal{H}$. There exists a unique projection-valued measure $P_A:\mathcal{B}(X)\rightarrow\mathcal{B}(\mathcal{H})$ on the Borel $\sigma$-algebra of the real line such that
        \begin{gather}
            A = \int_{\mathbb{R}}\lambda dP_A(\lambda).
        \end{gather}
    \end{theorem}
    \begin{remark}[Normal operators]
        The above theorem extends to normal operators if one replaces $\mathbb{R}$ by $\mathbb{C}$.
    \end{remark}
    \begin{property}[Spectrum and support]
        The spectrum of a self-adjoint operator $A$ coincides with the support of its associated spectral measure $P_A$. A number $\lambda\in\mathbb{R}$ belongs to the point spectrum of $A$ if and only if the PVM associated to $A$ does not vanish on $\{\lambda\}$. A number $\lambda\in\mathbb{R}$ belongs to the continuous spectrum of $A$ if the associated PVM vanishes on $\{\lambda\}$ but is nonvanishing on any open set containing $\lambda$.
    \end{property}

    The above property allows to compose self-adjoint operators with (measurable) functions similar to how one can compute $f(X)$ for finite-dimensional operators by applying $f$ to the eigenvalues of $X$:
    \begin{formula}[Function of operators]
        Let $f:\sigma(A)\rightarrow\mathbb{C}$ be a measurable function (with respect to the restriction of the Borel algebra on $\mathbb{R}$) and let $g:\mathbb{R}\rightarrow\mathbb{C}$ be any other measurable function that coincides with $f$ on $\sigma(A)$.
        \begin{gather}
            f(A) := \int_{\sigma(A)}f(\lambda)dP_A(\lambda) = \int_{\mathbb{R}}g(\lambda)dP_A(\lambda) =: g(A).
        \end{gather}
    \end{formula}
\chapter{Notes}

    This chapter contains notes written down during lunch talks, summer courses and conferences.

\section{Noether's Theorem and Gauge-Gravity Duality by S. De Haro}

    Date \& location: October 6 2018, London\\
    Conference: \textit{The Philosophy and Physics of Noether's Theorems}

\subsection{Pseudotensors}

    Maxwell theory has a Noether stress-energy tensor of the form
    \begin{gather}
        T^\mu_{\ \nu} := F^{\mu\lambda}\partial A_\lambda - \frac{1}{4}\delta^\mu_\nu F^{\lambda\kappa}F_{\lambda\kappa}
    \end{gather}
    and an associated weak conservation law
    \begin{gather}
        \partial_\mu T^{\mu\nu}\approx 0.
    \end{gather}
    Through the Schwarz theorem \ref{calculus:schwarz_theorem}, this tensor can be enlarged to a conserved \textit{Belinfante tensor}
    \begin{gather}
        \overline{T}^\mu_{\ \nu} := T^\mu_{\ \nu} + \partial_\lambda U^{[\mu\lambda]}_{\ \ \ \ \nu}.
    \end{gather}
    Using such an extension, the standard form
    \begin{gather}
        T^\mu_{\ \nu} := F^{\mu\lambda}F_{\nu\lambda} - \frac{1}{4}\delta^\mu_\nu F^{\lambda\kappa}F_{\lambda\kappa} + \text{eom}
    \end{gather}
    can be obtained.

    Now, consider the theory coupled to gravity. In this case partial derivatives have to be replaced by covariant derivatives and the (weak) conservation law becomes:
    \begin{gather}
        \nabla_\mu T^{\mu\nu}.
    \end{gather}
    However, this identity only contains the stress-energy tensor of matter, not of gravity itself. A possible solution, due to \textit{Einstein}, was to construct a stress-energy (pseudo)tensor\footnote{Pseudotensor here just means an object that does not transform tensorially.} from the Christoffel symbols, since these are already responsible for the coupling to gravity in the conservation law above:
    \begin{gather}
        \partial_\mu(\sqrt{g}T^\mu_{\ \nu}+t^\mu_{\ \nu}) = 0.
    \end{gather}
    As above this leads to a \textbf{superpotential}:
    \begin{gather}
        \sqrt{g}T^\mu_{\ \nu}+t^\mu_{\ \nu} = \partial_\lambda s^{\mu\lambda}_{\ \ \ \nu},
    \end{gather}
    where $\partial_\mu\partial_\lambda s^{\mu\lambda}_{\ \ \ \nu}=0$. All conservation laws will be preserved if $s^{\mu\lambda}_{\ \ \ \nu}=s^{[\mu\lambda]}_{\ \ \ \ \nu}$.

    The main issues with this new ``stress-energy tensor'' are:
    \begin{itemize}
        \item It is not a tensor.
        \item There are an infinite number of possibilities.
    \end{itemize}
    However, the superpotential can be related to boundary conditions and, therefore, has a physical intepretation. Consider the Hamiltonian
    \begin{gather}
        \overline{H}(n) := \int_\Sigma n^\mu H_\mu + \oint_{\partial\Sigma}B(n),
    \end{gather}
    where $n$ is the ADM-like shift vector that generates tangential motion along the spacelike hypersurface $\Sigma$ and $B(n)\sim n^\nu s^{\mu\lambda}_{\ \ \nu}$. Noether's theorem implies that $H_\mu$ is proportional to the EOM, which implies that $H(n)$ is a boundary term determined by the superpotential. It determines the quasi-local energy.

    A different approach is the \textbf{Brown-York pseudotensor}. Here, a bounded spacetime region $M$ is considered with two spacelike boundaries, the initial and final slices $\Sigma_\pm$, and a timelike hypersurface $N$. The matter-coupled gravitational action is given by:
    \begin{gather}
        S = \frac{1}{2\kappa}\int_Md^4x\sqrt{-g}R + \frac{1}{\kappa}\int_{\Sigma_\pm}d^3x\sqrt{h}K - \frac{1}{\kappa}\int_Nd^3x\sqrt{-\gamma}\Theta + S_\mathrm{matter} + S_\mathrm{ref},
    \end{gather}
    where $S_\mathrm{ref}$ is a reference action to regularize the action. Variation with respect to the metric gives:
    \begin{gather}
        \delta S = -\int_Md^4xE^{\mu\nu}\delta g_{\mu\nu} + \frac{1}{2}\int_{\Sigma_\pm}d^3x\sqrt{h}P_{ij}\delta h_{ij} + \frac{1}{2}\int_Nd^3x\sqrt{\gamma}\tau^{ij}_\mathrm{BY}\delta\gamma_{ij} +\text{matter terms}.
    \end{gather}
    The first term gives the Einstein field equations and vanishes on-shell. The other terms define the conjugate momenta. As the notation implied, the object $\tau^{ij}_\mathrm{BY}$ is the (quasilocal) Brown-York stress-energy tensor. For pure gravity this can be written as follows:
    \begin{gather}
        \tau^{ij}_\mathrm{BY} = -\frac{1}{\kappa\sqrt{\gamma}}(\Theta\gamma^{ij}-\Theta^{ij}) +\text{regularizing terms}.
    \end{gather}

\subsection{AdS-CFT duality}

    Now, consider an anti-de Sitter spacetime $M$. In the CFT description the stress-energy tensor can be obtained by a functional derivative of the partition function with respect to some fixed background metric $g_{(0)}$:
    \begin{gather}
        \langle T_{ij}(x) \rangle = \frac{2}{\sqrt{g_{(0)}}}\frac{\delta W[g_{(0)}]}{\delta g^{ij}_{(0)}(x)}.
    \end{gather}
    In the gravitational description the fixed metric rerepsents the asymptotic metric (up to conformal factors). A theorem by \textit{Fefferman-Graham} says that near the boundary, the metric admits a local Poincar\'e form:
    \begin{gather}
        ds^2 = \frac{l^2}{r^2}(dr^2 + g_{ij}(r,x)dx^idx^j),
    \end{gather}
    with $g_{ij}(0,x) = g_{(0)ij}(x)$. Holographic duality then shows that the qausilocal Brown-York tensor is equal to the holographic stress-energy tensor.

\section{A generalization of Noether's theorem and the information-theoretic approach to the study of symmetric dynamics by R. Spekkens}

    Date \& location: October 6 2018, London\\
    Conference: \textit{The Philosophy and Physics of Noether's Theorems}

    The general idea of this talk is that (quantum) entanglement can be considered from a \textit{resource theory} perspective. In this approach quantum entanglement is a \textbf{resource} and LOCC operations are the free operations, i.e. given sufficient entanglement they can be used to perform any quantum operation.

    To characterize asymmetry measures, the following principle is adopted:
    \begin{axiom}[Curie's principle]
        \textit{The symmetries of the cause are to be found in the effect}, i.e. the effect is at least as symmetric as the cause.
    \end{axiom}

    Now, consider an \textbf{asymmetry measure}, i.e. a function $\mu:S(A)\rightarrow\mathbb{R}$ such that $\mu(\rho)\geq\mu(\Phi(\rho))$ for all symmetric quantum channels $\Phi$ (quantum channels that commute with symmetry operations). For general dynamics (including open systems), where Noether's theorem does not apply, asymmetry measures give constraints on allowed evolutions. When restricting to closed systems, these measures turn into conserved quantities and for mixed states they are independent of Noether charges.
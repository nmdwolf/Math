\chapter{Astronomy}

\section{Ellipsoidal coordinates}

We start from folowing parametrized equation:
\begin{equation}\index{coordinates!ellipsoidal}
	\label{astronomy:ellipsoidal_defining_function}
	f(\tau) = \stylefrac{x^2}{\tau + \alpha} + \stylefrac{y^2}{\tau + \beta} + \stylefrac{z^2}{\tau + \gamma} - 1
\end{equation}
where $\alpha<\beta<\gamma<0$. By multiplying away the denominators and setting $f(\tau) = 0$ we obtain a polynomial equation of degree 3 in $\tau$. This polynomial can be formally factorised as
\begin{equation}
	-(\tau-\lambda)(\tau-\mu)(\tau-\nu)
\end{equation}
such that the solutions $(\lambda, \mu, \nu)$ obey the following rules:
\[
	\left\{
    \begin{array}{ccl}
		\nu&\in&]-\gamma, -\beta[\\
    	\mu&\in&]-\beta, -\alpha[\\
    	\lambda&\in&]-\alpha, +\infty[
	\end{array}
    \right.
\]
From previous two equations we can find a solution for $x^2$ by multiplying by $(\tau+\alpha)$ and letting $\tau\rightarrow-\alpha$. Solutions for $y^2$ and $z^2$ can be found in a similar way:
\begin{equation}
	\label{astronomy:ellipsoidal_coordinates}
	\left\{
    \begin{array}{ccl}
		x^2 &=& \stylefrac{(\lambda + \alpha)(\mu + \alpha)(\nu + \alpha)}{(\beta - \alpha)(\gamma - \alpha)}\\
        y^2 &=& \stylefrac{(\lambda + \beta)(\mu + \beta)(\nu + \beta)}{(\beta - \alpha)(\beta - \gamma)}\\
        z^2 &=& \stylefrac{(\lambda + \gamma)(\mu + \gamma)(\nu + \gamma)}{(\alpha - \gamma)(\beta - \gamma)}
	\end{array}
    \right.
\end{equation}

\noindent For these solutions multiple cases can be considered. We can define different surfaces by fixing $\tau$ at different values.

	\subsection{Ellipsoid\texorpdfstring{$\text{: }\tau = \lambda$}{}}\index{ellipsoid}\index{focal!ellipse}
    First we look at the surfaces defined by fixing $\tau = \lambda$ in equation \ref{astronomy:ellipsoidal_defining_function}. By noting that all denominators are positive in this case, we see that the obtained surface is an ellipsoid with the $x$-axis as the shortest axis. By letting $\lambda\rightarrow+\infty$ we obtain the equation of a sphere with radius $\sqrt{\lambda}$. If $\lambda\rightarrow-\alpha$ we get an ellipse in the $yz$-plane. This ellipse is called the \textbf{focal ellipse}.
    
    \subsection{One-sheet hyperboloid\texorpdfstring{$\text{: }\tau = \mu$}{}}\index{focal!hyperboloid}\index{hyperboloid}
    By fixing $\tau=\mu$ in \ref{astronomy:ellipsoidal_defining_function} we obtain the equation of one-sheet hyperboloid (also called a \textbf{hyperbolic hyperboloid}) around the $x$-axis. By letting $\mu\rightarrow-\alpha$ the hyperboloid collapses in the $yz$-plane and we obtain the surface outside the focal ellipse. If $\mu\rightarrow-\beta$ the hyperboloid becomes degenerate and we get the surface inside the \textbf{focal hyperbola} defined by
    \begin{equation}
    	\label{astronomy:focal_hyperbola}
    	\stylefrac{x^2}{\alpha-\beta} + \stylefrac{z^2}{\gamma-\beta} = 1
    \end{equation}
    This hyperbola intersects the $z$-plane in the foci of the focal ellipse.
    
    \subsection{Two-sheet hyperboloid\texorpdfstring{$\text{: }\tau = \nu$}{}}
    By fixing $\tau=\nu$ in \ref{astronomy:ellipsoidal_defining_function} we obtain the equation of two-sheet hyperboloid (also called an \textbf{elliptic hyperboloid}) around the $z$-axis. By letting $\nu\rightarrow-\beta$ the hyperboloid becomes degenerate and we obtain the surface outside the focal hyperbola \ref{astronomy:focal_hyperbola}. If $\nu\rightarrow-\gamma$ the two sheets coincide in the $xy$-plane.
    
    \subsection{Hamiltonian function}
    When wrting out the kinetic energy in ellipsoidal coordinates by applying the chain rule for differentiation to the Cartesian kinetic energy while noting that mixed terms of the form $\pderiv{x^a}{\lambda^i}\pderiv{x^a}{\lambda^j}$ cancel out when writing them out using \ref{astronomy:ellipsoidal_coordinates} it is clear that the Hamiltonian function can be spearated as follows:
    \begin{equation}
    	H = \frac{1}{2}\left(\stylefrac{p_\lambda^2}{Q_\lambda^2} + \stylefrac{p_\mu^2}{Q_\mu^2} + \stylefrac{p_\nu^2}{Q_\nu^2}\right) + V
    \end{equation}
    where $Q_j^2 = \sum_i\left(\pderiv{x^i}{\lambda^j}\right)^2$ are the metric coefficients in ellipsoidal coordinates.
    
    These coefficients can be calculated by noting that $\pderiv{x^i}{\lambda} = \frac{1}{x^i}\pderiv{(x^i)^2}{\lambda}$ and putting $\frac{1}{(\lambda + \alpha)(\lambda + \beta)(\lambda + \gamma)}$ in the front. Furthermore the coefficient belonging to $\lambda^2, \mu^2, \nu^2$, mixed terms and others can be calculated easily. By doing so we obtain following result
    \begin{equation}
    	Q_\lambda^2 = \frac{1}{4}\stylefrac{(\lambda - \mu)(\lambda - \nu)}{(\lambda + \alpha)(\lambda + \beta)(\lambda + \gamma)}
    \end{equation}
    which is also valid for $\mu$ and $\nu$ by applying cyclic permutation to the coordinates.
    
    Following from the St\"ackel conditions \ref{lagrange:stackel_condition} the potential must be of the form
    \begin{equation}
    	V = \sum_i\stylefrac{W_i(\lambda^i)}{Q_i^2}
    \end{equation}
    if we want to obtain a seperable Hamilton-Jacobi equation. Due to the disjunct nature of $\lambda, \mu$ and $\nu$ we can consider $W_\lambda, W_\mu$ and $W_\nu$ as three parts of a single function $G(\tau)$ given by:
    \begin{equation}
    	G(\tau) = -4(\tau + \beta)W_\tau(\tau)
    \end{equation}
    The 3D potential is thus completely determined by a 1D function $G(\tau)$.
    
    \subsection{Hamilton-Jacobi equation}
    If we consider a time-independent system we can use \ref{lagrange:time_independent_hje} as our starting point. If we multiply this equation by $(\lambda - \mu)(\lambda - \nu)(\mu - \nu)$ we obtain
    \begin{multline}
    	(\mu - \nu)\left[2(\lambda + \alpha)(\lambda + \beta)(\lambda + \gamma)\left(\deriv{S^\lambda(\lambda)^2}{\lambda}\right)\right.\\ - (\lambda + \alpha)(\lambda + \gamma)G(\lambda) - \lambda^2E \bigg] + \text{cyclic permutations} = 0
    \end{multline}
    where we rewrote the multiplication factor in the form $a\lambda^2 + b\mu^2 + c\nu^2$ before multiplying the RHS of \ref{lagrange:time_independent_hje}. This equation can be elegantly rewritten as
    \begin{equation}
    	(\mu-\nu)U(\lambda) + (\lambda - \mu)U(\nu) + (\nu - \lambda)U(\mu) = 0
    \end{equation}
    Differentiating twice with respect to any $\lambda^i$ gives $U''(\tau) = 0$ or equivalently
    \begin{equation}
    	U(\tau) = I_3 - I_2\tau
    \end{equation}
    where $I_2$ and $I_3$ are two new first integrals of motion.
    
    From the Hamiltonian-Jacobi equations of motion one can calculate the conjugate momenta $p_\tau = \deriv{S^\tau}{\tau}$. After a lengthy calculation we obtain
    \begin{equation}
    	p_\tau^2 = \stylefrac{1}{2(\tau + \beta)}\left[E - V_{\text{eff}}(\tau)\right]
    \end{equation}
    where the effective potential is given by
    \begin{equation}
    	\boxed{V_{\text{eff}} = \stylefrac{J}{\tau + \alpha} + \stylefrac{K}{\tau + \gamma} - G(\tau)}
    \end{equation}
    where $J$ and $K$ are two conserved quantities given by
    \[J = \stylefrac{\alpha^2E + \alpha I_2 + I_3}{\alpha - \gamma} \quad\text{and}\quad K = \stylefrac{\gamma^2E + \gamma I_2 + I_3}{\gamma - \alpha}\]
    To be physically acceptable, $p_\tau^2$ should be positive. This leads to following conditions on the energy:
    \begin{equation}
    	\begin{cases}
    		E\geq V_{\text{eff}}(\lambda)\\
            E\geq V_{\text{eff}}(\mu)\\
            E\leq V_{\text{eff}}(\nu)
    	\end{cases}
    \end{equation}The generating $G(\tau)$ function should also satisfy some conditions. First we note that we can rewrite our St\"ackel potential $V(\lambda, \mu, \nu)$ as
    \begin{equation}
    	\label{astronomy:potential2}
    	V = -\stylefrac{1}{\lambda - \nu}\left(\stylefrac{F(\lambda) - F(\mu)}{\lambda - \mu} - \stylefrac{F(\mu) - F(\nu)}{\mu - \nu}\right) \leq 0
    \end{equation}
    where $F(\tau) = (\tau + \alpha)(\tau + \gamma)G(\tau)$.
    
     For $\lambda\rightarrow+\infty$ (or $r^2\rightarrow+\infty$) we get $V \approx -\frac{F(\lambda)}{\lambda^2} \approx -G(\lambda)$. Because $V\sim \lambda^{-1}$ it is clear that $G(\tau)$ cannot decay faster than $\lambda^{-1/2}$ at infity. Furthermore we can interpret \ref{astronomy:potential2} as an approximation of $-F''(\tau)$. So it follows that $F(\tau)$ should be convex. For $\tau\rightarrow-\gamma$ we get
    \[\begin{cases}
    	\alpha + \tau < 0\\
        \tau + \gamma \rightarrow 0\\
    \end{cases}\]
    So if $G(\tau)$ decays faster than $\ds\frac{1}{\tau + \gamma}$ then $F(\tau)\rightarrow-\infty$ which is not possible for a convex function.
    
    To fullfil these conditions we assume that the generating function can be written as
    \begin{equation}
    	\boxed{G(\tau) = \stylefrac{GM}{\sqrt{\gamma_0 + \tau}}}
    \end{equation}
    where $G$ is the gravitational constant and $M$ is the galactic mass.
    
    \begin{theorem}[Kuzmin's theorem]
    	The spatial mass density function generated by a St\"ackel potential is completely determined by a function of the form $\rho(z)$.
    \end{theorem}
    \begin{result}
    	For triaxial mass models in ellipsoidal coordinates the axial ratios are inversely proportional to the axial ratios of the coordinate system.
    \end{result}

\chapter{Optics}

\section{General}
\subsection{Conservation of energy}

    From the law of conservation of energy, one can derive the following formula:
    \begin{gather}
        \label{optics:energy_conservation}
        T+R+A=1\,,
    \end{gather}
    where:
    \begin{itemize}
        \item $T$ is the transmission coefficient,
        \item $R$ is the reflection coefficient, and
        \item $A$ is the absorption coefficient.
    \end{itemize}

\section{Wave mechanics}

    \newformula{Wave equation}{\index{wave!equation}\index{d'Alembert!operator}
        In one spatial dimension the wave equation reads as follows:
        \begin{gather}
            \label{optics:wave_equation}
            \mpderiv{2}{f}{x} = \frac{1}{v^2}\mpderiv{2}{f}{t}\,.
        \end{gather}
        In higher dimensions this can be rewritten using the Laplacian (\cref{vector:laplacian}) as
        \begin{gather}
            \Delta f = \frac{1}{v^2}\mpderiv{2}{f}{t}\,.
        \end{gather}
        or using the \textbf{d'Alembertian}
        \begin{gather}
            \Box_v := \frac{1}{v^2}\mpderiv{2}{}{t}-\Delta
        \end{gather}
        as
        \begin{gather}
            \Box_v f = 0\,.
        \end{gather}
    }

    \newformula{d'Alemberts formula}{
        Consider the wave equation $\Box f=0$. By applying the method from \cref{section:characteristics}, it is clear that the characteristics are given by
        \begin{gather}
            \xi = x + ct \qquad\text{and}\qquad \eta = x - ct\,.
        \end{gather}
        Furthermore, it follows that the wave equation is a hyperbolic equation and, accordingly, can be rewritten in the canonical form
        \begin{gather}
            \label{optics:canonical_wave_equation}
            \frac{\partial^2u}{\partial\xi\partial\eta}(\xi,\eta) = 0\,.
        \end{gather}
        Integration with respect to $\xi$ and $\eta$ and rewriting the solution in terms of $x$ and $t$ gives
        \begin{gather}
            \label{pde:wave_solution}
            u(x,t) = f(x+ct) + g(x-ct)\,,
        \end{gather}
        where $f,g\in C^\infty(\mathbb{R})$ are arbitrary. This solution represents a superposition of a left-moving and a right-moving wave.

        Now, consider the wave equation subject to the general boundary conditions
        \begin{gather}
            u(x,0) = v(x) \qquad\text{and}\qquad \pderiv{u}{t}(x,0) = q(x)\,.
        \end{gather}
        By inserting these conditions in the solution~\eqref{pde:wave_solution}, it can be shown that the general solution, subject to the given boundary conditions, is given by
        \begin{gather}
            \label{optics:dalembert_solution}
            u(x,t) := \frac{1}{2}\bigl(v(x+ct) + v(x-ct)\bigr) + \frac{1}{2c}\Int_{x-ct}^{x+ct}q(z)\,dz\,.
        \end{gather}
    }

    \newdef{Harmonic wave}{\index{harmonic!wave}\index{amplitude}\index{frequency}\index{period}\index{wave!number}\index{velocity!phase}\index{phase}\label{optics:plane_wave}
        The wave form obtained by setting \[v(x\pm vt)=A\sin\left(\frac{2\pi}{\lambda}(x\pm vt)\right)\] in d'Alembert's formula, where the constants $A,\lambda$ are called the \textbf{amplitude} and \textbf{wave length}, respectively. The wave length determines the distance between consequent minima/maxima.

        This form can be expressed in various other forms:
        \begin{align*}
            \xi(x,t) &= A\sin\left(\frac{2\pi}{\lambda}(x\pm vt)\right)\\
            &= A\sin\left(2\pi(\frac{x}{\lambda}\pm \nu t)\right)\\
            &= A\sin\left(2\pi(\frac{x}{\lambda}\pm\frac{t}{\tau})\right)\\
            &= A\sin\left(kx\pm \omega t\right)\,.
        \end{align*}
        The constants $\omega,\nu,\tau$ and $k$ are respectively called the \textbf{circular frequency, frequency, period} and \textbf{wave number}. The speed at which such a wave moves is called the \textbf{phase velocity}:
        \begin{gather}
            v := \lambda\nu\,.
        \end{gather}
        One can also introduce an additional constant such that the wave form is shifted in space/time:
        \begin{gather}
            \xi(x,t) = A\sin\left(\frac{2\pi}{\lambda}(x\pm vt) + \varphi\right)\,.
        \end{gather}
        This constant $\varphi$ is called a \textbf{phase} (shift).
    }

    \newdef{Group velocity}{\index{velocity!group}
        Consider a superposition of two harmonic (cosine) waves with equal amplitudes:
        \begin{align*}
            u(x,t) &= A\cos(kx-\omega t) + A\cos(k'x-\omega't)\\
            &= 2A\cos\left(\frac{k'-k}{2}x - \frac{\omega'-\omega}{2}t\right)\cos\left(\frac{k+k'}{2}x - \frac{\omega+\omega'}{2}t\right)\,.
        \end{align*}
        When $\omega\neq\omega'$ or $k\neq k'$, the resulting wave is not purely harmonic. The result is a combination of a \textbf{carrier wave} with phase velocity $\frac{\omega+\omega'}{k+k'}$ and an \textbf{envelope} with phase velocity $\frac{\omega-\omega'}{k-k'}$. The phase velocity of the envelope is called the group velocity. This is the velocity with which the information contained in the modulated signal is transmitted.

        When the wave numbers are infinitesimally separated $k\approx k'$, the group velocity can be written as a derivative:
        \begin{gather}
            v_g := \deriv{\omega}{k}\,.
        \end{gather}
        This is the definition of the group velocity of an arbitrary wave packet \[u(x,t)=\Int_{-\infty}^\infty A(k)e^{i(kx-\omega t)}\,dk\,.\]
    }

    \begin{method}[Huygens's principle]\index{Huygens's principle}
        Every point of the wave front acts as a source for ``secondary'' spherical waves. The wave front at later points is obtained by taking the boundary of the convex hull of these secondary waves.
    \end{method}

\section{Electromagnetic radiation}

    Consider Maxwell's equations from \cref{section:maxwell_equations}. By inserting Maxwell's law in the rotor of Faraday's law, the following equation for the electric field is obtained:
    \begin{gather}
        \Delta\vector{E} = \varepsilon_0\mu_0\mpderiv{2}{\vector{E}}{t}\,.
    \end{gather}
    This has the form of the wave equation~\eqref{optics:wave_equation} with speed $c^2=\frac{1}{\varepsilon_0\mu_0}$. A similar derivation gives rise to a wave equation of the magnetic field. Moreover, through the Maxwell equations, the solutions of these wave equations are coupled. For example, if one considers harmonic solutions:
    \begin{align*}
        \vector{E}(x,t) &= \vector{E}_0\sin(kx-\omega t)\,,\\
        \vector{B}(x,t) &= \vector{B}_0\sin(kx-\omega t)\,,
    \end{align*}
    Faraday's law gives
    \begin{gather}
        \vector{E}_0k\cos(kx-\omega t) = \vector{B}_0\omega\cos(kx-\omega t)\,.
    \end{gather}
    This in turns gives rise to the following important identity:
    \begin{gather}
        \frac{\|\vector{E}_0\|}{\|\vector{B}_0\|} = \frac{\omega}{k} = c\,.
    \end{gather}

    \newformula{Energy}{\label{optics:energy}
        \begin{gather}
            E = h\nu = \hbar\omega = \frac{hc}{\lambda}
        \end{gather}
    }
    \newformula{Momentum}{\label{optics:momentum}
        \begin{gather}
            p = \frac{h}{\lambda} = \hbar k
        \end{gather}
    }

    \newdef{Polarization}{\index{polarization}
        The polarization of an electromagnetic wave is defined as the direction of the $\vector{E}$-component.

        Consider a superposition of two orthogonally polarized plane waves:
        \begin{align*}
            \vector{E}_1(x,t) &= E_1\sin(kx-\omega t)\symbf{\hat{e}_y}\,,\\
            \vector{E}_2(x,t) &= E_2\sin(kx-\omega t+\varphi)\symbf{\hat{e}_z}\,.
        \end{align*}
        If the phase shift $\varphi$ vanishes, the superposition is \textbf{linearly polarized}, because the resulting polarization is fixed. If the phase shift is $\varphi=\pm\frac{\pi}{2}$, i.e.~the waves are perfectly out of phase, the resulting wave is said to be \textbf{elliptically polarized}. The resulting polarization vector rotates in the $yz$-plane. When the amplitudes coincide, elliptically polarized light is also said to be \textbf{circularly polarized}, since the polarization vector draws a circular in the $yz$-plane.
    }

\section{Refraction}

    \newformula{Refractive index}{\index{refraction}\index{index!refractive}
        The relative refractive index between two materials is defined as follows:
        \begin{gather}
            \label{optics:refraction}
            n := \frac{v_1}{v_2}\,,
        \end{gather}
        where $v_1,v_2$ are the speeds of light in the first and second material, respectively. If the first material is chosen to be the vacuum, i.e.~$v_1=c$, the \textbf{absolute refractive index} is obtained.
    }
    \begin{formula}[Dielectric function]\index{dielectric constant}\label{optics:dielectric_function_non_magnetic}
        In the case of nonmagnetic materials ($\mu_r\approx1$), one can write the dielectric function as follows:
        \begin{gather}
            \epsilon = \epsilon_r + i\epsilon_i =: \widetilde{n}^2 = (n+ik)^2\,,
        \end{gather}
        where $\widetilde{n}$ is called the \textbf{complex refractive index} and $k$ the \textbf{extinction coefficient}.
    \end{formula}

    \begin{theorem}[Malus--Dupin]\index{Malus--Dupin}
        Light rays are always perpendicular to the wave fronts.
    \end{theorem}

    \begin{property}[Reflection and refraction laws]\index{Snell's law}
        For plane waves reflection and refraction are characterized by the following three rules:
        \begin{enumerate}
            \item The normal to the interface and the incoming, reflected and refracted rays all lie in one plane.
            \item The incoming and reflected angles coincide.
            \item The incoming and refracted angles satisfy \textbf{Snell's law}:
            \begin{gather}
                \frac{\sin(\theta_i)}{\sin(\theta_r)}=\frac{n_2}{n_1}\,.
            \end{gather}
        \end{enumerate}
        The second and third item can be derived from both Huygens's principle and Malus's law or Fermat's principle.
    \end{property}

\section{Absorption}

    \begin{theorem}[Law of Lambert--Beer]\index{Lambert--Beer}\label{optics:lambert_beer}
        Let $I(0)$ be the intensity of a light beam incident on a material. After penetrating a distance $x$ in the material, the intensity is given by
        \begin{gather}
            I(x) = I(0)\exp\left(-\frac{4\pi\nu k}{c}x\right);\,.
        \end{gather}
        \begin{mdframed}[roundcorner=10pt, linecolor=blue, linewidth=1pt]
            \begin{proof}
                From \cref{optics:dielectric_function_non_magnetic} it is known that the complex refractive index can be written as \[\widetilde{n} = n+ik\,,\] where $k$ is called the \textbf{extinction coefficient}. From classical optics it is also known that in a material the speed of light obeys the following relation: \[c = \widetilde{n}v\,.\] It readily follows that the wave number (sadly also denoted by the letter $k$) can be written as \[k = \frac{\omega}{v} = \widetilde{n}\frac{\omega}{c}\,.\] From electromagnetism one knows that a plane wave can be written as \[E(x,t) = \mathrm{Re}\bigl\{A\exp\left[i(kx - \omega t + \phi)\right]\bigr\}\,.\] So, after putting everything together, one obtains \[E(x,t) = \mathrm{Re}\left\{A\exp\left[i\left((n+ik)\frac{\omega}{c}x - \omega t + \phi\right)\right]\right\}\] or \[E(x,t) = \mathrm{Re}\left\{A\exp\left(in\frac{\omega}{c}x\right)\cdot\exp\left(-k\frac{\omega}{c}x\right)\cdot\exp\left(-i\omega t\right)\cdot\exp\left(i\phi\right)\right\}\,.\] It is also known that the intensity is given by the following relation:\[I(x) = |E(x)|^2\,.\] This implies that only the second exponential factor will remain. Dividing the result by its value at $x=0$ gives \[\frac{I(x)}{I(0)} = \frac{|E(x)|^2}{|E(0)|^2} = \exp\left(-\frac{2k\omega}{c}x\right) = \exp(-\alpha x)\,,\]
                where $\alpha$ is the absorption coefficient as defined in \cref{optics:absorption_coefficient}.
            \end{proof}
        \end{mdframed}
    \end{theorem}

    \newdef{Absorption coefficient}{\index{absorption}\label{optics:absorption_coefficient}
        The scale factor
        \begin{gather}
            \alpha = \frac{4\pi\nu k}{c}
        \end{gather}
        in the Lambert--Beer law is called the absorption coefficient.
    }

    \newdef{Scattering}{\index{Rayleigh!scattering}\index{Raman scattering}
        When scattering light off matter, two situations can occur:
        \begin{itemize}
            \item\textbf{Rayleigh scattering}: Photons are elastically scattered off the bound electrons. The incoming and outgoing frequencies are equal.
            \item\textbf{Raman scattering}: Photons are inelastically scattered.
        \end{itemize}
    }
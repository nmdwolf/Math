\chapter{Optics}

\section{General}
\subsection{Conservation of energy}

    From the law of conservation of energy one can derive the following formula:
    \begin{gather}
        \label{optics:energy_conservation}
        T+R+A=1,
    \end{gather}
    where
    \begin{itemize}
        \item $T$ is the transmission coefficient,
        \item $R$ is the reflection coefficient, and
        \item $A$ is the absorption coefficient.
    \end{itemize}

\section{Plane wave}

    \newdef{Wave number}{\index{wave!number}\label{optics:wave_number}
        Consider a (plane) wave with wavelength $\lambda$. Its wave number is defined as follows:
        \begin{gather}
            k := \frac{2\pi}{\lambda}.
        \end{gather}
    }
    \begin{formula}[Plane wave]\label{optics:plane_wave}
        The following equations represent a plane wave, (linearly) polarized in the $xy$-plane, that moves in the $x$-direction:
        \begin{align}
            \vector{E}(x,t) &= \mathrm{Re}\Big\{A\exp\left[i\left(kx - \omega t + \phi\right)\right]\Big\}\vector{e}_y\\
            &= \mathrm{Re}\left\{A\exp\left[2\pi i\left(\frac{x}{\lambda} - \frac{t}{T} + \frac{\phi}{2\pi}\right)\right]\right\}\vector{e}_y.
        \end{align}
    \end{formula}

\subsection{Photon}

    \newformula{Energy}{\label{optics:energy}
        \begin{gather}
            E = h\nu = \hbar\omega = \frac{hc}{\lambda}
        \end{gather}
    }
    \newformula{Momentum}{\label{optics:momentum}
        \begin{gather}
            p = \frac{h}{\lambda} = \hbar k,
        \end{gather}
        where Definition \ref{optics:wave_number} was used in the last step.
    }

\section{Refraction}

    \newformula{Refractive index}{\index{refraction}
        The relative refractive index between two materials is defined as follows:
        \begin{gather}
            \label{optics:refraction}
            n := \frac{v_1}{v_2},
        \end{gather}
        where $v_1,v_2$ are the speeds of light in the first and second material, respectively. If the first material is chosen to be the vacuum, i.e. $v_1=c$, the absolute refractive index is obtained.
    }
    \begin{formula}[Dielectric function]\index{dielectric constant}
        In the case of nonmagnetic materials ($\mu_r\approx1$), one can write the dielectric function as follows:
        \begin{gather}
            \label{optics:dielectric_function_non_magnetic}
            \epsilon = \epsilon_r + i\epsilon_i = \widetilde{n}^2 = (n+ik)^2,
        \end{gather}
        where $\widetilde{n}$ is the \textbf{complex refractive index} and $k$ is the \textbf{extinction coefficient}.
    \end{formula}

\section{Absorption}

    \begin{theorem}[Law of Lambert-Beer$^\dag$]\index{Lambert-Beer}\label{optics:lambert_beer}
        Let $I$ be the intensity of a light beam.
        \begin{gather}
            \frac{I(x)}{I(0)} = \exp\left(-\frac{4\pi\nu k}{c}x\right)
        \end{gather}
    \end{theorem}

    \begin{definition}[Absorption coefficient]\index{absorption}\label{optics:absorption_coefficient}
        The scale factor in the Lambert-Beer law is called the absorption coefficient:
        \begin{gather}
            \alpha = \frac{4\pi\nu k}{c}.
        \end{gather}
    \end{definition}
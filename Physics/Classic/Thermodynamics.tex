\chapter{Thermodynamics}

\section{General definitions}

    \newdef{System}{\index{system}
        The part of space that is of interest.
    }
    \newdef{Environment}{\index{environment}
        The complement of the system in space. More specifically, this denotes the part of space that has a potential influence on the system.
    }

    \newdef{Thermodynamic coordinate}{\index{coordinate}
        Macroscopical variable that describes the system. These are also called \textbf{state variables}.
    }
    \newdef{Intensive coordinate}{
        Coordinate that does not depend on the system size. The opposite notion is called an \textbf{extensive coordinate}.
    }
    \newdef{Thermodynamic equilibrium}{
        A system is said to be in thermodynamic equilibrium if it is simultaneously in thermal, mechanical and chemical equilibrium. In this case, it is fully described by a set of constant thermodynamic coordinates.
    }
    \begin{property}[Uniformity]
        During thermodynamic equilibrium all intensive coordinates are uniform throughout the system.
    \end{property}

    \newdef{Isolated system}{
        A system that cannot interact with its environment (e.g.~due to the presence of impenetrable walls).
    }
    \newdef{Diathermic wall}{\index{wall}
        A wall that only allows heat transfer. This should be distinguished from an \textbf{adiabatic wall}, i.e.~a wall that does not allow any transfer of heat.
    }
    \newdef{Heat bath}{\index{bath}\index{reservoir}
        A heat bath or \textbf{thermal reservoir} is a thermodynamic system (often part of the environment) for which the temperature remains constant during the exchange of heat, i.e.~it has a virtually infinite heat capacity.
    }
    \newdef{Open system}{\index{open!system}
        A system that is allowed to interact with its environment.
    }
    \newdef{Quasistatic process}{
        A sequence of equilibrium states separated by infinitesimal changes.
    }
    \newdef{Path}{\index{path}
        The sequence of equilibrium states in a thermodynamic process is called its path.
    }

\section{Postulates}

    \begin{axiom}[Zeroth law]
        If two objects are in thermal equilibrium with a third object, they are also in thermal equilibrium with each other.
    \end{axiom}
    \begin{axiom}[First law]
        The change in internal energy is given by
        \begin{gather}
            \label{thermo:first_law}
            \Delta U = Q + W\,,
        \end{gather}
        or, infinitesimally, by
        \begin{gather}
            \label{thermo:first_law_differential}
            \dr U = \delta Q + \delta W\,,
        \end{gather}
        where $W$ denotes the work done on the system and $Q$ denotes the heat that was extracted from the environment.
    \end{axiom}
    \sremark{The $\delta$ in the heat and work differentials implies that these are `inexact' differentials, i.e.~they are not the differential of functions of the thermodynamic coordinates alone. See \cref{section:forms} for more information on differential forms.}

    \begin{axiom}[Kelvin--Planck formulation of the second law]
        No machine can absorb an amount of heat and completely transform it into work.
    \end{axiom}
    \begin{axiom}[Clausius formulation of the second law]
        Heat cannot be passed from a cooler object to a warmer object without performing work.
    \end{axiom}

    \newformula{Clausius's inequality}{\index{Clausius!inequality}\label{thermo:clausius_inequality}
        In differential form, the inequality reads as
        \begin{gather}
            \frac{\delta Q}{T} \geq 0\,.
        \end{gather}

        \todo{COMPLETE THIS STATEMENT (WHAT INEQUALITY?)}
    }

    \begin{axiom}[Third law]
        No process can reach absolute zero through a finite sequence of operations.
    \end{axiom}

\section{Equations of state}

    \begin{property}[PV-system]
        Consider a system described by pressure $P$ and volume $V$. The first law becomes
        \begin{gather}
            \delta Q = \dr U + P\dr V\,.
        \end{gather}
        This formula defines a Pfaffian system (\cref{bundle:pfaff}). Since the state space is only two-dimensional, it is not too hard to see that it is integrable, i.e.~there exists an integrating factor $T$ and a one-form $\dr S$ such that
        \begin{gather}
            T\dr S = \dr U + P\dr V\,.
        \end{gather}
        $T$ and $S$ represent the temperature and entropy of the system, respectively. If the heat exchange is zero, the entropy is conserved.
    \end{property}

\section{Gases}

    \begin{formula}[Ideal gas law]\index{ideal!gas law}\label{thermo:ideal_gas_law}
        \begin{gather}
            PV = nRT\,,
        \end{gather}
        where $R$ is the \textbf{ideal gas constant} $R\approx8.314\frac{\text{J}}{\text{K\,mol}}$.
    \end{formula}
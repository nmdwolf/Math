\chapter{BRST Theory}\label{chapter:constrained_dynamics}

    The foundations for this subject were laid down by~\citet{dirac_generalized_1950}. By introducing constraints, the coordinates and their momenta become dependent. This implies, for example, that the Hamiltonian equations of motion have to be modified.

    \minitoc

\section{Constrained systems}\label{section:constrained_systems}
\subsection{Constraints}

    \newdef{Holonomic constraint}{\index{holonomic!constraint}
        A constraint $f(q,t) = 0$ that only depends on the coordinates $q^i$ and $t$ and not on the derivatives.
    }
    \begin{method}[Holonomic constraints]\index{Lagrange!multiplier}
        The Euler--Lagrange equations of a system with $k\in\mathbb{N}$ holonomic constraints
        \begin{gather}
            \zeta_k(q,t)=0
        \end{gather}
        can be obtained from the extended action functional
        \begin{gather}
            S_E[q,\lambda]:=\Int_{t_1}^{t_2}\left[L(q,\dot{q},t) - \sum_{i=1}^k\lambda_i(t)\zeta_i(q,t)\right]\,dt\,,
        \end{gather}
        where the $\lambda_i$ are Lagrange multipliers (\cref{section:lagrange_multipliers}). Extremizing with respect to these multipliers induces the constraints:
        \begin{gather}
            \frac{\delta S_E}{\delta\lambda_i}=\zeta_i=0\,.
        \end{gather}
    \end{method}

    First, recall the Lagrangian equations of motion~\eqref{classic:second_kind}. By expanding these equations, it can be shown that the accelerations $\ddot{q}$ are uniquely determined by the coordinates and velocities $(q,\dot{q})$ if and only if the Hessian of the Lagrangian is invertible. If the Hessian is not invertible, the definition of the conjugate momenta (\cref{classic:conjugate_momentum}) cannot be inverted to express velocities in terms of momenta. Alternatively, the coordinates $q$ and momenta $p$ are not independent and there must exist relations of the form
    \begin{gather}
        \zeta(q,p) = 0\,.
    \end{gather}
    Constraints of this type are called \textbf{primary constraints}. They do not serve to constrain the range of the coordinates $q$, they only couple the coordinates and the momenta.

    \begin{axiom}[Regularity conditions]\label{constraint:regularity}
        It will always be assumed that the independent constraints, i.e.~the minimal generating set, satisfy the following (equivalent) conditions:
        \begin{itemize}
            \item The constraints can locally serve as the first coordinates of a (regular) coordinate system.
            \item The differentials (gradients) $\dr\zeta_m$ are locally linearly independent.
            \item The variations $\delta\zeta_m$ are of the order $\varepsilon$ whenever the variations $\delta q^i,\delta p_i$ are of the order $\varepsilon$. (This is the original condition due to \indexauthor{Dirac}.)
        \end{itemize}
    \end{axiom}

    A constrained dynamical system consists of a dynamic system $(M,\omega,H)$ together with a finite collection of constraints $\zeta_m(q,p)=0,m\in I$. If this system is derived from a Lagrangian $L(q,\dot{q})$, the calculus of variations easily extends to these constrained systems, where it gives the following modified Hamiltonian equations:
    \begin{align}
        \dot{q}^i &= \pderiv{H}{p_i} + \sum_{m\in I}u_m\pderiv{\zeta_m}{p_i}\,,\\
        \dot{p}_i &= -\pderiv{H}{q^i} - \sum_{m\in I}u_m\pderiv{\zeta_m}{q^i}\,,
    \end{align}
    where the $u_m$ are functions of the coordinates and velocities that play a role similar to ordinary Lagrange multipliers. In terms of Poisson brackets, the constrained time evolution of a (time-independent) function is given by:
    \begin{gather}
        \label{constraint:modified_poisson_evolution}
        \dot{f} = \{H,f\} + \sum_{m\in I}u_m\{\zeta_m,f\}\,.
    \end{gather}

    \begin{remark}
        The above relations follow from the property that the general solutions to $\lambda_i\delta q^i + \mu_i\delta p^i = 0$ for variations $\delta q^i,\delta p_i$ tangent to the constraint surface are of the form
        \begin{gather}
            \begin{cases}
                &\lambda_i = \sum_{m\in I}u_m\pderiv{\zeta_m}{q^i}\,,\\
                &\mu^i = \sum_{m\in I}u_m\pderiv{\zeta_m}{p_i}\,.
            \end{cases}
        \end{gather}
        Combining this result with the usual derivation of Hamilton's equations from a Lagrangian action principle gives the above modified equations.
    \end{remark}

    \begin{method}[General Poisson brackets]
        Until now, Poisson brackets were only defined for functions depending on the canonical coordinates $(q,p)$. This definition can be generalized to arbitrary functions through the Poisson algebra properties (\cref{lie:poisson_algebra}). Furthermore, after working out the Poisson brackets, one can use the constraint equations to drop all terms that are proportional to $\zeta_m$.

        For example, \cref{constraint:modified_poisson_evolution} can be rewritten as
        \begin{gather}
            \dot{f} = \left\{H + \sum_{m\in I}u_m\zeta_m,f\right\}\,.
        \end{gather}
        To prove the equivalence, one can use the linearity and Leibniz properties. This involves the following equality
        \begin{gather}
            \{u_m\zeta_m, f\} = \{u_m,f\}\zeta_m + u_m\{\zeta_m,f\}\,.
        \end{gather}
        The Poisson brackets in the second term only involve functions depending on $(q,p)$ and can be calculated in the usual way. The first term, however, involves a Poisson bracket of the Lagrange multiplier $u_m$. In general, these do not simply depend on $q$ and $p$. Luckily, this does not pose a problem because the term is proportional to the constraints and, as such, vanishes on-shell. It is important that the constraints are only applied after the Poisson brackets have been fully worked out.
    \end{method}

    \begin{notation}[Weak equality]
        The constraints $\zeta_m$ only vanish on shell. To distinguish between functional equalities, i.e.~equalities that also hold off-shell, and on-shell equalities, also called \textbf{weak equalities}, the latter are often denoted by the $\approx$ symbol. For example, the condition $\zeta_m\approx0$ is only a weak equality.
    \end{notation}
    Using the above definitions, one can write an arbitrary time derivative as
    \begin{gather}
        \dot{f}\approx\{H_T,f\}\,,
    \end{gather}
    where $H_T := H + \sum_{m\in I}u_m\zeta_m$ is the total Hamiltonian.

    \begin{remark}[Closure]
        An important insight regarding weak equalities can be obtained by calculating the Poisson bracket with a function $f$ that is strongly zero, i.e.~a function that vanishes on shell and whose variation also vanishes. In this case, $\{f,g\}\approx0$ for all functions $g$, i.e.~the brackets only vanish weakly. Furthermore, if $f$ is only weakly zero, then $\{f,g\}$ does not even have to vanish at all.
    \end{remark}

    \begin{property}[Weakly vanishing functions]\label{constraint:weakly_vanishing_functions}
        Assuming that the constraints satisfy \cref{constraint:regularity}, a function that vanishes on shell is equal to some combination of the constraints (the coefficients might be functions themselves)~\citep[p.~70]{henneaux_quantization_1992}.
    \end{property}

    \begin{property}[Consistency conditions]
        By taking $f=\zeta_n$ for any $n\in I$ in \cref{constraint:modified_poisson_evolution}, a set of consistency conditions is obtained:
        \begin{gather}
            \label{constraint:inhomogeneous}
            \{H,\zeta_n\} + \sum_{m\in I}u_m\{\zeta_m,\zeta_n\}\approx 0\,.
        \end{gather}
        It is possible that this condition reduces to an inconsistency of the type $1\approx0$. In this case, the equations of motion are inconsistent and the theory is not physical. If this is not the case, multiple possibilities can arise:
        \begin{itemize}
            \item After imposing the primary constraints, a tautology $0=0$ is found. This gives no new information.
            \item The equation reduces to an equation not involving the Lagrange multipliers $u_m$. This gives an additional constraint
                \begin{gather}
                    \chi(q,p)=0\,.
                \end{gather}
                These are called \textbf{secondary constraints}.
            \item A condition on the coefficients $u_m$ is obtained.
        \end{itemize}
        After having found a set of secondary constraints, this procedure can be iterated until no new contraints or conditions are found. Because the consistency conditions are linear in the coefficients $u_m$, the general solution can be written as
        \begin{gather}
            u_m = U_m + v_aV^a _m\,,
        \end{gather}
        where $U_m$ is a solution of the inhomogeneous equation~\eqref{constraint:inhomogeneous} and the $V^a_m$ are linearly independent solutions of the homogeneous equation
        \begin{gather}
            \sum_{m\in I}u_m\{\zeta_m,\zeta_n\} = 0\,.
        \end{gather}
        The resulting coefficients $v_a$ are completely arbitrary functions of time and will be referred to as the \textbf{Lagrange multipliers}.\index{Lagrange!multiplier} Therefore, the total Hamiltonian can be written in the form
        \begin{gather}
            H_T = H'(q,p) + v^a(t)\zeta_a(q,p)\,,
        \end{gather}
        where $\zeta_a := \sum_{m\in I}V^a_m\zeta_m$. The occurence of arbitrary functions in the Hamiltonian implies that the evolution of the phase space variables is not unique and, accordingly, that the theory has a gauge freedom.
    \end{property}

    \newdef{First- and second-class}{\label{constraint:first_and_second_class}
        A function $f(q,p)$ is said to be first class if its Poisson bracket with every constraint (both primary and secondary) is weakly zero. The function is said to be second class otherwise. It can be shown that both the total Hamiltonian $H_T$ and the primary constraints $\zeta_a$ are first class. The number of Lagrange multipliers $v^a$ is equal to the number of primary, first-class constraints.

        For first-class constraints $\{\zeta_a\}_{a\in I}$, this leads to the following structure:
        \begin{gather}
            \label{constraint:first_class_bracket}
            \{\zeta_a,\zeta_b\} = C^c_{ab}\zeta_c + T_{ab}^i\frac{\delta S}{\delta x^i}\,,
        \end{gather}
        where $C^c_{ab}:M\rightarrow\mathbb{R}$ are the \textbf{structure functions} and the $T^i_{ab}:M\rightarrow\mathbb{R}$ are antisymmetric under $a\leftrightarrow b$. The structure of the constraint algebra depends on the properties of these coefficients:
        \begin{itemize}\index{soft!algebra}\index{open!algebra}
            \item $T_{ab}=0$ and $C^c_{ab}$ constant: \textbf{closed algebra} (this corresponds to Lie or $L_\infty$-algebras),
            \item $T_{ab}=0$: \textbf{soft algebra} (this corresponds to Lie or $L_\infty$- algebroids~\ref{hdg:lie_algebroid}), and
            \item in general: \textbf{open algebra} (this corresponds to \textit{central extensions of $L_\infty$-algebroids})\,.
        \end{itemize}
    }
    \begin{notation}
        To distinguish between first- and second-class constraints, the latter are often denoted by a separate symbol $\chi$.
    \end{notation}
    \begin{property}[Closure]
        The Poisson bracket of two primary, first-class functions is first-class. So is the Poisson bracket of the total Hamiltonian and a primary, first-class constraint.
    \end{property}
    \begin{result}[Lie algebra]
        The first-class constraints form a Lie algebra with respect to the Poisson bracket and the associated gauge transformations define a submanifold in phase space by Frobenius' theorem~\ref{bundle:frobenius}.
    \end{result}

    \begin{property}
        If the constraint algebra is closed, every gauge-invariant function is gauge equivalent to a strongly gauge-invariant function.
    \end{property}

    \begin{remark}[Dirac conjecture]\index{Dirac!conjecture}
        The primary, first-class constraints generate gauge transformations in the sense that variations in the coefficients, which are arbitrary, give rise to phase space variations that leave the physical state invariant. Some secondary constraints might also generate gauge transformations and \indexauthor{Dirac} even conjectured that this was the case for all constraints. However, counterexamples have been found. A common workaround is simply to restrict to systems where the conjecture is true and, from here on, the distinction between primary and secondary will be dropped.\mnote{\dbend} From this point of view, it makes sense to define the extended Hamiltonian
        \begin{gather}
            \label{constraint:extended_hamiltonian}
            H_E := H_T + v^b(t)\zeta_b(q,p)\,,
        \end{gather}
        where $b$ ranges over all secondary, first-class constraints. For gauge-invariant functions, i.e.~those functions whose Poisson bracket with all first-class constraints vanishes, evolution with all three Hamiltonians $H,H_T$ and $H_E$ is identical. For general functions, only $H_E$ takes into account the full gauge freedom.\footnote{Note that $H_T$ is the Hamiltonian that corresponds to a Lagrangian approach. $H_E$ gives a more general theory.}
    \end{remark}

    \begin{formula}[Degrees of freedom]
        The number of degrees of freedom is given by the following formula:
        \begin{align*}
            2\times\#\text{d.o.f.} =\ &\#\text{canonical coordinates}\\
            &- \#\text{second-class constraints}\\
            &- 2\times\#\text{first-class constraints}.
        \end{align*}
    \end{formula}

    \newdef{Dirac bracket}{\index{Dirac!bracket}\label{constraint:dirac_bracket}
        To take care of second-class constraints, \indexauthor{Dirac} introduced a modification of the Poisson bracket:
        \begin{gather}
            \{f,g\}_D := \{f,g\} - \{f,\chi_a\}C^{ab}\{\chi_b,g\}\,,
        \end{gather}
        where the $\chi_a$ are the second-class constraints and the (invertible) matrix $C^{ab}$ is the inverse of the matrix $C_{ab}:=\{\chi_a,\chi_b\}$.

        The benefit of using the Dirac bracket (after the Poisson bracket has been used to separate constraints in first-class and second-class constraints) is that second-class constraints become strong equalities, i.e.~they can be used even before evaluating further (Dirac) brackets. The Dirac bracket satisfies the same algebraic properties as the Poisson bracket, i.e.~it also defines a Poisson algebra. From here on, all constraints will be assumed to be first class, i.e.~the Poisson bracket will be assumed to be the one obtained after applying the Dirac procedure to all second-class constraints.
    }
    \begin{remark}
        Instead of splitting the constraints in first- and second-class instances and having to work with the nontrivial Dirac bracket, one can also try to remove second-class constraints in a different way. In the above formula for the degrees of freedom, the factor 2 on the right-hand side is obtained by the introduction of gauge-fixing conditions. What these actually do is turning first-class constraints into second-class ones. In fact, the converse is also possible. One can obtain all second-class constraints as gauge-fixed first-class constraints after enlarging the system (although this procedure is not unique). After doing this, there is no need for the Dirac bracket anymore and one can simply work with the Poisson bracket (with the added complexity that all constraints now only hold weakly).
    \end{remark}

    \newdef{Gauge-invariant functions}{\label{constraint:functions}
        Consider the algebra $C^\infty(M)$ of smooth functions on phase space. In the spirit of algebraic geometry, the space of functions on the constraint surface $\Sigma$ is given by the quotient algebra $C^\infty(\Sigma):=C^\infty(M)/\mathcal{N}$, where $\mathcal{N}$ is the ideal having $\Sigma$ as its zero locus, i.e.~$\mathcal{N}$ is the ideal generated by the constraints. The elements of $C^\infty(\Sigma)$ that are gauge-invariant, i.e.~first class with respect to first-class constraints, should be considered as the \textbf{classical observables}. Passing to this extension effectively amounts to solving a(n integrable) system through constants of motion.

        The restriction to gauge-invariant functions is also imperative if one wants to extend the Dirac bracket to $C^\infty(\Sigma)$. In general, the ideal $\mathcal{N}$ is not an ideal of the Dirac bracket. The gauge-invariant subalgebra is, in fact, the maximal subalgebra of $C^\infty(\Sigma)$ for which $\mathcal{N}$ is again an ideal.
    }

\subsection{Zero Hamiltonian}

    In the previous section, dynamical systems with constraints were considered. Using these tools, one can turn any system evolving under a physical, but nondynamical or external, time parameter $t$ into a system having time as a canonical coordinate. In this setting, the time variable is treated on the same footing as the other coordinates. Such a system is said to be \textbf{generally covariant}.\index{covariance!general}

    If one starts from the single-particle action
    \begin{gather}
        S[q,p] = \Int\left(p_i\dot{q}^i-H_0\right)\,dt,
    \end{gather}
    one can introduce time as a generalized coordinate with momentum $p_0$ by modifying the action as follows:
    \begin{gather}
        S[q,p,t,p_0,u] = \Int\bigl[p_0t'+p_iq'^i-u(p_0+H_0)\bigr]\,d\tau\,,
    \end{gather}
    where the primes indicate derivatives with respect to the parameter $\tau$ and $u$ acts as a Lagrange multiplier. It is easily checked that the resulting equations of motion are the same as for the original action.

    The system now involves a single constraint $H_0=-p_0$, which is first class. It is often called the \textbf{Hamiltonian constraint}.\index{Hamilton!constraint} Aside from this constraint, the extended action contains no first-class Hamiltonian. Evolution is solely determined by a (first-class) constraint and, therefore, is given by a gauge transformation.

    \begin{remark}[Nonholonomicity]
        All constraints were assumed to be holonomic, i.e.~they did not explicitly depend on time. The presence of time derivatives is not compatible with the Poisson/Dirac bracket. However, when passing to a generally covariant system as above, the time variable loses its peculiar character and all constraints can be handled in the same way.
    \end{remark}

    \begin{property}[Vanishing Hamiltonian]
        If the canonical coordinates $(q,p)$ transform as scalars under $\tau$-reparametrizations, the Hamiltonian is weakly zero. Conversely, zero Hamiltonians give rise to reparametrization-invariant systems.
    \end{property}

    Although in many cases, it is reasonable to introduce a certain split of the coordinates and break general covariance --- gauge-invariant observables and constants of motion are usually treated differently --- this is not required. For a (weakly) zero Hamiltonian, first class functions are exactly constant of motion.

\subsection{Field theories}

    Up until now, the theories were assumed to have a finite number of degrees of freedom. However, in field theory, the phase space is (a subspace of) the module of sections of some vector bundle and, as such, is infinite dimensional. To ensure the existence of a well-behaved BRST theory, one has to restrict to `local theories', i.e.~all relevant functions/functionals should only depend on the fields and their derivatives up to some finite order.
    \begin{axiom}[Locality]\index{locality}
        A Hamiltonian gauge theory is local if the following conditions are satisfied:
        \begin{enumerate}
            \item The Hamiltonian action is local:
            \begin{gather}
                \label{constraint:hamiltonian_action}
                S[z,\zeta] := \Int_{t_1}^{t_2}\Int_M\left(\dot{\phi}^I\pi_I-h_0-\lambda^a\zeta_a\right)\,d^nx\,dt\,,
            \end{gather}
            where the field momenta $\pi_I$, the Hamiltonian density $h_0$ and the constraints $\zeta_a$ are functions of the fields $\phi^I$ and their derivatives up to a finite order $k\in\mathbb{N}$, i.e.~they are elements of a finite jet bundle $J^k(M)$.
            \item The bracket $\{\phi^I(x),\phi^J(y)\}$ is local for all $I,J$ and $x,y\in M$.
            \item Any reducibility relation is again local.
        \end{enumerate}
        The second condition ensures that taking iterated brackets respects the locality of the theory.
    \end{axiom}

    \begin{axiom}[Local completeness]
        Any function that vanishes weakly is zero on the constraint surface by means differential identities of the constraints only, i.e.~it does not depend on the boundary conditions.
    \end{axiom}
    \begin{remark}
        Local completeness can always be ensured by adding constraints or by replacing constraints by their (integral) primitives.
    \end{remark}

    \Cref{constraint:regularity} can be extended to local field theories as follows.
    \begin{axiom}[Regularity condition]
        Let $V:=\{\zeta_a\}$ be a set of $k$-local constraints, with $k\in\mathbb{N}$. For every $n\in\mathbb{N}$, the jet prolongation $j^nV$ cuts out a submanifold of $J^{n+k}(M)$. It is assumed that there exists a generating set of $j^nV$ such that these independent constraints form a regular coordinate system of $j^nV$.
    \end{axiom}

    If the above axioms are satisfied, \cref{constraint:weakly_vanishing_functions} can be generalized to field theories.
    \begin{property}
        If a local function vanishes on the constraint surface, it is an element of some finite prolongation of the constraint algebra.
    \end{property}

\subsection{Constraint surface}

    \begin{property}[Geometric characterization]\index{phase space!reduced}
        Restricted to a first-class constraint surface $\Sigma$, the `symplectic' form becomes maximally degenerate with rank
        \begin{gather}
            \rk(\omega) = \dim(M) - 2\dim(\Sigma)\,.
        \end{gather}
        This essentially says that the constraint surface is coisotropic and, as a consequence, that the Poisson bracket is ill-defined (since this would involve the inverse of the symplectic form). After passing to the \textbf{reduced phase space}, i.e.~the leaf space of the Hamiltonian foliation generated by the constraints, one again obtains a well-defined Poisson bracket that coincides with the ordinary Poisson bracket without any constraints.

        The opposite situation arises for constraint surfaces that only involve second-class constraints. Here, the induced symplectic form is of maximal rank
        \begin{gather}
            \rk(\omega) = \dim(M) - \dim(\Sigma)\,,
        \end{gather}
        which implies that the surface is isotropic. Furthermore, the induced Poisson bracket coincides with the restriction of the Dirac bracket.
    \end{property}

    \begin{formula}[Reduced action]\label{constraint:reduced_action}
        Let $y^j(q^i)$ be regular coordinates on $\Sigma$. Let $\theta_j\dr y^j$ be a symplectic potential for the associated symplectic form (on $\Sigma$). The reduced action reads
        \begin{gather}
            S[y] = \Int_{t_1}^{t_2}(\theta_jy^j-H|_{\xi=0})\,dt\,.
        \end{gather}
        The reduced potential satisfies
        \begin{gather}
            \label{constraint:symplectic_potential}
            \theta_j\dr y^j \approx p_i\dr q^i + \dr M\,.
        \end{gather}
    \end{formula}

    \begin{remark}[Algebraic characterization]\index{co-!isotrope}\index{Poisson algebra!reduced}
        The fact that first-class constraints define a coisotropic submanifold is not a peculiarity. A multiplicative ideal of a Poisson algebra that is closed under the Poisson bracket, is often called a \textbf{coisotrope} (or \textbf{coisotropic ideal}). Coisotropic submanifolds of a Poisson manifold  (\cref{symplectic:poisson_manifold}) are exactly the zero loci of such coisotropes. In fact, one can restate the above constructions in purely algebraic terms. Given a Poisson algebra $P$ and a coisotrope $\mathcal{I}$, one can pass to the quotient $N(\mathcal{I})/\mathcal{I}$, where $N$ denotes the normalizer in $P$. This quotient is again a Poisson algebra, called the \textbf{reduced Poisson algebra}. This construction is strictly more general than the symplectic case considered above. (The sections further on could also be generalized to this setting.)
    \end{remark}

    \begin{theorem}[Abelianization]\index{Abelianization}
        Locally, one can always find a gauge transformation such that the constraint algebra becomes Abelian:
        \begin{gather}
            \{\zeta,\zeta'\}=0\,.
        \end{gather}
    \end{theorem}

\subsection{Fermionic systems}

    In this section, the study of constrained systems with `fermionic' or odd statistics is considered. For an introduction to Grassmann numbers, see \cref{section:berezin}. In general, the phase space will be assumed to be a supermanifold (\cref{hdg:supermanifold}).

    First, the ordinary Poisson bracket is extended to Grassmann-odd coordinates as follows:
    \begin{gather}
        \{\theta^i,\theta^j\} = 0 = \{\pi_i,\pi_j\}
    \end{gather}
    and
    \begin{gather}
        \{\theta^i,\pi_j\} = \delta^i_j = \{\pi_j,\theta^i\}\,.
    \end{gather}
    By defining the matrix $\sigma^{ij} := \{z^i,z^j\}$, where $z$ can be any of the $q,p,\theta$ or $\pi$, one can then succintly write the Poisson bracket of superfunctions as follows:
    \begin{gather}
        \{f,g\} := \pderiv{^Rf}{z^i}\sigma^{ij}\pderiv{^Lg}{z^j}\,.
    \end{gather}
    Note that this is virtually the same expression as the ordinary Poisson bracket, where the matrix $\sigma$ was the inverse of the symplectic matrix. Writing out all terms gives
    \begin{gather}
        \{f,g\} = \left(\pderiv{f}{p_i}\pderiv{g}{q^i}-\pderiv{f}{q^i}\pderiv{g}{p_i}\right) + (-1)^{\deg(f)}\left(\pderiv{f}{\pi_i}\pderiv{g}{\theta^i} + \pderiv{f}{\theta^i}\pderiv{g}{\pi_i}\right)\,.
    \end{gather}
    The algebraic properties of this generalized Poisson bracket are graded generalizations of those of the ordinary one:
    \begin{align}
        \{f,g\} &= -(-1)^{\deg(f)\deg(g)}\{g,f\}\\
        0 &= \{f,\{g,h\}\} + (-1)^{[\deg(f)+\deg(g)]\deg(h)}\{h,\{f,g\}\}\nonumber\\
        &\ \phantom{= \{f,\{g,h\}\} } + (-1)^{\deg(f)[\deg(g)+\deg(h)]}\{g,\{h,f\}\}\\\nonumber\\
        \{f,gh\} &= \{f,g\}h + (-1)^{\deg(f)\deg(g)}f\{g,h\}\\
        \deg(\{f,g\}) &= \deg(f)+\deg(g)\,.
    \end{align}
    The first two properties state that the generalized Poisson bracket gives rise to a Lie superalgebra (\cref{hda:lie_superalgebra}) and the third property states that it is a \textit{Poisson superalgebra}. In fact, this is the example that lends its name to the algebraic structure. Geometrically, the matrix $\sigma$ gives rise to a \textit{supersymplectic structure}.\index{Poisson!superalgebra}\index{super-!symplectic structure}

\section{Hamiltonian BRST theory}\label{section:classical_brst}
\subsection{Introduction}\label{section:brst_introduction}

    Consider a dynamical system $(M,\omega,H)$ together with a set of first-class constraints $\{\zeta_a\}_{a\in I}$. As was shown before, these constraints generate an algebra under the Poisson bracket. However, even more structure is present. To explore this structure, the phase space can be enlarged by introducing, for every constraint and every relation between constraints, a Grassmann-odd\footnote{In fact, one can generalize this section to phase spaces that already contain odd variables. In that case, the ghost variables should have the opposite parity of the associated constraints.} \textbf{ghost variable} $\eta^a$ and its canonical conjugate $\mathcal{P}_a$, i.e.
    \begin{gather}
        \{\mathcal{P}_a,\eta^b\} := -\delta^b_a\,.
    \end{gather}
    These come with two types of grading:
    \begin{enumerate}
        \item the \textbf{pure ghost number} generated by $\mathrm{pure\ gh}(\mathcal{P}_a) = 0$ and $\mathrm{pure\ gh}(\eta^a) = 1$, and
        \item the \textbf{antighost number} generated by $\mathrm{antigh}(\mathcal{P}_a) = 1$ and $\mathrm{antigh}(\eta^a) = 0$.
    \end{enumerate}
    The (total) \textbf{ghost number} is defined as the difference of these ghost numbers:\index{ghost}\index{antighost number}
    \begin{gather}
        \mathrm{gh}(f) := \mathrm{pure\ gh}(f) - \mathrm{antigh}(f)\,.
    \end{gather}
    Moreover, both the ghosts and ghost momenta have Grassmann parity $\varepsilon_a+1$.

    On this extended phase space, one can then construct a nilpotent function $\Omega$, the \textbf{BRST generator}, that induces a cohomology theory (\cref{homalg:homology}) through the Poisson bracket:
    \begin{gather}
        s := \{\cdot,\Omega\}\,.
    \end{gather}
    This cohomology theory characterizes the gauge structure (such as gauge-invariant functions).

    \begin{remark}[Nonminimal sectors]\index{Nakanishi--Lautrup field}\index{Faddeev--Popov!ghost}\index{ghost|see{Faddeev--Popov}}\index{minimal!sector}
        In certain situations, it is useful to extend the phase space even further by introducing additional conjugate pairs, e.g.~the Lagrange multipliers in an action principle. Such descriptions are said to belong to the nonminimal sector. An example would be the \textit{Nakanishi--Lautrup field} $B$, introduced when quantizing Yang--Mills theory (see \cref{section:gauge_auxiliary_fields}), which is conjugate to the \textit{Faddeev--Popov antighost field} $\overline{c}$, i.e.~$\{\overline{c},\Omega\}=B$. This will be covered further down in \cref{section:nonminimal_sectors}.
    \end{remark}

\subsection{Irreducible constraints}\label{section:irreducible_constraints}

    In this section, only systems with irreducible constraints are considered, i.e.~it will be assumed that no relations between the constraints exist. To formulate the BRST complex in terms of invariant geometric notions, a homological and differential-geometric approach will be adopted.

    A first step is to express the algebra of on-shell functions $C^\infty(\Sigma)$ in an invariant way. The idea is to rewrite the quotient $C^\infty(M)/\mathcal{N}$ as a homology group (which is invariant by its very nature). To this end, one passes to the Koszul complex (\cref{homalg:koszul_resolution}) associated to the first-class constraints $\zeta_a$: $C^\infty(M)\otimes\mathbb{C}[\mathcal{P}_a]$. (Independence of the constraints exactly says that they form a regular sequence and, hence, the complex gives a homological resolution.) One then finds that $H_0(\delta)\cong C^\infty(M)/\mathcal{N}\cong C^\infty(\Sigma)$, where $\delta$ is the Koszul differential.
    \newdef{Antighost number}{\index{ghost!number}
        The degree corresponding to $\delta$ is exactly the antighost number and the Koszul generators are the ghost momenta $\mathcal{P}_a$ associated to the constraints $\zeta_a$. This also implies that $\mathrm{antigh}(\delta)=-1$.
    }

    A second step is to characterize the gauge structure of the constraint surface $\Sigma$. To this end, a modified exterior derivative on the phase space is introduced. Although the gauge algebra spanned by the constraints $\zeta_a$ does not necessarily generate a closed gauge group on the full phase space, it does so when restricted to the constraint surface. The $|I|$-dimensional leaves of the foliation generated by the constraints are called the \textbf{gauge orbits} and the Hamiltonian vector fields associated to the first-class constraints are tangent to these orbits. Furthermore, by the irreducibility of the constraints, the vector fields form a frame field for the tangent bundle of the gauge orbits.
    \newdef{Longitudinal complex}{
        Vector fields that are tangent to the gauge orbits are said to be \textbf{longitudinal} or \textbf{vertical} (not to be confused with the vertical vector fields from \cref{section:connections}). A frame for these vector fields is given by the Hamiltonian vector fields
        \begin{gather}
            \label{constraint:longitudinal_basis}
            X_a:=\{\zeta_a,\cdot\}\,.
        \end{gather}
        The tangent bundle of $\Sigma$ in $M$ can be decomposed as follows:
        \begin{gather}
            TM|_\Sigma = T\mathcal{F}\oplus N\Sigma\,,
        \end{gather}
        where $T\mathcal{F}$ denotes the tangent bundle of the foliation generated by the constraints and $N\Sigma$ denotes the normal bundle to the foliation.

        This decomposition also turns the de Rham complex $\Omega(\Sigma):=\Omega(M)|_\Sigma$ into a bicomplex $\Omega^{\bullet,\bullet}(\Sigma)=\Lambda^\bullet T^*\mathcal{F}\otimes\Lambda^\bullet N^*\Sigma$. The longitudinal (or vertical) complex is given by the subcomplex $\Omega^{\bullet,0}(\Sigma)$, i.e.~the longitudinal forms $\eta^a$ are the multilinear duals of the longitudinal vector fields. The \textbf{longitudinal derivative} is the projection of the total de Rham differential on the longitudinal subcomplex, i.e.~the operator $\symbfsf{d}:\Omega^{\bullet,0}(\Sigma)\rightarrow\Omega^{\bullet+1,0}(\Sigma)$ given by
        \begin{align}
            \symbfsf{d}f &:= \dr f = \{\zeta_a,f\}\,\eta^a\,,\\
            \symbfsf{d}\eta^a &:= -\frac{1}{2}C^a_{bc}(q,p)\eta^b\wedge\eta^c\,,\label{constraint:chevalley_eilenberg}
        \end{align}
        where the $C^a_{bc}(q,p)$ are the structure functions of the constraint algebra. The Grassmann parity of the forms $\eta^a$ is taken to be $\varepsilon_a+1$. Note that the action of $\symbfsf{d}$ is exactly the action of the Chevalley--Eilenberg differential from \cref{section:lie_algebra_cohomology}. (This is not a coincidence as will be explained further on.)

        The longitudinal complex can be extended to all of $M$ by taking its elements to be the forms
        \begin{gather}
            A \equiv A_{i_1\ldots i_k}(q,p)\eta^{i_1}\wedge\cdots\wedge\eta^{i_k}\,,
        \end{gather}
        where the coefficients are equivalence classes of weakly equal functions in $C^\infty(M)$.
    }

    \newdef{Pure ghost number}{\index{ghost!number}
        The longitudinal differential forms correspond to the ghosts in \cref{section:brst_introduction} and, accordingly, their degree corresponds to the (pure) ghost number. This also implies that $\mathrm{pure\ gh}(\symbfsf{d})=1$.
    }

    Note that the number of Koszul generators is the same as the number of ghost fields, since both are induced by the constraints $\zeta_a$ (as foretold in \cref{section:brst_introduction}). To extend the Poisson bracket to the \textbf{extended phase space} containing phase space functions, ghost fields and ghost momenta, the following convention is introduced:
    \begin{gather}
        \{\mathcal{P}_a,\eta^b\} = -(-1)^{(\varepsilon_a+1)(\varepsilon_b+1)}\{\eta^b,\mathcal{P}_a\} := -\delta_a^b\,.
    \end{gather}

    \newdef{Ghost number}{\index{ghost!number}
        The total ghost number of an element in the coordinate superalgebra $C^\infty(M)\otimes\mathbb{C}[\eta^a]\otimes\mathbb{C}[\mathcal{P}_a]$ on the extended phase space is defined as follows:
        \begin{gather}
            \mathrm{gh}(A) := \mathrm{pure\ gh}(A) - \mathrm{antigh}(A)\,.
        \end{gather}
        It satisfies
        \begin{gather}
            \mathrm{gh}(AB) = \mathrm{gh}(A)+\mathrm{gh}(B)\,.
        \end{gather}
    }

    \begin{property}
        The ghost number is equal to the eigenvalue of the operator
        \begin{gather}
            \label{constraint:ghost_number}
            \mathcal{G}:=i\eta^a\mathcal{P}_a\,.
        \end{gather}
    \end{property}

    When passing to longitudinal forms on all of $M$, the operator $\symbfsf{d}$ fails to be a differential, i.e.~$\symbfsf{d}^2\neq0$ on $M$. It is only weakly zero outside $\Sigma$. Furthermore, when extending the longitudinal derivative $\symbfsf{d}$ to the extended phase space, one has the freedom to choose the action on the ghost momenta under the constraint that $\mathrm{gh}(\symbfsf{d})=1$ and $\mathrm{antigh}(\symbfsf{d})=0$.\footnote{Because $\delta$ is a boundary operator, i.e.~it decreases the degree, there is less freedom in defining $\delta\eta^a$. Only $\delta\eta^a=0$ is allowed.} By making the choice
    \begin{gather}
        \symbfsf{d}\mathcal{P}_a := (-1)^{\varepsilon_a}\eta^c C^b_{ca}\mathcal{P}_b\,,
    \end{gather}
    the Koszul differential and longitudinal derivative satisfy $[\delta,\symbfsf{d}]=0$. This also turns $\symbfsf{d}$ into a differential modulo $\delta$ (\cref{homalg:differential_modulus_differential}). The homology of $\delta$ can be generalized to the full extended phase space by tensoring the Koszul resolution of $C^\infty(\Sigma)$ by $\mathbb{C}[\eta^a]$, since the latter is a free and, in particular, projective module. The cohomology of $\symbfsf{d}$ modulo $\delta$ on $M$ can be shown to coincide with the cohomology of $\symbfsf{d}$ on $\Sigma$. This is exactly the \textbf{BRST cohomology} from the introduction.

    Homological perturbation theory (\cref{homalg:homological_perturbation}) now also says that there exists a true differential $s$ on the extended phase space generating BRST cohomology.
    \newdef{BRST operator}{\index{Becchi--Rouet--Stora--Tyutin}
        To any dynamical system governed by first-class constraints $\{\zeta_a\}_{a\in I}$, one can associate a BRST differential\footnote{BRST stands for \textit{Becchi, Rouet, Stora} and \textit{Tyutin}.}
        \begin{align}
            s &= \delta + \symbfsf{d} + \cdots\,,\nonumber\\
            \varepsilon(s) &= 1\,,\\
            \mathrm{gh}(s) &= 1\,.\nonumber
        \end{align}
        given by the Poisson bracket with a \textbf{BRST function} $\Omega$
        \begin{gather}
            sA = \{A,\Omega\}
        \end{gather}
        that satisfies the following conditions:
        \begin{enumerate}
            \item It is of ghost degree 1:
                \begin{gather}
                    \mathrm{gh}(\Omega) = 1\,.
                \end{gather}
            \item It is nilpotent with respect to the Poisson bracket:
                \begin{gather}
                    \label{constraint:BRST_nilpotency}
                    \{\Omega,\Omega\} = 0\,.
                \end{gather}
            \item It is real/Hermitian:
                \begin{gather}
                    \Omega^* = \Omega\,.
                \end{gather}
            \item It is proportional to the constraints to lowest order:
                \begin{gather}
                    \Omega = \eta^a\zeta_a + \text{terms of nonzero pure ghost number}\footnotemark\,.
                \end{gather}
        \end{enumerate}
        \footnotetext{To ensure that the ghost number of $\Omega$ is 1, this means that the extra terms are at least quadratic in the ghost fields.}
    }

    \begin{remark}[Off-shell closure]\index{on-shell}\index{Batalin--Fradkin--Vilkovisky|see{BFV theory}}\index{BFV theory}
        It should be noted that the nilpotency of the BRST operator not only holds on-shell, but everywhere on $M$. In this sense, the ghost momenta appearing in its definition are the fields necessary to close the algebra outside the constraint surface. In fact, it is important to work in the Hamiltonian formalism if one wants to achieve this off-shell nilpotency. It has been shown that in the Lagrangian formalism, this property cannot hold for gauge transformations that only close on-shell. (This latter property is related to the fact that the structure coefficients are generally functions of the canonical variables. Only when they are constants does the algebra of canonical transformations generated by the constraints close off-shell.) In fact, `the BRST complex' historically only refers to the longitudinal complex. It only gives a resolution for the quotient by the constraint algebra. The full complex as considered here is called the \textbf{Batalin--Fradkin--Vilkovisky complex} (BFV theory).
    \end{remark}

    It can be shown that the BRST operator only depends on the constraint surface and not on the choice of a local description.
    \begin{property}[Uniqueness]
        Any two BRST generators $\Omega,\Omega'$ for the same constraint surface are related by a canonical transformation in the extended phase space.
    \end{property}

    \begin{example}[Abelian constraint algebra]
        When the constraints form an Abelian algebra, i.e.~$\{\zeta_a,\zeta_b\}=0$, the terms involving higher ghost momenta vanish:
        \begin{gather}
            \Omega = \eta^a\zeta_a\,.
        \end{gather}
    \end{example}
    \begin{example}[Closed constraint algebra]
        When the constraints form a closed algebra, i.e.~$\{\zeta_a,\zeta_b\}=C^c_{ab}\zeta_c$ with $C^c_{ab}$ constant, the BRST operator has a slightly more complex expression since the zeroth order term in the BRST expansion is not nilpotent on its own. However, since the structure functions $C^c_{ab}$ are constants, all higher order terms still vanish:
        \begin{gather}
            \Omega = \eta^a\zeta_a -\frac{(-1)^{\varepsilon_b}}{2}\eta^b\eta^c C^a_{bc}\mathcal{P}_a\,.
        \end{gather}
        More generally, the BRST generator has this form in lowest degree:
        \begin{gather}
            \label{constraint:BRST_expansion}
            \Omega = \eta^a\zeta_a - \frac{(-1)^{\varepsilon_b}}{2}\eta^b\eta^cC^a_{cb}\mathcal{P}_a + \text{higher-order terms}\,.
        \end{gather}
    \end{example}

    \begin{property}[BRST cohomology]
        For all functions $f$ on the extended phase space, one has the following equality:
        \begin{gather}
            s^2f = \{\{f,\Omega\},\Omega\} = 0\,,
        \end{gather}
        i.e.~$s$ is a proper differential. In view of this structure, one says that a function is BRST closed function if
        \begin{gather}
            \{f,\Omega\} = 0
        \end{gather}
        and that it is BRST exactwhen it can be written as
        \begin{gather}
            f = \{g,\Omega\}
        \end{gather}
        for some function $g$. It is clear that the resulting BRST cohomology theory is gauge-invariant, since $\Omega$ is gauge-invariant.
    \end{property}

\subsection{Reducible constraints}

    In this section, the irreducibility requirement for the first-class constraints is relaxed. To recover the BRST complex as a homological object, one has to modify the construction from the previous section. First of all, the Koszul complex is not a resolution of $C^\infty(\Sigma)$ anymore. Because higher-order relations exist among the constraints, the higher-degree homology groups do not vanish. Mathematically, the issue is that the constraints do not form a regular sequence anymore. However, they still generate a module and so a Koszul--Tate resolution exists (\cref{homalg:koszul_tate_resolution}). Instead of only introducing ghost momenta corresponding to constraints, one also has to introduce ghost-of-ghosts.

    A second problem occurs when trying to define the longitudinal complex and trying to combine it with the Koszul--Tate complex. The number of ghost momenta is now greater than the number of (longitudinal) ghost fields and, furthermore, the longitudinal algebra is not a tensor product $C^\infty(\Sigma)\otimes\mathbb{C}[\eta^a]$ due to the existence of relations among the constraints. The solution here is again to pass to a larger structure that has the correct `homotopical' structure. In this case, this will be a Sullivan model (\cref{topology:sullivan_algebra}), i.e.~the Chevalley--Eilenberg algebra associated to an $L_\infty$-algebroid (\cref{hdg:l_infty_algebroid}).

    \begin{formula}[Ghost number]
        Due to the introduction of ghost-of-ghosts, \cref{constraint:ghost_number} has to be modified. Let $\eta^{a_1}$ denote the ghost fields, i.e.~fields of (pure) ghost number 1, $\eta^{a_2}$ the ghost-of-ghosts, i.e.~fields of (pure) ghost number 2, etc. The total ghost number operator is then given by
        \begin{gather}
            \mathcal{G} := i\sum_{n=1}^{+\infty}(n+1)\eta^{a_n}\mathcal{P}_{a_n} + \text{constants due to operator ordering}\,,
        \end{gather}
        where $\mathcal{P}_{a_n}$ are the ghost-of-ghost momenta of antighost number $i$.
    \end{formula}

    \begin{construct}[Longitudinal complex]
        Whereas there were as many Koszul generators as longitudinal basis forms in the case of irreducible algebras, this is not the case anymore for reducible algebras. Moreover, for reducible theories, the extended phase space can only locally be given a tensor product space. Sullivan models will allow to consider an equivalent dgca, with a global tensor product structure and as many longitudinal generators as Koszul--Tate generators, such that the associated cohomology coincides with the longitudinal cohomology.

        In reducible theories, the constraints are related as follows
        \begin{gather}
            Z^{a_0}_{a_1}\zeta_{a_0}=0
        \end{gather}
        for some functions $Z^{a_0}_{a_1}:M\rightarrow\mathbb{R}$. The coefficients in the overcomplete `basis' $X_a$ of longitudinal vector fields~\eqref{constraint:longitudinal_basis} then satisfy the following relations:
        \begin{gather}
            (-1)^{\varepsilon_{a_0}(\varepsilon_{a_1}+1)}Z^{a_0}_{a_1}X_{a_0}=0\,.
        \end{gather}
        Of course, higher-order reducibility then corresponds to further relations $Z^{a_1}_{a_2}$ among these relations. For every reducibility identity of order $k\in\mathbb{N}$, one then adds a ghost(-of-ghost) generator $\eta^{a_k}$ satisfying
        \begin{gather}
            \Delta\eta^{a_k} := (-1)^{\varepsilon_{a_k}+k+1}\eta^{a_k+1}Z^{a_k}_{a_k+1}\,.
        \end{gather}
        and
        \begin{gather}
            \mathrm{pure\ gh}(\eta^{a_k}) = k+1 \qquad\text{and}\qquad \varepsilon(\eta^{a_k}) = \varepsilon_{a_k}+k+1\,.
        \end{gather}
        Note that the $\eta^{a_k}$ are not given by differential forms as they were in the irreducible case!
    \end{construct}

    \Cref{constraint:BRST_expansion} can be generalized to the reducible setting by including the reducibiliy relations.
    \begin{formula}
        The BRST generator can be expanded as follows:
        \begin{gather}
            \Omega = \eta^{a_0}\zeta_{a_0} + \eta^{a_k}Z^{a_k-1}_{a_k}\mathcal{P}_{a_k-1} + \text{higher-order terms}\,.
        \end{gather}
    \end{formula}

    \begin{remark}[Chevalley--Eilenberg complex]\label{constraint:remark_chevalley_eilenberg}
        As has been noted before, there are some relations between BRST complexes and Chevalley--Eilenberg algebras. In fact, these relations are no mere coincidences. The constraint algebra
        \begin{gather}
            \{\zeta_a,\zeta_b\} = C^c_{ab}\zeta_c
        \end{gather}
        defines a Lie algebroid in the case of irreducible constraints and a (potentially truncated) $L_\infty$-algebroid in the case of reducible constraints. Similar to \cref{lie:zeroth_cohomology} in Lie algebra cohomology, one can characterize invariants of Lie algebroids in terms of their Chevalley--Eilenberg cohomology. From this point of view, gauge-invariant functions do not just resemble elements of the zeroth cohomology group of a Chevalley--Eilenberg differential, they are exactly that.
    \end{remark}

\subsection{Observables}

    By construction of the BRST complex, the $s$-cohomology coincides with the cohomology of the longitudinal differential. In negative ghost degree, it can be shown that BRST cohomology vanishes. In degree 0, one finds the physical observables.
    \begin{property}[Gauge-invariant functions]\label{constraint:brst_0}
        $H^0(s)$ is isomorphic to the set of equivalence classes of weakly equal, gauge-invariant functions.

        Given a BRST-closed function $f$ of ghost degree 0, the associated \textbf{classical observable} is obtained as the term of antighost number 0 in its BRST expansion. Conversely, any BRST-closed function of ghost number 0 is called a \textbf{BRST-invariant extension} of its term of antighost number 0.
    \end{property}
    \sremark{The interpretation of higher cohomology groups will be adressed in \cref{chapter:quantization}.}

    \begin{property}[Poisson algebra]\label{constraint:poisson_algebra}
        The Poisson bracket descends to $H^0(s)$ and defines a (graded) Poisson algebra structure. Furthermore, if $f$ and $g$ are BRST-invariant extensions of $f_0$ and $g_0$ respectively, the functions $fg$ and $\{f,g\}$ are BRST-invariant extensions of $f_0g_0$ and $\{f_0,g_0\}$, respectively.
    \end{property}

    \begin{example}[Extension of constraints]
        Consider a first-class constraint $\zeta_a$. A BRST-invariant extension is given by the Poisson bracket
        \begin{gather}
            G_a := \{-\mathcal{P}_a,\Omega\}\,.
        \end{gather}
        This immediately shows that the extension $G_a$ corresponds to the observable 0, since it is $s$-exact and, hence, vanishes in cohomology. If the constraints form a closed algebra, so do their extensions (due to the property above). However, in general, the BRST extensions do not obey any kind of algebra-like condition.
    \end{example}

    \begin{remark}[Higher cohomology]
        The higher cohomology groups $H^{\bullet\geq1}(s)$ do not have a straightforward physical explanation. $H^1(s)$ and $H^2(s)$ are related to symmetry breaking and anomalies.
    \end{remark}

\subsection{Actions and gauge fixing}

    Consider a classical Hamiltonian $H_0$ and its BRST extension $H$. Because of \cref{constraint:poisson_algebra}, Hamiltonian dynamics can be defined on the entire extended phase space:
    \begin{gather}
        \deriv{F}{t} := \{F,H\}\,.
    \end{gather}
    For BRST-invariant functions, this is equivalent to the ordinary equations of motion for their associated classical observables. Because these equations do not have any gauge redundancies, they are said to be \textbf{gauge fixed}.

    \newdef{Gauge fixing}{\index{gauge!fixing}
        One can change the Hamiltonian $H$ by a BRST-exact term $\{K,\Omega\}$ without changing the cohomology, i.e.~without changing the dynamics of BRST-invariant functions. However, the dynamics of noninvariant functions is modified. For this reason, the function $K$ is called the \textbf{gauge-fixing fermion} (since it has to be odd for $\{K,\Omega\}$ to be even).

        If the gauge-fixing fermion has the form
        \begin{gather}
            K = K_0 + k^{a_1}(z,\eta)\mathcal{P}_{a_1} + \text{higher antighost terms}\,,
        \end{gather}
        the transformation is given by $H\rightarrow H+k^{a_1}(z,\eta)\mathcal{P}_{a_1}$. Note that the gauge-fixed Hamiltonian might involve multighost terms.
    }

    Given a BRST invariant Hamiltonian and gauge-fixing fermion $K$, a BRST invariant action is given by
    \begin{gather}
        S_K[z,\eta,\mathcal{P}] := \Int_{t_1}^{t_2}\left(\dot{z}^\mu\theta_\mu(z)+\sum_{k=0}^{+\infty}\dot{\eta}^{a_k}\mathcal{P}_{a_k}-H-\{K,\Omega\}\right)\,dt\,,
    \end{gather}
    where $\theta_\mu$ are the coefficients of a symplectic potential (as in \cref{constraint:reduced_action}).

\subsection{Nonminimal sectors}\label{section:nonminimal_sectors}

    In certain situations, it is useful (or even necessary) to enlarge the extended phase space even more, without modifying $H^0(s)$. An example would be generally covariant gauges for relativistic (field) theories.
    
    In their most basic form, nonminimal sectors arise from including cohomologically trivial canonical pairs $(\alpha,\beta,\pi_\alpha,\pi_\beta)$:
    \begin{gather}
        \begin{aligned}
            s\alpha &= \beta\\
            s\pi_\beta &= \pi_\alpha\,.
        \end{aligned}
    \end{gather}
    The BRST generator in this nonminimal sector is then given by $\pi_\alpha\beta$.

    \begin{remark}
        The distinction between minimal and nonminimal is mostly artificial. For example, the trivial condition $\zeta\equiv0$, characterizing an unconstrained system, can be implemented by either $\Omega=0$ or $\Omega=\mathcal{P}\pi$. When $\zeta$ is not included as an explicit constraint, the latter choice is nonminimal, whereas it becomes minimal when $\zeta$ is included.
    \end{remark}

    The nonminimal sectors can also be chosen in such a way that they contain the `Lagrange multipliers' of the constraints in the extended Hamiltonian~\eqref{constraint:extended_hamiltonian}. The function $v^a$ has as canonical momentum a weakly vanishing function $b_a$. Including these pairs does, accordingly, not change the dynamics of the theory. To preserve BRST invariance, however, one should also include appropriate ghost and antighost fields (and momenta) $(\rho^a,\overline{C}_a)$.

\subsection{Faddeev--Popov action}\index{Faddeev--Popov!action}

    Consider a closed constraint algebra:
    \begin{gather}
        \{\zeta_a,\zeta_b\}=C^c_{ab}\zeta_c\,.
    \end{gather}
    In this case, the minimal BRST generator was given by
    \begin{gather}
        \Omega_0 = \eta^a\zeta_a - \frac{1}{2}\eta^a\eta^b C^c_{ba}\mathcal{P}_c\,.
    \end{gather}
    For scalar functions, the BRST differential acts through gauge variations:
    \begin{gather}
        \{f,\Omega_0\} = (-1)^{\varepsilon_a}\delta_\eta f\,.
    \end{gather}
    Note that, whereas the gauge variations usually have an infinitesimal parameter, this is here replaced by the ghosts $\eta$. A common (nonminimal) gauge-fixing fermion for Lagrangian field theory is
    \begin{gather}
        K = i\overline{C}_a\chi^a - \mathcal{P}_av^a + \frac{i}{2}\overline{C}_ag^{ab}b_b\,,
    \end{gather}
    where $g^{ab}$ is a `metric' (it is a proper metric for bosonic theories) and the $\chi^a$'s are real functions. This way, the gauge-fixed Hamiltonian contains a quadratic (kinetic) term in the momenta $b_a$ which allows to recover a Lagrangian formulation.\footnote{If the last term were note present, the gauge-fixed Hamiltonian would be linear in $b_a$ and, hence, the momenta can not be eliminated.} Such gauges are called \textbf{propagating gauges}, i.e.~the ghosts admit a well-defined propagator.
    \newdef{Propagating gauge}{\index{gauge!propagating}
        Propagating gauges, which are quadratic in the conjugate momenta, allow to eliminate these momenta through their own equations of motion. The resulting gauge-fixed action only has BRST invariance on shell. This type of BRST transformations are called \textbf{Lagrangian BRST transformations.} The Lagrangian BRST invariance has an associated Noether current: the BRST generator $\Omega$.
    }

    The equations of motion derived from the nonminimally extended action for the antighosts $\mathcal{P}_a$ and $\rho^a$ are:
    \begin{gather}
        \dot{\overline{C}}_a -i\mathcal{P}_ai^{\varepsilon_a}=0
    \end{gather}
    and
    \begin{gather}
        \dot{\eta}^a + (-1)^{\varepsilon_c}v^c\eta^bC^a_{bc}+i\rho^a(-i)^{\varepsilon_a}=0\,.
    \end{gather}
    These can be substituted back into the action to obtain the following gauge-fixed expression:
    \begin{gather}
        \label{constraint:FP_action}
        \begin{aligned}
            S'_K[z,v,b,\eta,\rho] &= S_E + S_{\text{gauge breaking}} + S_{\text{ghost}}\\
            &= \Int_{t_1}^{t_2}\left[\dot{z}^\mu\theta_\mu(z) - H_0 - v^a\zeta_a\right]\,dt\\
            &\quad+ \Int_{t_1}^{t_2}\left[\dot{v}^a+i^{\varepsilon_a}\chi^a+\frac{(-i)^{\varepsilon_a}}{2}b^a\right]b_a\,dt\\
            &\quad -i^{\varepsilon_a+1}\Int_{t_1}^{t_2}\overline{C}_a\delta_\eta\left[\dot{v}^a+i^{\varepsilon_a}\chi^a\right]\,dt\,.
        \end{aligned}
    \end{gather}
    This action, in turn, gives the following equations of motion for the momenta $b_a$ conjugate to the Lagrange multipliers $v^a$:
    \begin{gather}
        \dot{v}^a+i^{\varepsilon_a}\chi^a+(-i)^{\varepsilon_a}b^a = 0\,.
    \end{gather}
    Substituting this again, gives the following gauge-fixed ghost action:
    \begin{gather}
        S''_{\text{ghost}} = -\frac{i^{\varepsilon_a}}{2}\Int_{t_1}^{t_2}\left[\left(\dot{v}^a + i^{\varepsilon_a}\chi^a\right)\left(\dot{v}_a + i^{\varepsilon_a}\chi_a\right)\right]\,dt.
    \end{gather}
    The above expressions contain three important ingredients:
    \begin{itemize}
        \item the original gauge-invariant action,
        \item a gauge-symmetry breaking term that determines the Lagrange multipliers and, accordingly, freezes the gauge invariance of the action, and
        \item a Faddeev--Popov term that is quadratic in the ghosts.
    \end{itemize}
    The gauge symmetry-breaking term leads to a derivative gauge:
    \begin{gather}
        \label{constraint:derivative_gauge}
        \dot{v}^a + i^{\varepsilon_a}\chi_a = 0\,.
    \end{gather}

    \begin{property}[Derivative gauges]\index{gauge!derivative}
        Derivative gauge can be taken into account in the BRST-BV complex by including the Lagrange multipliers and their momenta as canonical coordinates.
    \end{property}
    Comparing \cref{constraint:FP_action} and \cref{constraint:derivative_gauge}, one can see that the ghost action contains the `ghost variation' of the gauge-fixing condition.
    \begin{formula}[Ghost action]
        When the constraint algebra is closed, the Faddeev--Popov ghost Lagangian is of the form
        \begin{gather}
            L_{\text{ghost}} = \overline{C}_a\delta_\eta F^a\,,
        \end{gather}
        where $F^a$ are the gauge-fixing conditions. For soft and open algebras, nonquadratic ghost vertices ruin this form. In fact, this is also the case for closed algebras for nonlinear gauges, i.e.~when the gauge-fixing fermion is nonlinear in $\mathcal{P}$ and $\overline{C}$.
    \end{formula}

\section{Lagrangian BRST theory}\label{section:BV_formalism}

    When considering (classical) constrained Hamiltonian systems (\cref{section:classical_brst}) and their \textit{quantization} (see \cref{section:quantum_constrained}), the phase space is extended by both ghosts and antighosts. The former corresponded to differential forms along the gauge orbits (giving a resolution of the quotient by the gauge group) and the antighosts correspond to the Koszul--Tate generators characterizing the zero locus of the field equations (i.e.~taking the intersection with the equations of motion). Now, what about Lagrangian field theories, where the solutions are not functions in $C^\infty(\mathbb{R},M)$, but sections of a vector bundle $E\rightarrow M$ that solve the field equations $\frac{\delta S}{\delta\phi}=0$? As shown below, such theories admit symmetries generated by the equations of motion themselves: the zilch symmetries. One can then play the same game as before, with the ordinary phase space replaced by the covariant phase space $\Sigma$ and the symmetry group replaced by the full gauge group (including true gauge symmetries and zilch symmetries). As a start, the formalism for finite-dimensional systems will be introduced, i.e.~the fields reduce to functions of time. Afterwards, the formalism, originally developed by \indexauthor{Peierls}, will be introduced in the absence of local gauge symmetries.

\subsection{Field theory}

    Since the classical notion of phase space as the set of $(q,p)$-points in coordinate-momentum space, at a given time $t$, is clearly not covariant (the choice of a time slice ruins any form of relativistic invariance). In the previous section, one possible solution was considered: including time as a canonical coordinate. Here, another approach is embraced which is more convenient in field theory.
    \newdef{Covariant phase space}{\index{phase space}
        Let $S$ be a local action functional. The covariant phase space $\Sigma$ associated to $S$ is the set of solutions of the equations of motion
        \begin{gather}
            \frac{\delta S}{\delta\phi^I}=0\,,
        \end{gather}
        i.e.~it is exactly the constraint surface.

        As before, the physical observables are defined as the smooth functions on this new phase space $\Sigma$. These can be described as before. Let $\mathcal{E}\supseteq\Sigma$ be the set of all field histories, e.g.~the space of sections $\Gamma(E)$ of a vector bundle. The ring of physical observables $C^\infty(\Sigma)$ is obtained as the quotient of $C^\infty(\mathcal{E})$ by the ideal of functions vanishing on-shell and by the gauge transformations.
    }

\subsection{Gauge algebra}

    Consider a local action $S$ on a space $M$. A \textbf{gauge transformation} of $S$ is a general field transformation, depending on $M$, that leaves the action invariant, i.e.~it is a vertical automorphism (\cref{bundle:vertical_automorphism}) of the field bundle. The most general form of such transformations is
    \begin{gather}
        \label{constraint:gauge_transformation}
        \delta_\varepsilon\phi^I = \overline{R}^I_{(0),\alpha}\varepsilon^\alpha + \overline{R}^{I,\mu}_{(1),\alpha}\partial_\mu\varepsilon^\alpha + \cdots + \overline{R}^{I,\mu_1\ldots\mu_s}_{(s),\alpha}\partial_{\mu_1\ldots\mu_s}\varepsilon^\alpha\equiv R^I_\alpha\varepsilon^\alpha\,,
    \end{gather}
    where the coefficients $\overline{R}_{(i)}$ are general functions of the coordinates and, in the last step, a new shorthand was introduced where the summation over $\alpha$ also includes an integral over $x$ (the \textbf{DeWitt convention}):\index{DeWitt!convention}
    \begin{align}
        R^I_\alpha\varepsilon^\alpha &:= \Int R^I_a(x,x')\varepsilon^a(x')\,dx'\label{constraint:dewitt_convention}\\
        &\phantom{:}= \Int\sum_{j,a_j}\left(\overline{R}^I_{(0),a_j}(x)\delta(x-x')+\overline{R}^I_{(1),a_j}(x)\delta'(x-x')+\overline{R}^I_{(2),a_j}(x)\delta^{(2)}(x-x')+\cdots\right)\varepsilon^{a_j}(x')\,dx'\,.\nonumber
    \end{align}
    Invariance of the action implies that
    \begin{gather}
        \delta_\varepsilon S = \frac{\delta S}{\delta\phi^I}\delta_\varepsilon\phi^I = \frac{\delta S}{\delta\phi^I}R^I_\alpha\varepsilon^\alpha = 0\,.
    \end{gather}
    Because this should hold for every value of the transformation parameters $\varepsilon^\alpha$, one immediately obtains the variational Noether identities.
    \begin{property}[Noether identities]\label{field:noether_identity}
        If a local action is invariant under the transformation~\eqref{constraint:gauge_transformation}, then
        \begin{gather}
            \frac{\delta S}{\delta\phi^I}R^I_\alpha = 0
        \end{gather}
        for all `indices' $\alpha$. In contrast to Noether's theorem~\ref{classic:noether_cyclic}, these identities do not imply conserved quantities. Instead, they imply that the equations of motion are not independent (cf.~Noether's second theorem~\ref{var:noether}).
    \end{property}

    The structure of the infinitesimal gauge transformations is easily seen to be that of a (real) Lie algebra $\overline{\mathcal{G}}$, whilst that of finite (exponentiated) transformations is a Lie group. However, the gauge algebra is very large (in fact, it is infinite-dimensional) and contains a lot of physically irrelevant information. The simplest example is that of the \textbf{zilch symmetries} as referred to by~\citet{van_proeyen_supergravity_2012}.
    \newdef{Zilch symmetry}{\index{zilch symmetry}
        All transformations of the form
        \begin{gather}
            \delta_\varepsilon\phi^I = \varepsilon^{IJ}\frac{\delta S}{\delta\phi^J}\,,
        \end{gather}
        where $\varepsilon^{IJ}$ is antisymmetric, are physically irrelevant since they vanish on-shell by the equations of motion. The trivial gauge transformations form an ideal $\mathcal{N}$ of the gauge algebra and the physically relevant algebra is the quotient $\mathcal{G} := \overline{\mathcal{G}}/\mathcal{N}$. However, for some reasons it might be convenient to retain the full gauge algebra.
    }

    Another problem with the gauge algebra is that independent transformations might lead to dependent Noether identities, wich implies that there is still some redundancy. To analyze this issue, one first finds a minimal set of generating transformations.
    \newdef{Generating set}{\index{open!algebra}
        A generating set of the gauge algebra (or \textbf{complete set of transformations}) is a set of transformations $\delta_\varepsilon\phi^I = R^I_\alpha\varepsilon^\alpha$ such that every gauge transformation can be written as follows:
        \begin{gather}
            \delta\phi^I = R^I_\alpha\mu^\alpha + M^{IJ}\frac{\delta S}{\delta\phi^J}\,,
        \end{gather}
        where $\mu^\alpha$ and $M^{IJ}=-M^{JI}$ may depend on the fields. Because the coefficients might be functions of the fields and their derivatives, the generating set is, in general, not a basis for the gauge algebra. However, due to the Lie algebra structure, there must exist structure functions $C^\gamma_{\alpha\beta}$ and $M^{IJ}_{\alpha\beta}$ such that
        \begin{gather}
            \label{field:lie_algebroid}
            R^J_\alpha\frac{\delta R^I_\beta}{\delta\phi^J} - R^I_\beta\frac{\delta R^J_\alpha}{\delta\phi^I} = C^\gamma_{\alpha\beta}R^I_\gamma + M^{IJ}_{\alpha\beta}\frac{\delta S}{\delta\phi^J}\,,
        \end{gather}
        where $M^{IJ}_{\alpha\beta}=-M^{JI}_{\alpha\beta}$.
    }
    In \cref{constraint:first_and_second_class}, the structure of the contraint algebra was considered. Comparing \cref{constraint:first_class_bracket} to \cref{field:lie_algebroid}, it is clear that the constraint algebra and the generating algebra of the gauge transformations play a similar role:
    \begin{itemize}\index{soft!algebra}\index{open!algebra}
        \item If all $M$ are zero, the algebra is said to be \textbf{closed}. (Even though the generating set itself might not be closed as a Lie algebra because the $C$'s generally are functions of the fields. The algebra is said to be \textbf{soft} in this case.)
        \item Otherwise, it is said to be \textbf{open}.
    \end{itemize}
    Similarly, A generating set is said to be \textbf{irreducible} if there exist no nontrivial combinations of elements:
    \begin{gather}
        R^I_\alpha\varepsilon^\alpha = M^{IJ}\frac{\delta S}{\delta\phi^J}\implies\varepsilon^\alpha = N^{\alpha I}\frac{\delta S}{\delta\phi^I}\,.
    \end{gather}

    The following remark is the Lagrangian counterpart of \cref{constraint:remark_chevalley_eilenberg} in the Hamiltonian treatment of constrained systems.
    \begin{remark}[Lie algebroids]
        If one restricts to closed gauge algebras, i.e.~ignores zilch symmetries, \cref{field:lie_algebroid} is exactly the closure condition for a Lie algebroid (\cref{hdg:lie_algebroid}). Higher relations between the generators, i.e.~a reducible theory, turn the gauge algebra into a Lie $n$-algebroid or even a $L_\infty$-algebroid.
    \end{remark}

\subsection{Antifields}

    In the Hamiltonian BRST formalism of \cref{section:classical_brst}, two (co)homological constructions were combined. On the one hand, the quotient of phase space by the constraint algebra, which was modeled by a Chevalley--Eilenberg complex, and, on the other hand, the critical locus of the constraints, modeled by a Koszul--Tate resolution. This led to the extended phase space containing three types of objects:
    \begin{enumerate}
        \item Canonical coordinates: $(q,p)$,
        \item Ghosts: $\eta$ (the Chevalley--Eilenberg generators), and
        \item Antighosts: $\mathcal{P}$ (the Koszul--Tate generators).
    \end{enumerate}
    In the Lagrangian setting, a very similar structure is obtained. However, now, the critical locus is not just the one induced by the Noether identities. One also has to model the equations of motion themselves. This implies that one also has to add antifields, i.e.~Koszul--Tate generators corresponding to the field equation $\frac{\delta S}{\delta\phi^I}=0$. This gives the \textbf{BV-BRST complex}.\footnote{BV stands for Batalin--Vilkovisky as it did in the case of BFV theory.}

    \newdef{Antifield}{\index{anti-!field}
        For every gauge symmetry $\widehat{G}_a$ in a generating set of the (proper) gauge group, one also introduces a set of antifields $\mathcal{P}_I$. These carry the following cohomological grading:
        \begin{gather}
            \begin{aligned}
                \varepsilon(\mathcal{P}_I) &= \varepsilon_I + 1\,,\\
                \mathrm{antigh}(\mathcal{P}_I) &= \mathrm{antigh}(\phi^I) + 1 = 1\,.
            \end{aligned}
        \end{gather}
        The antifields $\mathcal{P}_I$ are the Koszul--Tate generators associated to the zilch symmetries:
        \begin{gather}
            \delta\mathcal{P}_I := -\frac{\delta S}{\delta\phi^I}\,.
        \end{gather}
        The ghost antifields (i.e.~the antighosts from \cref{section:classical_brst}) arise as the Koszul--Tate generators induced by the Noether identities (\cref{field:noether_identity}), where the higher antifields correspond to reducibility relations among the symmetries, since the Noether identities $\delta(R^I_a\mathcal{P}_I)=0$ induce elements of $H^1(\delta)$.

        If one extends the above grading properties to the antighost fields, one obtains
        \begin{gather}
            \mathrm{antigh}(\mathcal{P}_a) = 2\,,
        \end{gather}
        so the antifields are shifted in degree by 1 when compared to the Hamiltonian setting. The Grassmann parity is also shifted compared to the Hamiltonian case:
        \begin{gather}
            \varepsilon(\mathcal{P}_I) = \varepsilon_I + 1\qquad\qquad\varepsilon(\mathcal{P}_a) = \varepsilon_a+1\qquad\qquad\cdots
        \end{gather}
    }
    \begin{notation}[Antifields]
        A common alternative notation for antifields is $\phi_I^*$.
    \end{notation}

    \begin{remark}[Interpretation]
        An interpretation for this parity shift and the extra antifields is that, in the Lagrangian framework, the field equations can be seen as the fundamental constraints and the Noether identities as reducibility relations. Hence, even without true gauge symmetry, the Lagrangian Koszul--Tate resolution is nontrivial even though the Hamiltonian one was trivial without constraints.
    \end{remark}

    In the Hamiltonian BRST complex, the natural Poisson bracket on phase space could be extended in a convenient way. The ghosts and antighosts paired up to form canonical pairs. However, in the Lagrangian setting, there is no natural Poisson structure. Luckily, there exists a modified bracket structure and, with respect to this new bracket, the BRST generator will still be the generator of canonical transformations.
    \newdef{Antibracket}{\label{field:antibracket}
        Instead of pairing ghosts and antighosts, there is a new natural pairing due to the apparent symmetry between fields and antifields.

        \begin{table}[h!]
            \centering
            \begin{tabular}{c|c}
                Fields&Antifields\\
                \hline
                $\phi^I$&$\mathcal{P}_I$\\
                $\eta^{a_0}$&$\mathcal{P}_{a_0}$\\
                $\eta^{a_1}$&$\mathcal{P}_{a_1}$\\
                $\vdots$&$\vdots$
            \end{tabular}
        \end{table}

        The antibracket is then defined to have the canonical form with respect to this ghost degree pairing. For anytwo functionals $f,g$ on $C^\infty(\Sigma)\otimes\mathbb{C}[\eta^a]\otimes\mathbb{C}[\mathcal{P}_I]\otimes\mathbb{C}[\mathcal{P}_a]$, it is defined as follows:\footnote{Where the partial derivatives should be read as functional derivatives.}
        \begin{gather}
            (f,g) := \left(\frac{\partial^Rf}{\partial\phi^I}\frac{\partial^Lg}{\partial\mathcal{P}_I}-\frac{\partial^Rf}{\partial\mathcal{P}_I}\frac{\partial^Lg}{\partial\phi^I}\right) + \left(\frac{\partial^Rf}{\partial\eta^a}\frac{\partial^Lg}{\partial\mathcal{P}_a}-\frac{\partial^Rf}{\partial\mathcal{P}_a}\frac{\partial^Lg}{\partial\eta^a}\right)\,,
        \end{gather}
        where the index $a$ denotes ghost (anti)fields of arbitrary degree.
    }
    \begin{property}[Odd symplectic structure]
        As before, if one considers the extended state space with the fields (and ghosts, ...) as `coordinates' and the antifields as `momenta', the antibracket gives the structure of an odd symplectic manifold and, in particular, that of a \textit{BV manifold} as defined below:
        \begin{gather}
            (f,g) = \frac{\partial^Rf}{\partial z^\mu}\omega^{\mu\nu}\frac{\partial^Lg}{z^\nu}\,,
        \end{gather}
        where $(z^\mu)\equiv(\phi^I,\eta^a,\mathcal{P}_I,\mathcal{P}_a)$ and
        \begin{gather}
            (\omega^{\mu\nu}) :=
            \begin{pmatrix}
                0&\delta^I_J+\delta^a_b\\
                -\delta^I_J-\delta^a_b&0
            \end{pmatrix}\,.
        \end{gather}
        The BV-antibracket, accordingly, has the following algebraic properties:
        \begin{itemize}
            \item It is BRST-odd: $\mathrm{gh}\bigl((f,g)\bigr) = \mathrm{gh}(f)+\mathrm{gh}(g)+1$.
            \item It induces the structure of a Gerstenhaber algebra (\cref{hda:gerstenhaber_algebra}), where the degree is the Grassmann parity.
        \end{itemize}
    \end{property}

    \begin{formula}[Classical master equation]\label{constraint:master_equation}
        As in the Hamiltonian framework, the BRST differential can be generated by a canonical transformation. The BRST generator $S$ satisfies the classical master equation~\eqref{hdg:classical_master_equation}, i.e.~it is nilpotent (cf.~\cref{constraint:BRST_nilpotency}):
        \begin{gather}
            (S,S)=0\,.
        \end{gather}
        This again allows to recursively solve for $S$ and obtain the following expression for the lowest-order terms:
        \begin{gather}
            S = S_0 + \mathcal{P}_IR^I_{a_0}\eta^{a_0} + \mathcal{P}_{a_0}Z^{a_0}_{a_1}\eta^{a_1} + \frac{1}{2}C^{a_0}_{b_0c_0}\eta^{b_0}\eta^{c_0} + \text{higher antighost terms}\,.
        \end{gather}
    \end{formula}
    Note that the lowest-order term is exactly the initial (gauge-invariant) action.

    \begin{remark}
        In the Hamiltonian setting, the Poisson bracket descended to BRST cohomology (\cref{constraint:poisson_algebra}) and showed that physical observables also formed a Poisson algebra. Here, the antibracket is of ghost number 1 and, hence, would move outside of $H^0(s)$. Moreover, when the gauge algebra is a proper Lie algebra, $(f,g)$ is always BRST exact, so the antibracket always gives cohomologically trivial results in this case.
    \end{remark}

    \begin{property}[Gauge invariance]
        Let there be $n\in\mathbb{N}$ fields (including ghosts, ghost-of-ghosts, etc.). The BRST generator $S$ admits exactly $n$ independent gauge symmetries, given by
        \begin{gather}
            \delta z^\mu = \omega^{\mu\nu}\frac{\partial^L\partial^R}{\partial z^\nu\partial z^\rho}\varepsilon^\rho\,.
        \end{gather}
        The $2n$ symmetries one would naively expect (due to presence of the antifields) are reduced in number to $n$ because the transformation matrix
        \begin{gather}
            R^\mu_\rho := \omega^{\mu\nu}\frac{\partial^L\partial^R}{\partial z^\nu\partial z^\rho}
        \end{gather}
        is on-shell nilpotent, i.e.~the symmetries are on-shell reducible. Note that every gauge symmetry of the BRST action $S$ can be obtained by differentiating it, with one symmetry corresponding to every field-antifield pair!
    \end{property}

    The above structure can be generalized as follows.
    \newdef{BV manifold}{\index{Batalin--Vilkovisky!manifold}\index{ghost!number}\label{field:bv_manifold}
        A \textbf{Batalin--Vilkovisky manifold} is a triple $(M,\omega,S)$ where $M$ is a graded manifold, $\omega$ is a degree-1 symplectic form and $S$ is a degree-0 function such that the classical master equation~\eqref{hdg:classical_master_equation} is satisfied:
        \begin{gather}
            (S,S)=0\,,
        \end{gather}
        where $\{\cdot,\cdot\}$ is the Poisson bracket induced by $\omega$ (\cref{hdg:poisson_manifold}).

        The most straightforward example of a BV manifold is, of course, the BV-BRST complex associated to a field theory as introduced above. For this reason, the Poisson bracket is more generally called the \textbf{antibracket}, while the grading is generally called the \textbf{ghost number} and denoted by $\mathrm{gh}$.
    }
    \begin{property}[Multivector fields]\label{constraint:antifield_vector_field}
        By \cref{hdg:dg_symplectic_manifold}, BV manifolds can be represented as spaces of multivector fields with the Poisson bracket mapping to the Schouten--Nijenhuis bracket. For the BV-BRST complex, the vector fields are given by the antifields. This correspondence can be seen by noting that, when the fields transform as
        \begin{gather}
            \phi^I\longrightarrow\widetilde{\phi}^J(\phi)\,,
        \end{gather}
        the antifields have to transform as
        \begin{gather}
            P_I\longrightarrow\widetilde{\mathcal{P}}_J(\phi,\widetilde{\phi}) = \widetilde{\mathcal{P}}_I\frac{\delta^R\phi^I}{\delta\widetilde{\phi}_J}
        \end{gather}
        to preserve the canonical structure.

    \end{property}

    \begin{example}[AKSZ model]\index{AKSZ model}
        The Alexandrov--Kontsevich--Schwarz--Zabronsky model is the (noninear) $\sigma$-model on a dg-manifold $(M,Q)$ with target space a dg-symplectic manifold $(N,\omega,X_H)$, where $X_H$ is Hamiltonian.

        For any graded manifold $\Sigma$, one can construct the source manifold by taking $M:=\Pi T\Sigma$ and $Q:=\dr$. A symplectic form on $C^\infty(M,N)$ is then given by
        \begin{gather}
            \Omega := \Int_{\Pi T\Sigma}\omega_{\mu\nu}\delta\Phi^\mu\delta\Phi^\nu\,\vol\,.
        \end{gather}
        The BV action is defined as follows:
        \begin{gather}
            S := \Int_{\Pi T\Sigma}\left(\alpha_\mu d\Phi^\mu+\Theta\right)\,\vol\,,
        \end{gather}
        where $\alpha$ is a symplectic potential for $\omega$, which necessarily exists globally by \cref{hdg:global_exactness} if $\mathrm{gh}(\omega)\neq0$.

        In general, the symplectic form on $C^\infty(M,N)$ is induced from that on $N$ by a pull-push operation. First, one pulls back this form along the evaluation map $\mathrm{ev}:C^\infty(M,N)\times M\rightarrow N$ and, then, one pushes it forward along the projection on the first factor (cf.~fibre integration (\cref{bundle:fibre_integration})).

        \todo{COMPLETE (check S for example)}
    \end{example}
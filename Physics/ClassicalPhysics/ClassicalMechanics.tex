\chapter{Classical Mechanics}\label{chapter:lagrange}

    The section about the geometric framework is mainly based on~\citet{palais_symmetries_1997,arnold_mathematical_2013}. For an introduction to differential geometry, see \cref{chapter:manifolds} and onwards.

    \minitoc

\section{Newtonian mechanics}
\subsection{Linear motion}

    \begin{axiom}[Newton's second law]\index{force}\label{classic:force}
        The force acting on a system is related to the change in momentum in the following way:
        \begin{gather}
            \vector{F}=\deriv{\vector{p}}{t}\,.
        \end{gather}
    \end{axiom}

    \newdef{Work}{\index{work}\label{classic:work}
        The work performed along a trajectory $\gamma$ is given by
        \begin{gather}
            W := \Int_\gamma\vector{F}\cdot d\vector{l}\,.
        \end{gather}
    }
    \newdef{Conservative force}{\index{conservative!force}\index{potential}
        If the work done by a force is independent of the path taken, the force is said to be \textbf{conservative}:
        \begin{gather}
            \label{classic:conservative_force_2}
            \Oint_C\vector{F}\cdot d\vector{l}=0\,.
        \end{gather}
        The Kelvin--Stokes theorem~\ref{vector:kelvin_stokes_theorem} together with \cref{vector:rotor_of_gradient} allows to rewrite the conservative force as the gradient of a scalar field $V:\mathbb{R}^3\rightarrow\mathbb{R}$:
        \begin{gather}
            \label{classic:conservative_force}
            \vector{F} = -\nabla V\,.
        \end{gather}
        Note that the \textbf{potential (energy)} $V$ is only defined up to a constant.
    }

    \newdef{Central force}{
        A force that only depends on the relative position of two objects:
        \begin{gather}
            \vector{F}_c = F\bigl(\|\vector{r}_2 - \vector{r}_1\|\bigr)\symbf{\hat{e}_r}\,.
        \end{gather}
    }

    \begin{formula}[Momentum]\index{momentum}
        Consider a mass $m$ with velocity $\vector{v}$. Its momentum is given by
        \begin{gather}
            \vector{p} = m\vector{v}\,.
        \end{gather}
        If the mass remains constant along the trajectory, Newton's second law (\cref{classic:force}) can be rewritten as follows:
        \begin{gather}
            \vector{F} = m\deriv{\vector{v}}{t} = m\vector{a}\,.
        \end{gather}
    \end{formula}

    \begin{formula}[Kinetic energy]\index{energy}\label{classic:kinetic_energy}
        For a free particle with total momentum $p$, the kinetic energy is given by the following formula:
        \begin{gather}
            E_{\text{kin}} := \frac{p^2}{2m}\,.
        \end{gather}
    \end{formula}

\subsection{Rotational motion}

    In this section, $r\in\mathbb{R}^+$ always denotes the distance from the object's center of mass to the axis around which the object rotates. More generally, the vector $\vector{r}\in\mathbb{R}^3$ will denote the position with respect to the axis of rotation.

    \newdef{Angular velocity}{\label{classic:angular_velocity}
        \begin{gather}
            \vector{\omega} := \frac{\vector{r}\times\vector{v}}{r^2}
        \end{gather}
    }
    \newdef{Angular frequency}{\label{classic:frequency}
        \begin{gather}
            \nu := \frac{\|\vector{\omega}\|}{2\pi}
        \end{gather}
    }

    \newdef{Moment of inertia}{\index{inertia}\label{classic:moment_of_inertia}
        For a (spherically) symmetric object, the moment of inertia is given by
        \begin{gather}
            I := \Int_Vr^2\rho(r)\,dV\,,
        \end{gather}
        where $\rho:\mathbb{R}^3\rightarrow\mathbb{R}^+$ denotes the mass density function. For a general body, the moment-of-inertia tensor is given by
        \begin{gather}
            \label{classic:inertia_tensor}
            \mathcal{I} := \Int_V\rho(\vector{r})\left(r^2\mathbbm{1}_{3\times3} - \vector{r}\otimes\vector{r}\right)\,dV\,.
        \end{gather}
    }

    \begin{example}[Azimuthal symmetry]
        Let $M,R$ denote the mass and the radius of the object, respectively. The (scalar) inertia of some common objects is given:
        \begin{itemize}
            \item Solid disk: $I = \frac{1}{2}MR^2$,
            \item Cylindrical shell: $I = MR^2$,
            \item Hollow sphere: $I = \frac{2}{3}MR^2$, and
            \item Solid sphere: $I = \frac{2}{5}MR^2$.
        \end{itemize}
        A proof for the solid disk and sphere are given:
        \begin{mdframed}[roundcorner=10pt, linecolor=blue, linewidth=1pt]
            \begin{proof}
                The volume of a (solid) disk is given by
                \begin{gather*}
                    V_{\text{disk}}=\pi R^2d\,,
                \end{gather*}
                where $R$ denotes the radius and $d$ denotes the thickness. The mass density is then given by
                \begin{gather*}
                    \rho=\frac{M}{\pi R^2d}\,.
                \end{gather*}
                Using cylindrical coordinates, the moment of inertia becomes
                \begin{align*}
                    I &= \frac{M}{\pi R^2d}\Int_0^{2\pi}d\varphi\Int_0^ddz\Int_0^Rr^3\,dr\\
                    &= \frac{M}{\pi R^2d}2\pi d\frac{R^4}{4}\\
                    &= \frac{1}{2}MR^2\,.
                \end{align*}
                The volume of a solid sphere is given by
                \begin{gather*}
                    V_{\text{sphere}} = \frac{4}{3}\pi R^3\,,
                \end{gather*}
                where $R$ denotes the radius. The mass density is then given by
                \begin{gather*}
                    \rho = \frac{M}{\frac{4}{3}\pi R^3}\,.
                \end{gather*}
                Spherical coordinates will be used to derive the moment of inertia, but one has to be careful. The $r$ in \cref{classic:moment_of_inertia} is the distance between a point in the body and the axis of rotation. Hence, it is not the same as the radial coordinate $r'$ in spherical coordinates, which is the distance between a point and the origin. However, the relation between these two quantities is easily found using basic geometry:
                \begin{gather*}
                    r=r'\sin\theta\,.
                \end{gather*}
                Now, one can calculate the moment of inertia as follows:
                \begin{align*}
                    I &= \frac{M}{\frac{4}{3}\pi R^3}\Int_0^{2\pi}d\varphi\Int_0^Rr'^{^4}\,dr'\Int_0^\pi\sin^3\theta\,d\theta\\
                    &= \frac{M}{\frac{4}{3}\pi R^3} 2\pi\frac{R^5}{5}\frac{4}{3}\\
                    &= \frac{2}{5}MR^2\,.
                \end{align*}
            \end{proof}
        \end{mdframed}
    \end{example}

    \newdef{Principal axes of inertia}{\index{principal!axis}
        Let $I$ be the matrix of inertia, i.e.~the matrix associated with the inertia tensor in \cref{classic:inertia_tensor}. This is a real symmetric matrix and, by \cref{linalgebra:diagonalizable_hermitian}, admits an eigendecomposition of the form
        \begin{gather}
            I = Q\Lambda Q^T\,.
        \end{gather}
        The columns of $Q$ determine the principal axes of inertia. The eigenvalues are called the \textbf{principal moments of inertia}.
    }

    \begin{theorem}[Parallel axis theorem\footnotemark]\index{Steiner}\index{parallel axis theorem}\label{classic:theorem:parallel_axis_theorem}
        \footnotetext{Also called \textbf{Steiner's theorem}.}
        Consider a rotation about an axis $\psi$ through a point $A$ and let $\psi_{\emph{CM}}$ be a parallel axis through the center of mass. The moment of inertia about $\psi$ is related to the moment of inertia about $\psi_{\text{CM}}$ in the following way:
        \begin{gather}
            I_\psi = I_{\text{CM}} + m\|\vector{r}_A - \vector{r}_{\text{CM}}\|^2\,,
        \end{gather}
        where $m$ is the mass of the rotating body.
    \end{theorem}

    \newdef{Angular momentum}{\label{classic:angular_momentum}
        \begin{gather}
            \vector{L} := \vector{r}\times\vector{p}
        \end{gather}
        Given the angular velocity vector $\vector{\omega}$, one can compute the angular momentum as follows:
        \begin{gather}
            \label{classic:angular_momentum_general}
            \vector{L} = \mathcal{I}(\vector{\omega})\,,
        \end{gather}
        where $\mathcal{I}$ is the inertia tensor. If $\vector{\omega}$ is parallel to a principal axis, the formula reduces to
        \begin{gather}
            \vector{L} = I\vector{\omega}\,,
        \end{gather}
        where $I$ is the corresponding principal moment of inertia.
    }

    \begin{formula}[Torque]\index{torque}\label{classic:torque}
        For angular momenta, there exists a formula analogous to Newton's second law:
        \begin{gather}
            \vector{\tau} :=\deriv{\vector{L}}{t}\,.
        \end{gather}
        For a constant mass, this formula can be rewritten as follows:
        \begin{gather}
            \vector{\tau} = \mathcal{I}(\vector{\alpha}) = \vector{r}\times\vector{F}\,,
        \end{gather}
        where
        \begin{gather}
            \vector{\alpha} := \deriv{\vector{\omega}}{t}
        \end{gather}
        is the \textbf{angular acceleration}.
    \end{formula}

    \begin{remark}
        From the previous definitions, it follows that both the angular momentum and torque vectors are in fact pseudovectors and, accordingly, change sign under coordinate transformations with negative determinant.
    \end{remark}

    \newformula{Rotational energy}{\index{energy}\label{classic:rotational_energy}
        \begin{gather}
            E_{\text{rot}} := \frac{1}{2}\mathcal{I}(\vector{\omega})\cdot\vector{\omega}
        \end{gather}
    }

\section{Lagrangian mechanics}
\subsection{Action}

    \newdef{Generalized coordinates}{\index{generalized!coordinate}
        The generalized coordinates are mutually independent coordinates that completely characterize the configuration of a system (relative to a reference configuration).

        When a system is characterized by $N$ parameters and $n_c$ constraints, there are $N-n_c$ generalized coordinates. Furthermore, every set of generalized coordinates describing the same system contains exactly $N-n_c$ coordinates. (In \cref{chapter:constrained_dynamics}, it is explained how the relevant degrees of freedom can be extracted.)
    }
    \newdef{Generalized velocities}{
        The generalized velocities $\dot{q}^k$ are the derivatives of the generalized coordinates $q^k$ with respect to time.
    }
    \newdef{Conjugate momentum}{\index{momentum!conjugate}\label{classic:conjugate_momentum}
        \begin{gather}
            p_k := \pderiv{L}{\dot{q}^k}
        \end{gather}
    }

    \begin{notation}\label{classic:notational_convention_1}
        Given a Lagrangian function $L:\mathbb{R}^{2n+1}\rightarrow\mathbb{R}$, depending on $n$ generalized coordinates and their associated velocities, the following shorthand notation is often used:
        \begin{gather}
            L(q,\dot{q},t)\equiv L\bigl(q_1(t),\ldots,q_n(t),\dot{q}_1(t),\ldots,\dot{q}_n(t),t\bigr)\,.
        \end{gather}
    \end{notation}
    \newdef{Action}{\index{action}\label{classic:action}
        Given a Lagrangian function $L$, the action is a functional on the space of paths in configuration space defined by integrating $L$:
        \begin{gather}
            S[q] := \Int_{t_1}^{t_2}L(q,\dot{q},t)\,dt\,.
        \end{gather}
    }

    Especially in continuum (fluid) physics, the total time derivative receives a proper name.
    \newdef{Lagrangian derivative\footnotemark}{\index{Lagrange!derivative|see{material derivative}}\index{material!derivative}\index{advective}\label{classic:lagrangian_derivative}
        \footnotetext{Also known as the \textbf{material derivative}.}
        Consider a quantity $A(q,\dot{q},t)$. The total time derivative along a path $\bigl(q(t),\dot{q}(t)\bigr)$ is given by
        \begin{gather}
            \begin{aligned}
                \Deriv{A}{t} :\equiv \left.\deriv{A}{t}\right|_{\bigl(q(t),\dot{q}(t)\bigr)} &= \lim_{\Delta t\rightarrow0}\frac{A(q+\Delta q,\dot{q}+\Delta\dot{q},t+\Delta t) - A(q,\dot{q},t)}{\Delta t}\\
                &= \pderiv{A}{t} + \dot{q}^i\pderiv{A}{q^i} + \deriv{\dot{q}^i}{t}\pderiv{A}{\dot{q}^i}\,.
            \end{aligned}
        \end{gather}
        The second term $\dot{\vector{q}}\cdot\nabla a$ in this equation is called the \textbf{advective term}. In most treatments of continuum mechanics, the last term is not included, since one usually follows particles through space and not phase space.
    }
    \begin{remark}[Convection]\index{convective}
        When $A(q,\dot{q},t)$ is a tensor field, the gradient $\nabla$ has to be replaced by the covariant derivative. The advective term is then sometimes called the \textbf{convective} term.
    \end{remark}

    \begin{example}
        For $A(q,\dot{q},t)=\vector{q}(t)$, one obtains
        \begin{gather}
            \Deriv{\vector{q}}{t} = \dot{\vector{q}}\,.
        \end{gather}
    \end{example}

\subsection{Euler--Lagrange equations}\index{Euler--Lagrange!equations}

    \begin{axiom}[d'Alembert's principle]\index{d'Alembert!principle}\index{virtual!displacement}\label{classic:dalembert_principle}
        \begin{gather}
            \sum_i(\vector{F}_i-\dot{\vector{p}}_i)\cdot\delta\vector{q}^i=0\,,
        \end{gather}
        where the $\delta\vector{q}^i$ denote the \textbf{virtual displacement} vectors, i.e.~the infinitesimal variations consistent with the contraints. In the spirit of calculus of variations (\cref{section:variational_calculus}), these are defined as follows. Consider a trajectory $\vector{q}:[0,1]\rightarrow\mathbb{R}^n$ and a variation $\vector{\gamma}$ of $\vector{q}$, i.e.~a smooth function $\vector{\gamma}:[0,1]\times[-\varepsilon,\varepsilon]\rightarrow\mathbb{R}^n$ such that $\vector{\gamma}(t,0)=\vector{q}(t)$. This function encodes how $\vector{q}$ can vary given the constraints of the system. The virtual displacement vector is then defined as the tangent vector
        \begin{gather}
            \delta\vector{q} := \left.\deriv{\vector{\gamma}}{\varepsilon}\right|_{\varepsilon=0}\,.
        \end{gather}
    \end{axiom}

    \begin{formula}[Euler--Lagrange equation of the first kind]\index{generalized!force}\index{Euler--Lagrange!equations}
        \begin{gather}
            \label{classic:first_kind}
            \deriv{}{t}\left(\pderiv{T}{\dot{q}^k}\right) - \pderiv{T}{q^k}=Q_k\,,
        \end{gather}
        where $T$ is the total kinetic energy and $Q_k$ are the \textbf{generalized forces}:
        \begin{gather}
            Q_k := \sum_i\vector{F}_i\cdot\pderiv{\vector{r}_i}{q^k}\,.
        \end{gather}
        Note that the constraint forces do not contribute to this quantity because they are always perpendicular to the motion.
    \end{formula}
    \begin{formula}[Euler--Lagrange equation of the second kind]
        \begin{gather}
            \label{classic:second_kind}
            \deriv{}{t}\left(\pderiv{L}{\dot{q}^k}\right) - \pderiv{L}{q^k} = 0
        \end{gather}
        Two proofs are provided:
        \begin{mdframed}[roundcorner=10pt, linecolor=blue, linewidth=1pt]
            \begin{proof}[Proof based on d'Alembert's principle]
                In the following derivation, the mass is assumed to be constant.
                \begin{align*}
                    &\sum_k\left(\vector{F}_k - \dot{\vector{p}}_k\right)\cdot\dot{\vector{r}}_k = 0\\
                    \iff&\sum_k\left(\vector{F}_k - \dot{\vector{p}}_k\right)\cdot\left(\sum_l\pderiv{\vector{r}_k}{q^l}\dot{q^l}\right) = 0\\
                    \iff&\sum_l\left(\sum_k\vector{F}_k\cdot\pderiv{\vector{r}_k}{q^l} - \sum_km\ddot{\vector{r}}_k\cdot\pderiv{\vector{r}_k}{q^l}\right)\dot{q^l} = 0\\
                    \iff&\sum_l\left(Q_l - \sum_km\ddot{\vector{r}}_k\cdot\pderiv{\vector{r}_k}{q^l}\right)\dot{q^l} = 0\,.
                \end{align*}
                Now, consider the following derivative:
                \begin{align*}
                    &\deriv{}{t}\left(\dot{\vector{r}}\cdot\pderiv{\vector{r}}{q^l}\right) = \ddot{\vector{r}}\cdot\pderiv{\vector{r}}{q^l} + \dot{\vector{r}}\cdot\deriv{}{t}\left(\pderiv{\vector{r}}{q^l}\right)\\
                    \iff&\ddot{\vector{r}}\cdot\pderiv{\vector{r}}{q^l} = \deriv{}{t}\left(\dot{\vector{r}}\cdot\pderiv{\vector{r}}{q^l}\right) - \dot{\vector{r}}\cdot\deriv{}{t}\left(\pderiv{\vector{r}}{q^l}\right)\\
                    \iff&\ddot{\vector{r}}\cdot\pderiv{\vector{r}}{q^l} = \deriv{}{t}\left(\dot{\vector{r}}\cdot{\color{red}\underbrace{\textcolor{black}{\pderiv{\vector{r}}{q^l}}}_A}\right) - \dot{\vector{r}}\cdot\pderiv{\dot{\vector{r}}}{q^l}\,.
                \end{align*}
                To evaluate the factor indicated by \textcolor{red}{A}, one can consider another derivative:
                \begin{align*}
                    \pderiv{\dot{\vector{r}}}{\dot{q}^l} &= \pderiv{}{\dot{q}^l}\left(\sum_k\pderiv{r}{q^k}\dot{q}^k\right)\\
                    &=\sum_k\pderiv{r}{q^k}\delta_{kl}\\
                    &=\pderiv{\vector{r}}{q^l}\\
                    &=\textcolor{red}{A}\,.
                \end{align*}
                Substituting this in the previous equation gives
                \begin{gather*}
                    %\label{lagrange_deriv:deriv3}
                    \begin{aligned}
                        \ddot{\vector{r}}\cdot\pderiv{\vector{r}}{q^l} &= \deriv{}{t}\left(\dot{\vector{r}}\cdot\pderiv{\dot{\vector{r}}}{\dot{q}^l}\right) - \dot{\vector{r}}\cdot\left(\pderiv{\dot{\vector{r}}}{q^l}\right)\\
                        &=\deriv{}{t}\left(\frac{1}{2}\pderiv{\dot{\vector{r}}^2}{\dot{q}^l}\right) - \frac{1}{2}\pderiv{\dot{\vector{r}}^2}{q^l}\,.
                    \end{aligned}
                \end{gather*}
                If one multiplies this by the mass $m$ and sums over all masses, the following expression is obtained:
                \begin{align*}
                    \sum_km_k\ddot{\vector{r}}_k\cdot\pderiv{\vector{r}_k}{q^l}=\ &\deriv{}{t}\pderiv{}{\dot{q}^l}\left(\sum_k\frac{1}{2}m\dot{\vector{r}}_k^2\right) - \pderiv{}{q^l}\left(\sum_k\frac{1}{2}m\dot{\vector{r}}_k^2\right)\\
                    =\ &\deriv{}{t}\pderiv{T}{\dot{q}^l} - \pderiv{T}{q^l}%\label{lagrange_deriv:deriv4}\,,
                \end{align*}
                where the total kinetic energy is denoted by $T$ in the last line. Plugging this result into the first equation gives
                \begin{gather*}
                    \sum_l\left(Q_l - \deriv{}{t}\pderiv{T}{\dot{q}^l} - \pderiv{T}{q^l}\right)\dot{q^l} = 0\,.
                \end{gather*}
                Because all the coordinates $q^l$ are independent, the following relation should hold for all $l$:
                \begin{gather*}
                    \begin{aligned}
                        &Q_l - \deriv{}{t}\left(\pderiv{T}{\dot{q}^l}\right) - \pderiv{T}{q^l} = 0\\
                        \iff&\deriv{}{t}\left(\pderiv{T}{\dot{q}^l}\right) - \pderiv{T}{q^l} = Q_l\,.
                    \end{aligned}
                \end{gather*}
                This last equation is the Euler--Lagrange equation of the first kind. If the system only contains conservative forces, the force on the $i^{th}$ mass can be written as
                \begin{gather*}
                    F_i = -\nabla_iV\,.
                \end{gather*}
                With this in mind, one can relate the partial derivatives of the potential to the generalized forces:
                \begin{gather*}
                    \begin{aligned}
                        \pderiv{V}{q^l} &= \sum_i\left(\nabla_iV\right)\cdot\pderiv{\vector{r}_i}{q^l}\\
                        &=-Q_l\,.
                    \end{aligned}
                \end{gather*}
                Furthermore, the derivative of $V$ with respect to the generalized velocities vanishes. This, combined with the Euler--Lagrange equation of the first kind, gives
                \begin{align*}
                    &\deriv{}{t}\left(\pderiv{T}{\dot{q}^l}\right) - \pderiv{T}{q^l} = Q_l\\
                    \iff&\deriv{}{t}\left(\pderiv{T}{\dot{q}^l}\right) - \pderiv{T}{q^l} = -\pderiv{V}{q^l} + \pderiv{V}{\dot{q}^l}\\
                    \iff&\deriv{}{t}\left(\pderiv{T}{\dot{q}^l} - \pderiv{V}{\dot{q}^l}\right) - \pderiv{}{q^l}\left(T - V\right) = 0\,.
                \end{align*}
                If one introduces the Lagrangian $L:=T-V$, one gets the Euler--Lagrange equation of the second kind:
                \begin{gather*}
                    \deriv{}{t}\left(\pderiv{L}{\dot{q}^l}\right) - \pderiv{L}{q^l} = 0\,.
                \end{gather*}$ $
            \end{proof}
        \end{mdframed}

        \begin{mdframed}[roundcorner=10pt, linecolor=blue, linewidth=1pt]
            \begin{proof}[Proof based on the principle of least action]\index{Hamilton!principle}
                First, recall \cref{classic:action} of the action:
                \begin{gather*}
                    S[q] := \Int_{t_1}^{t_2}L\bigl(q(t),\dot{q}(t),t\bigr)\,dt\,.
                \end{gather*}
                The principle of least action (also called \textbf{Hamilton's principle}) postulates that the action is minimal for the physically relevant path. To this end, define a family of paths
                \begin{gather*}
                    q(t,\alpha) = q(t) + \alpha\eta(t)\,,
                \end{gather*}
                where $\eta(t)$ is an arbitrary function satisfying the following boundary conditions:
                \begin{gather*}
                    \begin{cases}
                    &\eta(t_1) = 0\,,\\
                    &\eta(t_2) = 0\,.
                    \end{cases}
                \end{gather*}
                If the action integral is extended to such a family, the integral becomes a function of $\alpha$:
                \begin{gather*}
                    S(\alpha) := \Int_{t_1}^{t_2}L\bigl(q(t,\alpha),\dot{q}(t,\alpha),t\bigr)\,dt\,.
                \end{gather*}
                Requiring that the action integral is stationary for the physical path $q(t)$, where $\alpha=0$, is equivalent to requiring that the derivative at $\alpha=0$ vanishes:
                \begin{gather*}
                    \left.\deriv{S}{\alpha}\right|_{\alpha=0} = 0\,.
                \end{gather*}
                As one evaluates this derivative at $\alpha=0$, $q(t,\alpha)$ coincides with $q(t)$ in this expression:
                \begin{gather*}
                    \begin{aligned}
                        \deriv{S}{\alpha}&=\Int_{t_1}^{t_2}\sum_l\left[\pderiv{L}{q^l}\pderiv{q^l}{\alpha} + \pderiv{L}{\dot{q}^l}\pderiv{\dot{q}^l}{\alpha}\right]\,dt\\
                        &=\Int_{t_1}^{t_2}\sum_l\left[\pderiv{L}{q^l}\eta^l(t) + \pderiv{L}{\dot{q}^l}\dot{\eta}^l(t)\right]\,dt\,.
                    \end{aligned}
                \end{gather*}
                By applying integration by parts to the second term in this integral, one obtains
                \begin{align*}
                    \deriv{S}{\alpha}&=\Int_{t_1}^{t_2}\sum_l\left[\pderiv{L}{q^l}(t)\eta^l(t) + \pderiv{L}{\dot{q}^l}(t)\dot{\eta}^l(t)\right]\,dt\\
                    &=\Int_{t_1}^{t_2}\sum_l\pderiv{L}{q^l}(t)\eta^l(t)\,dt + \pderiv{L}{\dot{q}^l}(t_2)\eta^l(t_2) - \pderiv{L}{\dot{q}^l}(t_1)\eta^l(t_1) - \Int_{t_1}^{t_2}\deriv{}{t}\left(\pderiv{L}{\dot{q}^l}\right)\eta^l(t)\,dt\,.
                \end{align*}
                Due to the initial conditions for the function $\eta$, the second and third term vanish:
                \begin{gather*}
                    \deriv{S}{\alpha}=\Int_{t_1}^{t_2}\sum_l\left[\pderiv{L}{q^l} - \deriv{}{t}\left(\pderiv{L}{\dot{q}^l}\right)\right]\eta^l(t)\,dt\,.
                \end{gather*}
                Furthermore, because the function $\eta$ was arbitrary, the only possible way that this derivative can be zero is when the integrand itself is zero:
                \begin{gather*}
                    \pderiv{L}{q^l} - \deriv{}{t}\left(\pderiv{L}{\dot{q}^l}\right) = 0\,.
                \end{gather*}
                Comparing this result to \cref{classic:second_kind} shows that one can also obtain the Lagrangian equations of the second kind by starting from the principle of least action.
            \end{proof}
        \end{mdframed}
    \end{formula}

    \begin{definition}[Cyclic coordinate]\index{cyclic}
        If the Lagrangian $L$ does not explicitly depend on a coordinate $q^k$, the coordinate is said to be cyclic.
    \end{definition}

    \begin{theorem}[Noether]\index{Noether}\label{classic:noether_cyclic}
        The conjugate momentum of a cyclic coordinate is a conserved quantity:
        \begin{gather}
            \dot{p}_k \overset{\ref{classic:conjugate_momentum}}{=} \deriv{}{t}\left(\pderiv{L}{\dot{q}^k}\right)\overset{{\eqref{classic:second_kind}}}{=}\pderiv{L}{q^k}=0\,.
        \end{gather}
    \end{theorem}

\subsection{Kepler problem}\index{Kepler!problem}\label{section:kepler}

    \begin{formula}[Gravitational potential for a point mass]\index{gravity}\label{classic:gravitational_potential}
        \begin{gather}
            V = -G\frac{M}{r}\,,
        \end{gather}
        where $G=6,67\mathrm{e}{-11}\frac{\mathrm{Nm}^2}{\mathrm{kg}^2}$ is the \textbf{gravitational constant}.
    \end{formula}

    To solve the Kepler problem, it is useful to perform a coordinate transformation. Simultaneously, one passes to the \textbf{center-of-mass reference frame}\index{center of mass}
    \begin{gather}
        \vector{R} := \frac{m_1\vector{r}_1+m_2\vector{r}_2}{m_1+m_2}\qquad\qquad M:=m_1+m_2
    \end{gather}
    and also introduces the \textbf{reduced mass}:\index{mass!reduced}
    \begin{gather}
        \vector{r}:=\vector{r}_1-\vector{r}_2\qquad\qquad\mu:=\frac{m_1m_2}{m_1+m_2}\,.
    \end{gather}
    With these coordinates, the Kepler Lagrangian becomes:
    \begin{gather}
        L_{\text{Kepler}} = \frac{1}{2}M\dot{R}^2 + \frac{1}{2}\mu\dot{r}^2 + \frac{GM}{r}\,.
    \end{gather}
    The first consequence of this transformation is that the equations of motion decouple. The center of mass behaves as a free particle:
    \begin{gather}
        M\ddot{\vector{R}} = 0\,,
    \end{gather}
    whereas the displacement vector and reduced mass are those of a particle influenced by a gravitational force:
    \begin{gather}
        \mu\ddot{\vector{r}} = \frac{GM}{r^2}\symbf{\hat{e}_r}\,.
    \end{gather}

    \begin{theorem}[Bertrand]\index{Bertrand}
        The only central force problems with closed, bounded orbits, are given by the inverse potential
        \begin{gather}
            V(r)\sim\frac{1}{r}
        \end{gather}
        and harmonic potential
        \begin{gather}
            V(r)\sim r^2\,.
        \end{gather}
    \end{theorem}

    \newdef{Laplace--Runge--Lenz vector}{\index{Laplace--Runge--Lenz vector}\label{classic:lrl_vector}
        Although all conservative central force problems have some common symmetries (time translations and rotations), the central force problems with a $1/r$-potential have an extra symmetry. However, in contrast to the ordinary Euclidean symmetries, this new symmetry is harder to understand. It is a dynamical symmetry (\cref{symplectic:dynamical_symmetry}) instead of a kinematical one (\cref{symplectic:kinematical_symmetry}), i.e.~it cannot simply be deduced from the Lagrangian.

        Before explaining the symmetry itself, the associated Noether charge (\cref{symplectic:noether}), the Laplace--Runge--Lenz vector, is given:
        \begin{gather}
            \vector{A} := \vector{p}\times\vector{L} - mk\vector{r}\,,
        \end{gather}
        where $m$ denotes the mass and $k$ fixes the scale of the potential:
        \begin{gather}
            V(r) = \frac{k}{r}\,.
        \end{gather}
        The reason why this conserved quantity cannot be obtained from a cyclic coordinate of the ordinary Lagrangian is that it is actually associated to a 4-dimensional problem. The central problem with an inverse-square force law can be reformulated as the free motion of a particle in 4 dimensions and only in this description can the Noether charge be obtained from a cyclic coordinate.
    }

\section{Hamiltonian mechanics}

    \newdef{Canonical coordinates}{\index{canonical!coordinates}
        Consider the generalized coordinates $(q,\dot{q},t)$. From these, one can derive a new set of coordinates, called canonical coordinates, by exchanging the time derivatives $\dot{q}^i$ in favour of the conjugate momenta $p_i$.

        This is only equivalent if the transformation $(q,\dot{q})\leftrightarrow(q,p)$ is invertible. A sufficient condition is that the Hessian of $L$ with respect to the generalized velocities is invertible.
    }

    \newdef{Phase space}{\index{phase space}
        The set of all possible $2n$-tuples $(q^i,p_i)$ of generalized coordinates and associated momenta. Note that this set also includes the tuples that are not solutions of the equations of motion.
    }

    \newdef{Libration}{\index{libration}
        A closed trajectory in phase space for which the coordinates take on only a subset of the allowed values. It is the generalization of an oscillation. Topologically, it is characterized by a contractible, closed trajectory.
    }
    \newdef{Rotation}{\index{rotation}
        A closed trajectory in phase space for which at least one of the variables takes on all possible values. Topologicaly, it is characterized by a noncontractible, closed trajectory.

        \todo{CORRECT (for an ordinary rotation in spherical coordinates this indeed holds, however, for rotations in Cartesian coordinates this fails)}
    }

    \newdef{Separatrix}{\index{separatrix}
        When plotting (closed) trajectories in phase space, the curve that separates regions of librations and rotations is called the separatrix.\footnote{In general, the separatrix of a dynamical system is a curve that separates regions with different behaviour.}
    }

    \newdef{Hamiltonian function}{\index{Hamilton!function}\label{classic:hamiltonian}
        Given a Lagrangian $L$, the Hamiltonian function is defined by the following Legendre transformation (\cref{calculus:legendre}):
        \begin{gather}
            H(q,p,t) := \sum_ip_i\dot{q}^i - L(q,p,t)\,.
        \end{gather}
    }

    \newformula{Hamilton's equations}{\index{Hamilton!equations}\label{classic:hamilton_equations}
        Inserting the above definition in the action (\cref{classic:action}) and applying the variational principle results in the following equations:
        \begin{gather}
            \begin{cases}
                &\ds\dot{q}^k = \pderiv{H}{p_k}\,,\\
                &\ds\dot{p}_k = -\pderiv{H}{q^k}\,.
            \end{cases}
        \end{gather}
        Systems obeying these equations are called \textbf{Hamiltonian systems}. (See \cref{section:hamiltonian_dynamics} for the mathematical background).
    }

\subsection{Poisson bracket}

    \newdef{Poisson bracket}{\index{Poisson!bracket}
        To stick with the conventions of \cref{symplectic:poisson}, the Poisson bracket is defined as
        \begin{gather}
            \{A,B\} := \sum_k\pderiv{A}{p_k}\pderiv{B}{q^k} - \pderiv{A}{q^k}\pderiv{B}{p_k}\,,
        \end{gather}
        where $q,p$ are the canonical coordinates in the Hamiltonian formalism.
    }
    \sremark{As noted before, some authors define the Poisson bracket with the opposite sign. One should always pay attention to which convention is used.}

    \newformula{Total time derivative}{\label{classic:total_time_derivative}
        Hamilton's equations imply the following expression for the total time derivative:
        \begin{gather}
            \deriv{F}{t} = \pderiv{F}{t} + \{H,F\}\,,
        \end{gather}
        which is the evolution equation for a nonautonomous system (\cref{symplectic:nonautonomous_system}).
    }

    \newdef{Liouville operator}{\index{Liouville!operator}\label{classic:liouville_operator}
        The operator $\widehat{L}:C^\infty(M)\rightarrow C^\infty(M)$ defined as follows:
        \begin{gather}
            \widehat{L}f := \{H,f\}\,.
        \end{gather}
    }

\subsection{Hamilton--Jacobi equation}

    For a geometrical interpretation, see \cref{section:hamilton_jacobi}.

    \newdef{Canonical transformations}{\index{canonical!transformation}\label{classic:canonical_transformation}
        A transformation that leaves the Hamiltonian equations of motion unchanged. This means that the transformation leaves the action invariant up to a constant or, equivalently, it leaves the Lagrangian invariant up to a complete time-derivative:
        \begin{gather}
            \sum_ip_i\dot{q}^i - H(q,p,t) = \sum_iP_i\dot{Q}^i - K(Q,P,t) - \deriv{S}{t}(Q,P,t)\,.
        \end{gather}
        The function $S$ is called the \textbf{generating function} of the canonical transformation. The choice of generating function uniquely determines the transformation (the converse is, however, not true).
    }

    \newformula{Hamilton--Jacobi equation}{\index{Hamilton--Jacobi equation}\index{Hamilton!principal function}\index{Kamiltonian}
        Sufficient conditions for a function $S$ to be a generating function of canonical transformations are:
        \begin{gather}
            \begin{aligned}
                p_i &=\pderiv{S}{q^i}\,,\\
                Q^i &=\pderiv{S}{P_i}\,,
            \end{aligned}
        \end{gather}
        and
        \begin{gather}
            K = H + \pderiv{S}{t}\,.
        \end{gather}
        Choosing the new Hamiltonian function $K$, sometimes called the \textbf{Kamiltonian}, to be 0 gives the Hamilton--Jacobi equation:
        \begin{gather}
            \label{classic:hamilton_jacobi_equation}
            H\left(q,\pderiv{S}{q},t\right)+\pderiv{S}{t} = 0\,.
        \end{gather}
        In this case the function $S$ is called \textbf{Hamilton's principal function}.
    }

    \begin{property}
        The new coordinates $Q^i$ and $P_i$ are all constants of motion. This immediately follows from the choice $K=0$.
    \end{property}

    \newdef{Hamilton's characteristic function}{\index{Hamilton!characteristic function}\index{energy}
        For time-independent systems, the Hamilton--Jacobi equation can be rewritten as follows:
        \begin{gather}
            \label{classic:time_independent_hje}
            H\left(q,\pderiv{S}{q}\right) = -\pderiv{S}{t} =: E\,.
        \end{gather}
        After a redefinition of $H$, this is the same as \cref{symplectic:hamilton_jacobi}. One thus obtains the classical result that, for time-independent systems, the Hamiltonian function is a constant of motion (often called the \textbf{energy}\footnote{Note that, in general, this is not the physical energy.}). Integration with respect to time gives the following form of the principal function:
        \begin{gather}
            S(q,p,t) = W(q,p) - Et\,.
        \end{gather}
        The time-independent function $W$ is called Hamilton's characteristic function.
    }

    \begin{property}[St\"ackel condition]\index{St\"ackel potential}
        The Hamilton--Jacobi equation is separable if and only if the potential is of the form
        \begin{gather}
            \label{classic:stackel_condition}
            V(q) = \sum_{i=1}^n\ds\frac{W_i(q^i)}{G_i^2(q)}
        \end{gather}
        whenever the Hamiltonian function can be written as
        \begin{gather}
            H(q,p) = \frac{1}{2}\sum_i\frac{p_i^2}{G^2_i(q)} + V(q)\,.
        \end{gather}
        Potentials of this form are called \textbf{St\"ackel potentials}.
    \end{property}

\section{Continuum mechanics}
\subsection{Liouville's theorem}\label{section:lagrangian_liouville}

    Let $\rho:\mathbb{R}^{2n}\rightarrow\mathbb{R}_+$ be the phase space distribution function of some extensive quantity $F$ and $\vol$ be the canonical measure, i.e.~the quantity $F$ evaluated on the volume $V\subseteq\mathbb{R}^{2n}$ is given by
    \begin{gather}
        F(V) = \Int_V\rho\vol \equiv \Int_V\rho\,dV\,.
    \end{gather}

    \begin{formula}[Liouville's lemma]\index{incompressible}
        Consider a time-dependent phase space volume $V:\mathbb{R}\rightarrow\mathcal{B}(\mathbb{R}^{2n})$, evolving along a path $x(t)\equiv\bigl(q(t),\dot{q}(t)\bigr)$. The Jacobian $J\bigl(x(t)\bigr)$ associated with this motion is given by
        \begin{gather}
            J\bigl(x(t)\bigr) = \det\left(\pderiv{x(t)}{x(0)}\right) = \sum_{ijklmn}\varepsilon^{ijklmn}\pderiv{x^1}{x^i_0}\pderiv{x^2}{x^j_0}\pderiv{x^3}{x^k_0}\pderiv{x^4}{x^l_0}\pderiv{x^5}{x^m_0}\pderiv{x^6}{x^n_0}\,.
        \end{gather}
        The Lagrangian derivative of this Jacobian is
        \begin{gather}
            \label{classic:jacobian_derivative}
            \Deriv{J}{t} = (\nabla\cdot\dot{x})J\,,
        \end{gather}
        where $\nabla$ denotes the gradient in phase space. Flows for which the right-hand side vanishes, i.e.~for which $\nabla\cdot\dot{x}=0$, are said to be \textbf{incompressible}.
    \end{formula}

    \newformula{Reynolds' transport theorem}{\index{Reynolds' transport theorem}\index{Leibniz!integral rule}\label{classic:reynolds_transport_theorem}
        Consider a quantity
        \begin{gather}
            F(t) := \Int_{V(t)}f(q,\dot{q},t)\,dV\,,
        \end{gather}
        where $V:\mathbb{R}\rightarrow\mathcal{B}(\mathbb{R}^{2n})$ is a time-dependent volume in phase space. Combining \cref{classic:jacobian_derivative} with the divergence theorem~\ref{vector:divergence_theorem} gives
        \begin{gather}
            \Deriv{F}{t} = \Int_{V(t)}\left(\pderiv{f}{t}+\nabla\cdot(f\dot{\vector{q}})\right)\,dV = \Int_{V(t)}\pderiv{f}{t}\,dV + \Oint_{\partial V(t)}f\dot{\vector{q}}\cdot d\vector{S}\,,
        \end{gather}
        where $\dot{\vector{q}}$ denotes the velocity of the infinitesimal area element. This formula can be interpreted as a three-dimensional generalization of the Leibniz integral rule (\cref{calculus:leibniz_integral_rule}).
    }

    \newdef{Material volume}{\index{material!volume}
        A time-dependent volume $V:\mathbb{R}\rightarrow\mathcal{B}(\mathbb{R}^{2n})$ such that the velocity $\dot{\vector{q}}$ at every point matches the flow velocity.
    }
    \sremark{In most parts of the literature, Reynolds' transport theorem is stated for material volumes. For simplicity, this will be assumed from here on.}

    \begin{result}
        Note that Reynolds' transport theorem can be rewritten as
        \begin{gather}
            \Deriv{F}{t} = \Int_{V(t)}\left(\Deriv{f}{t}+f\nabla\cdot\dot{\vector{q}}\right)\,dV\,.
        \end{gather}
        Hence, for incompressible fluids, the material derivative can be taken inside the integral.
    \end{result}

    \newformula{Continuity equations}{\index{continuity!equation}
        For a conserved quantity $F$, Reynolds' transport theorem becomes:
        \begin{gather}
            \label{classic:eulerian_continuity_gather}
            \pderiv{f}{t} + \nabla\cdot(f\dot{\vector{q}}) = 0
        \end{gather}
        or, after using \cref{classic:lagrangian_derivative},
        \begin{gather}
            \label{classic:lagrangian_continuity_gather}
            \Deriv{f}{t} + (\nabla\cdot\dot{\vector{q}})f = 0\,.
        \end{gather}
        If one sets $f=\rho$ (the phase space density), the equations are called the \textbf{Eulerian continuity equation} and \textbf{Lagrangian continuity equation}, respectively.

        The difference between these two equations corresponds to the way the system is observed. In the Eulerian approach, one observes a fixed point in space and measures how a given quantity at that point evolves. In the Lagrangian approach, one observes a fixed element in the system and measures how a given quantity evolves around the chosen point as it moves throughout space.
    }
    \begin{result}
        Combining Reynolds' transport theorem with the continuity equation, and assuming that the velocity of $V(t)$ matches the flow velocity, gives the following identity for an arbitrary function $f:\mathbb{R}^{2n+1}\rightarrow\mathbb{R}$:
        \begin{gather}
            \label{classic:result1}
            \Deriv{}{t}\Int_{V(t)}\rho f\,dV = \Int_{V(t)}\rho\Deriv{f}{t}\,dV\,.
        \end{gather}
    \end{result}

    Although this result does imply that the total probability in any material volume is preserved, the preservation of the probability density function requires a slightly stronger assumption.
    \begin{theorem}[Liouville]\index{Liouville!theorem on phase spaces}\index{Liouville!equation}\label{classic:liouvilles_theorem}
        If the flow is incompressible, the density is constant along the flow:\footnote{This equation is also called the \textbf{Liouville equation}.}
        \begin{gather}
            \Deriv{\rho}{t} = 0\,.
        \end{gather}
        Moreover, the volume of any material volume is also constant:
        \begin{gather}
            \Deriv{\vol(V)}{t} = 0
        \end{gather}
        for all material volumes $V:\mathbb{R}\rightarrow\mathcal{B}(\mathbb{R}^{2n})$.
    \end{theorem}

    \begin{remark}[$\clubsuit$]
        By \cref{section:hamiltonian_dynamics}, every phase space admits a symplectic form $\omega$ and the induced canonical volume form on phase space is $\vol_\omega\sim\omega^n$. Accordingly, the above theorem (more specifically, the second part) easily follows from the fact that time evolution is the flow of the Hamiltonian vector field, which preserves the symplectic form.
    \end{remark}

    \newformula{Boltzmann's transport equation}{\index{Boltzmann!transport equation}\index{Vlasov equation}
        Let $F(q,\dot{q},t)$ be the mass distribution function:
        \begin{gather}
            M_{\text{tot}} = \Int_{V(t)}F(q,\dot{q},t)\,dV\,.
        \end{gather}
        From the conservation of mass, one can derive the following formula:
        \begin{gather}
            \label{classic:boltzmann_transport_gather}
            \Deriv{F}{t} = \pderiv{F}{t} + \dot{q}^i\cdot\pderiv{F}{q^i} - \nabla V^i\pderiv{F}{\dot{q}^i} = \left[\pderiv{F}{t}\right]_{\text{col}}\,,
        \end{gather}
        where the right-hand side gives the change of $F$ due to collisions.\footnote{The collisionless form of this equation is sometimes called the \textbf{Vlasov equation}.} This partial differential equation in 7 variables can be solved to obtain $F$.
    }

    A prime example of flows are given by time evolution in a Hamiltonian system. The Hamiltonian equations of motion can be shown to imply the incompressibility of the Hamiltonian flow and, accordingly, by Liouville's theorem, this flow gives rise to a volume- (or measure-)preserving map $g:\mathbb{R}^{2n}\rightarrow\mathbb{R}^{2n}$ (\cref{measure:measure_preserving}). This gives rise to the following theorem.
    \begin{theorem}[Poincar\'e recurrence theorem]\index{Poincar\'e!recurrence theorem}
        For every point $x_0\in\mathbb{R}^{2n}$ and for every neighbourhood $U\subseteq\mathbb{R}^{2n}$ of $x_0$, there exists a point $y\in U$ such that $g^n(y)\in U$ for every $n\in\mathbb{N}$.
    \end{theorem}

    \begin{theorem}[Strong Jeans's theorem\footnotemark]\index{Jeans}\index{isolating integrals}
        \footnotetext{Actually due to \indexauthor{Lynden-Bell}.}
        In a time-independent system for which almost all orbits are regular, the distribution function can be expressed in terms of three integrals of motion.
    \end{theorem}
    The constants in Jeans's theorem are called the \textup{\textbf{isolating integrals}} of the system.

\subsection{Fluid mechanics}

    \todo{ADD (e.g.~stress vector)}

    \begin{theorem}[Cauchy's stress theorem\footnotemark]\index{Cauchy!fundamental theorem}\index{stress}
        \footnotetext{Also known as \textbf{Cauchy's fundamental theorem}.}
        The stress vectors acting on the coordinate planes through a point determine the stress vector acting on an arbitrary plane passing through that point.
    \end{theorem}

    Cauchy's stress theorem is equivalent to the existence of the following tensor.
    \newdef{Cauchy stress tensor}{\index{Cauchy!stress tensor}
        The Cauchy stress tensor is a $(0,2)$-tensor $\mathbf{T}$ that gives the relation between a stress vector associated to a plane and the normal vector $\vector{n}$ to that plane:
        \begin{gather}
            \vector{t}_{(\vector{n})} = \mathbf{T}(\vector{n})\,.
        \end{gather}
    }
    \begin{example}
        For identical particles, the stress tensor is given by
        \begin{gather}
            \mathbf{T} = -\rho\langle\vector{w}\otimes\vector{w}\rangle\,,
        \end{gather}
        where $\vector{w}$ is the random component of the velocity vector and $\langle\cdot\rangle$ denotes the expectation value (\cref{prob:expectation_value}).
    \end{example}

    \begin{theorem}[Cauchy's lemma]\index{Cauchy!lemma}
        The stress vectors acting on opposite planes are equal in magnitude but opposite in direction:
        \begin{gather}
            \vector{t}_{(-\vector{n})} = -\vector{t}_{(\vector{n})}\,.
        \end{gather}
    \end{theorem}

    \newformula{Cauchy momentum equation}{\index{Cauchy!momentum equation}\index{shear stress}
        From Newton's second law (\cref{classic:force}), it follows that
        \begin{gather}
            \Deriv{\vector{P}}{t} = \Int_V\vector{f}(x,t)\,dV + \Oint_{\partial V}\vector{t}(x,t)\,dS\,,
        \end{gather}
        where $\vector{P}$ is the momentum density, $\vector{f}$ are the `body' forces and $\vector{t}$ are the `surface' forces (such as \textit{shear stress}). Using Cauchy's stress theorem and the divergence theorem~\ref{vector:divergence_theorem}, one obtains
        \begin{gather}
            \Deriv{\vector{P}}{t} = \Int_V\left[\vector{f}(x,t) + \nabla\cdot\mathbf{T}(x,t)\right]\,dV\,.
        \end{gather}
        The left-hand side can be rewritten using \cref{classic:result1} as
        \begin{gather}
            \int_V\rho\Deriv{\vector{v}}{t}\,dV = \Int_V\left[\vector{f}(x,t) + \nabla\cdot\mathbf{T}(x,t)\right]\,dV\,.
        \end{gather}
    }

\section{Koopman--von Neumann mechanics}\index{Koopman--von Neumann!mechanics}\index[author]{Koopman}\index[author]{von Neumann}

    After the invention and formalization of quantum mechanics (see \cref{part:qm}), some authors tried to reformulate classical mechanics in terms of operators and Hilbert spaces. A first step towards this goal was made by, amongst others, \textit{Koopman} and \textit{von Neumann}, who expressed the measure-preserving transformations (i.e.~the canonical transformation) on phase space as operators, which form a Hilbert space under the Hilbert--Schmidt inner product (\cref{linalgebra:hilbert_schmidt_norm}). However, notwithstanding this reformulation, the content of this section is actually of a different nature and the historical attribution is, therefore, incorrect. The idea of this section is to rephrase the classical probability distribution of phase space as a wave function.

    Recall the Liouville equation (\cref{classic:liouvilles_theorem})
    \begin{gather}
        \Deriv{\rho}{t}=0
    \end{gather}
    for the phase space density function. Using \cref{classic:liouville_operator}, this can be rewritten as
    \begin{gather}
        \pderiv{\rho}{t} = -\widehat{L}\rho\,.
    \end{gather}

    \newdef{Koopman--von Neumann wave function}{
        The KvN wave function $\psi:\mathbb{R}^{2n+1}\rightarrow\mathbb{C}$ is required to factorize the density function:
        \begin{gather}
            \rho = \overline{\psi}\psi
        \end{gather}
        and satisfy the Liouville equation:
        \begin{gather}
            \pderiv{\psi}{t} = \widehat{L}\psi\,.
        \end{gather}
    }

    An alternative approach start from the operator formulation of quantum mechanics, where the observables are represented as self-adjoint operators on a Hilbert space, as mentioned above. The axioms are as follows:\index{Born rule}
    \begin{enumerate}
        \item\textbf{Normalization}: $\braket{\psi}{\psi} = 1$.
        \item\textbf{Expectation}: $\expect{A} = \langle\psi\mid\widehat{A}\mid\psi\rangle$.
        \item\textbf{Born rule}: The probability of measuring the value $a\in\sigma(\widehat{A})$ is given by
        \begin{gather}
            p_a = |\braket{a}{\psi}|^2\,.
        \end{gather}
        \item\textbf{Compositionality}: The disjoint union of systems is represented by the tensor product of their wave functions.
    \end{enumerate}

    \begin{property}[Generality]
        The above set of axioms are, in fact, very general, in that they allow to recover both classical mechanics and quantum mechanics, depending on an additional set of relations. First of all, if Newton's second law (\cref{classic:force}) is postulated to hold on average (in an Ehrenfest-like manner, cf.~\cref{qm:ehrenfest}), i.e.
        \begin{gather}
            m\deriv{}{t}\langle x\rangle=\langle p\rangle \qquad\text{and}\qquad \deriv{}{t}\langle p\rangle=-\langle\nabla U(x)\rangle
        \end{gather}
        for some function $U:\mathbb{R}^n\rightarrow\mathbb{R}$, the following conditions are recovered:
        \begin{gather}
            m[\widehat{L},\widehat{x}]=\widehat{p} \qquad\text{and}\qquad [\widehat{L},\widehat{p}]=-\nabla U(\widehat{x})\,,
        \end{gather}
        where $\widehat{L}$ is the generator of time evolution according to Stone's theorem~\ref{functional:stone}.

        Now, if the classical commutation relation $[\widehat{x},\widehat{p}]=0$ is required to hold, the above equations cannot hold for any operator $\widehat{L}\equiv L(\widehat{x},\widehat{p})$. To solve this issue, an additional set of operators $\{\widehat{\lambda}_x,\widehat{\lambda}_p\}$ has to be introduced. These are required to form a Lie algebra, called the \textbf{KvN algebra}, with the following commutation relations:\index{Koopman--von Neumann!algebra}
        \begin{gather}
            \begin{aligned}
                [\widehat{\lambda}_x,\widehat{x}] &= [\widehat{\lambda}_p,\widehat{p}] = i\,,\\
                [\widehat{x},\widehat{p}] &= [\widehat{\lambda}_x,\widehat{p}] = [\widehat{\lambda}_p,\widehat{x}] = [\widehat{\lambda}_x,\widehat{\lambda}_p] = 0\,.
            \end{aligned}
        \end{gather}
        A form of $\widehat{L}$ that makes the above conditions hold is:
        \begin{gather}
            -i\widehat{L} = \frac{\widehat{p}}{m}\widehat{\lambda}_x - \nabla U(\widehat{x})\widehat{\lambda}_p\,.
        \end{gather}
        Using this expression, the classical rules of statistical mechanics can be recovered.

        To recover \textit{quantum mechanics} (see \cref{chapter:qm}), the \textit{canonical commutation relations} (see \cref{qm_formalism:CCR}) are postulated: $[\widehat{x},\widehat{p}]=i$. In this case, no additional operators have to be introduced and (up to a factor $\hbar$), the Liouville operator can be shown to coincide with standard Hamiltonian:
        \begin{gather}
            \hbar\widehat{L} = \frac{\widehat{p}^2}{2m}+U(\widehat{x})\,,
        \end{gather}
        for some function $U:\mathbb{R}^n\rightarrow\mathbb{R}$.
    \end{property}

    \begin{remark}[Linearity]
        Note that, in contrast to the \textit{Schr\"pdinger equation} (see, e.g., \cref{qm:TDSE_position}), both the Liouville and KdV equations are linear PDEs in classical mechanics.
    \end{remark}

    \todo{COMPLETE}

\section{Dynamical systems}

    The following property, although seemingly innocuous, is very important for the study of mechanical systems.
    \begin{property}
        For dynamical systems governed by ODEs satisfying the Picard--Lindel\"of conditions (\cref{ode:picard_lindelof}), different trajectories never intersect.
    \end{property}

    The above property has an important (but nontrivial) consequence.
    \begin{theorem}[Poincar\'e--Bendixson]\index{Poincar\'e--Bendixson}
        In a two-dimensional phase space, the only trajectories inside of a closed bounded subregion without fixed points are either closed orbits or trajectories spiralling into closed orbits.
    \end{theorem}
    \begin{result}\index{strange attractor}
        In two-dimensional (Cartesian) phase spaces, there cannot exist chaos, i.e.~no \textit{strange attractors} can exist.
    \end{result}

    \newdef{Lypanuov exponents}{\index{Lyapunov exponent}
        Consider two trajectories of a system and denote the distance between them at a time $t_0$ by $s_0:=s(t_0)$. (The initial time can be taken to be 0 without loss of generality.) If, after some time $t$, these trajectories satisfy
        \begin{gather}
            s(t)\approx e^{\lambda t}s_0\,,
        \end{gather}
        the parameter $\lambda$ is called the Lyapunov exponent of the system.
    }

    \newdef{Limit cycle}{\index{cycle|seealso{limit}}\index{limit!cycle}
        Consider a closed trajectory $C$. If there exist curves that asymptotically $(t\longrightarrow\pm\infty$) converge to $C$, i.e.~their \textbf{limit set} is $C$, one calls $C$ a limit cycle.
    }

    \newdef{Poincar\'e map}{\index{Poincar\'e!map}
        Consider a dynamical system determined by a phase flow $\phi$ and let $S$ be a codimension-1 hypersurface in the phase space $Q$ that is transversal to $\phi$, i.e.~all trajectories intersect $S$ at isolated points. Intuitively, the Poincar\'e map $P:S\rightarrow S$ is defined as the `map of first return', i.e.~the image of every point $x\in S$, if it exists, is given by $\phi_T(x)$ with $T=:\min\{t\in\mathbb{R}^+\mid\phi_t(x)\in S\}$.

        One can give a more formal definition (one that also avoids the fact that $P$ would only be partially defined). The Poincar\'e map $P$ is defined as follows:
        \begin{itemize}
            \item $P:U\rightarrow S$ is a differentiable map, where $U\subseteq S$ is open and connected.
            \item $P|_{P(U)}$ is a diffeomorphism.
            \item For every point $u\in U$, the positive semi-orbit of $u$ intersects $S$ for the first time at $P(u)$.
        \end{itemize}
        The usefulness of this map lies in the fact that it preserves (quasi)periodicity whilst reducing the dimensionality of the space. It is especially useful for the study of 3-dimensional spaces, where the section $S$ is 2-dimensional and, hence, easily visualized.
    }
    \begin{property}[Fixed points]
        Fixed points of the Poincar\'e map correspond to closed orbits.
    \end{property}

\section{Geometric mechanics}
\subsection{Newton's equations}

    In this section, the current chapter is reformulated in a differential-geometric framework (in the spirit of \cref{section:hamiltonian_dynamics}). The first step is to reformulate ordinary Newtonian mechanics. The general setting here is a Riemannian manifold $(M,g)$, where the metric is mainly used for defining the kinetic energy
    \begin{gather}
        \label{classic:kinetic_energy}
        E_{\text{kin}} := \frac{1}{2}g(v,v)\,,
    \end{gather}
    together with a second-order ODE in the form of a vector field $X\in \mathfrak{X}(TM)$ such that $\pi_*(X_v)=v$, where $\pi:TM\rightarrow M$ denotes the tangent bundle projection.

    For integral curves $\gamma$ in $TM$ of second-order ODEs, it is easy to show that they are the tangent vector fields of their projections. In particular, if $q^i(t)$ are the local coordinates of a curve $\sigma:=\pi(\gamma)$ on $M$, it can be shown that the tangent coordinates $\dot{q}^i$ of $\gamma$ are exactly the derivatives of the coordinates $q^i$:
    \begin{gather}
        \dot{q}^i(t) = \deriv{q^i}{t}(t)\,.
    \end{gather}
    As such, the abuse of notation $\dot{q}^i$ is justified. Conversely, it can be shown that a vector field on $TM$ is second order if and only if this is true for any local chart, i.e.~if the vector field $X\in\mathfrak{X}(TM)$ can be expressed as follows:
    \begin{gather}
        X = \dot{q}^i\pderiv{}{q^i} + F^i(q,\dot{q})\pderiv{}{\dot{q}^i}\,.
    \end{gather}
    By writing this vector field as a system of differential equations, a second-order ODE is obtained (hence the terminology):
    \begin{gather}
        \mderiv{2}{q^i}{t} = F^i(q,\dot{q})\,.
    \end{gather}
    The primary example of such a second-order ODE is the vector field generating the geodesic flow on $TM$, i.e.~the integral curves are the tangent curves of geodesics on $M$. Therefore, it should not be surprising that the above equation is similar to \cref{diff:geodesic_equation}. By adopting the notation of \cref{riemann:autoparallel_equation}, one can generalize the geodesic equation to obtain Newton's equations of motion for an arbitrary smooth potential $U:M\rightarrow\mathbb{R}$.
    \begin{formula}[Newton's equation]\index{Newton!equation}
        Let $(M,g)$ be a Riemannian manifold and let $U:M\rightarrow\mathbb{R}$ be a smooth function. Newton's equations for a curve $\sigma:[a,b]\rightarrow M$ read
        \begin{gather}
            \nabla_{\dot{\sigma}}\dot{\sigma} = -\mathrm{grad}(U)\,,
        \end{gather}
        where $\nabla$ indicates the Levi-Civita connection and $\mathrm{grad}$ denotes the gradient operator (\cref{vector:gradient}), which is here not denoted by $\nabla$ to avoid confusion with the covariant derivative.
    \end{formula}

    \begin{property}[Cogeodesic flow]
        Consider the kinetic Hamiltonian~\eqref{classic:kinetic_energy} expressed on the cotangent bundle $T^*M$ of a Riemannian manifold:
        \begin{gather}
            H_{\text{kin}}(q,p) := \frac{1}{2}\sum_{i,j}g^{ij}(q)p_ip_j\,,
        \end{gather}
        where $g^{ij}$ is the inverse of the (local) matrix representative of the metric $g$. The integral curves of the associated Hamiltonian flow project down to geodesics on $M$.
    \end{property}

\subsection{Lagrangian mechanics}

    Now it is time to turn to Noether's theorem and, in particular, the version concerning cyclic coordinates (\cref{classic:noether_cyclic}). Any diffeomorphism of $M$ induces a diffeomorphism on $TM$ by pushforward.
    \newdef{Lagrangian symmetry}{\index{symmetry!Lagrangian}
        A symmetry of a Lagrangian $L:TM\rightarrow\mathbb{R}$ is a diffeomorphism $\phi$ of $M$ such that $\phi^*L=L$. Infinitesimal symmetries, i.e.~the generators of symmetries, are vector fields for which the associated flow is a symmetry.
    }

    Given a complete vector field $X$, one can define the conjugate momentum $\widehat{X}:TM\rightarrow\mathbb{R}$ as follows:
    \begin{gather}
        \label{classic:metric_conjugate_momentum}
        \widehat{X}(v) := g(X_{\pi(v)},v)\,.
    \end{gather}
    Using this definition, Noether's theorem~\ref{classic:noether_cyclic} can be reformulated as follows.
    \begin{theorem}[Noether's theorem]\index{Noether!theorem}
        The conjugate momentum of an infinitesimal symmetry is a constant of motion.
    \end{theorem}

    If the conjugate momenta of the coordinate-induced vector fields $\partial_i$ are denoted by $p_i$, the nondegeneracy of $g$ implies that the set $\{q^i,p_i\}_{i\leq n}$ gives well-defined coordinate functions on $T^*M$. The equivalence of the Lagrangian action principle and the Newtonian equations of motion imply that the second-order ODE associated to the potential $U$ takes the following form:
    \begin{gather}
        X_U := \dot{q}^i\pderiv{}{q^i} + \pderiv{L}{q^i}\pderiv{}{p_i}\,.
    \end{gather}
    After performing the Legendre transformation $E:=p_i\dot{q}^i-L$ to obtain the (Hamiltonian) energy function, Newton's equations can be rewritten in Hamiltonian form:
    \begin{gather}
        X_E = \pderiv{E}{p_i}\pderiv{}{q^i}-\pderiv{E}{q^i}\pderiv{}{p_i}\,.
    \end{gather}
    The function $E$ will not be called a Hamiltonian function because this will be reserved for functions on the cotangent bundle.

\subsection{Hamiltonian mechanics}

    The procedure of mapping a (complete) vector field to its conjugate momentum can be generalized to an isomorphism $TM\rightarrow T^*M$ as follows.
    \newdef{Fibre derivative}{\index{fibre!derivative}
        Let $L:TM\rightarrow\mathbb{R}$ be a smooth Lagrangian. The fibre derivative of $L$ is defined as the directional (G\^ateaux) derivative (\cref{functional:gateaux}) of $L$:
        \begin{gather}
            \langle\mathbb{F}L(v),w\rangle := \left.\deriv{}{t}L(v+tw)\right|_{t=0}\,.
        \end{gather}
        Because $\mathbb{F}L(v)\in\mathcal{L}(TM,\mathbb{R})=T^*M$ by definition of the derivative, one can see that $\mathbb{F}L$ defines a map $TM\rightarrow T^*M$. In local coordinates $(q^i,\dot{q}^i)$, the fiber derivative is given by
        \begin{gather}
            \mathbb{F}L:(q^i,\dot{q}^i)\mapsto\left(q^i,\pderiv{L}{\dot{q}^i}\right)\equiv(q^i,p_i)\,.
        \end{gather}
        As such, the classical definition~\ref{classic:conjugate_momentum} for conjugate momenta is obtained. In the case of kinetic Lagrangians defined by a metric $g$, it is not hard to see that this boils down to \cref{classic:metric_conjugate_momentum}.
    }
    \begin{remark}[Legendre transform]\index{Legendre!transformation}
        The fibre derivative $\mathbb{F}L$ is often called the Legendre transform of $L$. Although this does not exactly coincide with \cref{calculus:legendre} or \cref{info:legendre}, the relation is simple enough. The Legendre transformation $L\mapsto L^*$ (on the tangent bundle) is implemented as
        \begin{gather}
            L^*(x) = \langle\mathbb{F}L(x),x\rangle - L(x)\,.
        \end{gather}
    \end{remark}

    Lagrangians for which the Legendre transformation is invertible, i.e.~for which $\mathbb{F}L$ is a diffeomorphism, give rise to equivalent mechanics encoded in the cotangent bundle. Given such a Lagrangian, we can construct the associated energy function $E$ by a Legendre transformation and map it to a Hamiltonian $H$ on the cotangent bundle as $H:=E\circ\mathbb{F}L^{-1}$. By abuse of notation, write $L\equiv L\circ\mathbb{F}L^{-1}$. This transformation also induces a Hamiltonian vector field $X_H:=\mathbb{F}L_*X_E$ on $T^*M$ that can alternatively be encoded as $X_E=\mathbb{F}L^*X_H$. If the cotangent coordinates $p_i(\alpha) := \alpha(\partial_i)$ are chosen, it can easily be seen that $p_i\circ\mathbb{F}L\equiv p_i$. This way the Hamiltonian equations remain virtually unchanged when transporting them to the cotangent bundle.

    For any choice of coordinates for which the symplectic form on $T^*M$ takes the standard Darboux form $\omega=dp_i\wedge dq^i$, the Newtonian equations of motion take on a Hamiltonian form. If there exists a coordinate chart in which the Hamiltonian function $H$ does not depend on any of the base coordinates $q^i$, the coordinates are called \textbf{action-angle variables} and the system is said to be \textbf{completely integrable} (cf.~\cref{section:integrability}).\index{action-angle coordinates}\index{integrability!complete}

\subsection{Contact structure}

    By extending the cotangent bundle of the configuration space with a time variable, one can also (re)obtain some interesting features. Consider the following one-form on the trivial line bundle $T^*M\times\mathbb{R}$:
    \begin{gather}
        \alpha := p_i\dr q^i - H\dr t\,.
    \end{gather}
    First of all, this endows the extended cotangent bundle with a contact structure (\cref{contact:contact_structure}). Moreover, it can be shown to be a relative integral invariant (\cref{bundle:integral_invariant}) of the following Pfaffian system:
    \begin{gather}
        \frac{\dr q^i}{\smallpderiv{H}{p_i}} = \dr t = \frac{-\dr p_i}{\smallpderiv{H}{q^i}}\,.
    \end{gather}
    The vector fields that leave the contact form invariant are exactly those generated by the Hamiltonian vector field $X_H$. Liouville's theorem~\ref{classic:liouvilles_theorem} is then simply a consequence of the absolute invariance of $\dr\alpha$.

    A canonical transformation (\cref{classic:canonical_transformation}) was originally defined as a transformation that leaves the Hamiltonian equations of motion invariant. In the contact setting, it can be defined as follows.
    \newdef{Canonical transformation}{\index{canonical!transformation}\index{symplectization}
        A transformation of the trivial line bundle $T^*M\times\mathbb{R}$ that leaves both the fibres and the \textbf{symplectization}\footnote{This form turns $T^*M\times\mathbb{R}^2$ into a symplectic manifold.}
        \begin{gather}
            \Omega := \dr(e^\lambda\alpha)
        \end{gather}
        invariant.

        \todo{CHECK THIS STATEMENT (cannot find source)}
    }
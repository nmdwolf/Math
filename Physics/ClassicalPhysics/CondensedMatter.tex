\chapter{Material Physics \& Condensed matter physics}

    \minitoc

\section{Crystals}\index{crystal}

    \begin{theorem}[Steno's law]\index{Steno}
        The angles between crystal faces of the same type are constant and do not depend on the total shape of the crystal.
    \end{theorem}

    \newdef{Zone}{\index{zone}
        The collection of faces parallel to a given axis. The axis itself is called the \textbf{zone axis}.
    }

\subsection{Analytic representation}

    \newdef{Miller indices}{\index{index!Miller}
        Let $a,b,c$ be the lengths of the (not necessarily orthogonal) basis vectors of the crystal lattice. The lattice plane intersecting the axes at $\left(\frac{a}{h},\frac{b}{k},\frac{c}{k}\right)$ is denoted by the Miller indices $(h\ k\ l)$.
    }
    \begin{notation}
        Negative numbers are often written as $\overline{a}$ instead of $-a$.
    \end{notation}

    \newformula{Axes}{
        Let $a,b,c$ denote the lengths of the basis vectors. The axis formed by the intersection of the planes $(h_1\ k_1\ l_1)$ and $(h_2\ k_2\ l_2)$ is denoted by $[u\ v\ w]$. Its direction is determined by the point $(au,bv,cw)$, where
        \begin{gather}
            u = \left|
            \begin{matrix}
                k_1&l_1\\
                k_2&l_2
            \end{matrix}\right|
            \qquad v = \left|
            \begin{matrix}
                l_1&h_1\\
                l_2&h_2
            \end{matrix}\right|
            \qquad w = \left|
            \begin{matrix}
                h_1&k_1\\
                h_2&k_2
            \end{matrix}\right|\,.
        \end{gather}
    }

    \begin{theorem}[Hauy's law of rational indices]\index{Hauy}
        The Miller indices of every natural face of a crystal will always have rational proportions.
    \end{theorem}

\section{Symmetries}

    \newdef{Equivalent planes/axes}{
        When applying crystal symmetries, it often happens that a set of equivalent planes and axes is obtained. These equivalence classes are denoted by $\{h\ k\ l\}$ and $\langle h\ k\ l\rangle$, respectively.
    }

    \newprop{Rotational symmetry}{
        Only $1,2,3,4$ and $6$-fold rotational symmetries can occur.
    }

\section{Crystal lattice}

    \begin{formula}\label{solid:d_hkl}
        For an orthogonal crystal lattice, the distance between planes of the family $(h\ k\ l)$ is given by
        \begin{gather}
            d_{hkl} = \frac{1}{\sqrt{\ds\left(\frac{h}{a}\right)^2 + \left(\frac{k}{b}\right)^2 + \left(\frac{l}{c}\right)^2}}\,.
        \end{gather}
    \end{formula}

\subsection{Bravais lattice}

    \newdef{Bravais lattice}{\index{Bravais lattice}
        A crystal lattice generated by a point group symmetry. There are 14 different Bravais lattices in 3 dimensions. These are the only possible ways to place (infinitely) many points in 3D space by applying symmetry operations consistent with the given point group.
    }

    \newdef{Wigner--Seitz cell}{\index{Wigner--Seitz cell}
        The part of space consisting of all points closer to a given lattice point than to any other.
    }

    \begin{theorem}[Neumann's principle]\index{Neumann!principle}
        The symmetry elements of the physical properties of a crystal should at least contain those of the point group of the crystal.
    \end{theorem}

\subsection{Reciprocal lattice}

    Since crystalline materials have a lattice periodicity, any well-defined function on this lattice $\Lambda$ has a Fourier decomposition:
    \begin{gather}
        f(\vector{r}) = \sum_{\vector{k}\in\Lambda^*}f_ke^{i\vector{k}\cdot\vector{r}}\,.
    \end{gather}
    Now, translation symmetry means that
    \begin{gather}
        f(\vector{r}) = f(\vector{r}+\vector{a})
    \end{gather}
    for any $\vector{a}\in\Lambda$ or, since Fourier decompositions are essentially unique,
    \begin{gather}
        \label{solid:orthogonality}
        \vector{k}\cdot\vector{a} \in 2\pi\mathbb{Z}\,.
    \end{gather}

    \newformula{Reciprocal basis vectors}{\index{reciprocal lattice}
        The reciprocal lattice corresponding to a Bravais lattice with primitive basis $\{\vector{a},\vector{b},\vector{c}\}$ is defined by the following reciprocal basis vectors:
        \begin{gather}
            \vector{a}^* := 2\pi\frac{\vector{b}\times\vector{c}}{\vector{a}\cdot(\vector{b}\times\vector{c})}\,.
        \end{gather}
        The vectors $\vector{b}^*$ and $\vector{c}^*$ are obtained by cyclic permutation of $(a,b,c)$. These vectors satisfy relation~\eqref{solid:orthogonality}:
        \begin{gather}
            \begin{aligned}
                \vector{a}\cdot\vector{a}^* &= 2\pi\,,\\
                \vector{b}\cdot\vector{b}^* &= 2\pi\,,\\
                \vector{c}\cdot\vector{c}^* &= 2\pi\,.
            \end{aligned}
        \end{gather}
    }
    \newnot{Reciprocal lattice vector}{
        The reciprocal lattice vector $\vector{r}^*_{hkl}$ is defined as follows:
        \begin{gather}
            \vector{r}^*_{hkl} := h\vector{a}^* + k\vector{b}^* + l\vector{c}^*\,.
        \end{gather}
    }

    \begin{property}
        The reciprocal lattice vector $\vector{r}^*_{hkl}$ has the following properties:
        \begin{itemize}
            \item $\vector{r}^*_{hkl}$ is perpendicular to the family of planes $(h\ k\ l)$ of the direct lattice, and
            \item $\|\vector{r}^*_{hkl}\| = \frac{2\pi n}{d_{hkl}}$.
        \end{itemize}
    \end{property}

\section{Diffraction}\index{diffraction}
\subsection{Constructive interference}

    \newformula{Laue conditions}{\index{Laue!conditions}
        Suppose that an incident beam makes angles $\alpha_0,\beta_0$ and $\gamma_0$ with the lattice axes. A diffracted beam, making angles $\alpha,\beta$ and $\gamma$ with the axes, will be observed if the following conditions are satisfied:
        \begin{gather}
            \begin{aligned}
                a(\cos\alpha - \cos\alpha_0) &= h\lambda\,,\\
                b(\cos\beta - \cos\beta_0) &= k\lambda\,,\\
                c(\cos\gamma - \cos\gamma_0) &= l\lambda\,.
            \end{aligned}
        \end{gather}
        If these conditions have been met, a diffracted beam of order $hkl$ will be observed.
    }
    \begin{remark}
        Further conditions can be imposed on the angles, such as the Pythagorean formula for orthogonal axes. This has the consequence that the only two possible ways to obtain a diffraction pattern are:
        \begin{itemize}
            \item a fixed crystal and a polychromatic beam, or
            \item a rotating crystal and a monochromatic beam.
        \end{itemize}
    \end{remark}

    \newformula{Vectorial Laue conditions}{\label{solid:vectorial_laue_condition}
        Let $\vector{k}_0,\vector{k}$ denote the wave vectors of the incident and diffracted beams respectively. The Laue conditions can be reformulated in the following way:
        \begin{gather}
            \vector{k} - \vector{k}_0 = \vector{r}^*_{hkl}\,.
        \end{gather}
    }

    \newformula{Bragg's law}{\index{Bragg's law}\label{solid:braggs_law}
        Another equivalent formulation of the Laue conditions is given by the following formula:
        \begin{gather}
            2d_{hkl}\sin\theta = n\lambda\,,
        \end{gather}
        where
        \begin{itemize}
            \item $\lambda$ is the wavelength of the incoming beam,
            \item $\theta$ is the \textbf{Bragg angle}, and
            \item $d_{hkl}$ is the distance between neighbouring planes.
        \end{itemize}
    }
    \begin{remark}
        The angle between the incident and diffracted beams is $2\theta$.
    \end{remark}

    A simple construction to determine where the maxima of Bragg diffraction will occur is the following.
    \begin{construct}[Ewald sphere]\index{Ewald sphere}
        Put the origin of the reciprocal lattice at the tip of the incident wave vector $\vector{k}_i$. Construct a sphere with radius $\frac{2\pi}{\lambda}$ centred on the initial point of $\vector{k}_i$. All points on the sphere that coincide with a reciprocal lattice point satisfy the vectorial Laue condition~\ref{solid:vectorial_laue_condition}. Therefore, Bragg diffraction will occur in the direction of all the intersections of the Ewald sphere and the reciprocal lattice.
    \end{construct}

\subsection{Intensity of diffracted beams}

    \newprop{Systematic extinctions}{\index{extinction}
        Every particle in the motive emits its own waves. These waves will interfere and some will cancel out. This leads to the absence of certain diffraction spots, the systematic extinctions.
    }

    \newdef{Atomic scattering factor}{\index{scattering!factor}
        The waves produced by the individual electrons of an atom, which can have a different phase, can be combined into a resulting wave. The amplitude of this wave is called the atomic scattering factor.
    }
    \newdef{Structure factor}{\index{structure!factor}\label{solid:structure_factor}
        The waves coming from the individual atoms in the motive can also be combined into a resulting wave (again, taking into account the different phases). The amplitude of this wave is called the structure factor. It is given by
        \begin{gather}
            F(hkl) = \sum_jf_j\exp\bigl(2\pi i(hx_j + ky_j + lz_j)\bigr)\,,
        \end{gather}
        where $f_j$ is the atomic scattering factor of the $j^{\text{th}}$ atom in the motive.
    }

    \begin{example}
        An important example of systematic extinctions is the structure factor of an FCC or BCC lattice. If $h+k+l$ is odd, $F(hkl) = 0$ for a BCC lattice. If $h,k$ and $l$ are not all even or all odd, $F(hkl) = 0$ for an FCC lattice.
    \end{example}

    \newdef{Laue indices}{\index{Laue!indices}
        Higher-order diffractions can be rewritten as a first-order diffraction in the following way:
        \begin{gather}
            2d_{nhnknl}\sin\theta = \lambda \qquad\text{with}\qquad d_{nhnknl} = \frac{d_{hkl}}{n}\,.
        \end{gather}
        The interpretation of Bragg's law as diffraction being a reflection in the lattice plane $(h\ k\ l)$ implies that one can introduce the (fictitious) plane with indices $(nh\ nk\ nl)$. These indices are called Laue indices.
    }
    \sremark{In contrast to Miller indices, which cannot possess common factors, the Laue indices obviously can.}

\section{Alloys}

    \begin{property}[Hume-Rothery conditions]\index{Hume-Rothery conditions}
        An element can be dissolved in a metal (forming a solid solution) if the following conditions are met:
        \begin{enumerate}
            \item the difference between the atomic radii is $\leq 15\%$,
            \item the crystal structures are the same,
            \item the elements have a similar electronegativity, and
            \item the valence is the same.
        \end{enumerate}
    \end{property}

\section{Lattice defects}

    \newdef{Interstitial}{\index{interstitial}
        An atom placed at a position that is not a lattice point.
    }

    \newdef{Vacancy}{\index{vacancy|see{Schottky defect}}\index{Schottky!defect}
        A lattice point where an atom is missing. This is also called a \textbf{Schottky defect}.
    }
    \begin{formula}[Concentration of Schottky defects]\label{solid:schottky_defects}
        Let $N,n\in\mathbb{N}$ denote the number of lattice points and vacancies, respectively. The following relation gives the temperature dependence of Schottky defects:
        \begin{gather}
            \frac{n}{n + N} = e^{-E_v/kT}\,,
        \end{gather}
        where $T$ denotes the temperature and $E_v$ the energy needed to create a vacancy. A similar relation holds for interstitials.
        \begin{mdframed}[roundcorner=10pt, linecolor=blue, linewidth=1pt]
            \begin{proof}
                Let $E_v$ be the energy needed to remove a particle from its lattice point and move it to the surface. All surface effects will be neglected and it will be assumed that the the energy $E_v$ is independent of the distance to the surface.

                The total energy of all vacancies is then given by $E = nE_v$. The number of possible microstates is
                \begin{gather}
                    \Omega = \frac{(N+n)!}{N!\,n!}\,,
                \end{gather}
                where the fact that the removal of $n$ particles creates $n$ more lattice points at the surface was used. Using Boltzmann's entropy formula~\eqref{statmech:boltzmann_formula} and Stirling's formula~\eqref{calculus:stirling} one obtains
                \begin{gather}
                    S(N,n) = k\ln\Omega = k\bigl[(N+n)\ln(N+n) -n\ln n - N\ln N\bigr]\,.
                \end{gather}
                Using \cref{statmech:temperature} of the temperature as the derivative of the energy gives
                \begin{gather}
                    \frac{1}{T} = \left(\pderiv{S}{E}\right)_{N, V} = \deriv{S}{n}\deriv{n}{E} = \frac{k}{E_v}\ln\frac{N+n}{n}
                \end{gather}
                or
                \begin{gather}
                    \frac{n}{N + n} = \exp\left(-\frac{E_v}{kT}\right)\,.
                \end{gather}
                The density of Frenkel pairs can be derived analogously.
            \end{proof}
        \end{mdframed}
    \end{formula}

    \newdef{Frenkel pair}{\index{Frenkel pair}
        An atom displaced from a lattice point to an interstitial location, thereby creating a vacancy-interstitial pair.
    }
    \begin{formula}[Concentration of Frenkel pairs]
        Let $n_i\in\mathbb{N}$ denote the number of atoms displaced from the bulk of the lattice to any of $N_i\in\mathbb{N}$ possible interstitial positions, thereby creating $n_i$ vacancies. The following relation holds:
        \begin{gather}
            \frac{n_i}{\sqrt{NN_i}} = e^{-E_{\text{fr}}/2kT}\,,
        \end{gather}
        where $E_{\text{fr}}$ denotes the energy needed to create a Frenkel pair.
    \end{formula}

    \remark{In compounds, the number of vacancies can be much higher than in mono-atomic lattices.}
    \remark{The existence of these defects creates the possibility of diffusion.}

\section{Electrical properties}
\subsection{Charge carriers}

    \newformula{Conductivity}{\index{conductivity}\label{solid:conductivity}
        \Cref{em:conductivity} can be modified to account for both positive and negative charge carriers:
        \begin{gather}
            \sigma := n_nq_n\mu_n + n_pq_p\mu_p\,.
        \end{gather}
    }
    \sremark{The difference between the concentration of positive and negative charge carriers can differ by orders of magnitude (up to 20) across different materials.}

\subsection{Band structure}

    \newdef{Valence band}{\index{valence band}
        The energy band corresponding to the outermost (partially) filled atomic orbital.
    }
    \newdef{Conduction band}{\index{conduction band}
        The first unfilled energy band.
    }
    \newdef{Band gap}{\index{band gap}
        The energy difference between the valence and conduction bands (if they do not overlap). It is the energy zone where no electron states can exist.\footnote{For a basic derivation, see~\cite{bransden_quantum_2000}.}
    }

    \newdef{Fermi level}{\index{Fermi!level}
        The energy level having a 50\% chance of being occupied at thermodynamic equilibrium.
    }
    \newformula{Fermi function}{\index{Fermi!function}\label{solid:fermi_function}
        The following distribution gives the probability of a state with energy $E_i$ being occupied by an electron:
        \begin{gather}
            f(E_i) = \frac{1}{e^{\beta(E_i - E_f)} + 1}\,,
        \end{gather}
        where $E_f$ is the Fermi level as defined above.
    }

    \begin{formula}
        Let $n\in\mathbb{N}$ denote the charge carrier density as before. The following temperature dependence can be found:
        \begin{gather}
            n \sim e^{-\beta E_g/2}\,,
        \end{gather}
        where $E_g$ is the band gap. This formula can be directly derived from the Fermi function by noting that, for intrinsic semiconductors, the Fermi level sits in the middle of the band gap, i.e.~$E_c-E_f = E_g/2$, and that, for most semiconductors, $E_g\gg kT$.
    \end{formula}

    \newdef{Doping}{\index{doping}
        Intentionally introducing impurities to modify the (electrical) properties.
    }
    \newdef{Acceptor}{\index{acceptor}
        A group-III element added to create an excess of holes in the valence band. The resulting semiconductor is called a \textbf{p-type semiconductor}.
    }
    \newdef{Donor}{\index{donor}
        A group-IV element added to create an excess of electrons in the valence band. The resulting semiconductor is called an \textbf{n-type semiconductor}.
    }

\subsection{Ferroelectricity}

    Some materials can exhibit phase transitions between a paraelectric and ferroelectric state. Paraelectric materials have the property that the polarization $\vector{P}$ and the electric field $\vector{E}$ are proportional. Ferroelectric materials have the property that they exhibit permanent polarization, even in the absence of an electric field. This permanent behaviour is the result of symmetry breaking. The ions in the lattice have been displaced from their `central' positions, which induces a permanent dipole moment.

    The temperature at which this phase transition occurs is called the \textbf{ferroelectric Curie temperature}. Above this temperature, the material will behave as a paraelectric material.

   \begin{remark}
       Ferroelectricity can only occur in crystals with unit cells that do not have a center of symmetry. This would rule out the possibility of having the asymmetry needed for the dipole moment.
   \end{remark}

    \newdef{Saturation polarization}{\index{polarization}
        The maximum polarization obtained by a ferroelectric material. It it obtained when the \textit{domain formation} reaches a maximum.
    }
    \newdef{Remanent polarization}{
        The residual polarization of the material when the external electric field is turned off.
    }
    \newdef{Coercive field}{\index{coercive field}\index{remanent}
        The electric field needed to cancel out the \textit{remanent} polarization.
    }

    \newdef{Piezoelectricity}{\index{piezoelectricity}
        Materials that obtain a polarization when exposed to mechanical stress are called piezoelectric materials.
    }
    \begin{remark}
        All ferroelectric materials are piezoelectric, but the converse is not true. Moreover, all crystals without a center of symmetry are piezoelectric. This property is, however, only a necessary (and not a sufficient) condition for ferroelectricity.
    \end{remark}

    \begin{example}[Transducer]\index{transducer}
        A device that converts electrical energy to mechanical energy (and vice versa).
    \end{example}

\section{Magnetic properties}

    \newdef{Diamagnetism}{\index{diamagnetism}
        In diamagnetic materials, the magnetization is oriented oppositely to the applied field, so $B<H$. The susceptibility is small, negative and independent of the temperature.
    }
    \begin{remark}
        All materials exhibit diamagnetic behaviour.
    \end{remark}

    \newdef{Paramagnetism}{\index{para-!magnetism}
        The susceptibility is small, positive and inversely proportional to the temperature.
    }

    \newdef{Ferromagnetism}{\index{ferromagentism}\index{Curie!temperature}
        Spontaneous magnetization can occur. The susceptibility is large and dependent on the applied field and temperature. Above a certain temperature, the \textbf{ferromagnetic Curie temperature}, the materials will behave as if they were only paramagnetic.
    }

\subsection{Paramagnetism}

    \newformula{Curie's law}{\index{Curie}\label{solid:curies_law}
        If the interactions between the particles can be neglected, the following relation is obtained for the susceptibility (\cref{em:magnetic_field}):
        \begin{gather}
            \chi = \frac{C}{T}\,.
        \end{gather}
        Materials that satisfy this law are called \textbf{ideal paramagnetics}.
    }
    \newformula{Curie--Weiss law}{\index{Weiss|see{Curie--Weiss}}\index{Curie--Weiss}\label{solid:curie_weiss__law}
        If the interactions between particles cannot be neglected, a correction of Curie's law is required:
        \begin{gather}
            \chi = \frac{C}{T-\theta}\,,
        \end{gather}
        where $\theta=CN_W$, with $N_W$ the \textbf{Weiss-constant}. This deviation from Curie's law is due to the intermolecular interactions that induce an internal magnetic field $H_m=N_WM$.
    }

    \begin{formula}[Brillouin function]\index{Brillouin function}\label{solid:brillouin_function}
        \begin{gather}
            B_J(y) := \frac{2J + 1}{2J}\coth\left(\frac{2J + 1}{2J}y\right) - \frac{1}{2J}\coth\left(\frac{y}{2J}\right)\,,
        \end{gather}
        where $y := \tfrac{g\mu_BJB}{k_BT}$.
    \end{formula}

    \begin{remark}\label{solid:absolute_saturation_magnetization}
        Because $\lim_{y\rightarrow\infty}\coth(y)=1$, one obtains:
        \begin{gather}
            T\longrightarrow0\implies M = \lim_{y\rightarrow\infty}Ng\mu_BJB_J(y) = Ng\mu_BJ\,.
        \end{gather}
        This value is called the \textbf{absolute saturation magnetization}.
    \end{remark}

\subsection{Ferromagnetism}

    Ferromagnetics are materials that have strong internal interactions that lead to large scale (with respect to the lattice constant) parallel ordering of the atomic magnetic (dipole) moments. This also leads to the spontaneous magnetization of the material and, consequently, to a nonzero total dipole moment. In reality, however, ferromagnetic materials do not always spontaneously possess a magnetic moment in the absence of an external field. Still, when stimulated by a small external field, they will display a magnetic moment much larger than paramagnetic materials would.

    \newdef{Domain}{\index{Weiss!domain}
        This behaviour is explained by the existence of \textbf{Weiss domains}. These are spontaneously magnetized regions in a magnetic material. The total dipole moment is the sum of the moments of the individual domains. If not all the domains have a parallel orientation, the total dipole moment can be 0. A small external field is, however, sufficient to change the domain orientation and produce a large total magnetization.
    }
    \newdef{Bloch walls}{\index{Bloch!walls}
        A wall between two magnetic domains.
    }

    \begin{remark}
        For ferromagnetic (and ferrimagnetic) materials, it is impossible to define a magnetic susceptibility, because the magnetization is nonzero even in the absence of a magnetic field.\footnote{This can be seen in \cref{em:M}: $M=\chi H$. The susceptibility should be infinite.} Above the critical temperature, it is, however, possible to define a susceptibility, because the materials become paramagnetic in this region.
    \end{remark}

\subsection{Antiferromagnetism}

    When the domains in a magnetic material have an antiparallel order (whenever this is energetically more favourable), the total dipole moment will be small. If the temperature rises, the thermal agitation will disturb the orientation of the domains and the magnetic susceptibility will increase.

    \newdef{N\'eel temperature}{\index{N\'eel!temperature}
        At the N\'eel temperature, the susceptibility will reach a maximum. Above this temperature, the material will become paramagnetic, satisfying the following formula:
        \begin{gather}
            \chi = \frac{C}{T+\theta}\,.
        \end{gather}
        This resembles a generalization of the Curie--Weiss law with a negative and, therefore, virtual critical temperature.
    }

\subsection{Ferrimagnetism}\index{ferrimagnetism}

    Materials that are not completely ferromagnetic nor antiferromagnetic, due to an imbalance between the sublattices, will have a nonzero dipole moment even in the absence of an external field. The magnitude of this moment will, however, be smaller than that of a ferromagnetic material. These materials are called \textbf{ferrimagnetic materials}.

    \newformula{N\'eel hyperbola}{\index{N\'eel!hyperbola}
        Above the N\'eel temperature, it is possible to define a susceptibility given by
        \begin{gather}
            \frac{1}{\chi} = \frac{T}{C} + \frac{1}{\chi_0} - \frac{\sigma}{T - \theta}\,,
        \end{gather}
        where $C$ is the Curie constant in the abscence of molecular interactions and $\chi_0,\sigma,\theta$ are functions of the molecular interactions. The last term is also known as the N\'eel hyperbola.
    }

\section{Topological phases of matter}

    \todo{ADD (group cohomology, Jordan--Wigner transformation, string nets, $F$-symbols, anyons, ...)}

\subsection{Tenfold way}

    Two operators of fundamental importance in physics, besides the Hamiltonian, are charge conjugation $\widehat{\mathcal{C}}$ and time reversal $\widehat{\mathcal{T}}$.\footnote{See also \cref{section:cpt}. Parity is not considered since condensed matter physics is not concerned with continuum field theories.} Physically, they square to $\mathbbm{1}$, i.e.~they are involutions. However, because of the rules of quantum mechanics (\cref{qm:projectivization}), operators forms a projective representation and, moreover, by Wigner's theorem~\ref{qm:wigner}, they are implemented antiunitarily. For clarity's sake, their relevant properties are summarized here:
    \begin{itemize}
        \item Charge conjugation (or \textbf{particle-hole symmetry}) $\widehat{\mathcal{C}}$:
        \begin{itemize}
            \item It is antiunitary.
            \item It anticommutes with the Hamiltonian: $[\widehat{\mathcal{C}},\widehat{H}]=0$.
        \end{itemize}
        \item Time reversal $\widehat{\mathcal{T}}$:
        \begin{itemize}
            \item It is antiunitary.
            \item It commutes with the Hamiltonian: $\{\widehat{\mathcal{C}},\widehat{H}\}_+=0$.
        \end{itemize}
    \end{itemize}
    It follows that both symmetries, whenever present, square to $\pm\mathbbm{1}$. All possible combinations sum up to 9. However, there is one last possibility. It might happen that there is no $\widehat{\mathcal{C}}$- or $\widehat{\mathcal{T}}$-symmetry, but that they do form a combined $\widehat{\mathcal{S}} := \widehat{\mathcal{C}}\widehat{\mathcal{T}}$ symmetry. This is called the \textbf{chiral symmetry} (or \textbf{sublattice symmetry}\footnote{This symmetry can be implemented on the lattice by doubling the vertices. It then corresponds to interchanging the two sublattices.}). As the product of two antiunitary involutions, it is a unitary involution and, hence, it can always be taken to square to $\mathbbm{1}$.\index{chiral!symmetry} If all symmetries are present, it can also be shown that $[\widehat{\mathcal{C}},\widehat{\mathcal{T}}]=0$.

    \begin{table}[ht!]
        \centering
        \begin{tabular}{c|ccc}
            \hline
            Class\footnotemark&$\widehat{\mathcal{T}}^2$&$\widehat{\mathcal{C}}^2$&$\widehat{\mathcal{S}}^2$\\
            \hline
            A&0&0&0\\
            AIII&0&0&1\\
            \hline
            AI&1&0&0\\
            BDI&1&1&1\\
            D&0&1&0\\
            DIII&-1&1&1\\
            AII&-1&0&0\\
            CII&-1&-1&1\\
            C&0&-1&0\\
            CI&1&-1&1\\
            \hline
        \end{tabular}
    \end{table}
    \footnotetext{This is also known as the \textbf{Cartan classification}.\index{Cartan!classification}}

    This classification can also be obtained in a different way. This was done by \indexauthor{Kitaev} and others. The algebras $\mathcal{A}$ generated by $\mathbb{1},\widehat{\mathcal{C}},\widehat{\mathcal{T}}$ and $\widehat{\mathcal{S}}$ can be represented as superalgebras or Clifford algebras (up to Morita equivalence), such that the complex Hilbert space $\mathcal{H}$ is a (super)representation of $\mathcal{A}$. This way, one obtains the tenfold classification as a combination of real and complex Bott periodicity (\cref{section:clifford_bott}).

    \begin{itemize}
        \item A: No symmetries, so the algebra of symmetries $\mathcal{A}$ is given by $\mathbb{C}$.
        \item AIII: Only chiral symmetry. In this case, $\mathcal{H}$ decomposes as a direct sum of the eigenspaces $\mathcal{H}_{\pm1}$ with the inclusion of an odd generator. This is a representation of $\mathcal{A}=\mathbb{C}\oplus\mathbb{C}e$ with $e^2=1$ and $[e,i]=0$.
        \item AI: Instead of getting an eigendecomposition as in the previous case, the antiunitarity of $\widehat{\mathcal{T}}$ entails that the imaginary unit $i$ maps $\mathcal{H}_{\pm1}$ to $\mathcal{H}_{\mp1}$. Accordingly, $\mathcal{H}$ is the complexification of the real vector space $\mathcal{H}_{\pm1}$, which is a representation of $\mathcal{A}=\mathbb{R}$.
        \item BDI: The presence of $\widehat{\mathcal{C}}$ entails that $\mathcal{H}$ is a super-Hilbert space on which $\widehat{\mathcal{C}}$ is represented as on odd operator squaring to $1$. Combining this result with the previous case, one obtains $\mathcal{A}=\mathbb{R}\oplus\mathbb{R}e$ with $e^2=1$.
        \item D: In this case, one only has a super-Hilbert space with an odd operator that anticommutes with $i$. It follows that $\mathcal{A}=\mathbb{C}\oplus\mathbb{C}e$ with $e^2=1$ and $\{i,e\}_+=0$.
        \item DIII: This is essentially the same as the previous case, except that one gets an additional square root of $-1$. If one denotes this square root by $j$, it combines with the imaginary unit $i$ to form the algebra $\mathbb{H}$. Adjoining the odd element $e$ with $e^2=1$, one obtains $\mathcal{A}=\mathbb{H}+\mathbb{H}\widetilde{e}$ with $\{e,i\}_+=0=\{e,j\}_+$.\footnote{$\widehat{\mathcal{C}}$ is represented by $eT$ to make it square to $-1$.}
        \item AII: As before, one gets an additional square root of $-1$ and, accordingly, $\mathcal{A}=\mathbb{H}$.
        \item CII: This is very similar to the case DIII. However, instead of $e^2=1$ one has to require $e^2=-1$.
        \item C: This is similar to the case AI but, instead of $e^2=1$, one has to require $e^2=-1$.
        \item CI: This is similar to the case BDI. However, instead of $e^2=1$, one has to resuire $e^2=-1$.
    \end{itemize}

    \begin{remark}[Super division algebras]\index{Brauer group}
        The above classification not only coincides with the Morita classification of CLifford algebras, but also with that of (super)division algebras. The 10 possibilities recover the (Morita-equivalence classes of) real superdivision algebras. It can be shown that BDI generates the so-called \textbf{super Brauer group}, i.e.~(the subgroup of invertible elements of) the monoid of Morita equivalence classes under tensor products.
    \end{remark}

    \begin{property}[Threefold way]
        If one only considers a single antiunitary symmetry, one recovers \indexauthor{Dyson}'s threefold way. This corresponds to the three real division algebras $\mathbb{R},\mathbb{C}$ and $\mathbb{H}$ (classes AI, A and AII).
    \end{property}

\subsection{Topological order}

    This subsection uses the language of \cref{chapter:quantum_computing}.

    One can define an equivalence relation on quantum states using local, \textbf{gapped} Hamiltonians, where gapped means that, even in the thermodynamic limit, there is an energy jump between the ground state and the first excited state.\index{gapped}
    \newdef{Topological order}{\index{order!topological}\label{solid:topological_order}
        Two quantum states $\ket{\psi_{0,1}}\in\mathcal{H}$ are said to be in the same topological order if there exists a continuous path of local, gapped Hamiltonians $\widehat{H}:[0,1]\rightarrow\End(\mathcal{H})$ such that $\ket{\psi_0}$ an $\ket{\psi_1}$ are ground states for $\widehat{H}_0$ and $\widehat{H}_1$, respectively.

        The trivial topological order is the equivalence class containing product states.
    }
    \begin{adefinition}
        Two quantum states are in the same topological order if they are related by a \textbf{local quantum circuit} or local unitary (LU) transformation, i.e.~a unitary transformation generated through finite-time evolution by a local Hamitonian.
    \end{adefinition}

    Although this is a powerful notion, it is not refined enough. There are quantum states with physically different behaviour that can be transformed into each other with a local quantum circuit. The crucial ingredient missing in this formulation is that of symmetries!
    
    \newdef{Symmetry-protected topological order}{\index{order!SPT}
        Two quantum states are in the same SPT phase if they are related by an LU transformation that preserves the symmetries of the system.
    }
    \begin{remark}
        Note that SPT phases are not necessarily (intrinsically) topologically ordered as mentioned above. There are SPT phases that are trivial in the sense of \cref{solid:topological_order}.
    \end{remark}
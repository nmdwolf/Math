\chapter{Tensor Networks}

\section{Matrix Product States}
\subsection{Finite-dimensional lattices}

	\newdef{Matrix product state}{\index{MPS}
		Let $\mathcal{H}_n$ be the local Hilbert spaces of dimension $d_n$ where $n\in\{1,...,N\}$. A state $|\psi\rangle$ in the total Hilbert space $\bigotimes_i\mathcal{H}_i$ is a matrix product state (with periodic boundary conditions) if there exist matrices $A^{i_n}(n)\in\mathcal{L}(\mathbb{C}^{D_n}, \mathbb{C}^{D_{n-1}})$ with $i_n\leq d_n$ such that:
		\begin{equation}
			|\psi\rangle = \sum_{\{i_k\}}\text{tr}\Big(\prod_n^NA^{i_n}(n)\Big)|i_1...i_N\rangle
		\end{equation}
		For each lattice site $n$ the set of matrices $\{A^{i_n}_{\alpha\beta}(n)\}$ can be regarded as the content of one rank-3 tensor. The periodic boundary condition requires that $D_0=D_N$ (otherwise the trace would be ill-defined). Different boundary conditions can be implemented by inserting an additional factor $X$ at the end of the trace.
	}
	\begin{notation}
		In the continuation of this chapter we will abbreviate matrix product states as \textbf{MPS}.
	\end{notation}
	
	\begin{remark}[Physical and virtual spaces]
		For each \textit{physical} index $i_n$ one can regard the matrix $A^{i_n}(n)$ as a linear map between \textit{virtual} (or ancilla) spaces $\mathbb{C}^{D_n}$.
	\end{remark}
	
	\newformula{MPS projector}{
		Consider an MPS given by tensors $\{A(n)\}_{n\leq N}$. The associated MPS projector is defined as:
		\begin{equation}
			\mathcal{P}(A) = \sum_{i, \alpha, \beta}A^i_{\alpha\beta}(n)|i\rangle\langle\alpha\beta|
		\end{equation}
	}
	
	\newformula{Transfer operator}{\index{transfer!operator}
		Give the MPS tensors $\{A(n)\}_{n\leq N}$ one can define a transfer operator:
		\begin{equation}
			\label{mps:transfer_operator}
			\mathbb{E}(n) = \sum_{i=1}^{d_n}A^i(n)\otimes\overline{A^i}(n)
		\end{equation}
	}
	\begin{formula}[Superoperator]\index{super!operator}
		More generally we can define for every local observable $\hat{O}_n$ a superoperator in $\mathcal{L}(\mathbb{C}^{D_n}\otimes\overline{\mathbb{C}^{D_n}}, \mathbb{C}^{D_{n-1}}\otimes\overline{\mathbb{C}^{D_{n-1}}})$:
		\begin{equation}
			\mathbb{E}_{O_n}(n) = \sum_{i,i'=1}^{d_n}\langle i|\hat{O}_n|i' \rangle A^{i'}(n)\otimes\overline{A^i}(n)
		\end{equation}
		Comparing with the definition of the transfer operator we see that $\mathbb{E}$ is given by the superoperator associated to the unit operator. Given two sets of MPS tensors $\{A(n)\}, \{B(n)\}$ we define a generalized superoperator by:
		\begin{equation}
			\mathbb{E}^A_B(n) = \sum_{i=1}^{d_n}A^i(n)\otimes\overline{B^i}(n)
		\end{equation}
	\end{formula}
	\begin{example}
		Using these definitions we can rewrite the formulas for expectation values more efficiently. Given a product operator $\hat{O}=\bigotimes_i^N\hat{O}_i$ we find that:
		\begin{equation}
			\langle\psi[A]|\hat{O}|\psi[A]\rangle = \text{tr}\Big(\mathbb{E}_{O_1}(1)\cdots\mathbb{E}_{O_N}(N)\Big)
		\end{equation}
	\end{example}
	
	\begin{formula}
		Associated to the superoperator $\mathbb{E}_O(n)$ one can define a map acting on the virtual operators:
		\begin{align}
			\mathcal{E}^{(n)}_{O_n}(\phi) &= \sum_{i, i'=1}^{d_n}\langle s|\hat{O}_n|s' \rangle A^{i'}(n)\phi A^i(n)^\dag\\
			\tilde{\mathcal{E}}^{(n)}_{O_n}(\phi) &= \sum_{i, i'=1}^{d_n}\langle s|\hat{O}_n|s' \rangle A^i(n)^\dag\sigma A^{i'}(n)
		\end{align}
		where $\phi\in\mathcal{L}(\mathbb{C}^{D_n}), \sigma\in\mathcal{L}(\mathbb{C}^{D_{n-1}})$.
	\end{formula}
	\begin{property}
		The map $\mathcal{E}_{\mathbbm{1}}^{(n)}$ associated to the transfer operator is a CP map\footnote{See definition \ref{operator:cp_map}.} and the associated Kraus operators are the MPS matrices $A^i(n)$.
	\end{property}

\subsection{Injectivity}\index{injective!MPS}

	For translation-invariant MPS one can use an easier definition:
	\newadef{Injective MPS}{
		A translation-invariant MPS is said to be injective if its transfer operator has a unique maximal eigenvalue.
	}
	
	In the next sections we will always assume the MPS to be injective unless stated otherwise.

\subsection{Gauge freedom and canonical forms}

	\begin{property}[Gauge freedom]
		As is clear from the construction of matrix product states there exists some freedom in the representation of the MPS tensors. One can always perform a transformation of the form $A(n)\rightarrow X^{-1}(n)A(n)X(n+1)$.
	\end{property}
	\begin{remark}
		If we use periodic boundary conditions then we must require that $X(L+1)=X(1)$ where $L$ is the lattice size.
	\end{remark}
	
	Using the gauge freedom in the representation of a generic MPS one can construct certain forms which have useful properties:
	\begin{construct}[Left canonical form\footnotemark]
		\footnotetext{Also called the \textbf{left isometric form}, \textbf{left orthogonal form} or just \textbf{left gauge}.}
		This form is specified by the following property:
		\begin{gather}
			\begin{tikzpicture}
				\node[rectangle,draw=black,minimum size=20] (A) at (0,1) {$A_L$};
				\node[rectangle,draw=black,minimum size=20] (Ad) at (0,-1) {$\overline{A_L}$};
				\draw (A) -- (Ad);
				\draw (A) to[out=200,in=160] (Ad);
				\draw (A) -- +(1,0);
				\draw (Ad) -- +(1,0);
				\node (E) at (1.5,0) {$=$};
				\node (A2) at (3,1) {};
				\node (Ad2) at (3,-1) {};
				\draw (A2) to[out=200,in=160] (Ad2);
			\end{tikzpicture}
		\end{gather}
		Any MPS can be brought in this form. First we construct the transfer operator $\mathbb{E}(n)$ for every site and find its maximal eigenvector. By the \textit{Perron-Frobenius theorem} this eigenvector (which is in fact a matrix itself) is positive and hence allows a decomposition of the form $\lambda(n)=L^\dag(n)L(n)$. The left orthogonal forms are then defined by
		\begin{gather}
			A_L(n) = L(n)A(n)L^{-1}(n+1)
		\end{gather}
	\end{construct}
	\sremark{In a similar manner one can construct the right orthogonal form $A_R$.}
	
	\begin{method}[Vidal]
		Given a general quantum state in terms of an $n$-leg tensor there exists an efficient way of constructing the left (or right) canonical forms introduced by Vidal \ref{VidalCanForm}. For this we 
	\end{method}
	
	\begin{construct}[Mixed canonical form]
		We can combine the left and right canonical forms. Let $L(n)$ and $R(n)$ be the decompositions of the left and right eigenvectors of the transfer operator at site $n$, i.e. $\lambda(n)=L^\dag(n)L(n)$ and $\rho(n)=R(n)R^\dag(n)$. The left and right canonical forms are then related by a matrix $C(n)$ in the following way: $A_L(n)C(n+1)=C(n)A_R(n)$. These matrices are given by
		\begin{gather}
			C(n)=L(n)R(n)
		\end{gather}
	\end{construct}
	
\subsection{Translation-invariant states}

	\newdef{Uniform MPS}{
		By setting all MPS tensors $A(n) = B$ for a given tensor $B$ one obtains a translation-invariant (TI) state, i.e. a state invariant under a shift of the index $n$. These MPS form the variational class of uniform MPS.
	}
	\begin{remark}[TIMPS]
		Not every TIMPS is uniform, there should only exist a local gauge transformation $A'(n) = U(n-1)A(n)U(n)^{-1}$ such that $A'(n)$ is uniform (in certain cases this is only possible by enlarging the bond dimension).
	\end{remark}


\section{Matrix product operators}

	\newdef{Matrix product operator\footnotemark}{\index{MPO}
		\footnotetext{As in the case of matrix product states we will abbreviate this as \textbf{MPO}.}
		Starting from the general form of an MPS one can easily construct more general objects. By replacing the rank-3 tensors $A^i(n)$ with rank-4 tensors $A^{i,j}(n)$ and $|i_1\cdots i_n\rangle$ by $|i_1\rangle\langle j_1|\otimes\cdots\otimes|i_n\rangle\langle j_n|$ one obtains the notion of a matrix product operator:
		\begin{gather}
			\hat{O} = \sum_{\{i_k,j_l\}}\text{tr}\Big(\prod_{m,n=1}^NO^{i_m,j_n}(n)\Big)|i_1\cdots i_N\rangle\langle j_1\cdots j_n|
		\end{gather}
		or in terms of a basis $\{\hat{O}_i\}$ for the space of local operators:
		\begin{gather}
			\hat{O} = \sum_{\{i_k\}}\text{tr}\Big(\prod_n^NA^{i_n}(n)\Big)\hat{O}_{i_n}
		\end{gather}
	}
	
	\begin{method}[Local Hamiltonian to MPO]
		Given a local Hamiltonian $\hat{H}=\sum_i\hat{H}^{(i)}$ one can build an MPO which generates this Hamiltonian\footnote{In fact one can use this procedure to turn any local operator into an MPO.}:
		\begin{gather}
			\hat{H} = \sum_{\{i_k,j_l\}}\text{tr}\Big(\prod_{m,n=1}^NA^{i_m,j_n}(n)\Big)|i_1\cdots i_N\rangle\langle j_1\cdots j_n|
		 \end{gather}
		 To obtain this MPO form one uses the concept of a cellular automaton. This is a set of possible states together with a set of rules that tell you how you can go from one state to another. To obtain the set of states in our case we look at a given site $i$. All distinct combinations of 1-site operators to the right of $i$ give rise to a dinstinct state $\mu$. The transition rules are obtained by looking at which operator can be placed at the site $i$ in a way consistent with the form of the given Hamiltonian.
	\end{method}
	\begin{example}
		Consider a 2-site Hamiltonian of the form \[\hat{A}\otimes\hat{B}\otimes\mathbbm{1}\otimes\cdots+\mathbbm{1}\otimes\hat{A}\otimes\hat{B}\otimes\mathbbm{1}\otimes\cdots+\cdots\] Looking at a specific site $i$ we obtain 3 distinct possibilities:
		\begin{enumerate}
			\item We have only identity operators acting to the right of $i$.
			\item Immediately to the right we have the operator $\hat{B}$ acting on $i+1$.
			\item Somewhere to the right we find the combination $\hat{A}\otimes\hat{B}$.
		 \end{enumerate}
		 The transition rules for this automaton are then given by the following list:
		 \begin{itemize}
		 	\item $1\rightarrow2$: $\mathbbm{1}$
		 	\item $1\rightarrow2$: $\hat{B}$
		 	\item $2\rightarrow3$: $\hat{A}$
		 	\item $3\rightarrow3$: $\mathbbm{1}$
		 \end{itemize}
		 What is useful for us is that this set of transition rules can be turned into a matrix: \[T=\begin{pmatrix}\mathbbm{1}&\hat{B}&0\\0&0&\hat{A}\\0&0&\mathbbm{1}\end{pmatrix}\] The MPO is then obtained by setting the MPO matrix $A$ equal to $T$ at every site.
	\end{example}
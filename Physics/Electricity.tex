\chapter{Electricity}

\section{Resistance \texorpdfstring{$R$}\ }
	\subsection{Conductivity}
    	\newdef{Drift velocity}{
        	The average speed of the independent charge carriers is the drift velocity $\vector{v_d}$. It is important to remark that $v_d$ is not equal to the propagation speed of the electric signal\footnote{It is several orders of magnitude smaller.}.
        }
        
        \newformula{Mobility}{
        	\begin{equation}
                	\label{electricity:mobility}
                	\mu = \stylefrac{v_d}{E}
		\end{equation}
	}
        \newformula{Conductivity}{
        	\begin{equation}
                	\label{electricity:conductivity}
                	\sigma = nq\mu
		\end{equation}
	}
        \newformula{Resistivity}{
		\begin{equation}
			\label{electricity:resistivity}
			\rho = \stylefrac{1}{\sigma}
		\end{equation}
        }

\subsection{Current density}

    	\begin{formula}
		Let $A$ be the cross-section of a conductor. Let $\vector{J}$ be the current density through $A$. The current through $A$ is then given by:
		\begin{equation}
			\label{electricity:current_density}
			I = \iint_A\vector{J}\cdot\hat{\mathbf{n}}dS
		\end{equation}
	\end{formula}
        \newformula{Free current}{
		The current density generated by free charges is given by:
		\begin{equation}
			\label{electricity:free_current_density}
			\vector{J} = nq\vector{v}_d
		\end{equation}
        }

\subsection{Pouillet's law}

    	\begin{equation}\index{Pouillet}
		\label{electricity:pouillet}
		R = \stylefrac{l}{A}\rho
	\end{equation}
        where:
        \begin{itemize}
        	\item $\rho$: resistivity of the material
		\item $l$: length of the resistor
		\item $A$: cross-sectional area
        \end{itemize}


\section{Ohm's law}

	\newformula{Ohm's law}{\index{Ohm}
		\begin{equation}
			\label{electricity:ohms_law}
			\boxed{\vector{J} = \sigma\cdot\vector{E}}
		\end{equation}
	        where $\sigma$ is the conductivity tensor.
	}
    
	\newformula{Ohm's law in wires}{
    		The following formula can be found by combining equations \ref{electricity:conductivity}, \ref{electricity:resistivity},\ref{electricity:current_density} and \ref{electricity:ohms_law} and by assuming that the conductivity tensor is a scalar (follows from the isotropic behaviour of common resistors):
		\begin{equation}
			\label{electricity:ohms_law_linear}
			U = RI
		\end{equation}
	}


\section{Capacitance \texorpdfstring{$C$}\ }

	\newdef{Capacitance}{\index{capacitance}
	    	The capacitance is a (geometrical) value that reflects the amount of charge an object can store:
	    	\begin{equation}
			\label{electricity:capacitance}
			C = \stylefrac{q}{V}
		\end{equation}
	}
        
\section{Electric dipole}

	\begin{formula}[Electric dipole]\index{dipole}
		\begin{equation}
			\label{electricity:dipole}
			\vector{p} = q\vector{a}
		\end{equation}
		where:
		\begin{itemize}
			\item $q$: charge of the positive particle
			\item $\vector{a}$: vector pointing from the negative to the positive particle
		\end{itemize}
	\end{formula}

	\begin{formula}[Energy]
	    	If an electric dipole is placed in an electric field, its potential energy is:
		\begin{equation}
			\label{electricity:dipole_energy}
			U = -\vector{p}\cdot\vector{E}
		\end{equation}
	\end{formula}

	\begin{formula}[Torque]
    		If an electric dipole is placed in an electric field, a torque is generated:
		\begin{equation}
			\label{electricity:dipole_torque}
			\vector{\boldsymbol{\tau}} = \vector{p}\times\vector{E}
		\end{equation}
	\end{formula}

\chapter{Electricity}

\section{Resistance \texorpdfstring{$R$}\ }
	\subsection{Conductivity}
    	\newdef{Drift velocity}{
        	The average speed of the independent charge carriers is the drift velocity $\vector{v_d}$. It is important to remark that $v_d$ is not equal to the propagation speed of the electricity\footnotemark. 
        }
        \footnotetext{It is several orders of magnitude smaller.}
        
        \newdef{Conductivity}{
        	\begin{equation}
                \label{electricity:conductivity}
                \text{conductivity: }\sigma = nq\mu
			\end{equation}
		}
        \newformula{Resistivity}{
            \begin{equation}
                \label{electricity:resistivity}
                \text{resistivity: }\rho = \stylefrac{1}{\sigma}
            \end{equation}
        }
        \newformula{Mobility}{
            \begin{equation}
                \label{electricity:mobility}
                \text{mobility: }\mu = \stylefrac{v_d}{E}
			\end{equation}
		}

	\subsection{Current density}
    	\begin{formula}
			Let $A$ be the cross section of a conductor. Let $\vector{J}$ be the current density though $A$. The current trough $A$ is then given by:
            \begin{equation}
				\label{electricity:current_density}
                I = \iint_A\vector{J}\cdot\hat{n}dS
			\end{equation}
		\end{formula}
        \newformula{Free current}{
        	The current density generated by free charges is given by:
            \begin{equation}
				\label{electricity:free_current_density}
                \vector{J} = nq\vector{v}_d
			\end{equation}
        }

	\subsection{Pouillet's law}
    	\begin{equation}
			\label{electricity:pouillet}
            R = \rho\ \stylefrac{l}{A}
		\end{equation}
        where:\begin{flalign*}
			\qquad&\rho:\text{resistivity of the material}&\\
            &l:\text{length of the resistor}&\\
            &A:\text{cross-sectional area}&
		\end{flalign*}

\section{Ohm's law}
    \newformula{Ohm's law}{
		\begin{equation}
			\label{electricity:ohms_law}
            \boxed{\vector{J} = \overset{\leftrightarrow}{\sigma}\cdot\vector{E}}
		\end{equation}
        where $\overset{\leftrightarrow}{\sigma}$ is the conductivity tensor.
	}
    
    \newformula{Ohm's law in wires}{
    	The following formula can be found by combining equations \ref{electricity:conductivity}, \ref{electricity:resistivity},\ref{electricity:current_density} and \ref{electricity:ohms_law} and by assuming that the conductivity tensor can be simplified to a scalar (follows from the isotropic behaviour of normal resistors):
		\begin{equation}
			\label{electricity:ohms_law_linear}
            U = RI
		\end{equation}
	}


\section{Capacitance \texorpdfstring{$C$}\ }
	\newdef{Capacitance}{
    	The capacitance is a (geometrical) value that reflects the amount of charge a certain body can store. 
    	\begin{equation}
			\label{electricity:capacitance}
            C = \stylefrac{q}{V}
		\end{equation}
    }
        
\section{Electric dipole}
	\begin{formula}[Electric dipole]
		\begin{equation}
			\label{electricity:dipole}
            \vector{p} = q\vector{a}
		\end{equation}
		Where:\begin{flalign*}
			\qquad&q:\text{charge of the positive particle}&\\
            &\vector{a}\ :\text{vector pointing from the negative to the positive particle}&\\
		\end{flalign*}
	\end{formula}
    \begin{formula}[Energy]
    	If an electric dipole is placed in an electric field, its potential energy is:
		\begin{equation}
			\label{electricity:dipole_energy}
            U = -\vector{p}\cdot\vector{E}
		\end{equation}
	\end{formula}
    \begin{formula}[Torque]
    	If an electric dipole is placed in an electric field, a torque is generated:
		\begin{equation}
			\label{electricity:dipole_torque}
            \vector{\boldsymbol{\tau}} = \vector{p}\times\vector{E}
		\end{equation}
	\end{formula}
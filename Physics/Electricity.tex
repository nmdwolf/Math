\chapter{Electricity and Magnetism}

\section{Resistance \texorpdfstring{$R$}{R}}

    \newdef{Drift velocity}{
        The average speed of the independent charge carriers is the drift velocity $\vector{v_d}$. It is important to remark that $v_d$ is not equal to the propagation speed of the electric signal\footnote{It is several orders of magnitude smaller.}.
    }

    \newformula{Mobility}{\label{electricity:mobility}
        \begin{gather}
            \mu := \stylefrac{v_d}{E}
        \end{gather}
    }
    \newformula{Conductivity}{\label{electricity:conductivity}
        \begin{gather}
            \sigma := nq\mu
        \end{gather}
    }
    \newformula{Resistivity}{\label{electricity:resistivity}
        \begin{gather}
            \rho := \stylefrac{1}{\sigma}
        \end{gather}
    }

    \begin{formula}[Pouillet's law]\index{Pouillet}
        \begin{gather}\index{Pouillet}
            \label{electricity:pouillet}
            R = \stylefrac{l}{A}\rho
        \end{gather}
        where:
        \begin{itemize}
            \item $\rho$ is the resistivity of the material
            \item $l$ is the length of the resistor
            \item $A$ is the cross-sectional area of the resistor.
        \end{itemize}
    \end{formula}

\section{Ohm's law}

    \begin{formula}
        Let $A$ be the cross-section of a conductor. Let $\vector{J}$ be the current density through $A$. The current through $A$ is then given by
        \begin{gather}
            \label{electricity:current_density}
            I = \iint_A\vector{J}\cdot\hat{\mathbf{n}}\ dS.
        \end{gather}
    \end{formula}
    \newformula{Free current}{
        The current density generated by free charges is given by
        \begin{gather}
            \label{electricity:free_current_density}
            \vector{J} = nq\vector{v}_d.
        \end{gather}
    }

    \newformula{Ohm's law}{\index{Ohm}\label{electricity:ohms_law}
        \begin{gather}
            \vector{J} = \sigma\cdot\vector{E}
        \end{gather}
        where $\sigma$ is the conductivity tensor.
    }

    \newformula{Ohm's law in wires}{
        The following formula can be found by combining equations \ref{electricity:conductivity}, \ref{electricity:resistivity},\ref{electricity:current_density} and \ref{electricity:ohms_law} and by assuming that the conductivity tensor is a scalar (this follows from the isotropic behaviour of common resistors):
        \begin{gather}
            \label{electricity:ohms_law_linear}
            U = RI.
        \end{gather}
    }

\section{Capacitance \texorpdfstring{$C$}{C}}

    \newdef{Capacitance}{\index{capacitance}\label{electricity:capacitance}
        The capacitance is a (geometrical) value that reflects the amount of charge an object can store:
        \begin{gather}
            C := \stylefrac{q}{V}
        \end{gather}
    }

\section{Electric dipoles}

    \begin{formula}[Electric dipole]\index{dipole}\label{electricity:dipole}
        \begin{gather}
            \vector{p} := q\vector{a}
        \end{gather}
        where:
        \begin{itemize}
            \item $q$ is the charge of the positive particle
            \item $\vector{a}$ is the vector pointing from the negative to the positive particle.
        \end{itemize}
    \end{formula}

    \begin{formula}[Energy]
        If an electric dipole is placed in an electric field, its potential energy is given by
        \begin{gather}
            \label{electricity:dipole_energy}
            U = -\vector{p}\cdot\vector{E}.
        \end{gather}
    \end{formula}

    \begin{formula}[Torque]
        If an electric dipole is placed in an electric field, the torque on this system is given by
        \begin{gather}
            \label{electricity:dipole_torque}
            \vector{\boldsymbol{\tau}} = \vector{p}\times\vector{E}.
        \end{gather}
    \end{formula}

\section{Magnetic fields}\index{magnetization}\index{induction}

    The \textbf{magnetizing field} $\vector{H}$ is the field generated by all external sources. When applying an external (magnetic) field, some materials will try to oppose this external influence. Similar to polarization in the case of electricity, one can define the \textbf{magnetization}:
    \begin{gather}
        \label{magnetism:M}
        \vector{M} := \chi\vector{H}
    \end{gather}
    where $\chi$ is the magnetic susceptibility.

    The \textbf{magnetic induction} $\vector{B}$ is the field generated by both the external sources and the internal magnetization. It is only this field that one can measure. In vacuum we have the following relation between the magnetic induction, the magnetizing field and the magnetization:
    \begin{gather}
        \label{magnetism:B}
        \vector{B} = \mu_0\left(\vector{H} + \vector{M}\right).
    \end{gather}
    By combining the previous two formulas we obtain\footnote{This equation is only valid in linear media.}
    \begin{gather}
        \label{magnetism:B_with_only_H}
        \vector{B} = \mu_0\left(1 + \chi\right)\vector{H}.
    \end{gather}
    The proportionality constant in formula \ref{magnetism:B_with_only_H} is called the \textbf{magnetic permeability}:
    \begin{gather}
        \label{magnetism:relative_permeability}
        \mu := \mu_0(1 + \chi)
    \end{gather}
    where $\mu_0$ is the magnetic permeability of the vacuum. The factor $1+\chi$ is called the \textbf{relative permeability} and it is often denoted by $\mu_r$.

    \begin{remark}[Tensorial formulation]
        In anisotropic materials we have to use a tensorial formulation:
        \begin{align}
            \label{magnetism:B_tensor}
            B_i &= \sum_j\mu_{ij}H_j\\
            \label{magnetism:M_tensor}
            M_i &= \sum_j\chi_{ij}H_j
        \end{align}
        Both $\mu$ and $\chi$ are tensors of rank 2.
    \end{remark}

\subsection{Electric charges in a magnetic field}

    \newformula{Gyroradius}{
        \begin{gather}
            \label{magnetism:gyroradius}
            r = \stylefrac{mv_{\perp}}{|q|B}
        \end{gather}
    }
    \newformula{Gyrofrequency\footnotemark}{\index{cyclotron}\index{Larmor frequency}
        \footnotetext{Also called the \textbf{Larmor} or \textbf{cyclotron frequency}.}
        \begin{gather}
            \label{magnetism:gyrofrequency}
            \omega = \stylefrac{|q|B}{m}
        \end{gather}
    }
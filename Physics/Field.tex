\chapter{Classical Field Theory}\label{chapter:classical_fields}

	Rigorous definitions and statements about the mathematical concepts used in this chapter can be found in chapters \ref{chapter:bundles}, \ref{chapter:vector_bundles}, \ref{chapter:riemann} and \ref{chapter:variation}.

\section{Lagrangian field theory}

	We will work on a (pseudo)Riemannian manifold $(M, g)$ and the fields will be sections of a fibre bundle $E\rightarrow M$ (mostly this will be the tangent bundle $TM$). In general a Lagrangian function is a mapping $L:J^\infty(E)\rightarrow\mathbb{R}$ from the infinity jet bundle to the real numbers.\footnote{In fact one uses the composition of a mapping $\Gamma(J^\infty E)\rightarrow\Omega^n(M)$ and integration of this form. So the Lagrangian will be an $n$-form obtained by multiplying the volume form (technically a tensor density) by a suitable (scalar) Lagrangian density.} However if our theory is sufficiently local, i.e. the Lagrangian depends only on a finite number of derivatives, then one can restrict the configuration space to a finite-dimensional jet bundle $J^k(E)$. (This is much cleaner than working with the infinite-dimensional space of sections $\Gamma(E)$).

	Associated to this manifold one can construct a cochain complex similar to the de Rham complex $\Omega^\bullet(M)$. This structure takes two geometric features into account. On one hand one has the ordinary de Rham differential $d$ on the base manifold $M$, while on the other hand one has a differential $\delta$ along the jet fibres, induced by the variation of fields. The total differential will be the sum of these as is standard in the context of bicomplexes. By taking the differentials to be anticommuting we obtain the variational bicomplex.

	We quickly recall some key concepts from the calculus of variations. The variational derivative or Euler-Lagrange derivative is defined as follows:
	\begin{gather}
		\frac{\delta L}{\delta \phi} = \pderiv{L}{\phi} - \partial_\mu\left(\pderiv{L}{\partial_\mu\phi}\right) + \partial_\mu\partial_\nu\left(\pderiv{L}{\partial_\mu\partial_\nu\phi}\right) -\cdots
	\end{gather}
	By comparing this formula to the formula for the variation of the Lagrangian $dL$ we obtain the following formula:
	\begin{gather}
		\delta L = \frac{\delta L}{\delta \phi^I}\partial\phi^I - d\Theta[\phi]
	\end{gather}
	The first term vanishes on-shell, $\frac{\delta L}{\delta \phi^I}$ is exactly the EL-equation associated to the field $\phi I$. The last term contains the boundary terms obtained by performing integration by parts. The $(n-1, 1)$-form $\Theta$ is called the \textbf{presymplectic potential}. The \textbf{presymplectic current}\footnote{In general this form is not symplectic.} $\omega$ is obtained by taking the variation:
	\begin{gather}
		\omega[\phi] = \delta\Theta[\phi]
	\end{gather}
	On-shell this current is closed, i.e. $d\omega\approx0$, and furthermore it can be shown that if the variations $\delta\phi^I$ satisfy the linearized equations of motion then for every gauge transformation $\xi$ there exists a $(n-2, 1)$-form $k_\xi[\phi]$ such that $\omega\approx dk[\phi]$. (More details can be found in \cite{compere}.)

\section{Covariant phase space}

	We will first introduce the formalism, originally developped by Peierls, in the absence of local gauge symmetries. Since the classical notion of phase space as the set of coordinate-momentum $(q,p)$ points at a given time $t$ is clearly not covariant (the choice of a time slice ruins any form of relativistic invariance) we have to embrace a new approach:
	\newdef{Covariant phase space}{\index{phase space}
		Let $S[\phi]$ be a local action functional. The covariant phase space $\mathcal{P}$ associated to $S$ is the set of solutions of the equations of motion $\frac{\delta S}{\delta \phi}=0$.
	}
	The physical observables are then defined as the smooth functions on this new phase space $\mathcal{P}$. These can be described more generally. Let $M$ be the set of all field histories (the covariant phase space is a submanifold of this space). The ring of physical observables $C^\infty(\mathcal{P})$ is then obtained as the quotient of $C^\infty(M)$ by the ideal of functions vanishing on-shell.
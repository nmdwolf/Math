\chapter{\difficult{Classical Field Theory}}\label{chapter:classical_fields}

    Rigorous definitions and statements about the mathematical concepts used in this chapter can be found in chapters \ref{chapter:bundles}, \ref{chapter:vector_bundles}, \ref{chapter:riemann} and \ref{chapter:variation}.

\section{Lagrangian field theory}

    We will work on a (pseudo-)Riemannian manifold $(M, g)$ and the fields will be sections of a fibre bundle $E\rightarrow M$ (this will mostly be the tangent bundle $TM$). In general a Lagrangian function is a mapping $L:J^\infty(E)\rightarrow\mathbb{R}$ from the infinity-jet bundle to the real numbers.\footnote{In fact one uses the composition of a mapping $\Gamma(J^\infty E)\rightarrow\Omega^n(M)$ and integration of this form. So the Lagrangian will be an $n$-form obtained by multiplying the volume form by a suitable (scalar) Lagrangian (density).} However, if our theory is sufficiently local, i.e. if the Lagrangian depends only on a finite number of derivatives, then one can restrict the configuration space to a finite-dimensional jet bundle $J^k(E)$. (This is much cleaner than working with the infinite-dimensional space of sections $\Gamma(E)$).

    Associated to this manifold one can construct a cochain complex similar to the de Rham complex $\Omega^\bullet(M)$. This structure takes two geometric features into account. On the one hand one has the ordinary de Rham differential $d$ on the base manifold $M$, while on the other hand one has a differential $\delta$ along the jet fibres, induced by the variation of fields. The total differential will be the sum of these as is standard in the context of bicomplexes. This defines the variational bicomplex.

    We quickly recall some key concepts from the calculus of variations. The variational derivative or Euler-Lagrange derivative is defined as follows:
    \begin{gather}
        \frac{\delta L}{\delta \phi} := \pderiv{L}{\phi} - \partial_\mu\left(\pderiv{L}{\partial_\mu\phi}\right) + \partial_\mu\partial_\nu\left(\pderiv{L}{\partial_\mu\partial_\nu\phi}\right) -\cdots.
    \end{gather}
    By comparing this formula to the formula for the variation of the Lagrangian $dL$ we obtain the following expression:
    \begin{gather}
        \delta L = \frac{\delta L}{\delta \phi^I}\delta\phi^I - d\Theta[\phi].
    \end{gather}
    The first term vanishes on-shell, $\frac{\delta L}{\delta \phi^I}$ is exactly the EL-equation associated to the field $\phi^I$. The last term contains the boundary terms obtained after performing integration by parts. The $(n-1, 1)$-form $\Theta$ is called the \textbf{presymplectic potential}. The \textbf{presymplectic current}\footnote{In general this form is not symplectic.} $\omega$ is obtained by taking the variation:\index{potential!presymplectic}\index{current!presymplectic}
    \begin{gather}
        \omega[\phi] = \delta\Theta[\phi].
    \end{gather}
    On-shell this current is closed, i.e. $d\omega\approx0$, and moreover it can be shown that if the variations $\delta\phi^I$ satisfy the linearised equations of motion, then for every gauge transformation $\xi$ there exists a $(n-2, 1)$-form $k_\xi[\phi]$ such that $\omega\approx dk[\phi]$. (More details can be found in \cite{compere}.)

\subsection{Noether's theorem for fields}\index{Noether}

    \begin{theorem}[Noether's first theorem$^\dag$]\index{Noether!charge}\label{qft:noethers_theorem}
        Consider a field transformation
        \begin{gather}
            \label{qft:noether}
            \phi(x)\rightarrow \phi(x) + \alpha\delta\phi(x)
        \end{gather}
        where $\alpha$ is an infinitesimal quantity and $\delta\phi$ is a small variation. In case of a symmetry we obtain the following conservation law:
        \begin{gather}
            \label{qft:conserved_current}
            \partial_\mu\left(\pderiv{\mathcal{L}}{(\partial_\mu\phi)}\delta\phi - \mathcal{J}^\mu\right) = 0.
        \end{gather}
        The factor between parentheses can be interpreted as a conserved current $j^\mu(x)$. Noether's (first) theorem states that every symmetry of the form \ref{qft:noether} leads to such a current.

        The conservation can also be expressed in terms of a charge\footnote{The conserved current and its associated charge are called the \textbf{Noether current} and \textbf{Noether charge}.}:
        \begin{gather}
            Q = \int_\Sigma j^0d^3x
        \end{gather}
        where $\Sigma$ is a spacelike hypersurface. The conservation law can then simply be restated as \[\deriv{Q}{t} = 0.\]
    \end{theorem}

    \begin{definition}[Stress-energy tensor]\index{stress-energy tensor}
        Consider a field transformation \[\phi(x)\rightarrow\phi(x+a) = \phi(x) + a^\mu\partial_\mu\phi(x)\] Because the Lagrangian is a scalar quantity it transforms as
        \begin{gather}
            \mathcal{L}\rightarrow\mathcal{L} + a^\mu\partial_\mu\mathcal{L} = \mathcal{L} + a^\nu\partial_\mu(\delta^\mu_{\ \nu}\mathcal{L}).
        \end{gather}
        This leads to the existence of 4 conserved currents. These can be used to define the stress-energy tensor:
        \begin{gather}
            \label{relativity:stress_energy_tensor}
            T^\mu_{\ \nu} = \pderiv{\mathcal{L}}{(\partial_\mu\phi)}\partial_\nu\phi - \mathcal{L}\delta^\mu_{\ \nu}.
        \end{gather}
    \end{definition}

\section{Covariant phase space}

    We will first introduce the formalism, originally developed by \textit{Peierls}, in the absence of local gauge symmetries. Since the classic notion of phase space as the set of $(q,p)$-points in coordinate-momentum space, at a given time $t$, is clearly not covariant (the choice of a time slice ruins any form of relativistic invariance) we have to embrace a new approach:
    \newdef{Covariant phase space}{\index{phase space}
        Let $S[\phi]$ be a local action functional. The covariant phase space $\mathcal{P}$ associated to $S$ is the set of solutions of the equations of motion
        \begin{gather}
            \frac{\delta S}{\delta \phi}=0.
        \end{gather}
    }
    The physical observables are then defined as the smooth functions on this new phase space $\mathcal{P}$. These can be described more generally: Let $M$ be the set of all field histories/configurations (the covariant phase space is a submanifold of this space). The ring of physical observables $C^\infty(\mathcal{P})$ is then obtained as the quotient of $C^\infty(M)$ by the ideal of functions vanishing on-shell.

    ?? COMPLETE ??
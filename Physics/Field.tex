\chapter{\difficult{Classical Field Theory}}\label{chapter:classical_fields}

    Rigorous definitions and statements about the mathematical concepts used in this chapter can be found in chapters \ref{chapter:bundles}, \ref{chapter:vector_bundles}, \ref{chapter:riemann} and \ref{chapter:variation}.

\section{Lagrangian field theory}

    The physical space will be assumed to be a (pseudo-)Riemannian manifold $(M,g)$ with the fields being sections of a fibre bundle $E\rightarrow M$ (often the tangent bundle $TM$). In general a Lagrangian (function) is a mapping $L:J^\infty(E)\rightarrow\mathbb{R}$ from the infinity-jet bundle to the real numbers. This is often given by a mapping $\Gamma(J^\infty E)\rightarrow\Omega^n(M)$ followed by integration of the top-dimensional form over $M$ (when expressed as $L\text{Vol}_g$ the function is $L$ is called the \textbf{Lagrangian density}). However, when the theory is sufficiently local, i.e. if the Lagrangian depends only on a finite number of derivatives, one can restrict the configuration space to a finite-dimensional jet bundle $J^k(E)$.

    Associated to this manifold one can construct a cochain complex similar to the de Rham complex $\Omega^\bullet(M)$. This structure takes two geometric features into account. On the one hand one has the ordinary de Rham differential $d$ on the base manifold $M$, while on the other hand one has a differential $\delta$ along the jet fibres, induced by the variation of fields. The total differential will be the sum of these as is standard in the context of bicomplexes. This defines the variational bicomplex (see section \ref{section:variational_bicomplex}).

    We quickly recall some key concepts from the calculus of variations. The variational derivative or Euler-Lagrange derivative \ref{var:euler_lagrange_derivative} is defined as follows (partial derivatives are denoted by subscripted commas, e.g. $\partial_\mu\partial_\nu\phi\equiv\phi_{,\mu\nu}$):
    \begin{gather}
        \frac{\delta L}{\delta \phi} := \pderiv{L}{\phi} - \partial_\mu\left(\pderiv{L}{\phi_{,\mu}}\right) + \partial_\mu\partial_\nu\left(\pderiv{L}{\phi_{,\mu\nu}}\right) -\cdots.
    \end{gather}
    By comparing this formula to the formula for the variation of the Lagrangian density one obtains the following expression (the first variational formula \ref{var:first_variational_formula}):
    \begin{gather}
        \delta L = \frac{\delta L}{\delta \phi^I}\delta\phi^I - d\Theta[\phi].
    \end{gather}
    The first term vanishes on-shell because it is proportional to the Euler-Lagrange equation associated to the field $\phi^I$. The last term contains the boundary terms obtained after performing integration by parts. The $(n-1,1)$-form $\Theta$ is called the \textbf{presymplectic potential}.

    The \textbf{presymplectic current}\footnote{In general this form is not symplectic.} $\omega$ is obtained by taking the variation of the presymplectic potential:\index{potential!presymplectic}\index{current!presymplectic}
    \begin{gather}
        \omega[\phi] = \delta\Theta[\phi].
    \end{gather}
    On-shell this form is closed, i.e. $d\omega\approx0$. It can be shown that if the variations $\delta\phi^I$ satisfy the linearised equations of motion, for every gauge transformation $\xi$ there exists a $(n-2, 1)$-form $k_\xi[\phi]$ such that $\omega\approx dk[\phi]$. (More details can be found in \cite{compere}.)

\subsection{Noether's theorem for fields}\index{Noether!theorem}

    In the context of field theory the Lagrangian density is often denoted by $\mathcal{L}$ instead of $L$. This convention will be adopted here as well.

    \begin{theorem}[Noether's first theorem$^\dag$]\index{Noether!charge}\label{qft:noethers_theorem}
        Consider a general field transformation
        \begin{gather}
            \label{qft:noether}
            \phi\rightarrow\phi+\alpha\delta\phi
        \end{gather}
        where $\alpha$ is an infinitesimal quantity and $\delta\phi$ is a small variation. In case of a symmetry one obtains a conservation law of the following form:
        \begin{gather}
            \label{qft:conserved_current}
            \partial_\mu\left(\pderiv{\mathcal{L}}{\phi_{,\mu}}\delta\phi - \mathcal{J}^\mu\right) = 0.
        \end{gather}
        The factor between parentheses can be interpreted as a conserved current $j^\mu(x)$.
    \end{theorem}
    The above conservation law can also be expressed in terms of a charge (such a current and its associated charge are generally called the \textbf{Noether current} and \textbf{Noether charge}):
    \begin{gather}
        Q[\Sigma] := \int_\Sigma j^0d^3x,
    \end{gather}
    where $\Sigma$ is a spacelike hypersurface. The conservation law can then simply be restated as \[\deriv{Q}{t} = 0.\]

    \begin{definition}[Stress-energy tensor]\index{stress-energy tensor}
        Consider the translation of a scalar field: \[\phi(x)\rightarrow\phi(x+a) = \phi(x) + a^\mu\partial_\mu\phi(x).\] Because the Lagrangian is a scalar quantity, it transforms in the same way as the fields:
        \begin{gather}
            \mathcal{L}\rightarrow\mathcal{L} + a^\mu\partial_\mu\mathcal{L} = \mathcal{L} + a^\nu\partial_\mu(\mathcal{L}\delta^\mu_{\ \nu}).
        \end{gather}
        On a $D$-dimensional manifold this leads to the existence of $D$ conserved currents. These can be used to define the stress-energy tensor:
        \begin{gather}
            \label{relativity:stress_energy_tensor}
            T^\mu_{\ \nu} = \pderiv{\mathcal{L}}{\phi_{,\mu}}\partial_\nu\phi - \mathcal{L}\delta^\mu_{\ \nu}.
        \end{gather}
    \end{definition}

\section{Covariant phase space}

    First the formalism for classical mechanical systems will be introduced, i.e. no fields will be considered. Then the formalism, originally developed by \textit{Peierls}, will be introduced in the absence of local gauge symmetries.

\subsection{Zero Hamiltonian}

    In Chapter \ref{chapter:constrained_dynamics} dynamical systems with constraints were considered. Using the tools from that chapter one can turn any system evolving under a physical, but nondynamical or external, time parameter $t$ into a system having time as a canonical parameter. In this setting the time variable is treated on the same footing as the other coordinates. Such a system is called a \textbf{generally covariant system}.

    If one starts from the action
    \begin{gather}
        S[q,p] = \int \left(p_i\dot{q}^i-H_0\right)dt,
    \end{gather}
    one can introduce time as a generalized coordinate with momentum $p_0$ by modifying the action as follows:
    \begin{gather}
        S[q,p,t,p_0,u] = \int \left[p_0t'+p_iq'^i-u(p_0+H_0)\right]d\tau,
    \end{gather}
    where the quotes indicate derivatives with respect to the parameter $\tau$. It is easily checked that the resulting equations of motion are the same as for the original action.

    The system now involves a single constraint $H_0=p_0$, which is first-class. It is often called the \textbf{Hamiltonian constraint}.\index{Hamilton!constraint} Aside from this constraint, the extended action contains no first-class Hamiltonian. So evolution is solely determined by a constraint and therefore is given by a gauge transformation.

    \begin{remark}[Nonholonomic constraints]
        In Chapter \ref{chapter:constrained_dynamics} all constraints were assumed to be holonomic, i.e. they did not explicitly depend on time. The presence of time derivatives is not compatible with the Poisson/Dirac bracket. However, when passing to a generally covariant system as above, the time variable loses its peculiar character and all constraints can be handled in the same way.
    \end{remark}

    \begin{property}[Vanishing Hamiltonian]
        If the canonical coordinates $(q,p)$ transform as scalars under $\tau$-reparametrizations, the Hamiltonian is weakly zero.
    \end{property}

\subsection{Field theory}

    Since the classical notion of phase space as the set of $(q,p)$-points in coordinate-momentum space, at a given time $t$, is clearly not covariant (the choice of a time slice ruins any form of relativistic invariance) one has to embrace a new approach:
    \newdef{Covariant phase space}{\index{phase space}
        Let $S[\phi]$ be a local action functional. The covariant phase space $\mathcal{P}$ associated to $S$ is the set of solutions of the equations of motion
        \begin{gather}
            \frac{\delta S}{\delta \phi}=0.
        \end{gather}
        The physical observables are then defined as the smooth functions on this new phase space $\mathcal{P}$. These can be described more generally: Let $M$ be the set of all field histories/configurations (the covariant phase space is a submanifold of this space). The ring of physical observables $C^\infty(\mathcal{P})$ is then obtained as the quotient of $C^\infty(M)$ by the ideal of functions vanishing on-shell.
    }

    ?? COMPLETE ??

\section{BRST quantization}
\subsection{Homological interpretation}

    Here the content of Section \ref{section:classical_brst} will be further formalized using the language of homological algebra (see Chapter \ref{chapter:hom_alg}). This will allow for extensions in the context of quantum field theory, especially when considering systems on curved backgrounds.

    Recall that a constrained system is a dynamical system $(M,\omega,H)$ together with a set of constraint equations $\phi^m(q,p)=0$. These define a constraint surface $\Sigma$ and the algebra of physically relevant functions can be at most the quotient $C^\infty(\Sigma):=C^\infty(M)/\mathcal{N}$ where the ideal $\mathcal{N}$ is generated by the constraint equations. In fact the constraint surface contains redundant information since it is foliated by the gauge orbits of the canonical transformations generated by the constraints. The classical observables are therefore given by the elements of $C^\infty(\Sigma)$ that are constant along these orbits. In Section \ref{section:classical_brst} these functions were constructed in terms of a BRST operator and its associated (co)homology.

    Now, regard $C^\infty(M)$ as the degree-0 part of a chain complex and define the differential $\delta$ on $C^\infty(M)$ by
    \begin{gather}
        \delta z^i := 0
    \end{gather}
    for all phase space variables $z^i$. The requirement that $\delta$ acts as a derivation implies $\ker(\delta)=C^\infty(M)$. The Koszul-Tate resolution \ref{homalg:koszul_tate_resolution} of this complex with respect to the ideal $\mathcal{N}$ is easily seen to satisfy $H_0(\delta)\cong C^\infty(\Sigma)$. The degree-1 generators that are added in this construction are exactly the ghost momenta $\overline{P}_m$. In this setting the degree is also called the \textbf{antighost number}.

    ?? COMPLETE (\cite{henneaux_teitelboim}) ??
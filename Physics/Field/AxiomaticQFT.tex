\chapter{\difficult{Axiomatic QFT}}

    For the sections on the Haag--Kastler framework and its extensions, the reader is referred to the work of \textit{Brunetti, Fredenhagen et al}. A reference for the remaining sections on algebraic QFT is~\citet{baez_introduction_2014}. The sections on cohomological quantization and quantization from a sheaf point of view are based on~\citet{nuiten_cohomological_2013}. For the sections on TQFTs and \textit{open-closed} TQFTs, the reader is referred to the original papers~\citet{atiyah_topological_1988,lauda_openclosed_2008}, respectively.

\section{Algebraic QFT}
\subsection{Haag--Kastler axioms}\label{section:haag_kastler}

    \begin{axiom}[Local net of observables]\index{microcausality|seealso{locality}}\index{Einstein!causality}\index{locality}\label{aqft:microcausality}
        To every causally closed set (\cref{relativity:causal_closure}) one associates a $C^*$-algebra. This assignment is required to satisfy the following conditions:
        \begin{enumerate}
            \item\textbf{Isotony}: If $O_1\subset O_2$, then $\mathcal{A}(O_1)\hookrightarrow\mathcal{A}(O_2)$.
            \item\textbf{(Causal) locality}\footnote{Also called \textbf{microcausality} or \textbf{Einstein causality}.}: If $O_1$ and $O_2$ are spacelike separated, then $\bigl[\mathcal{A}(O_1),\mathcal{A}(O_2)\bigr]=0$ (as a graded commutator) within a larger algebra $\mathcal{A}(O)$ such that $O_1,O_2\subset O$.
        \end{enumerate}
    \end{axiom}
    \remark{The isotony condition implies that local nets of observables are modelled by copresheaves $\mathbf{Mink}\rightarrow\mathbf{C^*Alg}$ that map (mono)morphisms to monomorphisms.}

    \begin{axiom}[Poincar\'e covariance]
        For all causally closed sets $O$ and Poincar\'e transformations $\Lambda$ there exists an isomorphism $\alpha^O_\Lambda:\mathcal{A}(O)\rightarrow\mathcal{A}(\Lambda O)$ such that the following conditions are satisfied:
        \begin{enumerate}
            \item If $O_1\subset O_2$, then $\alpha_\Lambda\circ\iota_{O_1,O_2} = \iota_{\Lambda O_1,\Lambda O_2}\circ\alpha_\Lambda$.
            \item The isomorphisms satisfy a composition rule: $\alpha^{\Lambda O}_{\Lambda'}\circ\alpha^O_\Lambda = \alpha^O_{\Lambda'\Lambda}$.
        \end{enumerate}
    \end{axiom}

    \begin{axiom}[Spectrum]
        For all spacetime regions $O$ one can construct a faithful $C^*$-algebra representation $\rho_O$ of $\mathcal{A}(O)$ on a fixed Hilbert space by the GNS construction (\cref{operators:gns}). The different representations should be compatible, i.e.~if $O_1\subset O_2$, the restriction of $\rho_{O_2}$ to $\mathcal{A}(O_1)$ should equal $\rho_{O_1}$. Furthermore, all spacetime translations are implemented unitarily:
        \begin{gather}
            U(a)\rho_O(c)U(a)^{-1} = \rho_{O+a}\left(\alpha^O_a(c)\right)
        \end{gather}
        for all $c\in\mathcal{A}(O)$, where $U$ is a unitary representation of the translation subgroup. In addition the generators of the translation subgroup are required to have a spectrum that is contained in the future light cone.
    \end{axiom}

    The following axiom is not part of the standard Haag--Kastler framework but can be added to introduce dynamics.
    \begin{axiom}[Time slice]
        Consider two spacetime regions $O_1,O_2$. If $O_1$ contains a Cauchy surface of $O_2$, the morphism $\mathcal{A}(O_1\hookrightarrow O_2)$ of $C^*$-algebras is an isomorphism.
    \end{axiom}

    \begin{axiom}[Haag duality]\index{Haag!duality}
        Let $\overline{O}$ denote the spacelike complement of $O$ and let $\mathcal{A}'$ denote the commutant of $\mathcal{A}$. Haag duality states that\footnote{Here it should be understood that $\mathcal{A}\left(\overline{O}\right)$ is the algebra generated by all algebras $\mathcal{A}(Q)$, where $Q$ ranges over the causally closed sets in $\overline{O}$.}
        \begin{gather}
            \mathcal{A}\bigl(\overline{O}\bigr)' = \mathcal{A}(O)
        \end{gather}j
        for all causally closed sets $O$.
    \end{axiom}
    \remark{Haag duality is known to hold for all free theories and even for some interacting theories. However, it is also known to fail in the case of symmetry breaking~\citet{roberts_spontaneously_1974}.}

    To generalize the above axiom system to globally hyperbolic space times, one must enter the realm of category theory. The notation\footnote{This could potentially cause confusion with other notations used in this text. $\mathbf{Loc}$ here has nothing to do with the category of locales from \cref{chapter:topology}.} of~\citet{calaque_mathematical_2015} (?? AND OTHERS ??) will be adopted. Let $\mathbf{Loc}$ be the category of globally hyperbolic space times with orientation- and causal structure-preserving isometries. Let $\mathbf{Obs}$ be the category of relevant algebras (commutative algebras for classical physics and $C^*$-algebras for quantum theories) together with suitable algebra morphisms. The assignment of algebras is then given by a functor $\func{\mathfrak{U}}{Loc}{Obs}$. The Haag--Kastler framework is recovered when $\mathfrak{U}$ is restricted to the subcategory on globally hyperbolic subsets of some manifold (with inclusions as morphisms).

\subsection{Weyl systems}

    \newdef{Weyl system}{\index{Weyl!system}
        Let $(L,\omega)$ be a symplectic vector space and let $K$ be a complex vector space. Consider a map $W$ from $L$ to the space of unitary operators on $K$. The pair $(K,W)$ is called a Weyl system over $(L,\omega)$ if it satisfies
        \begin{gather}
            W(z)W(z') = e^{\frac{i}{2}\omega(z,z')}W(z+z')
        \end{gather}
        for all $z,z'\in L$.
    }
    \remark{This is a generalization of the Weyl form of the canonical commutation relations~\ref{qm_formalism:CCR}.}

    \newdef{Heisenberg system}{\index{Heisenberg!system}
        The generators $\phi(z)$ of the maps $t\mapsto W(tz)$, which exist by Stone's theorem \ref{functional:stone}, are said to form a Heisenberg system. These operators satisfy the following properties:
        \begin{itemize}
            \item $\lambda\phi(z) = \phi(\lambda z)$ for all $\lambda>0$,
            \item $[\phi(z),\phi(z')] = -i\omega(z,z')$, and
            \item $\phi(z+z')$ is the closure \ref{functional:closure} of $\phi(z)+\phi(z')$.
        \end{itemize}
    }

\section{Perturbative AQFT}

    Perturbative algebraic quantum field theory (pAQFT) tries to formalize the notion of perturbation theory in QFT as is is nowadays used in particle physics and high energy physics by combining the formalism of AQFT with the causal pertubartion theory as introduced by \textit{Epstein} and \textit{Glaser}.

    In this section, the main object of interest will be the space of fields. In almost all cases this will be the space of sections $\mathcal{E}$ of a (graded) vector bundle $\bundle$. Such spaces are always (graded) Fr\'echet spaces by generalizing the construction in \cref{distribution:distribution}. Here one can use the fact that every manifold is $\sigma$-compact and admits a connection. Moreover, as before, the spacetime manifolds $M$ will always be assumed to be globally hyperbolic, i.e.~admit a Cauchy surface.

\subsection{Functional analysis}

    To handle the dynamics, one first has to generalize the notion of Lagrangians and actions from \cref{chapter:classical_fields}. Instead of simply considering a local action functional $S:\mathcal{E}\rightarrow\mathbb{R}$, one uses a generalization:
    \begin{gather}
        S:C^\infty_c(M)\rightarrow\mathcal{F}_{\text{loc}}(\mathcal{E})\,,
    \end{gather}
    i.e.~one uses local functional-valued distributions on $M$. Because local functionals are of the form
    \begin{gather}
        F:\phi\mapsto\Int_M (j^\infty\phi)^*L\vol\,,
    \end{gather}
    the standard example of such generalized action functionals are simply given by taking a product:
    \begin{gather}
        S[f]:\phi\mapsto\Int_Mf(j^\infty\phi)^*L\vol\,.
    \end{gather}
    This approach has to benefits. First of all, one does not have to restrict to compact sets to have a well-defined integration by parts formula, since all objects under the integral are now automatically compactly supported on $M$. Secondly, one has a straightforward way to implement adiabatic switching, an important ingredient for perturbation theory, since the function $f$ can be treated as a position-dependent coupling constant.

    \begin{remark}[Functional approach vs variational bicomplex]\index{local!functional}
        In \cref{chapter:classical_fields}, the variational bicomplex was implicitly assumed as the framework for doing variational calculus. The main assumption of this framework, however, is locality. The functional approach on the other hand does allow nonlocal functionals. For completeness, algebraically, a local functional is characterized by the following conditions:
        \begin{enumerate}
            \item For all $\phi,\psi$ and $\chi\in\mathcal{E}$:
                \begin{gather}
                    F(\phi+\psi+\chi)=F(\phi+\psi)-F(\psi)+F(\psi+\chi)
                \end{gather}
                whenever $\supp(\phi)\cap\supp(\chi)=\emptyset$.
            \item The wave front set of $F^{(n)}$ is orthogonal to the tangent bundle $T\Delta_n$ of the diagonal of $M$ for all $n\in\mathbb{N}$.
        \end{enumerate}
        For $n=1$ this implies that $F'(\phi)$ is a smooth function. Moreover, this also implies that derivatives of local functionals are supported on the diagonals $\Delta_n$.
    \end{remark}

    For many purposes in modern field theory one needs to enlarge the space of sections to a graded vector space (e.g.~to include fermionic fields). Functionals on such graded spaces are defined as follows.
    \newdef{Graded functional}{\index{functional!graded}\index{functional!multilocal}
        One can show that the (strong) dual to the space sections of an exterior tensor product is isomorphic to the completed projective tensor product of the (strong) dual of the the spaces of sections of the individual bundles:
        \begin{gather}
            \Gamma(E^{\boxtimes k})^*\cong\big(\Gamma(E)^*\big)^{\widehat{\otimes}_\pi k}\,.
        \end{gather}
        By restricting on the right-hand side to the subspaces of (anti)symmetric tensors, one can define the spaces of (anti)symmetric functionals on exterior tensor powers (after completion). The space of antisymmetric functionals on $\mathcal{E}$ is then defined as follows:
        \begin{gather}
            \mathcal{F}(\mathcal{E}) := C^\infty\left(V_0,\bigoplus_{\substack{k,l=0\\i_1+\cdots i_l=k}}\Gamma_{i_1|\underline{i_2}|i_3|\ldots}\left(\boxtimes^l_{j=1}E[-j]_0^{\boxtimes i_j}\right)^*\right)\,,
        \end{gather}
        where the sequence $i_1|\underline{i_2}|i_3|\ldots$ denotes symmetric and antisymmetric tensor powers (underlined indices denote antisymmetric tensors). Note that because of the desuspension operator (\cref{hda:suspension}), the tensor powers are taken over bundles of degree 0. Equivalently, the desuspension could have been left out such that only the symmetric tensor powers have to be used:
        \begin{gather}
            \mathcal{F}(\mathcal{E}) = C^\infty\left(V_0,\bigoplus_{\substack{k,l=0\\i_1+\cdots i_l=k}}\Gamma_{i_1|i_2|i_3|\ldots}\left(\boxtimes^l_{j=1}E_j^{\boxtimes i_j}\right)^*\right)\,,
        \end{gather}
        The closure of the subspace of local functionals, those of compact support that are induced by functions on some jet bundle, under wedge products is called the space of \textbf{multilocal functionals} $\mathcal{F}_{\text{ml}}(\mathcal{E})$. These model the physical observables of the theory.
    }

    Aside from the standard functional derivative, in the sense of Gateaux (\cref{functional:gateaux}), one can now also define a `graded derivative'.
    \begin{formula}[Graded derivative]\index{derivative!graded}\label{aqft:graded_derivative}
        Denote the functional target space \[\bigoplus_{\substack{l=0\\i_1+\cdots i_l=k}}\Gamma_{i_1|\underline{i_2}|i_3|\ldots}\left(\boxtimes^l_{j=1}E[-j]_0^{\boxtimes i_j}\right)^*\] by $\mathcal{A}^k(\mathcal{E})$ and consider a homogeneous element $F\in\mathcal{A}^k(\mathcal{E})$. The left (graded) derivative is defined as follows:
        \begin{gather}
            \begin{cases}
                \ds\left\langle\frac{\delta^lF}{\delta\phi}(v),w\right\rangle := F(w\wedge v)&k>0\,,\\
                &\\
                \ds\frac{\delta^lF}{\delta\phi} := 0&k=0\,.
            \end{cases}
        \end{gather}
        Similarly, the right (graded) derivative is defined by
        \begin{gather}
            \begin{cases}
                \ds\left\langle\frac{\delta^rF}{\delta\phi}(v),w\right\rangle := F(v\wedge w)&k>0\,,\\
                &\\
                \ds\frac{\delta^rF}{\delta\phi} := 0&k=0\,.
            \end{cases}
        \end{gather}
    \end{formula}

\subsection{Equations of motion}

    Using the functional approach, the Euler--Lagrange derivative $\delta_{\text{EL}}L:\mathcal{E}\rightarrow\mathcal{E}^*_c$ is given by the following formula:
    \begin{gather}
        \langle\delta_{\text{EL}}L(\phi),\psi\rangle := \langle L'[f](\phi),\psi \rangle\,,
    \end{gather}
    for any cutoff function $f$ on $\supp(\psi)$.
    \begin{remark}[Functoriality]
        A more categorical formulation is that a generalized Lagrangian is a natural transformation
        \begin{gather}
            L:C^\infty_c\Rightarrow\mathcal{F}_\text{loc}
        \end{gather}
        that satisfies the following locality conditions:
        \begin{enumerate}
            \item\textbf{Additivity}: $L[f+g+h] = L[f+g]-L[g]+L[g+h]$ if $\supp(f)\cap\supp(h)=\emptyset$, and
            \item\textbf{Support}: $\supp(L[f])\subseteq\supp(f)$.
        \end{enumerate}
        To obtain an action functional $S$, one quotients out by the relation
        \begin{gather}
            \supp(L_1[f]-L_2[f])\subset\supp(\dr f)\,.
        \end{gather}
        This identifies Lagrangians that differ up to a total divergence since the above equation for the Euler--Lagrange derivative is invariant under the addition of such terms. In fact it is invariant under the addition of any generalized Lagrangian that is supported outside the level set $f^{-1}(1)$. (This is exactly the condition that is quotiented out.)
    \end{remark}

    The second variational derivative of the action then satisfies:
    \begin{gather}
        \langle\delta^2S(\phi),\psi_1\otimes\psi_2 \rangle = \langle L''[f](\phi),\psi_1\otimes\psi_2 \rangle\,,
    \end{gather}
    where $f$ is a cutoff function on $\supp(\psi_1)\cup\supp(\psi_2)$. Because the Lagrangians are local, the linear maps $S''(\phi)$ can be extended from $\mathcal{E}_c\times\mathcal{E}_c$ to $\mathcal{E}\times\mathcal{E}_c$. By the Kernel Theorem~\ref{distribution:kernel_theorem}, this gives rise to a linear operator $P_S(\phi):\mathcal{E}\rightarrow\mathcal{E}^*$.

    \begin{axiom}[Normal hyperbolicity]\index{hyperbolic!normally}
        The operator $P_S(\phi)$ is normally hyperbolic, i.e.~a hyperbolic differential operator whose principal symbol defines a metric, for all solutions $\phi$.
    \end{axiom}

    This axiom implies that $P_S(\phi)$ admits unique Green's operators $\Delta^{R/A}_{S,\phi}:\mathcal{E}^*_c\rightarrow\mathcal{E}$:
    \begin{gather}
        \begin{aligned}
            P_S(\phi)\circ\Delta^{R/A}_{S,\phi} &= \mathbbm{1}_{\mathcal{E}^*_c}\\
            \Delta^{R/A}_{S,\phi}\circ P_S(\phi) &= \mathbbm{1}_{\mathcal{E}_c}
        \end{aligned}\\\nonumber\\
        \begin{aligned}
            \supp(\Delta^R_{S,\phi}(\psi)) &\subset J^-(\supp(\psi))\\
            \supp(\Delta^A_{S,\phi}(\psi)) &\subset J^+(\supp(\psi))\,.
        \end{aligned}
    \end{gather}
    These operator also define integral kernels through the kernel theorem:
    \begin{gather}
        \Delta^R_{S,\phi}(x,y) = \Delta^A_{S,\phi}(y,x)\,.
    \end{gather}
    Their difference gives the causal propagator:\index{propagator}
    \begin{gather}
        \Delta_{S,\phi} := \Delta^R_{S,\phi} - \Delta^A_{S,\phi}\,.
    \end{gather}

    \begin{property}\index{equation of motion!linearised}
        If the action $S$ is a quadratic functional, the operator $P_S(\phi)$ is constant:
        \begin{gather}
            \forall\phi\in\mathcal{E}:P_S(\phi)=P\,.
        \end{gather}
        Moreover, in this case the Euler--Lagrange equation becomes $P(\phi) = 0$. In general, the equation $P_S(\phi)\psi=0$ is called the \textbf{linearised equation of motion}.
    \end{property}

    The next step is to equip the space of functionals with a suitable Poisson bracket. As in \cref{chapter:classical_fields}, this will be the Peierls bracket.
    \newdef{Peierls bracket}{\index{Peierls bracket}
        Consider two multilocal functionals $F,G\in\mathcal{F}_{\text{ml}}(\mathcal{E})$ together with an action functional $S$.
        \begin{gather}
            \{F,G\}_S(\phi) := \left\langle F'(\phi),\Delta_{S,\phi}G'(\phi) \right\rangle\,.
        \end{gather}
    }
    The main problem with the Peierls bracket is that it does not define a Poisson algebra structure on $\mathcal{F}_{\text{ml}}(\mathcal{E})$ since this space is not closed under the bracket. The H\"ormander criterion for pointwise multiplication of distributions gives some intuition for how to construct such a space.

    \begin{property}\label{aqft:causal_support}
        The integral kernel of the causal propagator has the following wave front set:
        \begin{gather}
            \mathrm{WF}(\Delta_{(S,\phi)}) = \{(x,k,x',-k')\in T^*M^2\mid(x,k)\sim(x',k')\}\,,
        \end{gather}
        where $\sim$ indicates that the points are connected by the lift of a null geodesic whose fibrewise projection is given by the metric dual:
        \begin{gather}
            \bigl(x(t),k(t)\bigr)\equiv\bigl(\gamma(t),g(\dot{\gamma}(t),\cdot)\bigr)
        \end{gather}
        for some null geodesic $\gamma$ on $M$.
    \end{property}

    \newdef{Microcausal functionals}{\index{microcausality}
        A functional $F\in\mathcal{E}^*$ that is compactly supported and such that
        \begin{gather}
            \mathrm{WF}(F^{(n)}(\phi))\subset\Xi_n
        \end{gather}
        for all $n\in\mathbb{N}$ and $\phi\in\mathcal{E}$, where
        \begin{gather}
            \Xi_n := T^*M^n\backslash\{(x,k)\in T^*M^n\mid k\in\overline{V}^+(x)\cup\overline{V}^-(x)\}\,.
        \end{gather}
        Microcausal functionals hare, hence, exactly those functionals for which the wafront set excludes (the closure of) causal lightcones. Moreover, a microcausal functional is said to be \textbf{strongly microcausal} if $F'(\phi)$ is $C^1$ for all $\phi\in\mathcal{E}$ and $\phi\mapsto F'(\phi)$ is also $C^1$.
    }
    \begin{property}
        The space of strongly microcausal functionals, equipped with the Peierls bracket, forms a Poisson algebra. If $S$ is quadratic, i.e.~the causal propagator does not depend on the fields, the space of microcausal functionals also forms a Poisson algebra under the Peierls bracket.
    \end{property}

    \newdef{Wightmann propagator}{\index{Wightmann!propagator}\index{Hadamard|seealso{Wightmann}}
        Consider the causal propagator $\Delta_S$. By \cref{aqft:causal_support}, its wave front set consists of two part lying in the closed past and closed future lightcones. This corresponds to the following decomposition:
        \begin{gather}
            \frac{i}{2}\Delta_S = \Delta^H_S - H\,,
        \end{gather}
        where
        \begin{gather}
            \mathrm{WF}(\Delta^H_S) = \{(x,k,x',-k')\in T^*M^2\mid(x,k)\sim(x',k')\land k\in \overline{V}^+(x)\}
        \end{gather}
        and
        \begin{gather}
            H(x,y) = H(y,x)\,.
        \end{gather}
        The Wightmann or \textbf{Hadamard} propagator represents the positive-frequency part of the causal propagator.
    }

\subsection{Symmetries}

    Now, to include symmetries, one has to introduce vector fields. On general infinite-dimensional manifolds (\cref{section:infinie_dimensional}) multiple approaches exist. One way is to regard $\mathcal{E}$ as a trivial $\mathcal{E}$-manifold. This way, the space of vector fields, given as sections of the tangent bundle, is then simply $C^\infty(\mathcal{E},\mathcal{E})$.
    \newdef{Multilocal multivector fields}{
        The space of multivector fields is now simply defined as the space of functionals on the odd cotangent bundle:
        \begin{gather}
            \mathfrak{X}_{\text{loc}}(\mathcal{E}) := \mathcal{F}_{\text{loc}}\bigl(\mathcal{E}\oplus\mathcal{E}^*[1]\bigr)\,.
        \end{gather}
        The subspace $\mathfrak{X}^1_{\text{loc}}(\mathcal{E})$ consists of the derivations on $\mathcal{F}_{\text{loc}}(\mathcal{E})$. The algebraic completion under wedge products defines the space of multilocal multivector fields $\mathfrak{X}_{\text{ml}}(\mathcal{E})$. The subspace $\mathfrak{X}^1(\mathcal{E})$, the completion as a $\mathcal{F}(\mathcal{E})$-module of the local vector fields, are then de derivations of $\mathcal{F}_{\text{ml}}(\mathcal{E})$.
    }

    The action of a vector field $X\in\mathfrak{X}^1_{\text{ml}}(\mathcal{E})$ is given by
    \begin{gather}
        \partial_XF(\phi) := \langle F'(\phi),X(\phi) \rangle\,.
    \end{gather}
    In physics literature, this is often formally written as
    \begin{gather}
        \partial_XF(\phi) = \Int_MX_x(\phi)\frac{\delta F}{\delta\phi(x)}\,.
    \end{gather}

    Note that for any multilocal vector field, the following functional is trivial on the zero locus $\mathcal{E}_S$ of $\delta_{\text{EL}}S$:
    \begin{gather}
        \phi\mapsto\delta_S(X)(\phi):=\partial_XS(\phi)=\langle\delta_{\text{EL}}S(\phi),X(\phi)\rangle\,.
    \end{gather}
    These functionals form an ideal under pointwise multiplication and the quotient ideal gives the space of physically distinguishable observables, i.e.~the on-shell functionals:
    \begin{gather}
        \mathcal{F}_S := \mathcal{F}_{\text{ml}}/\delta_S(\mathfrak{X}^1_{\text{ml}})\,.
    \end{gather}
    As in \cref{chapter:constrained_dynamics} and \cref{chapter:classical_fields}, one can now characterize this space in homological terms. For this one constructs the Koszul complex
    \begin{gather}
        \cdots\overset{\delta_S}{\longrightarrow}\Lambda^2\mathcal{E}\overset{\delta_S}{\longrightarrow}\mathcal{E}\overset{\delta_S}{\longrightarrow}\mathcal{F}_{\text{ml}}\longrightarrow0\,,
    \end{gather}
    where $\delta_S$ is extended to multivector fields through the (right-)graded Leibniz rule. The first homology group of this complex represents the space of nontrivial symmetries, they finish identically on $\mathcal{E}_S$. The differential $\delta_S$ can also be (locally) generated by a Poisson-like bracket, the antibracket. In this formulation it is given by the Schouten-Nijenhuis bracket on multivector fields. \Cref{field:antibracket} for the antifield (ignoring the ghosts and antighosts for now) then corresponds to the expression
    \begin{gather}
        \{X,Y\}(\phi) := \left\langle\frac{\delta^rX}{\delta\phi},\frac{\delta^lY}{\delta\phi^\dagger}\right\rangle - \left\langle\frac{\delta^rX}{\delta\phi^\dagger},\frac{\delta^lY}{\delta\phi}\right\rangle\,,
    \end{gather}
    where the derivatives with respect to $\phi$ and $\phi^\dagger$ should be interpreted as functional (Gateaux) and graded derivatives (\cref{aqft:graded_derivative}) with respect to the base and fibre coordinates of the odd cotangent bundle $\mathcal{E}\oplus\mathcal{E}^*[1]$.

    ?? ADD SCHOUTEN BRACKET TO CHAPTER ON VECTOR BUNDLES ??

\subsection{Quantization}

    To quantize the above classical formalism, the framework of deformation quantization is used. Since all spaces considered in the previous sections are linear, Moyal quantization (\cref{quantization:moyal}) is applicable:
    \begin{gather}
        F\ast G := \mu\circ\exp(\tfrac{i\hbar}{2}\pi_\Delta)(F\otimes G)\,,
    \end{gather}
    where $\pi_\Delta$ is the Poisson bivector induced by the causal propagator (see the definition of Peierls bracket above). However, aside from the causal propagator $\Delta$, one also has the Wightmann propagator $\Delta^H$. This operator also induces a Moyal product:
    \begin{gather}
        \label{aqft:moyal_product}
        F\ast_HG := \mu\circ\exp(\hbar\pi_{\Delta^H})(F\otimes G)\,,
    \end{gather}

    \begin{property}
        On the space of regular functionals, the causal and Wightmann star products are isomorphic:
        \begin{gather}
            F\ast_HG = \alpha_H(\alpha_H^{-1}F\ast\alpha_H^{-1}G)\,,
        \end{gather}
        where
        \begin{gather}
            \alpha_H := \exp\left(\frac{\hbar}{2}\left\langle H,\frac{\delta^2}{\delta\phi^2}\right\rangle\right)\,.
        \end{gather}
        The transformation from one star product to the other corresponds to Wick's theorem in ordinary QFT.
    \end{property}

    The space of regular functionals is sequentially dense in the space of microcausal functionals, so it makes sense to consider sequences that converge to an element in $\mathcal{F}_{\mu C}$.
    \newdef{Quantum algebra}{\index{quantum!algebra}
        The quantum algebra $\mathfrak{U}^H$ of the free theory $S_0$ is defined as the extension of $\mathcal{F}_{\text{reg}}[[\hbar]]$ by elements of the form $\lim_{n\rightarrow\infty}\alpha_H^{-1}(F_n)$ for $\seq{F}$ a convergent sequence in $\mathcal{F}_{\mu C}$. The Moyal product $\ast_H$ is defined as above and involution is given by
        \begin{gather}
            F^* := \alpha_H^{-1}\left(\overline{\alpha_HF}\right)\,.
        \end{gather}
        The support of an element $F\in\mathfrak{U}^H$ is given by
        \begin{gather}
            \supp(F) := \supp(\alpha_HF)\,.
        \end{gather}
    }
    \begin{remark}
        The decomposition of the causal propagator in terms of the Wightmann propagator is not unique. For two such decompositions the difference $H-H'$ is a smooth symmetric bisolution:
        \begin{gather}
            P_x(H-H')(x,y) = P_y(H-H')(x,y)=0\,.
        \end{gather}
        The associated morphism $\alpha_{H-H'}$ is an isomorphism and relates the quantum algebras $\mathfrak{U}^H$ and $\mathfrak{U}^{H'}$. As abstract algebras these are isomorphic, but the choice of Wightmann propagator fixes how classical functionals are realized in the quantum algebra.
    \end{remark}

    \begin{property}
        Let $(M,g)$ be a globally hyperbolic manifold equipped with a vector bundle $\bundle$. If $S_0$ is a quadratic action functional that induces a formally self-adjoint operator $P$, the net of quantum algebras $O\mapsto\mathfrak{U}(O)$ of elements with support in $O$ satisfies the Haag--Kastler axioms.
    \end{property}

    As before the on-shell observables are defined as those elements of the quotient algebra
    \begin{gather}
        \mathfrak{U}_{S_0} := \mathfrak{U}/\mathfrak{J}_{S_0}\,,
    \end{gather}
    where $\mathfrak{J}_{S_0}$ is generated by the functionals $\langle\delta_{\text{EL}}S,X\rangle$ for $X\in\mathfrak{X}_{\text{loc}}$. The Moyal product is extended to microcausal multivector fields by replacing the pointwise multiplication in \cref{aqft:moyal_product} by the wedge product:
    \begin{align}
        \langle X\!&\ast_H\!Y(\phi),v_1\otimes\cdots\otimes v_{p+q} \rangle := \\
        &\sum_{i=0}^\infty\frac{\hbar^i}{i!p!q!}\sum_{\sigma\in S_{p+q}}\left\langle\!\!\left\langle\frac{\delta^iX}{\delta\phi^i}(\phi),v_{\sigma(1)}\otimes\cdots\otimes v_{\sigma(p)}\right\rangle,(\Delta^H_{S_0})^{\otimes i}\left\langle\frac{\delta^iY}{\delta\phi^i}(\phi),v_{\sigma(p+1)}\otimes\cdots\otimes v_{\sigma(p+q)}\right\rangle\!\!\right\rangle\,.\nonumber
    \end{align}

    ?? COMPLETE ??

\section{Topological QFT}

    For convenience, the reader is reminded of some of the notations that are used. $\mathbf{FinVect}$ will denote the category of finite-dimensional vector spaces over $\mathbb{C}$ and $\mathbf{Bord}^d_{d-1}$ denotes the category of $d$-dimensional (framed\footnote{A framing in this context means a framing of the $d$-stabilized tangent bundles $TM\oplus\mathbb{R}^{d-\dim(M)}$.}) cobordisms (\cref{manifold:cobordism}).

\subsection{Atiyah--Segal axioms}

    \begin{axiom}[Atiyah--Segal]\index{Atiyah--Segal axioms}\index{field!theory}
        \nomenclature[A_TQFT]{TQFT}{topological quantum field theory}
        A $d$-dimensional \textbf{topological quantum field theory} (TQFT) is a symmetric monoidal functor $F:\mathbf{Bord}_{d-1}^d\rightarrow\mathbf{FinVect}$. This means (among other things) that $F$ is a map satisfying the following axioms:
        \begin{enumerate}
            \item\textbf{Normalization}: $F(\emptyset)=\mathbb{C}$.
            \item\textbf{Disjoint union}: $F(M\sqcup M') = F(M)\otimes F(M')$.
            \item\textbf{Composition}: If $N=M\cup M'$, where $\partial M$ and $\partial M'$ have opposite orientations, then \[F(N) = F(M)\circ F(M').\]
            \item\textbf{Invariance}: If $f: M\rightarrow M'$ is a diffeomorphism rel boundary, then $F\circ f = F$.
        \end{enumerate}
        In the above conditions $M,M'$ are $d$-dimensional cobordisms between $(d-1)$-dimensional (closed) smooth manifolds.
    \end{axiom}

    \begin{example}[1D]
        In $d=1$, a TQFT functor gives rise to the following correspondence:
        \begin{gather*}
            \begin{array}{l|l}
                \text{point with orientation } + & \text{vector space } V\\
                \text{point with reversed orientation } - & \text{dual space }V^*\\
                \text{line between points} & \text{linear map }f:V\rightarrow V\\
                \text{cap between $\emptyset$ and points } +,- & \text{coevaluation } \mathbb{C}\rightarrow V\otimes V^*\\
                \text{cup between points $-,+$ and }\emptyset & \text{evaluation }V^*\otimes V\rightarrow\mathbb{C}
            \end{array}
        \end{gather*}
        Essentially, this gives the structure of a (finite-dimensional) vector space and its dual.
    \end{example}

    \begin{example}[2D]
        In $d=2$, one can obtain a similar result by drawing all possible configurations. However, the existence and combination of `pair of pants'-diagrams gives a richer structure. For 2D TQFTs the corresponding object is a (finite-dimensional) commutative and cocommutative Frobenius algebra (\cref{linalgebra:frobenius}).
    \end{example}

    In dimensions 3 and higher, the definition above is intractable. To allow the construction to be generalized to higher dimensions one considers the following (extended) definition.
    \newdef{Extended TQFT}{
        A $d$-dimensional extended TQFT is a symmetric monoidal $(\infty,d)$-functor $F:\mathbf{Bord}^d\rightarrow\mathbf{FinVect}$ satisfying the Atiyah-Segal axioms, where the invariance axiom is required only at the highest level of $k$-morphisms.
    }
    \begin{remark}
        One can replace the category $\mathbf{FinVect}$ by any other symmertric monoidal category $\mathbf{C}$.
    \end{remark}

    The following conjecture by \textit{Baez} and \textit{Dolan}, later proven by \textit{Lurie}, states that every extended TQFT is defined by a single object. This is motivated by the fact that all cobordisms between smooth manifolds can be obtained from gluing together (the geometric representation) of identity morphisms and taking disjoint unions of them.
    \begin{theorem}[Cobordism hypothesis]\index{cobordism!hypothesis}
        $\mathbf{Bord}^d$ is the free symmetric monoidal $(\infty,n)$-category with all duals on a single object $\mathrm{pt}$. In particular, every symmetric monoidal $(\infty,n)$-functor $F:\mathbf{Bord}^d\rightarrow\mathbf{C}$ is characterized by a single fully dualizable object $F(\mathrm{pt})\in\ob{C}$.
    \end{theorem}

\subsection{Open-closed TQFT}

    In the case of ordinary TQFTs as defined in the previous section, one considers cobordisms between closed manifolds. Hence, the relevant objects in this category are manifolds with boundary. A generalization is obtained by relaxing the constraint on the cobordisms and allowing for the notion of manifolds with corners (\cref{section:manifold_boundary}). For simplicity only the case of 2D TQFTs (as in the original definition) will be considered.

    ?? COMPLETE ??

\section{Prequantum field theory}

    Now, consider an (extended) TQFT. In full generality such a TQFT assigns to every manifold $M$ an ($\infty$-)stack of fields. For a cobordism, this induces a span (\cref{cat:span}) by restriction to the boundary components. Composition of cobordisms corresponds to pullback of spans. So, a TQFT induces a (monoidal) functor $\mathbf{Bord}^d\rightarrow\mathbf{Span}(\mathbf{H})$, where $\mathbf{H}$ is the $(\infty,1)$-category of $\infty\mathbf{Grpd}$-valued $\infty$-sheaves. When considering cobordisms between cobordisms, one obtains spans between spans. So, eventually, a $d$-dimensional TQFT can be seen to be equivalent to a symmetric monoidal $(\infty,n)$-functor $\mathbf{Bord}^d\rightarrow\mathbf{Corr}_d(\mathbf{H})$, where $\mathbf{Corr}_n(\mathbf{H})$ is the $(\infty,n)$-category of \textit{$n$-fold correspondences} in $\mathbf{H}$. By a theorem of \textit{Lurie}, correspondences admit a (symmetric) monoidal structure induced by that on $\mathbf{H}$. The cobordism hypothesis now immediately implies that any such functor is uniquely characterized by a (moduli) stack of fields. These theories are called \textbf{(local) prequantum field theories (pQFT)}.

    \begin{property}
        Any local pQFT is a (topological) $\sigma$-model:
        \begin{gather}
            \mathrm{Fields}(M)\cong\bigl[\smallint\!M,\mathbf{Fields}\bigr]\,,
        \end{gather}
        where $\mathbf{Fields}$ is the classifying stack and $\smallint$ sends a manifold to its homotopy type, i.e.~it is the shape modality (\cref{topos:cohesive_modalities}).
    \end{property}

    Now, to obtain a reasonable quantum theory, one also needs an action principle. To a $d$-dimensional manifold it should assign a phase, i.e.~a map $\mathbf{H}\rightarrow\mathrm{U}(1)$, to its boundary components it should assign a gauge field, i.e.~a map $\mathbf{H}\rightarrow\mathbf{B}\mathrm{U}(1)$, etc. Combining this with the pQFT, one obtains a functor $\mathbf{Bord}^d\rightarrow\mathbf{Corr}_d(\mathbf{H}/_{\mathbf{B}^d\mathrm{U}(1)})$ to the $d$-fold correspondences in the slice topos over the classifying stack of circle $d$-bundles. Because the classiyfing stacks are \textit{$E_\infty$-algebras} in $\mathbf{H}$, i.e.~commutative monoids in $\mathbf{H}$, the slice topos admits the structure of a symmetric monoidal category. Therefore one can extend the action functional to disjoint unions of cobordisms.
    \newdef{Action functional}{\index{action}
        An action functional for $\mathbf{Bord}^d\rightarrow\mathbf{Corr}_d(\mathbf{H})$ is a lift (of monoidal functors) to a symmetric monoidal $(\infty,n)$-functor $\mathbf{Bord}^d\rightarrow\mathbf{Corr}_d(\mathbf{H}/_{\mathbf{B}^d\mathrm{U}(1)})$.
    }
    By the cobordism hypothesis, such a functional is uniquely defined by a morphism $\mathbf{Fields}\rightarrow\mathbf{B}^d\mathrm{U}(1)$, i.e.~by a circle $d$-bundle on the space of fields.

    One can also rephrase this assignment in terms of (higher) holonomy functors. Any map $U\rightarrow\bigl[\smallint\!M,\mathbf{B}^d\mathrm{U}(1)\bigr]$ is given by th assignment of a flat $d$-bundle on $U\times M$, by the adjunction $\smallint\dashv\flat$. Taking the holonomy of this bundle defines a functor $[\smallint\!M,\mathbf{B}^d\mathrm{U}(1)]\rightarrow\mathbf{B}^{d-k}\mathrm{U}(1)$, where $k$ is the dimension of $U$. Composition with the classifying functor $\mathbf{Fields}\rightarrow\mathbf{B}^d\mathrm{U}(1)$ induces an assignment of $(d-k)$-bundles to the spaces of fields on $M$. This is the generalization of the (Lagrangian) action principle, where integration of Lagrangians is replaced by assigning higher holonomies.
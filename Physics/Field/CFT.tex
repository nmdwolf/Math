\chapter{Conformal Field Theory}\index{conformal!field theory}\label{chapter:cft}
\nomenclature[A_CFT]{CFT}{conformal field theory}

    References for this chapter are~\citet{schottenloher_mathematical_2008} and lecture notes by \textit{Schellekens}. For an introduction to Riemannian and conformal geometry, see \cref{chapter:riemann} and, in particular, \cref{section:conformal_structures}.

    \begin{property}[Stress-energy tensor]
        Consider a theory that is invariant under conformal transformations. The generator of general coordinate transformations is the stress-energy tensor $T$ (the associated current is $\mathcal{J}_\mu = T_{\mu\nu}\varepsilon^\nu$). Conformal invariance implies that the stress-energy tensor is traceless:
        \begin{gather}
            T^\mu_\mu = 0\,.
        \end{gather}
    \end{property}

\section{\texorpdfstring{In dimension $d=2$}{In dimension d=2}}

    In dimension 2 (in Euclidean signature) something special happens. By inserting $d=2$ in the conformal Killing equation~\eqref{riemann:conformal_killing}, the Cauchy--Riemann equations~\ref{complex:cauchy_riemann} are obtained. The scale factor can thus be written as
    \begin{gather}
        \kappa(z) = \left|\frac{\partial f}{\partial z}\right|^2
    \end{gather}
    for some analytic function $f:\mathbb{C}\rightarrow\mathbb{C}$. Because of this complex coordinates will be used from here on. Switching to complex coordinates also has important consequences for the metric and stress-energy tensor:
    \begin{align}
        g_{zz} = g_{\zbar\zbar} = 0&\qquad g_{z\zbar} = \frac{1}{2}\,,\\
        \partial_zT_{\zbar\zbar} = \partial_{\zbar} T_{zz} = 0&\qquad T_{z\zbar} = 0\,.
    \end{align}
    The stress-energy tensor thus contains a meromorphic\footnote{The literature often just calls this holomorphic.} component $T_{zz}\equiv T(z)$ and an antimeromorphic component $T_{\zbar\zbar}\equiv\overline{T}(\zbar)$.

    \newdef{Witt algebra}{\index{Witt algebra}\label{cft:witt_algebra}
        Infinitesimally this gives an infinite-dimensional algebra. The generators can be chosen to be
        \begin{align}
            l_n(z) &:= -z^{n+1}\partial_z\,,\\
            \overline{l}_n(\zbar)&:=-\zbar^{n+1}\partial_{\zbar}\,.
        \end{align}
        These generate the transformation $z\mapsto z-z^{n+1}$ and $\zbar\mapsto\zbar-\zbar^{n+1}$ respectively. They also give rise to isomorphic Lie algebras with the following commutation relation:
        \begin{gather}
            [l_m,l_n] = (m-n)l_{m+n}\,.
        \end{gather}
        This Lie algebra is called the \textbf{Witt algebra}.
    }

    \begin{remark}[Conformal group]
        Often one finds in the literature that the conformal group of a 2D CFT is infinite-dimensional. However, this statement is not entirely true. It is true that the Witt algebra is infinite-dimensional, but one cannot globally exponentiate all generators $l_m$. First of all it should be noted that the space of all holomorphic functions does not even have a group structure because the composition of holomorphic functions does not have to be holomorphic. The correct conformal group for 2D Euclidean CFTs is the M\"obius group $\mathrm{PSL}(2,\mathbb{C})$. This group is obtained as the Lie group generated by $l_0$ and $l_{\pm1}$, which are the only generators that can be exponentiated globally.

        What is also true is that the conformal group of two-dimensional Minkowski space is infinite-dimensional. It can be shown that $\mathrm{Conf}(\mathbb{R}^{1,1})$ is isomorphic to $\mathrm{Diff}(S^1)_+\,\times\,\mathrm{Diff}(S^1)_+$, where the orientation-preserving diffeomorphism group $\mathrm{Diff}(S^1)_+$ is an infinite-dimensional Lie group (see the intermezzo further below).

        At last it should be noted that although the conformal group of $\mathbb{R}^{2,0}$ is finite-dimensional, the infinite-dimensionality of the Witt algebra (and of its extensions) is sufficient for all physical purposes. The algebraic constraints turn this theory into an integrable theory and allow to solve it exactly.
    \end{remark}

    \newdef{Primary field}{\index{primary!field}\index{conformal!weight}
        A field that transforms tensorially under global conformal transformations:
        \begin{gather}
            \phi'(z',\zbar') = \left(\pderiv{f}{z}\right)^h\left(\pderiv{f}{\zbar}\right)^{\overline{h}}\phi\bigl(f(z),\overline{f}(\zbar)\bigr)\,.
        \end{gather}
        Fields that satisfy this relation for all conformal transformations are called primary fields. The tuple $(h,\overline{h})$ is called the \textbf{conformal weight} of the field.
    }

\subsection{Minkowski space}

    As mentioned in the remark above, the conformal group of 2D Minkowski space is infinite-dimensional. The theory of infinite-dimensional manifolds is, however, a bit more intricate than the theory of finite-dimensional manifolds. A little introduction is therefore in order.

    \newdef{Fr\'echet manifold}{\index{Fr\'echet!manifold}
        A Hausdorff space $M$ together with an atlas of coordinate charts $(U,\varphi)$ such that $\varphi:U\rightarrow F_U$ are homeomorphisms onto a Fr\'echet space and such that the transition functions are smooth maps (of Fr\'echet spaces).
    }
    \begin{remark}
        Because the dual of Fr\'echet space that is not Banach is never Fr\'echet itself, the treatment of the tangent and cotangent bundles becomes problematic. Moreover, existence and uniqueness properties of ODEs can fail in infinite dimensions, hence, many statements of ordinary differential geometry might also fail in this setting.
    \end{remark}

    Using the definition of Fr\'echet manifolds one can start to analyze the group $\mathrm{Diff}(S^1)_+$. First of all, look at the space $C^\infty(S^1,S^1)$. This space has the structure of a Fr\'echet manifold modelled on the Fr\'echet space $\mathfrak{X}(S^1)$ of vector fields on the circle\footnote{By this definition it can be seen that $\mathfrak{X}(S^1)$ is isomorphic (as a Lie algebra) to the mapping space $C^\infty(S^1,\mathbb{R})$. Henceforth, the vector fields $v$ will be identified with their corresponding smooth function $\xi$.}:
    \begin{gather}
        \mathfrak{X}(S^1) = \left\{\xi(\theta)\frac{\partial}{\partial\theta}\,\middle\vert\,\theta\in C^\infty(S^1)\right\}\,.
    \end{gather}
    Let $V_0$ be the set of vector fields that have norm $\|v\|\leq\pi$ and let $U_0$ be the set of smooth mappings $f\in C^\infty(S^1,S^1)$ such that $f(\theta)\neq-\theta$ for all $\theta\in S^1$. There exists a diffeomorphism $\psi:V_0\rightarrow U_0$ that assigns to any vector field $v$ the function $\psi_v:S^1\rightarrow S^1$ such that the arc between $\theta$ and $\psi_v(\theta)$ has length $\|v(\theta)\|$. If an open subset $U\subset U_0$ of diffeomorphisms is chosen, then a chart $(U,\psi^{-1})$ is obtained around the identity map. Charts around any diffeomorphism $f:S^1\rightarrow S^1$ are obtained by left multiplication of $U$.

    The Lie algebra of $\mathrm{Diff}(S^1)_+$ is, therefore, given by $\mathfrak{X}(S^1)$, but the induced Lie bracket is the commutator of vector fields with the opposite sign: \[[\cdot,\cdot]_{\text{Lie}} = -[\cdot,\cdot]_{\mathfrak{X}(S^1)}\,.\] Now, it is interesting to note that the Witt algebra is actually a subalgebra of $\mathfrak{X}(S^1)$. Consider the maps $\xi_n(\theta):=-ie^{in\theta}$ (the minus sign is a convention). The associated vector fields satisfy
    \begin{gather}
        \left[\xi_k(\theta)\frac{\partial}{\partial\theta}, \xi_l(\theta)\frac{\partial}{\partial\theta}\right] = -i(l-k)\xi_{k+l}(\theta)\frac{\partial}{\partial\theta}\,.
    \end{gather}
    These are exactly the relations for the Witt algebra.

\section{\texorpdfstring{Quantization in $d=2$}{Quantization in d = 2}}

    By \cref{riemann:conformal_group}, the conformal group of a pseudo-Euclidean space of signature $(p,q)$ is given by the special orthogonal group $\mathrm{SO}(p+1,q+1)$. For Minkowski space this becomes $\mathrm{SO}(4,2)$. However, with the current section in mind, it is important to look at the interaction between representation theory and quantum mechanics.

    When treating spin quantum-mechanically one is forced to pass from the rotation group $\mathrm{SO}(3)$ to its double cover $\mathrm{SU}(2)$. Then, when passing to special relativity, the Lorentz group $\mathrm{SO}^\uparrow(3,1)$ had to be replaced by its double cover $\mathrm{SL}(2,\mathbb{C})$. The same story repeats itself here: the conformal group should be replaced by its double cover $\mathrm{SU}(2,2)$. This is also how \textit{twistor theory} comes into the picture.\index{twistor theory}

\subsection{Radial quantization}

    The charge of a conserved current $\mathcal{J}^\mu$ is generally given by \cref{field:noether_charge}:
    \begin{gather}
        Q = \Int_\Sigma\mathcal{J}^0(x,t)\,,
    \end{gather}
    where $\Sigma$ is a spacelike hypersurface. Often the spatial dimension will be compactified (this can be seen as a regularization procedure) and the time dimension will be \textit{Wick-rotated}. By a conformal transformation one can then go back to the plane, mapping one end of the cylinder to the origin and the other side to a circle at $\infty$. After these transformations one obtains the following form for an operator on the plane:
    \begin{gather}
        Q_\varepsilon = \frac{1}{2\pi i}\Oint\varepsilon(z)T(z)\,dz + \frac{1}{2\pi i}\Oint\varepsilon(\zbar)\overline{T}(\zbar)\,d\zbar\,.
    \end{gather}
    As usual, an infinitesimal transformation of a field $\phi$ is given by the commutator $[Q_\varepsilon,\phi(w,\overline{w})]$ or, in integral form, by\footnote{A holomorphic split is implicitly assumed such that antiholomorphic contributions $\overline{T}(\zbar)$ can be ignored.} \[\frac{1}{2\pi i}\Oint\varepsilon(z)\bigl[T(z)\phi(w,\overline{w}) - \phi(w,\overline{w})T(z)\bigr]\,dz\,.\] However, because all the objects in this formula are operators, operator ordering should be taken into account. On the plane this is given by the so-called \textbf{radial ordering}:\index{radial ordering}
    \begin{gather}
        \mathcal{R}(A(z,\zbar)B(w,\overline{w})) :=
        \begin{cases}
            A(z,\zbar)B(w,\overline{w})&|z|>|w|\,,\\
            B(w,\overline{w})A(z,\zbar)&|w|>|z|\,.
        \end{cases}
    \end{gather}
    After a deformation of the integration contour the following general formula is obtained:
    \begin{gather}
        [Q_\varepsilon,\phi(w,\overline{w})] = \frac{1}{2\pi i}\Oint\varepsilon(z)\mathcal{R}\bigl(T(z)\phi(w,\overline{w})\bigr)\,dz\,\,,
    \end{gather}
    where the contour is a circle around the point $w$. For primary fields with conformal weight $h$, one can also write an infinitesimal transformation as \[\delta_\varepsilon\phi(z,\zbar) = h(\partial_z\varepsilon(z))\phi(z,\zbar) + \partial_z\phi(z,\zbar)\,.\] Comparing these two expressions leads to the following form of the operator product:
    \begin{gather}
        \mathcal{R}(T(z)\phi(w,\overline{w})) = \frac{h}{(z-w)^2}\phi(w,\overline{w}) + \frac{1}{z-w}\partial_w\phi(w,\overline{w}) + \text{higher order in }(z-w)\,.
    \end{gather}
    An expression of this form is called an \textbf{operator product expansion} (OPE).\nomenclature[A_OPE]{OPE}{operator product expansion}

\subsection{Virasoro algebra}\index{Virasoro algebra}

    When quantizing classical systems one has to replace symmetry actions by (unitary) projective representations, which are characterized by central extensions. As an example of the Lie group-Lie algebra correspondence, it can be shown that central extensions of Lie algebras (\cref{section:central_extension_algebra}) are in correspondence with central extensions of Lie groups.

    To equip a quantum theory with an action of the conformal group, one needs to construct a central extension of the Witt algebra. By applying \cref{lie:cocycle}, one obtains the \textbf{Virasoro algebra} as a (universal) central extension of the Witt algebra by\footnote{The reason to extend by $\mathbb{C}$ instead of $\mathbb{R}$ is that all Lie algebras are complexified in this section.} $\mathbb{C}$ associated to the cocycle
    \begin{gather}
        \Theta:(L_m,L_n)\mapsto\frac{c}{12}m(m^2-1)\delta_{m+n,0}\,.
    \end{gather}
    To obtain the Virasoro algebra from a more physical point of view, look at the stress-energy tensor. If the current $\mathcal{J}^n(z) := z^{n+1}T(z)$ is chosen, the transformation $z\longrightarrow z-z^{n+1}$ is obtained. In analogy with the generators of the Witt algebra, this generator is called $L_n$:
    \begin{gather}
        L_n := \frac{1}{2\pi i}\Oint z^{n+1}T(z)\,dz\,.
    \end{gather}
    This relation can be inverted using the Residue Theorem~\ref{complex:residue_theorem} to obtain the following expression for (the holomorphic component of) the stress-energy tensor:
    \begin{gather}
        T(z) = \sum_{n=-\infty}^\infty z^{-n-2}L_n\,.
    \end{gather}
    Using the above expression and the product operator expansion of $T(z)T(w)$, one obtains exactly the commutation relations of the Virasoro algebra:
    \begin{gather}
        [L_m,L_n] = (m-n)L_{m+n} + \frac{c}{12}m(m^2-1)\delta_{m+n,0}\,.
    \end{gather}
    The occurrence of the central charge $c$ gives a conformal anomaly on quantization (see the definition of the vacuum further below). However, the central charge does not affect the $\mathfrak{sl}(2,\mathbb{C})$ subalgebra spanned by $L_{-1},L_0$ and $L_1$. This implies that concepts such as the conformal weight are still well-defined after quantization.

\subsection{Representation theory}

    \newdef{Highest weight state}{
        A state with minimal eigenvalue for $L_0$. Equivalently, a state that is annihilated by all generators $L_n$ for $n\geq1$:
        \begin{gather}
            L_n|h\rangle = 0\,.
        \end{gather}
    }

    \newdef{Vacuum}{\index{vacuum}
        Consider the Virasoro generators $\{L_n\}_{n\in\mathbb{Z}}$. The vacuum $|0\rangle$ is defined as the maximally symmetric state. In terms of generators this means that \[L_n|0\rangle = 0\] for as many $n\in\mathbb{Z}$ as possible. However, due to the Virasoro commutation relations and, in particular, the central charge, this is not possible for all $n\in\mathbb{Z}$. Instead, one can only require that this expression vanishes for all $n\geq-1$.
    }

    \newdef{Descendants}{\index{descendant}
        By acting with the generators $L_n$, where $n\leq-1$, on a highest weight state $|h\rangle$, one obtains a whole family of states. These are called the descendants of $|h\rangle$ and together they span the Verma module (\cref{lie:verma_module}) associated to $|h\rangle$.
    }

\section{\difficult{Extensions}}
\subsection{TCFT}

    By the \textit{FRS theorem}, rational 2D CFTs are classified by pairs of modular tensor categories and internal (special symmetric) Frobenius algebras. By categorifying the latter, Calabi--Yau categories are obtained (\cref{hda:calabi_yau}). To get a fully weak formulation, one can pass to Calabi--Yau $A_\infty$-categories, where a cyclicity condition relates the higher composition maps and the trace functional. It can be shown that these classify topological conformal field theories (by a version of the \textit{Cobordism Hypothesis}).
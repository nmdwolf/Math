\chapter{Entanglement in QFT}

    The main reference is \cite{entanglement_entropy, tuybens}. This chapter should be seen as a generalization of the content of Chapter \ref{chapter:quantum_computing} to the continuum setting (in particular the characterization and computation of entanglement).

\section{Lattice theories}

    In this section the most important definitions and constructions in ordinary quantum information theory are recalled and applied to a lattice theory. Taking the lattice spacing to zero will (formally) allow to extend the definitions to continuum field theories (up to some technicalities that will be explained when necessary). For simplicity it will be assumed that the local Hilbert space is finite-dimensional.

    Consider a bipartite subdivision $A\cup A^c$ of the lattice, given by a codimension-1 hypersurface $\partial A$ called the \textbf{entangling surface}.\index{entangling surface} This induces a binary factorization of the total Hilbert space (all degrees of freedom are assumed to be confined to individual vertices) and, hence, one can compute the reduced density matrix for both $A$ and its complement $A^c$. The eigenvalues, which solely depend on the entangling surface $\partial A$, allow to calculate the von Neumann entropy:\footnote{Certain assumption ought to be made as to keep the entropy finite whenever the state-space is infinite-dimensional since it can be shown that the set of states with infinite von Neumann entropy is trace norm-dense (see \cite{Eisert_entropy}).}
    \begin{gather}
        S(\rho_A) := -\tr(\rho_A\ln\rho_A) = -\sum_i\rho_i\ln\rho_i.
    \end{gather}
    In the same way one can also introduce the R\'enyi $q$-entropy:\index{entropy!R\'enyi}
    \begin{gather}
        S_q(\rho_A) := \frac{1}{1-q}\ln\left(\sum_i\rho_i^q\right).
    \end{gather}
    \begin{property}[Limiting case]
         First of all one can analytically continue the definition of the $q$-entropy to arbitrary positive real numbers. The limit $q\longrightarrow1$ coincides with the von Neumann entropy.
    \end{property}

    ?? COMPLETE ??
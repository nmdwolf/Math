\chapter{Gauge Theory}\label{chapter:gauge_theory}

    References for this chapter are~\citet{sontz_principal_2015,schuller_lectures_2016,nash_topology_2011,belgun_gauge_2024}. The section on the Higgs mechanism is mainly based on~\citet{choquet-bruhat_analysis_2000}. Using the tools of differential geometry, as presented in \cref{chapter:bundles} and onwards, one can introduce a general formulation of gauge theories and, in particular, Yang--Mills theories.

\section{Gauge invariance}

    Consider a general Lie group $G$, often called the \textbf{gauge group}, acting on a vector bundle $E$ with typical fibre $\mathcal{H}$ over a base manifold $M$. This bundle is, in general, obtained as a bundle associated to the frame bundle $FM$. Locally, a general gauge transformation has the form
    \begin{gather}
        \label{gauge:gauge_transformation}
        \psi'(x) = U(x)\psi(x)\,,
    \end{gather}
    where $\psi,\psi':M\rightarrow\mathcal{H}$ are trivializations of local sections of $E$ and $U:M\rightarrow G$ encodes the local behaviour of the gauge transformation. It is assumed to be a unitary representation with respect to the Hilbert structure on $\mathcal{H}$. Globally, the gauge transformations are given by the vertical automorphisms.

    \begin{axiom}[Local gauge principle]\index{gauge!principle}
        The Lagrangian functional $L[\psi]$ is invariant under the action of the gauge group $G$:
        \begin{gather}
            L[U\psi] = L[\psi]\,.
        \end{gather}
    \end{axiom}

    Generally, this gauge invariance can be achieved in the following way. Denote the Lie algebra corresponding to $G$ by $\mathfrak{g}$. Because the gauge transformation is local, the information on how it varies from point to point should be able to propagate through space(time). This is done by introducing a new (local) field $B_\mu(x)$, called the \textbf{gauge field}. The most elegant formulation uses the concept of covariant derivatives.
    \newdef{Covariant derivative}{\index{covariant!derivative}\index{minimal!coupling}
        When gauging a symmetry group, the ordinary partial derivatives are replaced by the covariant derivative
        \begin{gather}
            \mathcal{D}_\mu = \partial_\mu + igB_\mu(x)\,,
        \end{gather}
        where $B_\mu:M\rightarrow\mathbb{R}$ are the coefficients of the gauge field with respect to a chosen basis of $\mathfrak{g}$. This procedure is called \textbf{minimal coupling}. It should be noted that the explicit action of the covariant derivative depends on the chosen representation of $\mathfrak{g}$ (or $G$) on $\mathcal{H}$. Furthermore, one should pay attention to the fact that the physicist convention was used, where one multiplies the gauge field $B$ by a factor $ig$.\footnote{The imaginary unit turns anti-Hermitian fields into Hermitian fields.}
    }

    So, to achieve gauge invariance, one should replace all derivatives by covariant derivatives. However, for this to be a well-defined operation, one should check that the covariant derivative itself satisfies the local gauge principle, i.e.~$\mathcal{D}'\psi'=U\mathcal{D}\psi$ (from here on the coordinate-dependence of all fields will be supressed).
    \begin{align}
        U^{-1}\left(\pderiv{}{x^\mu} + igB_\mu'\right)\psi' &= U^{-1}\left(\pderiv{}{x^\mu} + igB_\mu'\right)U\psi\nonumber\\
        &= U^{-1}\pderiv{U}{x^\mu}\psi + \pderiv{\psi}{x^\mu} + igU^{-1}B_\mu'U\psi\,.
    \end{align}
    This expression can only be equal to $\mathcal{D}\psi$ if
    \begin{gather}
        igB_\mu = U^{-1}\pderiv{U}{x^\mu} + igU^{-1}B_\mu'U\,,
    \end{gather}
    which can be rewritten as
    \begin{gather}
        B_\mu' = UB_\mu U^{-1} - \frac{1}{ig}(\partial_\mu U)U^{-1}
    \end{gather}
    or, in coordinate-independent form, as
    \begin{gather}
        B' = UBU^{-1} - \frac{1}{ig}\dr UU^{-1}\,.
    \end{gather}
    Up to scale conventions, this is exactly the content of \cref{bundle:local_compatibility} and \cref{bundle:mc_pullback} appearing in the study of principal connections. This should not come as a surprise, since the physical fields are sections of associated vector bundles and, hence, the principal bundle structure lurks in the background. Adding interactions is mathematically equivalent to coupling the space(time) manifold to a principal bundle.

    \begin{example}[QED]
        For quantum electrodynamics, which has $\mathrm{U}(1)\cong S^1$ as its gauge group, the parametrization $U(x)=e^{ie\chi(x)}$ is used with $\chi:\mathbb{R}^n\rightarrow\mathbb{R}$. Minimal coupling leads to
        \begin{align}
            \partial_\mu &\longrightarrow\mathcal{D}_\mu = \partial_\mu + ieA_\mu\,,\\
            A_\mu &\longrightarrow A_\mu' = A_\mu - \partial_\mu\chi\,,
        \end{align}
        where $A_\mu$ is the classic electromagnetic potential. These are the formulas introduced in \cref{chapter:em}.
    \end{example}

\section{Spontaneous symmetry breaking}

    \begin{theorem}[Goldstone]\index{Goldstone}
        Consider a field theory with gauge group $G$ and denote the generators of the corresponding Lie algebra by $X_a$. Generators that do not annihilate the vacuum, i.e.~$X_av\neq0$, or, equivalently, transformations that leave the vacuum invariant, correspond to massless scalar particles.
    \end{theorem}
    The massless bosons from this theorem are called \textbf{Goldstone bosons}.

\subsection{Higgs mechanism}\index{Higgs!mechanism}\index{vacuum!Higgs}\label{section:higgs_mechanism}

    In \cref{bundle:section_bijection}, the equivariant maps corresponding to global sections of a principal bundle were called Higgs fields. In this section, a clarification for this terminology is given. The \textbf{Higgs vacuum} of a $G$-gauge theory, described by a principal bundle $P$, with a $G$-invariant potential $V$ is given by the solutions of the following equations:
    \begin{align}
        V(\phi) &= 0\\
        \nabla\phi &= 0\,,
    \end{align}
    where $\nabla$ is a covariant derivative on $P$ and $\phi$ is a section of some associated (finite-rank) vector bundle $P\times_\rho\mathcal{H}$. If the space of solutions $\mathcal{M}$ to the first equation admits a transitive $G$-action, i.e.~it is a homogeneous space, then, by \cref{group:transitive_action_property}, one can write
    \begin{gather}
        \mathcal{M}\cong G/H\,,
    \end{gather}
    where $H$ is the isotropy group of any given solution. More generally, when the action is not transitive, the solution manifold is still the union of $G$-orbits, all of the form $G/H$ with $H$ the isotropy group of a point in the orbit.

    Now, consider a specific choice of vacuum $m_0\in\mathcal{M}$. If the whole theory were to be $G$-invariant, like the potential $V$, this corresponds to an equivariant map
    \begin{gather}
        \phi:P\rightarrow \mathcal{M}_0\cong G/H\,,
    \end{gather}
    where $\mathcal{M}_0$ is the orbit of $m_0$. This field is called the \textbf{Higgs field} in the physics literature. (For this reason, all such equivariant morphism and their associated sections are called Higgs fields.) The specific choice of vacuum, which generically has the smaller symmetry group $H$, induces by \cref{bundle:reduction_classification} a reduction of the structure group from $G$ to $H$ and, consequently, the symmetry group is said to be \textbf{broken} to $H$.

    After reduction, the $G$-connection can locally be decomposed as follows:
    \begin{gather}
        \iota^*\omega_{\mathfrak{g}} = \omega_{\mathfrak{h}} + \gamma\,,
    \end{gather}
    where $\iota:P_H\hookrightarrow P$ denotes the reduction morphism and $\gamma$ is a tensorial $(\mathrm{Ad}_H,\mathfrak{m})$-form with $\mathfrak{m}$ the complement of $\mathfrak{h}$ in $\mathfrak{g}$.\footnote{To make $\mathrm{Ad}_H$ a well-defined representation on $\mathfrak{m}$, the latter is usually constructed as an orthogonal complement with respect to an $\mathrm{Ad}$-invariant metric on $\mathfrak{g}$. In general, the pair $(\mathfrak{g},\mathfrak{h})$ is required to be reductive (\cref{bundle:klein_reductive}).}

    For the Higgs field $\phi:P\rightarrow\mathcal{M}_0$ and, in fact, for any equivariant map $\phi:P\rightarrow\mathcal{M}_0$ such that $\nabla^H(\phi\circ\iota)=0$, the covariant derivative satisfies
    \begin{gather}
        \nabla_X\phi = (\rho_{e,\ast}\circ\gamma)(X)m_0\,.
    \end{gather}
    The generators $\rho_{e,\ast}(\gamma^i_\mu)m_0$, for $i=1,\ldots,\dim(\mathfrak{m})$, are called the \textbf{(Nambu--)Goldstone bosons}. Since $\dim(\mathfrak{m})=\dim(G)-\dim(H)$, there are $\dim(G)-\dim(H)$ Goldstone fields.\index{Goldstone!boson}\index{Nambu|seealso{Goldstone boson}} As seen above, after reduction, the connection form (gauge field) splits into a connection form for the smaller symmetry group and a set of new (massive) fields. The new connection form is obtained by trivially extending $\omega_{\mathfrak{h}}$ to a connection on $P$ through $G$-equivariance. For such connections, \cref{bundle:connection_reducibility} implies that $\nabla\phi=0$ (this also follows from the expression above since $\gamma$ vanishes for this kind of connection). This is exactly the second condition for the Higgs vacuum.
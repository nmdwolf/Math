\chapter{Gauge Theory}\label{chapter:gauge_theory}

    References for this chapter are~\citet{sontz_principal_2015,schuller_lectures_2016,nash_topology_2011,belgun_gauge_2024}. The section on the Higgs mechanism is mainly based on~\citet{choquet-bruhat_analysis_2000}. Using the tools of differential geometry, as presented in \cref{chapter:bundles} and onwards, one can introduce a general formulation of gauge theories and, in particular, Yang--Mills theories.

\section{Gauge invariance}

    Consider a general Lie group $G$, often called the \textbf{gauge group}, acting on a vector bundle $E$ with typical fibre $\mathcal{H}$ over a base manifold $M$. This bundle is, in general, obtained as a bundle associated to the frame bundle $FM$. Locally, a general gauge transformation has the form
    \begin{gather}
        \label{gauge:gauge_transformation}
        \psi'(x) = U(x)\psi(x)\,,
    \end{gather}
    where $\psi,\psi':M\rightarrow\mathcal{H}$ are trivializations of local sections of $E$ and $U:M\rightarrow G$ encodes the local behaviour of the gauge transformation. It is assumed to be a unitary representation with respect to the Hilbert structure on $\mathcal{H}$. Globally, the gauge transformations are given by the vertical automorphisms.

    \begin{axiom}[Local gauge principle]\index{gauge!principle}
        The Lagrangian functional $L[\psi]$ is invariant under the action of the gauge group $G$:
        \begin{gather}
            L[U\psi] = L[\psi]\,.
        \end{gather}
    \end{axiom}

    Generally, this gauge invariance can be achieved in the following way. Denote the Lie algebra corresponding to $G$ by $\mathfrak{g}$. Because the gauge transformation is local, the information on how it varies from point to point should be able to propagate through space(time). This is done by introducing a new (local) field $B_\mu(x)$, called the \textbf{gauge field}. The most elegant formulation uses the concept of covariant derivatives.
    \newdef{Covariant derivative}{\index{covariant!derivative}\index{minimal!coupling}
        When gauging a symmetry group, the ordinary partial derivatives are replaced by the covariant derivative
        \begin{gather}
            \mathcal{D}_\mu = \partial_\mu + igB_\mu(x)\,,
        \end{gather}
        where $B_\mu:M\rightarrow\mathbb{R}$ are the coefficients of the gauge field with respect to a chosen basis of $\mathfrak{g}$. This procedure is called \textbf{minimal coupling}. It should be noted that the explicit action of the covariant derivative depends on the chosen representation of $\mathfrak{g}$ (or $G$) on $\mathcal{H}$. Furthermore, one should pay attention to the fact that the physicist convention was used, where one multiplies the gauge field $B$ by a factor $ig$.\footnote{The imaginary unit turns anti-Hermitian fields into Hermitian fields.}
    }

    So, to achieve gauge invariance, one should replace all derivatives by covariant derivatives. However, for this to be a well-defined operation, one should check that the covariant derivative itself satisfies the local gauge principle, i.e.~$\mathcal{D}'\psi'=U\mathcal{D}\psi$ (from here on the coordinate-dependence of all fields will be suppressed).
    \begin{align}
        U^{-1}\left(\pderiv{}{x^\mu} + igB_\mu'\right)\psi' &= U^{-1}\left(\pderiv{}{x^\mu} + igB_\mu'\right)U\psi\nonumber\\
        &= U^{-1}\pderiv{U}{x^\mu}\psi + \pderiv{\psi}{x^\mu} + igU^{-1}B_\mu'U\psi\,.
    \end{align}
    This expression can only be equal to $\mathcal{D}\psi$ if
    \begin{gather}
        igB_\mu = U^{-1}\pderiv{U}{x^\mu} + igU^{-1}B_\mu'U\,,
    \end{gather}
    which can be rewritten as
    \begin{gather}
        B_\mu' = UB_\mu U^{-1} - \frac{1}{ig}(\partial_\mu U)U^{-1}
    \end{gather}
    or, in coordinate-independent form, as
    \begin{gather}
        B' = UBU^{-1} - \frac{1}{ig}\dr UU^{-1}\,.
    \end{gather}
    Up to scale conventions, this is exactly the content of \cref{bundle:local_compatibility} and \cref{bundle:mc_pullback} appearing in the study of principal connections. This should not come as a surprise, since the physical fields are sections of associated vector bundles and, hence, the principal bundle structure lurks in the background. Adding interactions is mathematically equivalent to coupling the space(time) manifold to a principal bundle.

    \begin{example}[QED]
        For quantum electrodynamics, which has $\mathrm{U}(1)\cong S^1$ as its gauge group, the parametrization $U(x)=e^{ie\chi(x)}$ is used with $\chi:\mathbb{R}^n\rightarrow\mathbb{R}$. Minimal coupling leads to
        \begin{align}
            \partial_\mu &\longrightarrow\mathcal{D}_\mu = \partial_\mu + ieA_\mu\,,\\
            A_\mu &\longrightarrow A_\mu' = A_\mu - \partial_\mu\chi\,,
        \end{align}
        where $A_\mu$ is the classic electromagnetic potential. These are the formulas introduced in \cref{chapter:em}.
    \end{example}

\section{Spontaneous symmetry breaking}

    \begin{theorem}[Goldstone]\index{Goldstone}
        Consider a field theory with gauge group $G$ and denote the generators of the corresponding Lie algebra by $X_a$. Generators that do not annihilate the vacuum, i.e.~$X_av\neq0$, or, equivalently, transformations that leave the vacuum invariant, correspond to massless scalar particles.
    \end{theorem}
    The massless bosons from this theorem are called \textbf{Goldstone bosons}.

    \begin{theorem}[Elitzur]\index{Elitzur}\label{gauge:elitzur}
        The only operators in a lattice gauge theory\footnote{A proof for continuum field theories does not exist. However, since nonperturbative field theories are usually constructed through a limit procedure of lattice theories, this does have an impact.} with a nonvanishing VEV are those invariant with respect to local gauge transformations.
    \end{theorem}
    \begin{result}
        Gauge symmetries cannot be spontaneaously broken.
    \end{result}

\subsection{Higgs mechanism}\index{Higgs!mechanism}\index{vacuum!Higgs}\label{section:higgs_mechanism}

    In \cref{bundle:section_bijection}, the equivariant maps corresponding to global sections of a principal bundle were called Higgs fields. In this section, a clarification for this terminology is given. The \textbf{Higgs vacuum} of a $G$-gauge theory, described by a principal bundle $P$, with a $G$-invariant potential $V$ is given by the solutions of the following equations:
    \begin{align}
        V(\phi) &= 0\\
        \nabla\phi &= 0\,,
    \end{align}
    where $\nabla$ is a covariant derivative on $P$ and $\phi$ is a section of some associated (finite-rank) vector bundle $P\times_\rho\mathcal{H}$. If the space of solutions $\mathcal{M}$ to the first equation admits a transitive $G$-action, i.e.~it is a homogeneous space, then, by \cref{group:transitive_action_property}, one can write
    \begin{gather}
        \mathcal{M}\cong G/H\,,
    \end{gather}
    where $H$ is the isotropy group of any given solution. More generally, when the action is not transitive, the solution manifold is still the union of $G$-orbits, all of the form $G/H$ with $H$ the isotropy group of a point in the orbit.

    Now, consider a specific choice of vacuum $m_0\in\mathcal{M}$. If the whole theory were to be $G$-invariant, like the potential $V$, this corresponds to an equivariant map
    \begin{gather}
        \phi:P\rightarrow \mathcal{M}_0\cong G/H\,,
    \end{gather}
    where $\mathcal{M}_0$ is the orbit of $m_0$. This field is called the \textbf{Higgs field} in the physics literature. (For this reason, all such equivariant morphism and their associated sections are called Higgs fields.) The specific choice of vacuum, which generically has the smaller symmetry group $H$, induces by \cref{bundle:reduction_classification} a reduction of the structure group from $G$ to $H$ and, consequently, the symmetry group is said to be \textbf{broken} to $H$.

    After reduction, the $G$-connection can locally be decomposed as follows:
    \begin{gather}
        \iota^*\omega_{\mathfrak{g}} = \omega_{\mathfrak{h}} + \gamma\,,
    \end{gather}
    where $\iota:P_H\hookrightarrow P$ denotes the reduction morphism and $\gamma$ is a tensorial $(\mathrm{Ad}_H,\mathfrak{m})$-form with $\mathfrak{m}$ the complement of $\mathfrak{h}$ in $\mathfrak{g}$.\footnote{To make $\mathrm{Ad}_H$ a well-defined representation on $\mathfrak{m}$, the latter is usually constructed as an orthogonal complement with respect to an $\mathrm{Ad}$-invariant metric on $\mathfrak{g}$. In general, the pair $(\mathfrak{g},\mathfrak{h})$ is required to be reductive (\cref{bundle:klein_reductive}).}

    For the Higgs field $\phi:P\rightarrow\mathcal{M}_0$ and, in fact, for any equivariant map $\phi:P\rightarrow\mathcal{M}_0$ such that $\nabla^H(\phi\circ\iota)=0$, the covariant derivative satisfies
    \begin{gather}
        \nabla_X\phi = (\rho_{e,\ast}\circ\gamma)(X)m_0\,.
    \end{gather}
    The generators $\rho_{e,\ast}(\gamma^i_\mu)m_0$, for $i=1,\ldots,\dim(\mathfrak{m})$, are called the \textbf{(Nambu--)Goldstone bosons}. Since $\dim(\mathfrak{m})=\dim(G)-\dim(H)$, there are $\dim(G)-\dim(H)$ Goldstone fields.\index{Goldstone!boson}\index{Nambu|seealso{Goldstone boson}} As seen above, after reduction, the connection form (gauge field) splits into a connection form for the smaller symmetry group and a set of new (massive) fields. The new connection form is obtained by trivially extending $\omega_{\mathfrak{h}}$ to a connection on $P$ through $G$-equivariance. For such connections, \cref{bundle:connection_reducibility} implies that $\nabla\phi=0$ (this also follows from the expression above since $\gamma$ vanishes for this kind of connection). This is exactly the second condition for the Higgs vacuum.

    Now, what about Elitzur's theorem~\ref{gauge:elitzur}? If its generalization to field theories holds, the above considerations should not hold. Two solutions exist:
    \begin{enumerate}
        \item Realize that the Higgs mechanism can be restated without symmetry breaking.
        \item Realize that the symmetry breaking applies to a global symmetry group and not a local one.
    \end{enumerate}
    Although the first option is a very interesting approach, only the second one will be covered here. \todo{MIGHT ADD FIRST OPTION TOO}

    The crucial point is that the group being broken is the \textbf,{residual symmetry group}, i.e.~the symmetry group that remains after gauge-fixing the theory. When fully fixing a gauge, this residual group coincides with the center of the gauge group $G$, e.g.~$\mathrm{U}(1)$ for $\mathrm{U}(1)$ or $\mathbb{Z}_n$ for $\mathrm{SU}(n)$. Consider for simplicity the typical Mexican hat potential in Yang--Mills theory:
    \begin{gather}
        \mathcal{L} := -\frac{1}{2}\mathrm{tr}\bigl(F_{\mu\nu}F^{\mu\nu}\bigr)+|D_\mu\phi|^2 - V(\phi)\,,
    \end{gather}
    where $V(\phi):=m^2|\phi|^2+\lambda|\phi|^4$ with $\lambda>0$ (note that $m^2<0$ is required for symmetry breaking, i.e.~the mass term is required to be `tachyonic'). The minimum of this potential is achieved for fields of modulus
    \begin{gather}
        |\phi|^2 = -\frac{m^2}{\lambda} \equiv \frac{\nu}{\sqrt{2}}\,.
    \end{gather}
    The states within the level set $V^{-1}(\nu)$ can be parametrized as follows:
    \begin{gather}
        \phi(x) = \frac{\nu}{\sqrt{2}}e^{i\pi(x)/\nu}\,,
    \end{gather}
    where $\pi(x)$ represents the (massless) Nambu--Goldstone boson.\index{Goldstone!boson}\index{Nambu|seealso{Goldstone boson}} Inserting this expression into the Lagrangian gives an expression where the gauge fields $A_\mu$ are replaced by $A_\mu-\frac{1}{q\nu}\partial_\mu\pi$, where $q$ is the charge factor. To get to the Lagrangian of a massive gauge field, the unitary gauge is chosen at this point:
    \begin{gather}
        A_\mu-\frac{1}{q\nu}\partial_\mu\pi\longrightarrow W_\mu\,.
    \end{gather}
    This gives
    \begin{gather}
        \mathcal{L}\sim -\frac{1}{2}\mathrm{tr}\bigl(F_{\mu\nu}F^{\mu\nu}\bigr)+(q\nu)^2W_\mu W^\mu\,.
    \end{gather}
    However, it should be noted that the gauge-fixed field $W_\mu$ actually has a gauge-invariant representation:
    \begin{gather}
        W_\mu = \frac{i}{q}\widetilde{\phi}^*D_\mu\widetilde{\phi}\,,
    \end{gather}
    with
    \begin{gather}
        \widetilde{\phi} := \frac{\phi}{|\phi|}\,.
    \end{gather}

    \todo{COMPLETE}

\section{Topological effects}
\subsection{Large gauge transformations}

    As stated in \cref{bundle:vertical_automorphism}, the gauge transformations in a general gauge theory are given by the vertical automorphisms of the underlying principal bundle. When quantizing the theory as in \cref{section:quantum_constrained}, one always starts with a constraint algebra. By \cref{lie:prop_connected} and \cref{lie:exp_result}, however, the exponential map only generates the identity component of the full gauge group. It follows that only the transformations in this connected component give rise to physically equivalent states. The gauge transformations not homotopic to the identity are called \textbf{large gauge transformations}. 

    Consider a Yang--Mills theory with model Lie group $G$. In vacuum, the field strength should vanish and, hence, the solutions are `pure gauge', i.e.
    \begin{gather}
        A_\mu = ig(x)\partial_\mu g^{-1}(x)
    \end{gather}
    for some smooth function $g:\mathbb{R}^n\rightarrow G$. To obtain a finite (Euclidean) action, $g$ should be a constant at infinity and, hence, we obtain a map $g:S^n\rightarrow G$ from the compactification of spacetime to the model Lie group. It follows that the vacua of such a theory are classified by $\pi_n(G)$. For example, when working with $G=\mathrm{SU}(2)$ in the \textit{temporal gauge} such that $n=3$, the vacua are classified by their Pontryagin index (or \textbf{winding number}) since $\pi_3(S^3)\cong\mathbb{Z}$.\footnote{In fact, this holds for all $\mathrm{SU}(N)$ with $N\geq2$ as the homotopy group $\pi_3$ stabilizes at $N=2$.}\index{Pontryagin!index}\index{winding number}\index{gauge!temporal} If the total gauge group were connected, the winding number of a vacuum would be fixed. However, in general, large gauge transformations can map vacua with different winding numbers into each other. These vacua are also called \textbf{topological vacua}.\index{vacuum!topological}

    \todo{COMPLETE (e.g. Witten effect)}

\subsection{Instantons}\index{instanton}

    Consider the topological vacua $|n\rangle$  from the previous section. Since large gauge transformations do not preserve the winding number, these vacua are not gauge invariant and, accordingly, are not proper vacua. The solution is to take a coherent superposition:
    \begin{gather}
        |\theta\rangle := \sum_{n\in\mathbb{Z}}e^{in\theta}|n\rangle\,,
    \end{gather}
    with $\theta\in[0,2\pi[$. These vacua, which are gauge invariant, are called \textbf{$\theta$-vacua}.\index{vacuum!$\theta$} It should be noted that states with different values of $\theta$ belong to different superselection sectors.
    
    The different topological vacua are also connected by so-called \textbf{instanton solutions}, i.e.~solutions with finite (Euclidean) action. Instantons that connect vacua $|n\rangle$ to $|n\pm1\rangle$ are called \textbf{Belavin--Polyakov--Schwarz--Tyupkin (BPST) instantons}.\index{instanton!BPST}

    The Pontryagin index in the previous section can also be expressed as the second Chern number (cf.~\cref{bundle:chern_class}) when regarding the instantons as solutions and, hence, principal $\mathrm{SU}(2)$-bundles over compactified spacetime $S^4$:
    \begin{gather}
        \{S^4\rightarrow B\mathrm{SU}(2)\}_{/\sim} \cong \pi_4\bigl(B\mathrm{SU}(2)\bigr)\cong\pi_3\bigl(\mathrm{SU}(2)\bigr)\,,
    \end{gather}
    where \cref{bundle:delooping} and \cref{topology:desuspension} were used in the second step. The second Chern number can be expressed as follows:
    \begin{gather}
        c_2(F) = \frac{\mathrm{tr}(F\wedge F)}{8\pi^2}\,.
    \end{gather}
    This term can also be added to the Lagrangian to obtain
    \begin{gather}
        \mathcal{L}_\theta := \mathcal{L}_{\text{YM}} + \frac{\theta}{16\pi^2}\Int_{\mathbb{R}^4}\mathrm{tr}(F\wedge F)\,.
    \end{gather}
    The problem here is that this extra term is not, for example, invariant under inversions (parity transformations), unlike the standard Yang--Mills term $\mathrm{tr}(F\wedge\ast F)$. The same issue holds for charge conjugation. The only situation where no \textbf{CP-problem} occurs, is when $\theta=0$.\index{CP-problem} In nature, however, no such violation is observed and, hence, a fine-tuning problem arises (the magnitude of $\theta$ is experimentally bounded by $10^{-10}$)!

    \todo{COMPLETE}
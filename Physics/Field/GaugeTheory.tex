\chapter{Gauge Theory}\label{chapter:gauge_theory}

    References for this chapter are \cite{principal_bundles, sen_nash, schuller, gauge1}. The section on the \textit{Higgs mechanism} is mainly based on \cite{AMP2}. Using the tools of differential geometry, as presented in Chapter \ref{chapter:bundles} and onwards, one can introduce a general formulation of gauge theories and, in particular, Yang-Mills theories.

\section{Gauge invariance}

    Consider a general Lie group $G$, often called the \textbf{gauge group}, acting on a vector bundle with typical fibre $\mathcal{H}$ over a base manifold $M$. This bundle is in general obtained as an associated bundle of the frame bundle $FM$. A general gauge transformation has the form
    \begin{gather}
        \label{gauge:gauge_transformation}
        \psi'(x) = U(x)\psi(x),
    \end{gather}
    where $\psi,\psi':M\rightarrow\mathcal{H}$ are sections of $\mathcal{H}$ and $U:M\rightarrow G$ encodes the local behaviour of the gauge transformation. It is assumed to be a unitary representation with respect to the Hilbert structure on $\mathcal{H}$. As such, a gauge transformation constitutes a vertical automorphism of the vector bundle.

    \begin{axiom}[Local gauge principle]
        The Lagrangian functional $\mathcal{L}[\psi]$ is invariant under the action of the gauge group $G$:
        \begin{gather}
            \mathcal{L}[U\psi] = \mathcal{L}[\psi].
        \end{gather}
    \end{axiom}

    Generally this gauge invariance can be achieved in the following way. Denote the Lie algebra corresponding to $G$ by $\mathfrak{g}$. Because the gauge transformation is local, the information on how it varies from point to point should be able to propagate through space(time). This is done by introducing a new field $B_\mu(x)$, called the \textbf{gauge field}. The most elegant formulation uses the concept of covariant derivatives:
    \newdef{Covariant derivative}{\index{covariant!derivative}\index{minimal!coupling}
        When gauging a symmetry group, the ordinary partial derivatives are replaced by the covariant derivative
        \begin{gather}
            \mathcal{D}_\mu = \partial_\mu + igB_\mu(x),
        \end{gather}
        where $B_\mu:M\rightarrow\mathfrak{g}$ is a new field with values in the Lie algebra of the gauge group. This procedure is called \textbf{minimal coupling}. It should be noted that the explicit action of the covariant derivative depends on the chosen representation of $\mathfrak{g}$ on $\mathcal{H}$. Furthermore, one should pay attention to the fact that the physics convention was used where one multiplies the gauge field $B$ by a factor $ig$.\footnote{The imaginary unit turns anti-Hermitian fields into Hermitian fields.}
    }

    So, to achieve gauge invariance one should replace all derivatives by covariant derivatives. However, for this to be a well-defined operation, one should check that the covariant derivative itself satisfies the local gauge principle, i.e. $\mathcal{D}'\psi' = U\mathcal{D}\psi$ (from here the coordinate-dependence of all fields will be supressed):
    \begin{align}
        U^{-1}\left(\pderiv{}{x^\mu} + igB_\mu'\right)\psi' &= U^{-1}\left(\pderiv{}{x^\mu} + igB_\mu'\right)U\psi\nonumber\\
        &= U^{-1}\pderiv{U}{x^\mu}\psi + \pderiv{\psi}{x^\mu} + igU^{-1}B_\mu'U\psi.
    \end{align}
    This expression can only be equal to $\mathcal{D}\psi$ if
    \begin{gather}
        igB_\mu = U^{-1}\pderiv{U}{x^\mu} + igU^{-1}B_\mu'U,
    \end{gather}
    which can be rewritten as
    \begin{gather}
        B_\mu' = UB_\mu U^{-1} - \frac{1}{ig}(\partial_\mu U)U^{-1}
    \end{gather}
    or, in coordinate-independent form, as
    \begin{gather}
        \mathbf{B}' = U\mathbf{B}U^{-1} - \frac{1}{ig}dUU^{-1}.
    \end{gather}
    Up to conventions this is exactly the content of Equations \eqref{bundle:local_compatibility} and \eqref{bundle:mc_pullback} appearing in the study of connections on principal bundles. This should not come as a surprise since the physical fields are sections of associated vector bundles and, hence, the principal bundle structure lurks in the background. Adding interactions is mathematically equivalent to coupling the physical manifold to a principal bundle.

    \begin{example}[QED]
        For quantum electrodynamics, which has $\mathrm{U}(1)\cong S^1$ as its gauge group, the parametrization $U(x) = e^{ie\chi(x)}$ is used with $\chi:\mathbb{R}^n\rightarrow\mathbb{R}$. Minimal coupling leads to
        \begin{align}
            \partial_\mu &\longrightarrow\mathcal{D}_\mu = \partial_\mu + ieA_\mu\\
            A_\mu &\longrightarrow A_\mu' = A_\mu - \partial_\mu\chi,
        \end{align}
        where $A_\mu$ is the classic electromagnetic potential. These are the formulas that introduced in Chapter \ref{chapter:maxwell}.
    \end{example}

\section{Spontaneous symmetry breaking}

    \begin{theorem}[Goldstone]\index{Goldstone}
        Consider a field theory with Lie group $G$ and denote the generators of the corresponding Lie algebra by $X_a$. Generators that do not destroy the vacuum $X_av\neq0$ or, equivalently, transformations that leave the vacuum invariant, correspond to massless scalar particles.
    \end{theorem}
    The massless bosons from this theorem are called \textbf{Goldstone bosons}.

\subsection{Higgs mechanism}\index{Higgs!mechanism}

    In Property \ref{bundle:section_bijection} the equivariant maps corresponding to global sections of a principal bundle were called \textbf{Higgs fields}. In this section a clarification of the terminology is given.

    The Higgs vacuum of a $G$-gauge theory, described by a principal bundle $P$, with a $G$-invariant potential $V$ is given by the solutions of the following equations:
    \begin{align}
        V(\phi) &= 0,\\
        \nabla\phi &= 0,
    \end{align}
    where $\nabla$ is the covariant derivative and $\phi$ is a section of some associated finite-rank vector bundle $P\times_\rho E$. If the space of solutions $\mathcal{M}$ to the first equation admits a transitive $G$-action, i.e. it is a homogeneous space, by Property \ref{group:transitive_action_property} one can write
    \begin{gather}
        \mathcal{M}\cong G/H,
    \end{gather}
    where $H$ is the isotropy group of any solution. More generally, when the action is not transitive, the solution manifold is still the union of $G$-orbits, all of the form $G/H$ with $H$ the isotropy group of a point of the orbit.

    Now, consider a specific choice of vacuum $m_0\in\mathcal{M}$. If the whole theory were to be $G$-invariant, like the potential $V$, this corresponds to an equivariant map $\phi:P\rightarrow \mathcal{M}_0\cong G/H$, where $\mathcal{M}_0$ is the orbit of $m_0$. This field is called the \textbf{Higgs field} in the physics literature (for this reason all such equivariant morphism (and their associated sections) are called Higgs fields). The specific choice of vacuum, which generically has a smaller symmetry group $H$, induces by Property \ref{bundle:reduction_classification} a reduction of the structure group from $G$ to $H$ (the symmetry group is said to be \textbf{broken} to $H$).

    After reduction, the $G$-connection can locally be decomposed as follows:
    \begin{gather}
        \omega_\mathfrak{g} = \omega_\mathfrak{h} + \gamma,
    \end{gather}
    where the left-hand side denotes the pullback of the $G$-connection along the reduction $\iota:P_H\hookrightarrow P$ and $\gamma$ is a tensorial form of type $(\mathrm{Ad}_H,\mathfrak{m})$ with $\mathfrak{m}$ the complement of $\mathfrak{h}$ in $\mathfrak{g}$.\footnote{To make $\mathrm{Ad}_H$ a well-defined representation on $\mathfrak{m}$, the latter is usually constructed as an orthogonal complement with respect to an $\mathrm{Ad}$-invariant metric on $\mathfrak{g}$. In general, the pair $(\mathfrak{g},\mathfrak{h})$ is required to be reductive \ref{bundle:klein_reductive}.}

    For the Higgs field $\phi:P\rightarrow\mathcal{M}_0$, in fact for any equivariant map $\phi:P\rightarrow\mathcal{M}_0$ such that $\nabla^H(\phi\circ\iota)=0$, the covariant derivative satisfies
    \begin{gather}
        \nabla_X\phi = (\rho_{e,\ast}\circ\gamma)(X)m_0.
    \end{gather}
    The generators $\rho_{e,\ast}(\gamma^i_\mu)m_0$, where $i=1,\ldots,\dim(\mathfrak{m})$, are called the \textbf{(Nambu-)Goldstone bosons}. Since $\dim(\mathfrak{m})=\dim(G)-\dim(H)$, there are $\dim(G)-\dim(H)$ Goldstone fields.\index{Goldstone!boson}\index{Nambu|seealso{Goldstone boson}} After reduction, the connection forms (gauge field) splits into a connection form for the smaller symmetry group and set of new (massive) fields. The new connection form is obtained by trivially extending $\omega_\mathfrak{h}$ to a connection on $P$ through $G$-equivariance. For such connection that are obtained as the extension of an $H$-connection on the reduction $P_H$, Property \ref{bundle:connection_reducibility} says that $\nabla\phi=0$ (this also follows from the expression above since $\gamma$ vanishes for this kind of connection). This is exactly the second condition for the Higgs vacuum.
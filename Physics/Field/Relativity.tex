\chapter{Special Relativity}

    In this chapter, as will be the case in the chapters on quantum field theory, the mostly-minuses convention for the Minkowski signature is adopted unless stated otherwise, i.e.~the signature is $(+,-,-,-)$. Furthermore, natural units will be used unless stated otherwise, i.e.~$\hbar = c = 1$. This follows the introductory literature such as~\citet{greiner_field_1996,peskin_introduction_1995}.

\section{Lorentz transformations}

    \begin{notation}\index{Lorentz!factor}
        In the context of special relativity it is often useful to introduce the following quantities:
        \begin{align}
            \beta &:= \frac{v}{c}\,,\\
            \label{relativity:lorentz_factor}
            \gamma &:= \frac{1}{\sqrt{1 - \beta^2}}\,.
        \end{align}
        The latter quantity is called the \textbf{Lorentz factor}.
    \end{notation}
    \newformula{Lorentz transformations}{\label{relativity:lorentz_transformations}
        Let $\mathbf{V}$ be a 4-vector. A Lorentz boost along the $x^1$-axis is given by the following transformation:
        \begin{gather}
            \begin{aligned}
                V'^0 &= \gamma\left(V^0 - \beta V^1\right)\\
                V'^1 &= \gamma\left(V^1 - \beta V^0\right)\\
                V'^2 &= V^2\\
                V'^3 &= V^3.
            \end{aligned}
        \end{gather}
    }
    \begin{remark}
        Setting $c=+\infty$ in the previous formulas recovers the Galilei transformations from classical mechanics (cf.~In\"on\"u--Wigner contractions (\cref{lie:inonu_wigner})).
    \end{remark}

\section{Energy and momentum}

    \newformula{4-velocity}{\index{velocity}\label{relativity:4_velocity}
        In analogy to the definition of velocity in classical mechanics, the 4-velocity is defined as follows:
        \begin{gather}
            \mathbf{U} := \left(\deriv{x^0}{\tau},\deriv{x^1}{\tau},\deriv{x^2}{\tau},\deriv{x^3}{\tau}\right)\,.
        \end{gather}
        By applying the formulas for proper time and time dilatation, we obtain:
        \begin{gather}
            \mathbf{U} = \left(\gamma c,\gamma\vector{u}\right)\,.
        \end{gather}
    }
    \newformula{4-momentum}{\index{momentum}\label{relativity:4_momentum}
        The 4-momentum is defined as follows:
        \begin{gather}
            \mathbf{p} = m_0\mathbf{U}\,,
        \end{gather}
        or, after defining $E := cp^0$:
        \begin{gather}
            \mathbf{p} = \left(\frac{E}{c},\gamma m_0\vector{u}\right)\,.
        \end{gather}
    }

    \newdef{Relativistic mass}{\index{mass}
        The factor $m:=\gamma m_0$ in the momentum 4-vector is called the relativistic mass. By introducing this quantity, the classical formula $\vector{p} = m\vector{u}$ for the 3-momentum can be generalized to 4-momenta $\mathbf{p}$.
    }

    \begin{formula}[Relativistic energy relation]\index{energy}\index{Einstein!energy relation}\label{relativity:relativistic_energy}
        \begin{gather}
            E^2 = p^2c^2 + m^2c^4
        \end{gather}
        This formula is often called the \textbf{Einstein relation}.
    \end{formula}

\section{Action principle}

    The main guiding principles for writing down a relativistic action for a point particle are locality and geometry. The latter means that one should only use geometric quantities, i.e.~diffeomorphism invariant quantities, while the former means that these should only depend on local information. For a single particle the most obvious action would be one that is proportional to the proper time along the worldline of the particle or, more invariantly, the arc length of the worldline:
    \begin{gather}
        S_{\text{point}}\sim\Int_\gamma ds\,.
    \end{gather}
    To get the units right, one should multiply by suitable Lorentz-invariant constants:
    \begin{gather}
        \label{relativity:worldline_action}
        S_{\text{point}}:=mc\Int_\gamma ds\,.
    \end{gather}
    By reparametrization-invariance one can choose a specific time-coordinate, e.g.~$\tau=ct$. In this coordinate system the action becomes
    \begin{gather}
        S_{\text{point}}=mc\Int_\gamma\sqrt{1-\frac{v^2}{c^2}}c\,dt\,,
    \end{gather}
    with $v$ the speed of the particle. The Lagrangian density can be Taylor expanded as
    \begin{gather}
        L_{\text{point}}=mc^2 - \frac{1}{2}mv^2 + \cdots\,,
    \end{gather}
    which recovers (up to a constant) the classical Lagrangian of a massive point particle when the speed is small $v\ll c$.

    When trying to quantize this action, however, a problem occurs. After a Legendre transformation, the Hamiltonian becomes $H=\sqrt{p^2+m^2c^4}$ (this is just the Einstein relation~\ref{relativity:relativistic_energy}). When applying the ordinary Dirac procedure $p_\mu\longrightarrow i\partial_\mu$, this becomes a nonlocal operator (the whole reason for why Dirac introduced spinors).

    Instead of passing to a spinor framework, one can try to write down an equivalent action that gives rise to a local Hamiltonian. One possibility is to pass to \textbf{light cone coordinates}:\index{light cone!coordinates}
    \begin{gather}
        x^\pm := x^0\pm x^1\,.
    \end{gather}
    However, here one makes a specific split of coordinates, which ruins Lorentz-invariance. A better idea is to introduce a dynamical Lagrange multiplier (from here on natural units are used):
    \begin{gather}
        S_{\text{point}} := \frac{1}{2}\Int_\gamma\left[\eta^{-1}\left(\deriv{x^\mu}{\tau}\right)^2-m^2\eta\right]d\tau\,.
    \end{gather}
    The equation of motion for $\eta$ is algebraic and gives
    \begin{gather}
        \eta = \frac{1}{m}\sqrt{-g_{\mu\nu}\deriv{x^\mu}{\tau}\deriv{x^\nu}{\tau}}\,.
    \end{gather}
    To find an interpretation of this multiplier, it is useful to consider the case where a metric $h$ is introduced on the worldline. This implies that the action has to be `covariantized':
    \begin{gather}
        \Int_\gamma\left(g_{\mu\nu}\deriv{x^\mu}{\tau}\deriv{x^\mu}{\tau}+m^2\right)d\tau\longrightarrow\Int_\gamma\sqrt{h}\left(hg_{\mu\nu}\deriv{x^\mu}{\tau}\deriv{x^\mu}{\tau}+m^2\right)d\tau\,.
    \end{gather}
    From this perspective it is clear that the Lagrange multiplier can be viewed as the square root of a dynamical metric on the worldline. It is an example of a \textit{vielbein} (in this case an `\textit{einbein}'). Furthermore, this action is a so-called \textit{nonlinear $\sigma$-model}.


\chapter{General Relativity}\label{chapter:GR}

    References for this chapter are~\citet{rovelli_covariant_2014,misner_gravitation_2017}. See \cref{chapter:riemann} for an introduction to the theory of Riemannian geometry. The mathematical background for the section on the \textit{tetradic formulation} of GR can be found in \cref{section:cartan_geometry}.

    In this chapter, the signature convention of the previous chapter is reversed. For general relativity it is often more convenient to use the mostly-pluses convention (this simply reduces the number of minus signs).

\section{Causal structure}\index{causal}

    \newdef{Null coordinate}{\index{null!vector}\index{lightlike}\index{timelike}\index{spacelike}
        Consider a vector $v\in T_pM$ on a Lorentzian manifold $(M,g)$. This vector is said to be null or \textbf{lightlike} if it satisfies the following condition:
        \begin{gather}
            g_p(v,v) = 0\,.
        \end{gather}
        One can also define \textbf{timelike} and \textbf{spacelike} vectors in a similar way as those vectors having negative and positive norm, respectively.\footnote{For a mostly-minuses signature one should interchange these definitions.} Spacelike, lightlike and timelike curves are defined as curves for which every tangent vector is respectively spacelike, lightlike or timelike. A curve is said to be \textbf{causal} if its tangent vectors are time- or lightlike.
    }

    \newdef{Time-orientability}{\index{orientation!time}
        A Lorentzian manifold is said to be time-orientable if there exists a nowhere-vanishing, timelike vector field. It should be noted that, in contrast to ordinary orientability, this notion is not purely topological. Moreover, neither orientability nor time-orientability implies the other. They are independent notions.

        The choice of a time-orienting vector field $\tau$ divides the set of timelike vectors at a point into two equivalence classes. A curve $\gamma$ is said to be future-directed (resp.~past-directed) if $g(\tau,\dot{\gamma})<0$ (resp.~$g(\tau,\dot{\gamma})>0$).
    }

    \newdef{Causal cone}{
        Let $M$ be a Lorentzian manifold. The causal cone of a point $p\in M$ is defined as the set $J^-(p)\cup J^+(p)\subset M$ of points that are connected to $p$ by a (smooth) causal curve. The past and future cones are respectively defined as the sets of points that can be connected to $p$ by a future-directed or past-directed casual curve. The boundaries of these causal cones are called the causal \textbf{lightcones}, sometimes denoted by $V^\pm(p)$.
    }
    \newdef{Causal closure}{\label{relativity:causal_closure}
        Let $S$ be a subset of a Lorentzian manifold. The \textbf{causal complement} of $S$ consists of all points that cannot be causally connected to any point in $S$. The causal closure of $S$ is defined as the causal complement of the causal complement of $S$. A \textbf{causally closed set} is then defined as a set which is equal to its causal closure.
    }

    \newdef{Globally hyperbolic manifold}{\index{hyperbolic!manifold}
        A Lorentzian manifold $M$ that does not contain closed causal curves and for which $J^+(p)\cap J^-(q)$ is compact for any two points $p,q\in M$.
    }

    \newdef{Stationary spacetime}{
        A spacetime $(M,g)$ is called stationary if there exists a timelike Killing vector. By the \textit{flowbox theorem}, there always exists a coordinate chart such that locally one can choose the Killing vector field to be $\partial_0$ and, hence, a spacetime is stationary if one can find a coordinate system for which the metric coefficients are time-independent.
    }

\section{Einstein field equations}

    \newformula{Einstein field equations}{\index{Einstein!field equations}\index{stress-energy tensor}\label{relativity:einstein_field_equations}
        The Einstein field equations without a cosmological constant $\Lambda$ read as follows (all fundamental constants are shown for completeness):
        \begin{gather}
            G_{\mu\nu} = \frac{8\pi G}{c^4}T_{\mu\nu}\,,
        \end{gather}
        where $G_{\mu\nu}$ is the Einstein tensor~\eqref{riemann:einstein_tensor} and $T_{\mu\nu}$ is the stress-energy tensor~\eqref{field:stress_energy_tensor}.

        By taking the trace of both sides one obtains $T = -R$ and, hence, the Einstein field equations can be rewritten as
        \begin{gather}
            R_{\mu\nu} = \widehat{T}_{\mu\nu}\,,
        \end{gather}
        where $\widehat{T}_{\mu\nu} := T_{\mu\nu} - \frac{1}{2}g_{\mu\nu}T$ is the \textbf{reduced stress-energy tensor}.
    }

    \begin{formula}[Einstein--Hilbert action]\index{action!Einstein--Hilbert}
        The (vacuum) field equations can be obtained by applying the variational principle to the following action:
        \begin{gather}
            S_{\text{EH}}[g_{\mu\nu}] := \Int_M\sqrt{-g}R\,\vol.
        \end{gather}
        For manifolds with boundary one needs an extra term to make the boundary contributions vanish (as to obtain a well-defined variational problem). This term is due to \textit{Gibbons}, \textit{Hawking} and \textit{York}:\footnote{Einstein had in fact already introduced a variant, the $\Gamma\Gamma$-\textit{Lagrangian}.}
        \begin{gather}
            S_{\text{GHY}}[g_{\mu\nu}] := \Oint_{\partial M}\epsilon\sqrt{h}K\,\vol,
        \end{gather}
        where $h_{ab}$ is the induced metric on the boundary, $K_{ab}$ is the extrinsic curvature and $\epsilon=\pm1$ is a `function' depending on whether the boundary is timelike or spacelike.
    \end{formula}

\section{Black holes}

    \newformula{Schwarzschild metric}{\index{Schwarzschild metric}
        \begin{gather}
            \label{relativity:schwarzschild_metric}
            ds^2 := \left(1-\frac{R_s}{r}\right)c^2dt^2 - \left(1-\frac{R_s}{r}\right)^{-1}dr^2 - r^2d\Omega^2\,,
        \end{gather}
        where $R_s$ is the Schwarzschild radius
        \begin{gather}
            R_s := \frac{2GM}{c^2}\,.
        \end{gather}
    }

    \begin{theorem}[Birkhoff]\index{Birkhoff}
        The Schwarzschild metric is the unique solution of the vacuum field equation under the additional constraints of asymptotic flatness and staticity.
    \end{theorem}

    \newformula{Reissner--Nordstr\"om metric}{\index{Reissner--Nordstr\"om metric}
        If the black hole is allowed to have an electric charge $Q$, the Schwarzschild metric must be modified in the following way:
        \begin{gather}
            ds^2 := \left(1-\frac{2GM}{r} + \frac{GQ^2}{4\pi r^2}\right)c^2dt^2 - \left(1-\frac{2GM}{r} + \frac{GQ^2}{4\pi r^2}\right)^{-1}dr^2 - r^2d\Omega^2\,.
        \end{gather}
    }

    \begin{remark}
        The electric field generated by a Reissner--Nordstr\"om black hole is given by
        \begin{gather}
            E^r = \frac{Q}{4\pi r^2}\,.
        \end{gather}
        Although the coordinate $r$ is not the proper distance, it still acts as a parameter for the surface of a sphere (as it does in a Euclidean or Schwarzschild metric). This explains why the above formula is the same as the one in classical electromagnetism.
    \end{remark}

    @@ ADD KERR--NEWMAN, ERGOSPHERE, PENROSE MECHANISM, KRUSKAL @@

\section{Conserved quantities}

    Before raising any hope that conserved quantities are to be found everywhere in GR, the following result is given.
    \begin{property}
        By Noether's third theorem~\ref{var:noether_third_theorem}, there exist no proper (local) conservation laws such as those of momentum and energy in classical mechanics, because the translation group is a subgroup of the infinite-dimensional symmetry group $\mathrm{Diff}(M)$.
    \end{property}

\section{Tetradic formulation}

    We will start from the geometric interpretation of the (weak) equivalence principle, i.e.~spacetime is locally modelled on Minkowski space. The natural language for this kind of geometry is that of Cartan geometries. By the Erlangen program it is known that the Minkowski spacetime $M^4$ can be described as the coset space $\mathrm{ISO}(3,1)/\mathrm{SO}(3,1)$. The natural generalization is given by a Cartan geometry with model geometry $\bigl(\mathfrak{iso}(3,1),\mathfrak{so}(3,1)\bigr)$.

    \begin{property}[Cartan connection]
        This way a $\mathrm{SO}(3,1)$-structure on the spacetime manifold $M$ is obtained, i.e.~a choice of Lorentzian metric $g$. The Cartan connection $\widetilde{\nabla}$ can also be decomposed as $\nabla+\mathbf{e}$ where
        \begin{itemize}
            \item $\nabla$ defines a $\mathfrak{so}(3,1)$-valued principal connection, and
            \item $\mathbf{e}$ defines a $\mathcal{M}^4$-valued solder form.
        \end{itemize}
        The principal connection $\nabla$ is called the \textbf{spin connection} and $\mathbf{e}$ is called the \textbf{vierbein} or \textbf{tetrad}.\index{spin!connection}\index{vielbein}\index{tetrad} These objects are well-known in general relativity. The connection $\nabla$ is the ordinary Levi-Civita connection associated to the Lorentzian manifold $M$ (in case of vanishing torsion) and $\mathbf{e}$ gives the isometry between local (flat) Minkowski coordinates and `global' coordinates:
        \begin{gather}
            g := \mathbf{e}^*\eta
        \end{gather}
        or, locally,
        \begin{gather}
            g_{\mu\nu} = e^i_\mu e^j_\nu\eta_{ij}\,.
        \end{gather}
    \end{property}

    Using the tetrad field one can rewrite the Einstein-Hilbert action in a very elegant way. To this end, a new curvature form is defined:
    \begin{gather}
        F^i_{\ j\mu\nu} := e^i_\rho e_j^\sigma R^\rho_{\ \sigma\mu\nu}\,,
    \end{gather}
    where $R^\rho_{\ \sigma\mu\nu}$ is the ordinary Riemann curvature tensor. The Einstein-Hilbert action is then equivalent\footnote{At least in the case of pure gravity~\citet{rovelli_covariant_2014}.} to the following Yang-Mills-like action:
    \newformula{Palatini action}{\index{Palatini action}
        \begin{gather}
            S[e,\nabla] := \Int_M\mathbf{e}\wedge\mathbf{e}\wedge\ast F\,.
        \end{gather}
        This action is sometimes called the \textbf{tetradic Palatini action} and the resulting formulation of general relativity is called the \textbf{first order formulation}. If one considers the same action but only as a functional of the tetrad field, one obtains the \textbf{second order formulation} of gravity.\footnote{These formulations are equivalent for pure gravity. However, when coupling the theory to fermions, they differ by a four-fermion vertex. This follows from the introduction of torsion due to the fermions.}

        Variation of the Palatini action gives the following EOM:
        \begin{itemize}
            \item $\delta\nabla$: $T(\mathbf{e})=0$ or, equivalently, $\nabla(\mathbf{e})\equiv\nabla$, i.e.~the torsion vanishes and the connection $\nabla$ is on-shell equal to the Levi-Civita connection on $M$.
            \item $\delta\mathbf{e}$: The metric $g$ satisfies the Einstein field equations.
        \end{itemize}
    }

    Because of its importance in general relativity, the first factor in the Palatini action deserves a name.
    \newdef{Plebanski form}{\index{Plebanski form}
        \begin{gather}
            \Sigma := \mathbf{e}\wedge\mathbf{e}
        \end{gather}
        Because of its internal antisymmetric Lorentz indices, one can interpret this object as a $\mathfrak{so}(3,1)$-valued two-form.
    }
    As was the case for 4D Yang--Mills theory, one can introduce a topological term that leaves the EOM invariant (up to boundary terms).
    \newdef{Holst action\footnotemark}{\index{action!Holst}\index{Barbero--Immirzi constant}
        \footnotetext{Holst was actually the second author to include this term.}
        \begin{align}
            S[\mathbf{e},\nabla] :&= \Int_M\mathbf{e}\wedge\mathbf{e}\wedge\ast F + \frac{1}{\gamma}\Int_M\mathbf{e}\wedge\mathbf{e}\wedge F\nonumber\\
            &= \Int_M\left(\ast\textbf{e}\wedge\mathbf{e} + \frac{1}{\gamma}\mathbf{e}\wedge\mathbf{e}\right)\wedge F.
        \end{align}
        The coupling constant $\gamma$ is called the \textbf{Barbero--Immirzi} constant.
    }

    @@ COMPLETE @@

\section{Spinning particle}

    In this section, the action of spinning particle is written down in a supersymmetric way~\citep{berezin_particle_1977}, and it is explained how the existence of spin structures on a manifold (\cref{section:spinor_bundles}) is related to a global anomaly in the quantization of the theory~\citep{witten_global_1985}.

    Consider the following action:
    \begin{gather}
        S_{\text{spin}}[x,\psi] := \frac{1}{2}\Int g_{\mu\nu}\left(\dot{x}^\mu\dot{x}^\nu + i\psi^\mu\nabla_{\dot{x}}\psi^\nu\right)\,d\tau = \frac{1}{2}\Int g_{\mu\nu}\left(\dot{x}^\mu\dot{x}^\nu + i\psi^\mu\dot{\psi}^\nu\right)\,d\tau\,,
    \end{gather}
    where $\psi$ is a Grassmann-odd vector field (\underline{not} a spinor field) and $\nabla$ is the spin connection. The second term is the simplest (Grassmann-even) term, first-order in time derivatives, in the Grassmann variables that can be written down.

    This action possesses a particular symmetry:
    \begin{align}
        \delta x^\mu &:= i\psi^\mu\varepsilon\,,\\
        \delta\psi^\mu &:= \left(\dot{x}^\mu - i\omega^{\ \mu\ }_{\kappa\ \nu}\psi^\kappa\psi^\nu\right)\varepsilon\,,
    \end{align}
    where $\varepsilon$ is a Grassmann-odd constant and $\omega$ is the (local) spin-connection one-form. Although this theory is phsyically not supersymmetric, it does possess a `worldline supersymmetry'.\index{super-!symmetry} The associated Noether charge is
    \begin{gather}
        Q := \psi_\mu\dot{x}^\mu\,.
    \end{gather}
    After canonical quantization, the Grassmann variables $\psi^\mu$ satisfy:
    \begin{gather}
        \{\psi_\mu,\psi_\nu\}_+ = g_{\mu\nu}\,,
    \end{gather}
    i.e.~they can be interpreted as dynamical gamma matrices~\eqref{dirac:clifford_relation}. This gives a physical explanation of why Clifford algebras are said to be `quantizations' of Grassmann algebras.

    The conjugate momentum of $x^\mu$ is given by
    \begin{gather}
        p_\mu = g_{\mu\nu}\dot{x}^\nu + \frac{i}{4}\omega_{\mu\nu\kappa}[\psi^\nu,\psi^\kappa]\,.
    \end{gather}
    In the typical representations of quantum mechanics the relation $p_\mu = -i\partial_\mu$ holds, so the above relation can be rewritten as
    \begin{gather}
        g_{\mu\nu}\dot{x}^\nu = -i\nabla_\mu := \partial_\mu + \frac{1}{4}\omega_{\mu\nu\kappa}[\psi^\nu,\psi^\kappa]\,,
    \end{gather}
    i.e.~the spinorial covariant derivative (up to a factor $-i$). The supersymmetry Noether charge is thus given by the Dirac operator.

    @@ FINISH (not clear right now)  @@
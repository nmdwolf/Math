\chapter{Extensions and Unification}\label{chapter:unification}

    The main reference for supersymmetric quantum mechanics is the seminal paper~\citet{witten_supersymmetry_1982}. For an introduction to algebraic superstructures, see \cref{section:graded_spaces}.

\section{Supersymmetry}

    This section is not meant to be an in-depth study of supersymmetry and its implications for (particle) physics. It will only introduce some concepts and constructions that are widely used in the study of supersymmetric theories. It also contains some sections on certain interesting mathematical properties that arise while studying supersymmetry.

\subsection{Supersymmetric quantum mechanics}

    In this section, a general graded Hilbert space $\mathcal{H}$ will be considered. This space is equipped with an algebra of bounded operator $A\subset\mathcal{B}(\mathcal{H})$ together with a set of $N$ odd self-adjoint operators $\{D^i\}_{i\leq N}$. More precisely, a spectral triple $(\mathcal{H},A,\{D^i\}_{i\leq N})$ is considered (\cref{ncg:spectral_triple}).

    This data defines an SQM system if the Hamiltonian $H$ satisfies the following condition:
    \begin{gather}
        \{D^i,D^j\}_+ = 2\delta^{ij}H\,.
    \end{gather}
    For $N=2$ the whole theory can be rephrased in terms of a nilpotent operator $\dr\sim D^1+iD^2$ and its adjoint:
    \begin{gather}
        \{\dr,\dr^\dagger\}_+\sim H\,.
    \end{gather}

    \begin{example}[Particle on a manifold]
        The archetypal example of $N=2$ systems is the situation where $d$ is the exterior derivative on a smooth manifold $M$, $A$ is the algebra of smooth functions $C^\infty(M)$ and $\mathcal{H}$ is the Hilbert space of square-integrable forms with respect to the Hodge metric
        \begin{gather}
            \langle\alpha\mid\beta\rangle = \Int_M\alpha\wedge\ast\beta\,.
        \end{gather}
    \end{example}

    The above superalgebra is that of a $D=1,N=2$ theory, i.e.~there are two generators acting on a one-dimensional manifold. However, this system can be deformed to represent a $D=2,N=1$ theory. Let $e^{W(t)}$ be a one-parameter subgroup of invertible operators. The $W$-deformed operators are defined as follows:
    \begin{align}
        \dr^W &:= e^{-W(t)}\circ\dr\circ e^{W(t)}\,,\\
        (\dr^W)^\dagger &:= e^{W(t)^\dagger}\circ\dr^\dagger\circ e^{-W(t)^\dagger}\,.
    \end{align}
    It is not hard to see that these deformed operators preserve the superalgebra. Although many authors assume $W$ to be a smooth function, this is not necessary. In fact many interesting examples involve more exotic choices. For example, \citet{schreiber_loop_2005} considers a loop space $\Omega M$ --- a theory of closed strings --- where the deformation operator at a point $\gamma\in\Omega M$ is given by
    \begin{gather}
        W(\gamma)\omega := \Int_\gamma B_{\mu\nu}\bigl(\gamma(t)\bigr)\,\dr x^\mu(t)\wedge\dr x^\nu(t)\wedge\omega(t)\,\dr t\,,
    \end{gather}
    i.e.~the operator takes the exterior product with a given two-form field (e.g.~the \textit{Kalb--Ramond field}) and integrates over the loop $\gamma$ (after pairing with a set of vector fields). When restrictied to the class of skew-Hermitian operators, these deformations can be shown to be pure gauge.\index{Kalb--Ramond field}

\subsection{Loop space mechanics}

    An interesting setting for supersymmetric quantum mechanics is the situation mentioned at the end of the previous section, namely where the base manifold is a loop space $\Omega M$. In this case the tangent space $T_p\Omega M$ is the space of vector fields along the path $p$. As such they carry two indices, one with respect to a (local) frame field on $M$ and one coming from the $S^1$-parametrization of loops. A holonomic basis is given by functional derivatives:
    \begin{gather}
        \partial_{\mu,\sigma} := \frac{\delta}{\delta X^\mu(\sigma)}\,,
    \end{gather}
    where $X^\mu(\sigma)$ is the $\mu^{th}$ coordinate of the loop at the parameter $\sigma\in[0,2\pi[\,$.

\subsection{Extensions of the Standard Model}

    \begin{theorem}[Coleman--Mandula]\index{Coleman--Mandula}
        Consider a quantum field theory satisfying the following conditions:
        \begin{enumerate}
            \item There exists a mass gap.
            \item For every mass scale $M$ there exist only finitely many particle species with mass $m\leq M$.
            \item The two-point scattering amplitudes are nonvanishing for almost every energy.
            \item The (two-point) scattering amplitudes are analytic in the particle momenta.
        \end{enumerate}
        If the symmetry group of the $S$-matrix contains a subgroup isomorphic to the Poincar\'e group\footnote{Technically, its universal cover should be considered.}, it can be written as the direct product of the Poincar\'e group and an internal gauge group.
    \end{theorem}
    \sremark{In other words, it is impossible to combine the Poincar\'e group in a nontrivial way with the internal symmetry group.}

    Now the question arises if one can do better, i.e.~is there a nontrivial way to extend this total symmetry group? A first possibility was given by conformal field theories in \cref{chapter:cft}. CFTs do not admit an $S$-matrix and, hence, the above theorem is clearly not applicable. However, a second and more intricate possibility is given by supersymmetry. Here, one does not work with an ordinary symmetry Lie algebra\footnote{See the original paper~\citet{coleman_all_1967} for why exactly the algebra plays an essential role.} but with a Lie superalgebra (\cref{hda:lie_superalgebra}). By allowing superspaces or, equivalently, by allowing fermionic symmetry generators, one can generalize the Coleman-Mandula theorem. The resulting generalization was proven by \textit{Sohnius, Lopusza\'nski} and \textit{Haag}.\index{Sohnius--Lopusza\'nski--Haag}

\section{Chern--Simons theory}

\subsection{Holomorphic Chern--Simons theory}

    Consider a complex manifold of dimension 3 equipped with a holomorphic volume form $\vol$, i.e.~a Calabi--Yau three-fold (\cref{complex:calabi_yau}). For every complex vector bundle $E\rightarrow M$ and connection $\nabla$ on $E$, one can consider the antiholomorphic connection one-form $B$. This form induces the following holomorphic Chern-Simons action:
    \begin{gather}
        S_{\text{CS}}[B]:=\Int_M\left(\langle B,\delbar B \rangle+\frac{2}{3}
        \langle B,[B\wedge B] \rangle\right)\wedge\vol_M\,.
    \end{gather}
    The critical points of this action are the holomorphically flat connections, i.e.~the connections that satisfy $F^{0,2}=0$. By the Koszul--Malgrange theorem~\ref{complex:koszul_malgrange} these correspond exactly to the holomorphic structures on $E$.

\section{Four-manifolds}

    In this section, the base manifold $M$ is assumed to be closed, orientable and Riemannian. The main part will be about the (differential) topology of four-manifolds.

    First of all, one can define an intersection form $Q:H^2(M;\mathbb{Z})\times H^2(M;\mathbb{Z})$ by pairing the cup product of two cohomology classes with the fundamental form $[M]$. In de Rham cohomology this corresponds to integration:
    \begin{gather}
        Q([\omega],[\nu]) := \Int_M\omega\wedge\nu\,.
    \end{gather}
    By Poincar\'e duality, $Q$ can also be extended to homology. It is a quadratic form and the signature $(q_+,q_-)$ of $Q$ is called the signature $\sigma(M)$ of $M$. (Some authors define the signature of $M$ as $q_+-q_-$.) The rank of $Q$ is equal to the second Betti number of $M$: $\rk(Q)=q_++q_-=b_2(M)$. An important result by \textit{Freedman} states that two simply-connected smooth\footnote{For topological manifolds and odd intersection forms, two classes exist.} 4-manifolds are homeomorphic if and only if they have the same intersection form.

    Because $M$ is assumed to be Riemannian, one can also define the Hodge operator $\ast$. It can be shown that the spaces of self- and anti-selfdual forms corresponds to the positive and negative subspaces of the intersection form. By Hodge theory, the signature $(q_+,q_-)$ corresponds to the number of self- and anti-selfdual harmonic forms on $M$.

    The first Pontryagin class $p_1(M)\in H^4(M;\mathbb{Z})$ induces a characteristic number (after integration) and by the \textit{Hirzebruch theorem} one has
    \begin{gather}
        p_1(M) = 2\sigma(M)\,.
    \end{gather}
    Moreover, if $M$ is equipped with an almost complex structure, one can calculate its Chern classes. The second Chern class is just the Euler class $c_2(M)=e(M)\in H^4(M;\mathbb{Z})$. The first Chern number admits an expression similar to that of the Pontryagin number by \cref{bundle:chern_pontryagin}:
    \begin{gather}
        c^2_1(M) = 2\chi(M) + 3\sigma(M)\,.
    \end{gather}

\subsection{Donaldson--Witten theory}\index{Donaldson--Witten theory}

    In four dimensions the classification of manifolds is considerably more intricate than in higher dimensions (cf.~\cref{manifold:dim4}). To find invariants \textit{Donaldson} studied instanton configurations of non-Abelian gauge theories on 4-manifolds.

    Recall the physics definition of a Yang-Mills instanton. It is a connection (one-form) $A$ with self-dual curvature vanishing at infinity. The latter means that it extends to the one-point compactification of $M$.

    Consider for example the case of a $\mathrm{SU}(n)$-bundle $P\rightarrow M$. In this case, the second Chern number reads (by Chern--Weil theory)
    \begin{gather}
        c_2(P)[M] = \Int_M\frac{\tr(F^2)-\tr(F)^2}{8\pi^2}\,.
    \end{gather}
    Because elements of $\mathfrak{su}(n)$ are traceless, this reduces to
    \begin{gather}
        c_2(P)[M] = \Int_M\frac{\tr(F^2)}{8\pi^2}=-\frac{1}{8\pi^2}\int_M\left(\left\|F^+\right\|^2-\left\|F^-\right\|^2\right)\vol_M\,,
    \end{gather}
    where the curvature was decomposed in a selfdual and anti-selfdual part $F^\pm$ and the Killing metric $K(X,Y)=-\tr(XY)$ is used. Now, consider the Yang--Mills action
    \begin{align*}
        S_{\text{YM}}[A] &= -\Int_M\tr(F\wedge\ast F)\\
        &= \Int_M\|F\|^2\,\vol_M\\
        &= \Int_M\left(\left\|F^+\right\|^2+\left\|F^-\right\|^2\right)\vol_M\,.
    \end{align*}{gather}
    So, the action is bounded below by (the absolute value of) the second Chern number. Depending on the sign of the second Chern number, the following minimzation conditions are found:\footnote{Note that on Lorentzian manifolds the sign of the Yang--Mills action is reversed. Some authors consequently also reverse the sign of the Chern class and, hence, also reverse this characterization.}
    \begin{itemize}
        \item $c_2(P)[M]\geq0$: anti-selfdual connections, i.e.~$F^+=0$, minimize the action.
        \item $c_2(P)[M]\leq0$: selfdual connections, i.e.~$F^-=0$, minimize the action.
    \end{itemize}
    When $c_2(P)[M]=1$, the (anti-)selfdual connections are called \textbf{(Yang-Mills) instantons}. In general, when the solution constitutes an absolute minimum, i.e.~when the action equals the second Chern number, the solution is called a \textbf{Bogomol'nyi--Prasad--Sommerfield instanton} (BPS). \index{instanton!Yang--Mills}\index{instanton!BPS} @@ CHECK THIS @@

    \begin{property}
        The intersection form $Q$ of $M$ is positive-definite if and only if there exist no anti-selfdual harmonic forms on $M$.
    \end{property}

    Now, consider the moduli space of (anti-)selfdual connections:
    \begin{gather}
        \mathcal{M}^\pm(P) := \{A\in\mathcal{A}(P)\mid\ast F_A=\pm F_A\}\, / \Aut_V(P)\,.
    \end{gather}
    One should, however, pay attention when forming this quotient. In general the action of the automorphism group will not be free, i.e.~there will be connections with a nontrivial stabilizer. The stabilizer of a connection is given by the covariantly constant sections of $\Aut_V(P)$. Elements for which the stabilizer is a subgroup of the center $Z(G)$ are said to be \textbf{irreducible}. In general the reducible connections correspond to singular points of $\mathcal{M}^\pm$.

    According to \cref{bundle:connection_gauge_transformation}, the stabilizer of $A$ can be characterized as the set of sections of $\Aut_V(P)$ that are covariantly constant (with respect to $A$). The Lie algebra of this group is given by $\Gamma(\mathrm{ad}(P))$ and, hence, the reducible connections are those that correspond to the nonzero kernel of
    \begin{gather}
        \nabla^A:\Gamma(\mathrm{ad}(P))\rightarrow\Omega^1(M;\mathrm{ad}(P))\,,
    \end{gather}
    where the codomain is exactly the model space of the affine space of connections $\mathcal{A}(P)$ and, accordingly, also the (model) tangent space $T_A\mathcal{A}(P)$.

    Now, to characterize the tangent space $T_{[A]}\mathcal{M}^\pm$, one needs to find the directions in $\mathcal{M}^\pm$ that preserve the (anti-)selfdual property and that are not gauge orbits. The second part can be done by orthogonally decomposing the tangent space with respect to $\nabla^A$:
    \begin{gather}
        \Omega^1(M;\mathrm{Ad}(P))=\im(\nabla)\oplus\ker(\nabla^\dagger)\,.
    \end{gather}
    This shows that the neighbourhood of $A$ (consisting of irreducible connections) is modelled on $\ker(\nabla^\dagger)$. Moreover, by linearizing the condition for $F_{A+a}$ to be (anti-)selfdual, where $a\in\Omega^1(M;\mathrm{ad}(P))$, one obtains the condition
    \begin{gather}
        \pr^\mp(\nabla^Aa)=0\,.
    \end{gather}
    This operator is simply the differential of the function that maps a connection to the (anti-)selfdual part of its curvature. It is also clear that the image of $\nabla^A$ lies in the kernel of $\pr^\mp\nabla^A$ since $\nabla^2=F$. This gives rise to the \textbf{Atiyah--Hitchin--Singer complex} or \textbf{instanton deformation complex} $C^\bullet_{\text{AHS}}$:
    \begin{gather}
        0\longrightarrow\Gamma(\ad(P))\overset{\nabla^A}{\longrightarrow}\Omega^1(M;\ad(P))\overset{\pr^\mp\nabla^A}{\longrightarrow}\Omega^{2,\pm}(M;\ad(P))\longrightarrow0\,.
    \end{gather}
    By the above arguments, one obtains the following model:
    \begin{gather}
        T_{[A]}\mathcal{M}^\pm\cong H^1(C^\bullet_{\text{AHS}})\,.
    \end{gather}
    The above complex is an elliptic complex and its index, minus the Euler characteristic, can be calculated by the Atiyah--Singer index theorem.

    @@ COMPLETE (dimension, ...) @@

\subsection{Instanton Floer homology}

    In the previous sections two important tools were introduced, that of Chern--Simons theory and that of instantons. The idea of instanton Floer theory is to consider the Morse--Floer homology of the Chern--Simons action functional on the moduli space $\mathcal{M}^-$ of (irreducible) ASD instantons. There are some technical subtleties involved, such as the fact that the functional might not be Morse (this can be resolved by adding a so-called \textit{holonomy perturbation}), which will be omitted.

    As in \cref{section:morse_homology}, the chain complex is generated (over $\mathbb{Z}$) by the critical points of $S_{\text{CS}}$, i.e.~the gauge equivalence classes of flat connections. For every path of connections $A:[0,1]\rightarrow\mathcal{M}^-$ one can define the \textbf{spectral flow} $\mathrm{sf}(A)$ as the number of eigenvalues that change from negative to positive minus the number of eigenvalues that change from positive to negative along the path. This is the infinite-dimensional analogue of the difference in Morse indices.

    Consider the subset of $\mathcal{M}^-(\mathbb{R}\times P)$ consisting of those connections that satisfy the finite energy condition:
    \begin{gather}
        \|F_A\|^2_2 := \Int_{\mathbb{R}\times M}\|F_A\|^2\,\vol_M<+\infty\,.
    \end{gather}
    As for ordinary Morse homology, the flow lines admit a free $\mathbb{R}$-action, which should be quotiented out. The resulting moduli space $\overline{\mathcal{M}}(A_-,A_+)$ has the structure of a finite-dimensional (oriented) smooth manifold of dimension $\mathrm{sf}(A_-,A_+)-1$. The boundary operator is defined as follows:
    \begin{gather}
        \partial A_- := \sum_{\substack{A_+\in\mathcal{M}^-_{\text{flat}}(P)\\\mathrm{sf}(A_-,A_+)=1}}\left|\,\overline{\mathcal{M}}(A_-,A_+)\right|\,\langle A_+ \rangle\,.
    \end{gather}
    The resulting homology theory is called \textbf{instanton Floer homology}.

    @@ COMPLETE @@

\subsection{Seiberg--Witten theory}\index{Seiberg--Witten theory}

    In this section $M$ will denote an oriented, Riemannian 4-manifold equipped with a $\mathrm{Spin}^{\mathbb{C}}$-structure $P\rightarrow M$. It can be shown that $\mathrm{Spin}^{\mathbb{C}}(4)$ is isomorphic to $\mathrm{SU}(2)\times\mathrm{SU}(2)\times\mathrm{U}(1)/\mathbb{Z}_2$. This group admits two two-dimensional representations
    \begin{align}
        s_+&:\mathrm{Spin}^{\mathbb{C}}(4)\rightarrow\mathrm{U}(2):[s^+,s^-,s]\mapsto[s_+,s]\,,\\
        s_-&:\mathrm{Spin}^{\mathbb{C}}(4)\rightarrow\mathrm{U}(2):[s^+,s^-,s]\mapsto[s_-,s]\,,
    \end{align}
    and the one-dimensional determinant representation
    \begin{gather}
        \det:\mathrm{Spin}^{\mathbb{C}}(4)\rightarrow\mathrm{U}(1):[s^+,s^-,s]\mapsto s^2\,.
    \end{gather}
    These representations in turn induce associated $\mathrm{Spin}^{\mathbb{C}}(4)$-bundles, two spinor bundles $S^\pm$ and one line bundle $\det(P)$. The spinor bundles are a kind of square root of the line bundle (by some representation-theoretic arguments): $\Lambda^2S^\pm\cong\det(P)$. Moreover, because the representations are unitary, these bundles admit a (Hermitian) metric.

    Any connection on $\det(P)$ induces a connection on an associated $\mathrm{Spin}^{\mathbb{C}}(4)$-bundle after addition by the Levi-Civita connection, since \[\mathrm{Lie}(\mathrm{Spin}^{\mathbb{C}}(4))\cong\mathrm{Lie}(\mathrm{SO}(4))\oplus\mathrm{Lie}(\mathrm{U}(1))\,.\] So choose such a connection $A$ and choose a self-dual two-form $\mu$ (this two-form acts as a perturbation parameter). The Seiberg--Witten equations associated to $\mu,\nabla:=\nabla^\mathrm{LC}+\nabla^A$ and a spinor field $\psi\in\Gamma(S^+)$ are
    \begin{align}
        \mathrm{D}\psi &= 0\,,\\
        F_A^+ &= q(\psi) + i\mu\,,\label{unification:sw2}
    \end{align}
    where $\mathrm{D}$ is the Dirac operator associated to $\nabla$, $F_A$ is the curvature of $A$ and $q:S^+\rightarrow i\Lambda^2T^*M$ is the quadratic form obtained as the adjoint of the Clifford multiplication extended to an action of $\Lambda^{2,+}_{\mathbb{C}}TM$.

    \begin{construct}[Squaring map]
        The squaring map $q$ deserves some more attention. Through Clifford multiplication, traceless self-adjoint spinor endomorphisms are identified with (imaginary) two-forms. So the first step of defining $q$ is to map a spinor field to a traceless operator:
        \begin{gather}
            \psi\mapsto\psi\otimes\psi^\dagger - \frac{1}{2}|\psi|^2\,.
        \end{gather}
        One can then apply the inverse of the Clifford map to obtain the two-form. For this reason, the second Seiberg--Witten equation is often written as\footnote{Another reason is that in dimensions different from 4, the spinor endomorphisms cannot be identified with two-forms and the equations are not equivalent anymore.}
        \begin{gather}
            \iota_{C\ell}(F_A^+-i\mu) = \psi\otimes\psi^\dagger-\frac{1}{2}|\psi|^2\,.
            \tag{\ref*{unification:sw2}b}
        \end{gather}
        Another way is to work with the adjoint of the Clifford map (since the $\partial_i$ are unit vectors, their image is unitary):
        \begin{align*}
            \langle\iota_{C\ell}^\dagger(\psi\otimes\psi^\dagger)\mid\partial_i\wedge\partial_j\rangle &= \langle\psi\otimes\psi^\dagger\mid\iota_{C\ell}(\partial_i\wedge\partial_j)\rangle\\
            &= \langle\psi\mid\iota_{C\ell}(\partial_i\wedge\partial_j)\psi\rangle\\
            &= \langle\psi\mid\gamma_i\gamma_j\cdot\psi\rangle\,,
        \end{align*}
        where it was used that $\gamma_1\gamma_2-\gamma_2\gamma_1=2\gamma_1\gamma_2$ since $i\neq j$. Since $i,j$ are arbitrary, one obtains:\footnote{Note that a wide variety of signs and prefactor can be found in the literature.}
        \begin{gather}
            q(\psi)\sim\frac{1}{2}\sum_{i\neq j}\langle\psi\mid\gamma_i\gamma_j\cdot\psi\rangle\drx^i\wedge\drx^j\,.
        \end{gather}
        A third approach uses the isomorphism $\mathrm{SU}(2)\cong\mathrm{Sp}(1)$ and identifies the $\mathbb{C}^2$-bundles with $\mathbb{H}$-bundles. Imaginary two-forms can then be identified with sections of an $\mathrm{Im}(\mathbb{H})$-subbundle, where $\mathrm{Im}(\mathbb{H}):=\{x\in\mathbb{H}\mid\overline{x}=-x\}$. The squaring map is induced by the following morphism:
        \begin{gather}
            q:\mathbb{H}\rightarrow\mathrm{Im}(\mathbb{H}):x\mapsto\frac{1}{2}\overline{x}ix\,.
        \end{gather}
    \end{construct}

    \begin{remark}[Hyper-K\"ahler manifolds]
        The study of Seiberg--Witten equations can be widely generalized exactly due to the relation between $\mathrm{Spin}^{\mathbb{C}}(4)$ and the quaternions $\mathbb{H}$ that was used to give an alternative definition of the squaring map.

        In this context one looks at hypercomplex or, in particular, hyper-K\"ahler manifolds. The circle group $\mathrm{U}(1)$ acts on $\mathbb{H}$ by scalar multiplication and preserves the hyper-K\"ahler structure. The associated moment map is exactly the morphism inducing the squaring map:
        \begin{gather}
            \mu:\mathbb{H}\rightarrow\mathrm{Sp}(1):x\mapsto\frac{1}{2}\overline{x}ix\,.
        \end{gather}
    \end{remark}

    \begin{property}[Weitzenb\"ock identity]\index{Weitzenb\"ock identity}
        The Lichnerowicz formula~\ref{riemann:lichnerowicz_formula} for the Dirac operator can easily be extended to a formula for twisted Dirac operators on a vector bundle of the form $S\otimes E$. If $F$ is the curvature form of a connection on $E$, the Weitzenb\"ock identity\footnote{In general one used the term \textbf{Weitzenb\"ock identity} for any relation between two symmetric/elliptic second-order partial differential operators (i.e.~\textit{generalized Laplacians}) with the same principal symbol.} for the twisted Dirac operator reads\footnote{Again, some authors might use different conventions.}
        \begin{gather}
            \mathrm{D}^2 = \nabla^*\nabla+\frac{1}{4}R+\frac{1}{4}F\,.
        \end{gather}
        In the case of a proper $\mathrm{Spin}(4)$-structure on $M$, the half-spinor bundles $S^\pm$ can be identified with the tensor product $S\otimes\det^{1/2}(P)$. In this case the Dirac operator on these bundles satisfy the above Weitzenb\"ock identity. However, if only a $\mathrm{Spin}^{\mathbb{C}}(4)$-structure exists, the Dirac operator cannot be obtained as a twisted operator, but it still satisfies this identity.
    \end{property}

    Now, consider the solution space $\overline{\mathcal{M}}_{S,\psi,\mu}\subset\mathcal{A}(\det(P))\times\Gamma(S^+)$ of solutions of the Seiberg--Witten equations ($S$ denotes the choice of $\mathrm{Spin}^{\mathbb{C}}$-structure). As usual, this space admits some gauge symmetries. Any function $g:M\rightarrow\mathrm{U}(1)$ acts on the connection $\nabla^L$ by ordinary gauge transformations:
    \begin{gather}
        A\longrightarrow A-2g^{-1}\dr g\qquad\qquad\psi\longrightarrow g\psi\,.
    \end{gather}
    The factor 2 follows from the fact that a connection one-form $A$ on $\det(P)$ would induce a connection one-form $\frac{1}{2}A$ on $\det^{1/2}(P)$ (if the latter existed). In general the action of the gauge group is free, expect at elements with $\psi=0$, which have a $\mathrm{U}(1)$ stabilizer. To remove these gauge symmetries, one should pass to a suitable quotient. Either by all of $C^\infty(M,\mathrm{U}(1))$ or by the subset of these functions that fix a fixed basepoint, giving respectively rise to $\mathcal{M}$ and $\mathcal{M}_0$. The topology on these spaces is induced by the $C^\infty$-Fr\'echet topology.

    \begin{property}[Moduli space]
        The moduli spaces $\mathcal{M}$ and $\mathcal{M}_0$ satisfy the following properties:
        \begin{itemize}
            \item $\mathcal{M}$ is compact.
            \item For almost all $\mu$, $\mathcal{M}_0$ is a (smooth) finite-dimensional manifold equipped with a circle action. Moreover, consider the \textit{intersection form} $Q:H^2(M;\mathbb{Z})\times H^2(M;\mathbb{Z})$ defined by integrating the wedge product of closed two-forms. This form is quadratic and, hence, there exists a maximal positive subspace of $H^2(M;\mathbb{Z})$. If this subspace has nonzero dimension, $\mathcal{M}$ is also smooth and $\mathcal{M}\rightarrow\mathcal{M}_0$ is a principal $\mathrm{U}(1)$-bundle.
            \item For almost all $\mu$, $\mathcal{M}$ is orientable and its dimension is given by
                \begin{gather}
                    \dim(\mathcal{M}) = \dim(H^1(M)) - 1 - q_+ + \frac{Q\bigl(c_1(\det(P)),c_1(\det(P))\bigr)-q_++q_-}{4}\,,
                \end{gather}
                where $(q_+,q_-)$ is the signature of $Q$.
        \end{itemize}
    \end{property}

\section{String theory}
\subsection{Worldsheets}

    Similar to how a relativistic particle is characterized by an action that is proportional to the arc length of its worldline as defined in \cref{relativity:worldline_action}, i.e.
    \begin{gather}
        S_{\text{point}}=-mc\Int_\gamma ds\,,
    \end{gather}
    the action of the relativistic string is proportional to the area of its worldsheet.
    \newformula{Nambu--Goto action}{\index{Nambu--Goto action}
        \begin{gather}
            S_{\text{NG}} := -T_0\Int_\Sigma d\Sigma = -\frac{1}{2\pi\alpha'}\Int_\Sigma d\Sigma\,,
        \end{gather}
    }
    where $T_0$ is the \textbf{string tension} and $\alpha'$ is the \textbf{slope parameter} (the latter parametrization is a remnant from the work on \textit{Regge trajectories} in nuclear physics).

    The relativistic action of a particle could be written in covariant form as a nonlinear $\sigma$-model after introducing dynamical gravity on the worldline. In the same way, the Nambu--Goto action can be rewritten as follows.
    \newformula{Polyakov action}{\index{Polyakov action}
        \begin{gather}
            S_{\text{Polyakov}} := \frac{T_0}{2}\Int_\Sigma\sqrt{-h}h^{ab}g_{\mu\nu}(x)\partial_ax^\mu\partial_bx^\nu\,d^2\tau\,.
        \end{gather}
    }

    \begin{property}[Conformal symmetry]
        The 2D $\sigma$-model defined by the Polyakov action has even more structure. It is not too hard to see that this action is conformally invariant, i.e.~it is invariant under Weyl transformations on the worldsheet $h\longrightarrow\Omega h$. It follows that the stress-energy tensor
        \begin{gather}
            T_{ab} := \frac{-2}{\sqrt{-h}}\frac{\delta S}{\delta h_{ab}}
        \end{gather}
        is traceless.
    \end{property}

\subsection{Branes}

    Before moving to branes in string theory, an interpretation in CFTs is given. Consider the class of 2D rational CFTs, i.e.~those characterized by the \textit{FRS theorem}. Such theories are defined by a choice of modular tensor category (\cref{hda:modular_category}) and an internal special symmetric Frobenius algebra (\cref{qa:frobenius}). To every boundary of a 2D cobordism the CFT assigns a module of the Frobenius algebra. This is exactly a brane in the CFT.

    @@ ADD (Chan--Paton, DBI action, ...) @@

\subsection{Duality}

    @@ ADD (S, T, Montonen--Olive, ...) @@

    The first type of duality that one can consider is so-called $T$-duality. In its simplest form this the string spectrum on two torus bundles. Consider two spacetime manifolds (locally) of the form $\mathbb{R}^d\times S^1(r)$ and $\mathbb{R}^d\times S^1(1/r)$. Considering the $\sigma$-model action it can be seen that it is invariant under the transformation $r\longrightarrow 1/r$.

    More generally, consider two circle bundles $\pi:P\rightarrow M$ and $\pi':P'\rightarrow M$ together with their $B$-fields, which by differential cohomology (\cref{section:differential_cohomology}) are represented by degree-3 integral cohomology classes $\alpha,\alpha'\in H^3(M;\mathbb{Z})$. These two bundles with B-field are said to be $T$-dual if they satisfy the following conditions:
    \begin{gather}
        \pi_*\alpha=c_1(P')\qquad\qquad\pi'_*\alpha'=c_1(P)\,,
    \end{gather}
    where $c_1$ denotes the first Chern class. If this condition is satisfied, there exists an isomorphism between twisted $K$-theories:
    \begin{gather}
        K^\bullet_\alpha(P)\cong K^{\bullet-1}_{\alpha'}(P')\,.
    \end{gather}

\subsection{Superstrings}

    @@ ADD (GSO, 5 theories, ...) @@

\section{M-theory}

    @@ ADD @@
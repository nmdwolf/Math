\chapter{Astronomy \& Cosmology}

    The content of this chapter consists of two main (related) topics. The first one will be astronomy, which can mostly be treated by classical mechanics (and some general relativity). For some background, see \namecrefs{chapter:lagrange}~\nameref{chapter:lagrange} and~\nameref{chapter:GR}. The second part will consist of more quantum-mechanical systems such as black holes and inflation.

    \minitoc

\section{Stars}

    \todo{COURSES UGENT}

\section{Galaxies}

    \todo{COURSES UGENT}

\section{Black holes}

    \begin{formula}[Black hole entropy]\index{entropy!black hole}\index{Bekenstein--Hawking formula}\index[author]{Bekenstein}\index[author]{Hawking}
        The \textbf{Bekenstein--Hawking formula} for the entropy of a black hole states that the entropy is proportional to the surface of the horizon:
        \begin{gather}
            S_{\text{BH}} = \frac{k_BA}{4\ell_p^2}\,,
        \end{gather}
        where $k_B$ is the Boltzmann constant and $\ell_p$ is the Planck length.
    \end{formula}

    \begin{conjecture}[Ryu--Takayanagi formula]\index{Ryu--Takayanagi formula}\index[author]{Ryu}\index[author]{Takayanagi}
        Consider a spatial region $\Sigma\subset\mathrm{AdS}_4$ and let $S_A$ denotes the (CFT) entanglement entropy of a subset $A\subset\partial\Sigma$ with its complement.
        \begin{gather}
            S_A = \frac{\mathrm{Area}(\gamma_A)}{4G}\,,
        \end{gather}
        where $\gamma_A$ is a surface satisfying:
        \begin{enumerate}
            \item $\partial\gamma_A = \partial A$,
            \item $[\gamma_A]=[A]$ in $H_2(\Sigma)$, and
            \item $\gamma_A$ is extremal (in terms of area).
        \end{enumerate}
        The surface $\gamma_A$ is usually called a \textbf{minimal surface}.
    \end{conjecture}

    \todo{ADD (e.g. Thermodynamics, Hawking radiation, ...)}

\section{Tegmark multiverses}\index[author]{Tegmark}\index{multi-!verse}

    In contemporary physics, there are various notions of multiverse. \textit{Tegmark} has introduced a hierarchy that orders some of these notions.

    \begin{enumerate}
        \item In most theories of inflation, the local (observable) universe is only a subset of the entire universe since the speed of light is finite and the expansion of space outpaced this limit. The collection of all these local universes (i.e.~the initial conditions) constitutes the Level I-multiverse.
        \item The foundational equations of the universe (be it Standard Model + Einstein Equations, LQG, string theory, ...) admit various solutions (i.e.~physical constants). Nothing, in theory, prevents nature from realizing all these solutions (an idea that is particularly popular in string theory) by, for example, spontaneous symmetry breaking. This leads to the Level II-multiverse.
        \item In the many-worlds interpretation of QM, the idea of physically branching universes leads to an additional level of multiverses. Not only are all (or, at least, various) initial conditions (Level I) and parameter configurations (Level II) realized, so are all possible histories. This collection forms the Level III-multiverse. (Note that the only difference between Level I/II and Level III-universes is the \textit{indexical information} about which branch someone lives on. It does not add any physical content to the multiverse.)
        \item On top of these physical distinctions, one can add a further layer of abstraction. Instead of merely realizing all possible parameter configurations and histories, the Level IV-multiverse also combines all possible mathematical equations and/or structures that could describe a universe.
    \end{enumerate}

    The Level IV-multiverse is based on the following idea.
    \begin{axiom}[Mathematical Universe Hypothesis]
        The universe is a mathematical object. Moreover, one can strengthen this hypothesis to the \textbf{Computable Universe Hypothesis}, which says that the universe, as a mathematical object, is defined by \textit{computable functions} (see \cref{section:computability}).
    \end{axiom}
\chapter{Classical Field Theory}\label{chapter:classical_fields}

    Rigorous definitions and statements about the mathematical concepts used in this chapter can be found in \namecrefs{chapter:bundles}~\ref{chapter:bundles}, \ref{chapter:vector_bundles}, \ref{chapter:riemann} and~\ref{chapter:variation}.

    \minitoc

\section{Lagrangian field theory}\index{action}

    The physical space will be assumed to be a (pseudo-)Riemannian, $n$-dimensional manifold $(M,g)$ with the fields being sections of a vector bundle $E\rightarrow M$. In general, an \textbf{action} is a function $S:\Gamma_c(E)\rightarrow\mathbb{R}$ from the space of (compactly supported) sections of $E$ to the real numbers. This is often given by local functionals for a local Lagrangian (\cref{var:local_lagrangian}):
    \begin{gather}
        S:\Gamma_c(E)\rightarrow\mathbb{R}:\phi\mapsto\Int_M(j^\infty\phi)^*L\vol_M\,.
    \end{gather}

    Associated to the manifold $M$, one can construct a cochain complex similar to the de Rham complex $\Omega^\bullet(M)$. This structure takes two geometric features into account. On the one hand, one has the ordinary de Rham differential $\dr$ on the base manifold $M$, whereas, on the other hand, one has a differential $\delta$ induced by the variation of fields along the jet fibres. The total differential will be the sum of these as is standard in the context of bicomplexes. This defines the variational bicomplex.

    Some key concepts from the calculus of variations are recalled here. The variational derivative or Euler--Lagrange derivative~\eqref{var:euler_lagrange_derivative} is defined as follows (partial derivatives are denoted by subscripted commas, e.g.~$\partial_\mu\partial_\nu\phi\equiv\phi_{,\mu\nu}$):\footnote{Total derivatives with respect to spatial coordinates are still denoted by $\partial$, in contrast to \cref{var:euler_lagrange_derivative}, to allign with most works in the literature.}
    \begin{gather}
        \frac{\delta L}{\delta \phi} := \pderiv{L}{\phi} - \pderiv{}{x^\mu}\left(\pderiv{L}{\phi_{,\mu}}\right) + \frac{\partial^2}{\partial x^\mu\partial x^\nu}\left(\pderiv{L}{\phi_{,\mu\nu}}\right) -\cdots\,.
    \end{gather}
    By comparing this formula to the formula for the variation of the Lagrangian density, one obtains the first variational formula~\eqref{var:first_variational_formula}:
    \begin{gather}
        \delta L = \frac{\delta L}{\delta\phi^I}\delta\phi^I - \dr\Theta[\phi]\,.
    \end{gather}
    The first term vanishes on-shell because it is proportional to the Euler--Lagrange equation associated to the field $\phi^I$. The last term contains the boundary terms obtained after performing integration by parts. The $(n-1,1)$-form $\Theta$ is called the \textbf{presymplectic potential}.\index{potential!presymplectic}

    The \textbf{presymplectic current} $\omega$ is obtained by taking the variation of the presymplectic potential:\index{current!presymplectic}
    \begin{gather}
        \omega[\phi] := \delta\Theta[\phi]\,.
    \end{gather}
    On-shell, this form is closed, i.e.~$\symbfsf{d}\omega\approx0$. Off-shell, this does not necessarily hold and, hence, the form is not properly symplectic. It can be shown that, if the variations $\delta\phi^I$ satisfy the linearised equations of motion, then for every gauge transformation $\xi$, there exists an $(n-2,1)$-form $k_\xi[\phi]$ such that $\omega\approx\dr k_\xi[\phi]$.\footnote{More details can be found in e.g.~\citet{compere_advanced_2019}.}

    \todo{EXPLAIN this last statement better}

\subsection{Noether's theorem}\index{Noether!theorem}

    \begin{theorem}[Noether's first theorem]\label{field:noethers_theorem}
        Consider an infinitesimal field transformation
        \begin{gather}
            \phi\longrightarrow\phi+\alpha\delta\phi\,,
        \end{gather}
        where the Lagrangian $L$ depends on the fields and their first-order derivatives.\footnote{An extension to higher-order derivatives can be obtained by including further boundary terms.} In case of a symmetry, one obtains a conservation law of the following form:
        \begin{gather}
            \label{field:conserved_current}
            \partial_\mu\left(\pderiv{L}{\phi_{,\mu}}\delta\phi - \mathcal{J}^\mu\right) = 0\,.
        \end{gather}
        The factor between parentheses can be interpreted as a conserved current $j^\mu(x)$.
        \begin{mdframed}[roundcorner=10pt, linecolor=blue, linewidth=1pt]
            \begin{proof}
                The general transformation rule for the Lagrangian is
                \begin{gather}
                    \label{noether_deriv:1}
                    L\longrightarrow L + \alpha\delta L\,.
                \end{gather}
                To have a symmetry, i.e.~to keep the action invariant, the deformation factor has to be a divergence:
                \begin{gather}
                    \label{noether_deriv:2}
                    L\longrightarrow L + \alpha\partial_\mu\mathcal{J}^\mu\,.
                \end{gather}

                To obtain the conservation law~\eqref{field:conserved_current}, the Lagrangian is varied explicitly:
                \begin{align*}
                    \delta L &= \pderiv{L}{\phi}\delta\phi + \pderiv{L}{\phi_{,\mu}}\delta\phi_{,\mu}\\
                    &= \pderiv{L}{\phi}\delta\phi + \partial_\mu\left(\pderiv{L}{\phi_{,\mu}}\delta\phi\right) - \partial_\mu\left(\pderiv{L}{\phi_{,\mu}}\right)\delta\phi\\
                    &= \partial_\mu\left(\pderiv{L}{\phi_{,\mu}}\delta\phi\right) + \left[\pderiv{L}{\phi} - \pderiv{L}{\phi_{,\mu}}\right]\delta\phi\,.
                \end{align*}
                The second term vanishes due to the Euler--Lagrange equation~\eqref{classic:second_kind}. Combining these formulas gives
                \begin{gather}
                    \partial_\mu\left(\pderiv{L}{\phi_{,\mu}}\delta\phi\right) - \partial_\mu\mathcal{J}^\mu(x) = 0\,.
                \end{gather}
                From this equation, one can conclude that the current
                \begin{gather}
                    j^\mu(x) = \pderiv{L}{\phi_{,\mu}}\delta\phi - \mathcal{J}^\mu(x)
                \end{gather}
                is conserved.
            \end{proof}
        \end{mdframed}
    \end{theorem}
    The above conservation law can also be expressed in terms of a charge (such a current and its associated charge are generally called the \textbf{Noether current} and \textbf{Noether charge}):\index{Noether!charge}
    \begin{gather}
        \label{field:noether_charge}
        Q[\Sigma] := \Int_\Sigma j^0\,d^{n-1}x\,,
    \end{gather}
    where $\Sigma$ is a spacelike hypersurface. The conservation law can then simply be restated as
    \begin{gather}
        \deriv{Q}{t} = 0\,.
    \end{gather}

    \newdef{Stress-energy tensor}{\index{stress-energy tensor}
        Consider the translation of a scalar field:
        \begin{gather}
            \phi(x)\longrightarrow\phi(x+a) = \phi(x) + a^\mu\partial_\mu\phi(x)\,.
        \end{gather}
        Because the Lagrangian is a scalar quantity, it transforms in the same way as the fields:
        \begin{gather}
            L\longrightarrow L + a^\mu\partial_\mu L = L + a^\nu\partial_\mu(L\delta^\mu_{\ \nu})\,.
        \end{gather}
        On an $n$-dimensional manifold, this leads to the existence of $n\in\mathbb{N}$ conserved currents. These can be used to define the stress-energy tensor:
        \begin{gather}
            \label{field:stress_energy_tensor}
            T^\mu_{\ \nu} = \pderiv{L}{\phi_{,\mu}}\partial_\nu\phi - L\delta^\mu_{\ \nu}\,.
        \end{gather}
    }
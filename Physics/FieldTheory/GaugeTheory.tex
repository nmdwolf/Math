\chapter{Gauge Theory}\label{chapter:gauge_theory}

    References for this chapter are~\citet{sontz_principal_2015,schuller_lectures_2016,nash_topology_2011,belgun_gauge_2024}. The section on the Higgs mechanism is mainly based on~\citet{choquet-bruhat_analysis_2000}. Using the tools of differential geometry, as presented in \labelref{chapter:bundles} and onwards, one can introduce a general formulation of gauge theories and, in particular, Yang--Mills theories.

    \minitoc

\section{Gauge invariance}
\subsection{Gauge principle}

    Consider a general Lie group $G$, often called the \textbf{gauge group}, acting on a vector bundle $E$ with typical fibre $\mathcal{H}$ over a base manifold $M$ (be wary of \cref{bundle:vertical_automorphism}). This bundle is, in general, obtained as a bundle associated to the frame bundle $FM$. Locally, a general gauge transformation has the form
    \begin{gather}
        \label{gauge:gauge_transformation}
        \psi'(x) = U(x)\psi(x)\,,
    \end{gather}
    where $\psi,\psi':M\rightarrow\mathcal{H}$ are trivializations of local sections of $E$ and $U:M\rightarrow G$ encodes the local behaviour of the gauge transformation. It is assumed to be a unitary representation with respect to the Hilbert structure on $\mathcal{H}$. Globally, the gauge transformations are given by the vertical automorphisms.

    \begin{axiom}[Local gauge principle]\index{gauge!principle}
        The Lagrangian functional $\mathcal{L}[\psi]$ is invariant under the action of the gauge group $G$:
        \begin{gather}
            \mathcal{L}[U\psi] = \mathcal{L}[\psi]\,.
        \end{gather}
    \end{axiom}

    Generally, this gauge invariance can be achieved in the following way. Denote the Lie algebra corresponding to $G$ by $\mathfrak{g}$. Because the gauge transformation is local, the information on how it varies from point to point should be able to propagate through space(time). This is done by introducing a new (local) field $A(x)$, called the \textbf{gauge field}. The most elegant formulation uses the concept of covariant derivatives.
    \newdef{Covariant derivative}{\index{covariant!derivative}\index{minimal!coupling}
        When gauging a symmetry group, the ordinary partial derivatives are replaced by the covariant derivative
        \begin{gather}
            \Dr_\mu = \partial_\mu + igA_\mu(x)\,,
        \end{gather}
        where $A_\mu:M\rightarrow\mathfrak{g}$ are the components of the ($\mathfrak{g}$-valued) gauge field (often expressed through coefficients $A_\mu^a$ with respect to a chosen basis of $\mathfrak{g}$). This procedure is called \textbf{minimal coupling} (conform \cref{qm:minimal_coupling}). It should be noted that the explicit action of the covariant derivative depends on the chosen representation of $\mathfrak{g}$ (or $G$) on $\mathcal{H}$. Furthermore, one should pay attention to the fact that the physicist convention was used, where one multiplies the gauge field $A$ by a factor $ig$.\footnote{The imaginary unit turns anti-Hermitian fields into Hermitian fields.}
    }

    So, to achieve gauge invariance, one should replace all derivatives by covariant derivatives. However, for this to be a well-defined operation, one should check that the covariant derivative itself satisfies the local gauge principle, i.e.~$\Dr'\psi'=U\Dr\psi$ (from here on the coordinate-dependence of all fields will be suppressed).
    \begin{align}
        U^{-1}\left(\pderiv{}{x^\mu} + igA_\mu'\right)\psi' &= U^{-1}\left(\pderiv{}{x^\mu} + igA_\mu'\right)U\psi\nonumber\\
        &= U^{-1}\pderiv{U}{x^\mu}\psi + \pderiv{\psi}{x^\mu} + igU^{-1}A_\mu'U\psi\,.
    \end{align}
    This expression can only be equal to $\Dr\psi$ if
    \begin{gather}
        igA_\mu = U^{-1}\pderiv{U}{x^\mu} + igU^{-1}A_\mu'U\,,
    \end{gather}
    which can be rewritten as
    \begin{gather}
        \label{gauge:gauge_transformed_connection}
        A_\mu' = UA_\mu U^{-1} - \frac{1}{ig}(\partial_\mu U)U^{-1}
    \end{gather}
    or, in coordinate-independent form, as
    \begin{gather}
        A' = UAU^{-1} - \frac{1}{ig}\dr UU^{-1}\,.
    \end{gather}
    Up to scale conventions, this is exactly the content of \cref{bundle:local_compatibility} and \cref{bundle:mc_pullback} appearing in the study of principal connections. This should not come as a surprise, since the physical fields are sections of associated vector bundles and, hence, the principal bundle structure lurks in the background. Adding interactions is mathematically equivalent to coupling the space(time) manifold to a principal bundle.

    \begin{example}[QED]
        For quantum electrodynamics, which has $\mathrm{U}(1)\cong S^1$ as its gauge group, the parametrization $U(x)=e^{iq\chi(x)}$ is used with $\chi:\mathbb{R}^n\rightarrow\mathbb{R}$. Minimal coupling is given by
        \begin{gather}
            \partial_\mu\longrightarrow\Dr_\mu = \partial_\mu + iqA_\mu\,,
        \end{gather}
        under the consistency condition
        \begin{gather}
            A_\mu\longrightarrow A_\mu' = A_\mu - \partial_\mu\chi\,,
        \end{gather}
        where $A_\mu$ is the classic electromagnetic potential. These are the formulas introduced in \labelref{chapter:em}.
    \end{example}

\subsection{Gauge fixing}\index{gauge!fixing}

    \newdef{Gauge-fixing condition}{
        Consider the space of all gauge connections $\mathcal{A}$. The group of gauge transformations $\mathcal{G}\equiv\Aut_V(P)$ acts on $\mathcal{A}$ and the quotient space $\mathcal{A}/\mathcal{G}$ is called the \textbf{moduli space of gauge connections}. When calculating, for example, the path integral, it is important to select one representative connection from each orbit.

        Algebro-differential conditions of the form
        \begin{gather}
            \Theta(A)=0\,,
        \end{gather}
        for a \textbf{gauge-fixing function} $\Theta:\mathcal{A}\rightarrow\mathbb{C}$, are called a gauge-fixing conditions.
    }

    \begin{example}[Temporal gauge\footnotemark]\index{gauge!temporal}
        \footnotetext{Also known as the \textbf{Weyl gauge}.}
        On any globally hyperbolic manifold $M\cong\mathbb{R}\times\Sigma^{n-1}$, the gauge field can be transformed to the temporal gauge, i.e.~$A_0=0$.

        \Cref{gauge:gauge_transformed_connection} gives the ODE
        \begin{gather}
            \partial_0U = -A_0U
        \end{gather}
        for the required gauge transformation, which, similar to \eqref{bundle:horizontal_ode_derivative}, can be integrated to obtain
        \begin{gather}
            U(t,x) = \mathcal{T}\exp\left(i\Int_0^tA_0(t',x)\,dt'\right)\,.
        \end{gather}
    \end{example}
    \begin{property}
        The temporal gauge still leaves some residual gauge freedom. It is preserved by any time-independent gauge transformation.
    \end{property}

    The following two gauge-fixing conditions are very similar and both are common in the literature.
    \begin{example}[Lorenz gauge]\index{gauge!Lorenz}
        \begin{gather}
            \partial_\mu A^\mu = 0
        \end{gather}
    \end{example}
    \begin{example}\index{gauge!Coulomb}
        Consider a globally hyperbolic manifold $M\cong\mathbb{R}\times\Sigma^{n-1}$.
        \begin{gather}
            \partial_iA^i= 0
        \end{gather}
        Note that the Coulomb gauge only looks at the `spatial components' of the connection.
    \end{example}

    \newdef{Gribov ambiguity}{\index{Gribov ambiguity}\label{gauge:gribov_ambiguity}
        Given a gauge-fixing function $\Theta:\mathcal{A}\rightarrow\mathbb{C}$, the level set $\Theta^{-1}(0)$ does not necessarily intersect the orbits of the moduli space $\mathcal{A}/\mathcal{G}$ exactly once. It can occur that some orbits are intersected multiple times or that orbits are never reached. These situations, especially the former, are called Gribov ambiguities.
    }

    In the path integral, gauge freedom is usually accounted for through the Fadeev--Popov procedure (but this issue persists in the BRST treatment). However, when Gribov ambiguities are present, this procedure still overcounts the number of inequivalent solutions.
    
    \todo{ADD possible solutions (Gribov horizon, fundamental region, Gribov--Zwanziger framework, ...) + work by N. Vandersickel}

    The following important result is due to \indexauthor{Singer} and shows how generic Gribov ambiguities acutally are.
    \begin{property}
        If the gauge group $G$ is non-Abelian and attention is restricted to gauge fields that extend to the one-point compactification of spacetime, any gauge-fixing condition suffers from Gribov ambiguities.
    \end{property}

\subsection{Anomalies}

    \newdef{Anomaly}{\index{anomaly}
        A classical symmetry that is broken when passing to quantum theory (either due to radiative corrections or nonperturbative corrections).
    }

    \todo{COMPLETE (e.g.~ABJ, Witten/SU(2))}

\section{Yang--Mills theory}\index{Yang--Mills theory}\label{section:yang_mills_theory}
\subsection{Yang--Mills equations}

    The general Yang--Mills action for a compact gauge group $G$ (in vacuum) reads as follows:
    \begin{gather}
        \label{gauge:YM_lagrangian}
        S_{\text{YM}}[A] := -\frac{1}{4}\langle F,F\rangle = -\frac{1}{4}\Int_M\tr(F\wedge\ast F) = -\frac{1}{4}\Int_MF_{\mu\nu}F^{\mu\nu}
    \end{gather}
    for some curvature 2-form $F\in\Omega^2(M;\ad(P))$ as defined through \cref{bundle:cartan_structure_equation}. In the above equation, $\langle\cdot,\cdot\rangle$ combines integration over $M$ and the Killing form (\cref{lie:killing_form}) on $\mathfrak{g}$.

    The variational principle leads to the following equations of motion:
    \begin{gather}
        \label{gauge:YM_equation}
        \mathrm{D}_A\ast F = 0\,,
    \end{gather}
    where $D_A$ is the exterior covariant derivative (\cref{section:exterior_covariant_derivative}). WIth the notation $\widetilde{F}:=\ast F$ (common in physics), the Yang--Mills equations can be written more explicitly as
    \begin{gather}
        \dr\widetilde{F} + A\barwedge\widetilde{F} = 0\,.
    \end{gather}
    By taking the Hodge dual of this expression, the following local coordinate expression can be found:
    \begin{gather}
        \partial_\mu F^{\mu\nu} + [A_\mu,F^{\mu\nu}] = 0\,.
    \end{gather}
    
    \begin{example}[QED]
        Consider the case $G=\mathrm{U}(1)$ as in the previous section. Since this group is Abelian, the second term in the Yang--Mills equation vanishes and one obtains
        \begin{gather}
            \partial_\mu F^{\mu\nu} = 0\,,
        \end{gather}
        Together with the second Bianchi identity (\cref{bundle:second_bianchi_identity}), this gives the Maxwell equations (\cref{section:maxwell_equations}) in disguise, conform \cref{section:diffgeom_electromagnetism}.
    \end{example}

\subsection{Currents and matter coupling}

    For most purposes (such as the Standard Model), one wants to couple the gauge fields to matter. Matter fields are given by sections of associated bundles: $\phi\in\Gamma(P\times_\rho V)$. The kinematic matter Lagrangian is given by
    \begin{gather}
        S_{\text{matter}}[\phi] := \langle\mathrm{D}\phi,D\phi\rangle\,.
    \end{gather}
    The resulting Euler--Lagrange equation is a generalization of the Laplace equation:
    \begin{gather}
        \mathrm{D}\ast\mathrm{D}\phi\,.
    \end{gather}

    If the full Lagrangian $\mathcal{L}[A,\phi]:=\mathcal{L}_{\text{YM}}[A]+\mathcal{L}_{\text{matter}}[\phi]$ is used, the Yang--Mills equation~\eqref{gauge:YM_equation} gets modified by a matter term:
    \begin{gather}
        \mathrm{D}\ast F + \ast[A,\ast\mathrm{D}\phi]=0\,.
    \end{gather}
    These are also sometimes called the \textbf{Yang--Mills--Higgs equations}.\index{Yang--Mills--Higgs equations}

    \todo{ADD gerbes for magnetic currents}

\subsection{2 dimensions}

    In 2D, Yang--Mills theory has some interesting properties. Note that the action still makes sense in 2D, $F$ is a 2-form and $\ast F$ is a 0-form. Using the results for Gaussian integrals (\cref{section:gaussian_integrals}), it can be shown that the following action gives the same partition function (and, hence, the same physics) as the 2D Yang--Mills action:
    \begin{gather}
        S[\phi,A] := -\frac{g^2}{4}\Int_M\langle\phi,\phi\rangle_{\mathfrak{g}}\vol_M-i\Int_M\langle\phi,F\rangle_{\mathfrak{g}}\,,
    \end{gather}
    where $\phi\in\Omega^0(M;\mathfrak{g})$ is called the \textbf{background field}. This first-order formulation of 2D Yang--Mills theory as a background field (BF) theory shows that it is quasi-topological, i.e.~topological when the coupling $g\longrightarrow0$.\index{BF theory} Even though the Yang--Mills term itself depends on the metric and, hence, is not diffeomorphism-invariant, the second term is entirely topological. In this BF theory, the background field $\phi$ acts as a Lagrange multiplier and leads, as for Chern--Simons theory, to flat connections.

\section{Spontaneous symmetry breaking}

    \begin{theorem}[Goldstone]\index{Goldstone}
        Consider a field theory with gauge group $G$ and denote the generators of the corresponding Lie algebra by $X_a$. Generators that do not annihilate the vacuum, i.e.~$X_av\neq0$, or, equivalently, transformations that leave the vacuum invariant, correspond to massless scalar particles.
    \end{theorem}
    The massless bosons from this theorem are called \textbf{Goldstone bosons}.

    \begin{theorem}[Elitzur]\index{Elitzur}\label{gauge:elitzur}
        The only operators in a lattice gauge theory\footnote{A proof for continuum field theories does not exist. However, since nonperturbative field theories are usually constructed through a limit procedure of lattice theories, this does have an impact.} with a nonvanishing VEV are those invariant with respect to local gauge transformations.
    \end{theorem}
    \begin{result}
        Gauge symmetries cannot be spontaneaously broken.
    \end{result}

\subsection{Higgs mechanism}\index{Higgs!mechanism}\index{vacuum!Higgs}\label{section:higgs_mechanism}

    In \cref{bundle:section_bijection}, the equivariant maps corresponding to global sections of a principal bundle were called Higgs fields. In this section, a clarification for this terminology is given. The \textbf{Higgs vacuum} of a $G$-gauge theory, described by a principal bundle $P$, with a $G$-invariant potential $V$ is given by the solutions of the following equations:
    \begin{align}
        V(\phi) &= 0\\
        \nabla\phi &= 0\,,
    \end{align}
    where $\nabla$ is a covariant derivative on $P$ and $\phi$ is a section of some associated (finite-rank) vector bundle $P\times_\rho\mathcal{H}$. If the space of solutions $\mathcal{M}$ to the first equation admits a transitive $G$-action, i.e.~it is a homogeneous space, then, by \cref{group:transitive_action_property}, one can write
    \begin{gather}
        \mathcal{M}\cong G/H\,,
    \end{gather}
    where $H$ is the isotropy group of any given solution. More generally, when the action is not transitive, the solution manifold is still the union of $G$-orbits, all of the form $G/H$ with $H$ the isotropy group of a point in the orbit.

    Now, consider a specific choice of vacuum $m_0\in\mathcal{M}$. If the whole theory were to be $G$-invariant, like the potential $V$, this corresponds to an equivariant map
    \begin{gather}
        \phi:P\rightarrow \mathcal{M}_0\cong G/H\,,
    \end{gather}
    where $\mathcal{M}_0$ is the orbit of $m_0$. This field is called the \textbf{Higgs field} in the physics literature. (For this reason, all such equivariant morphism and their associated sections are called Higgs fields.) The specific choice of vacuum, which generically has the smaller symmetry group $H$, induces by \cref{bundle:reduction_classification} a reduction of the structure group from $G$ to $H$ and, consequently, the symmetry group is said to be \textbf{broken} to $H$.

    After reduction, the $G$-connection can locally be decomposed as follows:
    \begin{gather}
        \iota^*\omega_{\mathfrak{g}} = \omega_{\mathfrak{h}} + \gamma\,,
    \end{gather}
    where $\iota:P_H\hookrightarrow P$ denotes the reduction morphism and $\gamma$ is a tensorial $(\mathrm{Ad}_H,\mathfrak{m})$-form with $\mathfrak{m}$ the complement of $\mathfrak{h}$ in $\mathfrak{g}$.\footnote{To make $\mathrm{Ad}_H$ a well-defined representation on $\mathfrak{m}$, the latter is usually constructed as an orthogonal complement with respect to an $\mathrm{Ad}$-invariant metric on $\mathfrak{g}$. In general, the pair $(\mathfrak{g},\mathfrak{h})$ is required to be reductive (\cref{bundle:klein_reductive}).}

    For the Higgs field $\phi:P\rightarrow\mathcal{M}_0$ and, in fact, for any equivariant map $\phi:P\rightarrow\mathcal{M}_0$ such that $\nabla^H(\phi\circ\iota)=0$, the covariant derivative satisfies
    \begin{gather}
        \nabla_X\phi = (\rho_{e,\ast}\circ\gamma)(X)m_0\,.
    \end{gather}
    The generators $\rho_{e,\ast}(\gamma^i_\mu)m_0$, for $i=1,\ldots,\dim(\mathfrak{m})$, are called the \textbf{(Nambu--)Goldstone bosons}. Since $\dim(\mathfrak{m})=\dim(G)-\dim(H)$, there are $\dim(G)-\dim(H)$ Goldstone fields.\index{Goldstone!boson}\index{Nambu|seealso{Goldstone boson}} As seen above, after reduction, the connection form (gauge field) splits into a connection form for the smaller symmetry group and a set of new (massive) fields. The new connection form is obtained by trivially extending $\omega_{\mathfrak{h}}$ to a connection on $P$ through $G$-equivariance. For such connections, \cref{bundle:connection_reducibility} implies that $\nabla\phi=0$ (this also follows from the expression above since $\gamma$ vanishes for this kind of connection). This is exactly the second condition for the Higgs vacuum.

    Now, what about Elitzur's theorem~\ref{gauge:elitzur}? If its generalization to field theories holds, the above considerations should not hold. Two solutions exist:
    \begin{enumerate}
        \item Realize that the Higgs mechanism can be restated without symmetry breaking.
        \item Realize that the symmetry breaking applies to a global symmetry group and not a local one.
    \end{enumerate}
    Although the first option is a very interesting approach, only the second one will be covered here. \todo{MIGHT ADD FIRST OPTION TOO}

    The crucial point is that the group being broken is the \textbf,{residual symmetry group}, i.e.~the symmetry group that remains after gauge-fixing the theory. When fully fixing a gauge, this residual group coincides with the center of the gauge group $G$, e.g.~$\mathrm{U}(1)$ for $\mathrm{U}(1)$ or $\mathbb{Z}_n$ for $\mathrm{SU}(n)$. Consider for simplicity the typical Mexican hat potential in Yang--Mills theory:
    \begin{gather}
        \mathcal{L} := -\frac{1}{2}\mathrm{tr}\bigl(F_{\mu\nu}F^{\mu\nu}\bigr)+|D_\mu\phi|^2 - V(\phi)\,,
    \end{gather}
    where $V(\phi):=m^2|\phi|^2+\lambda|\phi|^4$ with $\lambda>0$ (note that $m^2<0$ is required for symmetry breaking, i.e.~the mass term is required to be `tachyonic'). The minimum of this potential is achieved for fields of modulus
    \begin{gather}
        |\phi|^2 = -\frac{m^2}{\lambda} \equiv \frac{\nu}{\sqrt{2}}\,.
    \end{gather}
    The states within the level set $V^{-1}(\nu)$ can be parametrized as follows:
    \begin{gather}
        \phi(x) = \frac{\nu}{\sqrt{2}}e^{i\pi(x)/\nu}\,,
    \end{gather}
    where $\pi(x)$ represents the (massless) Nambu--Goldstone boson.\index{Goldstone!boson}\index{Nambu|seealso{Goldstone boson}} Inserting this expression into the Lagrangian gives an expression where the gauge fields $A_\mu$ are replaced by $A_\mu-\frac{1}{q\nu}\partial_\mu\pi$, where $q$ is the charge factor. To get to the Lagrangian of a massive gauge field, the unitary gauge is chosen at this point:
    \begin{gather}
        A_\mu-\frac{1}{q\nu}\partial_\mu\pi\longrightarrow W_\mu\,.
    \end{gather}
    This gives
    \begin{gather}
        \mathcal{L}\sim -\frac{1}{2}\mathrm{tr}\bigl(F_{\mu\nu}F^{\mu\nu}\bigr)+(q\nu)^2W_\mu W^\mu\,.
    \end{gather}
    However, it should be noted that the gauge-fixed field $W_\mu$ actually has a gauge-invariant representation:
    \begin{gather}
        W_\mu = \frac{i}{q}\widetilde{\phi}^*D_\mu\widetilde{\phi}\,,
    \end{gather}
    with
    \begin{gather}
        \widetilde{\phi} := \frac{\phi}{|\phi|}\,.
    \end{gather}

\section{Instantons}\label{gauge:instantons}
\subsection{Instanton sectors}

    \newdef{Instanton}{\index{instanton}
        A Yang--Mills solution with finite Euclidean action.
    }

    To attain finite Euclidean energy, the solution should asymptotically be equal to the vacuum, i.e.~the field strength should vanish at infinity and, hence, the solutions are asymptotically (locally) `pure gauge':
    \begin{gather}
        A_\mu = ig(x)\partial_\mu g^{-1}(x)
    \end{gather}
    for some smooth function $g:\mathbb{R}^n\rightarrow G$. Now, since $F$ should vanish at infinity to obtain a finite (Euclidean) action, it extends to a curvature 2-form on the one-point compactification $S^4$. By the clutching construction (\cref{bundle:clutching_theorem}), the underlying principal bundles are classified by maps $\chi:S^{n-1}\rightarrow G$ up to homotopy or, equivalently, by $\pi_{n-1}(G)$. For example, when working with $G=\mathrm{SU}(2)\cong S^3$ over Minkowski spacetime, the vacua are classified by their first Pontryagin number (which coincides with the second Chern number for traceless Lie algebras) since $\pi_3(S^3)\cong\mathbb{Z}$, which, in this context, is also called the \textbf{instanton number} $k\in\mathbb{Z}$.\footnote{In fact, this holds for all $\mathrm{SU}(N)$ with $N\geq2$ since the homotopy groups $\pi_3(\mathrm{SU}(N))$ stabilize at $N=2$.}\index{instanton!number} In fact, the instanton number can also be calculated through the clutching function $\chi$ (this is essentially the Chern--Simons construction from \cref{section:chern_simons}):
    \begin{gather}
        \label{gauge:winding_number}
        k = -\frac{1}{24\pi^2}\Int_{S^3}\tr\left(\chi\dr\chi^{-1}\wedge\chi\dr\chi^{-1}\wedge\chi\dr\chi^{-1}\right)\,.
    \end{gather}

    \begin{remark}[Literature]
        In the literature, one can often find `physical' explanations for the clutching construction where the `spacetime boundary' at infinity is considered, often also working in the temporal gauge to remove the time dependence. It should be noted that that is only a rough analogy and should not be taken literally. 
    \end{remark}

\subsection{Large gauge transformations}

    \newdef{Topological vacuum}{\index{vacuum!topological}
        A Yang--Mills solution where $F=0$ everywhere. All vacua belong to the zeroth instanton sector.
    }

    As stated in \cref{bundle:vertical_automorphism}, the gauge transformations in a general gauge theory are given by the vertical automorphisms of the underlying principal bundle. When quantizing the theory as in \cref{section:quantum_constrained}, one always starts with a constraint algebra. By \cref{lie:prop_connected} and \cref{lie:exp_result}, however, the exponential map only generates the identity component of the full gauge group. It follows that only the transformations in this connected component give rise to physically equivalent states. The gauge transformations not homotopic to the identity are called \textbf{large gauge transformations}.

    Even though all vacua belong the same instanton sector, there are topological distinctions. Locally, vacua admit the form
    \begin{gather}
        A_\mu = ig(x)\partial_\mu g^{-1}(x)
    \end{gather}
    as remarked for instantons at infinity. As such, one can also calculate their winding number through \cref{gauge:winding_number}. If the total gauge group were connected, the winding number of a vacuum would be fixed. However, in general, large gauge transformations can map vacua with different winding numbers into each other.
    
    Because the topological vacua are not gauge invariant, they are actually not proper vacua (the dinstinction is merely a remnant from the nonconnectedness of gauge symmetries after all). The solution is to take a coherent superposition:
    \begin{gather}
        \label{gauge:theta_vacuum}
        \ket{\theta} := \sum_{k\in\mathbb{Z}}e^{in\theta}\ket{0}_k\,,
    \end{gather}
    with $\theta\in[0,2\pi[$. These gauge-invariant vacua are called \textbf{$\theta$-vacua}\index{vacuum!$\theta$}. It should be noted that states with different values of $\theta$ belong to different superselection sectors.
    
    The different topological vacua can actually also be connected by instanton solutions, which are, here, related to tunneling amplitudes in path integral calculations. Consider a chunk of 4-dimensional spacetime of the form $[-T,T]\times S^3$ for some $T\in\mathbb{R}_0$.\footnote{In \textit{TQFTs} (see \cref{section:tqft}), this can be interpreted as a propagator between spaces of states on spatial slices $S^3$.} Through Chern--Simons theory, the instanton number of a solution $A$ on this manifold can be calculated as
    \begin{gather}
        k \sim \Int_{[-T,T]\times S^3}\dr\mathrm{CS}(A_-,A_+) = \Int_{\{T\}\times S^3}\mathrm{CS}(A_+)-\Int_{\{-T\}\times S^3}\mathrm{CS}(A_-)\,.
    \end{gather}
    If the initial and final states (the $S^3$'s) represent vacua, the gauge fields are locally pure gauge, i.e.
    \begin{gather}
        A_\pm = i\chi_\pm\dr\chi_\pm^{-1}\,,
    \end{gather}
    the Chern--Simons forms give the winding numbers of the `clutching functions' $\chi_\pm:S^3\rightarrow G$:
    \begin{gather}
        k = k_+-k_-\,.
    \end{gather}
    The instanton (with instanton number $k\in\mathbb{Z}$) on the cylinder can, hence, be seen as interpolating between spatial slices of vacua of winding numbers $k_-$ and $k_+=k_-+k$, respectively. (This interpretation will be recalled in \cref{section:chern_simons_theory}.)
    
    \begin{example}[BPST instanton]\index{instanton!BPST}
        Instantons that connect vacua $\ket{0}_n$ to $\ket{0}_{n\pm1}$ and, accordingly, have winding number 1, are called \textbf{Belavin--Polyakov--Schwarz--Tyupkin (BPST) instantons}.
    \end{example}

    \todo{ADD (e.g.~$n$-vacua, BPST solution)}

\subsection{Selfduality}

    Consider the Yang--Mills equations (in vacuum):
    \begin{align*}
        \Dr F &= 0\\
        \Dr\ast F &= 0\,.
    \end{align*}
    The first equation, the Bianchi identity, holds for all connection 1-forms and, hence, any (anti)selfdual connection (\cref{vector:self_dual_form}) is a solution to the Yang--Mills equations. In fact, one has more.
    \begin{property}[Minimizers]
        Consider $G=\mathrm{SU}(N)$. Within a topological sector, i.e.~for a fixed second Chern number
        \begin{gather}
            c_2(F) \sim \frac{\mathrm{tr}(F\wedge F)}{8\pi^2} = \frac{\mathrm{tr}(F^+\wedge F^+)}{8\pi^2} - \frac{\mathrm{tr}(F^-\wedge F^-)}{8\pi^2}\,,
        \end{gather}
        the Yang--Mills action is minimized for (anti)selfdual connections, since
        \begin{gather*}
            F\wedge\ast F = (F^+\wedge\ast F^+)+(F^-\wedge\ast F^-)\,,
        \end{gather*}
        where the notation of \cref{vector:self_dual_form} is used. For negative $c_2(F)$, the antiselfdual connections are minimizers, whereas for positive $c_2(F)$ the selfdual connections are minimizers. For $c_2(F)=0$, one recovers the flat connections.
    \end{property}

\subsection{'t Hooft--Polyakov instantons}\index{instanton!'t Hooft--Polyakov}

    The definition of instantons can, of course, be generalized to gauge theories with matter couplings (or, in fact, to any Lagrangian theory). Consider Yang--Mills--Higgs theory, i.e.~ordinary Yang--Mills theory coupled to a Higgs field $\phi:M\rightarrow V$. The Lagrangian of this theory reads as follows:
    \begin{gather}
        S_{\text{YMH}} = S_{\text{YM}} + S_{\text{Higgs}} = -\frac{1}{4}\langle F\wedge\ast F\rangle + \langle\nabla\phi,\nabla\phi\rangle + V(\phi)\,.
    \end{gather}
    As mentioned before, instantons are defined as the finite-action (in Euclidean signature) solutions of the equations of motions. This is equivalent to finite-energy solutions.

\subsection{Dyons}

    \newdef{Dyon}{\index{dyon}
        A particle carrying both an electric and a magnetic charge.
    }

    \begin{property}[Schwinger--Zwanziger condition]\index{quantization!Schwinger--Zwanziger condition}\index{Schwinger--Zwanziger|see{quantization condition}}
        Recall Dirac's quantization condition~\ref{quantization:dirac_condition} for the electric charge. It can be extended to dyons as follows:
        \begin{gather}
            q_1g_2-g_1q_2 \in 2\pi\mathbb{Z}\,.
        \end{gather}
        This condition implies the following constraints on the charges:
        \begin{enumerate}
            \item The magnetic charges are quantized.
            \item The electric charges differ by an integral multiple of the minimal electric charge $e$.
        \end{enumerate}
    \end{property}

    Now, recall the definition of $\theta$-vacua in \cref{gauge:theta_vacuum}. These give the vacuum for a Yang--Mills Lagrangian to which the (second) Chern number has been added:
    \begin{gather}
        S_\theta := S_{\text{YM}} + \frac{\theta}{16\pi^2}\langle F\wedge F\rangle\,.
    \end{gather}
    The problem here is that this extra (topological) term is not invariant under, for example, inversions (i.e.~parity transformations), unlike the standard Yang--Mills term $\mathrm{tr}(F\wedge\ast F)$. The same issue holds for charge conjugation. The only situation where no \textbf{CP-problem} occurs, is when $\theta=0$.\index{CP-problem} In nature, however, no such violation is observed and, hence, a fine-tuning problem arises (the magnitude of $\theta$ is experimentally bounded by $10^{-10}$)!

    However, assuming the topological Chern term is included, the Schwinger--Zwanziger is modified.
    \begin{property}[Witten effect]\index{Witten}
        When the CP-violating $\theta$-term is added to the Lagrangian, the quantization condition for the electric charge is modified as follows:
        \begin{gather}
            q = ne + \frac{e\theta}{2\pi}N
        \end{gather}
        for some $n,N\in\mathbb{Z}$. In particular, when $\theta\neq0$, the electric charges are integral multiples of the minimal charge $e$.
    \end{property}

\subsection{Electric-magnetic duality}

    \todo{ADD (see blog post and \cref{maxwell:em_duality})}

\section{Chern--Simons theory}\index{Chern--Simons theory}\label{section:chern_simons_theory}

    The Chern--Simons action on a 3-manifold $\Sigma$ is given by integration of the Chern--Simons form (\cref{section:chern_simons} and \cref{bundle:killing_transgression} in particular):
    \begin{gather}
        S_{\mathrm{CS}}[A_\mu] := \frac{k}{4\pi}\Int_\Sigma\tr\left(\dr A\wedge A + \frac{2}{3}A\wedge A\wedge A\right)\,.
    \end{gather}
\chapter{General Relativity}\label{chapter:GR}

    References for this chapter are~\citet{rovelli_covariant_2014,misner_gravitation_2017}. See \cref{chapter:riemann} for an introduction to the theory of Riemannian geometry. The mathematical background for the section on the tetradic formulation of GR can be found in \cref{section:cartan_geometry}.

    In this chapter, the signature convention of the previous chapter is reversed. For general relativity, it is often more convenient to use the mostly-pluses convention (this simply reduces the number of minus signs).

\section{Causal structure}\index{causal}

    \newdef{Null coordinate}{\index{null!vector}\index{lightlike}\index{timelike}\index{spacelike}
        Consider a vector $v\in T_pM$ on a Lorentzian manifold $(M,g)$. This vector is said to be null or \textbf{lightlike} if it satisfies the following condition:
        \begin{gather}
            g_p(v,v) = 0\,.
        \end{gather}
        One can also define \textbf{timelike} and \textbf{spacelike} vectors in a similar way as those vectors having negative and positive norm, respectively.\footnote{For a mostly-minuses signature, one should interchange these definitions.} Spacelike, lightlike and timelike curves are defined as curves for which every tangent vector is respectively spacelike, lightlike or timelike. A curve is said to be \textbf{causal} if its tangent vectors are time- or lightlike.
    }
    \begin{remark}
        Note that Riemannian manifolds do not admit timelike and lightlike vectors as their metric is positive definite.
    \end{remark}

    \newdef{Time orientability}{\index{orientation!time}
        A Lorentzian manifold is said to be time orientable if there exists a nowhere-vanishing timelike vector field. It should be noted that, in contrast to ordinary orientability, this notion is not purely topological. Moreover, neither orientability nor time orientability implies the other. They are independent notions.

        The choice of a time-orienting vector field $\tau$ divides the set of timelike vectors at a point into two equivalence classes. A curve $\gamma$ is said to be \textbf{future directed} (resp.~\textbf{past directed}) if $g(\tau,\dot{\gamma})<0$ (resp.~$g(\tau,\dot{\gamma})>0$).
    }

    \newdef{Causal cone}{
        Let $M$ be a Lorentzian manifold. The causal cone of a point $p\in M$ is defined as the set $J^-(p)\cup J^+(p)\subset M$ of points that are connected to $p$ by a (smooth) causal curve. The past and future cones $J^\pm(p)$ are respectively defined as the sets of points that can be connected to $p$ by a future-directed or past-directed casual curve. The boundaries of these causal cones are called the causal \textbf{lightcones}, sometimes denoted by $V^\pm(p)$.
    }
    \newdef{Causal closure}{\label{relativity:causal_closure}
        Let $S$ be a subset of a Lorentzian manifold. The \textbf{causal complement} of $S$ consists of all points that cannot be causally connected to any point in $S$. The causal closure of $S$ is defined as the causal complement of the causal complement of $S$. A \textbf{causally closed set} is then defined as a set that is equal to its causal closure.
    }

    \newdef{Globally hyperbolic manifold}{\index{hyperbolic!manifold}
        A Lorentzian manifold $M$ that does not contain closed causal curves and for which $J^+(p)\cap J^-(q)$ is compact for any two points $p,q\in M$.
    }

    This concept can also be stated in a different way, through the relation between causal structures and initial value problems.
    \newdef{Cauchy surface}{\index{Cauchy!surface}\index{inextensible}\label{gr:cauchy_surface}
        Consider a Lorentzian manifold $M$. A curve $\gamma:\,]a,b[\rightarrow M$ is said to be an \textbf{inextensible timelike curve} if:
        \begin{enumerate}
            \item $\dot{\gamma}(t)$ is timelike for all $t\in\,]a,b[$.
            \item $\gamma$ does not approach a limit towards $a$ or $b$.
        \end{enumerate}
        Given a subset $S\subseteq M$, the \textbf{domain of dependence} $D(S)$ is given by the points $p\in M$ such that every inextensible timelike curve through $p$ also intersects $S$.

        A subset $S\subset M$ is called a Cauchy surface if every inextensible timelike curve in $M$ intersects $S$ exactly once and $D(S)=M$.
    }
    \begin{property}
        For a Lorentzian manifold $M$, the following conditions are equivalent:
        \begin{itemize}
            \item $M$ is globally hyperbolic, and
            \item $M$ admits a Cauchy surface $S$.
        \end{itemize}
        Moreover, in this case, there exists a homeomorphism (or even diffeomorphism if $S$ is taken to be smooth) $M\cong S\times\mathbb{R}$. A time orientation of $M$ is then equivalent to an orientation of the factor $\mathbb{R}$.
    \end{property}

    \newdef{Stationary spacetime}{\index{stationary}
        A Lorentzian manifod $M$ is said to be stationary if there exists a timelike Killing vector. By the \textit{flowbox theorem}, there always exists a coordinate chart such that, locally, one can choose the Killing vector field to be $\partial_0$ and, hence, $M$ is stationary if one can find a coordinate system for which the metric coefficients are time independent.
    }

    \newdef{Celestial sphere}{\index{sphere!celestial}
        Consider a 4-dimensional Lorentzian manifold $M$. The celestial sphere at a point $p\in M$ is given by the projectivization $\mathbb{P}\bigl(J^+(p)\bigr)\cong\mathbb{CP}^1\cong S^2$, i.e.~it consists of all rays in the future (or, equivalently, past) lightcone at $p$. As~\citet{schreiber_nlab_2008} so nicely phrases it ``\textit{your celestial sphere (the one around the point where your head is) is the sphere of which you directly perceive a portion when you look.}''.
    }
    \begin{property}
        By the above definition and \cref{dirac:exceptional_isomorphism}, the celestial sphere is given by $\mathbb{CP}^1$ or, in the spinorial setting, by the Weyl spinors modulo rescalings.
    \end{property}

\section{Einstein field equations}

    \newformula{Einstein field equations}{\index{Einstein!field equations}\index{stress-energy tensor}\label{relativity:einstein_field_equations}
        The Einstein field equations with a cosmological constant $\Lambda$ read as follows (all fundamental constants are shown for completeness):
        \begin{gather}
            G_{\mu\nu} = \frac{8\pi G}{c^4}T_{\mu\nu} + \Lambda g_{\mu\nu}\,,
        \end{gather}
        where $G_{\mu\nu}$ is the Einstein tensor~\eqref{riemann:einstein_tensor} and $T_{\mu\nu}$ is the stress-energy tensor~\eqref{field:stress_energy_tensor}.

        By taking the trace of both sides, one obtains $T = -R$ and, hence, the Einstein field equations can be rewritten as
        \begin{gather}
            R_{\mu\nu} = \widehat{T}_{\mu\nu}\,,
        \end{gather}
        where $\widehat{T}_{\mu\nu} := T_{\mu\nu} - \frac{1}{2}g_{\mu\nu}T$ is the \textbf{reduced stress-energy tensor}.
    }

    \begin{formula}[Einstein--Hilbert action]\index{action!Einstein--Hilbert}
        The (vacuum) field equations can be obtained by applying the variational principle to the following action:
        \begin{gather}
            S_{\text{EH}}[g_{\mu\nu}] := \Int_M\sqrt{-g}R\,\vol.
        \end{gather}
        For manifolds with boundary, one needs an extra term to make the boundary contributions vanish (as to obtain a well-defined variational problem). This term is due to \indexauthor{Gibbons}, \indexauthor{Hawking} and \indexauthor{York}:\footnote{\indexauthor{Einstein} had, in fact, already introduced a variant: the $\Gamma\Gamma$-\textit{Lagrangian}.}
        \begin{gather}
            S_{\text{GHY}}[g_{\mu\nu}] := \Oint_{\partial M}\epsilon\sqrt{h}K\,\vol,
        \end{gather}
        where $h_{ab}$ is the induced metric on the boundary, $K_{ab}$ is the extrinsic curvature and $\epsilon=\pm1$ is a `function' depending on whether the boundary is timelike or spacelike.
    \end{formula}

\section{Black holes}

    \newformula{Schwarzschild metric}{\index{Schwarzschild metric}
        \begin{gather}
            \label{relativity:schwarzschild_metric}
            ds^2 := \left(1-\frac{R_s}{r}\right)c^2dt^2 - \left(1-\frac{R_s}{r}\right)^{-1}dr^2 - r^2d\Omega^2\,,
        \end{gather}
        where $R_s$ is the Schwarzschild radius
        \begin{gather}
            R_s := \frac{2GM}{c^2}\,.
        \end{gather}
    }

    \begin{theorem}[Birkhoff]\index{Birkhoff}
        The Schwarzschild metric is the unique solution of the vacuum field equation under the additional constraints of asymptotic flatness and staticity.
    \end{theorem}

    \newformula{Reissner--Nordstr\"om metric}{\index{Reissner--Nordstr\"om metric}
        If the black hole is allowed to have an electric charge $Q$, the Schwarzschild metric must be modified in the following way:
        \begin{gather}
            ds^2 := \left(1-\frac{2GM}{r} + \frac{GQ^2}{4\pi r^2}\right)c^2dt^2 - \left(1-\frac{2GM}{r} + \frac{GQ^2}{4\pi r^2}\right)^{-1}dr^2 - r^2d\Omega^2\,.
        \end{gather}
    }

    \begin{remark}
        The electric field generated by a Reissner--Nordstr\"om black hole is given by
        \begin{gather}
            E^r = \frac{Q}{4\pi r^2}\,.
        \end{gather}
        Although the coordinate $r$ is not the proper distance, it still acts as a parameter for the surface of a sphere (as it does in a Euclidean or Schwarzschild metric). This explains why the above formula is the same as the one in classical electromagnetism.
    \end{remark}

    \todo{ADD (KERR--NEWMAN, ERGOSPHERE, PENROSE MECHANISM, KRUSKAL)}

\section{Conserved quantities}

    Before raising any hope that conserved quantities are to be found in GR, the following result is given.
    \begin{property}
        By Noether's third theorem~\ref{var:noether_third_theorem}, there exist no proper (local) conservation laws such as those of momentum and energy in classical mechanics, since the translation group is a subgroup of the infinite-dimensional symmetry group $\mathrm{Diff}(M)$.
    \end{property}

    \todo{COMPLETE}

\section{Tetradic formulation}

    This section starts from the geometric interpretation of the (weak) equivalence principle, i.e.~spacetime is locally modelled on Minkowski space. The natural language for this kind of geometry is that of Cartan geometries (\cref{section:cartan_geometry}). By the Erlangen program, it is known that Minkowski spacetime $M^4$ can be described as the coset space $\mathrm{ISO}(3,1)/\mathrm{SO}(3,1)$. The natural generalization is given by a Cartan geometry with model geometry $\bigl(\mathfrak{iso}(3,1),\mathfrak{so}(3,1)\bigr)$.

    \begin{property}[Cartan connection]
        This way, a $\mathrm{SO}(3,1)$-structure on the spacetime manifold $M$ is obtained, i.e.~a choice of Lorentzian metric $g$. The Cartan connection $\widetilde{\nabla}$ can also be decomposed as $\nabla+\mathrm{e}$ where:
        \begin{itemize}
            \item $\nabla$ defines a $\mathfrak{so}(3,1)$-valued principal connection, and
            \item $\mathrm{e}$ defines an $M^4$-valued solder form.
        \end{itemize}
        The principal connection $\nabla$ is called the \textbf{spin connection} and $\mathrm{e}$ is called the \textbf{vierbein} or \textbf{tetrad}.\footnote{In other dimensions, it is called a \textbf{vielbein}.}\index{spin!connection}\index{vielbein}\index{tetrad} These objects are well known in general relativity. In case of vanishing torsion, the connection $\nabla$ is the ordinary Levi-Civita connection associated to the Lorentzian manifold $M$, and $\mathrm{e}$ gives the isometry between local (flat) Minkowski coordinates and `global' coordinates:
        \begin{gather}
            g := \mathrm{e}^*\eta
        \end{gather}
        or, locally,
        \begin{gather}
            g_{\mu\nu} = e^i_\mu e^j_\nu\eta_{ij}\,.
        \end{gather}
    \end{property}

    Using the tetrad field, one can rewrite the Einstein--Hilbert action in a very elegant way. To this end, a new curvature form is defined:
    \begin{gather}
        F^i_{\ j\mu\nu} := e^i_\rho e_j^\sigma R^\rho_{\ \sigma\mu\nu}\,,
    \end{gather}
    where $R^\rho_{\ \sigma\mu\nu}$ is the ordinary Riemann curvature tensor. The Einstein--Hilbert action is then equivalent\footnote{At least in the case of pure gravity~\citep{rovelli_covariant_2014}.} to the following Yang--Mills-like action.
    \newformula{Palatini action}{\index{Palatini action}
        \begin{gather}
            S[e,\nabla] := \Int_M\mathrm{e}\wedge\mathrm{e}\wedge\ast F\,.
        \end{gather}
        This action is sometimes called the \textbf{tetradic Palatini action} and the resulting formulation of general relativity is called the \textbf{first-order formulation}. If one considers the same action but only as a functional of the tetrad field, one obtains the \textbf{second-order formulation} of gravity.\footnote{These formulations are equivalent for pure gravity. However, when coupling the theory to fermions, they differ by a four-fermion vertex. This follows from the introduction of torsion due to the fermions.}

        Variation of the Palatini action gives the following EOM:
        \begin{itemize}
            \item $\delta\nabla$: $T(\mathrm{e})=0$ or, equivalently, $\nabla(\mathrm{e})\equiv\nabla$, i.e.~the torsion vanishes and the connection $\nabla$ is, on-shell, equal to the Levi-Civita connection on $M$.
            \item $\delta\mathrm{e}$: The metric $g$ satisfies the Einstein field equations.
        \end{itemize}
    }

    Because of its importance in general relativity, the first factor in the Palatini action deserves a name.
    \newdef{Plebanski form}{\index{Plebanski form}
        \begin{gather}
            \Sigma := \mathrm{e}\wedge\mathrm{e}
        \end{gather}
        Because of its internal antisymmetric Lorentz indices, one can interpret this object as an $\mathfrak{so}(3,1)$-valued 2-form.
    }

    As is the case for \textit{4D Yang--Mills theory} (see \cref{section:yang_mills_theory}), one can introduce a topological term that leaves the EOM invariant (up to boundary terms).
    \newdef{Holst action\footnotemark}{\index{action!Holst}\index{Barbero--Immirzi constant}
        \footnotetext{\indexauthor{Holst} was actually the second author to include this term.}
        \begin{align}
            S[\mathrm{e},\nabla] :&= \Int_M\mathrm{e}\wedge\mathrm{e}\wedge\ast F + \frac{1}{\gamma}\Int_M\mathrm{e}\wedge\mathrm{e}\wedge F\nonumber\\
            &= \Int_M\left(\ast\mathrm{e}\wedge\mathrm{e} + \frac{1}{\gamma}\mathrm{e}\wedge\mathrm{e}\right)\wedge F.
        \end{align}
        The coupling constant $\gamma$ is called the \textbf{Barbero--Immirzi} constant.
    }
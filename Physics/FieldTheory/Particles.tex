\chapter{Particle Physics}\label{chapter:particle_physics}

    This chapter summarizes the content of the foregoing chapters and tries to relate it to a more phenomenological approach to modern physics.

    \minitoc

\section{Overview}

    In this section we give a short overview of the particles in the Standard Model of particle physics. This will not include any possible extensions that are currently being researched.

    \newdef{Leptons}{\index{leptons}\index{neutrino}
        The leptons are fermionic particles, i.e.~characterized by the Dirac equation~\ref{dirac:dirac_equation}. They are spin-$\tfrac{1}{2}$ particles that are charged under both the electromagnetic and weak interactions:
        \begin{gather*}
            \begin{pmatrix}
                e^-\\
                \nu_e
            \end{pmatrix}
            \qquad\qquad
            \begin{pmatrix}
                \mu^-\\
                \nu_\mu
            \end{pmatrix}
            \qquad\qquad
            \begin{pmatrix}
                \tau^-\\
                \nu_\tau
            \end{pmatrix}
        \end{gather*}
        and their antiparticles
        \begin{gather*}
            \begin{pmatrix}
                e^+\\
                \overline{\nu}_e
            \end{pmatrix}
            \qquad\qquad
            \begin{pmatrix}
                \mu^+\\
                \overline{\nu}_\mu
            \end{pmatrix}
            \qquad\qquad
            \begin{pmatrix}
                \tau^+\\
                \overline{\nu}_\tau
            \end{pmatrix}\,.
        \end{gather*}
        The particles are grouped in weak doublets. The upper particles in a doublet are always electromagnetically charged, while the lower particles are neutral. From left to right one has the \textbf{electron}, \textbf{mu(on)} and \textbf{tau} particles (and their corresponding \textbf{neutrinos}).
    }

    \newdef{Quarks}{\index{quark}
        A second class of fundamental particles is given by those that not only act through the electromagnetic and weak interactions, but also through the strong interaction. These quarks are also fermionic spin-$\tfrac{1}{2}$ particles:
        \begin{gather*}
            \begin{pmatrix}
                u\\
                d
            \end{pmatrix}
            \qquad\qquad
            \begin{pmatrix}
                s\\
                c
            \end{pmatrix}
            \qquad\qquad
            \begin{pmatrix}
                t\\
                b
            \end{pmatrix}\,.
        \end{gather*}
        The quarks are again paired in weak doublets. The electromagnetic charge of the upper particles is $\tfrac{2}{3}$, while that of the lower particles is $-\tfrac{1}{3}$. Every quark also has strong (\textbf{colour}) charge. The antipartners of these doublets are not shown, but also exist. From left to right and top to bottom one has the \textbf{up}, \textbf{strange}, \textbf{top}, \textbf{down}, \textbf{charm} and \textbf{bottom} quarks.
    }

    \newdef{Gauge bosons}{\index{foton}
        The last class of particles is that of the force carriers. These are bosonic in nature and are thus not characterized by the Dirac equation.
        \begin{itemize}
            \item The electromagnetic interaction mediated by an uncharged spin-1 boson, the \textbf{foton} $\gamma$.
            \item The weak interaction is mediated by three particles:
                \begin{gather*}
                    Z^0\qquad\qquad
                    \begin{pmatrix}
                        W^+\\
                        W^-
                    \end{pmatrix}.
                \end{gather*}
                The $Z^0$-boson is again uncharged, while the the $W$-doublet carries both electromagnetic and weak charges ($\pm1$). These particles are sometimes called \textbf{intermediate vector bosons}.
            \item The strong interaction is mediated by bosons that are only strongly charged, the \textbf{gluons} $g$.
        \end{itemize}
    }
    \begin{remark}
        Note that gravity is not mentioned in the above classification. Although some of the particles above have a mass, gravity is not part of the Standard Model. Mass is simply a parameter in the model.
    \end{remark}

    \newdef{Higgs boson}{
        A last constituent of the Standard Model is the Higgs boson. This scalar field is responsible for giving a small contribution to the lepton masses. It is a spin-0 particle that has no charges at all.
    }
    \begin{remark}[Origin of mass: Yukawa interaction]\index{Yukawa interaction}
        Contrary to what many popular sources state, the Higgs boson is not responsible for `the' mass of matter. The mass of fundamental particles indeed comes from the \textbf{Yukawa interaction} with the Higgs field, but most of the mass of compound particles actually comes from the `nuclear force' that holds them together (gluon and virtual meson exchange). So it is CQD, rather than the Higgs mechanism that is important for the mass of ordinary matter.

        It should be noted that the Yukawa couplings
        \begin{gather}
            \mathcal{L}_{\text{Yukawa}}(\phi,\psi)\sim\overline{\psi}\phi\psi
        \end{gather}
        are introduced in the Standard Model by hand. As of yet, there is no accepted mechanism that generates these terms in the Lagrangian.

        The Higgs particle is, however, important for another aspect of the Standard Model. In the next section it will be explained how this particle is responsible for electroweak symmetry breaking and the mass of the intermediate vector bosons.
    \end{remark}

\section{Standard model}
\subsection{Group theory}

    In this section, the structure of the different particles and multiplets in the Standard Model is related to group theory. Recall that the fundamental interactions are described by the following (Lie) groups:
    \begin{itemize}
        \item Electromagnetic interaction: $\mathrm{U}(1)$,
        \item Weak interaction: $\mathrm{SU}(2)$, and
        \item Strong interaction: $\mathrm{SU}(3)$.
    \end{itemize}

    The fact that all particles can be organised in terms of representations of these groups has some important consequences.
    \begin{property}[Electromagnetism]
        All (electromagnetically interacting) fundamental particles transform under the fundamental representation of $\mathrm{U}(1)$. Since this representation is one-dimensional, interactions with a photon do not change particle types.
    \end{property}
    \begin{property}[Weak interactions]
        All (weakly interacting) fundamental particles can be organised into fundamental representations of $\mathrm{SU}(2)$. Since this representation is two-dimensional, all particles can be organised into doublets. These are the pairs from the previous section. Interactions with the $W^\pm$-bosons interchange the particles in a doublet.

        At this point we have to make an important remark. As was discovered in the 50s and 60s, the weak interaction behaves in a special way with respect to parity transformations. Left-handed and right-handed (Weyl) spinors are treated in a different way. Only the left-handed particles couple to $W^\pm$-bosons. So the doublets shown in the previous section are actually only the left-handed particles. The right-handed particles transform in the trivial, one-dimensional representation of $\mathrm{SU}(2)$, e.g.:
        \begin{gather*}
            \begin{pmatrix}
                e^-\\
                \nu_e
            \end{pmatrix}_L
            \qquad\qquad e^-_R\qquad\qquad\nu_{e,R}\,.
        \end{gather*}
        This is characterized by a so-called V-A theory (\textit{vector minus axial}-theory), where the coupling to the weak interaction is governed by the chirality operator $1-\gamma$ (right-handed spinors are eigenvectors of $\gamma$ with eigenvalue 1).

        In the next section it is explained why the intermediate vector bosons do not form an adjoint triplet $(W^-,W^0,W^+)$ as would be expected from ordinary gauge theory, but rather form a doublet $W^\pm$ and a singlet $Z^0$. To understand this one needs to unify the forces into an electroweak theory and then break symmetry.
    \end{property}
    \begin{property}[Strong interaction]
        All (strongly interaction) fundamental particles can be organised into fundamental representations of $\mathrm{SU}(3)$. Since this representation is three-dimensional all particle types come in three version: red, green and blue. Contrary to the weak interaction, the strong interaction does not interchange particle types, it only changes the `colour' of a particle.

        The gauge bosons mediating the strong interaction form an adjoint representation of $\mathrm{SU}(3)$. The dimension formula for the adjoint representation gives $N^2-1=8$ gluons. Note that free colour charges do not occur outside of interactions due to confinement in QCD, so in practice all particles that can be observed will be white. This also explains why, with three colours, one does not obtain 9 colour-anticolour gluon states. If the singlet state $r\overline{r}+g\overline{g}+b\overline{b}$ would exist, it could couple colourless states to colourful states, which is not allowed.
    \end{property}

\subsection{Symmetry breaking}

    At our energy scale, the gauge bosons represent three distinct fundamental forces. However, at higher energies, these forces are in fact not distinct. They come from a unified force, but due to spontaneous symmetry breaking at lower energies, one observes this distinction.

    The `true' gauge group of the Standard Model is
    \begin{gather}
        \mathrm{U}(1)\times\mathrm{SU}(2)\times\mathrm{SU}(3)/\mathbb{Z}_6\,.
    \end{gather}
    The center $\mathbb{Z}_6$ acts trivially on all particle types. It is generated by the element
    \begin{gather}
        (e^{\pi i/3},-1,e^{2\pi i/3})\,.
    \end{gather}
    As noted in the previous section, the second factor only acts on left-handed particles, so a better notation would be $\mathrm{SU}(2)_L$. However, the first factor is also not what it seems. The $\mathrm{U}(1)$ gauge group is not simply that of the electromagnetic interaction. If $\mathrm{U}(1)$ was the electromagnetic gauge group, by the direct product structure the electromagnetic and weak interactions would commute and, accordingly, the particles in a weak multiplet would necessarily have the same charge. This is clearly not the case. Correctly interpreting these groups will also explain why the weak gauge bosons ($W^\pm$ and $Z^0$) seemingly do not form an adjoint triplet.

    In the 50s a new formula was proposed to be able to organize the dozens of particles that were experimentally detected. This was the so-called \textbf{Gell-Mann--Nishijima formula}\index{Gell-Mann--Nishijima formula}:
    \begin{gather}
        Q = I_3 + \tfrac{1}{2}Y\,,
    \end{gather}
    where the new \textbf{hypercharge} quantum number $Y$ was introduced.\index{hyper-!charge} Originally introduced for the strong interaction, it later became apparent that this formula is actually better suited for describing the weak interaction, $Q$ being the electric charge and $I_3$ being the weak isospin. The weak hypercharge $Y_W$ is now exactly the quantum number related to the $\mathrm{U}(1)$ gauge group. After spontaneous symmetry breaking, the associated $B$-particle mixes up with the $W^0$-particle from the weak adjoint triplet to give the photon $\gamma$ and the $Z^0$-boson:
    \begin{align}
        \gamma &= W^0 + \tfrac{1}{2}B\,,\\
        Z^0 &= W^0 - \tfrac{1}{2}B\,.
    \end{align}
    This recovers the Gell-Mann--Nishijima formula.

    The reason for electroweak symmetry breaking from $\mathrm{SU}(2)_L\times\mathrm{U}(1)_Y$ to $\mathrm{U}(1)_{\text{em}}$ is given by the Higgs mechanism (see \cref{section:higgs_mechanism}). Before symmetry breaking, the Higgs field is a $\mathrm{SU}(2)_L$-doublet. This couples to the electroweak interaction (the doublet has $I=\tfrac{1}{2}$ and $Y_W=1$). However, if the Higgs field has a nonzero vacuum expectation value, the total symmetry group is broken. This has two consequences. First of all, the gauge fields are reduced and the new generator $\gamma$ is obtained (the photon). Secondly, the remaining components of the gauge fields give rise to three massive Goldstone bosons: $W^\pm$ and $Z^0$. It follows that the $\mathrm{SU}(2)_L$-doublets and the intermediate vector bosons in the Standard model do not constitute a gauge theory. Only the electromagnetic sector or the complete unified electroweak theory are proper gauge theories.

\section{Shortcomings}

    Although the Standard Model if often celebrated as the most accurate theory every written down, it might not be flawless. In fact, we know it is not perfect. There are a couple of well-known phenomena that cannot be explained by the model and there are also some hints toward problems within the model itself.

\subsection{Open problems}

    There are two very big open problems in the field of particle physics at this time. Both are related to cosmology and are mainly of importance on larger scales.

    The first, and the most obvious one, is the unification problem for the Standard Model with gravity. No quantummechanical model exists for gravity, but it is known that (almost) all particles interact gravitationally. So the ultimate physical theory should describe both aspects of nature. The current way around this problem is the fact that gravity is extremely weak, the coupling constant (Newton's constant $G$) is much smaller than even the weak coupling constant, and can be ignored for particle physics. However, when trying to describe for example baryogenesis shortly after the Big Bang, an era where the universe was completely different in terms of energy scales, it is expected that all interactions have to be taken into account. (See \cref{chapter:unification} for possible extensions of the Standard Model.)

    Another problem, related to the gravitational one, is the existence of \textbf{dark matter} and \textbf{dark energy}. Astronomical studies have indicated that the visible content of the universe, hadronic matter and light, cannot explain the gravitational phenomena that are observed. The mass required for this phenomena just cannot be accounted for only by matter and energy as we know them. In fact, it is estimated that only about 5\% of the known universe exists out of hadronic matter and light. The remaining 95\% contains about 20\% dark matter and 75\% dark energy.

\subsection{Might-be problems}

    Since the advent of the Standard Model scientists have tried to test its limits. One of the biggest surprises was exactly how robust and accurate this theory was. Even the most precise measurements seemed to agree with the theoretical predictions. However, as was to be expected, there are hints that even this theory might not be the end (even for particle physics).

    The first issue is related to neutrinos. As currently understood, these particles are not charged under the electromagnetic and strong interactions. However, because of weak symmetry breaking, the right-handed neutrinos also do not interact weakly. This means that the only possible way to observe them would be through gravitational interactions. Because of \textit{neutrino oscillations} it is known that neutrinos have an intrinsic mass, but this does not really increase the hope of detecting right-handed particles. One needs an extremely large amount of matter in the detectors in order to have a reasonable chance of detecting gravitational interactions with neutrinos.

    \todo{ADD (e.g.~sterile neutrinos)}

    A second possible issue arose a coupe of years ago after FermiLab released measurements of the magnetic moment of muons, the muon $g-2$ experiment. From basic quantum theory, it is known that spin induces a magnetic dipole moment:
    \begin{gather}
        \vector{B} = g\frac{e}{2m}\vector{S}\,.
    \end{gather}
    \todo{ADD (spin-orbit coupling to [QM])}
    The factor $g$ is called the \textbf{anomalous magnetic moment}. Without quantum field theory, this factor would be equal to 2, but due to higher-order loop corrections, the value is slightly greater than 2. The latest measurements at FermiLab, however, indicated that the true value differs from the predicted one.

    Even more recently, a third possible deviation from the Standard Model was found. Measurements of the $W$-boson mass seemed to indicate that it did not exactly match the predictions of the Standard Model, the mass was found to be slightly larger. If found to be true, this is especially problematic since the value of the weak mixing angle is fixed and the mass of the $Z$-boson is measured very accurately.
\chapter{Special Relativity}

    In this chapter, as will be the case in the chapters on quantum field theory, the mostly-minuses convention for the Minkowski signature is adopted unless stated otherwise, i.e.~the signature is $(+,-,-,-)$. Furthermore, natural units will be used unless stated otherwise, i.e.~$\hbar = c = 1$. This follows the introductory literature such as~\citet{greiner_field_1996,peskin_introduction_1995}.

    \minitoc

\section{Lorentz transformations}

    \begin{notation}\index{Lorentz!factor}
        In the context of special relativity, it is often useful to introduce the following quantities:
        \begin{align}
            \beta &:= \frac{v}{c}\,,\\
            \label{relativity:lorentz_factor}
            \gamma &:= \frac{1}{\sqrt{1 - \beta^2}}\,.
        \end{align}
        The latter quantity is called the \textbf{Lorentz factor}.
    \end{notation}
    \newformula{Lorentz transformations}{\label{relativity:lorentz_transformations}
        Let $\mathbf{V}$ be a 4-vector. A Lorentz boost along the $x^1$-axis is given by the following transformation:
        \begin{gather}
            \begin{aligned}
                V'^0 &= \gamma\left(V^0 - \beta V^1\right)\\
                V'^1 &= \gamma\left(V^1 - \beta V^0\right)\\
                V'^2 &= V^2\\
                V'^3 &= V^3.
            \end{aligned}
        \end{gather}
    }
    \begin{remark}
        Setting $c=+\infty$ in the previous formulas recovers the Galilei transformations from classical mechanics (cf.~In\"on\"u--Wigner contractions (\cref{lie:inonu_wigner})).
    \end{remark}

\section{Energy and momentum}

    \newformula{4-velocity}{\index{velocity}\label{relativity:4_velocity}
        In analogy to the definition of velocity in classical mechanics, the 4-velocity is defined as follows:
        \begin{gather}
            \mathbf{U} := \left(\deriv{x^0}{\tau},\deriv{x^1}{\tau},\deriv{x^2}{\tau},\deriv{x^3}{\tau}\right)\,.
        \end{gather}
        By applying the formulas for proper time and time dilatation, one obtains:
        \begin{gather}
            \mathbf{U} = \left(\gamma c,\gamma\vector{u}\right)\,.
        \end{gather}
    }
    \newformula{4-momentum}{\index{momentum}\label{relativity:4_momentum}
        The 4-momentum is defined as follows:
        \begin{gather}
            \mathbf{p} = m_0\mathbf{U}\,,
        \end{gather}
        or, after defining $E := cp^0$:
        \begin{gather}
            \mathbf{p} = \left(\frac{E}{c},\gamma m_0\vector{u}\right)\,.
        \end{gather}
    }

    \newdef{Relativistic mass}{\index{mass}
        The factor $m:=\gamma m_0$ in the momentum 4-vector is called the relativistic mass. By introducing this quantity, the classical formula $\vector{p} = m\vector{u}$ for the 3-momentum can be generalized to 4-momenta $\mathbf{p}$.
    }

    \begin{formula}[Relativistic energy relation]\index{energy}\index{Einstein!energy relation}\label{relativity:relativistic_energy}
        \begin{gather}
            E^2 = p^2c^2 + m^2c^4
        \end{gather}
        This formula is often called the \textbf{Einstein relation}.
    \end{formula}

\section{Action principle}

    The main guiding principles for writing down a relativistic action for a point particle are locality and invariance. The latter means that one should only use geometric quantities, i.e.~diffeomorphism-invariant quantities, while the former means that these should only depend on local information. For a single particle, the most obvious choice of action would be one that is proportional to the proper time along the worldline of the particle or, more invariantly, the arc length of the worldline:
    \begin{gather}
        S_{\text{point}}\sim\Int_\gamma ds\,.
    \end{gather}
    To get the units right, one should multiply by suitable Lorentz-invariant constants:
    \begin{gather}
        \label{relativity:worldline_action}
        S_{\text{point}}:=mc\Int_\gamma ds\,.
    \end{gather}
    By reparametrization invariance, one can choose a specific time coordinate, e.g.~$\tau=ct$. In this coordinate system, the action becomes
    \begin{gather}
        S_{\text{point}}=mc\Int_\gamma\sqrt{1-\frac{v^2}{c^2}}c\,dt\,,
    \end{gather}
    with $v$ the speed of the particle. The Lagrangian density can be Taylor expanded as
    \begin{gather}
        L_{\text{point}}=mc^2 - \frac{1}{2}mv^2 + \cdots\,,
    \end{gather}
    which recovers (up to a constant) the classical Lagrangian of a massive point particle when the speed is small $v\ll c$.

    When trying to quantize this action, however, a problem occurs. After a Legendre transformation, the Hamiltonian becomes $H=\sqrt{p^2+m^2c^4}$ (this is just the Einstein relation~\ref{relativity:relativistic_energy}). When applying the ordinary Dirac procedure $p_\mu\longrightarrow i\partial_\mu$, this becomes a nonlocal operator (the whole reason for why Dirac introduced spinors).

    Instead of passing to a spinor framework, one can try to write down an equivalent action that gives rise to a local Hamiltonian. One possibility is to pass to \textbf{light cone coordinates}:\index{light cone!coordinates}
    \begin{gather}
        x^\pm := x^0\pm x^1\,.
    \end{gather}
    However, here one makes a specific split of coordinates, which ruins Lorentz invariance. A better idea is to introduce a dynamical Lagrange multiplier (from here on, natural units are used):
    \begin{gather}
        S_{\text{point}} := \frac{1}{2}\Int_\gamma\left[\eta^{-1}\left(\deriv{x^\mu}{\tau}\right)^2-m^2\eta\right]d\tau\,.
    \end{gather}
    The equation of motion for $\eta$ is algebraic and gives
    \begin{gather}
        \eta = \frac{1}{m}\sqrt{-g_{\mu\nu}\deriv{x^\mu}{\tau}\deriv{x^\nu}{\tau}}\,.
    \end{gather}
    To find an interpretation of this multiplier, it is useful to consider the case where a metric $h$ is introduced on the worldline. This implies that the action has to be `covariantized':
    \begin{gather}
        \Int_\gamma\left(g_{\mu\nu}\deriv{x^\mu}{\tau}\deriv{x^\mu}{\tau}+m^2\right)d\tau\longrightarrow\Int_\gamma\sqrt{h}\left(hg_{\mu\nu}\deriv{x^\mu}{\tau}\deriv{x^\mu}{\tau}+m^2\right)d\tau\,.
    \end{gather}
    From this perspective, it is clear that the Lagrange multiplier can be viewed as the square root of a dynamical metric on the worldline. It is an example of a \textit{vielbein} (in this case, an `\textit{einbein}'). Furthermore, this action is a so-called \textit{nonlinear $\sigma$-model}.
\chapter{Fluid Mechanics}

\section{Cauchy stress tensor}

	\begin{theorem}[Cauchy's stress theorem\footnotemark]\index{Cauchy!stress theorem}
		\footnotetext{Also known as \textbf{Cauchy's fundamental theorem}.}
	    	Knowing the stress vectors acting on the coordinate planes through a point $A$ is sufficient to calculate the stress vector acting on an arbitrary plane passing through $A$.
	\end{theorem}

	The \textit{Cauchy stress theorem} is equivalent to the existence of the following tensor:
	\newdef{Cauchy stress tensor}{\index{Cauchy!stress tensor}
    		The Cauchy stress tensor is a $(0,2)$-tensor $\mathbf{T}$ that gives the relation between a stress vector associated to a plane and the normal vector $\vector{n}$ to that plane:
	        \begin{gather}
        		\vector{t}_{(\vector{n})} = \mathbf{T}(\vector{n})
	        \end{gather}
	}
	\begin{example}
    		For identical particles, the stress tensor is given by:
	        \begin{gather}
        		\mathbf{T} = -\rho \langle\vector{w}\otimes\vector{w}\rangle
	        \end{gather}
	        where $\vector{w}$ is the random component of the velocity vector and $\langle\cdot\rangle$ denotes the expectation value (see \ref{prob:expectation_value}).
	\end{example}

	\begin{theorem}[Cauchy's lemma]\index{Cauchy!lemma}
    		The stress vectors acting on opposite planes are equal in magnitude but opposite in direction:
	        \begin{gather}
        		\vector{t}_{(-\vector{n})} = -\vector{t}_{(\vector{n})}
	        \end{gather}
	\end{theorem}

	\newformula{Cauchy momentum equation}{\index{Cauchy!momentum equation}
    		From Newton's second law \ref{forces:force} it follows that:
	        \begin{gather}
        		\Deriv{\vector{P}}{t} = \int_V\vector{f}(\vector{x}, t)dV + \oint_S\vector{t}(\vector{x}, t)dS
	        \end{gather}
        	where $\vector{P}$ is the momentum density, $\vector{f}$ are body forces and $\vector{t}$ are surface forces (such as shear stress). Using Cauchy's stress theorem and the divergence theorem \ref{vectorcalculus:divergence_theorem} we get
	        \begin{gather}
        		\Deriv{\vector{P}}{t} = \int_V\left[\vector{f}(\vector{x}, t) + \nabla\cdot\mathbf{T}(\vector{x}, t)\right]dV
	        \end{gather}
        	The left-hand side can be rewritten using \ref{fluidum:result1} as
	        \begin{gather}
        		\int_V\rho\Deriv{\vector{v}}{t}dV = \int_V\left[\vector{f}(\vector{x}, t) + \nabla\cdot\mathbf{T}(\vector{x}, t)\right]dV
	        \end{gather}
	}
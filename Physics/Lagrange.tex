\chapter{Lagrangian and Hamiltonian Mechanics}\label{chapter:lagrange}

\section{Action functional}

    \begin{definition}[Generalized coordinates]\index{generalized!coordinates}
        The generalized coordinates $q_k$ are independent coordinates that completely describe the current configuration of a system relative to a reference configuration.

        When a system has $N$ degrees of freedom and $n_c$ constraints, there are $(N - n_c)$ generalized coordinates. Furthermore, every set of generalized coordinates describing the same system contains exactly $(N - n_c)$ coordinates.
    \end{definition}
    \begin{definition}[Generalized velocities]
        The generalized velocities $\dot{q}_k$ are the derivatives of the generalized coordinates with respect to time.
    \end{definition}
    \begin{notation}
        Given a Lagrangian function, depending on $n$ generalized coordinates and their associated velocities, we often introduce the following shorthand notation:
        \begin{gather}
            \label{lagrange:notational_convention_1}
            L\left(q(t), \dot{q}(t), t\right) \equiv L\left(q_1(t), ..., q_n(t), \dot{q}_1(t), ..., \dot{q}_n(t), t\right)
        \end{gather}
    \end{notation}

    \begin{definition}[Action]\index{action}\label{lagrange:action}
        \begin{gather}
            S[q] := \int_{t_1}^{t_2}L\left(\vector{q}(t), \dot{\vector{q}}(t), t\right)dt.
        \end{gather}
    \end{definition}

\section{Euler-Lagrange equations\texorpdfstring{$^\dag$}\ }\index{Lagrange!equations of motion}\index{Euler-Lagrange|see{Lagrange}}

    \begin{formula}[Euler-Lagrange equation of the first kind]
        \begin{gather}
            \label{lagrange:first_kind}
            \deriv{}{t}\left(\pderiv{T}{\dot{q}^k}\right) - \pderiv{T}{q^k} = Q_k
        \end{gather}
        where $T$ is the total kinetic energy.
    \end{formula}
    \begin{formula}[Euler-Lagrange equation of the second kind]
        \begin{gather}
            \label{lagrange:second_kind}
            \deriv{}{t}\left(\pderiv{L}{\dot{q}^k}\right) - \pderiv{L}{q^k} = 0
        \end{gather}
    \end{formula}

\section{Conservation laws and symmetry properties}

    \begin{definition}[Conjugate momentum]\index{momentum!conjugate}\label{lagrange:conjugate_momentum}
        Also called the \textbf{canonically} conjugate momentum.
        \begin{gather}
            p_k = \pderiv{L}{\dot{q}^k}
        \end{gather}
    \end{definition}
    \begin{definition}[Cyclic coordinate]
        If the Lagrangian $L$ does not explicitly depend on a coordinate $q_k$, the coordinate is said to be cyclic.
    \end{definition}

    \begin{property}[Noether's theorem]\index{Noether}\label{lagrange:noether_cyclic}
        The conjugate momentum of a cyclic coordinate is a conserved quantity:
        \begin{gather}
            \dot{p}_k \overset{\ref{lagrange:conjugate_momentum}}{=} \deriv{}{t}\left(\pderiv{L}{\dot{q}^k}\right) \overset{{\ref{lagrange:second_kind}}}{=}\pderiv{L}{q^k} \overset{\underset{\text{\tiny{coord.}}}{\text{\tiny{cyclic}}}}{=} 0.
        \end{gather}
    \end{property}

\section{Hamilton's equations}

    \newdef{Canonical coordinates}{\index{canonical!coordinates}
        Consider the generalized coordinates $(q, \dot{q}, t)$ from the Lagrangian formalism. Using these we can define a new set of coordinates, called canonical coordinates, by exchanging the time-derivatives $\dot{q}^i$ in favour of the conjugate momenta $p_i$ (see definition \ref{lagrange:conjugate_momentum}).
    }

    \newdef{Hamiltonian function}{\index{Hamilton!function}\label{hamilton:hamiltonian}
        Given a Lagrangian $L$ one defines the Hamiltonian function as follows:
        \begin{gather}
            H(q, p, t) := \sum_ip_i\dot{q}^i - L(q, p, t).
        \end{gather}
    }

    \newformula{Hamilton's equations\footnotemark}{\index{Hamilton!equations of motion}\label{lagrange:hamilton_equations}
        \footnotetext{Also known as the \textit{canonical equations of Hamilton}.}
        \begin{align}
            \dot{q}^i &= \pderiv{H}{p_i}\\
            \dot{p_i} &= -\pderiv{H}{q^i}
        \end{align}
        Systems obeying these equations are called \textbf{Hamiltonian systems} (see section \ref{section:hamiltonian_dynamics} for a formal introduction).
    }

    The formula to obtain the Hamiltonian from the Lagrangian is an application of the following more general \textbf{Legendre transformation}:
    \newdef{Legendre transformation}{\index{Legendre!transformation}
        Consider an equation of the following form:
        \begin{gather}
            \label{lagrange:legendre1}
            df = udx + vdy
        \end{gather}
        where $u = \pderiv{f}{x}$ and $v = \pderiv{f}{y}$.

        Suppose we want to perform a coordinate transformation $(x, y)\rightarrow (u, y)$ that preserves the general form of \ref{lagrange:legendre1}. To this end we consider the function
        \begin{gather}
            \label{lagrange:legendre}
            g = f - ux.
        \end{gather}
        Differentiation gives
        \begin{gather*}
            dg = vdy - xdu
        \end{gather*}
        which has the form of \ref{lagrange:legendre1} as desired. The quantities $v$ and $x$ are now given by
        \begin{gather}
            x = -\pderiv{g}{u}\qquad\text{and}\qquad v=\pderiv{g}{y}.
        \end{gather}
        The transition $f\longrightarrow g$ defined by equations \ref{lagrange:legendre1} and \ref{lagrange:legendre} is called a Legendre transformation.
    }
    \begin{remark}
        Although the previous derivation used only 2 coordinates, the definition of Legendre transformations can easily be generalized to more coordinates.
    \end{remark}

\subsection{Poisson brackets}

    \newdef{Poisson bracket}{\index{Poisson!bracket}
        \nomenclature[O_sympoi]{$\{\cdot, \cdot\}$}{Poisson bracket}
        \begin{gather}
            \{A, B\} = \pderiv{A}{p}\pderiv{B}{q} - \pderiv{B}{p}\pderiv{A}{q}
        \end{gather}
        where $q, p$ are the generalized coordinates in the Hamiltonian formalism.
    }
    \remark{Some authors define the Poisson bracket with the opposite sign. One should always pay attention to which convention is used.}
    \newformula{Total time derivative}{
        Hamilton's equations imply the following expression for the total time derivative:
        \begin{gather}
            \deriv{F}{t} = \pderiv{F}{t} + \{H, F\}
        \end{gather}
        where $\{\cdot,\cdot\}$ is the Poisson bracket as defined above and $H$ is the Hamiltonian \ref{hamilton:hamiltonian}.
    }

\section{Hamilton-Jacobi equation}

    For a formal introduction see section \ref{section:hamilton_jacobi}.

\subsection{Canonical transformations}

    \newdef{Canonical transformations}{\index{canonical!transformation}
        A canonical transformation is a transformation that leaves the Hamiltonian equations of motion unchanged. Mathematically this means that the transformations leave the action invariant up to a constant, or equivalently, they leave the Lagrangian invariant up to a complete time-derivative:
        \begin{gather}
            \sum_i \dot{q}^ip_i - H(q, p, t) = \sum_i \dot{Q}^iP_i - K(Q, P, t) - \deriv{G}{t}(Q, P, t).
        \end{gather}
        The function $G$ is called the \textbf{generating function} of the canonical transformation. The choice of $G$ uniquely determines the transformation (the converse is not true).
    }

    \newformula{Hamilton-Jacobi equation}{\index{Hamilton-Jacobi}\index{Hamilton!principal function}
        Sufficient conditions for the generating function $S$ are given by:
        \begin{align*}
            P_i &=\ds\pderiv{S}{Q^i}\\
            Q^i &=\ds\pderiv{S}{P_i}
        \end{align*}
        and \[K = H + \pderiv{S}{t}.\]
        Choosing the new Hamiltonian function $K$ to be 0 gives the Hamilton-Jacobi equation:
        \begin{gather}
            \label{lagrange:hamilton_jacobi_equation}
            H\left(q, \pderiv{S}{q}, t\right)+\pderiv{S}{t} = 0.
        \end{gather}
        The function $S$ is called \textbf{Hamilton's principal function}.
    }

    \begin{property}
        The new coordinates $P_i$ and $Q^i$ are all constants of motion. This follows immediately from the choice $K = 0$.
    \end{property}

    \newdef{Hamilton's characteristic function}{\index{Hamilton!characteristic function}\index{energy}
        For time-independent systems the HJE can be rewritten as follows:\footnote{Note that this form is the same as equation \ref{diff:hamilton_jacobi} after a redefinition of $H$.}
        \begin{gather}
            \label{lagrange:time_independent_hje}
            H\left(q, \pderiv{S}{q}\right) = -\pderiv{S}{t} := E.
        \end{gather}
        We thus obtain the classical result that for time-independent systems the Hamiltonian function is a constant of motion (the \textbf{energy}). Integration with respect to time then gives the following form of the principal function:\footnote{Note that often $E$ will be the energy, however this is not a general fact.}
        \begin{gather}
            S(q, p, t) = W(q, p) - Et.
        \end{gather}
        The time-independent function $W$ is called Hamilton's characteristic function.
    }

\subsection{St\"ackel potentials}

    \begin{remark}
        If the principal function can be separated into $n$ equations, the HJE splits up into $n$ equations of the form
        \begin{gather}
            h_i\left(q^i, \pderiv{S}{q^i}, \alpha_i\right) = 0.
        \end{gather}
        The partial differential equation for $S$ can thus be rewritten as a system of $n$ ordinary differential equations.
    \end{remark}

    \begin{property}[St\"ackel condition]\index{St\"ackel potential}
        The Hamilton-Jacobi equation is separable if and only if the potential is of the following form:
        \begin{gather}
            \label{lagrange:stackel_condition}
            V(q) = \sum_{i=1}^n\ds\frac{1}{G_i^2(q)}W_i(q^i)
        \end{gather}
        whenever the Hamiltonian function can be written as
        \begin{gather}
            H(q, p) = \frac{1}{2}\sum_i\stylefrac{p_i^2}{G^2_i(q)} + V(q).
        \end{gather}
        Potentials of this form are called \textbf{St\"ackel potentials}.
    \end{property}

\section{Analytical mechanics}
\subsection{Phase space}

    \newdef{Phase space}{\index{phase space}
        The set of all possible $n$-tuples\footnote{Not only those as given by the equations of motion.} $(q^i, p_i)$ of generalized coordinates and associated momenta is called the phase space of the system. Given a system living on a smooth manifold $Q$, its phase space is modelled by the cotangent bundle $T^*Q$.
    }

    \newdef{Libration}{\index{libration}
        A closed trajectory for which the coordinates take on only a subset of the allowed values. It is the generalization of an oscillation. Topologically it is characterized by a closed trajectory that is contractible.
    }
    \newdef{Rotation}{\index{rotation}
        A closed trajectory in which at least one of the variables takes on all possible values. Topologicaly these are characterized as those closed trajectories that are noncontractible.
    }

    \newdef{Separatrix}{\index{separatrix}
        When plotting different (closed) trajectories in the phase space of a system, the curve that separates regions of librations and rotations is called the separatrix.\footnote{In general the separatrix of a dynamical system is a curve that separates regions with different behaviour.}
    }

\subsection{Material derivative}

    \newdef{Lagrangian derivative\footnotemark}{\index{Lagrange!derivative|see{material derivative}}\index{material!derivative}\label{phasespace:lagrangian_derivative}
        \footnotetext{Also known as the \textbf{material derivative}, especially when applied to fluid mechanics.}
        Let $a(\vector{r}, \vector{v}, t)$ be a property defined at every point of the system. The Lagrangian derivative along a path $(\vector{r}(t), \vector{v}(t))$ in phase space is given by
        \begin{align}
            \ds\Deriv{a}{t} &= \lim_{\Delta t\rightarrow0}\ds\frac{a(\vector{r} + \Delta\vector{r}, \vector{v} + \Delta\vector{v}, t+\Delta t) - a(\vector{r}, \vector{v}, t)}{\Delta t}\nonumber\\
            &= \pderiv{a}{t} + \deriv{\vector{r}}{t}\cdot\pderiv{a}{\vector{r}} + \deriv{\vector{v}}{t}\cdot\pderiv{a}{\vector{v}}\nonumber\\
            &= \pderiv{a}{t} + \vector{v}\cdot\nabla a + \deriv{\vector{v}}{t}\cdot\pderiv{a}{\vector{v}}.
        \end{align}
        The second term $\vector{v}\cdot\nabla a$ in this equation is called the \textbf{advective} term.
    }
    \begin{remark}
        In the case that $a(\vector{r}, \vector{v}, t)$ is a tensor field the gradient $\nabla$ has to be replaced by the covariant derivative. The advective term is then called the \textbf{convective} term.
    \end{remark}

    \begin{result}
        If we take $a(\vector{r}, \vector{v}, t) = \vector{r}$ we obtain
        \begin{gather}
            \Deriv{\vector{r}}{t} = \vector{v}.
        \end{gather}
    \end{result}

\subsection{Liouville's theorem}

    \begin{formula}[Liouville's lemma]
        Consider a phase space volume element $dV_0$ moving along a path $(\vector{r}(t), \vector{v}(t))\equiv\vector{x}(t)$. The Jacobian $J(\vector{x}, t)$ associated with this motion is given by
        \begin{gather}
            J(\vector{x}(t)) = \deriv{V}{V_0} = \det\left(\pderiv{\vector{x}}{\vector{x}_0}\right) = \sum_{ijklmn}\varepsilon^{ijklmn}\pderiv{x^1}{x^i_0}\pderiv{x^2}{x^j_0}\pderiv{x^3}{x^k_0}\pderiv{x^4}{x^l_0}\pderiv{x^5}{x^m_0}\pderiv{x^6}{x^n_0}.
        \end{gather}
        The Lagrangian derivative of this Jacobian then becomes
        \begin{gather}
            \label{fluidum:jacobian_derivative}
            \Deriv{J}{t} = (\nabla\cdot\vector{x})J.
        \end{gather}
        Furthermore, using Hamilton's equations \ref{lagrange:hamilton_equations} it is easy to prove that
        \begin{gather}
            \nabla\cdot\vector{x} = 0
        \end{gather}
        and hence that the material derivative of $J$ vanishes.
    \end{formula}

    \begin{theorem}[Liouville]\index{Liouville!theorem on phase spaces}\label{fluidum:liouvilles_theorem}
        Let $V(t)$ be a phase space volume containing a fixed set of particles. Applying Liouville's lemma gives
        \begin{gather}
            \Deriv{V}{t} = \Deriv{}{t}\int_{\Omega(t)}d^6x = \Deriv{}{t}\int_{\Omega_0}J(\vector{x}, t)d^6x_0 = 0.
        \end{gather}
        It follows that the phase space volume of a Hamiltonian system is invariant with respect to time-evolution.
    \end{theorem}

    \newformula{Boltzmann's transport equation}{\index{Boltzmann!transport equation}\index{Vlasov equation}
        Let $F(\vector{r}, \vector{v}, t)$ be the mass distribution function
        \begin{gather}
            M_{tot} = \int_{\Omega(t)}F(\vector{r}, \vector{v}, t)d^6x.
        \end{gather}
        From the conservation of mass we can derive the following formula:
        \begin{gather}
            \label{fluidum:boltzmann_transport_gather}
            \Deriv{F}{t} = \pderiv{F}{t} + \deriv{\vector{r}}{t}\cdot\pderiv{F}{\vector{r}} - \nabla V\cdot\pderiv{F}{\vector{v}} = \left[\pderiv{F}{t}\right]_{col}
        \end{gather}
        where the right-hand side gives the change of $F$ due to collisions.\footnote{The collisionless form of this equation is sometimes called the \textbf{Vlasov equation}.} This partial differential equation in 7 variables can be solved to obtain $F(\vector{r}, \vector{v}, t)$.
    }

    Consider a Hamiltonian system with a phase space $\mathcal{V}$. By Liouville's theorem, the phase flow generated by the equations of motion is a measure- or volume-preserving map $g:\mathcal{V}\rightarrow\mathcal{V}$. This leads us to the following theorem:
    \begin{theorem}[Poincar\'e recurrence theorem]\index{Poincar\'e!recurrence theorem}
        Let $\mathcal{V}_0$ be the phase space volume of the system. For every point $x_0\in\mathcal{V}_0$ and for every neighbourhood $U$ of $x_0$ there exists a point $y\in U$ such that $g^n(y)\in U$ for every $n\in\mathbb{N}$.
    \end{theorem}

    \begin{theorem}[Strong Jeans's theorem\footnotemark]\index{Jeans}\index{isolating integrals}
        \footnotetext{Actually due to Donald Lynden-Bell.}
        The distribution function $F(\vector{r}, \vector{v})$ of a time-independent system for which almost all orbits are regular can be expressed in terms of 3 integrals of motion.
    \end{theorem}
    The constants in Jeans's theorem are called the \textup{\textbf{isolating integrals}} of the system.

\subsection{Continuity equation}

    \newformula{Reynolds transport theorem\footnotemark}{\index{Reynolds!transport theorem}\index{Leibniz!integral rule}\label{fluidum:reynolds_transport_theorem}
        \footnotetext{This is a 3D extension of the \textit{Leibniz integral rule}.}
        Consider a quantity \[F = \int_{V(t)}f(\vector{r}, \vector{v}, t)dV.\] Using equation \ref{fluidum:jacobian_derivative} together with the divergence theorem \ref{vectorcalculus:divergence_theorem} gives us
        \begin{gather}
            \Deriv{F}{t} = \int_V\pderiv{f}{t}dV + \oint_Sf\vector{v}\cdot d\vector{S}.
        \end{gather}
    }
    \newformula{Continuity equations}{\index{continuity!equation}
        For a conserved quantity the equation above becomes:
        \begin{gather}
            \label{fluidum:lagrangian_continuity_gather}
            \Deriv{f}{t} + (\nabla\cdot\vector{v})f = 0
        \end{gather}
        \begin{gather}
            \label{fluidum:eulerian_continuity_gather}
            \pderiv{f}{t} + \nabla\cdot(f\vector{v}) = 0.
        \end{gather}
        If we set $f = \rho$ (the mass density) then the first equation is called the \textbf{Lagrangian continuity equation} and the second equation is called the \textbf{Eulerian continuity equation}. Both equations can be found by pulling the Lagrangian derivative inside the integral on the left-hand side of \ref{fluidum:reynolds_transport_theorem}.

        The difference between these two equations corresponds to the way we observe the system. In the Eulerian approach one observes a fixed point in space and measures how a given quantity at that point evolves. In the Lagrangian approach one observes a given point (or particle) in the system and measures how a given quantity evolves around the chosen point as it moves throughout space.
    }
    \begin{result}
        Combining the Reynolds transport theorem with the Lagrangian continuity equation gives the following identity for an arbitrary function $f$:
        \begin{gather}
            \label{fluidum:result1}
            \Deriv{}{t}\int_V\rho fdV = \int_V\rho\Deriv{f}{t}dV.
        \end{gather}
    \end{result}

\section{Dynamical systems}

    The following property, although seemingly innocuous, is rather important:
    \begin{property}
        For dynamical systems governed by ODEs satisfying the Picard-Lindel\"of conditions \ref{diffeq:picard_lindelof}, different trajectories never intersect (this follows from the uniqueness part).
    \end{property}

    The above property has an important consequence\footnote{The proof is a bit more involved than that.}
    \begin{theorem}[Poincar\'e-Bendixson]\index{Poincar\'e-Bendixson}
        In a phase plane, i.e. a 2D phase space, the only trajectories inside of a closed bounded subregion without fixed points are either closed orbits or trajectories spiralling into closed orbits.
    \end{theorem}
    \begin{result}
        In 2D (Cartesian) phase spaces there cannot exist chaos, i.e. no strange attractors can exist.
    \end{result}

    \newdef{Lypanuov exponents}{
        Consider two trajectories of a system throughout phase space. Let $s_0 := s(t_0)$ be the distance between these trajectories at a time $t_0$ (we can take this time to be 0 without loss of generality). If after some time $t$ we can write
        \begin{gather}
            s(t) \approx e^{\lambda t}s_0
        \end{gather}
        then we call $\lambda$ the Lyapunov exponent of the system.
    }

    \newdef{Limit cycle}{\index{cycle|seealso{limit}}\index{limit!cycle}
        Consider a closed trajectory $C$. If there exist curves that asymptotically $(t\rightarrow\pm\infty$) converge to $C$, i.e. their \textbf{limit set} is $C$, then we call $C$ a limit cycle.
    }

    \newdef{Poincar\'e map}{\index{Poincar\'e!map}
        Consider a dynamical system determined by a phase flow $\phi$. Let $S$ be a codimension-1 hypersurface in the phase space $Q$ that is transversal to $\phi$, i.e. all trajectories intersect $S$ at isolated points.

        Intuitively, the Poincar\'e map $P:S\rightarrow S$ is defined as the ''first return map'', i.e. for every point $x\in S$ the point $P(x)$, if it exists, is given by $\phi_T(x)$ with $T=:\min\{t\in\mathbb{R}^+:\phi_t(x)\in S\}$.

        One can give a more formal definition (one that also avoids the fact that $P$ would only be partially defined). The Poincar\'e map $P$ is defined as follows:
        \begin{enumerate}
            \item A differentiable map $P:U\rightarrow S$ where $U\subset S$ is open and connected.
            \item $P|_{P(U)}$ is a diffeomorphism.
            \item For every point $u\in U$ the positive semi-orbit of $u$ intersects $S$ for the first time at $P(u)$.
        \end{enumerate}

        The usefulness of this map lies in the fact that it preserves (quasi)periodicity whilst reducing the dimensionality of the space. Its main use lies in the study of 3D spaces where the section $S$ is 2-dimensional and hence easily visualized.
    }
    \begin{property}
        Fixed points of the Poincar\'e map correspond to closed orbits.
    \end{property}

\section{Fluid Mechanics}

    \begin{theorem}[Cauchy's stress theorem\footnotemark]\index{Cauchy!fundamental theorem}
        \footnotetext{Also known as \textbf{Cauchy's fundamental theorem}.}
        Knowing the stress vectors acting on the coordinate planes through a point $A$ is sufficient to calculate the stress vector acting on an arbitrary plane passing through $A$.
    \end{theorem}

    The \textit{Cauchy stress theorem} is equivalent to the existence of the following tensor:
    \newdef{Cauchy stress tensor}{\index{Cauchy!stress tensor}
        The Cauchy stress tensor is a $(0,2)$-tensor $\mathbf{T}$ that gives the relation between a stress vector associated to a plane and the normal vector $\vector{n}$ to that plane:
        \begin{gather}
            \vector{t}_{(\vector{n})} = \mathbf{T}(\vector{n}).
        \end{gather}
    }
    \begin{example}
        For identical particles the stress tensor is given by
        \begin{gather}
            \mathbf{T} = -\rho \langle\vector{w}\otimes\vector{w}\rangle
        \end{gather}
        where $\vector{w}$ is the random component of the velocity vector and $\langle\cdot\rangle$ denotes the expectation value (see \ref{prob:expectation_value}).
    \end{example}

    \begin{theorem}[Cauchy's lemma]\index{Cauchy!lemma}
        The stress vectors acting on opposite planes are equal in magnitude but opposite in direction:
        \begin{gather}
            \vector{t}_{(-\vector{n})} = -\vector{t}_{(\vector{n})}.
        \end{gather}
    \end{theorem}

    \newformula{Cauchy momentum equation}{\index{Cauchy!momentum equation}
        From Newton's second law \ref{forces:force} it follows that
        \begin{gather}
            \Deriv{\vector{P}}{t} = \int_V\vector{f}(\vector{x}, t)dV + \oint_S\vector{t}(\vector{x}, t)dS
        \end{gather}
        where $\vector{P}$ is the momentum density, $\vector{f}$ are body forces and $\vector{t}$ are surface forces (such as \textit{shear stress}). Using Cauchy's stress theorem and the divergence theorem \ref{vectorcalculus:divergence_theorem} we get
        \begin{gather}
            \Deriv{\vector{P}}{t} = \int_V\left[\vector{f}(\vector{x}, t) + \nabla\cdot\mathbf{T}(\vector{x}, t)\right]dV.
        \end{gather}
        The left-hand side can be rewritten using \ref{fluidum:result1} as
        \begin{gather}
            \int_V\rho\Deriv{\vector{v}}{t}dV = \int_V\left[\vector{f}(\vector{x}, t) + \nabla\cdot\mathbf{T}(\vector{x}, t)\right]dV.
        \end{gather}
    }

\section{Geometric description}

    In this section we reformulate the current chapter in a differential geometric framework (for an introduction to differential geometry see chapter \ref{chapter:manifolds} and onwards). This section is based on \cite{palais_solitons}.

    We first begin by reformulating ordinary Newtonian mechanics. The general setting here is a Riemannian manifold\footnote{The metric is mainly for defining the kinetic term $\frac{1}{2}g(v, v)$.} $(M, g)$ together with a second-order ODE in the form of a vector field on $TM$ such that $\pi_*(X_v) = v$ (where $\pi$ denotes the tangent bundle projection). For integral curves $\gamma$ of second-order ODEs it is easy to show that they are the tangent vector fields of their projections. If $q_i(t)$ are the local coordinates of the base curve $\sigma:=\pi(\gamma)$ then it can be shown that the tangent coordinates $\dot{q}^i$ of $\gamma$ are exactly the derivatives of the coordinates $q_i$:
    \begin{gather}
        \dot{q}^i(t) = \deriv{q^i}{t}(t)
    \end{gather}
    and as such the abuse of notation $\dot{q}^i$ is justified. Furthermore, it can be shown that a vector field on $TM$ is second-order if and only if this is true for any local chart, i.e. if the vector field $X\in\mathfrak{X}(TM)$ can be expressed as follows:
    \begin{gather}
        X = \dot{q}^i\pderiv{}{q^i} + F^i(q,\dot{q})\pderiv{}{\dot{q}^i}.
    \end{gather}
    By writing this vector field as a system of differential equations we get the second-order ODE (hence the terminology)
    \begin{gather}
        \mderiv{2}{q^i}{t} = F^i(q,\dot{q}).
    \end{gather}
    The prime example of such a second-order ODE is the vector field generating the geodesic flow on $TM$, i.e. the integral curves are the tangent curves of geodesics on $M$. The similarity between the above equation and equation \ref{diff:geodesic_equation} is therefore striking. By adopting the notation of equation \ref{riemann:geodesic} one can generalize the geodesic equation to obtain Newton's equation for an arbitrary smooth ''potential'' $U:M\rightarrow\mathbb{R}$:
    \begin{formula}[Newton's equation]\index{Newton!equation}
        Let $(M, g)$ be a Riemannian manifold and let $U:M\rightarrow\mathbb{R}$ be a smooth function. Newton's equation for a curve $\sigma:[a,b]\rightarrow M$ reads
        \begin{gather}
            \nabla_{\dot{\sigma}}\dot{\sigma} = -\text{grad}(U)
        \end{gather}
    \end{formula}
    where $\nabla$ indicates the Levi-Civita connection and $\text{grad}$ denotes the gradient operator.

\subsection{Lagrangian formalism}

    We now turn to Noether's theorem and in particular the version concerning cyclic coordinates \ref{lagrange:noether_cyclic}. Any diffeomorphism of $M$ induces a diffeomorphism on $TM$ by pushforward. A symmetry of the Lagrangian function $L:TM\rightarrow\mathbb{R}$ is a diffeomorphism $\phi$ of $M$ such that $\phi^*L = L$. Infinitesimal symmetries (or infinitesimal symmetry generators) are then the vector fields for which the flow is a symmetry. Given a complete vector field $X$, one can define the conjugate momentum $\widehat{X}:TM\rightarrow\mathbb{R}$ as follows:
    \begin{gather}
        \label{lagrange:metric_conjugate_momentum}
        \widehat{X}(v) := g(X_{\pi(v)}, v).
    \end{gather}
    Using this definition we can reformulate Noether's theorem \ref{lagrange:noether_cyclic} as follows:
    \begin{theorem}[Noether's theorem]\index{Noether}
        The conjugate momentum of an infinitesimal symmetry is a constant of motion.
    \end{theorem}

    If one denotes the conjugate momenta of the coordinate-induced vector fields $\partial_i$ by $P_i$, the nondegeneracy of $g$ implies that the set $\{q^i, P_i\}_{i\leq n}$ gives well-defined coordinate functions on $T^*M$. The equivalence of the Lagrangian action principle and the Newtonian equations of motion imply that the second-order ODE associated to the potential $U$ takes the following form:
    \begin{gather}
        X^U := \dot{q}^i\pderiv{}{q^i} + \pderiv{L}{q^i}\pderiv{}{P_i}.
    \end{gather}
    After performing the Legendre transformation $E:=P_i\dot{q}^i-L$ to obtain the (Hamiltonian) energy function\footnote{We refrain from calling the Hamiltonian function, as we reserve this terminology for objects on the cotangent bundle.}, we can rewrite Newton's equations in the Hamiltonian form:
    \begin{gather}
        X^E = \pderiv{E}{P_i}\pderiv{}{q^i}-\pderiv{E}{q^i}\pderiv{}{P_i}.
    \end{gather}

\subsection{Hamiltonian formalism}

    The procedure of mapping a (complete) vector field to its conjugate momentum can be generalized to an isomorphism $TM\rightarrow T^*M$ as follows:
    \newdef{Fibre derivative}{\index{fibre!derivative}
        Let $L:TM\rightarrow\mathbb{R}$ be a smooth Lagrangian. The fiber derivative of $L$ is defined as the Fr\'echet derivative
        \begin{gather}
            \langle FL(v), w\rangle := \left.\deriv{}{t}\right|_{t=0}L(v + tw).
        \end{gather}
        Because $FL(v)\in\mathcal{L}(TM, \mathbb{R})\equiv T^*M$ by definition of the derivative, we see that $FL$ defines a map $TM\rightarrow T^*M$. In local coordinates $(q^i, \dot{q}^i)$ the fiber derivative is given by
        \begin{gather}
            FL:(q^i, \dot{q}^i)\mapsto\left(q^i, \pderiv{L}{\dot{q}^i}\right)\equiv(q^i, p_i)
        \end{gather}
        As such we obtain the classical definition \ref{lagrange:conjugate_momentum} for conjugate momenta. In the case of kinetic Lagrangians defined by a metric $g$, it is not hard to see that this boils down to equation \ref{lagrange:metric_conjugate_momentum} of conjugate momenta given in the previous paragraph.
    }
    \begin{remark}[Legendre transform]\index{Legendre!transformation}
        The fibre derivative $FL$ is often called the Legendre transformation of $L$. Although this does not exactly coincide with definitions \ref{info:legendre} or \ref{lagrange:legendre}, the relation is simple enough. The Legendre transformation $L\mapsto E$ (on the tangent bundle) is implemented as $E(X) = \langle FL(X), X\rangle - L(X)$.
    \end{remark}
    Lagrangians for which the Legendre transformation is invertible, i.e. for which $FL$ is a diffeomorphism, give rise to mechanics on the cotangent bundle: Given such a Lagrangian one constructs the associated energy function $E$ by a Legendre transformation and maps it to a Hamiltonian $H$ on the cotangent bundle as $H:=E\circ FL^{-1}$. (By abuse of notation we set $L\equiv L\circ FL^{-1}$.) The transformation also induces a Hamiltonian vector field on $T^*M$ by $X^H:=FL_*X^E$ or alternatively by $X^E=FL^*X^H$. If we choose cotangent coordinates $p_i(\alpha) := \alpha(\partial_i)$, then we easily see that $p_i\circ FL=P_i$. This way the Hamiltonian equations remain virtually unchanged when transporting them to the cotangent bundle.

    For any choice of coordinates such that the symplectic form on $T^*M$ takes the standard Darboux form $\omega=dp_i\wedge dq^i$, the Newtonian equations of motion take on a Hamiltonian form. If there exists a coordinate chart in which the Hamiltonian function $H$ does not depend on any of the base coordinates $q^i$, then we call the coordinates \textbf{action-angle} variables and the system is said to be \textbf{completely integrable}.\index{action-angle coordinates}

\subsection{Symplectic structure on infinite-dimensional systems}

    Although we could have put this section in chapter \ref{chapter:symplectic}, we thought it better fit here, since the study of these manifolds is almost always related to the study of physical phenomena such as \textit{solitons}.

    The general definition of a symplectic manifold $(M, \omega)$ remains the same, i.e. it is a smooth manifold $M$ equipped with a closed, nondegenerate $2$-form $\omega$. Even though we remain in the 2-plectic setting, the content of remark \ref{symplectic:hamiltonian_forms} applies also to infinite-dimensional systems. If we restrict to the space of Hamiltonian functions (and vector fields) we can define a Poisson structure as follows:
    \begin{gather}
        \{F, G\} := \omega(X^G, X^F).
    \end{gather}

    ?? COMPLETE (e.g. Palais, cursus Antwerpen)??
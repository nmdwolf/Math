\chapter{Lagrangian and Hamiltonian Mechanics}\label{chapter:lagrange}

\section{Action functional}

	\begin{definition}[Generalized coordinates]\index{generalized!coordinates}
		The generalized coordinates $q_k$ are independent coordinates that completely describe the current configuration of a system relative to a reference configuration.
		
		When a system has $N$ degrees of freedom and $n_c$ constraints, there are $(N - n_c)$ generalized coordinates. Furthermore, every set of generalized coordinates, describing the same system, should contain exactly $(N - n_c)$ coordinates.
	\end{definition}
    \begin{definition}[Generalized velocities]
		The generalized velocities $\dot{q}_k$ are the derivatives of the generalized coordinates with respect to time.
	\end{definition}
    \begin{notation}
		\begin{gather}
			\label{lagrange:notational_convention_1}
            L\left(\vector{q}(t), \dot{\vector{q}}(t), t\right) \equiv L\left(q_1(t), ..., q_n(t), \dot{q}_1(t), ..., \dot{q}_n(t), t\right)
		\end{gather}
	\end{notation}
    
	\begin{definition}[Action]\index{action}
		\begin{gather}
			\label{lagrange:action}
            \boxed{S = \int_{t_1}^{t_2}L\left(\vector{q}(t), \dot{\vector{q}}(t), t\right)dt}
		\end{gather}
	\end{definition}

\section{Euler-Lagrange equations\texorpdfstring{$^\dag$}\ }\index{Lagrange!equations of motion}\index{Euler-Lagrange|see{Lagrange}}
	\begin{formula}[Euler-Lagrange equation of the first kind]
    	\begin{gather}
        	\label{lagrange:first_kind}
			\boxed{\deriv{}{t}\left(\pderiv{T}{\dot{q}^k}\right) - \pderiv{T}{q^k} = Q_k}
		\end{gather}
		where $T$ is the total kinetic energy.
    \end{formula}
	\begin{formula}[Euler-Lagrange equation of the second kind]
    	\begin{gather}
        	\label{lagrange:second_kind}
            \boxed{\deriv{}{t}\left(\pderiv{L}{\dot{q}^k}\right) - \pderiv{L}{q^k} = 0}
        \end{gather}
    \end{formula}

\section{Conservation laws and symmetry properties}

	\begin{definition}[Conjugate momentum]\index{momentum!conjugate}
		Also called the \textbf{canonically conjugate momentum}.
	    	\begin{gather}
			\label{lagrange:conjugate_momentum}
			p_k = \pderiv{L}{\dot{q}^k}
		\end{gather}
	\end{definition}
	\begin{definition}[Cyclic coordinate]
	    	If the lagrangian $L$ does not explicitly depend on a coordinate $q_k$, the coordinate is called a cyclic coordinate.
	\end{definition}
    
	\begin{property}
		The conjugate momentum of a cyclic coordinate is a conserved quantity.
		\begin{gather}
			\dot{p}_k \overset{\ref{lagrange:conjugate_momentum}}{=} \deriv{}{t}\left(\pderiv{L}{\dot{q}^k}\right) \overset{{\ref{lagrange:second_kind}}}{=}\pderiv{L}{q^k} \overset{\underset{\text{\tiny{coord.}}}{\text{\tiny{cyclic}}}}{=} 0
		\end{gather}
	\end{property}

\subsection{Noether's theorem}\index{Noether!theorem}

	\begin{theorem}[Noether's theorem$^\dag$]\label{qft:noethers_theorem}\index{Noether!charge}
		Consider a field transformation
		\begin{gather}
			\label{qft:noether}
			\phi(x)\rightarrow \phi(x) + \alpha\delta\phi(x)
		\end{gather}
		where $\alpha$ is an infinitesimal quantity and $\delta\phi$ is a small deformation. In case of a symmetry we obtain the following conservation law:
		 \begin{gather}
		 	\label{qft:conserved_current}
		 	\partial_\mu\left(\pderiv{\mathcal{L}}{(\partial_\mu\phi)}\delta\phi - \mathcal{J}^\mu\right) = 0
		 \end{gather}
		 The factor between parentheses can be interpreted as a conserved current $j^\mu(x)$. Noether's theorem states that every symmetry of the form \ref{qft:noether} leads to such a current.
		 
		 The conservation can also be expressed in terms of a charge\footnote{The conserved current and its associated charge are called the \textbf{Noether current} and \textbf{Noether charge}.}:
		 \begin{gather}
		 	\deriv{Q}{t} = \deriv{}{t}\int j^0d^3x = 0
		 \end{gather}
	\end{theorem}

	\begin{definition}[Stress-energy tensor]\index{stress-energy tensor}
		Consider a field transformation
		\[
			\phi(x)\rightarrow\phi(x+a) = \phi(x) + a^\mu\partial_\mu\phi(x)
		\]
		Because the Lagrangian is a scalar it transforms similarly:
		\begin{gather}
			\mathcal{L}\rightarrow\mathcal{L} + a^\mu\partial_\mu\mathcal{L} = \mathcal{L} + a^\nu\partial_\mu(\delta^\mu_{\ \nu}\mathcal{L})
		\end{gather}
		This leads to the existence of 4 conserved currents. These can be used to define the stress-energy tensor:
		\begin{gather}
			\label{relativity:stress_energy_tensor}
			\boxed{T^\mu_{\ \nu} = \pderiv{\mathcal{L}}{(\partial_\mu\phi)}\partial_\nu\phi - \mathcal{L}\delta^\mu_{\ \nu}}
		\end{gather}
	\end{definition}

\section{Hamilton's equations}

	\newdef{Canonical coordinates}{\index{canonical!coordinates}
		Consider the generalized coordinates $(q, \dot{p}, t)$ from the Lagrangian formalism. Using these we can define a new set of coordinates, called canonical coordinates, by exchanging the time-derivatives $\dot{q}^i$ in favour of the conjugate momenta $p_i$ (see definition \ref{lagrange:conjugate_momentum}) and leaving the coordinates $q^i$ and $t$ invariant.
	}

	\newdef{Hamiltonian function}{\index{Hamilton!Hamiltonian function}
		The (classical) Hamiltonian function is defined as follows:
		\begin{gather}
    			\label{hamilton:hamiltonian}
			H(q, p, t) = \sum_ip_i\dot{q}^i - L(q, p, t)
		\end{gather}
	}

	\newformula{Hamilton's equations\footnotemark}{\index{Hamilton!equations of motion}\footnotetext{Also known as the \textit{canonical equations of Hamilton}.}
    		\begin{empheq}[box=\widefbox]{align}
	        	\label{lagrange:hamilton_equations}
    			\dot{q}^i &= \pderiv{H}{p_i}\\
    			\dot{p_i} &= -\pderiv{H}{q^i}\\
    			\pderiv{L}{t} &= -\pderiv{H}{t}
	    	\end{empheq}
	}

	The formula to obtain the Hamiltonian from the Lagrangian is an application of the following more general \textbf{Legendre transformation}:
	\newdef{Legendre transformation}{\index{Legendre!transformation}
		Consider an equation of the following form:
		\begin{gather}
			\label{lagrange:legendre1}
			df = udx + vdy
		\end{gather}
		where $u = \pderiv{f}{x}$ and $v = \pderiv{f}{y}$.
		
		Suppose we want to perform a coordinate transformation $(x, y)\rightarrow (u, y)$ while preserving the general form of \ref{lagrange:legendre1} for differential quantities. To do this we consider the function
		\begin{gather}
			\label{lagrange:legendre}
			g = f - ux
		\end{gather}
		Differentiating gives
		\begin{align}
			dg &= df - udx - xdu\nonumber\\
			&= (udx + vdy) - udx - xdu\nonumber\\
			&= vdy - xdu\nonumber
		\end{align}
		which has the form of \ref{lagrange:legendre1} as desired. The quantities $v$ and $x$ are now given by
		\begin{gather}
			x = -\pderiv{g}{u}\qquad\text{and}\qquad v=\pderiv{g}{y}
		\end{gather}
		
		The transition $f\longrightarrow g$ defined by equations \ref{lagrange:legendre1} and \ref{lagrange:legendre} is called a Legendre transformation.
	}
	\begin{remark}
		Although the previous derivation used only 2 coordinates, the definition of Legendre transformations can easily be generalized to more coordinates.
	\end{remark}

\subsection{Poisson brackets}

	\newdef{Poisson bracket}{\index{Poisson!bracket}
		\nomenclature[O_sympoi]{$\{\cdot, \cdot\}$}{Poisson bracket}
		\begin{gather}
			\boxed{\{A, B\} = \pderiv{A}{p}\pderiv{B}{q} - \pderiv{B}{p}\pderiv{A}{q}}
		\end{gather}
		where $q, p$ are the generalized coordinates in the Hamiltonian formalism.
	}
	\newformula{Total time derivative}{
		\begin{gather}
			\deriv{F}{t} = \pderiv{F}{t} + \{H, F\}
		\end{gather}
		where $\{\cdot,\cdot\}$ is the Poisson bracket as defined above and $H$ is the Hamiltonian \ref{hamilton:hamiltonian}.
	}

\section{Hamilton-Jacobi equation}

	For a mathematical introduction see section \ref{section:hamilton_jacobi}.

\subsection{Canonical transformations}

	\newdef{Canonical transformations}{\index{canonical!transformation}
		A canonical transformation is a transformation that leaves the Hamiltonian equations of motion unchanged. Mathematically this means that the transformations leave the action invariant up to a constant, or equivalently, they leave the Lagrangian invariant up to a complete time-derivative:
		\begin{gather}
			\sum_i \dot{q}^ip_i - H(q, p, t) = \sum_i \dot{Q}^iP_i - K(Q, P, t) - \deriv{G}{t}(Q, P, t)
		\end{gather}
		The function $G$ is called the generating function of the canonical transformation. The choice of $G$ uniquely determines the transformation.
	}
    
	\newformula{Hamilton-Jacobi equation}{\index{Hamilton-Jacobi}\index{Hamilton!principal function}
		Sufficient conditions for the generating function $S$ are given by:
		\begin{align}
			P_i &=\ds\pderiv{S}{Q^i}\nonumber\\
			Q^i &=\ds\pderiv{S}{P_i}\nonumber
		\end{align}
		and \[K = H + \pderiv{S}{t}\]
		Choosing the new Hamiltonian function $K$ to be 0 gives the Hamilton-Jacobi equation:
		\begin{gather}
			\label{lagrange:hamilton_jacobi_equation}
			\boxed{H\left(q, \pderiv{S}{q}\right)+\pderiv{S}{t} = 0}
		\end{gather}
		The function $S$ is called \textbf{Hamilton's principal function}.
	}
	
	\begin{property}
		The new coordinates $P_i$ and $Q^i$ are all constants of motion. This follows immediately from the choice $K = 0$.
	\end{property}
    
	\newdef{Hamilton's characteristic function}{\index{Hamilton!characteristic function}\index{energy}
		For time-independent systems the HJE can be rewritten as follows:\footnote{Note that this form is the same as equation \ref{diff:hamilton_jacobi} after a redefinition of $H$.}
		\begin{gather}
			\label{lagrange:time_independent_hje}
			H\left(q, \pderiv{S}{q}\right) = -\pderiv{S}{t} := E.
		\end{gather}
		We thus obtain the classical result that for time-independent systems the Hamiltonian function is a constant of motion (the \textbf{energy}). Integration with respect to time then gives the following form of the principal function:
		\begin{gather}
			S(q, p, t) = W(q, p) - Et
		\end{gather}
		where $E$ is the energy. The time-independent function $W$ is called Hamilton's characteristic function.
	}

\subsection{St\"ackel potentials}
	
	\begin{remark}
		If the principal function can be separated into $n$ equations, the HJE splits up into $n$ equations of the form
		\begin{gather}
			h_i\left(q^i, \deriv{S}{q^i}, \alpha_i\right) = 0
		\end{gather}
		The partial differential equation for $S$ can thus be rewritten as a system of $n$ ordinary differential equations.
	\end{remark}

	\begin{property}[St\"ackel condition]\index{St\"ackel potential}
		The Hamilton-Jacobi equation is separable if and only if the potential is of the following form:
		\begin{gather}
			\label{lagrange:stackel_condition}
			V(q) = \sum_{i=1}^n\ds\frac{1}{G_i^2(q)}W_i(q^i)
		\end{gather}
		whenever the Hamiltonian function can be written as
		\begin{gather}
			H(q, p) = \frac{1}{2}\sum_i\stylefrac{p_i^2}{G^2_i(q)} + V(q)
		\end{gather}
		Potentials of this form are called \textbf{St\"ackel potentials}.
	\end{property}

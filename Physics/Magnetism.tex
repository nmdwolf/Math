\chapter{Magnetism}

\section{Magnetic field}
\subsection{Fields}\index{magnetization}\index{induction}
        
        The \textbf{magnetizing field} $\vector{H}$ is the field resulting from all external sources. When applying an external (magnetic) field, some materials will try to oppose this external influence. Similar to polarization in electricity one can measure the \textbf{magnetization}:
	\begin{equation}
		\label{magnetism:M}
		\vector{M} = \chi\vector{H}
	\end{equation}
	where $\chi$ is the magnetic susceptibility.

        The \textbf{magnetic induction} $\vector{B}$ is the field resulting from external sources and internal magnetization. (It is the 'real', detectable field.) In vacuum we have the following relation between the magnetic induction, the magnetizing field and the magnetization:
        \begin{equation}
            \label{magnetism:B}
            \vector{B} = \mu_0\left(\vector{H} + \vector{M}\right)
        \end{equation}
        By combining this formula with formula \ref{magnetism:M} we get\footnote{This equation is only valid in linear media.}:
        \begin{equation}
            \label{magnetism:B_with_only_H}
            \vector{B} = \mu_0\left(1 + \chi\right)\vector{H}
        \end{equation}

	The proportionality constant in formula \ref{magnetism:B_with_only_H} is called the \textbf{magnetic permeability}:
	\begin{equation}
		\label{magnetism:relative_permeability}
		\mu = \mu_0(1 + \chi)
	\end{equation}
	where $\mu_0$ is the magnetic permeability of the vacuum. The factor $1+\chi$ is called the \textbf{relative permeability} and it is often denoted by $\mu_r$.
    
    \subsection{Tensorial formulation}
    	In anistropic materials we have to use the tensorial formulation.
    	\begin{equation}
    		\label{magnetism:B_tensor}
        	B_i = \sum_j\mu_{ij}H_j
		\end{equation}
        \begin{equation}
    		\label{magnetism:M_tensor}
        	M_i = \sum_j\chi_{ij}H_j
		\end{equation}
        Both $\mu$ and $\chi$ are tensors of rank 2.
        
\section{Magnetic multipoles}
	\subsection{Dipole}
        \begin{equation}
            \label{magnetism:dipole}
            \vector{m} = IS\vector{u}_n
        \end{equation}
        
\section{Electric charges in a magnetic field}
	\subsection{Cyclotron}
    	\newformula{Gyroradius}{
        	\begin{equation}
            	\label{magnetism:gyroradius}
                r = \stylefrac{mv_{\perp}}{|q|B}
            \end{equation}
		}
        \newformula{Gyrofrequency\footnotemark}{
        	\begin{equation}
            	\label{magnetism:gyrofrequency}
                \omega = \stylefrac{|q|B}{m}
            \end{equation}
		}
        \footnotetext{Also called the Larmor frequency.}

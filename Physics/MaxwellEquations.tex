\chapter{Maxwell Equations}
\section{Lorentz force}

	\begin{formula}[Lorentz force]
	        \begin{equation}
		        \label{maxwell:lorentz_force}
		        \vector{F} = q\left(\vector{E} + \vector{v}\times\vector{B}\right)
        	\end{equation}
	\end{formula}
    
	\begin{formula}[Lorentz force density]
		\begin{equation}
		    	\label{maxwell:lorentz_force_density}
		        \vector{f} = \rho\vector{E} + \vector{J}\times\vector{B}
		\end{equation}
	\end{formula}
    
\section{Differential Maxwell equations}

	\newformula{Gauss' law for electricity}{
	    	\begin{equation}
			\label{maxwell:gauss_electricity}
		        \nabla\cdot\vector{E} = \stylefrac{\rho}{\varepsilon}
		\end{equation}
	}
	\newformula{Gauss' law for magnetism}{
    		\begin{equation}
			\label{maxwell:gauss_magnetism}
        		\nabla\cdot\vector{B} = 0
		\end{equation}
	}
	\newformula{Faraday's law}{
	    	\begin{equation}
			\label{maxwell:faraday}
		        \nabla\times\vector{E} = -\pderiv{\vector{B}}{t}
		\end{equation}
	}
	\newformula{Maxwell's law\footnotemark}{
		\footnotetext{Also called the law of Maxwell-Amp\`ere.}
	    	\begin{equation}
			\label{maxwell:maxwell}
		        \nabla\times\vector{B} = \varepsilon\mu\pderiv{\vector{E}}{t} + \mu\vector{J}
		\end{equation}
	}
    
\section{Potentials}
\subsection{Decomposition in potentials}

        Following the Helmholtz decomposition \ref{vectorcalculus:helmholtz_decomposition} we can derive the following general form for $\vector{B}$ starting from Gauss' law \ref{maxwell:gauss_magnetism}:
        \begin{equation}
	        \label{maxwell:magnetic_potential}
	        \boxed{\vector{B} = \nabla\times\vector{A}}
        \end{equation}
        where $\vector{A}$ is the magnetic potential.

        Combining equation \ref{maxwell:magnetic_potential} with Faraday's law \ref{maxwell:faraday} and rewriting it a bit gives the following general form for $\vector{E}$:
        \begin{equation}
	        \label{maxwell:electric_potential}
	        \boxed{\vector{E} = -\nabla V - \pderiv{\vector{A}}{t}}
        \end{equation}
        where $V$ is the electrostatic potential.
        
	\begin{property}
    		Substituting the expressions \ref{maxwell:magnetic_potential} and \ref{maxwell:electric_potential} into Gauss' law \ref{maxwell:gauss_electricity} and Maxwell's law \ref{maxwell:maxwell} gives the following two (coupled) conditions for the electromagnetic potentials:
        	\begin{align}
        		\label{maxwell:potential_conditions_A}
			&\lap\vector{A} - \varepsilon\mu\mpderiv{2}{\vector{A}}{t} = \nabla\left(\nabla\cdot\vector{A} + \varepsilon\mu\pderiv{V}{t}\right) - \mu\vector{J}\\
        		\label{maxwell:potential_conditions_V}
		        &\lap V - \varepsilon\mu\mpderiv{2}{V}{t} = -\pderiv{}{t}\left(\nabla\cdot\vector{A} + \varepsilon\mu\pderiv{V}{t}\right) - \stylefrac{\rho}{\varepsilon}
		\end{align}
	\end{property}

\subsection{Gauge transformations}\index{gauge!transformation}

    	Looking at equation \ref{maxwell:magnetic_potential}, it is clear that a transformation $\vector{A}\longrightarrow\vector{A} + \nabla\psi$ has no effect on $\vector{B}$ due to property \ref{vectorcalculus:rotor_of_gradient}. To compensate this in equation \ref{maxwell:electric_potential}, we also have to perform a transformation $V\longrightarrow V - \pderiv{\psi}{t}$.
        
        The (scalar) function $\psi(\vector{r}, t)$ is called a \textbf{gauge function}. The transformations are called \textbf{gauge transformations}.
        \newdef{Gauge fixing conditions}{
	        Conditions that fix a certain gauge (or class of gauge transformations) are called gauge fixing conditions. These select one of many physically equivalent configurations.
        }

	\begin{example}[Lorenz gauge]\index{Lorenz!gauge}
                A first example of a gauge fixing condition is the Lorenz gauge\footnote{Named after Ludvig Lorenz. Not to be confused with Hendrik Lorentz.}:
        	\begin{equation}
        		\label{maxwell:lorenz_gauge}
		        \boxed{\nabla\cdot\vector{A} + \varepsilon\mu\pderiv{V}{t} = 0}
		\end{equation}
        	When using this gauge fixing condition, equations \ref{maxwell:potential_conditions_A} and \ref{maxwell:potential_conditions_V} become uncoupled and can be rewritten as:
        	\begin{align}
			&\Box\vector{A} = -\mu\vector{J}\\
        		&\Box V = -\stylefrac{\rho}{\varepsilon}
		\end{align}
        	To see which gauge functions $\psi$ are valid in this case we perform a transformation as explained above:
        	\begin{equation}
        		\label{maxwell:gauge_transformations}
        		\vector{A}' = \vector{A} + \nabla\psi\qquad\text{ and }\qquad V'=V-\pderiv{\psi}{t}
        	\end{equation}
        	Substituting these transformations in equation \ref{maxwell:lorenz_gauge} and using the fact that both sets of potentials $(\vector{A}, V)$ and $(\vector{A}', V')$ satisfiy the Lorenz gauge \ref{maxwell:lorenz_gauge} gives the following condition for the gauge function $\psi$:
        	\begin{equation}
			\label{maxwell:lorenz_gauge_condition}
        		\Box\psi = 0
		\end{equation}
	\end{example}
        
	\begin{example}[Coulomb gauge]\index{Coulomb!gauge}
        	Apart from the Lorenz gauge \ref{maxwell:lorenz_gauge}, there is also the Coulomb gauge:
        	\begin{equation}\index{Coulomb!gauge}
			\label{maxwell:coulomb_gauge}
        		\nabla\cdot\vector{A} = 0
		\end{equation}
	\end{example}
        
\section{Energy and momentum}

	\newdef{Poynting vector}{\index{Poynting vector}
	    	\begin{equation}
			\label{maxwell:poynting_vector}
			\boxed{\vector{S} = \vector{E}\times\vector{H}}
		\end{equation}
	}
    
	\newdef{Energy density}{
	    	\begin{equation}
			\label{maxwell:energy_density}
		        \boxed{W = \stylefrac{1}{2}\left(\vector{E}\cdot\vector{D} + \vector{B}\cdot\vector{H}\right)} 
		\end{equation}
	}
    
\section{Differential geometric perspective}

	Using the tools given (e.g. differential forms) in chapter \ref{diff:chapter:vector_bundles} we can rewrite all of the above formulas in a more elegant form, which will also allow us to generalize them to higher dimensions and more general settings. See for example \cite{principal_bundles} for a complete derivation and interpretation. It should be noted that we used \textit{Gaussian units} throughout this section.
	
	\newdef{Field strength}{\index{field!strength}
		Let \[\mathbf{E} = E_1dx^1 + E_2dx^2 + E_3dx^3\] and \[\mathbf{B} = B_1dx^2\wedge dx^3 + B_2dx^3\wedge dx^1 + B_3dx^1\wedge dx^2\] be the electric and magnetic field forms respectively. Using these forms we can define the field strength as follows:
		\begin{equation}
			\mathbf{F} = \mathbf{B} - dt \wedge \mathbf{E}
		\end{equation}
	}
	
	\begin{formula}[Maxwell's equations]
		Denote the electric 4-current by \[\mathbf{J} = \rho dt - J_1dx^1 - J_2dx^2 - J_3dx^3\] Maxwell's equations can now be rewritten as follows:
		\begin{align}
			d\mathbf{F} &= 0\label{maxwell:diff_homogeneous}\\
			\ast d(\ast \mathbf{F}) &= 4\pi\mathbf{J}
		\end{align}
		where $\ast$ is the Hodge operator \ref{tensor:explicit_hodge_star}.
	\end{formula}
	
	\newdef{Potential}{\index{potential}\index{gauge!field}
		The homogeneous equation \ref{maxwell:diff_homogeneous} together with Poincar\'e's lemma\footnote{See theorem \ref{forms:theorem:poincare}.} implies that there exists a differential 1-form $\mathbf{A}$ such that
		\begin{equation}
			\mathbf{F} = d\mathbf{A}
		\end{equation}
		This 1-form is called the potential or \textbf{gauge field}. This field can be related to the ordinary scalar potential $V$ and vectorial potential $\vector{A}$ as follows:
		\begin{equation}
			\mathbf{A} = -Vdt + A_1dx^1 + A_2dx^2 + A_3dx^3
		\end{equation}
	}
	\begin{property}[Gauge transformation]\index{gauge!transformation}
		Because $d^2 = 0$ the above equation is invariant under a transformation $\mathbf{A}\longrightarrow\mathbf{A}+df$ for any $f\in C^\infty$. This gives us exactly the gauge transformations from equation \ref{maxwell:gauge_transformations} when written out in coordinates.
	\end{property}

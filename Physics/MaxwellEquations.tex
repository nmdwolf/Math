\chapter{Maxwell equations}
\section{Lorentz force}
	\begin{formula}[Lorentz force]
        \begin{equation}
            \label{maxwell:lorentz_force}
            \vector{F} = q\left(\vector{E} + \vector{v}\times\vector{B}\right)
        \end{equation}
    \end{formula}
    
    \begin{formula}[Lorentz force density]
    \begin{equation}
    	\label{maxwell:lorentz_force_density}
        \vector{f} = \rho\vector{E} + \vector{J}\times\vector{B}
    \end{equation}
    \end{formula}
    
\section{Differential Maxwell equations}
	\newformula{Gauss' law for electricity}{
    	\begin{equation}
			\label{maxwell:gauss_electricity}
            \nabla\cdot\vector{E} = \stylefrac{\rho}{\varepsilon}
		\end{equation}
    }
    \newformula{Gauss' law for magnetism}{
    	\begin{equation}
			\label{maxwell:gauss_magnetism}
            \nabla\cdot\vector{B} = 0
		\end{equation}
    }
    \newformula{Faraday's law}{
    	\begin{equation}
			\label{maxwell:faraday}
            \nabla\times\vector{E} = -\pderiv{\vector{B}}{t}
		\end{equation}
    }
    \newformula{Maxwell's law\footnotemark}{
    	\begin{equation}
			\label{maxwell:maxwell}
            \nabla\times\vector{H} = \pderiv{\vector{D}}{t} + \vector{J}
		\end{equation}
    }
    \footnotetext{Also called the law of Maxwell-Amp\`ere.}
    
\section{Potentials}
	\subsection{Decomposition in potentials}
        Remembering the Helmholtz decomposition (equation \ref{vectorcalculus:helmholtz_decomposition}) we can derive the following general form for $\vector{B}$ starting from Gauss' law \ref{maxwell:gauss_magnetism}:

        \begin{equation}
            \label{maxwell:magnetic_potential}
            \boxed{\vector{B} = \nabla\times\vector{A}}
        \end{equation}
        where $\vector{A}$ is the magnetic potential.

        Combining equation \ref{maxwell:magnetic_potential} with Faraday's law \ref{maxwell:faraday} and rewriting it a bit, gives the following general form for $\vector{E}$:
        \begin{equation}
            \label{maxwell:electric_potential}
            \boxed{\vector{E} = -\nabla V - \pderiv{\vector{A}}{t}}
        \end{equation}
        where $V$ is the electrostatic potential.
        
	\subsection{Conditions}
    	Substituting the expressions \ref{maxwell:magnetic_potential} and \ref{maxwell:electric_potential} into Gauss' law \ref{maxwell:gauss_electricity} and Maxwell's law \ref{maxwell:maxwell} gives the following two (coupled) conditions for the electromagnetic potentials:
        \begin{align}
        	\label{maxwell:potential_conditions_A}
			&\lap\vector{A} - \varepsilon\mu\mpderiv{2}{\vector{A}}{t} = \nabla\left(\nabla\cdot\vector{A} + \varepsilon\mu\pderiv{V}{t}\right) - \mu\vector{J}\\
            \label{maxwell:potential_conditions_V}
            &\lap V - \varepsilon\mu\mpderiv{2}{V}{t} = -\pderiv{}{t}\left(\nabla\cdot\vector{A} + \varepsilon\mu\pderiv{V}{t}\right) - \stylefrac{\rho}{\varepsilon}
		\end{align}

	\subsection{Gauge transformations}\index{gauge}
    	Looking at equation \ref{maxwell:magnetic_potential}, it is clear that a transformation $\vector{A}\rightarrow\vector{A} + \nabla\psi$ has no effect on $\vector{B}$ due to property \ref{vectorcalculus:rotor_of_gradient}. To compensate this in equation \ref{maxwell:electric_potential}, we also have to perform the transformation $V\rightarrow V - \pderiv{\psi}{t}$.
        
        The (scalar) function $\psi(\vector{r}, t)$ is called a \textbf{gauge function}. The transformations are called \textbf{gauge transformations}.

        \newdef{Gauge fixing conditions}{
            Conditions that fix a certain gauge (or class of gauge transformations) are called gauge fixing conditions. These select one of many physically equivalent configurations.
        }

	\subsection{Lorenz gauge}\index{Lorenz!gauge}
        \noindent A first example of a gauge fixing condition is the Lorenz gauge\footnotemark\ :
        \begin{equation}
			\label{maxwell:lorenz_gauge}
            \boxed{\nabla\cdot\vector{A} + \varepsilon\mu\pderiv{V}{t} = 0}
		\end{equation}
        \footnotetext{Named after Ludvig Lorenz. Not to be confused with Hendrik Lorentz.}
        
        \noindent When using this gauge fixing condition, equations \ref{maxwell:potential_conditions_A} and \ref{maxwell:potential_conditions_V} become uncoupled and can be rewritten as:
        \begin{align}
			&\Box\vector{A} = -\mu\vector{J}\\
            &\Box V = -\stylefrac{\rho}{\varepsilon}
		\end{align}
        
        \noindent To see which gauge functions $\psi$ are valid in this case we perform a transformation as explained above:\[\vector{A}' = \vector{A} + \nabla\psi\qquad\text{ and }\qquad V'=V-\pderiv{\psi}{t}\]
        Substituting these transformations in equation \ref{maxwell:lorenz_gauge} and using the fact that both sets of potentials $(\vector{A}, V)$ and $(\vector{A}', V)$ satisfiy the Lorenz gauge \ref{maxwell:lorenz_gauge} gives the following condition for the gauge function $\psi$:
        \begin{equation}
			\label{maxwell:lorenz_gauge_condition}
            \Box\psi = 0
		\end{equation}
        
	\begin{example}[Alternative gauges]
        Apart from the Lorenz gauge \ref{maxwell:lorenz_gauge}, there is also the Coulomb gauge:
        \begin{equation}\index{Coulomb!gauge}
			\label{maxwell:coulomb_gauge}
            \nabla\cdot\vector{A} = 0
		\end{equation}
	\end{example}
        
\section{Energy and momentum}
	\newdef{Poynting vector}{\index{Poynting vector}
    	\begin{equation}
			\label{maxwell:poynting_vector}
            \boxed{\vector{S} = \vector{E}\times\vector{H}}
		\end{equation}
    }
    
    \newdef{Energy density}{
    	\begin{equation}
			\label{maxwell:energy_density}
            \boxed{W = \stylefrac{1}{2}\left(\vector{E}\cdot\vector{D} + \vector{B}\cdot\vector{H}\right)} 
		\end{equation}
    }
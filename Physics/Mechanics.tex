\chapter{Equations of Motion}\label{chapter:EOM}

\section{General quantities}
\subsection{Linear quantities}

	\begin{formula}[Force]\index{force}
		\begin{equation}
			\label{forces:force}
        		\boxed{\vector{F} = \deriv{\vector{p}}{t}}
		\end{equation}
	\end{formula}
	\sremark{In classical mechanics, this formula is the content of Newton's second law (and is as such an axiom).}

	\begin{formula}[Work]\index{work}
		\begin{equation}
			\label{forces:work}
            		W = \int\vector{F}\cdot d\vector{l}
		\end{equation}
	\end{formula}
	\begin{definition}[Conservative force]\index{conservative}
	    	If the work done by a force is independent of the path taken, the force is said to be \textbf{conservative}.
        	\begin{equation}
			\label{forces:conservative_force_2}
        		\oint_C\vector{F}\cdot d\vector{l}=0
		\end{equation}
	        Stokes' theorem \ref{vectorcalculus:stokes_theorem} together with relation \ref{vectorcalculus:rotor_of_gradient} lets us rewrite the conservative force as the gradient of a scalar field:
		\begin{equation}
			\label{forces:conservative_force}
        		\boxed{\vector{F} = -\nabla V}
		\end{equation}
	\end{definition}
    
	\begin{formula}[Kinetic energy]\index{energy}
		For a free particle with momentum $\vector{p}$ the kinetic energy is given by the following formula:
		\begin{equation}
			\label{forces:kinetic_energy}
			E_{kin} = \stylefrac{p^2}{2m}
		\end{equation}
	\end{formula}

\subsection{Angular quantities}

	In this section $r$ always denotes the the distance from the object's center of mass to the axis around which the object rotates.
	
	\newformula{Angular velocity}{
	    	\begin{equation}
			\label{forces:angular_velocity}
		        \omega = \stylefrac{v}{r}
		\end{equation}
	}
	\newformula{Angular frequency}{
	    	\begin{equation}
			\label{forces:frequency}
		        \nu = \stylefrac{\omega}{2\pi}
		\end{equation}
	}
	
	\newformula{Moment of inertia}{\index{inertia}
		For a symmetric object the moment of inertia is given by:
    		\begin{equation}
			\label{forces:moment_of_inertia}
        		I = \int_V r^2\rho(r)dV
		\end{equation}
		For a general body we can define the moment of inertia tensor:
		\begin{equation}
			\label{forces:inertia_tensor}
			\boxed{\mathcal{I} = \int_V\rho(\vector{r})\left(r^2\mathbbm{1} - \vector{r}\otimes\vector{r}\right)dV}
		\end{equation}
	}
    
	\newdef{Principal axes of inertia}{\index{principal!axis}
		Let $I$ be the matrix of inertia, i.e. the matrix associated with the inertia tensor \ref{forces:inertia_tensor}. This is a real symmetric matrix and as such admits an eigendecomposition\footnote{See property \ref{linalgebra:diagonalizable_hermitian}.} of the form:
		\begin{equation}
			I = Q\Lambda Q^T
		\end{equation}
		The columns of $Q$ determine the principal axes of inertia. The eigenvalues are called the \textbf{principal moments of inertia}.
	}

	\begin{example}[Objects with azimuthal symmetry$^\dag$]
		Let $r$ denote the radius of the object.
		\begin{itemize}
			\item Solid disk: $I = \frac{1}{2}mr^2$
			\item Cylindrical shell: $I = mr^2$
			\item Hollow sphere: $I = \frac{2}{3}mr^2$
			\item Solid sphere: $I = \frac{2}{5}mr^2$
		\end{itemize}
	\end{example}

	\begin{theorem}[Parallel axis theorem\footnotemark]\index{Steiner}
		\footnotetext{Also called \textbf{Steiner's theorem}.}
		Consider a rotation about an axis $\psi$ through a point $A$. Let $\psi_{CM}$ be a parallel axis through the center of mass. The moment of inertia about $\psi$ is related to the moment of inertia about $\psi_{CM}$ in the following way:
		\begin{equation}
			\label{forces:theorem:parallel_axis_theorem}
			\boxed{I_A = I_{CM} + M||\vector{r}_A - \vector{r}_{CM}||^2}
		\end{equation}
		where $M$ is the mass of the rotating body.
	\end{theorem}

	\begin{formula}[Angular momentum]
		\begin{equation}
			\label{forces:angular_momentum}
			\boxed{\vector{L} = \vector{r}\times\vector{p}}
		\end{equation}
		Given the angular velocity vector we can compute the angular momentum as follows:
		\begin{equation}
			\label{forces:angular_momentum_general}
			\vector{L} = \mathcal{I}(\vector{\omega})
		\end{equation}
		where $\mathcal{I}$ is the moment of inertia tensor. If $\vector{\omega}$ is parallel to a principal axis, then the formula reduces to:
		\begin{equation}
			\vector{L} = I\vector{\omega}
		\end{equation}
	\end{formula}
    
	\begin{formula}[Torque]\index{torque}
		For angular momenta there exists a formula analogous to Newton's second law:
		\begin{equation}
			\label{forces:torque}
		        \vector{\tau} =\deriv{\vector{L}}{t}
		\end{equation}
		For constant bodies, this formula can be rewritten as follows:
		\begin{equation}
			\vector{\tau} = I\vector{\alpha} = \vector{r}\times\vector{F}
		\end{equation}
	\end{formula}
    
	\begin{remark}
		From the previous definitions it follows that both the angular momentum and torque vectors are in fact pseudovectors and accordingly change sign under coordinate transformations with $\det = -1$.
	\end{remark}
    
	\newformula{Rotational energy}{\index{energy}
		\begin{equation}
			\label{forces:rotational_energy}
			E_{\text{rot}} = \frac{1}{2}I\omega^2
		\end{equation}
	}

\section{Central force}

	\begin{definition}[Central force]
		A central force is a force that only depends on the relative position of two objects:
		\begin{equation}
			\vector{F}_c \equiv F\Big(||\vector{r}_2 - \vector{r}_1||\Big)\boldsymbol{\hat{e}_r}
		\end{equation}
	\end{definition}

\section{Kepler problem}\index{Kepler!problem}

	\begin{formula}[Potential for a point mass]\index{gravity}
		\begin{equation}
	        	\label{forces:gravity:potential}
			\boxed{V = -G\stylefrac{M}{r}}
		\end{equation}
	        where $G = \num{6,67E-11}\stylefrac{Nm^2}{\text{kg}^2}$ is the \textbf{gravitational constant}.
	\end{formula}
    
    
\section{Harmonic oscillator}

	\begin{formula}[Harmonic potential]
		\begin{equation}
	        	\label{forces:harmonic_oscillator:potential}
			\boxed{V = \stylefrac{1}{2}kx^2}
		\end{equation}
		or
	        \begin{equation}
        		\label{forces:harmonic_oscillator:potential_2}
			V = \stylefrac{1}{2}m\omega^2x^2
		\end{equation}
        	where we have set $\omega = \sqrt{\stylefrac{k}{m}}$.
	\end{formula}
    
	\begin{formula}[Solution]
		\begin{align}
        		\label{forces:harmonic_oscillator:solution}
			x(t) &= A\sin\omega t + B\cos\omega t\\
			&=Ce^{i\omega t} + De^{-i\omega t}
		\end{align}
	\end{formula}

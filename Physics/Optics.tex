\chapter{Optics}
\section{General}
	\subsection{Conservation of energy}
    	From the law of conservation of energy we can derive the following formula:
        \begin{equation}
			\label{optics:energy_conservation}
            \boxed{T+R+A=1}
		\end{equation}
        where
        \begin{flalign*}
			\qquad&T:\text{Transmission coefficient}&\\
            &R:\text{Reflection coefficient}&\\
            &A:\text{Absorption coefficient}&
		\end{flalign*}
        
	\subsection{Photon}
    	\newformula{Energy}{
            \begin{equation}
                \label{optics:photon:energy}
                E = h\nu = \hbar\omega = \stylefrac{hc}{\lambda}
            \end{equation}
        }
        \newformula{Momentum}{
            \begin{equation}
                \label{optics:photon:momentum}
                p = \stylefrac{h}{\lambda} = \hbar k
            \end{equation}
            where formula \ref{optics:wave_number} was used in the last step.
        }
        \begin{remark*}
        	These formulas can also be (approximately) used for particles for which the rest mass (energy) is negligible.
        \end{remark*}

\section{Plane wave}

	\newformula{Wave number}{
    	\begin{equation}
			\label{optics:wave_number}
            k = \stylefrac{2\pi}{\lambda}
		\end{equation}
    }
	\begin{formula}[Plane wave]
		Following equations represent a plane wave moving in the $x$-direction and polarized in the $xy$-plane:
        \begin{equation}
			\label{optics:plane_wave}
            \vector{E}(x, t) = \text{Re}\left\{A\ \text{exp}\left[i\left(kx - \omega t + \phi\right)\right]\right\}\vector{e}_y
		\end{equation}
        \begin{equation}
			\label{optics:plane_wave_2}
            \vector{E}(x, t) = \text{Re}\left\{A\ \text{exp}\left[2\pi i\left(\stylefrac{x}{\lambda} - \stylefrac{t}{T} + \stylefrac{\phi}{2\pi}\right)\right]\right\}\vector{e}_y
		\end{equation}
	\end{formula}

\section{Refraction}
    \newformula{Refraction}{
    	\begin{equation}
			\label{optics:refraction}
            v_2 = \stylefrac{v_1}{n}
		\end{equation}
	}
    \begin{formula}[Di\"electric function]
		In the case of non-magnetic materials ($\mu_r\approx1$), we can write the di\"electric function as following:
       	\begin{equation}
			\label{optics:dielectric_function_non_magnetic}
            \epsilon = \epsilon_r + i\epsilon_i = \widetilde{n}^2 = (n+ik)^2
		\end{equation}
        Where $\widetilde{n}$ is the \textbf{complex refractive index} and $k$ is the \textbf{extinction coefficient}.
	\end{formula}
    

\section{Absorption}
    \begin{theorem}[Law of Lambert-Beer$^\dag$]\index{Lambert-Beer}
        \begin{equation}
        	\label{optics:lambert_beer}
	        \stylefrac{I(x)}{I(0)} = \text{exp}\left(-\stylefrac{4\pi\nu k}{c}x\right)
        \end{equation}
    \end{theorem}

    \begin{definition}[Absorption coefficient]\index{absorption}
        The constant factor in the Lambert-Beer law is called the absorption coefficient.
        \begin{equation}
	        \label{optics:absorption_coefficient}
    	    \alpha = \stylefrac{4\pi\nu k}{c}
        \end{equation}
    \end{definition}

\section{Diffraction}
	
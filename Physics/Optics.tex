\chapter{Optics}

\section{General}
\subsection{Conservation of energy}

    	From the law of conservation of energy we can derive the following formula:
        \begin{gather}
		\label{optics:energy_conservation}
		T+R+A=1
	\end{gather}
        where
        \begin{itemize}
		\item $T$ is the transmission coefficient
		\item $R$ is the reflection coefficient
		\item $A$ is the absorption coefficient
	\end{itemize}

\section{Plane wave}

	\newdef{Wave number}{\index{wave!number}\label{optics:wave_number}
		Consider a (plane) wave with wavelength $\lambda$. Its wave number is defined as follows:
		\begin{gather}
			k = \stylefrac{2\pi}{\lambda}.
		\end{gather}
	}
	\begin{formula}[Plane wave]\label{optics:plane_wave}
		Following equations represent a plane wave, polarized in the $xy$-plane, moving in the $x$-direction:
        	\begin{align}
			\vector{E}(x, t) &= \text{Re}\Big\{A\ \text{exp}\left[i\left(kx - \omega t + \phi\right)\right]\Big\}\vector{e}_y\\
			&= \text{Re}\left\{A\ \text{exp}\left[2\pi i\left(\stylefrac{x}{\lambda} - \stylefrac{t}{T} + \stylefrac{\phi}{2\pi}\right)\right]\right\}\vector{e}_y
		\end{align}
	\end{formula}

\subsection{Photon}

    	\newformula{Energy}{
		\begin{gather}
                	\label{optics:photon:energy}
                	E = h\nu = \hbar\omega = \stylefrac{hc}{\lambda}
		\end{gather}
        }
        \newformula{Momentum}{
		\begin{gather}
                	\label{optics:photon:momentum}
	                p = \stylefrac{h}{\lambda} = \hbar k
		\end{gather}
		where definition \ref{optics:wave_number} was used in the last step.
        }
        \begin{remark*}
        	These formulas can also be (approximately) used for particles for which the rest mass (energy) is negligible.
        \end{remark*}

\section{Refraction}

	\newformula{Refraction}{
    		\begin{gather}
			\label{optics:refraction}
			v_2 = \stylefrac{v_1}{n}
		\end{gather}
	}
	\begin{formula}[Dielectric function]
		In the case of non-magnetic materials ($\mu_r\approx1$), we can write the di\"electric function as following:
       		\begin{gather}
			\label{optics:dielectric_function_non_magnetic}
			\epsilon = \epsilon_r + i\epsilon_i = \widetilde{n}^2 = (n+ik)^2
		\end{gather}
		Where $\widetilde{n}$ is the \textbf{complex refractive index} and $k$ is the \textbf{extinction coefficient}.
	\end{formula}
    

\section{Absorption}

	\begin{theorem}[Law of Lambert-Beer$^\dag$]\index{Lambert-Beer}\label{optics:lambert_beer}
        	\begin{gather}
		        \stylefrac{I(x)}{I(0)} = \text{exp}\left(-\stylefrac{4\pi\nu k}{c}x\right)
		\end{gather}
	\end{theorem}

	\begin{definition}[Absorption coefficient]\index{absorption}
        	The constant factor in the Lambert-Beer law is called the absorption coefficient.
	        \begin{gather}
		        \label{optics:absorption_coefficient}
			\alpha = \stylefrac{4\pi\nu k}{c}
		\end{gather}
	\end{definition}
\chapter{Angular Momentum}

    In this chapter the general angular momentum operator $\hat{J} = \left(\hat{J}_x,\hat{J}_y,\hat{J}_z\right)$ is considered.

\section{General operator}

    \begin{property}[Lie algebra]
       The angular momentum operators generate a Lie algebra \ref{lie:lie_algebra}. The Lie bracket is defined by the following commutation relation:
       \begin{gather}
           \label{angular_momentum:commutation}
           \left[\hat{J}_i,\hat{J}_j\right] = i\hbar\varepsilon_{ijk}\hat{J}_k.
       \end{gather}
       Since rotations correspond to actions of the orthogonal group $\mathrm{SO}(3)$ it should not come as a surprise that the above relation is that of the Lie algebra $\mathfrak{so}(3)$ (Example \ref{lie:so3}).
    \end{property}

    \begin{property}
       The mutual eigenbasis of $\hat{J}^2$ and $\hat{J}_z$ is defined by the following two eigenvalue equations:
       \begin{align}
           \label{angular_momentum:j}
           \hat{J}^2|j,m\rangle &= j(j+1)\hbar^2\,|j,m\rangle\\
           \label{angular_momentum:m}
           \hat{J}_z|j,m\rangle &= m\hbar\,|j,m\rangle.
        \end{align}
    \end{property}

    \newdef{Ladder operators\footnotemark}{\index{ladder operators}
        \footnotetext{Also called the \textbf{creation} and \textbf{annihilation} operators (especially in quantum field theory).}
        The raising and lowering operators $\hat{J}_+$ and $\hat{J}_-$ are defined as follows:
        \begin{gather}
            \hat{J}_+ := \hat{J}_x + i\hat{J}_y \qquad\text{and}\qquad \hat{J}_- := \hat{J}_x - i\hat{J}_y.
        \end{gather}
        These operators only change the quantum number $m_z$, not the total angular momentum.
    }
    \begin{result}
        From the commutation relations of the angular momentum operators, one can derive the commutation relations of the ladder operators:
        \begin{gather}
            \left[\hat{J}_+,\hat{J}_-\right] = 2\hbar\hat{J}_z.
        \end{gather}
    \end{result}

    \begin{formula}
        The total angular momentum operator $\hat{J}^2$ can now be expressed in terms of $\hat{J}_z$ and the ladder operators using the commutation relation \eqref{angular_momentum:commutation}:
        \begin{gather}
            \hat{J}^2 = \hat{J}_+\hat{J}_- + \hat{J}_z^2 - \hbar\hat{J}_z.
        \end{gather}
    \end{formula}
    \begin{remark}[Casimir operator]\index{Casimir!invariant}
        From the definition of $\hat{J}^2$ it follows that this operator is a Casimir invariant \ref{lie:casimir_invariant} of $\mathfrak{so}(3)$.
    \end{remark}

\section{Rotations}
\subsection{Infinitesimal rotation}

    \begin{formula}
        An infinitesimal rotation $\hat{R}(\delta\vector{\varphi})$ is given by the following formula:
        \begin{gather}
            \label{angular_momentum:infinitesimal_rotation}
            \hat{R}(\delta\vector{\varphi}) = \mathbbm{1} - \frac{i}{\hbar}\vector{J}\cdot\delta\vector{\varphi}.
        \end{gather}
        A finite rotation can be generated by applying this infinitesimal rotation repeatedly:
        \begin{gather}
            \label{angular_momentum:finite_rotation}
            \hat{R}(\vector{\varphi}) = \left(\mathbbm{1} - \frac{i}{\hbar}\vector{J}\cdot\frac{\vector{\varphi}}{n}\right)^n = \exp\left(-\frac{i}{\hbar}\vector{J}\cdot\vector{\varphi}\right).
        \end{gather}
    \end{formula}

    \newformula{Matrix elements}{\index{Wigner!$D$-functions}
        Applying a rotation over an angle $\varphi$ about the $z$-axis to a state $|j,m\rangle$ gives
        \begin{gather}
            \hat{R}(\varphi\vector{e}_z)|j,m\rangle = \exp\left(-\frac{i}{\hbar}\hat{J}_z\varphi\right)|j,m\rangle = \exp\left(-\frac{i}{\hbar}m\varphi\right)|j,m\rangle.
        \end{gather}
        Multiplying these states with a bra $\langle j',m'|$ and using the orthonormality of the eigenstates gives the matrix elements of the rotation operator:
        \begin{gather}
            \hat{R}_{ij}(\varphi\vector{e}_z) = \exp\left(-\frac{i}{\hbar}m\varphi\right)\delta_{jj'}\delta_{mm'}.
        \end{gather}
        From the expression of the angular momentum operators and the rotation operator it is clear that a general rotation has no effect on the total angular momentum number $j$. This means that the rotation matrix will be block diagonal with respect to $j$. This amounts to the following reduction of the representation of the rotation group:
        \begin{gather}
            \langle j,m'|\hat{R}(\varphi\vector{n})|j,m\rangle = \mathcal{D}^{(j)}_{m,m'}(\hat{R}),
        \end{gather}
        where the functions $\mathcal{D}^{(j)}_{m,m'}(\hat{R})$ are called the \textbf{Wigner $D$-functions}. For every value of $j$ there are $(2j+1)$ values for $m$. This implies that the matrix $\mathcal{D}^{(j)}(\hat{R})$ is a $(2j+1)\times(2j+1)$-matrix.
    }

\subsection{Spinor representation}

    \newdef{Pauli matrices}{\index{Pauli!matrix}\label{angular_momentum:pauli_matrices}
        \begin{gather}
            \sigma_x :=
            \begin{pmatrix}
                0&1\\
                1&0
            \end{pmatrix}
            \qquad
            \sigma_y :=
            \begin{pmatrix}
                0&-i\\
                i&0
            \end{pmatrix}
            \qquad
            \sigma_z :=
            \begin{pmatrix}
                1&0\\
                0&-1
            \end{pmatrix}
        \end{gather}
        From this definition it is clear that the Pauli matrices are Hermitian and unitary. Together with the $2\times2$ identity matrix, they form a basis for the space of $2\times2$ Hermitian matrices. For this reason the identity matrix is often denoted by $\sigma_0$ (especially in the context of relativistic QM).
    }

    \begin{formula}
        In the spinor representation ($J=\frac{1}{2}$) the Wigner-$D$ matrix reads as follows:
        \begin{gather}
            \mathcal{D}^{(1/2)}(\varphi\vector{e}_z) =
            \begin{pmatrix}
                e^{-i/2 \varphi}&0\\
                0&e^{i/2\varphi}
            \end{pmatrix}.
        \end{gather}
    \end{formula}

\section{Coupling of angular momenta}

    Due to the tensor product structure of a coupled Hilbert space, the angular momentum operator $\hat{J}_i$ should now be interpreted as $\mathbbm{1}\otimes\cdots\otimes\hat{J}_i\otimes\cdots\otimes\mathbbm{1}$ (cf. Notation \ref{vector:tensor_abuse}). Because the angular momentum operators $\hat{J}_{l\neq i}$ do not act on the space $\mathcal{H}_i$, one can pull these operators through the tensor product: \[\hat{J}_i|j_1\rangle\otimes\cdots\otimes|j_n\rangle = |j_1\rangle\otimes\cdots\otimes\hat{J}_i|j_i\rangle\otimes\cdots\otimes|j_n\rangle.\] The basis used above is called the \textbf{uncoupled basis}.

    For simplicity the total Hilbert space is from here on assumed to be that of a two-particle system. Let $\hat{J}$ denote the total angular momentum defined as
    \begin{gather}
        \hat{J} = \hat{J}_1 + \hat{J}_2.
    \end{gather}
    With this operator one can define a \textbf{coupled} state $|J,M\rangle$, where $M$ is the total magnetic quantum number which ranges from $-J$ to $J$.

    \newformula{Clebsch-Gordan coefficients}{\index{Clebsch-Gordan coefficient}
        Because both bases (coupled and uncoupled) span the total Hilbert space $\mathcal{H}$, there exists an invertible transformation between them. The transformation coefficients can be found by using the resolution of the identity:
        \begin{gather}
            \label{angular_momentum:clebsch-gordan}
            |J,M\rangle = \sum_{m_1=-j_1}^{j_1}\sum_{m_2=-j_2}^{j_2}|j_1,j_2,m_1,m_2\rangle\langle j_1,j_2,m_1,m_2|J,M \rangle.
        \end{gather}
        These coefficients are called the Clebsch-Gordan coefficients.
    }

    \begin{property}
        By acting with the operator $\hat{J}_z$ on both sides of Equation \eqref{angular_momentum:clebsch-gordan} it is possible to prove that the Clebsch-Gordan coefficients are nonzero if and only if $M = m_1 + m_2$.
    \end{property}
\chapter{\difficult{Axiomatic QFT}}

    For the sections on the Haag-Kastler framework and its extensions we refer to the work of \textit{Brunetti, Fredenhagen et al}. A reference for the remaining sections on algebraic QFT is \cite{baez_aqft}. For the section on TQFTs we refer to the original paper of \textit{Atiyah} \cite{atiyah}.

\section{Algebraic QFT}
\subsection{Haag-Kastler axioms}

    \begin{axiom}[Local net of observables]\index{microcausality|see{locality}}\index{Einstein!causality}\index{locality}\label{qft:microcausality}
        To every causally closed set\footnote{See definition \ref{relativity:causal_closure}.} $O$ one associates a $C^*$-algebra $\mathcal{A}(O)$. We require this assignment to satisfy the following conditions:
        \begin{enumerate}
            \item \textbf{Isotony}: If $O_1\subset O_2$, then $\mathcal{A}(O_1)\hookrightarrow\mathcal{A}(O_2)$.
            \item \textbf{(Causal) locality}\footnote{Also called \textbf{microcausality} or \textbf{Einstein causality}.}: If $O_1$ and $O_2$ are spacelike separated, then $[\mathcal{A}(O_1),\mathcal{A}(O_2)] = 0$ (as a graded commutator) within a larger algebra $\mathcal{A}(O)$ such that $O_1, O_2\subset O$.
        \end{enumerate}
    \end{axiom}
    \sremark{The isotony condition implies that local nets of observables are modelled by copresheaves $\mathbf{Mink}\rightarrow\mathbf{C^*Alg}$ that map (mono)morphisms to monomorphisms.}

    \begin{axiom}[Poincar\'e covariance]
        For all causally closed sets $O$ and Poincar\'e transformations $\Lambda$ there exists an isomorphism $\alpha^O_\Lambda:\mathcal{A}(O)\rightarrow\mathcal{A}(\Lambda O)$ such that the following conditions are satisfied:
        \begin{enumerate}
            \item If $O_1\subset O_2$, then $\alpha_\Lambda\circ\iota_{O_1,O_2} = \iota_{\Lambda O_1, \Lambda O_2}\circ\alpha_\Lambda$.
            \item The isomorphisms satisfy a composition rule: $\alpha^{\Lambda O}_{\Lambda'}\circ\alpha^O_\Lambda = \alpha^O_{\Lambda'\Lambda}$.
        \end{enumerate}
    \end{axiom}

    \begin{axiom}[Spectrum]
        For all spacetime regions $O$ one can construct a faithful $C^*$-algebra representation $\rho_O$ of $\mathcal{A}(O)$ on a fixed Hilbert space (see the GNS construction \ref{operators:gns}). The different representations should be compatible, i.e. if $O_1\subset O_2$ then the restriction of $\rho_{O_2}$ to $\mathcal{A}(O_1)$ should equal $\rho_{O_1}$. Furthermore, all spacetime translations are to be implemented unitarily:
        \begin{gather}
            U(a)\rho_O(c)U(a)^{-1} = \rho_{O+a}(\alpha^O_a(c))
        \end{gather}
        for all $c\in\mathcal{A}(O)$, where $U$ is a unitary representation of the translation subgroup. In addition we require that the generators of the translation subgroup have a spectrum that is contained in the future light cone.
    \end{axiom}

    The following axiom is not part of the standard Haag-Kastler framework but can be added to introduce a notion of time evolution:
    \begin{axiom}[Time slice]
        Consider two spacetime regions $O_1, O_2$. If $O_1$ contains a Cauchy surface of $O_2$ then the morphism $\mathcal{A}(O_1\hookrightarrow O_2)$ of $C^*$-algebras is an isomorphism.
    \end{axiom}

    \begin{axiom}[Haag duality]\index{Haag!duality}
        Let $\overline{O}$ denote the spacelike complement of $O$ and let $\mathcal{A}'$ denote the commutant of $\mathcal{A}$. Haag duality states that\footnote{Here it should be understood that $\mathcal{A}\left(\overline{O}\right)$ is the algebra generated by all algebras $\mathcal{A}(Q)$ where $Q$ ranges over the causally closed sets in $\overline{O}$.}
        \begin{equation}
            \mathcal{A}\left(\overline{O}\right)' = \mathcal{A}(O)
        \end{equation}
        for all causally closed sets $O$.
    \end{axiom}
    \remark{Haag duality is known to hold for all free theories and even for some interacting theories. However, it is also known to fail in the case of symmetry breaking \cite{Roberts_haag}.}

    To generalize the above axiom system to globally hyperbolic space times, we must enter the realm of category theory. We will follow the notation\footnote{This may cause confusion with other notations used in this text. $\mathbf{Loc}$ here has nothing to do with the category of locales from chapter \ref{chapter:topology}.} of \cite{cal_strobl} (?? AND OTHERS ??): Let $\textbf{Loc}$ be the category of globally hyperbolic space times with orientation- and causal structure-preserving isometries. Let $\textbf{Obs}$ be the category of relevant algebras\footnote{Commutative algebras for classical physics and $C^*$-algebras for quantum theories.} together with suitable algebra morphisms. The assignment of algebras is then given by a functor $\func{\mathfrak{U}}{Loc}{Obs}$. The Haag-Kastler framework is recovered when we restrict $\textbf{Loc}$ to the subcategory of globally hyperbolic subsets of some manifold (with inclusions as morphisms).

\subsection{Weyl systems}

    \newdef{Weyl system}{\index{Weyl!system}
        Let $(L, \omega)$ be a symplectic vector space and let $K$ be a complex vector space. Consider a map $W$ from $L$ to the space of unitary operators on $K$. The pair $(K, W)$ is a Weyl system over $(L, \omega)$ if it satisfies
        \begin{equation}
            W(z)W(z') = e^{\frac{i}{2}\omega(z, z')}W(z+z')
        \end{equation}
        for all $z,z'\in L$.
    }
    \remark{This is a generalization of the canonical commutation relations in their Weyl form. (See definition \ref{qm_formalism:CCR}.)}

    \newdef{Heisenberg system}{\index{Heisenberg!system}
        The generators (which exist by Stone's theorem \ref{operator:stone}) $\phi(z)$ of the maps $t\mapsto W(tz)$ are said to form a Heisenberg system. These operators satisfy the following properties:
        \begin{itemize}
            \item $\lambda\phi(z) = \phi(\lambda z)$ for all $\lambda>0$,
            \item $[\phi(z),\phi(z')] = -i\omega(z, z')$, and
            \item $\phi(z+z')$ is the closure \ref{operator:closure} of $\phi(z)+\phi(z')$.
        \end{itemize}
    }

\section{Topological QFT}

    For convenience we remind the reader of some of the notations that we use. $\mathbf{FinVect}$ will denote the category of finite-dimensional vector spaces over $\mathbb{C}$. $\mathbf{Bord}^d_{d-1}$ will denote the category of $d$-dimensional cobordisms (see definition \ref{manifolds:cobordism}).

\subsection{Atiyah-Segal axioms}

    \begin{axiom}[Atiyah-Segal]\index{Atiyah-Segal}\index{TQFT}
        \nomenclature[A_TQFT]{TQFT}{topological quantum field theory}
        A $d$-dimensional topological quantum field theory (TQFT) is a symmetric monoidal functor $F:\textbf{Bord}_{d-1}^d\rightarrow\textbf{FinVect}$. This means (among other things) that $F$ is a map satisfying the following axioms:
        \begin{enumerate}
            \item \textbf{Normalization}: $F(\emptyset)=\mathbb{C}$.
            \item \textbf{Disjoint union}: $F(M\sqcup M') = F(M)\otimes F(M')$.
            \item \textbf{Composition}: If $N=M\cup M'$, where $\partial M$ and $\partial M'$ have opposite orientations, then \[F(N) = F(M)\circ F(M').\]
            \item \textbf{Invariance}: If $f: M\rightarrow M'$ is a diffeomorphism rel boundary then $F\circ f = F$.
        \end{enumerate}
        In the above conditions $M, M'$ are $d$-dimensional cobordisms between $(d-1)$-dimensional (closed) smooth manifolds.
    \end{axiom}

    \begin{example}[1D]
        In 1D a TQFT functor gives rise to the following correspondence:
        \begin{equation*}
            \begin{array}{l|l}
                \text{Point with orientation } + & \text{Vector space } V\\
                \text{Point with reversed orientation } - & \text{Dual space } V^*\\
                \text{Line between points} & \text{Linear map } f:V\rightarrow V\\
                \text{Cap between $\emptyset$ and points } +, - & \text{Coevaluation } \mathbb{C}\rightarrow V\otimes V^*\\
                \text{Cup between points $-, +$ and }\emptyset & \text{Evaluation } V^*\otimes V\rightarrow\mathbb{C}
            \end{array}
        \end{equation*}
        Essentially this gives us the structure of a (finite-dimensional) vector space and its dual.
    \end{example}

    \begin{example}[2D]
        In 2D one can obtain a similar result by drawing all possible configurations. However, the existence and combination of "pair of pants"-diagrams gives a richer structure. For 2D TQFTs the corresponding object is a (finite-dimensional) commutative and cocommutative Frobenius algebra (see definition \ref{linalgebra:frobenius}).
    \end{example}

    In dimensions 3 and higher the definition above is intractable. To allow the construction to be generalized to higher dimensions one considers the following (extended) definition:
    \newdef{Extended TQFT}{\index{TQFT}
        A $d$-dimensional extended TQFT is a symmetric monoidal functor $F:\textbf{Bord}_1^d\rightarrow\textbf{FinVect}$ satisfying the Atiyah-Segal axioms where the invariance axiom is required only at the highest level of $k$-morphisms.
    }

\subsection{Open-closed TQFT}

    For a complete definition we refer to the original paper \cite{open_closed}.

    In the case of ordinary TQFTs as defined in the previous section one considers cobordisms between closed manifolds, hence the relevant objects in this category are manifolds with boundary. A generalization is obtained by relaxing the constraint on the cobordisms and allowing the notion of manifolds with corners (see section \ref{section:manifold_boundary}). For simplicity we will only consider the case of 2D TQFTs (as in the original definition).

    ?? COMPLETE ??
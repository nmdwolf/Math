\chapter{Conformal Field Theory}\index{conformal|seealso{CFT}}

	References for this chapter are \cite{CFT} and the lecture notes by Schellekens.

\section{Conformal invariance}
\subsection{Conformal transformations}
	
	\newdef{Conformal transformation}{
		Consider two (pseudo-)Riemannian manifold $(M, g)$ and $(M', g')$. A smooth map $\varphi:M\rightarrow M'$ if it leaves the metric invariant up to a scale transformation:\footnote{Compare this to definition \ref{diff:conformal_map}.}
		\begin{gather}
			\varphi^*g' = g.
		\end{gather}
	}
	
	Infinitesimally these maps are described by a specific type of vector field:
	\newdef{Conformal Killing vector}{\index{Killing!conformal vector}
		Consider a pseudo-Riemannian manifold $(M, g)$. A vector field $X$ is said to be conformal with conformal factor $\Omega:M\rightarrow\mathbb{R}$ if it satisfies
		\begin{equation}
			\mathcal{L}_Xg = \Omega g.
		\end{equation}
		In local coordinates this amounts to
		\begin{equation}
			\nabla_\mu X_\nu + \nabla_\nu X_\mu = \Omega g_{\mu\nu}
		\end{equation}
		where $\nabla$ is the Levi-Civita connection associated to $(M, g)$. Equivalently, a vector field is conformal if its flow is a conformal map.
	}
	
	By parametrizing an infinitesimal transformation as $x^\mu\rightarrow x^\mu+\varepsilon^\mu$ one obtains the following infinitesimal generators:
	\begin{itemize}
		\item Translations: $a^\mu\partial_\mu$
		\item Rotations (orthogonal transformations): $\omega^\mu_{\ \nu} x^\nu\partial_\mu$
		\item Dilations: $\lambda x^\mu\partial_\mu$
		\item Special conformal transformations: $x^2b^\mu\partial_\mu - 2(b\cdot x)x^\mu\partial_\mu$
	\end{itemize}
	As usual, exponentiating these generators gives the finite transformations. The conformal group in signature $(p, q)$ is isomorphic to SO$(p+1, q+1)$. One immediately notices that the Poincar\'e group is a subgroup of the conformal group.

\subsection{In dimension $d=2$}

	In dimension 2 (with Euclidean signature) something special happens. By inserting $d=2$ in the conformal Killing equation we obtain the Cauchy-Riemann equations. The scale factor $\Omega(x)$ becomes $\left|\frac{\partial f}{\partial z}\right|^2$ where $f$ is analytic in $z = x_1+ix_2$.

	Infinitesimally this gives us an infinite-dimensional algebra. As generators\footnote{These generate the transformation $z\mapsto z-z^{n+1}$ and $\zbar\mapsto\zbar-\zbar^{n+1}$ respectively.} we choose
	\begin{align}
		l_n(z) &= -z^{n+1}\partial_z\\
		\overline{l}_n(\zbar)&=-\zbar^{n+1}\partial_{\zbar}
	\end{align}
	These generators generate isomorphic Lie algebras with the following commutation relation:
	\begin{gather}
		[l_m, l_n] = (m-n)l_{m+n}
	\end{gather}
	This conformal algebra is called the \textbf{Witt-algebra}.\index{Witt algebra}
	
	\newdef{Conformal group}{\index{conformal!group}
		The conformal group of a pseudo-Riemannian manifold $M$ is not just the conformal diffeomorphism group of $M$. The conformal group Conf$(M)$ is defined as the connected component of the identity element in the conformal diffeomorphism group of the conformal compactification of $M$.
	}
	
	\begin{remark}
		Often one finds in the literature that the conformal group of a 2D CFT is infinite-dimensional. However this statement is not entirely true. It is true that the Witt algebra is infinite-dimensional, but one cannot globally exponentiate all generators $l_m$. First of all it should be remarked that the space of all holomorphic functions does not have a group structure, i.e. the composition of holomorphic functions does not have to be holomorphic. The correct notion of a conformal group for 2D Euclidean CFT's is the M\"obius group PSL$(2, \mathbb{C})$. This group is obtained as the Lie group generated by $l_0$ and $l_{\pm1}$, the only generators which can be globally exponentiated.
		
		What is however true is that the conformal group of 2D Minkowski space is infinite-dimensional. It can be shown that Conf$(\mathbb{R}^{1, 1})$ is isomorphic to $\text{Diff}(S^1)_+\times\text{Diff}(S^1)_+$ and the (orientation-preserving) diffeomorphism group Diff$(S^1)_+$ is indeed an infinite-dimensional Lie group (see further below).
		
		At last we should comment that although the conformal group of $\mathbb{R}^{2, 0}$ is finite-dimensional, the infinite-dimensionality of the Witt algebra (and of its extensions) is sufficient for all physical purposes.
	\end{remark}

\subsection{Minkowski space}

	As mentioned in the remark above, the conformal group of 2D Minkowski space is infinite-dimensional. The theory of infinite-dimensional manifolds is however a bit more intricate then the theory of finite-dimensional manifolds. A little introduction is therefore in order.

	\newdef{Fr\'echet space}{\index{Fr\'echet!space}
		A complete locally convex Haussdorf topological vector space which admits a translation-invariant metric. Equivalently, a complete Haussdorf topological vector space for which the topology is induced by a countable family of seminorms.
	}
	\newdef{Smooth map}{\index{smooth!function}\index{G\^ateaux}
		The \textit{G\^ateaux derivative} of a continuous map of Fr\'echet spaces $f:F\rightarrow G$ is defined as follows:
		\begin{gather}
			D_hf(a) = \lim_{t\rightarrow0}\frac{f(a+th) - f(a)}{t}
		\end{gather}
		If this limit exists and is continuous in both $a, h$ then $f$ is said to be of class $C^1$. By iterating this construction one can define smooth maps of Fr\'echet spaces.
	}
	\newdef{Fr\'echet manifold}{
		A Fr\'echet manifold is a Haussdorf space $M$ together with an atlas of coordinate charts $(U, \varphi)$ such that $\varphi:U\rightarrow F_U$ are homeomorphisms onto a Fr\'echet space and such that the transition functions are smooth maps (of Fr\'echet spaces).
	}
	
	Using the above definitions we can start to analyze the group Diff$(S^1)_+$. First of all we look at the space of all smooth maps $S^1\rightarrow S^1$. This space has a natural structure of Fr\'echet manifold modelled on the Fr\'echet space $\mathfrak{X}(S^1)$ of vector fields on the circle\footnote{By this definition we see that $\mathfrak{X}(S^1)$ is isomorphic (as a Lie algebra) to the mapping space $C^\infty(S^1, \mathbb{R})$. As such we will identify vector fields $v$ with their corresponding function $\xi$.}:
	\begin{gather}
		\mathfrak{X}(S^1) = \left\{\xi(\theta)\frac{\partial}{\partial\theta}:\theta\in C^\infty(S^1)\right\}
	\end{gather}
	Let $V_0$ be the set of vector fields which have norm $||v||\leq\pi$ and let $U_0$ be the set of smooth mappings $f\in C^\infty(S^1, S^1)$ such that $f(\theta)\neq-\theta$ for all $\theta\in S^1$. Then there exists a diffeomorphism $\psi:V_0\rightarrow U_0$ which assigns to any vector field $v$ the function $\psi_v:S^1\rightarrow S^1$ such that the arc between $\theta$ and $\psi_v(\theta)$ has length $||v(\theta)||$. If we choose an open subset $U\subset U_0$ of diffeomorphism then we obtain a chart $(U, \psi^{-1})$ around the identity map. Charts around any diffeomorphism $f:S^1\rightarrow S^1$ are obtained by left multiplication of $U$.
	
	The Lie algebra of Diff$(S^1)_+$ is accordingly given by $\mathfrak{X}(S^1)$ but the induced Lie bracket is the commutator of vector fields with the opposite sign: \[[\cdot, \cdot]_{Lie} = -[\cdot, \cdot]_{\mathfrak{X}(S^1)}\]
	
	Now it is interesting to note that the Witt algebra is actually a subalgebra of $\mathfrak{X}(S^1)$. Consider the maps $\xi_n(\theta):=-ie^{in\theta}$ (the minus sign is a convention). The associated vector fields then satisfy:
	\begin{gather}
		\left[\xi_k(\theta)\frac{\partial}{\partial\theta}, \xi_l(\theta)\frac{\partial}{\partial\theta}\right] = -i(l-k)\xi_{k+l}(\theta)\frac{\partial}{\partial\theta}
	\end{gather}
	These are exactly the relations for the Witt algebra.

\section{Quantization}

	\sremark{From here one we will always work in 2 dimensions.}

\subsection{Virasoro algebra}\index{Virasoro algebra}

	When going from classical systems to quantum systems we have to replace symmetry operations by (unitary) projective representations. As in the case of ordinary groups, the projective representations of Lie groups are related to central extensions of Lie groups. As an example of the Lie group-Lie algebra correspondence it can be shown that central extensions of Lie algebras (see section \ref{section:central_extension_algebra}) are in correspondence with central extensions of Lie groups.
	
	If we now want a quantum theory equipped with an action of the ''conformal group'', then we need to construct a central extension of the Witt algebra. By applying the construction of section \ref{section:central_extension_algebra} we obtain the \textbf{Virasoro algebra} as a (universal) central extension of the Witt algebra by\footnote{The reason why we extend by $\mathbb{C}$ instead of $\mathbb{R}$ is that we are working with complexified Lie algebras.} $\mathbb{C}$ associated to the cocycle \[\Theta:(L_m, L_n)\mapsto \frac{c}{12}m(m^2-1)\delta_{m+n, 0}\]

	To obtain the Virasoro algebra from a more physical point of view we look at the stress-energy tensor. One can expand the stress-energy tensor in modes as follows:
	\begin{gather}
		T(z) = \sum_{n\in\mathbb{Z}}z^{-n-2}L_n
	\end{gather}
	where the exponent is chosen such that the mode $L_{-n}$ has scaling dimension $n$. This relation can be inverted using the residue theorem to obtain the following expression:
	\begin{gather}
		L_n = \oint\frac{dz}{2\pi i}z^{n+1}T(z).
	\end{gather}
	Using the above expression and the product operator expansion of $T(z)T(w)$ one obtains exactly the commutation relation of the Virasoro algebra:
	\begin{gather}
		[L_m, L_n] = (m-n)L_{m+n} + \frac{c}{12}m(m^2-1)\delta_{m+n, 0}
	\end{gather}
	
	However there remains some freedom in the definition of the generators $L_n$. We can fix the definitions by requiring that the subalgebra spanned by $L_0$ and $L_{\pm1}$ still spans $\mathfrak{sl}(2, \mathbb{C})$ or equivalently that SL$(2, \mathbb{C})$ is still a global symmetry group even though the quantum algebra contains a central charge. ??EXPLAIN??

\subsection{Radial quantization}


\subsection{Representation theory}

	\newdef{Highest weight state}{
		A state with minimal eigenvalue for $L_0$. Equivalently, a state that is annihilated by all generators $L_n$ for $n\geq1$:
		\begin{gather}
			L_n|h\rangle = 0
		\end{gather}
	}

	\newdef{Vacuum}{\index{vacuum}
		Consider the Virasoro generators $\{L_n\}_{n\in\mathbb{Z}}$. The vacuum $|0\rangle$ is defined as the maximally symmetric state. In terms of generators this means that we want to have \[L_n|0\rangle = 0\] for as many $n$ as possible. However due to the Virasoro commutation relations this is not possible for all $n\in\mathbb{Z}$. Instead one can only require the vanishing for $n\geq-1$.
	}
	
	\newdef{Descendants}{
		By acting with the generators $L_n$, $n\leq-1$ on a highest weight state $|h\rangle$ one obtains a whole family of states. These are called the descendants of $|h\rangle$ and together they span the Verma module\footnote{See definition \ref{lie:verma_module}.} (associated to $|h\rangle$).
	}
	
\section{Partition functions}
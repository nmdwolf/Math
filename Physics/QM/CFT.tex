\chapter{Conformal Field Theory}\index{conformal|seealso{CFT}}

\section{Conformal invariance}


	From here one we will always work in 2 dimensions.

\section{Quantization}


\section{Representation theory}

	\newdef{Highest weight state}{
		A state with minimal eigenvalue for $L_0$. Equivalently, a state that is annihilated by all generators $L_n$ for $n\geq1$:
		\begin{gather}
			L_n|h\rangle = 0
		\end{gather}
	}

	\newdef{Vacuum}{\index{vacuum}
		Consider the Virasoro generators $\{L_n\}_{n\in\mathbb{Z}}$. The vacuum $|0\rangle$ is defined as the maximally symmetric state. In terms of generators this means that we want to have \[L_n|0\rangle = 0\] for as many $n$ as possible. However due to the Virasoro commutation relations this is not possible for all $n\in\mathbb{Z}$. Instead one can only require the vanishing for $n\geq-1$.
	}
	
	\newdef{Descendants}{
		By acting with the generators $L_n$, $n\leq-1$ on a highest weight state $|h\rangle$ one obtains a whole family of states. These are called the descendants of $|h\rangle$ and together they span the Verma module\footnote{See definition \ref{lie:verma_module}.} (associated to $|h\rangle$).
	}
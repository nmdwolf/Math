\chapter{Quantum Information Theory}\label{chapter:quantum_computing}

\section{Entanglement}
\subsection{Schmidt decomposition}

    \begin{construct}[Schmidt decomposition]\index{Schmidt!decomposition}\index{rank}
        Consider a bipartite state $|\psi\rangle\in\mathcal{H}_1\otimes\mathcal{H}_2$. For any such state there exist orthonormal sets $\big\{|e_i\rangle, |f_j\rangle\big\}_{i,j\leq\kappa}$ such that
        \begin{gather}
            |\psi\rangle = \sum_{i=1}^k\lambda_i|e_i\rangle\otimes|f_i\rangle,
        \end{gather}
        where the coefficients $\lambda_i$ are nonnegative real numbers. Together with the expansion vectors they can be obtained from a singular value decomposition of the coefficient matrix $\mathbf{C}$ of $|\psi\rangle$ in some bases of $\mathcal{H}_1$ and $\mathcal{H}_2$. The number $\kappa$ is called the \textbf{Schmidt rank} of $|\psi\rangle$.
    \end{construct}

    \newdef{Entangled states}{\index{separable!state}\index{entanglement}
        Consider a state $|\psi\rangle$ and consider its Schmidt decomposition. If the Schmidt rank is 1, i.e. the state can be written as $|\psi\rangle = |v\rangle\otimes|w\rangle$, the state is said to be \textbf{separable}. Otherwise the state is said to be entangled.
    }

\subsection{Bell states}

    \newdef{Bell state}{\index{Bell state}\index{Einstein-Podolsky-Rosen|see{Bell state}}
        \nomenclature[A_EPR]{EPR}{Einstein-Podolsky-Rosen}
        A (binary) Bell state (also called a \textbf{cat state} or \textbf{Einstein-Podolsky-Rosen pair}) is defined as the following entangled state:
        \begin{gather}
            |\Phi^+\rangle := \frac{1}{\sqrt{2}}\Big(|00\rangle+|11\rangle\Big).
        \end{gather}
        In fact this state can be extended to a full maximally entangled basis for the 2-qubit Hilbert space:
        \begin{gather}
            |\Phi^-\rangle := \frac{1}{\sqrt{2}}\Big(|00\rangle-|11\rangle\Big)\nonumber\\
            |\Psi^+\rangle := \frac{1}{\sqrt{2}}\Big(|01\rangle+|10\rangle\Big)\\
            |\Psi^-\rangle := \frac{1}{\sqrt{2}}\Big(|01\rangle-|10\rangle\Big).\nonumber
        \end{gather}
    }

    \begin{method}[Dense coding\footnotemark]\index{dense!coding}
        \footnotetext{Sometimes called \textbf{superdense coding}.}
        Consider the Bell state $|\Phi^+\rangle$. By acting with one of the (unitary) spin-flip operators $X,Y,Z$ one can obtain any of the other three Bell states:
        \begin{align}
            X|\Phi^+\rangle &= |\Phi^-\rangle\nonumber\\
            Y|\Phi^+\rangle &= |\Psi^+\rangle\\
            Z|\Phi^+\rangle &= |\Psi^-\rangle.\nonumber
        \end{align}
        In a typical Alice-and-Bob-style experiment one can ask the question if this observation allows to achieve a better-than-classical communication channel. If Alice performs a spin flip on her qubit, although the resulting state has instantly ``changed'' (cf. \textit{spooky action at a distance}), Bob still cannot uniquely determine what this state is (since the resulting state is still maximally entangled). However, if Alice sends her qubit to Bob, the latter can perform a measurement on the composite system to find out what the state is and in this way determine which operation Alice performed ($\mathbbm{1},X,Y,Z$). Alice has thus effectively sent 2 classical bits of information through 1 qubit. Note that due to the fact that Alice still has to send her qubit through classical means, no faster-than-light communication is achieved.
    \end{method}

    \newdef{GHZ\footnotemark state}{\index{Greenberger-Horne-Zeilinger state}
        \footnotetext{\textit{Greenberger-Horne-Zeilinger}}
        \nomenclature[A_GHZ]{GHZ}{Greenberger-Horne-Zeilinger}
        The GHZ state is defined as the multiparticle qudit ($d,N>2$) version of the Bell state above and is, hence, also referenced to as a cat state:
        \begin{gather}
            |\mathrm{GHZ}\rangle = \frac{1}{\sqrt{d}}\sum_{i=0}^{d-1}|i\rangle^{\otimes N}.
        \end{gather}
    }

\section{Density operators}

	\newdef{Reduced density operator}{\index{density!operator}
		Let $|\Psi\rangle_{AB}$ be the state of a bipartite system. The reduced density operator $\hat\rho_A$ of $A$ is defined as follows:
		\begin{gather}
			\hat\rho_A := \mathrm{tr}_B|\Psi\rangle_{ABAB}\langle\Psi|.
		\end{gather}
	}

	\newdef{Purification}{\index{purification}
		Let $\hat\rho_A$ be the density operator of a system $A$. A purification of $\hat\rho_A$ is a pure state $|\Psi\rangle_{AB}$ of some composite system $AB$ such that
		\begin{gather}
			\hat\rho_A = \mathrm{tr}_B|\Psi\rangle_{ABAB}\langle\Psi|.
		\end{gather}
	}
	\begin{property}
		Any two purifications of the same density operator $\hat\rho_A$ are related by a transformation $\mathbbm{1}_A\otimes\hat{V}$ with $\hat{V}$ a unitary operator on $\mathcal{H}_B$.
	\end{property}

    ?? COMPLETE ??
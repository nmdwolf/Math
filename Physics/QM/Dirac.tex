\chapter{Dirac Equation}\label{chapter:dirac}

    References for this chapter are~\citet{van_proeyen_supergravity_2012}. (Note that the authors use the mostly-pluses signature there.) For the mathematical background of Clifford algebras and Spin groups, see \cref{chapter:clifford} and, in particular, \cref{section:spin}. For the extension to (pseudo-)Riemannian manifolds, see \cref{section:spinor_bundles}.

\section{Dirac matrices}

    \newdef{Dirac matrices}{\index{Dirac!algebra}\index{Dirac!matrix|seealso{gamma matrix}}\index{Dirac!basis}\index{Weyl!basis}\index{chiral!basis|see{Weyl basis}}
        The Dirac (or \textbf{gamma}) matrices are defined by the following equality:
        \begin{gather}
            \label{dirac:clifford_relation}
            \{\gamma^\mu,\gamma^\nu\}_+ = 2\eta^{\mu\nu}\mathbbm{1}\,,
        \end{gather}
        where $\eta^{\mu\nu}$ is the Minkowski metric. This has the form of \cref{clifford:inner_product}, i.e.~the Dirac matrices form the generating set of a Clifford algebra, called the \textbf{Dirac algebra}.

        There exist multiple distinct representations of the Clifford generators in signature $(1,3)$. The first one is called the \textbf{Dirac representation}. Here, the timelike Dirac matrix $\gamma^0$ is defined as
        \begin{gather}
            \gamma^0 :=
            \begin{pmatrix}
                \mathbbm{1}_2&0\\
                0&-\mathbbm{1}_2
            \end{pmatrix}\,.
        \end{gather}
        The spacelike Dirac matrices $\gamma^k$ ($k=1,2,3$) are defined using the Pauli matrices (\cref{angular_momentum:pauli_matrices}) $\sigma^k$:
        \begin{gather}
            \gamma^k :=
            \begin{pmatrix}
                0&\sigma^k\\
                -\sigma^k&0
            \end{pmatrix}\,.
        \end{gather}
        The \textbf{Weyl} or \textbf{chiral} representation\footnote{This representation is preferred in quantum field theory and supergravity.} is defined by replacing the timelike matrix $\gamma^0$ by
        \begin{gather}
            \gamma^0 :=
            \begin{pmatrix}
                0&\mathbbm{1}_2\\
                \mathbbm{1}_2&0
            \end{pmatrix}\,.
        \end{gather}
        In signature $(3,1)$, one obtains the Weyl representation by defining $\sigma^\mu:=(\mathbbm{1},\sigma_i)$ and $\overline{\sigma}^\mu:=\sigma_\mu$:
        \begin{gather}
            \gamma^\mu :=
            \begin{pmatrix}
                0&\sigma_\mu\\
                \overline{\sigma}_\mu&0
            \end{pmatrix}\,.
        \end{gather}
    }
    \sremark{In the remainder of this compendium, the Weyl representation will be used.}

    \newnot{Feynman slash notation}{\index{Feynman!slash}
        Let $\mathbf{a}\equiv a^\mu\mathbf{e}_\mu\in M^4$ be a general 4-vector. The Feynman slash is defined as follows:
        \begin{gather}
            \slashed{a} := a^\mu\gamma_\mu\,.
        \end{gather}
        In fact, this is just the embedding of Minkowski space in its Clifford algebra:
        \begin{gather}
            /:M^4\rightarrow C\ell(M^4,\eta):a^\mu\mathbf{e}_\mu\mapsto a^\mu\gamma_\mu\,.
        \end{gather}
    }

\section{Spinors}
\subsection{Dirac equation}

    \newformula{Dirac equation}{\index{Dirac!equation}\label{dirac:dirac_equation}
        In covariant form the Dirac equation reads as
        \begin{gather}
            (i\hbar\slashed\partial - mc)\psi = 0\,,
        \end{gather}
        where $m$ denotes the mass and $c$ denotes the speed of light.
    }

    \newdef{Dirac adjoint}{\index{Dirac!adjoint}
        \begin{gather}
            \overline{\psi} := i\psi^\dag\gamma^0
        \end{gather}
        When working in the Dirac representation, the factor $i$ should be dropped.
    }
    \newdef{Majorana adjoint}{\index{Majorana!adjoint}
        In the context of SUSY it is often convenient to work with a different adjoint spinor. Let $\mathcal{C}:=i\gamma^3\gamma^1$ denote the charge conjugation operator. The Majorana adjoint is defined by
        \begin{gather}
            \overline{\psi} := \psi^tC\,.
        \end{gather}
    }

    \newformula{Parity}{\index{parity}
        The parity operator is defined as follows:
        \begin{gather}
            \hat{P}(\psi) := \gamma^0\psi\,.
        \end{gather}
    }

\subsection{Chiral spinors}

    In an even number of dimensions, one can define an additional matrix that satisfies \cref{dirac:clifford_relation}.
    \newdef{Chiral matrix}{\index{chiral!matrix}\index{helicity}
        Assume that the dimension, given by $d=m+n$, is even. The chiral (helicity) matrix can be defined as follows:\footnote{Some authors add a constant factor to this definition.}
        \begin{gather}
            \gamma_{d+1} := \gamma_1\gamma_2\cdots\gamma_d\,.
        \end{gather}
        In an odd number of dimensions ($d=2m+1$), a generating set for the Clifford algebra can be obtained by taking the generating set from one dimension lower and adjoining the element $k\gamma_\ast$ where $k^2=(-1)^{n+d/2}$. This gives two inequivalent representations of the Clifford algebra (depending on the sign). From here on, the following redefinition will be used:
        \begin{gather}
            \gamma_{d+1}\longrightarrow k\gamma_{d+1}\,.
        \end{gather}
        This has the benefit that $\gamma_{d+1}^2 = \mathbbm{1}$.

        In $d=3+1$ one generally takes the following representation for $\gamma_5$:\footnote{Such a block diagonal form can always be chosen by working in a \textit{helicity-adapted basis}.}
        \begin{gather}
            \gamma_{d+1} :=
            \begin{pmatrix}
                \mathbbm{1}&0\\
                0&-\mathbbm{1}
            \end{pmatrix}\,.
        \end{gather}
    }

    \newdef{Chiral projection}{
        The chiral projections of a spinor $\psi$ are defined as follows:
        \begin{gather}
            \psi_L := \frac{1+\gamma_{d+1}}{2}\psi
        \end{gather}
        and
        \begin{gather}
            \psi_R := \frac{1-\gamma_{d+1}}{2}\psi\,.
        \end{gather}
        Every spinor can then be written as a sum of its chiral parts:
        \begin{gather}
            \psi = \psi_L + \psi_R.
        \end{gather}
    }

\subsection{\texorpdfstring{Dirac algebra in $d=4$}{Dirac algebra in d=4}}

    For a lot of calculations, especially in quantum electrodynamics, one needs the properties of the gamma matrices. The most relevant relations in $d=3+1$ are listed below:
    \begin{formula}[Trace algebra]
        \begin{align}
            \tr(\gamma^\mu) = \tr(\gamma^\mu\gamma^\nu\gamma^\rho) &= 0\\
            \tr(\gamma^\mu\gamma^\nu) &= 4\eta^{\mu\nu}\\
            \tr(\gamma^\mu\gamma^\nu\gamma^\kappa\gamma^\lambda) &= 4(\eta^{\mu\nu}\eta^{\kappa\lambda} - \eta^{\mu\kappa}\eta^{\nu\lambda} + \eta^{\mu\lambda}\eta^{\nu\kappa})\\
            \tr(\gamma^5) = \tr(\gamma^\mu\gamma^5) = \tr(\gamma^\mu\gamma^\nu\gamma^5) = \tr(\gamma^\mu\gamma^\nu\gamma^\rho\gamma^5)&= 0\\
            \tr(\gamma^\mu\gamma^\nu\gamma^\kappa\gamma^\lambda\gamma^5) &= -4i\varepsilon^{\mu\nu\kappa\lambda}\\
            \tr(\gamma^{\mu_1}\cdots\gamma^{\mu_k}) &= \tr(\gamma^{\mu_k}\cdots\gamma^{\mu_1})
        \end{align}
    \end{formula}

    \begin{formula}[Contraction identities]
        \begin{align}
            \gamma^\mu\gamma_\mu &= 4\\
            \gamma^\mu\gamma^\nu\gamma_\mu &= -2\gamma^\nu\\
            \gamma^\mu\gamma^\nu\gamma^\rho\gamma_\mu &= 4\eta^{\nu\rho}\\
            \gamma^\mu\gamma^\nu\gamma^\kappa\gamma^\lambda\gamma_\mu &= -2\gamma^\lambda\gamma^\kappa\gamma^\nu
        \end{align}
    \end{formula}

\subsection{Fierz identities}\index{Hierz identity}

    Using a spinor $u\in S$ and a cospinor $\overline{v}\in S^*$ one can build a bilinear form $\overline{v}u$. However, for two spinors $u,\omega$ and two cospinors $\overline{v},\overline{\rho}$ one can interpret the expression $(\overline{v}u)(\overline{\rho}\omega)$ either as a quadrilinear form on $u\otimes\overline{v}\otimes\omega\otimes\overline{\rho}$ or as a quadrilinear form on $\omega\otimes\overline{v}\otimes u\otimes\overline{\rho}$. Because $C\ell_{3,1}(\mathbb{C})$ is isomorphic to the endomorphism ring on $S$, there must exist coefficients $a^{ij}$ where $i,j=1,\ldots,2^D$ such that
    \begin{gather}
        (\overline{v}u)(\overline{\rho}\omega) = \sum_{i,j=1}^{2^d}a^{ij}(\overline{v}\gamma_i\omega)(\overline{\rho}\gamma_ju)\,.
    \end{gather}
    By using the trace orthogonality relationsv, one can find that
    \begin{gather}
        \alpha^{ij} =
        \begin{cases}
            0&i\neq j\,,\\
            \frac{1}{2^{\lfloor d/2 \rfloor}}&i=j\,.
        \end{cases}
    \end{gather}
    The above equality can then also be rewritten as follows:
    \begin{gather}
        \delta_b^a\delta_d^c = \frac{1}{2^{\lfloor d/2 \rfloor}}\sum_{i=1}^{2^d}(\gamma_i)_d^a(\gamma_i)_b^c\,.
    \end{gather}
    This expression (and the techniques used to find it) allows one to rearrange almost all multilinear expressions involving spinors and cospinors.
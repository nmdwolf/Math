\chapter{Dirac Equation}

    References for this chapter are \cite{supergravity}. (Note that one uses the mostly-pluses signature there.) For the mathematical background on Clifford algebras and Spin groups, see chapter \ref{chapter:clifford} and in particular section \ref{clifford:section:spin}.

\section{Dirac matrices}

    \begin{property}\index{Dirac!algebra}
        The Dirac matrices are defined by the following equality:
        \begin{gather}
            \label{dirac:clifford_relation}
            \{\gamma^\mu,\gamma^\nu\}_+ = 2\eta^{\mu\nu}\mathbbm{1}.
        \end{gather}
        where $\eta^{\mu\nu}$ is the Minkowski metric. This has the form of equation \ref{clifford:inner_product}. The Dirac matrices can thus be used as the generating set of a Clifford algebra\footnote{See definition \ref{clifford:clifford_algebra}.}, called the \textbf{Dirac algebra}.
    \end{property}

    \newdef{Dirac matrices\footnotemark}{\index{Dirac!matrix}\index{Dirac!basis}\index{Weyl!basis}\index{chiral!basis|see{Weyl basis}}\index{gamma matrix|see{Dirac matrix}}
        \footnotetext{Often just called the \textbf{gamma matrices}.}
        There exist multiple different representations of the Clifford generators in signature $(1, 3)$. The first one is the \textbf{Dirac representation}. Here, the timelike Dirac matrix $\gamma^0$ is defined as
        \begin{gather}
            \gamma^0 :=
            \begin{pmatrix}
                \mathbbm{1}_2&0\\
                0&-\mathbbm{1}_2
            \end{pmatrix}.
        \end{gather}
        The spacelike Dirac matrices $\gamma^k$ ($k= 1, 2, 3$) are defined using the Pauli matrices\footnote{See definition \ref{QM:angular_momentum:pauli_matrices}.} $\sigma^k$:
        \begin{gather}
            \gamma^k :=
            \begin{pmatrix}
                0&\sigma^k\\
                -\sigma^k&0
            \end{pmatrix}.
        \end{gather}

        The \textbf{Weyl} or \textbf{chiral} representation\footnote{This representation is widely used in advanced field theory and supergravity.} is defined by replacing the timelike matrix $\gamma^0$ by
        \begin{gather}
            \gamma^0 :=
            \begin{pmatrix}
                0&\mathbbm{1}_2\\
                \mathbbm{1}_2&0
            \end{pmatrix}.
        \end{gather}
        In signature $(3, 1)$ one obtains the Weyl representation by defining $\sigma^\mu:=(\mathbbm{1}, \sigma_i)$ and $\overline{\sigma}^\mu:=\sigma_\mu$:
        \begin{gather}
            \gamma^\mu :=
            \begin{pmatrix}
                0&\sigma_\mu\\
                \overline{\sigma}_\mu&0
            \end{pmatrix}.
        \end{gather}
    }
    \remark{In the remainder of this compendium we will adopt the Weyl representation.}

    \newnot{Feynman slash notation}{\index{Feynman!slash}
        Let $\mathbf{a} = a^\mu\mathbf{e}_\mu\in V$ be a general 4-vector. The Feynman slash $\slashed{a}$ is defined as follows:
        \begin{gather}
            \slashed{a} := a^\mu\gamma_\mu.
        \end{gather}
        In fact this is just a vector space morphism:
        \begin{gather}
            / : V\rightarrow C\ell(V, \eta) : a^\mu\mathbf{e}_\mu\mapsto a^\mu\gamma_\mu.
        \end{gather}
    }

\section{Spinors}
\subsection{Dirac equation}

    \newformula{Dirac equation}{\index{Dirac!equation}
        In covariant form the Dirac equation reads
        \begin{gather}
            (i\hbar\slashed\partial - mc)\psi = 0
        \end{gather}
        where $m$ is the mass of the fermion and $c$ the speed of light.
    }

    \newdef{Dirac adjoint}{\index{Dirac!adjoint}
        \begin{gather}
            \overline{\psi} := \psi^\dag\gamma^0
        \end{gather}
        When working in the Weyl representation one should add a factor $i$ to this definition.
    }
    \newdef{Majorana adjoint}{\index{Majorana!adjoint}
        In the context of SUSY it is often convenient to work with a different adjoint spinor. Let $\mathcal{C}:=i\gamma^3\gamma^1$ denote the charge conjugation operator. The Majorana adjoint is then defined by
        \begin{gather}
            \overline{\psi} := \psi^tC.
        \end{gather}
    }

    \newformula{Parity}{\index{parity}
        The parity operator is defined as follows:
        \begin{gather}
            \hat{P}(\psi) = \gamma^0\psi.
        \end{gather}
    }

\subsection{Chiral spinors}

    In even dimensions one can define an additional matrix \footnote{In $d=4$ this matrix is often denoted by $\gamma_5$.} which also satisfies equation \ref{dirac:clifford_relation}:
    \newdef{Chiral matrix}{\index{chiral!matrix}
        Assume that the dimension, given by $d=m+n$, is even. The chiral matrix can then be defined as follows:\footnote{Some authors add a constant to this definition.}
        \begin{gather}
            \gamma_\ast := \gamma_0\gamma_1...\gamma_{d-1}.
        \end{gather}
        In odd dimensions ($d=2m+1$) a generating set for the Clifford algebra can be obtained by taking the generating set from one dimension lower and adjoining the element $k\gamma_\ast$ where $k^2=(-1)^{n+d/2}$. This gives two inequivalent representations of the Clifford algebra (depending on the sign).

        In $d=3+1$ one generally takes the following representation for $\gamma_\ast$:
        \begin{gather}
            \gamma_\ast :=
            \begin{pmatrix}
                \mathbbm{1}&0\\
                0&-\mathbbm{1}
            \end{pmatrix}.
        \end{gather}
    }

    \newdef{Chiral projection}{
        The chiral projections of a spinor $\psi$ are defined as follows:
        \begin{gather}
            \psi_L := \frac{1+\gamma_\ast}{2}\psi
        \end{gather}
        and
        \begin{gather}
            \psi_R := \frac{1-\gamma_\ast}{2}\psi.
        \end{gather}
        Every spinor can then be written as a sum of its chiral parts:
        \begin{gather}
            \psi = \psi_L + \psi_R.
        \end{gather}
    }

\subsection{\texorpdfstring{Dirac algebra in $D=4$}{Dirac algebra in D=4}}

    For a lot of calculations, especially in quantum electrodynamics, one needs the properties of the gamma matrices. Therefore we list the most relevant relations in $D=3+1$:
    \begin{formula}[Trace algebra]
        \begin{align}
            \text{tr}(\gamma^\mu) = \text{tr}(\gamma^\mu\gamma^\nu\gamma^\rho) &= 0\\
            \text{tr}(\gamma^\mu\gamma^\nu) &= 4\eta^{\mu\nu}\\
            \text{tr}(\gamma^\mu\gamma^\nu\gamma^\kappa\gamma^\lambda) &= 4(\eta^{\mu\nu}\eta^{\kappa\lambda} - \eta^{\mu\kappa}\eta^{\nu\lambda} + \eta^{\mu\lambda}\eta^{\nu\kappa})\\
            \text{tr}(\gamma^5) = \text{tr}(\gamma^\mu\gamma^5) = \text{tr}(\gamma^\mu\gamma^\nu\gamma^5) = \text{tr}(\gamma^\mu\gamma^\nu\gamma^\rho\gamma^5)&= 0\\
            \text{tr}(\gamma^\mu\gamma^\nu\gamma^\kappa\gamma^\lambda\gamma^5) &= -4i\varepsilon^{\mu\nu\kappa\lambda}\\
            \text{tr}(\gamma^{\mu_1}\cdots\gamma^{\mu_k}) &= \text{tr}(\gamma^{\mu_k}\cdots\gamma^{\mu_1})
        \end{align}
    \end{formula}

    \begin{formula}[Contraction identities]
        \begin{align}
            \gamma^\mu\gamma_\mu &= 4\\
            \gamma^\mu\gamma^\nu\gamma_\mu &= -2\gamma^\nu\\
            \gamma^\mu\gamma^\nu\gamma^\rho\gamma_\mu &= 4\eta^{\nu\rho}\\
            \gamma^\mu\gamma^\nu\gamma^\kappa\gamma^\lambda\gamma_\mu &= -2\gamma^\lambda\gamma^\kappa\gamma^\nu
        \end{align}
    \end{formula}

\subsection{Fierz identities}

    Using a spinor $u$ and a cospinor $\overline{v}$ one can build a bilinear form $\overline{v}u$. ?? COMPLETE ??
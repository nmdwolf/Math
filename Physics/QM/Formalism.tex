\chapter{Mathematical Formalism}\label{chapter:mathematical_formalism_qm}

\section{Postulates}

?? FINISH THIS ??

\subsection{Postulate 6: Eigenfunction expansion}

    \newdef{Observable}{\index{observable}
        An operator $\hat{A}$ which possesses a complete set of eigenfunctions is called an observable.
    }

    \begin{formula}\index{eigenfunction!expansion}
        Let $|\Psi\rangle$ be an arbitrary wave function representing the system. Let the set $\{|\psi_n\rangle\}_{i\in I}$ be a complete set of eigenfunctions of an observable of the system. The wave function $|\Psi\rangle$ can be expanded as a linear combination of those eigenfunctions:
        \begin{gather}
            \label{qm_formalism:eigenfunction_expansion}
            |\Psi\rangle = \sum_nc_n|\psi_n\rangle + \int c_a|\psi_a\rangle da
        \end{gather}
        where the summation ranges over the discrete spectrum and the integral over the continuous spectrum.
    \end{formula}

    \begin{formula}[Closure relation]\index{closure!relation}\index{resolution!of the identity}
        For a complete set of discrete eigenfunctions the closure relation\footnote{This relation is also called the \textbf{resolution of the identity}.} reads:
        \begin{gather}
            \label{qm_formalism:closure}
            \sum_n|\psi_n\rangle\langle\psi_n| = \mathbbm{1}.
        \end{gather}
        For a complete set of continuous eigenfunctions we have the following counterpart:
        \begin{gather}
            \label{qm_formalism:closure_continuouos}
            \int|i\rangle\langle i|di = \mathbbm{1}.
        \end{gather}
        For a mixed set of eigenfunctions a similar relation is obtained by summing over the discrete eigenfunctions and integrating over the continuous eigenfunctions.
    \end{formula}
    \sremark{To simplify the notation we will almost always use the notation of equation \ref{qm_formalism:closure} but implicitly integrate over the continuous spectrum.}

\section{Uncertainty principle}

    \newdef{Compatible observables}{\index{observable!compatibile observables}
        Let $\hat{A}, \hat{B}$ be two observables. If there exists a complete set of functions $|\psi_n\rangle$ that are eigenfunctions of both $\hat{A}$ and $\hat{B}$ then the two operators are said to be compatible.
    }

    \newformula{Uncertainty relation}{\index{Heisenberg!uncertainty relation}
        Let $\hat{A}, \hat{B}$ be two observables and let $\Delta A, \Delta B$ be the corresponding uncertainties. The (Robertson) uncertainty relation reads
        \begin{gather}
            \label{qm_formalism:uncertainty_relation}
            \Delta A\Delta B \geq \stylefrac{1}{4}\left|\left\langle\left[\hat{A}, \hat{B}\right]\right\rangle\right|^2.
        \end{gather}
    }

\section{Matrix representation}

    \begin{formula}
        The following formula gives the matrix representation of $\hat{A}$ with respect to the orthonormal basis $\{\psi_n\}$:
        \begin{gather}
            \label{qm_formalism:matrix_entry}
            A_{mn} = \langle\psi_m|\hat{A}|\psi_n\rangle.
        \end{gather}
    \end{formula}
    \begin{remark}
         The basis $\{\psi_n\}$ need not consist out of eigenfunctions of $\hat{A}$.
    \end{remark}

\section{Slater determinants}

    \begin{theorem}[Symmetrization postulate]\index{symmetrization postulate}
        Let $\mathcal{H}$ be the Hilbert space belonging to a single particle. A system of $n$ identical particles is described by a wave function $\Psi$ belonging to either $S^n(\mathcal{H})$ or $\Lambda^n(\mathcal{H})$.
    \end{theorem}
    \begin{remark}
        In ordinary quantum mechanics this is a postulate, but in quantum field theory this is a consequence of the spin-statistics theorem of Pauli.
    \end{remark}

    \begin{formula}
        Consider the permutation group $\{\sigma\}\equiv S_n$ on $n$ elements. Let $|\psi\rangle$ be a single-particle wave function. Bosonic systems are described by a wave function of the form
        \begin{gather}
            |\Psi_B\rangle = \sum_{\sigma}|\psi_{\sigma(1)}\rangle\cdots|\psi_{\sigma(n)}\rangle.
        \end{gather}
        Fermionic systems are described by a wave function of the form
        \begin{gather}
            |\Psi_F\rangle = \sum_{\sigma}\sgn(\sigma)|\psi_{\sigma(1)}\rangle\cdots|\psi_{\sigma(n)}\rangle.
        \end{gather}
    \end{formula}

    \newdef{Slater determinant}{\index{Slater!determinant}\index{spin!orbitals}\index{permanent}
        Let $\{\phi_i(\vector{q})\}_{i\leq N}$ be a set of wave functions, called \textbf{spin orbitals}, describing a system of $N$ identical fermions. The totally antisymmetric wave function of the system is given by\footnote{For bosonic systems one can define a similar function using the concept of \textit{permanents}.}
        \begin{gather}
            \label{qm_formalism:slater_determinant}
            \psi(\vector{q}_1, ..., \vector{q}_N) = \frac{1}{\sqrt{N!}}\det\left(
            \begin{array}{ccc}
                    \phi_1(\vector{q}_1)&\cdots&\phi_N(\vector{q}_1)\\
                    \vdots&\ddots&\vdots\\
                    \phi_1(\vector{q}_N)&\cdots&\phi_N(\vector{q}_N)
            \end{array}
            \right).
         \end{gather}
    }

\section{Interaction picture}\index{interaction!picture}\label{qm:interaction_picture}

    Let $\hat{H}(t) = \hat{H}_0 + \hat{V}(t)$ be the total Hamiltonian of a system where $\hat{V}(t)$ is the interaction potential. Let $|\psi(t)\rangle$ and $\hat{O}$ denote a state and operator in the Schr\"odinger picture.
    \begin{formula}
        In the interaction picture we define the state vector as follows:
        \begin{gather}
            |\psi(t)\rangle_I = e^{\frac{i}{\hbar}\hat{H}_0t}|\psi(t)\rangle.
        \end{gather}
        From this it follows that the operators in the interaction picture are given by
        \begin{gather}
            \hat{O}_I(t) = e^{\frac{i}{\hbar}\hat{H}_0t}\hat{O}e^{-\frac{i}{\hbar}\hat{H}_0t}.
        \end{gather}
    \end{formula}
    \newformula{Schr\"odinger equation}{
        Using the previous formulas, the Schr\"odinger equation can be rewritten as follows:
        \begin{gather}
            i\hbar\deriv{}{t}|\psi(t)\rangle_I = \hat{V}_I(t)|\psi(t)\rangle_I.
        \end{gather}
        The time-evolution of operators in the interaction picture is given by
        \begin{gather}
            \deriv{}{t}\hat{O}_I(t) = \frac{i}{\hbar}\left[\hat{H}_0, \hat{O}_I(t)\right].
        \end{gather}
    }

    \begin{theorem}[Adiabatic theorem]\index{adiabatic!theorem}
        If a perturbation is acting slowly enough on a system such that the system can adapt its configuration at every single moment then the system will remain in the same eigenstate.
    \end{theorem}

\section{Quantum mechanics on curved backgrounds}

    Using the tools of distribution theory and differential geometry, as presented in chapters \ref{chapter:distributions}, \ref{chapter:bundles} and onward, we can introduce quantum mechanics on curved backgrounds\footnote{That is space, not spacetime!}. The main reference for this chapter is \cite{schuller}.

    \begin{remark}
        A first important remark that we have to make is that the classical definition of the wave function as an element of $L^2(\mathbb{R}^d, \mathbb{C})$ is not sufficient, even in flat Cartesian space. A complete description requires the introduction of the so-called \textit{Gelfand triple}\footnote{Often called a \textit{rigged Hilbert space}.} where we replace the space of square-integrable functions by the Schwartz space\footnote{See definition \ref{distribution:schwartz_space}.} of rapidly decreasing functions. The linear functionals on this space are then given by the tempered distributions.
    \end{remark}

    \begin{construct}
        When working on curved spaces (or even in non-Cartesian coordinates on flat space) one can encounter problems with the definition of the self-adjoint operators $\hat{q}^i$ and $\hat{p}_i$. The naive definition $\hat{q}^i = q^i, \hat{p}_i = -i\partial_i$ gives rise to extra terms when calculating inner products that break the canonical commutation relations and the self-adjointness of the operators (e.g. the angular position operator $\hat{\varphi}$ on the circle together with its conjugate $\hat{L}$).

        An elegant solution to this problem is found by giving up the definition of the wave function as a function $\psi:\mathbb{R}^d\rightarrow\mathbb{C}$. Assume that we are working on a Riemannian manifold $(M, g)$ and that our ''naive'' wave functions take values in a vector space $V$. We then construct a vector bundle $E$ with typical fibre $V$ over $M$. On the frame bundle $F(E)$ associated to this vector bundle there lives an Ehresmann connection $\omega$ that we can use to define a (local) covariant derivative $\nabla$. The ''true'' wave function is now defined as a map $\Psi: F(E)\rightarrow V$ or locally as the pullback $\psi := \varphi^*\Psi$ for some local section $\varphi:U\subseteq M\rightarrow F(E)$.

        Now we introduce a general inner product
        \begin{gather}
            \langle\psi, \phi\rangle := \int d^dx\sqrt{\det(g)}\psi^*(x)\phi(x).
        \end{gather}
        Because the factor $\sqrt{\det(g)}$ transforms in the inverse manner of the measure $d^dx$, the integrand is invariant under coordinate transforms (which is something we generally require of our physical laws). Using this new inner product we can check the self-adjointness of the momentum operator $\hat{P}_i := -i\nabla_i$:
        \begin{align*}
            \langle\psi, \hat{P}_i\phi\rangle = &\int d^dx\sqrt{\det(g)}\psi^*(x)(-i\nabla_i)\phi(x)\\
            \overset{\ref{diff:prin:local_covariant_derivative}}{=} &\int d^dx\sqrt{\det(g)}\psi^*(x)(-i\partial_i - i\omega_i)\phi(x)\\
            = &i\int d^dx\left(\partial_i\sqrt{\det(g)}\right)\psi^*(x)\phi(x) + \int d^dx\sqrt{\det(g)}(-i\partial_i\psi)^*(x)\phi(x)\\
                &\hspace{2cm} -i \int d^dx\sqrt{\det(g)}\psi^*(x)\omega_i\phi(x)\\
            = &\langle\hat{P}_i\psi, \phi\rangle -i \int d^dx\sqrt{\det(g)}\psi^*(x)\omega_i^*\phi(x)\\
                &\hspace{2cm} + i\int d^dx\left(\partial_i\sqrt{\det(g)}\right)\psi^*(x)\phi(x) -i \int d^dx\sqrt{\det(g)}\psi^*(x)\omega_i\phi(x).
        \end{align*}
        Self-adjointness then requires that
        \begin{gather}
            \sqrt{\det(g)}(\omega_i + \omega_i^*) = \partial_i\sqrt{\det(g)}
        \end{gather}
        or using the identity $(\ln f)' = \frac{f'}{f}$:
        \begin{gather}
            2\text{Re}(\omega_i) = \partial_i\ln\left(\sqrt{\det(g)}\right).
        \end{gather}
    \end{construct}

    ?? COMPLETE ??

\section{Geometric Quantization}\label{section:geometric_quantization}
\subsection{Prequantization}

    \newdef{Prequantum line bundle}{\index{prequantum line bundle}
        Consider a classical phase space $M$ equipped with its symplectic form $\omega$. A prequantum line bundle for this manifold is a line bundle equipped with a connection $\nabla$ such that $\omega=F_\nabla$ (where $F$ denotes the curvature 2-form).
    }
    \begin{property}\index{Weil!integrality}\index{Dirac!quantization condition}
        Such a line bundle exists if and only if the symplectic form is integral\footnote{See definition \ref{diff:integral_form}.}. For simply-connected manifolds this is equivalent to
        \begin{gather}
            \int_S\omega \in 2\pi\mathbb{Z}
        \end{gather}
        for every 2-cycle $S\subset M$. This condition resembles the ''old'' \textit{Bohr-Sommerfeld condition} and is in general known as the \textbf{Weil integrality condition}.

        One can also derive the Dirac quantization condition from Weil integrality. If we couple the system to a gauge potential then the connection form should contain the gauge field and hence the curvature is transformed as $\omega\longrightarrow\omega + eF_{\text{EM}}$. Weil integrality then implies that $e$ is an integer.
    \end{property}

    \newdef{Prequantum Hilbert space}{
        Consider a symplectic manifold $(M, \omega)$ together with its prequantum line bundle $(L, \nabla)$. The prequantum Hilbert space $\mathcal{H}_{\mathbb{P}}$ is defined as the space of square-integrable sections of $L$ (with respect to the volume form induced by $\omega$).
    }

    \newdef{Quantization}{\index{quantization}
        Consider a symplectic manifold $(M, \omega)$. A quantization of $(M, \omega)$ is given by a prequantum line bundle $(L, \nabla)$ together with a polarization $P$ of $M$. The quantum state space $\mathcal{H}$ is given by the subspace of $\mathcal{H}_{\mathbb{P}}$ of those sections that are covariantly constant along $P$, i.e. those sections $s\in\Gamma(L)$ such that $\nabla_Xs=0$ for all $X\in P$.
    }

    \newdef{Prequantum operator}{
        Consider a symplectic manifold $(M, \omega)$ together with its prequantum line bundle $(L, \nabla)$. To every smooth function $f\in C^\infty(M,\mathbb{C})$ we associate a (prequantum) operator $\hat{f}:\Gamma(L)\rightarrow\Gamma(L)$ by the following formula:
        \begin{gather}
            \hat{f}:\psi\rightarrow -i\nabla_{X^f}\,\psi + f\cdot\psi
        \end{gather}
        where $X^f$ is the Hamiltonian vector field associated to $f$ (see definition \ref{diff:hamilton_vectorfield}) and $\nabla = d - i\theta$ with $\theta$ the Liouville 1-form.

        If this operator ought to preserve the polarization of $M$ then we should have \[\nabla_Xs=0\implies\nabla_X(\hat{f}s)=0\] for all sections $s\in\Gamma(L)$ and $X\in P$. Using the general formula for second covariant derivatives and the fact that the leaves of $P$ are Lagrangian we find the following condition:
        \begin{gather}
            [X, X^f]=0
        \end{gather}
        for all $X\in P$.
    }
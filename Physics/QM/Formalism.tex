\chapter{Mathematical Formalism}

\section{Postulates}
\subsection{Postulate 6: eigenfunction expansion}

	\newdef{Observable}{\index{observable}
    		An operator $\hat{A}$ which possesses a complete set of eigenfunctions is called an observable.
	}
    
	\begin{formula}\index{eigenfunction!expansion}
    		Let $|\Psi\rangle$ be an arbitrary wavefunction representing the system. Let the set $\{|\psi_n\rangle\}$ be a complete set of eigenfunctions of an observable of the system. The wavefunction $|\Psi\rangle$ can then be expanded as a linear combination of those eigenfunctions:
		\begin{equation}
	        	\label{qm_formalism:eigenfunction_expansion}
			\boxed{|\Psi\rangle = \sum_nc_n|\psi_n\rangle + \int c_a|\psi_a\rangle da}
		\end{equation}
	        where the summation ranges over the discrete spectrum and the integral over the continuous spectrum.
	\end{formula}
   
	\begin{formula}[Closure relation]\index{closure!relation}\index{resolution!of the identity}
	    	For a complete set of discrete eigenfunctions the closure relation\footnotemark\ reads:
		\begin{equation}
		        \label{qm_formalism:closure}
			\sum_n|\psi_n\rangle\langle\psi_n| = \mathbbm{1}
		\end{equation}
	        For a complete set of continuous eigenfunctions we have the following counterpart:
	        \begin{equation}
		        \label{qm_formalism:closure_continuouos}
			\int|i\rangle\langle i|di = \mathbbm{1}
		\end{equation}
	        For a mixed set of eigenfunctions a similar relation is obtained by summing over the discrete eigenfunctions and integrating over the continuous eigenfunctions.
	        \footnotetext{This relation is also called the \textbf{resolution of the identity}.}
	\end{formula}
	\sremark{To simplify the notation we will almost always use the notation of equation \ref{qm_formalism:closure} but implicitly integrate over the continuous spectrum.}

\section{Uncertainty relations}
	
	\newdef{Commutator}{\index{commutator}
    		Let $\hat{A}, \hat{B}$ be two operators. We define the commutator of $\hat{A}$ and $\hat{B}$ as follows:
    		\begin{equation}
			\label{qm_formalism:commutator}
        		\boxed{\comm{A}{B} = \hat{A}\hat{B} - \hat{B}\hat{A}}
		\end{equation}
	}
	\begin{formula}
	        \begin{equation}
			\label{qm_formalism:commutator_left}
		        \left[\hat{A}\hat{B}, \hat{C}\right] = \hat{A}\comm{B}{C} + \comm{A}{C}\hat{B}
		\end{equation}
	\end{formula}
    
	\newdef{Anticommutator}{\index{anticommutator}
	    	Let $\hat{A}, \hat{B}$ be two operators. We define the anticommutator of $\hat{A}$ and $\hat{B}$ as follows:
	    	\begin{equation}
			\label{qm_formalism:anticommutator}
		        \boxed{\left\{\hat{A},\hat{B}\right\}_+ = \hat{A}\hat{B} + \hat{B}\hat{A}}
		\end{equation}
	}
    
	\newdef{Compatible observables}{\index{observable!compatibile observables}
	    	Let $\hat{A}, \hat{B}$ be two observables. If there exists a complete set of functions $|\psi_n\rangle$ that are eigenfunctions of both $\hat{A}$ and $\hat{B}$ then the two operators are said to be compatible.
	}

	\newformula{Heisenberg uncertainty relation}{\index{Heisenberg!uncertainty relation}
	    	Let $\hat{A}, \hat{B}$ be two observables. Let $\Delta A, \Delta B$ be the corresponding uncertainties.
	    	\begin{equation}
			\label{qm_formalism:uncertainty_relation}
		        \boxed{\Delta A\Delta B = \stylefrac{1}{4}\left|\left\langle\left[\hat{A}, \hat{B}\right]\right\rangle\right|^2}
		\end{equation}
	}

\section{Matrix representation}

	\begin{formula}
	    	The following formula gives the $(m,n)$-th element of the matrix representation of $\hat{A}$ with respect to the orthonormal basis $\{\psi_n\}$:
		\begin{equation}
			\label{qm_formalism:matrix_entry}
		        \boxed{A_{mn} = \langle\psi_m|\hat{A}|\psi_n\rangle}
		\end{equation}
	\end{formula}
	\begin{remark}
		 The basis $\{\psi_n\}$ need not consist out of eigenfunctions of $\hat{A}$.
	\end{remark}
    
\section{Slater determinants}

	\begin{theorem}[Symmetrization postulate]\index{symmetrization postulate}
	    	Let $\mathcal{H}$ be the Hilbert space belonging to a single particle. A system of $n$ identical particles is described by a wave function $\Psi$ belonging to either $S^n(\mathcal{H})$ or $\Lambda^n(\mathcal{H})$.
	\end{theorem}
	\begin{remark}
	    	In ordinary quantum mechanics this is a postulate, but in quantum field theory this is a consequence of the spin-statistics theorem of Pauli.
	\end{remark}
    
	\begin{formula}
	    	Let $\{\sigma\}$ be the set of all permutations of the sequence $(1, ..., n)$. Let $|\psi\rangle$ be the single-particle wave function. Bosonic systems are described by a wave function of the form
	        \begin{equation}
	        	|\Psi_B\rangle = \sum_{\sigma}|\psi_{\sigma(1)}\rangle\cdots|\psi_{\sigma(n)}\rangle
	        \end{equation}
	        Fermionic systems are described by a wave function of the form
	        \begin{equation}
	        	|\Psi_F\rangle = \sum_{\sigma}\sgn(\sigma)|\psi_{\sigma(1)}\rangle\cdots|\psi_{\sigma(n)}\rangle
	        \end{equation}
	\end{formula}

	\newdef{Slater determinant}{\index{Slater!determinant}\index{spin!orbitals}
	    	Let $\{\phi_i(\vector{q})\}_{i\leq N}$ be a set of wave functions, called \textbf{spin orbitals}, describing a system of $N$ identical fermions. The totally antisymmetric wave function of the system is given by:
	        \begin{equation}
	        	\label{qm_formalism:slater_determinant}
		        \boxed{\psi(\vector{q}_1, ..., \vector{q}_N) = \frac{1}{\sqrt{N!}}\det\left(
		        \begin{array}{ccc}
            			\phi_1(\vector{q}_1)&\cdots&\phi_N(\vector{q}_1)\\
                		\vdots&\ddots&\vdots\\
                		\phi_1(\vector{q}_N)&\cdots&\phi_N(\vector{q}_N)
		        \end{array}
		        \right)}
     		\end{equation}
	}
	
\section{Interaction picture}\index{interaction!picture}\label{qm:interaction_picture}

	Let $\hat{H}(t) = \hat{H}_0 + \hat{V}(t)$ be the total Hamiltonian of a system where $\hat{V}(t)$ is the interaction potential. Let $|\psi(t)\rangle$ and $\hat{O}$ denote a state and operator in the Schr\"odinger picture.
	\begin{formula}
		In the interaction picture we define the state vector as follows:
		\begin{equation}
			|\psi(t)\rangle_I = e^{\frac{i}{\hbar}\hat{H}_0t}|\psi(t)\rangle
		\end{equation}
		From this it follows that the operators in the interaction picture are given by:
		\begin{equation}
			\hat{O}_I(t) = e^{\frac{i}{\hbar}\hat{H}_0t}\hat{O}e^{-\frac{i}{\hbar}\hat{H}_0t}
		\end{equation}
	\end{formula}
	\newformula{Schr\"odinger equation}{
		Using the previous formulas, the Schr\"odinger equation can be rewritten as follows:
		\begin{equation}
			i\hbar\deriv{}{t}|\psi(t)\rangle_I = \hat{V}_I(t)|\psi(t)\rangle_I
		\end{equation}
		The time-evolution of operators in the interaction picture is given by:
		\begin{equation}
			\deriv{}{t}\hat{O}_I(t) = \frac{i}{\hbar}\left[\hat{H}_0, \hat{O}_I(t)\right]
		\end{equation}
	}
	
\subsection{Adiabatic switching}

	\begin{theorem}[Adiabatic theorem]\index{adiabatic!theorem}
		If a perturbation is acting slowly enough on a system such that the system can adapt its configuration at every single moment then the system will remain in the same eigenstate.
	\end{theorem}
	
\section{Quantum mechanics on curved space}

	Using the tools of differential geometry, as presented in chapters \ref{diff:chapter:bundles} and onward, we can introduce quantum mechanics on curved space\footnote{Not space-time!}. We heavily used the course material from \cite{schuller}.

	\begin{remark}
		A first important remark that we have to make is that the classical definition of the wave function as an element of $L^2(\mathbb{R}^d, \mathbb{C})$ is not sufficient, even in flat cartesian space. A complete description requires the introduction of the so-called \textit{Gelfand triple}\footnote{See also \textit{rigged Hilbert space}.} where we replace the space of square-integrable functions by the Schwartz space\footnote{See definition \ref{distribution:schwartz_space}.} of rapidly decreasing functions. The linear functionals on this space are then given by the \textit{tempered distributions}.
	\end{remark}

	\begin{construct}
		When working on curved spaces (or even in non-cartesian coordinates on flat space) there arise problems with the definition of the self-adjoint operators $\hat{q}^i$ and $\hat{p}_i$. The naive definition $\hat{q}^i = q^i, \hat{p}_i = -i\partial_i$ gives rise to extra terms when calculating inner products that break the canonical commutation relations and the self-adjointness of the operators.
	
		An elegant solution to this problem is found by giving up the definition of the wave function as a function $\psi:\mathbb{R}^d\rightarrow\mathbb{C}$. Assume that we are working on a Riemannian base manifold $(M, g)$ and that our 'naive' wave functions where living in a vector space $V$. We then construct a vector bundle $E$ with typical fibre $V$ over $M$. Associated to this vector bundle we then find a frame bundle $F(E)$ and an Ehresmann connection $\omega$ that we can use to define a (local) covariant derivative $\nabla$. The wave function is now defined as a map $\Psi: F(E)\rightarrow V$ or locally as the pullback $\psi := \varphi^*\Psi$ for some local section $\varphi:U\subseteq M\rightarrow F(E)$.
		
		First we introduce the general inner product
		\begin{equation}
			\langle\psi, \phi\rangle = \int d^dx\sqrt{\det(g)}\psi^*(x)\phi(x)
		\end{equation}
		Because the factor $\sqrt{\det(g)}$ transforms in the inverse manner of the measure $d^dx$, the integrand is invariant under coordinate transforms which is something we generally require of our physical laws. Using this new inner product we can check the self-adjointness of the new momentum operator $\hat{P}_i = -i\nabla_i$:
		\begin{align*}
			\langle\psi, \hat{P}_i\phi\rangle = &\int d^dx\sqrt{\det(g)}\psi^*(x)(-i\nabla_i)\phi(x)\\
			\overset{\ref{diff:prin:local_covariant_derivative}}{=} &\int d^dx\sqrt{\det(g)}\psi^*(x)(-i\partial_i - i\omega_i)\phi(x)\\
			= &i\int d^dx\left(\partial_i\sqrt{\det(g)}\right)\psi^*(x)\phi(x) + \int d^dx\sqrt{\det(g)}(-i\partial_i\psi)^*(x)\phi(x)\\
				&\hspace{2cm} -i \int d^dx\sqrt{\det(g)}\psi^*(x)\omega_i\phi(x)\\
			= &\langle\hat{P}_i\psi, \phi\rangle -i \int d^dx\sqrt{\det(g)}\psi^*(x)\omega_i^*\phi(x)\\
				&\hspace{2cm} + i\int d^dx\left(\partial_i\sqrt{\det(g)}\right)\psi^*(x)\phi(x) -i \int d^dx\sqrt{\det(g)}\psi^*(x)\omega_i\phi(x)
		\end{align*}
		Self-adjointness then requires that
		\begin{equation}
			\sqrt{\det(g)}(\omega_i + \omega_i^*) = \partial_i\sqrt{\det(g)}
		\end{equation}
		or using the well known identity $(\ln f)' = \frac{f'}{f}$:
		\begin{equation}
			2\text{Re}(\omega_i) = \partial_i\ln\left(\sqrt{\det(g)}\right)
		\end{equation}
	\end{construct}
	
\section{Geometric Quantization}

\subsection{Prequantization}\label{section:geometric_quantization}

	\newdef{Prequantum line bundle}{\index{prequantum line bundle}
		Consider a classical phase space $M$ equipped with its symplectic form $\omega$. A prequantum line bundle for this manifold is a line bundle equipped with a connection $\nabla$ such that $\omega=F_\nabla$ (where $F$ denotes the curvature 2-form).
	}
	\begin{remark}\index{Weil integrality}\index{Dirac!quantization condition}
		Such a line bundle exist if and only if the symplectic form is integral\footnote{See definition \ref{diff:integral_form}.}. For simply connected manifolds this is equivalent to
		\begin{gather}
			\int_S\omega \in 2\pi\mathbb{Z}
		\end{gather}
		for every 2-cycle $S$ in $M$ which resembles the ''old'' \textit{Bohr-Sommerfeld condition}. This condition is known as the \textbf{Weil integrality condition}.
		
		One can also derive the Dirac quantization condition from Weil integrality. If we couple the system to a gauge potential then the connection form should contain the gauge field and hence the curvature is transformed as $\omega\longrightarrow\omega + eF_{\text{EM}}$. Weil integrality then implies that $e$ is an integer.
	\end{remark}
	
	\newdef{Prequantum Hilbert space}{
		Consider a symplectic manifold $(M, \omega)$ together with its prequantum line bundle $(L, \nabla)$. The prequantum Hilbert space $\mathcal{H}_{\mathbb{P}}$ is defined as the space of square-integrable sections of $L$.
	}
	
	\newdef{Quantization}{\index{quantization}
		Consider a symplectic manifold $(M, \omega)$. A quantization of $(M, \omega)$ is given by a prequantum line bundle $(L, \nabla)$ together with a polarization $P$ of $M$. The quantum state space $\mathcal{H}$ is given by the subspace of $\mathcal{H}_{\mathbb{P}}$ of those sections that are covariantly constant along $P$, i.e. those sections $s\in\Gamma(L)$ such that $\nabla_Xs=0$ for all $X\in P$.
	}
	
	\newdef{Prequantum operator}{
		Consider a symplectic manifold $(M, \omega)$ together with its prequantum line bundle $(L, \nabla)$. To every smooth function $f\in C^\infty(M,\mathbb{C})$ we associate a (prequantum) operator $\hat{f}:\Gamma(L)\rightarrow\Gamma(L)$ by the following formula:
		\begin{gather}
			\hat{f}:\psi\rightarrow -i\nabla_{X^f}\,\psi + f\cdot\psi
		\end{gather}
		where $X^f$ is the Hamiltonian vector field associated to $f$ (see definition \ref{diff:hamilton_vectorfield}) and $\nabla = d - i\theta$ with $\theta$ the Liouville 1-form.
		
		If this operator ought to preserve the polarization of $M$ then we should have $\nabla_Xs=0\implies\nabla_X(\hat{f}s)=0$ for all sections $s\in\Gamma(L)$ and $X\in P$. Using the general formula for second covariant derivatives and the fact that the leaves are Lagrangian we find the following condition:
		\begin{gather}
			[X, X^f]=0
		\end{gather}
		for all $X\in P$.
	}

\subsection{K\"ahler quantization}

	

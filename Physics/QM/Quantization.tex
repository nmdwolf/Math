\chapter{\difficult{Quantization}}

\section{Geometric Quantization}\label{section:geometric_quantization}
\subsection{Prequantization}

    \newdef{Prequantum line bundle}{\index{prequantum line bundle}
        Consider a classical phase space $M$ equipped with its symplectic form $\omega$. A prequantum line bundle for this manifold is a line bundle equipped with a connection $\nabla$ such that $\omega=F_\nabla$ where $F_\nabla$ denotes the curvature two-form associated to $\nabla$.
    }
    \begin{property}\index{Weil!integrality}\index{Dirac!quantization condition}
        A prequantum line bundle exists if and only if the symplectic form is integral \ref{diff:integral_form}. For simply-connected manifolds this is equivalent to
        \begin{gather}
            \int_S\omega \in 2\pi\mathbb{Z}
        \end{gather}
        for every 2-cycle $S\subset M$. This condition resembles the ''old'' \textit{Bohr-Sommerfeld condition} and is in general known as the \textbf{Weil integrality condition}.
    \end{property}
    \begin{result}[Dirac quantization condition]
        One can also derive the Dirac quantization condition from Weil integrality. If one couples the system to a gauge potential, the connection form should contain the gauge field. This implies that the curvature is transformed as $\omega\longrightarrow\omega + eF_{\text{EM}}$. Weil integrality then implies that $e$ is an integer.
    \end{result}

    \newdef{Prequantum Hilbert space}{
        Consider a symplectic manifold $(M,\omega)$ together with its prequantum line bundle $(L,\nabla)$. The prequantum Hilbert space $\mathcal{H}_{\mathbb{P}}$ is defined as the space of square-integrable sections of $L$ (with respect to the volume form induced by $\omega$).
    }

    \newdef{Quantization}{\index{quantization}
        Consider a symplectic manifold $(M,\omega)$. A quantization of $(M,\omega)$ is given by a prequantum line bundle $(L,\nabla)$ together with a polarization $P$ of $M$. The quantum state space $\mathcal{H}$ is given by the subspace of $\mathcal{H}_{\mathbb{P}}$ of those sections that are covariantly constant along $P$, i.e. those sections $s\in\Gamma(L)$ that satisfy $\nabla_Xs=0$ for all $X\in P$.
    }

    \newdef{Prequantum operator}{
        Consider a symplectic manifold $(M,\omega)$ together with its prequantum line bundle $(L,\nabla)$. To every smooth function $f\in C^\infty(M,\mathbb{C})$ one can associate a (\textbf{Kostant-Souriau}) prequantum operator $\hat{f}:\Gamma(L)\rightarrow\Gamma(L)$ by the following formula:
        \begin{gather}
            \hat{f}:\psi\rightarrow -i\nabla_{X_f}\,\psi + f\cdot\psi
        \end{gather}
        where $X_f$ is the Hamiltonian vector field \ref{symplectic:hamilton_vectorfield} associated to $f$ and $\nabla = d - i\theta$ with $\theta$ the Liouville one-form.

        If these operator ought to represent genuine operators on the quantum state space, one should have \[\nabla_Xs=0\implies\nabla_X(\hat{f}s)=0\] for all sections $s\in\Gamma(L)$ and $X\in P$. Using the general formula for second covariant derivatives and the fact that the leaves of $P$ are Lagrangian, one finds the following condition:
        \begin{gather}
            [X,X_f]=0
        \end{gather}
        for all $X\in P$.
    }
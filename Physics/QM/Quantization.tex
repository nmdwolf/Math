\chapter{\difficult{Quantization}}

    Given the content of chapter \ref{chapter:mathematical_formalism_qm}, the general quantization procedure can be axiomatized as follows:

    \begin{method}[Abstract quantization]
        A quantization of a symplectic manifold $(M,\omega)$ is a pair $(\mathcal{H},\mathcal{O})$ such that:
        \begin{enumerate}
            \item $\mathcal{H}$ is a separable (complex) Hilbert space.
            \item $\mathcal{O}$ takes real functions $C^\infty(M)$ to self-adjoint operators.
            \item $\mathcal{O}$ is $\mathbb{C}$-linear.
            \item $\mathcal{O}(1) = \mathbbm{1}_{\mathcal{H}}$.
            \item \textbf{Dirac correspondence}: $[\mathcal{O}(f),\mathcal{O}(g)] = i\hbar\mathcal{O}(\{f,g\})$.
        \end{enumerate}
    \end{method}

    Because a Hilbert space forms an irreducible representation of any complete set of observables, it makes sense to additionally require the following axiom:
    \begin{axiom}[Irreducibility postulate]
        Let $(M,\omega)$ be a $2n$-dimensional phase space. If the observables $\{f_i\}_{i\leq n}$ form a complete set, i.e. any function that Poisson commutes with all $f_i$ is necessarily constant, the quantum state space $\mathcal{H}$ is required to be irreducible with respect to the action of $\{\hat{f}_i\}_{i\leq n}$. Equivalently, if a group $G$ acts transitively on $M$, the state space is required to be an irreducible representation of a $\text{U}(1)$-central extension of $G$.
    \end{axiom}
    The simplest example is again $T^*\mathbb{R}^n$. Here, the usual choice of observables are the coordinate and momentum functions $\{q^i,p_i\}_{i\leq n}$. The group $G$ is given by the Heisenberg group, a $\text{U}(1)$-central extension of the translation group $\mathbb{R}^{2n}$. Irreducible representations are characterized by the Stone-von Neumann theorem \ref{qm_formalism:stone_von_neumann}.

    Although the above procedure sounds reasonable, it is known to be impossible impractice. It was shown by \textit{Groenewold} and \textit{Van Hove} that there exists no map $\mathcal{O}$ that takes the entire (Lie) algebra of classical observables to the Lie algebra corresponding quantum observables.\index{Groenewold-Van Hove}

\section{Geometric Quantization}\label{section:geometric_quantization}

    In this section the constant $\hbar$ has been set to 1.

\subsection{Prequantization}

    \newdef{Prequantum line bundle}{\index{prequantum line bundle}
        Consider a classical phase space $M$ equipped with its symplectic form $\omega$. A prequantum line bundle for this manifold is a Hermitian line bundle equipped with a connection $\nabla$ such that $\omega=F_\nabla$ where $F_\nabla$ denotes the curvature two-form associated to $\nabla$.
    }
    \begin{property}\index{Weil!integrality}\index{Dirac!condition}\index{Bohr-Sommerfeld}
        Complex line bundles are classified by their (first) Chern class $c_1\in H^2(M;\mathbb{Z})$, which is proportional to the curvature form through Chern-Weil theory \ref{section:chern_weil}. Therefore, a prequantum line bundle exists if and only if the symplectic form is integral \ref{diff:integral_form} (up to a factor of $2\pi$). For simply-connected manifolds this is equivalent to
        \begin{gather}
            \int_S\omega \in 2\pi\mathbb{Z}
        \end{gather}
        for every 2-cycle $S\subset M$. This condition resembles the ''old'' \textit{Bohr-Sommerfeld condition} and is in general known as the \textbf{Weil integrality condition}.

        The space of (inequivalent) prequantum line bundles forms an affine space over that of flat line bundles. The latter is classified by the \v{C}ech cohomology group $H^1(M; \text{U}(1))$. There are two contributions to this moduli space. First of all there can exist topologically nonequivalent bundles. Another possibility is that there can exist nonequivalent (flat) connections on the same bundle. Two flat connections are nonequivalent if they differ by a closed one-form that is neither integral nor exact.
    \end{property}
    \begin{result}[Dirac quantization condition]
        One can derive the Dirac quantization condition from Weil integrality. If one couples the system to a gauge potential, the minimal coupling procedure gives $\omega\longrightarrow\omega+eF$. Weil integrality then implies that $e$ is an integer.
    \end{result}

    \newdef{Prequantum Hilbert space}{
        Consider a symplectic manifold $(M,\omega)$ together with its prequantum line bundle $L$. The prequantum Hilbert space $\mathcal{H}_{\mathbb{P}}$ is defined as (the $L^2$-completion of) the space of square-integrable sections of $L$ (with respect to the metric on $L$ and the Liouville volume form on $M$).

        To every smooth function $f\in C^\infty(M,\mathbb{C})$ one can associate a (\textbf{Segal-Kostant-Souriau}) prequantum operator $\hat{f}:\Gamma(L)\rightarrow\Gamma(L)$ by the following formula:
        \begin{gather}
            \hat{f}:\psi\rightarrow -i\nabla_{X_f}\,\psi + f\cdot\psi
        \end{gather}
        where $X_f$ is the Hamiltonian vector field \ref{symplectic:hamilton_vectorfield} associated to $f$ and, locally, $\nabla = d - i\theta$ with $\theta$ the Liouville one-form (symplectic potential). This operator can also be interpreted in terms of a Hamiltonian flow. The Hamiltonian flow of $X_f$ can be lifted (up to a phase) to an automorphism $\psi^f_t$ on $L$ that preserves both the metric and the connection. The prequantum operator $\hat{f}$ is then simply given by
        \begin{gather}
            \hat{f}s = -i\left.\deriv{}{t}(\psi^f_ts)\right|_{t=0}.
        \end{gather}
    }

    \begin{example}[Spinning particle]
        Consider as phase space the 2-sphere $S^2(r)$ with radius $r\in\mathbb{R}$. In this case the symplectic form can be written as $\omega=r^2\sin\theta d\theta\wedge d\varphi$. This form is only integral for a discrete set of values of $r$, namely for $r\in\mathbb{Z}/2$. Up to a factor $\hbar$ this is exactly the quantization rule for angular momentum. The reason for this is that $S^2$ is a homogeneous space for $\text{SU}(2)$, the group characterizing spinning particles. In fact, using the theory of coadjoint orbits (see further below), one can show that this quantization procedure coincides with the KKS quantization of the coadjoint orbit $S^2=\text{SU}(2)/\text{U}(1)$.
    \end{example}

    At this point it can easily be seen that there is a problem with the dimension of the prequantum state space. For the cotangent bundle $T^*\mathbb{R}^n = \mathbb{R}^n\times\mathbb{R}^n$ the resulting state space would be $L^2(\mathbb{R}^{2n})$. However, from ordinary quantum theory it is well-known that the right Hilbert space is $L^2(\mathbb{R}^n)$. In general the above procedure would give wave functions that depend on $2n$ variables instead of the $n$ coordinates of configuration space that are normally found in quantum mechanics.

    A solution is obtained by making a choice of ''configuration space'' or, in terms of ordinary quantum mechanics, to choose a ''representation'' of the system:
    \begin{construct}[Quantization]\index{quantization}
        Consider a symplectic manifold $(M,\omega)$. A quantization of $(M,\omega)$ is given by a prequantum line bundle $L$ together with a polarization $P$ of $M$. The ''naive'' quantum state space $\mathcal{H}$ would be given by the subspace of $\mathcal{H}_{\mathbb{P}}$ of those sections that are covariantly constant along $P$, i.e. those sections $s\in\Gamma(L)$ that satisfy $\nabla_Xs=0$ for all $X\in P$. These sections are also called \textbf{polarized sections}.

        The fact that a polarization is required, and not merely an $n$-dimensional involutive distribution, follows from an additional consistency condition imposed by the condition $\nabla_Xs=0$ for all $X\in\mathcal{P}$. Because $\omega$ also represents the curvature of the connection on $L$, one obtains
        \begin{gather}
            \omega(X,Y)s = [\nabla_X,\nabla_Y]s - \nabla_{[X,Y]}s = 0
        \end{gather}
        for all $X,Y\in\mathcal{P}$. This implies that $\mathcal{P}$ defines an isotropic submanifold. For a completely integrable system $\ref{symplectic:CIS}$, a natural choice would be given by the distribution spanned by the Hamiltonian vector fields.

        Now, if the prequantum operators ought to represent genuine operators on the quantum state space, one should have \[\nabla_Xs=0\implies\nabla_X(\hat{f}s)=0\] for all sections $s\in\Gamma(L)$ and $X\in P$. Using the general formula for iterated covariant derivatives and the fact that the leaves of $P$ are Lagrangian, one finds the following condition:
        \begin{gather}
            [X,X_f]=0
        \end{gather}
        for all $X\in P$. So in general one should restrict to the subspace of $C^\infty(M)$ on those functions whose Hamiltonian flow preserves $P$.

        There is, however, a problem with this construction. Nothing ensures that $\mathcal{H}_{\mathbb{P}}$ contains any polarized sections. As an example consider a cotangent bundle with its vertical polarization. In this case the polarized sections are given by functions that only depend on the base coordinates $q^i$ and not on the fibre (momentum) coordinates $p_i$. However, because the fibres are noncompact, the integral of such a section with respect to the Liouville measure will always diverge. For this example this can be solved by integrating over $M\cong T^*M/D$.
    \end{construct}

    However, even though a possible divergence coming from noncompact fibres is resolved, another problem arises. To integrate over $M$ one needs a volume form, but there is not always a canonical choice. A solution is given by working with densities (see section \ref{section:integration_manifolds}):
    \begin{method}[Half-form quantization]
        For this method the polarization is assumed to be real and have simply-connected leaves. Furthermore, the manifold $M$ is assumed to admit a metaplectic structure or, by virtue of property \ref{complex:metaplectic}, $P$ is assumed to admit a metalinear structure. Using this structure one can define the half-form bundle $\delta^{1/2}$. Given a prequantum line bundle $L$, one defines the twisted bundle of $L$-valued half-forms $L\otimes\delta^{1/2}$. A \textbf{wave function} is defined as a section $\psi$ of $L\otimes\delta^{1/2}$ such that locally $\psi=\lambda\otimes\mu$ with $\nabla_X\lambda=0$ and $\mathcal{L}_X\mu=0$ for all $X\in P$. By pairing two wave functions one obtains a 1-density on $M$ which can be integrated. The quantum state space is then defined as the $L^2$-completion of the space of wave functions.

        To extend the definition of operators to half-form quantized manifolds, one simply needs to extend the definition to density bundles. Because $\hat{f}$ represents the Hamiltonian flow, a natural choice is the Lie derivative:
        \begin{gather}
            \hat{f}(s\mu) := (\hat{f}s)\otimes\mu - is\otimes\mathcal{L}_{X_f}\mu.
        \end{gather}
    \end{method}

    \begin{example}[K\"ahler quantization]
        For this method the polarization is assumed to be positive K\"ahler, i.e. $P$ is the antiholomorphic tangent bundle of a K\"ahler manifold. By taking the (local) symplectic potential to be the holomorphic derivative of the K\"ahler potential, one obtains the space of holomorphic sections as the prequantum state space (a different choice of potential results in a phase transformation). It can be shown that a natural inner product is given by
        \begin{gather}
            \langle \psi_1|\psi_2 \rangle = \int_{\mathbb{C}^n}\overline{\psi_1(z)}\psi_2(z)\exp(-|z|^2/2)dz^n.
        \end{gather}
        This representation has many names, including \textbf{Bargmann}, \textbf{Segal-Bargmann} and \textbf{Bargmann-Fock} representations. The coordinates $z,\overline{z}$ are represented by the operators $z$ and $\pderiv{}{z}$. These are easily interpreted as creation and annihilation operators.
    \end{example}
    \begin{remark}
        If the positivity assumption would be dropped, the K\"ahler potential $K$, which was $-|z|^2/2$ above, would be indefinite. This would in turn imply that the integral diverges.
    \end{remark}

    \begin{method}[Bohr-Sommerfeld quantization]\index{Bohr-Sommerfeld}
        Here the polarization $P$ is again assumed to be real, but the leaves are allowed to not be simply-connected. The partial connection $\nabla$ along $P$ is flat when restricted to a single leaf $\Lambda_m\subset M$. When $\Lambda_m$ is not simply-connected, the holonomy group can be nontrivial. However, the defining condition of $\mathcal{H}$ is that sections should be covariantly constant. This implies that either the section is zero or that the holonomy around any loop vanishes. The support of all sections in $\mathcal{H}$ is therefore given by the union $S$ of all leaves on which $\nabla$ is trivial. This space is called the Bohr-Sommerfeld variety.

        Vanishing holonomy implies
        \begin{gather}
            \exp\left(i\oint_\gamma\theta\right) = 1
        \end{gather}
        for all loops $\gamma$, where $\theta$ is the symplectic potential/connection one-form. In terms of Darboux coordinates this gives
        \begin{gather}
            \oint_\gamma p_idq^i\in\mathbb{Z}.
        \end{gather}
        This is exactly the old Bohr-Sommerfeld quantization condition. When using half-density quantization, an additional contribution coming from the covariant derivative on densities would have to be added to the right-hand side:
        \begin{gather}
            \oint_\gamma p_idq^i = 2\pi(k_\gamma + d_\gamma)
        \end{gather}
        where $k_\gamma\in\mathbb{Z}$.
    \end{method}

\subsection{Coadjoint orbits}

    ?? COMPLETE (SEE ALSO \ref{section:coadjoint_orbits}) ??

\section{BFV quantization}

    Recall the content of Section \ref{section:constrained_dynamics} and consider a dynamical system $(M,\omega,H)$ with a Lie algebra of first-class constraints $\{\phi_m\}_{m\in I}$. The extended phase space $M_{\text{ext}}$ is defined by introducing dynamical Lagrange multipliers and their momenta:
    \begin{gather}
        \{\lambda^m,\pi_n\} := \delta^m_n,
    \end{gather}
    together with two collections of (homologically) odd-degree ghosts and their momenta:
    \begin{gather}
        \{C^i,\overline{P}_j\} := \delta^i_j\\
        \{P^i,\overline{C}_j\} := \delta^i_j
    \end{gather}
    Note that the Poisson brackets for the ghost is a graded Poisson bracket. The subalgebra on the variables $(q,p,C,\overline{P})$ is called the \textbf{minimal subalgebra}, while the subalgebra on the variables $(\lambda,\pi,P,\overline{C})$ is called the \textbf{auxiliary subalgebra}.

    A quantized algebra of observables is obtained through the (graded) Dirac correspondence. This algebra is naturally graded with respect to the ghost number defined by the self-adjoint operator
    \begin{gather}
        \mathcal{G} := \frac{1}{2}\left(C^i\overline{P}_i - \overline{P}^iC_i + P^i\overline{C}_i - \overline{C}_iP^i\right).
    \end{gather}
    This operator acts on observables as follows:
    \begin{gather}
        [\mathcal{G},A] = i\hbar\,\text{gh}(A)A.
    \end{gather}

    BFV quantization is obtained by defining a nilpotent BRST operator $\Omega=\Omega_{\text{min}}+\Omega_{\text{aux}}$.
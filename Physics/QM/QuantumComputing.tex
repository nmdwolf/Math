\chapter{Quantum Information Theory}\label{chapter:quantum_computing}

\section{Entanglement}
\subsection{Schmidt decomposition}

    \begin{construct}[Schmidt decomposition]\index{Schmidt!decomposition}\index{rank!Schmidt}
        Consider a bipartite state $|\psi\rangle\in\mathcal{H}_1\otimes\mathcal{H}_2$. For any such state there exist orthonormal sets $\big\{|e_i\rangle, |f_j\rangle\big\}_{i,j\leq\kappa}$ such that
        \begin{gather}
            |\psi\rangle = \sum_{i=1}^\kappa\lambda_i|e_i\rangle\otimes|f_i\rangle,
        \end{gather}
        where the coefficients $\lambda_i$ are nonnegative real numbers. All objects in this expression can be obtained from a singular value decomposition of the coefficient matrix $\mathbf{C}$ of $|\psi\rangle$ in some bases of $\mathcal{H}_1$ and $\mathcal{H}_2$. The number $\kappa$ is called the \textbf{Schmidt rank} of $|\psi\rangle$.
    \end{construct}

    \newdef{Entangled states}{\index{separable!state}\index{entanglement}
        Consider a state $|\psi\rangle$ and consider its Schmidt decomposition. If the Schmidt rank is 1, i.e.~the state can be written as $|\psi\rangle = |v\rangle\otimes|w\rangle$, the state is said to be \textbf{separable}. Otherwise the state is said to be entangled.
    }

\subsection{Bell states}

    \newdef{Bell state}{\index{Bell state}\index{Einstein-Podolsky-Rosen|see{Bell state}}
        \nomenclature[A_EPR]{EPR}{Einstein-Podolsky-Rosen}
        A (binary) Bell state (also called a \textbf{cat state} or \textbf{Einstein-Podolsky-Rosen pair}) is defined as the following entangled state:
        \begin{gather}
            |\Phi^+\rangle := \frac{1}{\sqrt{2}}\Big(|00\rangle+|11\rangle\Big).
        \end{gather}
        In fact this state can be extended to a full maximally entangled basis for the 2-qubit Hilbert space:
        \begin{gather}
            |\Phi^-\rangle := \frac{1}{\sqrt{2}}\Big(|00\rangle-|11\rangle\Big)\nonumber\\
            |\Psi^+\rangle := \frac{1}{\sqrt{2}}\Big(|01\rangle+|10\rangle\Big)\\
            |\Psi^-\rangle := \frac{1}{\sqrt{2}}\Big(|01\rangle-|10\rangle\Big).\nonumber
        \end{gather}
    }

    \begin{method}[Dense coding\footnotemark]\index{dense!coding}
        \footnotetext{Sometimes called \textbf{superdense coding}.}
        Consider the Bell state $|\Phi^+\rangle$. By acting with one of the (unitary) spin-flip operators $X,Y,Z$ one can obtain any of the other three Bell states:
        \begin{align}
            X|\Phi^+\rangle &= |\Phi^-\rangle\nonumber\\
            Y|\Phi^+\rangle &= |\Psi^+\rangle\\
            Z|\Phi^+\rangle &= |\Psi^-\rangle.\nonumber
        \end{align}
        In a typical Alice-and-Bob-style experiment one can ask the question if this observation allows to achieve a better-than-classical communication channel. If Alice performs a spin flip on her qubit, although the resulting state has instantly ``changed'' (cf. \textit{spooky action at a distance}), Bob still cannot uniquely determine what this state is (since the resulting state is still maximally entangled). However, if Alice sends her qubit to Bob, the latter can perform a measurement on the composite system to find out what the state is and in this way determine which operation Alice performed ($\mathbbm{1},X,Y,Z$). Alice has thus effectively sent 2 classical bits of information through 1 qubit. Note that due to the fact that Alice still has to send her qubit through classical means, no faster-than-light communication is achieved.
    \end{method}

    \newdef{GHZ\footnotemark\ state}{\index{Greenberger-Horne-Zeilinger state}
        \footnotetext{\textit{Greenberger-Horne-Zeilinger}}
        \nomenclature[A_GHZ]{GHZ}{Greenberger-Horne-Zeilinger}
        The GHZ state is defined as the multiparticle qudit ($d,N>2$) version of the Bell state above and is, hence, also referenced to as a cat state:
        \begin{gather}
            |\mathrm{GHZ}\rangle = \frac{1}{\sqrt{d}}\sum_{i=0}^{d-1}|i\rangle^{\otimes N}.
        \end{gather}
    }

\section{Density operators}\index{density!operator}

    \newdef{Density operator}{
        Consider a (finite-dimensional) Hilbert space $\mathcal{H}$. A density operator on $\mathcal{H}$ is a linear operator $\rho\in\End(\mathcal{H})$ satisfying the following properties:
        \begin{enumerate}
            \item\textbf{Positivity}: $\langle v|\rho v \rangle\geq0$ for all $v\in\mathcal{H}$.
            \item\textbf{Hermiticity}: $\rho^\dag=\rho$.
            \item\textbf{Unit trace}: $\tr(\rho)=1$.
        \end{enumerate}
        More concisely, density operators are the representing objects of normal states \ref{operators:normal_state} on $\mathcal{B}(\mathcal{H})$.
    }

    \begin{example}[Classical probability]
        A diagonal density matrix corresponds to a (classical) discrete probability distribution.
    \end{example}

    \newdef{Pure state}{\index{pure}
        A state is said to be pure if it is described by an outer product of a state vector or, equivalently, by an idempotent density matrix. A density matrix that is not of this form gives rise to a \textbf{mixed state}.
    }

	\newdef{Reduced density operator}{
		Let $|\Psi\rangle\in\mathcal{H}_A\otimes\mathcal{H}_B$ be the state of a bipartite system. The reduced density operator $\hat\rho_A$ of $A$ is defined as follows:
		\begin{gather}
			\hat\rho_A := \tr_B|\Psi\rangle\langle\Psi|.
		\end{gather}
	}

	\newdef{Purification}{\index{purification}
		Let $\hat\rho_A$ be the density operator of a system $A$. A purification of $\hat\rho_A$ is a pure state $|\Psi\rangle$ of some composite system $AB$ such that
		\begin{gather}
			\hat\rho_A = \tr_B|\Psi\rangle\langle\Psi|.
		\end{gather}
	}
	\begin{property}
		Any two purifications of the same density operator $\hat\rho_A$ are related by a transformation $\mathbbm{1}_A\otimes\hat{V}$ with $\hat{V}$ an isometry.
	\end{property}

\section{Operations}

    The following definition generalizes the content of Section \ref{section:PVM} to a setting of partial information:
    \newdef{Positive operator-valued measure}{\index{measurement!positive-operator valued}
        \nomenclature[A_POVM]{POVM}{positive operator-valued measure}
        First, let $\mathcal{H}$ be a finite-dimensional Hilbert space. A POVM on $\mathcal{H}$ consists of a finite set of positive (semi)definite operators $\{P_i\}_{i\leq n}$ such that
        \begin{gather}
            \sum_{i=1}^nP_i=\mathbbm{1}_\mathcal{H}.
        \end{gather}
        The probability to obtain state $i$, given a general state $\hat{\rho}$, is given by $\tr(\hat{\rho}P_i)$. Note that the operators are not necessarily orthogonal projectors, so $n$ can be greater than $\dim(\mathcal{H})$.

        Now, consider a measurable space $(X,\Sigma)$ and a (possibly infinite-dimensional) Hilbert space $\mathcal{H}$. A POVM on $X$ consists of a function $P:\Sigma\rightarrow\mathcal{B}(\mathcal{H})$ satisfying the following conditions:
        \begin{enumerate}
            \item $P_E$ is positive and self-adjoint for all $E\in\Sigma$,
            \item $P_X=\mathbbm{1}_\mathcal{H}$, and
            \item for all disjoint $\seq{E}\subset\Sigma$:
                \begin{gather}
                    \sum_{n\in\mathbb{N}}P_{E_n} = P_{\cup_{n\in\mathbb{N}}E_n}.
                \end{gather}
        \end{enumerate}
    }

    The following theorem can be derived from the Stinespring theorem \ref{operators:stinespring}:
    \begin{theorem}[Naimark dilation theorem]\index{Naimark dilation theorem}
        Every POVM $P$ on $\mathcal{H}$ can be realized as a PVM $\Pi$ on a possibly larger Hilbert space $\mathcal{K}$, i.e.~there exists a bounded operator $V:\mathcal{K}\rightarrow\mathcal{H}$ such that
        \begin{gather}
            P(\cdot) = V\Pi(\cdot)V^\dagger.
        \end{gather}
        In the finite-dimensional setting $V$ can be chosen to be an isometry.
    \end{theorem}

    \newdef{Completely positive trace-preserving}{\index{quantum!channel}
        \nomenclature[A_CPTP]{CPTP}{completely positive, trace-preserving}
        Consider a map $\Phi:\mathcal{B}(\mathcal{H}_1)\rightarrow\mathcal{B}(\mathcal{H}_2)$ between (trace-class) operators on two (finite-dimensional) Hilbert spaces. This map preserves density matrices if it positive \ref{operators:positive_map} and if it is trace-preserving \ref{operators:trace_preserving}. Furthermore, to ensure that an operation applied to a subsystem does not interfere with the positivity of the complete system, they are also required to be completely positive \ref{operators:cp_map}.

        Completely positive, trace-preserving (CPTP) maps are often called \textbf{quantum channels}.
    }

    The following property can be derived from the Stinespring theorem \ref{operators:stinespring}:
    \begin{property}[Kraus decomposition]\index{Kraus decomposition}\label{qc:kraus}
        Let $\mathcal{H}_1,\mathcal{H}_2$ be Hilbert spaces of dimensions $m$ and $n$, respectively. A linear map $\Phi:\mathcal{B}(\mathcal{H}_1)\rightarrow\mathcal{B}(\mathcal{H}_2)$ is completely positive if and only if there exist bounded operators $\{A_i\}_{i\leq mn}$ such that
        \begin{gather}
            \Phi(B) = \sum_{i=1}^{mn}A^\dag_iBA_i.
        \end{gather}
        Furthermore, it is trace-preserving if and only if
        \begin{gather}
            \sum_{i=1}^{mn}A^\dag_iA_i = \mathbbm{1}.
        \end{gather}
        A decomposition of the above form is also often called an \textbf{operator-sum decomposition}.
    \end{property}
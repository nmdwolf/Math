\chapter{Supersymmetry}

    This chapter is not meant to be an in-depth study of supersymmetry and its implications for (particle) physics. It will only introduce some concepts and constructions that are widely used in the study of supersymmetric theories. It also contains some sections on certain interesting mathematical properties that arise while studying supersymmetry.

    The main reference for supersymmetric quantum mechanics is the seminal paper \cite{witten_morse} by \textit{Witten}. For an introduction to algebraic superstructures, see section \ref{section:graded_spaces}.

\section{Supersymmetric quantum mechanics}

    In this section we consider a general graded Hilbert space $\mathcal{H}$. We equip this space with an algebra of bounded operator $A\subset\mathcal{B}(\mathcal{H})$ together with a set of $N$ odd self-adjoint operators $\{D^i\}_{i\leq N}$. More precisely we consider a \textit{spectral triple} $(\mathcal{H}, A, \{D^i\}_{i\leq N})$.

    This data defines an SQM system if the Hamiltonian $H$ satisfies the following condition:
    \begin{gather}
        \{D^i, D^j\}_+ = 2\delta^{ij}H.
    \end{gather}
    For $N=2$ we can rephrase the whole theory in terms of a nilpotent operator $d\sim D^1+iD^2$ and its adjoint:
    \begin{gather}
        \{d, d^\dagger\}_+\sim H.
    \end{gather}

    \begin{example}[Particle on a manifold]
        The archetypal example of such $N=2$ systems is the situation where $d$ is the exterior derivative on a smooth manifold $M$, $A$ is the algebra of smooth functions $C^\infty(M)$ and $\mathcal{H}$ is the Hilbert space of square-integrable forms with respect to the Hodge metric
        \begin{gather}
            \langle\alpha|\beta\rangle = \int_M\alpha\wedge\ast\beta.
        \end{gather}
    \end{example}

    The above superalgebra is that of a $D=1,N=2$ theory, i.e. there are two generators acting on a one-dimensional manifold. However, this system can be deformed to represent a $D=2,N=1$ theory.

    An important feature of these theories is that they can be deformed. Let $e^{W(t)}$ be a one-parameter subgroup of invertible operator. The $W$-deformed operators are defined as follows:
    \begin{align}
        d^W &:= e^{-W(t)}\circ d\circ e^{W(t)}\\
        (d^W)^\dagger &:= e^{W(t)^\dagger}\circ d^\dagger\circ e^{-W(t)^\dagger}.
    \end{align}
    It is not hard to see that these deformed operators preserve the superalgebra. Although many authors assume $W$ to be a smooth function, this is not necessary. In fact many interesting examples involve more exotic choices. For example, \cite{phd_schreiber} considers a loop space $\Omega M$ (i.e. he considers a theory of closed strings) where the the deformation operator at a point $\gamma\in\Omega M$ is given by
    \begin{gather}
        W(\gamma)\omega := \int_\gamma dt\ B_{\mu\nu}(\gamma(t))dx^\mu(t)\wedge dx^\nu(t)\wedge\omega(t),
    \end{gather}
    i.e. the operator takes the exterior product with a given 2-form field (e.g. the \textit{Kalb-Ramond field}) and integrates over the loop $\gamma$ (at least after pairing with a set of vector fields). When restricting to the class of skew-Hermitian operators, the deformations can be shown to be pure gauge.

\subsection{Loop space}

    An interesting setting for supersymmetric quantum mechanics is the situation mentioned at the end of the previous section, namely where the base manifold is a loop space $\Omega M$. In this case the tangent space $T_p\Omega M$ is the space of vector fields along the path $p$. As such they carry two indices, one with respect to a (local) frame field on $M$ and one coming from the $S^1$-parametrization of loops. A holonomic basis is given by functional derivatives:
    \begin{gather}
        \partial_{\mu,\sigma} := \frac{\delta}{\delta X^\mu(\sigma)}
    \end{gather}
    where $X^\mu(\sigma)$ is the $\mu^{th}$ coordinate of the loop at the parameter $\sigma\in[0,2\pi[$.



\subsection{Morse theory}

    ?? INSERT WITTEN ??

\section{Extensions of the Standard Model}

    \begin{theorem}[Coleman-Mandula]\index{Coleman-Mandula}
        Consider a quantum field theory with the following constraints:
        \begin{enumerate}
            \item There exists a mass gap.
            \item For every mass $M$ there exist only finitely many particle species with mass $\leq M$.
            \item The two-point scattering amplitudes are nonvanishing for almost every energy.
            \item The (two-point) scattering amplitudes are analytic in the particle momenta.
        \end{enumerate}
        If the symmetry group (of the $S$-matrix) contains a subgroup isomorphic to the Poincar\'e group\footnote{To be precise: its universal cover.}, then it can be written as the direct product of the Poincar\'e group and an internal gauge group.
    \end{theorem}
    \sremark{In other words, it is impossible to combine the Poincar\'e group in a nontrivial way with the internal symmetry group.}

    Now the question arises if one can do better, i.e. is there a nontrivial way to extend this total symmetry group. A first possibility was given by conformal field theories in chapter \ref{chapter:CFT}. CFTs do not admit an $S$-matrix and hence the above theorem is clearly not applicable. However, a second and more intricate possibility is given by supersymmetry. Here one does not work with an ordinary symmetry Lie algebra\footnote{See the original paper \cite{coleman_mandula} for why exactly the algebra plays an essential role.} but with a Lie superalgebra. By allowing superspaces, or equivalently fermionic symmetry generators, one can generalize the Coleman-Mandula theorem. The resulting generalization was proven by \textit{Sohnius, Lopusza\'nski} and \textit{Haag}.
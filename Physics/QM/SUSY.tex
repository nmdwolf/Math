\chapter{Supersymmetry}

    For an introduction to superstructures in algebra, see section \ref{section:graded_spaces}.

\section{Extensions of the Standard Model}

    \begin{theorem}[Coleman-Mandula]\index{Coleman-Mandula}
        Consider a quantum field theory with the following constraints:
        \begin{enumerate}
            \item There exists a mass gap.
            \item For every mass $M$ there exist only finitely many particle species with mass $\leq M$.
            \item The two-point scattering amplitudes are nonvanishing for almost every energy.
            \item The (two-point) scattering amplitudes are analytic in the particle momenta.
        \end{enumerate}
        If the symmetry group (of the $S$-matrix) contains a subgroup isomorphic to the Poincar\'e group\footnote{To be precise: its universal cover.}, then it can be written as the direct product of the Poincar\'e group and an internal gauge group.
    \end{theorem}
    \sremark{In other words, it is impossible to combine the Poincar\'e group in a nontrivial way with the internal symmetry group.}

    Now the question arises if one can do better, i.e. is there a nontrivial way to extend this total symmetry group. A first possibility was given by conformal field theories in chapter \ref{chapter:CFT}. CFT's do not admit an $S$-matrix and hence the above theorem is clearly not applicable. However, a second and more intricate possibility is given by supersymmetry. Here one does not work with an ordinary symmetry Lie algebra\footnote{See the original paper \cite{coleman_mandula} for why exactly the algebra plays an essential role.} but with a Lie superalgebra. By allowing superspaces, or equivalently fermionic symmetry generators, one can generalize the Coleman-Mandula theorem. The resulting generalization was proven by \textit{Sohnius, Lopusza\'nski} and \textit{Haag}.
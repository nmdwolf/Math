\chapter{Scattering Theory}

\section{Cross sections}

    \newformula{Differential cross section}{\index{cross!section}\label{scattering:cross_section}
        \begin{gather}
            \deriv{\sigma}{\Omega} = \frac{N(\theta,\varphi)}{F}\,,
        \end{gather}
        where $F$ is the incoming flux and $N$ the detected flow rate. Because $N$ is not defined as a flux but as a rate, the differential cross section has the dimension of area.
    }

    \newformula{Fermi's golden rule}{\index{Fermi!golden rule}\label{scattering:fermi_golden_rule}
        The transition probability from an initial state to a final state is given by
        \begin{gather}
            \Gamma_{i\rightarrow f} = \frac{2\pi}{\hbar}|\langle f|\widehat{H}|i\rangle|^2\deriv{n}{E_f}\,.
        \end{gather}
    }

\section{Lippman--Schwinger equations}

    In this section Hamiltonians of the form $\widehat{H} = \widehat{H}_0 + \widehat{V}$ are considered, where $\widehat{H}_0$ is the free Hamiltonian and $\widehat{V}$ the scattering potential. It will also be assumed that both the total Hamiltonian and the free Hamiltonian have the same eigenvalues (as in the previous chapter).

    \newformula{Lippman--Schwinger equation}{\index{Lippman--Schwinger equation}
        \begin{gather}
            |\psi^{(\pm)}\rangle = |\varphi\rangle + \frac{1}{E - \widehat{H}_0 \pm i\varepsilon}\widehat{V}|\psi^{(\pm)}\rangle\,,
        \end{gather}
        where $|\varphi\rangle$ is an eigenstate of the free Hamiltonian with the same energy as $|\psi\rangle$, i.e.~$\widehat{H}_0|\varphi\rangle = E|\varphi\rangle$.
    }
    \begin{remark}
        The term $\pm i\varepsilon$ is added to the denominator because otherwise it would be singular. The term has no real physical meaning.
    \end{remark}

    \newformula{Born series}{\index{Born!series}\label{scattering:born_series}
        If the Lippman--Schwinger equation is rewritten as
        \begin{gather}
            |\psi\rangle = |\varphi\rangle + \widehat{G}_0\widehat{V}|\psi\rangle\,,
        \end{gather}
        where $\widehat{G}_0$ is the Green's operator (\cref{pde:green_function}) associated to $\widehat{H}_0$, one can derive the following series expansion by iterating the above expression:
        \begin{gather}
            |\psi\rangle = |\varphi\rangle + \widehat{G}_0\widehat{V}|\varphi\rangle + \left(\widehat{G}_0\widehat{V}\right)^2|\varphi\rangle + \cdots\,.
        \end{gather}
        Convergence issues of this series can be resolved through a Borel resummation procedure (\cref{calculus:borel_transform}).
    }
    \newformula{Born approximation}{\index{Born!approximation}\label{scattering:born_approximation}
        If the Born series is truncated at the first-order term in $\widehat{V}$, the Born approximation is obtained:
        \begin{gather}
            |\psi\rangle = |\varphi\rangle + \widehat{G}_0\widehat{V}|\varphi\rangle\,.
        \end{gather}
    }
\chapter{Scattering Theory}

    \newformula{Differential cross section}{\index{cross!section}
        \begin{gather}
            \label{QM:scattering:cross_section}
            \deriv{\sigma}{\Omega} = \frac{N(\theta, \varphi)}{F}
        \end{gather}
        where $F$ is the incoming particle flux and $N$ the detected flow rate\footnote{Because $N$ is not defined as a flux but as a rate, the differential cross section has the dimension of area.}.
    }

    \newformula{Fermi's golden rule}{\index{Fermi!golden rule}
        The transition probability from state $i$ to state $f$ is given by
        \begin{gather}
            \label{QM:scattering:fermi_golden_rule}
            \Gamma_{i\rightarrow f} = \frac{2\pi}{\hbar}|\langle f|\hat{H}|i\rangle|^2\deriv{n}{E_f}.
        \end{gather}
    }

\section{Lippman-Schwinger equations}

    In this section we consider Hamiltonians of the following form: $\hat{H} = \hat{H}_0 + \hat{V}$ where $\hat{H}_0$ is the free Hamiltonian and $\hat{V}$ the scattering potential. We will also assume that both the total Hamiltonian and the free Hamiltonian have the same eigenvalues.

    \newformula{Lippman-Schwinger equation}{\index{Schwinger!Lippman-Schwinger equation}
        \begin{gather}
            |\psi^{(\pm)}\rangle = |\varphi\rangle + \stylefrac{1}{E - \hat{H}_0 \pm i\varepsilon}\hat{V}|\psi^{(\pm)}\rangle
        \end{gather}
        where $|\varphi\rangle$ is an eigenstate of the free Hamiltonian with the same energy as $|\psi\rangle$, i.e. $\hat{H}_0|\varphi\rangle = E|\varphi\rangle$.
    }
    \begin{remark}
        The term $\pm i\varepsilon$ is added to the denominator because otherwise it would be singular. The term has no real physical meaning.
    \end{remark}

    \newformula{Born series}{\index{Born!series}
        If we rewrite the Lippman-Schwinger equation as
        \begin{gather}
            |\psi\rangle = |\varphi\rangle + \hat{G}_0\hat{V}|\psi\rangle
        \end{gather}
        where $\hat{G}_0$ is the Green's operator, then we can derive the following series expansion by iterating the above expression:
        \begin{gather}
            \label{QM:cattering:born_series}
            |\psi\rangle = |\varphi\rangle + \hat{G}_0\hat{V}|\varphi\rangle + \left(\hat{G}_0\hat{V}\right)^2|\varphi\rangle + ....
        \end{gather}
        Convergence issues of this series can be resolved through a Borel resummation procedure (see method \ref{calculus:borel_transform}).
    }
    \newformula{Born approximation}{\index{Born!approximation}
        If we cut off the Born series at the first order term in $\hat{V}$ then we obtain the Born approximation:
        \begin{gather}
            \label{QM:scattering:born_approximation}
            |\psi\rangle = |\varphi\rangle + \hat{G}_0\hat{V}|\varphi\rangle.
        \end{gather}
    }
\chapter{Wave and Matrix Mechanics}

\section{Schr\"odinger picture}
\subsection{One dimension}

    \begin{formula}[TISE]
        The following partial differential equation is called the TISE or \textbf{time-independent Schr\"odinger equation}:
        \begin{gather}
            \label{schrodinger:1D:TISE}
            \hat{H}\psi(x) = E\psi(x)
        \end{gather}
        where $\hat{H}$ is the Hamiltonian of the system and $\psi$ is an element of the vector space $L^2(\mathbb{R})\otimes\mathcal{H}$ with $\mathcal{H}$ the internal Hilbert space (describing for example the spin or charge of a particle). This is an eigenvalue equation for the energy levels of the system.
    \end{formula}

    \newprop{Orthogonality}{
        Let $\{\psi_i\}_{i\in I}\subset L^2(\mathbb{R})$ be a set of eigenfunctions of the TISE (we ignore the internal space for convenience). These functions can be normalised such that they obey the following orthonormality relations:
        \begin{gather}
            \label{schrodinger:orthogonality}
            \int_{\mathbb{R}}\psi_i^*(x)\psi_j(x)dx = \delta_{ij}.
        \end{gather}
    }

    The time evolution of a wave function is defined through the following equation:
    \newformula{TDSE}{\index{Schr\"odinger!equation}
        The following partial differential equation is called the (time-dependent) Schr\"odinger equation:
        \begin{gather}
            \label{schrodinger:TDSE}
            i\hbar\pderiv{}{t}\psi(x, t) = \hat{H}\psi(x, t).
        \end{gather}
        The TISE can be obtained from this equation by separation of variables\footnote{See section \ref{section:separation_of_variables}.} (if $\hat{H}$ is time-independent).
    }
    \begin{example}[Massive particle in a stationary potential]
        \begin{gather}
            \label{schrodinger:1D:TDSE_position}
            i\hbar\pderiv{}{t}\psi(x, t) = \left(\stylefrac{\hat{p}^2}{2m} + \hat{V}(x)\right)\psi(x, t)
        \end{gather}
    \end{example}

    \begin{formula}[General solution]
        A general solution of the time-dependent Schr\"odinger equation (for time-independent Hamiltonians) is given by the following formula\footnote{See also equation \ref{diffeq:first_order_general_solution}.}:
        \begin{gather}
            \label{schrodinger:1D:general_solution}
            \psi(x, t) = \sum_Ec_E\psi_E(x)e^{-\frac{i}{\hbar}Et}
        \end{gather}
        where the functions $\psi_E(x)$ are the eigenfunctions of the TISE \ref{schrodinger:1D:TISE}. The coefficients $c_E$ can be found using the orthogonality relations
        \begin{gather}
            \label{schrodinger:1D:general_solution_coefficients}
            c_E=\left(\int_{\mathbb{R}}\psi_E^*(x')\psi(x', t_0)dx'\right)e^{\frac{i}{\hbar}Et_0}.
        \end{gather}
    \end{formula}

    ?? COMPLETE ??

\section{Heisenberg picture}

    Let us first reflect on what the wave function is and how it relates to the state of the general system. At every point it gives us the probability of observing a particle (or whatever object we are studying). But what if we want to express our information not in terms of positions, but instead in terms of momenta? The information about our state should not depend on the chosen ''representation''. Therefore we introduce a state vector $|\psi\rangle$ that represents the state of our system as an abstract vector in the Hilbert space.
    \begin{notation}[Dirac notation]\index{Dirac!notation}
        This notation was introduced by Dirac and is often called the \textit{braket} notation. State vectors $|\psi\rangle$ are called \textbf{ket}'s and their duals $\langle\psi|$ are called \textbf{bra}'s. An inner product is then simply denoted by $\langle\phi|\psi\rangle$.
    \end{notation}

    But then how do we obtain the position (or configuration) representation $\psi(x)$? This is simply a projection of the state vector $|\psi\rangle$ on the ''basis function'' $|x\rangle$ in $L^2(\mathbb{R})$, i.e. $\psi(x)$ represents an expansion coefficient.\footnote{See the next section for a more formal explanation.} In the same way one can obtain a momentum space representation $\psi(p)$ by projecting on the plane waves $e^{ipx}$.

    From now on we will adopt this new notation and work independent from any specific state representation.

\subsection{Matrix representation}

    \begin{formula}[Matrix representation]
        The following formula gives the matrix representation of an operator $\hat{A}$ with respect to the orthonormal basis $\big\{|\psi_i\rangle\big\}_{i\in I}$ (where $I$ is assumed to be finite here\footnote{A rigorous treatment will be given in the next chapter.}) cf. section \ref{linalgebra:matrix_representation}:
        \begin{gather}
            \label{qm_formalism:matrix_entry}
            A_{ij} := \langle\psi_i|\hat{A}|\psi_j\rangle.
        \end{gather}
    \end{formula}
    \begin{remark}
         The basis $\big\{|\psi_i\rangle\big\}_{i\in I}$ need not consist out of eigenfunctions of $\hat{A}$.
    \end{remark}

    ?? COMPLETE ??
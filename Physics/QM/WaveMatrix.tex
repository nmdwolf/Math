\chapter{Waves and matrices}

\section{Schr\"odinger picture}
\subsection{One dimension}

    \begin{formula}[TISE]
        The following partial differential equation is called the TISE or \textbf{time-independent Schr\"odinger equation}:
        \begin{gather}
            \label{schrodinger:1D:TISE}
            \hat{H}\psi(x) = E\psi(x)
        \end{gather}
        where $\hat{H}$ is the Hamiltonian of the system.
    \end{formula}

    \newprop{Orthogonality}{
        Let $\{\psi_i\}_{i\in I}\subset L^2(\mathbb{R})$ be a set of eigenfunctions of the TISE. These functions can be normalised as to obey the following orthonormality relations:
        \begin{gather}
            \label{schrodinger:orthogonality}
            \int\psi_i^*(x)\psi_j(x)dx = \delta_{ij}.
        \end{gather}
    }

    \newformula{TDSE}{\index{Schr\"odinger!equation}
        The following partial differential equation is called the (time-dependent) Schr\"odinger equation:
        \begin{gather}
            \label{schrodinger:TDSE}
            i\hbar\pderiv{\psi}{t} = \hat{H}\psi.
        \end{gather}
    }
    \begin{example}[Massive particle in a stationary potential]
        \begin{gather}
            \label{schrodinger:1D:TDSE_position}
            i\hbar\pderiv{}{t}\psi(x, t) = \left(\stylefrac{\hat{p}^2}{2m} + \hat{V}(x)\right)\psi(x, t)
        \end{gather}
    \end{example}

    \begin{formula}[General solution]
        A general solution of the time-dependent Schr\"odinger equation (for time-independent Hamiltonians) is given by the following formula\footnote{See also equation \ref{diffeq:first_order_general_solution}.}:
        \begin{gather}
            \label{schrodinger:1D:general_solution}
            \psi(x, t) = \sum_Ec_E\psi_E(x)e^{-\frac{i}{\hbar}Et}
        \end{gather}
        where the functions $\psi_E(x)$ are the eigenfunctions of the TISE \ref{schrodinger:1D:TISE}. The coefficients $c_E$ can be found using the orthogonality relations
        \begin{gather}
            \label{schrodinger:1D:general_solution_coefficients}
            c_E=\left(\int\psi_E^*(x')\psi(x', t_0)dx'\right)e^{\frac{i}{\hbar}Et_0}.
        \end{gather}
    \end{formula}

    ?? COMPLETE ??

\section{Heisenberg picture}
\subsection{Matrix representation}

    \begin{formula}
        The following formula gives the matrix representation of an operator $\hat{A}$ with respect to the orthonormal basis $\big\{|\psi_n\rangle\big\}$:
        \begin{gather}
            \label{qm_formalism:matrix_entry}
            A_{mn} := \langle\psi_m|\hat{A}|\psi_n\rangle.
        \end{gather}
    \end{formula}
    \begin{remark}
         The basis $\big\{|\psi_n\rangle\big\}$ need not consist out of eigenfunctions of $\hat{A}$.
    \end{remark}

    ?? COMPLETE ??
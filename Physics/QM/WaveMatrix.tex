\chapter{Wave and Matrix Mechanics}

    The main reference for this chapter is \cite{bransden}. In this chapter the two basic formalisms of Quantum Mechanics are introduced: wave and matrix mechanics.

\section{Schr\"odinger picture}

    \begin{formula}[Time-independent Schr\"odinger equation]\index{Schr\"odinger!equation}\index{Hamilton!function}\label{wavematrix:TISE}
        \begin{gather}
            \hat{H}\psi(x) = E\psi(x).
        \end{gather}
        The operator $\hat{H}$ is called the \textbf{Hamiltonian} of the system and $\psi$ is an element of the vector space $L^2(\mathbb{R},\mathbb{C})\otimes\mathcal{H}$ with $\mathcal{H}$ the internal Hilbert space (describing for example the spin or charge of a particle). This is an eigenvalue equation for the energy levels of the system.
    \end{formula}

    \newprop{Orthogonality}{
        Let $\{\psi_i\}_{i\in I}\subset L^2(\mathbb{R})$ be a collection of eigenfunctions of the TISE (the internal space is hidden for convenience). These functions can be normalized with respect to the inner product \eqref{lebesgue:L2_inner_product} such that they obey the following relation:
        \begin{gather}
            \label{wavematrix:orthogonality}
            \int_\mathbb{R}\overline{\psi_i}(x)\psi_j(x)dx = \delta_{ij}.
        \end{gather}
    }

    The time evolution of a wave function is defined through the following equation:
    \newformula{TDSE}{\index{Schr\"odinger!equation}\label{wavematrix:TDSE}
        The following partial differential equation is called the \textbf{(time-dependent) Schr\"odinger equation}:
        \begin{gather}
            i\hbar\pderiv{}{t}\psi(x,t) = \hat{H}\psi(x,t).
        \end{gather}
        In case $\hat{H}$ is time-independent, the TISE can be obtained from this equation by separation of variables (see Derivation \ref{section:tdse_tise}).
    }
    \begin{example}[Massive particle in a stationary potential]
        \begin{gather}
            \label{wavematrix:TDSE_position}
            i\hbar\pderiv{}{t}\psi(x,t) = \left(\frac{\hat{p}^2}{2m} + \hat{V}(x)\right)\psi(x,t)
        \end{gather}
    \end{example}

    \begin{formula}[General solution]
        A general solution of the time-dependent Schr\"odinger equation (for time-independent Hamiltonians) is given by the following formula (cf. Formula \ref{ode:first_order_general_solution}):
        \begin{gather}
            \label{wavematrix:general_solution}
            \psi(x,t) = \sum_Ec_E\psi_E(x)e^{-\frac{i}{\hbar}Et},
        \end{gather}
        where the functions $\psi_E(x)$ are the eigenfunctions of the TISE \ref{wavematrix:TISE}. The coefficients $c_E$ can be found using the orthogonality relations
        \begin{gather}
            \label{wavematrix:general_solution_coefficients}
            c_E=\left(\int_\mathbb{R}\overline{\psi_E}(x)\psi(x,t_0)dx\right)e^{\frac{i}{\hbar}Et_0}.
        \end{gather}
    \end{formula}

    ?? COMPLETE ??

\section{Heisenberg picture}

    This is the right place to reflect on what the wave function is and how it relates to the state of a system. At every point it gives the probability of observing a particle (or whatever object is being studied). But what if one wants to express this information in terms of momenta instead of positions? The information about the state should not depend on the chosen ``representation''. To this end, a state vector $|\psi\rangle$ that represents the state of the system as an abstract vector in some Hilbert space is introduced.
    \begin{notation}[Dirac notation]\index{bra-ket notation}\index{Dirac!notation}
        This notation is often called the \textbf{braket notation}. State vectors $|\psi\rangle$ are called \textbf{ket}'s and their duals $\langle\psi|$ are called \textbf{bra}'s. The inner product of a state $|\phi\rangle$ and a state $|\psi\rangle$ is denoted by $\langle\phi|\psi\rangle$.
    \end{notation}
    But then, how does one recover the position (configuration) representation $\psi(x)$? This is simply the projection of the state vector $|\psi\rangle$ on the ``basis function'' $\delta(x)$, i.e. $\psi(x)$ represents an expansion coefficient in terms of a ``basis'' for the physical Hilbert space. In the same way one can obtain the momentum representation $\psi(p)$ by projecting on the plane waves $e^{ipx}$.

    \begin{remark}
        It should be noted that neither the ``basis states'' $\delta(x)$, nor the plane waves $e^{ipx}$ are square-integrable and, hence, they are not elements of the Hilbert space $L^2(\mathbb{R},\mathbb{C})$. In the next chapter this issue will be resolved through the concept of \textit{rigged Hilbert spaces}.
    \end{remark}

    \begin{formula}[Matrix representation]
        The following formula gives the matrix representation of an operator $\hat{A}$ with respect to the orthonormal basis $\{|\psi_i\rangle\}_{i\leq n}$ cf. Construction \ref{linalgebra:matrix_representation}:
        \begin{gather}
            \label{wavematrix:matrix_entry}
            A_{ij} := \langle\psi_i|\hat{A}|\psi_j\rangle.
        \end{gather}
    \end{formula}
    \begin{remark}
         The basis $\big\{|\psi_i\rangle\big\}_{i\leq n}$ need not consist out of eigenfunctions of $\hat{A}$.
    \end{remark}

    ?? COMPLETE ??

\section{Uncertainty principle}

    \newdef{Compatible observables}{\index{observable!compatibility}
        Two observables are said to be compatible if they share a complete set of eigenvectors.
    }

    \newdef{Expectation value}{
        The expectation value of an operator $\hat{A}$ in a state $|\psi\rangle$ is defined as
        \begin{gather}
            \langle\hat{A}\rangle_\psi := \langle\psi|\hat{A}|\psi\rangle.
        \end{gather}
        The subscript $\psi$ is often left implicit. As in ordinary statistics \eqref{statistics:variance_without_sum}, the uncertainty or variance is defined as follows:
        \begin{gather}
            \Delta A := \langle\hat{A}^2\rangle - \langle\hat{A}\rangle^2.
        \end{gather}
    }

    \newformula{Uncertainty relation}{\index{Heisenberg!uncertainty relation}\index{Robertson|see{Heisenberg uncertainty relation}}
        Let $\hat{A},\hat{B}$ be two operator and let $\Delta A,\Delta B$ be the corresponding uncertainties. The (\textbf{Robertson}) uncertainty relation reads as follows:
        \begin{gather}
            \label{wavematrix:uncertainty_relation}
            \Delta A\Delta B\geq\frac{1}{4}\left|\left\langle\left[\hat{A},\hat{B}\right]\right\rangle\right|^2.
        \end{gather}
    }
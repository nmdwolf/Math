\chapter{Gauge Theory}

\section{Gauge invariance}

    Using the tools of differential geometry, as presented in chapter \ref{chapter:bundles} and onwards, we can introduce the general formulation of gauge theories and in particular Yang-Mills theories. Valuable references for these subjects are \cite{principal_bundles,sen_nash,schuller,gauge1}.

    Consider a general Lie group\footnote{This group is in the physics literature often called the \textbf{gauge group}.} $G$, acting on a Hilbert bundle\footnote{As explained in chapter \ref{chapter:classical_fields} this bundle is in general obtained as an associated bundle of the frame bundle (or a reduction thereof), possibly tensored with another principal bundle when we want extra symmetries.} $\mathcal{H}$ of physical states over a base manifold $M$. A general gauge transformation has the form
    \begin{gather}
        \label{qft:gauge_transformation}
        \psi'(x) = U(x)\psi(x)
    \end{gather}
    where $\psi, \psi':M\rightarrow\mathcal{H}$ are sections of the physical Hilbert bundle and $U:M\rightarrow G$ encodes the local behaviour of the gauge transformation ($U$ is assumed to be a unitary representation with respect to the Hilbert structure on $\mathcal{H}$). As such a gauge transformation constitutes a vertical automorphism of the Hilbert bundle.

    \begin{axiom}[Local gauge principle]
        The Lagrangian functional $\mathcal{L}[\psi]$ is invariant under the action of the gauge group $G$:
        \begin{gather}
            \mathcal{L}[U\psi] = \mathcal{L}[\psi].
        \end{gather}
    \end{axiom}

    Generally this gauge invariance can be achieved in the following way. Denote the Lie algebra corresponding to $G$ by $\mathfrak{g}$. Because the gauge transformation is local, the information on how it varies from point to point should be able to propagate through space. This is done by introducing a new field $B_\mu(x)$, called the \textbf{gauge field}. The most elegant formulation uses the concept of covariant derivatives:
    \newdef{Covariant derivative}{\index{covariant!derivative}\index{minimal!coupling}
        When gauging a certain symmetry group one replaces the ordinary partial derivatives by the following covariant derivative (this is called \textbf{minimal coupling}):
        \begin{gather}
            \mathcal{D}_\mu = \partial_\mu + igB_\mu(x)
        \end{gather}
        where $B_\mu:M\rightarrow\mathfrak{g}$ is a new field with values in the Lie algebra of the gauge group. Here we should also note that the explicit action of the covariant derivative depends on the chosen representation of $\mathfrak{g}$ on $\mathcal{H}$. Furthermore, one should pay attention to the fact that we used the physics convention where one multiplies\footnote{The imaginary unit turns anti-Hermitian fields into Hermitian fields.} the gauge field $B$ by a factor $ig$.
    }

    So to achieve gauge invariance one should replace all derivatives by the covariant derivative. However, for this to be a well-defined operation, one should check that the covariant derivative itself satisfies the local gauge principle, i.e. $\mathcal{D}'\psi' = U\mathcal{D}\psi$ (from here on we will suppress the coordinate dependence of all fields):
    \begin{align}
        U^{-1}\left(\pderiv{}{x^\mu} + igB_\mu'\right)\psi' &= U^{-1}\left(\pderiv{}{x^\mu} + igB_\mu'\right)U\psi\nonumber\\
        &= U^{-1}\pderiv{U}{x^\mu}\psi + \pderiv{\psi}{x^\mu} + igU^{-1}B_\mu'U\psi.
    \end{align}
    This expression can only be equal to $\mathcal{D}\psi$ if
    \begin{gather}
        igB_\mu = U^{-1}\pderiv{U}{x^\mu} + igU^{-1}B_\mu'U
    \end{gather}
    which can be rewritten as
    \begin{gather}
        B_\mu' = UB_\mu U^{-1} - \frac{1}{ig}(\partial_\mu U)U^{-1}
    \end{gather}
    or in coordinate-independent form as
    \begin{gather}
        \mathbf{B}' = U\mathbf{B}U^{-1} - \frac{1}{ig}dUU^{-1}.
    \end{gather}
    Up to conventions this is exactly the content of equations \ref{diff:prin:local_compatibility} and \ref{diff:prin:mc_pullback} appearing in the study of connections on principal bundles. This should not come as a surprise since the physical fields are sections of associated vector bundles and hence the principal bundle structure lurks in the background. We conclude that adding interactions is mathematically equivalent to coupling the physical manifold to a principal bundle.

    \begin{example}[QED]
        For quantum electrodynamics, which has U$(1)\cong S^1$ as its gauge group, we use the parametrization $U(x) = e^{ie\chi(x)}$ where $\chi:\mathbb{R}^n\rightarrow\mathbb{R}$. By minimal coupling we obtain
        \begin{align}
            \partial_\mu &\longrightarrow \mathcal{D}_\mu = \partial_\mu + ieA_\mu\\
            A_\mu &\longrightarrow A_\mu' = A_\mu - \partial_\mu\chi
        \end{align}
        where $A_\mu$ is the classic electromagnetic potential. These are indeed the formulas that we introduced in chapter \ref{chapter:maxwell}.
    \end{example}

\section{Spontaneous symmetry breaking}

    \begin{theorem}[Goldstone]\index{Goldstone bosons}
        Consider a QFT with Lie group $G$. Denote the generators of the corresponding Lie algebra by $\mathbf{X}_a$. Generators that do not destroy the vacuum\footnote{This corresponds to a transformation that leaves the vacuum invariant.}, i.e. $\mathbf{X}_av\neq0$, correspond to massless scalar particles.
    \end{theorem}
    The massless bosons from this theorem are called \textbf{Goldstone bosons}.

    ?? COMPLETE ??
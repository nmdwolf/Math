\chapter{Dirac Equation}\label{chapter:dirac}

    References for this chapter are~\citet{van_proeyen_supergravity_2012}. (Note that these authors use the mostly-pluses signature.) For the mathematical background of Clifford algebras and Spin-groups, see \cref{chapter:clifford} and, in particular, \cref{section:spin}. For the extension to (pseudo-)Riemannian manifolds, see \cref{section:spinor_bundles}.

    \minitoc

\section{Dirac matrices}

    \newdef{Dirac matrices}{\index{Dirac!algebra}\index{Dirac!matrix|seealso{gamma matrix}}\index{Dirac!basis}\index{Weyl!basis}\index{chiral!basis|see{Weyl basis}}
        The Dirac (or \textbf{gamma}) matrices are defined by the following equality:
        \begin{gather}
            \label{dirac:clifford_relation}
            \{\gamma^\mu,\gamma^\nu\}_+ = 2\eta^{\mu\nu}\mathbbm{1}\,,
        \end{gather}
        where $\eta^{\mu\nu}$ is the Minkowski metric. This has the form of \cref{clifford:inner_product}, i.e.~the Dirac matrices form the generating set of a Clifford algebra, called the \textbf{Dirac algebra}.

        There exist multiple distinct representations of the Clifford generators in signature $(1,3)$. The first one is called the \textbf{Dirac representation}. Here, the timelike Dirac matrix $\gamma^0$ is defined as
        \begin{gather}
            \gamma^0 :=
            \begin{pmatrix}
                \mathbbm{1}_2&0\\
                0&-\mathbbm{1}_2
            \end{pmatrix}\,.
        \end{gather}
        The spacelike Dirac matrices $\gamma^k$ ($k=1,2,3$) are defined using the Pauli matrices (\cref{angular_momentum:pauli_matrices}) $\sigma^k$:
        \begin{gather}
            \gamma^k :=
            \begin{pmatrix}
                0&\sigma^k\\
                -\sigma^k&0
            \end{pmatrix}\,.
        \end{gather}
        The \textbf{Weyl} or \textbf{chiral} representation\footnote{This representation is preferred in quantum field theory and supergravity.} is defined by replacing the timelike matrix $\gamma^0$ by
        \begin{gather}
            \gamma^0 :=
            \begin{pmatrix}
                0&\mathbbm{1}_2\\
                \mathbbm{1}_2&0
            \end{pmatrix}\,.
        \end{gather}
        In signature $(3,1)$, one obtains the Weyl representation by defining $\sigma^\mu:=(\mathbbm{1},\sigma_i)$ and $\overline{\sigma}^\mu:=(\mathbbm{1},-\sigma_i)$:
        \begin{gather}
            \gamma^\mu :=
            \begin{pmatrix}
                0&\sigma^\mu\\
                \overline{\sigma}^\mu&0
            \end{pmatrix}\,.
        \end{gather}
    }
    \sremark{In the remainder of this compendium, the Weyl representation will be used.}

    \newnot{Feynman slash notation}{\index{Feynman!slash}
        Let $\symbf{a}\equiv a^\mu\symbf{e}_\mu\in M^4$ be a general 4-vector. The Feynman slash is defined as follows:
        \begin{gather}
            \slashed{\symbf{a}} := a^\mu\gamma_\mu\,.
        \end{gather}
        In fact, this is just the embedding of Minkowski space in its Clifford algebra:
        \begin{gather}
            /:M^4\rightarrow C\ell(M^4,\eta):a^\mu\symbf{e}_\mu\mapsto a^\mu\gamma_\mu\,.
        \end{gather}
    }

\section{Spinors}
\subsection{Dirac equation}

    \newformula{Dirac equation}{\index{Dirac!equation}\index{Weyl!equation}\label{dirac:dirac_equation}
        In covariant form, the Dirac equation reads as
        \begin{gather}
            (i\hbar\slashed\partial - mc)\psi = 0\,,
        \end{gather}
        where $m$ denotes the mass and $c$ denotes the speed of light. In the case of massless particles, the \textbf{Weyl equation} is recovered:
        \begin{gather}
            \slashed{\partial}\psi = 0\,.
        \end{gather}
    }

    \newdef{Dirac adjoint}{\index{Dirac!adjoint}
        \begin{gather}
            \overline{\psi} := i\psi^\dag\gamma^0
        \end{gather}
        When working in the Dirac representation, the factor $i$ should be dropped.
    }
    \newdef{Majorana adjoint}{\index{Majorana!adjoint}
        In the context of supersymmetry, it is often convenient to work with a different adjoint spinor. Let $\widehat{\mathcal{C}}:=i\gamma^3\gamma^1$ denote the charge conjugation operator. The Majorana adjoint is defined by
        \begin{gather}
            \overline{\psi} := \psi^t\widehat{\mathcal{C}}\,.
        \end{gather}
    }

    \newformula{Parity}{\index{parity}
        The parity operator is defined as follows:
        \begin{gather}
            \widehat{P}(\psi) := \gamma^0\psi\,.
        \end{gather}
    }

\subsection{Chiral spinors}

    In an even number of dimensions, one can define an additional matrix that satisfies \cref{dirac:clifford_relation}.
    \newdef{Chiral matrix}{\index{chiral!matrix}\index{helicity}
        Assume that the dimension $d=m+n$ is even. The chiral (helicity) matrix can be defined as follows:\footnote{Some authors add a constant factor to this definition.}
        \begin{gather}
            \gamma_{d+1} := \gamma_1\gamma_2\cdots\gamma_d\,.
        \end{gather}
        In an odd number of dimensions ($d=2m+1$), a generating set for the Clifford algebra can be obtained by taking the generating set from one dimension lower and adjoining the element $k\gamma_\ast$ where $k^2=(-1)^{n+d/2}$. This gives two inequivalent representations of the Clifford algebra (depending on the sign). From here on, the following redefinition will be used:
        \begin{gather}
            \gamma_{d+1}\longrightarrow k\gamma_{d+1}\,.
        \end{gather}
        This has the benefit that $\gamma_{d+1}^2 = \mathbbm{1}$. In $d=3+1$, one generally takes the following representation for $\gamma_5$:\footnote{Such a block diagonal form can always be chosen by working in a \textit{helicity-adapted basis}.}
        \begin{gather}
            \gamma_{d+1} :=
            \begin{pmatrix}
                \mathbbm{1}&0\\
                0&-\mathbbm{1}
            \end{pmatrix}\,.
        \end{gather}
        \todo{CHECK THIS MATRIX (currently equal to $\gamma^0$)}
    }

    \newdef{Chiral projection}{\index{helicity}
        The chiral projections of a spinor $\psi$ are defined as follows:
        \begin{gather}
            \psi_- := \frac{1+\gamma_{d+1}}{2}\psi
        \end{gather}
        and
        \begin{gather}
            \psi_+ := \frac{1-\gamma_{d+1}}{2}\psi\,.
        \end{gather}
        Every spinor can then be written as a sum of its chiral parts:
        \begin{gather}
            \psi = \psi_+ + \psi_-\equiv
            \begin{pmatrix}
                \psi_+\\\psi_-    
            \end{pmatrix}\,.
        \end{gather}
    }

    In general, since the Dirac equation is linear, it has a plane wave solution of the form
    \begin{gather}
        \psi(x) = u(p)e^{-ip\cdot x} + v(p)e^{ip\cdot x}\,,
    \end{gather}
    the first term corresponding to (incoming) fermions and the second corresponding to (outgoing) antifermions. (Taking the Dirac adjoint switches incoming and outgoing states.) For massless particles, the Dirac equation reduces to the Weyl equation and has two solutions (of opposite \textbf{helicity}):
    \begin{gather}
        \label{dirac:chiral_solutions}
        \slashed{p}u_\pm(p) = 0\qquad\text{and}\qquad\slashed{p}v_\pm(p)=0\,,
    \end{gather}
    where the helicity is defined as follows:
    \begin{gather}
        \widehat{\lambda} := \frac{\widehat{p}\cdot\vector{\sigma}}{|\widehat{p}|}\,.
    \end{gather}
    Since massless particles move at the speed of light, their helicity is independent of a choice of reference frame and, hence, the helicity is a well-defined quantity. Moreover, in the case of massless particles, a \textit{crossing symmetry} is found:\index{crossing symmetry}
    \begin{gather}
        u_\pm = v_\mp\,.
    \end{gather}

    \begin{property}[Bispinor representation]\label{dirac:exceptional_isomorphism}
        The exceptional isomorphism $\mathrm{Spin}(3,1)\cong\mathrm{SL}(2,\mathbb{C})$ from \cref{clifford:exceptional_isomorphisms} allows to express the Lorentz norm as a determinant:
        \begin{gather}
            2\|(x^\mu)\| = \det(\slashed{x}) = 2\det((x^{\alpha\beta}))\,,
        \end{gather}
        where
        \begin{gather}
            x_{\alpha\beta} := x_\mu(\sigma^\mu)_{\alpha\beta}\,.
        \end{gather}
        It follows that if $x$ is lightlike, it is equivalent to the outer product of a (Weyl) spinor $\xi$ and its conjugate $\overline{\xi}$:
        \begin{gather}
            x^{\alpha\beta} = u_+^\alpha\overline{u}_+\vphantom{u}^\beta\,.
        \end{gather}
    \end{property}

    For future use, some new notations are introduced (akin to the standard Dirac notation for quantum states).
    \begin{notation}[Weyl notation]
        By the property above, every lightlike 4-vector can be expressed as the product of a Weyl spinor and its adjoint. Given this representation, one defines
        \begin{gather}
            \begin{aligned}
                \ket{p} &:= u_+(p)\,,\\
                [p|&:= \overline{u}_+(p)\,, 
            \end{aligned}
        \end{gather}
        and, accordingly,
        \begin{gather}
            \begin{aligned}
                \bra{p} &= \overline{u}_-(p)\,,\\
                |p] &= u_-(p)\,, 
            \end{aligned}
        \end{gather}
        This also implies that $\slashed{p}$, which essentially contains two copies of $(p ^{\alpha\beta})$ can also be expressed in the bispinor representation through the following completeness relation:\footnote{The addition in the first step should be interpreted as a direct sum.}
        \begin{gather}
            \slashed{p} = \ket{p}[p| + |p]\bra{p} = u_+(p)\overline{u}_+(p) + u_-(p)\overline{u}_-(p)\,.
        \end{gather}
        The Weyl equation itself becomes:
        \begin{gather}
            p^{\alpha\beta}\ket{p}_\beta = 0\,.
        \end{gather}
    \end{notation}

    Using the Weyl notation, one can also introduce spinor bilinears that are similar in nature to inner products, with the main difference that they are antisymmetric:
    \begin{gather}
        \label{dirac:spinor_bilinears}
        \begin{aligned}
            \langle p\,q\rangle &:= \braket{p}{q} = \overline{u}_-(p)u_+(q)\,,\\
            [p\,q] &:= [p\mid q] = \overline{u}_+(p)u_-(q)\,.
        \end{aligned}
    \end{gather}

    \begin{remark}
        Mixed bilinears, with one angle variable and one bracket variable, vanish. In fact, if they would not vanish, they would not be Lorentz invariant. To obtain an invariant quantity, a Dirac matrix should be introduced, e.g.~$\bra{p}\gamma^\mu|q]$.
    \end{remark}

    \begin{property}[Inner products]
        The inner product of two lightlike 4-vectors can be expressed in terms of spinor bilinears:
        \begin{gather}
            p\cdot q = \frac{1}{2}(p+q)^2 = \frac{1}{2}\langle p\,q\rangle[p\,q]\,.
        \end{gather}
    \end{property}
    \begin{property}[Schouten identity]\index{Schouten!identity}
        Consider four lightlike 4-vectors $\{p_1,p_2,p_3,p_4\}$. Given the bispinor decomposition $p_i=\ket{i}[i|$, the following equality is satisfied:
        \begin{gather}
            \langle1\,2\rangle\langle3\,4\rangle + \langle1\,3\rangle\langle4\,2\rangle + \langle1\,4\rangle\langle2\,3\rangle = 0\,.
        \end{gather}
    \end{property}

\subsection{\texorpdfstring{Dirac algebra in $d=4$}{Dirac algebra in d=4}}

    For a lot of calculations, especially in quantum electrodynamics, one needs the properties of the gamma matrices. The most relevant relations in $d=3+1$ are listed below:
    \begin{formula}[Trace algebra]
        \begin{align}
            \tr(\gamma^\mu) = \tr(\gamma^\mu\gamma^\nu\gamma^\rho) &= 0\\
            \tr(\gamma^\mu\gamma^\nu) &= 4\eta^{\mu\nu}\\
            \tr(\gamma^\mu\gamma^\nu\gamma^\kappa\gamma^\lambda) &= 4(\eta^{\mu\nu}\eta^{\kappa\lambda} - \eta^{\mu\kappa}\eta^{\nu\lambda} + \eta^{\mu\lambda}\eta^{\nu\kappa})\\
            \tr(\gamma^5) = \tr(\gamma^\mu\gamma^5) = \tr(\gamma^\mu\gamma^\nu\gamma^5) = \tr(\gamma^\mu\gamma^\nu\gamma^\rho\gamma^5)&= 0\\
            \tr(\gamma^\mu\gamma^\nu\gamma^\kappa\gamma^\lambda\gamma^5) &= -4i\varepsilon^{\mu\nu\kappa\lambda}\\
            \tr(\gamma^{\mu_1}\cdots\gamma^{\mu_k}) &= \tr(\gamma^{\mu_k}\cdots\gamma^{\mu_1})
        \end{align}
    \end{formula}

    \begin{formula}[Contraction identities]
        \begin{align}
            \gamma^\mu\gamma_\mu &= 4\\
            \gamma^\mu\gamma^\nu\gamma_\mu &= -2\gamma^\nu\\
            \gamma^\mu\gamma^\nu\gamma^\rho\gamma_\mu &= 4\eta^{\nu\rho}\\
            \gamma^\mu\gamma^\nu\gamma^\kappa\gamma^\lambda\gamma_\mu &= -2\gamma^\lambda\gamma^\kappa\gamma^\nu
        \end{align}
    \end{formula}

\subsection{Fierz identities}\index{Fierz identity}

    Using a spinor $u\in S$ and a cospinor $\overline{v}\in S^*$, one can build a bilinear form $\overline{v}u$. However, for two spinors $u,\omega$ and two cospinors $\overline{v},\overline{\rho}$, one can interpret the expression $(\overline{v}u)(\overline{\rho}\omega)$ either as a quadrilinear form on $u\otimes\overline{v}\otimes\omega\otimes\overline{\rho}$ or as a quadrilinear form on $\omega\otimes\overline{v}\otimes u\otimes\overline{\rho}$. Because $C\ell_{3,1}(\mathbb{C})$ is isomorphic to the endomorphism ring on $S$, there must exist coefficients $a^{ij}$, with $i,j=1,\ldots,2^d$, such that
    \begin{gather}
        (\overline{v}u)(\overline{\rho}\omega) = \sum_{i,j=1}^{2^d}a^{ij}(\overline{v}\gamma_i\omega)(\overline{\rho}\gamma_ju)\,.
    \end{gather}
    By using the orthogonality relations of the trace, one can find that
    \begin{gather}
        \alpha^{ij} =
        \begin{cases}
            0&\cif i\neq j\,,\\
            \frac{1}{2^{\lfloor d/2 \rfloor}}&\cif i=j\,.
        \end{cases}
    \end{gather}
    The above equality can then also be rewritten as follows:
    \begin{gather}
        \delta_b^a\delta_d^c = \frac{1}{2^{\lfloor d/2 \rfloor}}\sum_{i=1}^{2^d}(\gamma_i)_d^a(\gamma_i)_b^c\,.
    \end{gather}
    This expression, and the techniques used to find it, allows one to rearrange almost any multilinear expression involving spinors and cospinors.
\chapter{Quantum Information Theory}\label{chapter:quantum_computing}

    The section on (quantum) reference frames is based on~\citet{de_la_hamette_quantum_2020}.

    \minitoc

\section{Entanglement}
\subsection{Schmidt decomposition}

    \begin{construct}[Schmidt decomposition]\index{Schmidt!decomposition}\index{rank!Schmidt}
        Consider a bipartite state $\ket{\psi}\in\mathcal{H}_1\otimes\mathcal{H}_2$. For any such state, there exist orthonormal sets $\bigl\{\ket{e_i},\ket{f_j}\bigr\}_{i,j\leq\kappa}$ such that
        \begin{gather}
            \ket{\psi} = \sum_{i=1}^\kappa\lambda_i\ket{e_i}\otimes\ket{f_i}\,,
        \end{gather}
        where the coefficients $\lambda_i$ are nonnegative real numbers. All objects in this expression can be obtained from a singular value decomposition of the coefficient matrix $\mathbf{C}$ of $\ket{\psi}$ in some bases of $\mathcal{H}_1$ and $\mathcal{H}_2$. The number $\kappa$ is called the \textbf{Schmidt rank} of $\ket{\psi}$.
    \end{construct}

    \newdef{Entangled states}{\index{separable!state}\index{entanglement}
        Consider a state $\ket{\psi}$ and consider its Schmidt decomposition. If the Schmidt rank is 1, i.e.~the state can be written as $\ket{\psi} = \ket{v}\otimes\ket{w}$, the state is said to be \textbf{separable}. Otherwise, the state is said to be entangled.
    }

\subsection{Bell states}

    \newdef{Bell state}{\index{Bell state}\index{Einstein--Podolsky--Rosen|see{Bell state}}
        A (binary) Bell state (also called a \textbf{cat state} or \textbf{Einstein--Podolsky--Rosen pair}) is defined as the following entangled state:
        \begin{gather}
            \ket{\Phi^+} := \frac{1}{\sqrt{2}}\bigl(\ket{00}+\ket{11}\bigr)\,.
        \end{gather}
        In fact, this state can be extended to a full maximally entangled basis for the 2-qubit Hilbert space:
        \begin{gather}
            \begin{aligned}
                \ket{\Phi^-} &:= \frac{1}{\sqrt{2}}\bigl(\ket{00}-\ket{11}\bigr)\,,\\
                \ket{\Psi^+} &:= \frac{1}{\sqrt{2}}\bigl(\ket{01}+\ket{10}\bigr)\,,\\
                \ket{\Psi^-} &:= \frac{1}{\sqrt{2}}\bigl(\ket{01}-\ket{10}\bigr)\,.
            \end{aligned}
        \end{gather}
    }

    \begin{method}[Dense coding\footnotemark]\index{dense!coding}\index{action!spooky}
        \footnotetext{Sometimes called \textbf{superdense coding}.}
        Consider the Bell state $\ket{\Phi^+}$. By acting with one of the (unitary) spin-flip operators $X,Y,Z$, one can obtain any of the other three Bell states:
        \begin{gather}
            \begin{aligned}
                X\ket{\Phi^+} &= \ket{\Phi^-}\,,\\
                Y\ket{\Phi^+} &= \ket{\Psi^+}\,,\\
                Z\ket{\Phi^+} &= \ket{\Psi^-}\,.
            \end{aligned}
        \end{gather}
        In a typical Alice-and-Bob-style experiment, one can ask whether this observation allows to achieve a better-than-classical communication channel. If Alice performs a spin flip on her qubit, although the resulting state has instantly `changed' (cf.~\textit{spooky action at a distance}), Bob still cannot uniquely determine what this state is (since the resulting state is still maximally entangled). However, if Alice sends her qubit to Bob, the latter can perform a measurement on the composite system to find out what the state is and in this way determine which operation Alice performed ($\mathbbm{1},X,Y,Z$). Alice has thus effectively sent 2 classical bits of information through 1 qubit. Note that due to the fact that Alice still has to send her qubit through classical means, no faster-than-light communication is achieved.
    \end{method}

    \newdef{GHZ state}{\index{Greenberger--Horne--Zeilinger state}
        The Greenberger--Horne--Zeilinger state is defined as the multiparticle qudit ($d,N>2$) version of the Bell state and is, therefore, also referenced to as a cat state:
        \begin{gather}
            \ket{\mathrm{GHZ}} = \frac{1}{\sqrt{d}}\sum_{i=0}^{d-1}\ket{i}^{\otimes N}\,.
        \end{gather}
    }

\subsection{SRE states}\label{section:sre_states}

    \todo{ADD}

\section{Density operators}\index{density!operator}

    \newdef{Density operator}{
        Consider a (finite-dimensional) Hilbert space $\mathcal{H}$. A density operator on $\mathcal{H}$ is a linear operator $\rho\in\End(\mathcal{H})$ satisfying the following properties:
        \begin{enumerate}
            \item\textbf{Positivity}: $\braket{v}{\rho v}\geq0$ for all $v\in\mathcal{H}$,
            \item\textbf{Hermiticity}: $\rho^\dag=\rho$, and
            \item\textbf{Unit trace}: $\tr(\rho)=1$.
        \end{enumerate}
        More concisely, density operators are the representing objects of normal states (\cref{operators:normal_state}) on $\mathcal{B}(\mathcal{H})$.
    }

    \begin{example}[Classical probability]
        A diagonal density matrix corresponds to a (classical) discrete probability distribution.
    \end{example}

    \newdef{Pure state}{\index{pure}
        A state is said to be pure if it is described by an outer product of a state vector or, equivalently, by an idempotent density matrix. A density matrix that is not of this form gives rise to a \textbf{mixed state}.
    }

	\newdef{Reduced density operator}{
		Let $\ket{\Psi}\in\mathcal{H}_A\otimes\mathcal{H}_B$ be the state of a bipartite system. The reduced density operator $\rho_A$ of $A$ is defined as follows:
		\begin{gather}
			\rho_A := \tr_B\ket{\Psi}\bra{\Psi}\,.
		\end{gather}
	}

	\newdef{Purification}{\index{purification}
		Let $\rho_A$ be the density operator of a system $A$. A purification of $\rho_A$ is a pure state $\ket{\Psi}$ of some composite system $AB$ such that
		\begin{gather}
			\rho_A = \tr_B\ket{\Psi}\bra{\Psi}\,.
		\end{gather}
	}
	\begin{property}
		Any two purifications of the same density operator $\rho_A$ are related by a transformation $\mathbbm{1}_A\otimes\hat{V}$ with $\hat{V}$ an isometry.
	\end{property}

\section{Operations}

    The following definition generalizes the content of \cref{section:PVM} to a setting of partial information.
    \newdef{Positive operator-valued measure}{\index{measurement!positive-operator valued}
        First, let $\mathcal{H}$ be a finite-dimensional Hilbert space. A POVM on $\mathcal{H}$ consists of a finite set of positive (semi)definite operators $\{P_i\}_{i\leq n}$ such that
        \begin{gather}
            \sum_{i=1}^nP_i=\mathbbm{1}_{\mathcal{H}}\,.
        \end{gather}
        The probability to obtain state $i$, given a general state $\widehat{\rho}$, is given by $\tr(\widehat{\rho}P_i)$. Note that the operators are not necessarily orthogonal projectors, so $n$ can be greater than $\dim(\mathcal{H})$.

        Now, consider a measurable space $(X,\Sigma)$ and a (possibly infinite-dimensional) Hilbert space $\mathcal{H}$. A POVM on $X$ consists of a function $P:\Sigma\rightarrow\mathcal{B}(\mathcal{H})$ satisfying the following conditions:
        \begin{enumerate}
            \item $P_E$ is positive and self-adjoint for all $E\in\Sigma$,
            \item $P_X=\mathbbm{1}_{\mathcal{H}}$, and
            \item for all disjoint $\seq{E}\subset\Sigma$:
                \begin{gather}
                    \sum_{n\in\mathbb{N}}P_{E_n} = P_{\cup_{n\in\mathbb{N}}E_n}\,.
                \end{gather}
        \end{enumerate}
    }

    The following theorem can be derived from Stinespring's theorem~\ref{operators:stinespring}.
    \begin{theorem}[Naimark dilation theorem]\index{Naimark dilation theorem}
        Every POVM $P$ on $\mathcal{H}$ can be realized as a PVM $\Pi$ on a, possibly larger, Hilbert space $\mathcal{K}$, i.e.~there exists a bounded operator $V:\mathcal{K}\rightarrow\mathcal{H}$ such that
        \begin{gather}
            P(\cdot) = V\Pi(\cdot)V^\dagger\,.
        \end{gather}
        In the finite-dimensional setting, $V$ can be chosen to be an isometry.
    \end{theorem}

    \newdef{Completely positive trace-preserving}{\index{quantum!channel}
        Consider a map $\Phi:\mathcal{B}(\mathcal{H}_1)\rightarrow\mathcal{B}(\mathcal{H}_2)$ between (trace-class) operators on two (finite-dimensional) Hilbert spaces. This map preserves density matrices if it positive (\cref{operators:positive_map}) and if it is trace-preserving (\cref{operators:trace_preserving}). Furthermore, to ensure that an operation applied to a subsystem does not interfere with the positivity of the complete system, they are also required to be completely positive (\cref{operators:cp_map}).

        Completely positive, trace-preserving (CPTP) maps are often called \textbf{quantum channels}.
    }

    The following property can also be derived from the Stinespring theorem~\ref{operators:stinespring}.
    \begin{property}[Kraus decomposition]\index{Kraus decomposition}\label{qc:kraus}
        Let $\mathcal{H}_1,\mathcal{H}_2$ be Hilbert spaces of dimensions $m$ and $n$, respectively. A linear map $\Phi:\mathcal{B}(\mathcal{H}_1)\rightarrow\mathcal{B}(\mathcal{H}_2)$ is completely positive if and only if there exist bounded operators $\{A_i\}_{i\leq mn}$ such that
        \begin{gather}
            \Phi(B) = \sum_{i=1}^{mn}A^\dag_iBA_i\,.
        \end{gather}
        Furthermore, it is trace-preserving if and only if
        \begin{gather}
            \sum_{i=1}^{mn}A^\dag_iA_i = \mathbbm{1}\,.
        \end{gather}
        A decomposition of the above form is also often called an \textbf{operator-sum decomposition}.
    \end{property}

\section{Quantum reference frames}\index{reference frame!quantum}

    An important notion in classical physics is that of a \textit{reference frame}, i.e.~a choice of axes and scales. Usually, this corresponds to choosing an observer, relative to which one expresses the motion of all other objects. In relativity, the relative treatment of physics was the grand breakthrough by Einstein. However, although this notion had been left aside for a long time in the treatment of quantum mechanics and a specific choice of reference frame was silently assumed, this assumption was not as innocuous as it appears. Superposition and complementarity make a definite choice of absolute reference frame impossible.

    For example, consider three observers: Alice, Bob and Charlie. Assume that each observer has a spin-$\tfrac{1}{2}$ particle and that, relative to Alice, the joint state is given by
    \begin{gather}
        \ket{\psi}^A_{ABC} = \ket{\uparrow}^A_A\left(\ket{\uparrow}^A_B+\ket{\downarrow}^A_B\right)\ket{\downarrow}^A_C\,.
    \end{gather}
    Note that this state is separable. Now, what would the state be relative to Bob? If one supposes that changes of reference frame are \textit{coherent} (to be formalized below), the joint state will be
    \begin{gather}
        \ket{\psi}^B_{ABC} = \ket{\uparrow}^B_B\left(\ket{\uparrow}^B_A\ket{\downarrow}^B_C+\ket{\downarrow}^B_A\ket{\uparrow}^B_C\right)\,.
    \end{gather}
    A mere change of reference frame, an operation that would classically leave the physics invariant, has transformed a product state into an entangled state.

    \begin{axiom}[Relational physics]
        Given $n\in\mathbb{N}$ systems\footnote{An abstraction of the notion of observer.}, any state is described relative to one of these systems. Given a choice of `observing system', let it be system $i$, the state of system $i$ is given by a fiducial state $\ket{0}^i_i$.
    \end{axiom}

    \begin{axiom}[Coherent change]\index{coherence}
        Consider a change of reference frame $0\longrightarrow i$ such that
        \begin{gather}
            \begin{cases}
                &\ket{\psi}^0\longrightarrow\ket{\psi}^i\\
                &\ket{\phi}^0\longrightarrow\ket{\phi}^i\,.
            \end{cases}
        \end{gather}
        Then
        \begin{gather}
            \alpha\ket{\psi}^0+\beta\ket{\phi}^0\longrightarrow\alpha\ket{\psi}^i+\beta\ket{\phi}^i
        \end{gather}
        for all $\alpha,\beta\in\mathbb{C}$.
    \end{axiom}

    Abstractly, a (classical) reference frame is defined as follows in the spirit of \cref{section:smooth_spaces} and \cref{section:space_and_quantity}.
    \newdef{Reference frame}{\index{reference frame}
        Let $X$ be an object of interest. Whereas a coordinate chart on $X$, modeled on an object $Y$, is given by a morphism $Y\rightarrow X$, a \textbf{coordinate system} on $X$ is given by an isomorphism $Y\cong X$, i.e.~a global coordinate chart. A reference frame is coordinate system for which $Y$ corresponds the a physical system.
    }

    Let the system of interest $X$ admit a group action that is both free and transitive, turning it into a $G$-torsor (\cref{group:torsor}). At the level of sets, one has $X\cong G$ and a choice of origin, i.e.~a specific choice of isomorphism, corresponds to a choice of reference frame (the identity element corresponding to the fiducial state above). A change of reference frames $s^0\longrightarrow s^i$, frome system $0$ to system $i$, is given by the right regular action of the relative coordinate of $i$ on all relative coordinates:
    \begin{gather}
        \phi^{0\longrightarrow i}(e,g^0_1,\ldots,g^0_n)\mapsto(g^i_0,g^0_1g^i_0,\ldots,e,\ldots,g^0_ng^i_0)\,,
    \end{gather}
    where the relation $g^0_i=(g^i_0)^{-1}$ was used. It should be noted that this boils down to a \textit{passive transformation}. When passing to the quantization of these systems, one should assume that $G$ is locally compact and comes equipped with the canonical Haar measure (\cref{distribution:haar_theorem}). In this case, a quantization is given by the space of square-integrable functions $L^2(G)$, where basis states are labeled by group elements.

    \todo{VERIFY THIS STATEMENT}

    The change-of-reference-frame operator is given as follows:
    \begin{gather}
        \widehat{U}^{0\longrightarrow i} := \mathrm{SWAP}_{0,i}\circ\Int_G\mathbbm{1}_{L^2(G)}\otimes \widehat{U}_R(g_i^0)^{\otimes i-2}\otimes\ket{g^i_0}\bra{g^0_i}\otimes\widehat{U}_R(g_i^0)^{\otimes n-i-2}\,dg^0_i\,,
    \end{gather}
    where
    \begin{gather}
        \widehat{U}_R(g):\ket{x}\mapsto\ket{xg^{-1}}
    \end{gather}
    is the unitary implementation of the right regular action and $dg$ denotes integration with respect to the Haar measure on $G$. It can be shown that $\widehat{U}^{0\longrightarrow i}$ is unitary, its inverse being given by $\widehat{U}^{i\longrightarrow 0}$ and composition is transitive. It can be shown that this procedure can be extended to any one-particle Hilbert space $\mathcal{H}$ as long as the inclusion $G\rightarrow\mathcal{H}$ is injective and maps $G$ to an orthonormal basis of (a subset of) $\mathcal{H}$.

    One of the applications of QRFs is the resolution of the following paradox.
    \newdef{Wigner's friend}{\index{Wigner!friend}
        Consider two observers, Wigner and his friend, performing an experiment as shown diagrammatically in \cref{fig:wigners_friend}. One envisions Wigner standing outside the laboratory, having no way to observe what happens inside the lab, and his friend who performs an experiment inside the lab. The paradox arises from the two ways one can describe the sequence of the friend performing a measurement and Wigner checking up on the results in the classical (Copenhagen) interpretation.

        \begin{figure}[ht!]
            \centering
            \begin{tikzpicture}
                \draw (0, 0) rectangle (10, 6) node[above]{universe};
                \draw (5, 1) rectangle (9, 5) node[above]{lab};
                \draw[fill = black] (2.5, 3) circle (.1) node[above]{Wigner};
                \draw[fill = black] (6.5, 4) circle (.1) node[above]{friend};
                \draw[fill = blue] (7.5, 2) circle (.1) node[below, color = blue]{experiment};
            \end{tikzpicture}
            \caption{Wigner's friend thought experiment.}
            \label{fig:wigners_friend}
        \end{figure}

        From the point of view of the friend, the measurement (projection) axiom states that the total wave function collapses at the time of the experiment. However, from the point of view of Wigner, the collapse only happens as soon as Wigner decides to walk into the laboratory.

        \todo{COMPLETE}
    }
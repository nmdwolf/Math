\chapter{Special Relativity}

In this and the following chapters we adopt the standard Minkowskian signature $(+, -, -, -)$ unless otherwise stated. This follows the introductory literature such as \cite{Peskin, greiner_qft}. Furthermore, we also work in natural units unless stated otherwise, i.e. $\hbar = c = 1$.

\section{Lorentz transformations}

	\begin{notation}
		In the context of special relativity it is often useful to introduce the following notation:
		\begin{gather}
			\beta = \frac{v}{c}.
		\end{gather}
	\end{notation}
	\newnot{Lorentz factor}{\index{Lorentz!factor}
	    	\begin{gather}
			\label{rel:lorentz_factor}
        		\gamma = \stylefrac{1}{\sqrt{1 - \beta^2}}
		\end{gather}
	}
	\newformula{Lorentz transformations}{
	    	Let $\mathbf{V}$ be a 4-vector. A Lorentz boost along the $x^1$-axis is given by the following transformation:
	    	\begin{gather}
	        	\label{rel:lorentz_transformations}
	        	\begin{array}{ccl}
				V'^0 &=& \gamma\left(V^0 - \beta V^1\right)\\
		                V'^1 &=& \gamma\left(V^1 - \beta V^0\right)\\
		                V'^2 &=& V^2\\
		                V'^3 &=& V^3
			\end{array}
		\end{gather}
	}
	\begin{remark}
		Putting $c=+\infty$ in the previous formulas recovers the Galilean transformations from classical mechanics.
	\end{remark}
    
\section{Energy and momentum}\index{energy}\index{momentum}

	\newformula{4-velocity}{
		In analogy to the definition of velocity in classical mechanics we define the 4-velocity as follows:
	    	\begin{gather}
        		U^\mu = \left(\deriv{x^0}{\tau}, \deriv{x^1}{\tau}, \deriv{x^2}{\tau}, \deriv{x^3}{\tau}\right).
    		\end{gather}
        	By applying the formulas for proper time and time dilatation we obtain:
        	\begin{gather}
    			\label{rel:4_velocity}
        		U^\mu = \left(\gamma c, \gamma \vector{u}\right).
	    	\end{gather}
	}
	\newformula{4-momentum}{
	    	\begin{gather}
			\label{rel:4_momentum}
		        p^\mu = m_0U^\mu
	    	\end{gather}
        	or by setting $E = cp^0$:
        	\begin{gather}
        		p^\mu = \left(\frac{E}{c}, \gamma m_0 \vector{u}\right)
        	\end{gather}
	}

	\newdef{Relativistic mass}{
	    	The factor $\gamma m_0$ in the momentum 4-vector is called the relativistic mass. By introducing this quantity (and denoting it by $m$), the classic formula $\vector{p} = m\vector{u}$ for the 3-momentum can be generalized to 4-momenta $p^\mu$.
	}

	\begin{formula}[Relativistic energy relation]\index{Einstein!energy relation}
		\begin{gather}
	        	\label{forces:relativistic_energy}
		        E^2 = p^2c^2 + m^2c^4
		\end{gather}
	        This formula is often called the \textbf{Einstein relation}.
	\end{formula}



\chapter{General Relativity}

	\newdef{Stationary spacetime}{
		A spacetime $g_{\mu\nu}$ is called stationary if there exists a timelike Killing vector\footnote{See definition \ref{diff:killing_vector}.}. By the \textit{flowbox theorem} there always exists a coordinate chart such that locally one can choose the Killing vector field to be $\partial_0$ and hence we see that a spacetime is stationary if we can find a coordinate system for which the metric coefficients are time-independent.
	}

\section{Einstein field equations}

	\newformula{Einstein field equations}{\index{Einstein!field equations}\index{stress-energy tensor!reduced}
	    	The Einstein field equations without the cosmological constant $\Lambda$ read:
	    	\begin{gather}
	    		\label{rel:einstein_field_equations}
		        G_{\mu\nu} = \stylefrac{8\pi G}{c^4}T_{\mu\nu}
	    	\end{gather}
	    	where $G_{\mu\nu}$ is the Einstein tensor \ref{diff:manifolds:einstein_tensor} and $T_{\mu\nu}$ is the stress-energy tensor \ref{relativity:stress_energy_tensor}.
	    	
	    	By taking the trace of both sides one obtains $T = -R$ (in units $c = 1, G = \frac{1}{8\pi}$) and hence we can rewrite the Einstein field equations as:
	    	\begin{gather}
	    		R_{\mu\nu} = \hat{T}_{\mu\nu}
	    	\end{gather}
	    	where $\hat{T}_{\mu\nu} = T_{\mu\nu} - \frac{1}{2}g_{\mu\nu}T$ is the \textbf{reduced stress-energy tensor}.
	}
	
	\begin{formula}[Einstein-Hilbert action]\index{Einstein-Hilbert}
		The (vacuum) field equations can be obtained by applying the variational principle to the following action:
		\begin{gather}
			S_{EH}[g_{\mu\nu}] = \int_M\sqrt{-g}R
		\end{gather}
		When working on noncompact manifolds (such as Minkowski space) one needs an extra term to make the boundary contributions vanish. This term is due to Gibbons and Hawking:
		\begin{gather}
			S_{GH}[g_{\mu\nu}] = \oint_{\partial M}\epsilon\sqrt{h}K
		\end{gather}
		where $h_{ab}$ is the induced metric on the boundary, $K_{ab}$ is the extrinsic curvature and $\epsilon = \pm1$ is a constant depending on whether the boundary is timelike or spacelike.
	\end{formula}
    
\section{Schwarzschild metric}

	\newformula{Schwarzschild metric}{\index{Schwarzschild}
	    	\begin{gather}
	    		\label{rel:schwarzschild_metric}
		        ds^2 = \left(1-\stylefrac{R_s}{r}\right)c^2dt^2 - \left(1-\stylefrac{R_s}{r}\right)^{-1}dr^2 - r^2d\Omega^2
	    	\end{gather}
        	where $R_s$ is the Schwarzschild radius given by
        	\begin{gather}
        		R_s = \stylefrac{2GM}{c^2}
        	\end{gather}
	}
    
	\begin{theorem}[Birkhoff]\index{Birkhoff}
	    	The Schwarzschild metric is the unique solution of the vacuum field equation under the additional constraints of asymptotic flatness and staticity.
	\end{theorem}
	
	\newformula{Reissner-Nordstr\"om metric}{
		If we allow the black hole to have an electric charge $Q$, the Schwarzschild metric is modified:
		\begin{gather}
			ds^2 = \left(1-\stylefrac{2GM}{r} + \frac{GQ^2}{4\pi r^2}\right)c^2dt^2 - \left(1-\stylefrac{2GM}{r} + \frac{GQ^2}{4\pi r^2}\right)^{-1}dr^2 - r^2d\Omega^2
		\end{gather}
	}
	
	\begin{remark}
		A computation of the electric field generated by the black hole gives us:
		\begin{gather}
			E^r = \frac{Q}{4\pi r^2}.
		\end{gather}
		Although the coordinate $r$ is not the proper distance, it still acts as a parameter for the surface of a sphere (as it does in a Euclidean or Schwarzschild metric). This explains why the above formule is the same as the one in classical electromagnetism.
	\end{remark}
	
\section{Causal structure}

	\newdef{Null coordinate}{\index{null!coordinate}
		Consider a coordinate $x$ on a pseudo-Riemannian manifold $(M,g)$. This coordinate is said to be a null coordinate (or a \textbf{lightlike} coordinate) if it satisfies the following condition at every point in $M$:
		\begin{gather}
			g\left(\pderiv{}{x}, \pderiv{}{x}\right) = 0
		\end{gather}
	}
	\newdef{Causal curve}{
		Consider a curve $\gamma:]0, 1[\rightarrow (M, g)$. This curve is said to be causal if it satisfies\footnote{Here we adopt the mostly minus convention.}
		\begin{gather}
			g\left(\dot{\gamma}(t), \dot{\gamma}(t)\right) \geq 0
		\end{gather}
		for all $t\in]0, 1[$. If the inequality is replaced by a strict inequality then we obtain the notion of a \textbf{timelike} curve. Hence causal curves can be partially lightlike and timelike.
	}

	\newdef{Causal cone}{\index{causal!cone}
		Let $M$ be a Lorentzian manifold. The causal cone of a point $p\in M$ is defined as the set of points $S\subset M$ such that every point $s\in S$ is connected to $p$ by a smooth curve that is everywhere causal.
	}
	\newdef{Causal closure}{\index{causal!closure}\label{relativity:causal_closure}
		Let $S$ be a subset of a Lorentzian manifold. The causal closure of $S$ is defined as the causal complement of the causal complement of $S$. A \textbf{causally closed set} is then defined as a set which is equal to its causal closure.
	}

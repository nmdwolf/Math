\chapter{Special relativity}
\section{Lorentz transformations}

	\begin{formula}
		\begin{equation}
			\beta = \frac{v}{c}
		\end{equation}
	\end{formula}
	\newformula{Lorentz factor}{\index{Lorentz!factor}
	    	\begin{equation}
			\label{rel:lorentz_factor}
        		\gamma = \stylefrac{1}{\sqrt{1 - \beta^2}}
		\end{equation}
	}
	\newformula{Lorentz transformations}{
	    	Let $\mathbf{V}$ be a 4-vector. A Lorentz boost along the $x^1$-axis is given by the following transformation:
	    	\begin{equation}
	        	\label{rel:lorentz_transformations}
	        	\boxed{\begin{array}{ccl}
				V'^0 &=& \gamma\left(V^0 - \beta V^1\right)\\
		                V'^1 &=& \gamma\left(V^1 - \beta V^0\right)\\
		                V'^2 &=& V^2\\
		                V'^3 &=& V^3
			\end{array}}
		\end{equation}
	}
	\begin{remark}
		Putting $c=+\infty$ in the previous transformation formulas gives the Galilean transformations from classical mechanics.
	\end{remark}
    
\section{Energy and momentum}\index{energy}\index{momentum}

	\newformula{4-velocity}{
	    	\begin{equation}
        		U^\mu = \left(\deriv{x^0}{\tau}, \deriv{x^1}{\tau}, \deriv{x^2}{\tau}, \deriv{x^3}{\tau}\right)
    		\end{equation}
        	or by applying the formulas for proper time and time dilatation we obtain:
        	\begin{equation}
    			\label{rel:4_velocity}
        		U^\mu = \left(\gamma c, \gamma \vector{u}\right)
	    	\end{equation}
	}
	\newformula{4-momentum}{
	    	\begin{equation}
			\label{rel:4_momentum}
		        p^\mu = m_0U^\mu
	    	\end{equation}
        	or by setting $E = cp^0$:
        	\begin{equation}
        		p^\mu = \left(\frac{E}{c}, \gamma m_0 \vector{u}\right)
        	\end{equation}
	}

	\newdef{Relativistic mass}{
	    	The factor $\gamma m_0$ in the momentum 4-factor is called the relativistic mass. By introducing this quantity (and denoting it by $m$), the classic formula $\vector{p} = m\vector{u}$ for the 3-momentum can be generalized to 4-momenta $p^\mu$.
	}

	\begin{formula}[Relativistic energy relation]\index{Einstein!energy relation}
		\begin{equation}
	        	\label{forces:relativistic_energy}
		        \boxed{E^2 = p^2c^2 + m^2c^4}
		\end{equation}
	        This formula is often called the \textbf{Einstein relation}.
	\end{formula}



\chapter{General Relativity}
\section{Einstein field equations}

	\newformula{Einstein field equations}{\index{Einstein!field equations}
	    	The Einstein field equations without the cosmological constant $\Lambda$ read:
	    	\begin{equation}
	    		\label{rel:einstein_field_equations}
		        \boxed{G_{\mu\nu} = \stylefrac{8\pi G}{c^4}T_{\mu\nu}}
	    	\end{equation}
	    	where $G_{\mu\nu}$ is the Einstein tensor \ref{diff:manifolds:einstein_tensor} and $T_{\mu\nu}$ is the stress-energy tensor \ref{relativity:stress_energy_tensor}.
	}
    
\section{Schwarzschild metric}

	\newformula{Schwarzschild metric}{\index{Schwarzschild!metric}\index{Schwarzschild!radius}
	    	\begin{equation}
	    		\label{rel:schwarzschild_metric}
		        ds^2 = \left(1-\stylefrac{R_s}{r}\right)c^2dt^2 - \left(1-\stylefrac{R_s}{r}\right)^{-1}dr^2 - r^2d\Omega^2
	    	\end{equation}
        	where $R_s$ is the Schwarzschild radius given by
        	\begin{equation}
        		R_s = \stylefrac{2GM}{c^2}
        	\end{equation}
	}
    
	\begin{theorem}[Birkhoff]
	    	The Schwarzschild metric is the unique solution of the vacuum field equation under the additional constraints of asymptotic flatness and staticity.
	\end{theorem}
	
	\newformula{Reissner-Nordstr\"om metric}{
		If we allow the black hole to have an electric charge $Q$, the Schwarzschild metric is modified:
		\begin{equation}
			ds^2 = \left(1-\stylefrac{2GM}{r} + \frac{GQ^2}{4\pi r^2}\right)c^2dt^2 - \left(1-\stylefrac{2GM}{r} + \frac{GQ^2}{4\pi r^2}\right)^{-1}dr^2 - r^2d\Omega^2
		\end{equation}
	}
	
	\begin{remark}
		A computation of the electric field generated by the black hole gives us:
		\begin{equation}
			E^r = \frac{Q}{4\pi r^2}
		\end{equation}
		Although the coordinate $r$ is not the proper distance, it still acts as a parameter for the surface of a sphere (as it does in a Euclidean or Schwarzschild metric). This explains why the above formule is the same as the one in classical electromagnetism.
	\end{remark}

\chapter{Special relativity}
\section{Lortenz transformations}

	\newformula{Lorentz factor}{
    	\begin{equation}
			\label{rel:lorentz_factor}
            \gamma = \stylefrac{1}{\sqrt{1 - \frac{v^2}{c^2}}}
		\end{equation}
    }
	\newformula{Lorentz transformations}{
    	Let $\mathbf{V}$ be a 4-vector. A Lorentz boost along the $x^1$-axis is given by the following transformation:
    	\begin{equation}
        	\label{rel:lorentz_transformations}
        	\boxed{\begin{array}{ccl}
				V'^0 &=& \gamma\left(V^0 - \stylefrac{v}{c}V^1\right)\\
                V'^1 &=& \gamma\left(V^1 - \stylefrac{v}{c}V^0\right)\\
                V'^2 &=& V^2\\
                V'^3 &=& V^3
			\end{array}
            }
		\end{equation}
    }
    \remark{Putting $c=+\infty$ in the previous transformation formulas gives the known Galilean transformations from classical mechanics.}
    
\section{Energy and momentum}\index{energy}\index{momentum}

	\newformula{4-velocity}{
    	\begin{equation}
        	U^\mu = \left(\deriv{x^0}{\tau}, \deriv{x^1}{\tau}, \deriv{x^2}{\tau}, \deriv{x^3}{\tau}\right)
    	\end{equation}
        or by applying the formulas for proper time and time dilatation we obtain:
        \begin{equation}
    		\label{rel:4_velocity}
            U^\mu = \left(\gamma c, \gamma \vector{u}\right)
    	\end{equation}
    }
    \newformula{4-momentum}{
    	\begin{equation}
			\label{rel:4_momentum}
            p^\mu = m_0U^\mu
    	\end{equation}
        or by setting $E = cp^0$:
        \begin{equation}
        	p^\mu = \left(\frac{E}{c}, \gamma m_0 \vector{u}\right)
        \end{equation}
    }
    \newdef{Relativistic mass}{
    	The factor $\gamma m_0$ in the momentum 4-factor is called the relativistic mass. By introducing this quantity (and denoting it by $m$), the classic formula $\vector{p} = m\vector{u}$ for the 3-momentum is preserved.
    }

	\begin{formula}[Relativistic energy relation]\index{Einstein!energy relation}
		\begin{equation}
        	\label{forces:relativistic_energy}
            \boxed{E^2 = m^2c^4 + p^2c^2}
		\end{equation}
        This is also sometimes called the Einstein relation for energy.
	\end{formula}



\chapter{General Relativity}
\section{Einstein field equations}
	\newformula{Einstein field equations}{\index{Einstein!field equations}
    	The Einstein field equations without the cosmological constant $\Lambda$ read:
    	\begin{equation}
    		\label{rel:einstein_field_equations}
            \boxed{G_{\mu\nu} = \stylefrac{8\pi G}{c^4}T_{\mu\nu}}
    	\end{equation}
    }
    
\section{Schwarzschild metric}
	\newformula{Schwarzschild metric}{\index{Schwarzschild!metric}\index{Schwarzschild!radius}
    	\begin{equation}
    		\label{rel:schwarzschild_metric}
            ds^2 = \left(1-\stylefrac{R_s}{r}\right)c^2dt^2 - \left(1-\stylefrac{R_s}{r}\right)^{-1}dr^2 - r^2d\Omega^2
    	\end{equation}
        where $R_s$ is the Schwarzschild radius given by $R_s = \stylefrac{2GM}{c^2}$.
    }
    
    \begin{theorem}[Birkhoff's theorem]
    	The Schwarzschild metric is the unique solution of the vacuum field equation with the additional constraints of asymptotic flatness and staticity.
    \end{theorem}
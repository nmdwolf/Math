\chapter{Special Relativity}

    In this chapter, as will be the case in the chapters on quantum field theory, the mostly-minuses convention for the Minkowski signature is adopted unless stated otherwise, i.e. the signature is $(+,-,-,-)$. Furthermore, natural units will be used unless stated otherwise, i.e. $\hbar = c = 1$. This follows the introductory literature such as \cite{Peskin, greiner_qft}.

\section{Lorentz transformations}

    \begin{notation}\index{Lorentz!factor}
        In the context of special relativity it is often useful to introduce the following quantities:
        \begin{align}
            \beta &:= \frac{v}{c}\\
            \label{relativity:lorentz_factor}
            \gamma &:= \frac{1}{\sqrt{1 - \beta^2}}.
        \end{align}
        The latter quantity is called the \textbf{Lorentz factor}.
    \end{notation}
    \newformula{Lorentz transformations}{\label{relativity:lorentz_transformations}
        Let $\mathbf{V}$ be a 4-vector. A Lorentz boost along the $x^1$-axis is given by the following transformation:
        \begin{gather}
            \begin{align}
                V'^0 &= \gamma\left(V^0 - \beta V^1\right)\\
                V'^1 &= \gamma\left(V^1 - \beta V^0\right)\\
                V'^2 &= V^2\\
                V'^3 &= V^3.
            \end{align}
        \end{gather}
    }
    \begin{remark}
        Putting $c=\infty$ in the previous formulas recovers the Galilei transformations from classical mechanics (cf. the In\"on\"u-Wigner contraction \ref{lie:inonu_wigner}).
    \end{remark}

\section{Energy and momentum}

    \newformula{4-velocity}{\index{velocity}\label{relativity:4_velocity}
        In analogy to the definition of velocity in classical mechanics, the 4-velocity is defined as follows:
        \begin{gather}
            \mathbf{U} := \left(\deriv{x^0}{\tau},\deriv{x^1}{\tau},\deriv{x^2}{\tau},\deriv{x^3}{\tau}\right).
        \end{gather}
        By applying the formulas for proper time and time dilatation we obtain:
        \begin{gather}
            \mathbf{U} = \left(\gamma c,\gamma\vector{u}\right).
        \end{gather}
    }
    \newformula{4-momentum}{\index{momentum}\label{relativity:4_momentum}
        The 4-momentum is defined as follows:
        \begin{gather}
            \mathbf{p} = m_0\mathbf{U},
        \end{gather}
        or, after defining $E := cp^0$:
        \begin{gather}
            \mathbf{p} = \left(\frac{E}{c},\gamma m_0\vector{u}\right).
        \end{gather}
    }

    \newdef{Relativistic mass}{\index{mass}
        The factor $m:=\gamma m_0$ in the momentum 4-vector is called the relativistic mass. By introducing this quantity, the classical formula $\vector{p} = m\vector{u}$ for the 3-momentum can be generalized to 4-momenta $\mathbf{p}$.
    }

    \begin{formula}[Relativistic energy relation]\index{energy}\index{Einstein!energy relation}\label{relativity:relativistic_energy}
        \begin{gather}
            E^2 = p^2c^2 + m^2c^4
        \end{gather}
        This formula is often called the \textbf{Einstein relation}.
    \end{formula}

\chapter{General Relativity}\label{chapter:GR}

    References for this chapter are \cite{gravitation, lqg}. See Chapter \ref{chapter:riemann} for an introduction to the theory of Riemannian geometry. The mathematical background for the section on the \textit{tetradic formulation} of GR can be found in Section \ref{section:cartan_geometry}.

    In this chapter the signature convention of the previous chapter is reversed. For general relativity it is often more convenient to use the mostly-pluses convention (this simply reduces the number of minus signs).

\section{Causal structure}

    \newdef{Null coordinate}{\index{null!vector}\index{lightlike}\index{timelike}\index{spacelike}
        Consider a vector $v\in T_pM$ on a Lorentzian manifold $(M,g)$. This vector is said to be null or \textbf{lightlike} if it satisfies the following condition:
        \begin{gather}
            g_p(v,v) = 0.
        \end{gather}
        One can also define \textbf{timelike} and \textbf{spacelike} vectors in a similar way as those vectors having negative and positive norm, respctively. \footnote{For a mostly-minuses signature one interchanges these definitions.} Spacelike, lightlike and timelike curves are defined as curves for which every tangent vector is respectively spacelike, lightlike or timelike.
    }

    \newdef{Time-orientability}{\index{orientation!time}
        A Lorentzian manifold is said to be time-orientable if there exists a nowhere-vanishing, timelike vector field. It should be noted that, in contrast to ordinary orientability, this notion is not purely topological. Moreover, neither orientability nor time-orientability implies the other. They are independent notions.
    }

    \newdef{Causal curve}{\index{curve!causal}
        Consider a smooth curve $\gamma:\ ]0, 1[\ \rightarrow(M,g)$. This curve is said to be causal if it satisfies
        \begin{gather}
            g\left(\dot{\gamma}(t),\dot{\gamma}(t)\right)\leq 0
        \end{gather}
        for all $t\in\ ]0,1[$. If the inequality is replaced by a strict inequality, the definition of a timelike curve is recovered. In general causal curves can be partially lightlike and timelike.
    }

    \newdef{Causal cone}{\index{causal!cone}
        Let $M$ be a Lorentzian manifold. The causal cone of a point $p\in M$ is defined as the set $S_p\subset M$ of points that are connected to $p$ by a (smooth) causal curve.
    }
    \newdef{Causal closure}{\index{causal!closure}\label{relativity:causal_closure}
        Let $S$ be a subset of a Lorentzian manifold. The causal closure of $S$ is defined as the causal complement of the causal complement of $S$. A \textbf{causally closed set} is then defined as a set which is equal to its causal closure.
    }

    \newdef{Stationary spacetime}{
        A spacetime $(M,g)$ is called stationary if there exists a timelike Killing vector. By the \textit{flowbox theorem} there always exists a coordinate chart such that locally one can choose the Killing vector field to be $\partial_0$ and hence we see that a spacetime is stationary if we can find a coordinate system for which the metric coefficients are time-independent.
    }

\section{Einstein field equations}

    \newformula{Einstein field equations}{\index{Einstein!field equations}\index{stress-energy tensor}
        The Einstein field equations without a cosmological constant $\Lambda$ read as follows (all fundamental constants are shown for completeness):
        \begin{gather}
            \label{relativity:einstein_field_equations}
            G_{\mu\nu} = \frac{8\pi G}{c^4}T_{\mu\nu},
        \end{gather}
        where $G_{\mu\nu}$ is the Einstein tensor \eqref{riemann:einstein_tensor} and $T_{\mu\nu}$ is the stress-energy tensor \eqref{field:stress_energy_tensor}.

        By taking the trace of both sides one obtains $T = -R$ and, hence, the Einstein field equations can be rewritten as
        \begin{gather}
            R_{\mu\nu} = \widehat{T}_{\mu\nu},
        \end{gather}
        where $\widehat{T}_{\mu\nu} = T_{\mu\nu} - \frac{1}{2}g_{\mu\nu}T$ is the \textbf{reduced stress-energy tensor}.
    }

    \begin{formula}[Einstein-Hilbert action]\index{action!Einstein-Hilbert}
        The (vacuum) field equations can be obtained by applying the variational principle to the following action:
        \begin{gather}
            S_\mathrm{EH}[g_{\mu\nu}] := \int_M\sqrt{-g}R.
        \end{gather}
        For manifolds with boundary one needs an extra term to make the boundary contributions vanish (as to obtain a well-defined variational problem). This term is due to \textit{Gibbons}, \textit{Hawking} and \textit{York}:\footnote{Einstein had in fact already introduced a variant, the $\Gamma\Gamma$-\textit{Lagrangian}.}
        \begin{gather}
            S_\mathrm{GHY}[g_{\mu\nu}] := \oint_{\partial M}\epsilon\sqrt{h}K,
        \end{gather}
        where $h_{ab}$ is the induced metric on the boundary, $K_{ab}$ is the extrinsic curvature and $\epsilon = \pm1$ is a ``function'' depending on whether the boundary is timelike or spacelike.
    \end{formula}

\section{Black holes}

    \newformula{Schwarzschild metric}{\index{Schwarzschild metric}
        \begin{gather}
            \label{relativity:schwarzschild_metric}
            ds^2 := \left(1-\frac{R_s}{r}\right)c^2dt^2 - \left(1-\frac{R_s}{r}\right)^{-1}dr^2 - r^2d\Omega^2,
        \end{gather}
        where $R_s$ is the Schwarzschild radius given by
        \begin{gather}
            R_s := \frac{2GM}{c^2}.
        \end{gather}
    }

    \begin{theorem}[Birkhoff]\index{Birkhoff}
        The Schwarzschild metric is the unique solution of the vacuum field equation under the additional constraints of asymptotic flatness and staticity.
    \end{theorem}

    \newformula{Reissner-Nordstr\"om metric}{
        If the black hole is allowed to have an electric charge $Q$, the Schwarzschild metric must be modified in the following way:
        \begin{gather}
            ds^2 := \left(1-\frac{2GM}{r} + \frac{GQ^2}{4\pi r^2}\right)c^2dt^2 - \left(1-\frac{2GM}{r} + \frac{GQ^2}{4\pi r^2}\right)^{-1}dr^2 - r^2d\Omega^2.
        \end{gather}
    }

    \begin{remark}
        The electric field generated by a Reissner-Nordstr\"om black hole is given by
        \begin{gather}
            E^r = \frac{Q}{4\pi r^2}.
        \end{gather}
        Although the coordinate $r$ is not the proper distance, it still acts as a parameter for the surface of a sphere (as it does in a Euclidean or Schwarzschild metric). This explains why the above formula is the same as the one in classical electromagnetism.
    \end{remark}

    ?? ADD KERR-NEWMAN, ERGOSPHERE, PENROSE MECHANISM, KRUSKAL ??

\section{Conserved quantities}

    Before raising any hope that conserved quantities are to be found everywhere in GR, the following result is given:
    \begin{property}
        By Noether's third theorem \ref{var:noether_third_theorem}, there exist no proper (local) conservation laws such as those of momentum and energy, because the translation group is a subgroup of the infinite-dimensional symmetry group $\mathrm{Diff}(M)$.
    \end{property}

\section{Tetradic formulation}

    We will start from the geometric interpretation of the (weak) equivalence principle, i.e. spacetime is locally modelled on Minkowski space. The natural language for this kind of geometry is that of Cartan geometries. By the Erlangen program we know that the Minkowski spacetime $M^4$ can be described as the coset space $\mathrm{ISO}(3,1)/\mathrm{SO}(3,1)$. The natural generalization is given by a Cartan geometry with model geometry $(\mathfrak{iso}(3,1),\mathfrak{so}(3,1))$.

    \begin{property}[Cartan connection]
        This way a $\mathrm{SO}(3,1)$-structure on the spacetime manifold $M$ is obtained, i.e. a choice of Lorentzian metric $g$. The Cartan connection $\widetilde{\nabla}$ can also be decomposed as $\nabla+\mathbf{e}$ where
        \begin{itemize}
            \item $\nabla$ defines a $\mathfrak{so}(3,1)$-valued principal connection, and
            \item $\mathbf{e}$ defines a $\mathcal{M}^4$-valued solder form.
        \end{itemize}
        The principal connection $\nabla$ is called the \textbf{spin connection} and $\mathbf{e}$ is called the \textbf{vierbein} or \textbf{tetrad}.\index{spin!connection}\index{vielbein}\index{tetrad} These objects are well-known in general relativity. The connection $\nabla$ is the ordinary Levi-Civita connection associated to the Lorentzian manifold $M$ (in case of vanishing torsion) and $\mathbf{e}$ gives the isometry between local (flat) Minkowski coordinates and ``global'' coordinates:
        \begin{gather}
            g := \mathbf{e}^*\eta
        \end{gather}
        or, locally,
        \begin{gather}
            g_{\mu\nu} = e^i_\mu e^j_\nu\eta_{ij}.
        \end{gather}
    \end{property}

    Using the tetrad field one can rewrite the Einstein-Hilbert action in a very elegant way. To this end, a new curvature form is defined:
    \begin{gather}
        F^i_{\ j\mu\nu} := e^i_\rho e_j^\sigma R^\rho_{\ \sigma\mu\nu},
    \end{gather}
    where $R^\rho_{\ \sigma\mu\nu}$ is the ordinary Riemann curvature tensor. The Einstein-Hilbert action is then equivalent\footnote{At least in the case of pure gravity \cite{lqg}.} to the following Yang-Mills-like action:
    \newformula{Palatini action}{\index{Palatini action}
        \begin{gather}
            S[e,\nabla] := \int_M\mathbf{e}\wedge\mathbf{e}\wedge\ast F.
        \end{gather}
        This action is sometimes called the \textbf{tetradic Palatini action} and the resulting formulation of general relativity is called the \textbf{first order formulation}. If one considers the same action but only as a functional of the tetrad field, one obtains the \textbf{second order formulation} of gravity.\footnote{These formulations are equivalent for pure gravity. However, when coupling the theory to fermions, they differ by a four-fermion vertex. This follows from the introduction of torsion due to the fermions.}

        Variation of the Palatini action gives the following EOM:
        \begin{itemize}
            \item $\delta\nabla$: $T(\mathbf{e})=0$ or, equivalently, $\nabla(\mathbf{e})\equiv\nabla$, i.e. the torsion vanishes and the connection $\nabla$ is on-shell equal to the Levi-Civita connection on $M$.
            \item $\delta\mathbf{e}$: The metric $g$ satisfies the Einstein field equations.
        \end{itemize}
    }

    Because of its importance in general relativity the first factor in the Palatini action deserves a name:
    \newdef{Plebanski form}{\index{Plebanski form}
        \begin{gather}
            \Sigma := \mathbf{e}\wedge\mathbf{e}
        \end{gather}
        Because of its internal antisymmetric Lorentz indices, one can interpret this object as a $\mathfrak{so}(3,1)$-valued two-form.
    }
    As was the case for 4D Yang-Mills theory, one can introduce a topological term that leaves the EOM invariant (up to boundary terms):
    \newdef{Holst action\footnotemark}{\index{action!Holst}\index{Barbero-Immirzi constant}
        \footnotetext{Holst was actually the second author to include this term.}
        \begin{align}
            S[\mathbf{e},\nabla] :&= \int_M\mathbf{e}\wedge\mathbf{e}\wedge\ast F + \frac{1}{\gamma}\int_M\mathbf{e}\wedge\mathbf{e}\wedge F\nonumber\\
            &= \int_M\left(\ast\textbf{e}\wedge\mathbf{e} + \frac{1}{\gamma}\mathbf{e}\wedge\mathbf{e}\right)\wedge F.
        \end{align}
        The coupling constant $\gamma$ is called the \textbf{Barbero-Immirzi} constant.
    }

    ?? COMPLETE ??
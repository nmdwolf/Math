\chapter{Special Relativity}

    In this and the following chapters we adopt the standard Minkowski signature $(+, -, -, -)$ unless stated otherwise. This follows the introductory literature such as \cite{Peskin, greiner_qft}. Furthermore, we also work in natural units unless stated otherwise, i.e. $\hbar = c = 1$.

\section{Lorentz transformations}

    \begin{notation}\index{Lorentz!factor}
        In the context of special relativity it is often useful to introduce the following quantities:
        \begin{align}
            \beta &:= \frac{v}{c}.\\
            \label{rel:lorentz_factor}
            \gamma &:= \stylefrac{1}{\sqrt{1 - \beta^2}}
        \end{align}
        The latter one is called the \textbf{Lorentz factor}.
    \end{notation}
    \newformula{Lorentz transformations}{
        Let $\mathbf{V}$ be a 4-vector. A Lorentz boost along the $x^1$-axis is given by the following transformation:
        \begin{gather}
            \label{rel:lorentz_transformations}
            \begin{align}
                V'^0 &= \gamma\left(V^0 - \beta V^1\right)\\
                V'^1 &= \gamma\left(V^1 - \beta V^0\right)\\
                V'^2 &= V^2\\
                V'^3 &= V^3.
            \end{align}
        \end{gather}
    }
    \begin{remark}
        Putting $c=+\infty$ in the previous formulas recovers the Galilei transformations from classical mechanics.
    \end{remark}

\section{Energy and momentum}\index{energy}\index{momentum}

    \newformula{4-velocity}{
        In analogy to the definition of velocity in classical mechanics we define the 4-velocity as follows:
        \begin{gather}
            \mathbf{U} := \left(\deriv{x^0}{\tau}, \deriv{x^1}{\tau}, \deriv{x^2}{\tau}, \deriv{x^3}{\tau}\right).
        \end{gather}
        By applying the formulas for proper time and time dilatation we obtain:
        \begin{gather}
            \label{rel:4_velocity}
            \mathbf{U} = \left(\gamma c, \gamma \vector{u}\right).
        \end{gather}
    }
    \newformula{4-momentum}{
        The 4-momentum is defined as follows:
        \begin{gather}
            \label{rel:4_momentum}
            \mathbf{p} = m_0\mathbf{U}
        \end{gather}
        or by defining $E = cp^0$:
        \begin{gather}
            \mathbf{p} = \left(\frac{E}{c}, \gamma m_0 \vector{u}\right).
        \end{gather}
    }

    \newdef{Relativistic mass}{
        The factor $\gamma m_0$ in the momentum 4-vector is called the relativistic mass. By introducing this quantity (and denoting it by $m$), the classic formula $\vector{p} = m\vector{u}$ for the 3-momentum can be generalized to 4-momenta $\mathbf{p}$.
    }

    \begin{formula}[Relativistic energy relation]\index{Einstein!energy relation}
        \begin{gather}
            \label{forces:relativistic_energy}
            E^2 = p^2c^2 + m^2c^4
        \end{gather}
        This formula is often the \textbf{Einstein relation}.
    \end{formula}

\chapter{General Relativity}

    See chapter \ref{chapter:riemann} for the theory of Riemannian manifolds. References for this chapter are \cite{gravitation, lqg}.

\section{Causal structure}

    \newdef{Null coordinate}{\index{null!vector}\index{lightlike}\index{timelike}\index{spacelike}
        Consider a vector $v\in T_pM$ on a pseudo-Riemannian manifold $(M,g)$. This vector at $p\in M$ is said to be null (or \textbf{lightlike}) if it satisfies the following condition:
        \begin{gather}
            g(v,v) = 0.
        \end{gather}
        One can also define \textbf{timelike} and \textbf{spacelike} vectors. Assume that the signature is mostly-pluses, then a timelike vector has negative norm while a spacelike vector has positive norm.\footnote{For a mostly-minuses signature one interchanges these definitions.}

        Spacelike, lightlike and timelike curves are defined as curves for which every tangent vector is spacelike, lightlike or timelike.
    }

    \newdef{Time-orientability}{\index{orientation!time}
        A Lorentzian manifold is said to be time-orientable if there exists a nowhere vanishing timelike vector field. It should be noted that, in contrast to ordinary orientability, this notion is not purely topological. Moreover, neither orientability nor time-orientability implies the other.
    }

    \newdef{Causal curve}{\index{curve!causal}
        Consider a curve $\gamma:\ ]0, 1[\rightarrow(M, g)$. This curve is said to be causal if it satisfies\footnote{Here we adopt the mostly minus convention.}
        \begin{gather}
            g\left(\dot{\gamma}(t), \dot{\gamma}(t)\right) \geq 0
        \end{gather}
        for all $t\in\ ]0, 1[$. If the inequality is replaced by a strict inequality then we recover the definition of a timelike curve. Hence causal curves can be partially lightlike and timelike.
    }

    \newdef{Causal cone}{\index{causal!cone}
        Let $M$ be a Lorentzian manifold. The causal cone of a point $p\in M$ is defined as the set of points $S\subset M$ such that every point $s\in S$ is connected to $p$ by a smooth curve that is everywhere causal.
    }
    \newdef{Causal closure}{\index{causal!closure}\label{relativity:causal_closure}
        Let $S$ be a subset of a Lorentzian manifold. The causal closure of $S$ is defined as the causal complement of the causal complement of $S$. A \textbf{causally closed set} is then defined as a set which is equal to its causal closure.
    }

    \newdef{Stationary spacetime}{
        A spacetime $(M, g)$ is called stationary if there exists a timelike Killing vector. By the \textit{flowbox theorem} there always exists a coordinate chart such that locally one can choose the Killing vector field to be $\partial_0$ and hence we see that a spacetime is stationary if we can find a coordinate system for which the metric coefficients are time-independent.
    }

\section{Einstein field equations}

    \newformula{Einstein field equations}{\index{Einstein!field equations}\index{stress-energy tensor!reduced}
        The Einstein field equations (without the cosmological constant $\Lambda$) read as follows:
        \begin{gather}
            \label{rel:einstein_field_equations}
            G_{\mu\nu} = \stylefrac{8\pi G}{c^4}T_{\mu\nu}
        \end{gather}
        where $G_{\mu\nu}$ is the Einstein tensor \ref{diff:manifolds:einstein_tensor} and $T_{\mu\nu}$ is the stress-energy tensor \ref{relativity:stress_energy_tensor}.

        By taking the trace of both sides one obtains $T = -R$ (in units $c = 1, G = \frac{1}{8\pi}$) and hence we can rewrite the Einstein field equations as
        \begin{gather}
            R_{\mu\nu} = \widehat{T}_{\mu\nu}
        \end{gather}
        where $\widehat{T}_{\mu\nu} = T_{\mu\nu} - \frac{1}{2}g_{\mu\nu}T$ is the \textbf{reduced stress-energy tensor}.
    }

    \begin{formula}[Einstein-Hilbert action]\index{action!Einstein-Hilbert}
        The (vacuum) field equations can be obtained by applying the variational principle to the following action:
        \begin{gather}
            S_{EH}[g_{\mu\nu}] := \int_M\sqrt{-g}R.
        \end{gather}
        When to manifolds with boundary one needs an extra term to make the boundary contributions vanish (and obtain a well-defined variational problem). This term is due to \textit{Gibbons}, \textit{Hawking} and \textit{York}:\footnote{Einstein had in fact already introduced a variant, the $\mathit{\Gamma\Gamma}$-\textit{Lagrangian}.}
        \begin{gather}
            S_{GH}[g_{\mu\nu}] := \oint_{\partial M}\epsilon\sqrt{h}K
        \end{gather}
        where $h_{ab}$ is the induced metric on the boundary, $K_{ab}$ is the extrinsic curvature and $\epsilon = \pm1$ is a constant depending on whether the boundary is timelike or spacelike.
    \end{formula}

\section{Schwarzschild metric}

    \newformula{Schwarzschild metric}{\index{Schwarzschild}
        \begin{gather}
            \label{rel:schwarzschild_metric}
            ds^2 := \left(1-\stylefrac{R_s}{r}\right)c^2dt^2 - \left(1-\stylefrac{R_s}{r}\right)^{-1}dr^2 - r^2d\Omega^2
        \end{gather}
        where $R_s$ is the Schwarzschild radius given by
        \begin{gather}
            R_s := \stylefrac{2GM}{c^2}.
        \end{gather}
    }

    \begin{theorem}[Birkhoff]\index{Birkhoff}
        The Schwarzschild metric is the unique solution of the vacuum field equation under the additional constraints of asymptotic flatness and staticity.
    \end{theorem}

    \newformula{Reissner-Nordstr\"om metric}{
        If we allow the black hole to have an electric charge $Q$, the Schwarzschild metric must be modified in the following way:
        \begin{gather}
            ds^2 := \left(1-\stylefrac{2GM}{r} + \frac{GQ^2}{4\pi r^2}\right)c^2dt^2 - \left(1-\stylefrac{2GM}{r} + \frac{GQ^2}{4\pi r^2}\right)^{-1}dr^2 - r^2d\Omega^2.
        \end{gather}
    }

    \begin{remark}
        A computation of the electric field generated by the black hole gives us
        \begin{gather}
            E^r = \frac{Q}{4\pi r^2}.
        \end{gather}
        Although the coordinate $r$ is not the proper distance, it still acts as a parameter for the surface of a sphere (as it does in a Euclidean or Schwarzschild metric). This explains why the above formula is the same as the one in classical electromagnetism.
    \end{remark}

\section{Tetradic formulation}

    The mathematical background for this section can be found in section \ref{section:cartan_geometry}.

    We will start from the geometric interpretation of the (weak) equivalence principle, i.e. spacetime is locally modelled on Minkowski space. The natural language for this kind of geometry is that of Cartan geometries. By the Erlangen program we know that the Minkowski space $\mathcal{M}$ can be described as the coset space $\text{ISO}(3,1)/\text{SO}(3,1)$. The natural generalization is given by a Cartan geometry with model geometry $(\mathfrak{iso}(3,1), \mathfrak{so}(3,1))$.

    This way we obtain a $SO(3,1)$-structure on the spacetime manifold $M$, i.e. a choice of Lorentzian metric $g$. The Cartan connection $\widetilde{\nabla}$ can also be decomposed as $\nabla+\mathbf{e}$ where
    \begin{itemize}
        \item $\nabla$ defines a $\mathfrak{so}(3,1)$-valued principal connection.
        \item $\mathbf{e}$ defines a $\mathcal{M}$-valued solder form.
    \end{itemize}
    The principal connection $\nabla$ is called the \textbf{spin connection} and $\mathbf{e}$ is called the \textbf{vierbein} or \textbf{tetrad}.\index{spin!connection}\index{vielbein}\index{tetrad} These objects are well-known in general relativity. The connection $\nabla$ is the ordinary Levi-Civita connection associated to the (pseudo-)Riemannian manifold $M$ (in case of vanishing torsion) and $\mathbf{e}$ gives the isometry between local (flat) Minkowski coordinates and ''global'' coordinates:
    \begin{gather}
        g := \mathbf{e}^*\eta
    \end{gather}
    or locally
    \begin{gather}
        g_{\mu\nu} = e^i_\mu e^j_\nu\eta_{ij}.
    \end{gather}

    Using the tetrad field we can rewrite the Einstein-Hilbert in a very elegant way. First we define a new curvature object:
    \begin{gather}
        F^i_{\ j\mu\nu} := e^i_\rho e_j^\sigma R^\rho_{\ \sigma\mu\nu}
    \end{gather}
    where $R^\rho_{\ \sigma\mu\nu}$ is the ordinary Riemann curvature tensor. The Einstein-Hilbert action is then equivalent\footnote{At least in the case of pure gravity (see \cite{lqg}).} to the following Yang-Mills-like action:
    \newformula{Palatini action}{\index{action!Palatini}
        \begin{gather}
            S[e, \nabla] := \int_M\mathbf{e}\wedge\mathbf{e}\wedge\ast F.
        \end{gather}
        This action is sometimes called the \textbf{tetradic Palatini action} and the resulting formulation of general relativity is called the \textbf{first order formulation}. If one considers the same action but only as a functional of the tetrad field, one obtains the \textbf{second order formulation} of gravity.\footnote{These formulations are equivalent for pure gravity, however, when coupling the theory to fermions they differ by a four-fermion vertex.}

        Variation of the Palatini action gives the following EOM:
        \begin{itemize}
            \item $\delta\nabla$: $T(\mathbf{e})=0$ or equivalently $\nabla(\mathbf{e})\equiv\nabla$, i.e. the torsion vanishes and hence the connection $\nabla$ is on-shell equal to the Levi-Civita connection on $M$.
            \item $\delta\mathbf{e}$: The metric $g$ satisfies the Einstein field equations.
        \end{itemize}
    }

    Because of its importance in general relativity we give the first factor in the Palatini action a name:
    \newdef{Plebanski form}{\index{Plebanski form}
        \begin{gather}
            \Sigma := \mathbf{e}\wedge\mathbf{e}
        \end{gather}
        Because of its internal antisymmetric Lorentz indices one can interpret this object as a $\mathfrak{so}(3,1)$-valued two-form.
    }
    As was the case for 4D Yang-Mills theory one can introduce a topological term that leaves the EOM invariant (up to boundary terms):
    \newdef{Holst action\footnotemark}{\index{action!Holst}\index{Barbero-Immirzi constant}
        \footnotetext{Holst was actually the second author to include this term.}
        \begin{align}
            S[\mathbf{e}, \nabla] :&= \int_M\mathbf{e}\wedge\mathbf{e}\wedge\ast F + \frac{1}{\gamma}\int_M\mathbf{e}\wedge\mathbf{e}\wedge F\nonumber\\
            &= \int_M\left(\ast\textbf{e}\wedge\mathbf{e} + \frac{1}{\gamma}\mathbf{e}\wedge\mathbf{e}\right)\wedge F.
        \end{align}
        The coupling constant $\gamma$ is called the \textbf{Barbero-Immirzi} constant.
    }

    ?? COMPLETE ??
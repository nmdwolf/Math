\chapter{Material physics}

\section{Crystals}
	
	\begin{theorem}[Steno's law]\index{Steno's law}

		The angles between crystal faces of the same type are constant and do not depend on the total shape of the crystal.
	\end{theorem}
    
	\newdef{Zone}{\index{zone}
    		The collection of faces parallel to a given axis, is called a zone. The axis itself is called the zone axis.
	}
    
\subsection{Analytic representation}
	
	\newdef{Miller indices}{\index{Miller indices}
    		Let $a, b, c$ be the lengths of the (not necessarily orthogonal) basis vectors of the crystal lattice. The lattice plane intersecting the axes at $\left(\frac{a}{h},\frac{b}{k},\frac{c}{k}\right)$ is denoted by the Miller indices $(h\ k\ l)$.
    	}
    	\begin{notation}
		Negative numbers are written as $\overline{a}$ instead of $-a$.
	\end{notation}
    
	\newformula{Coordinates of axes}{
    		Let $a, b, c$ denote the lengths of the basis vectors. The axis formed by the intersection of the planes $(h_1\ k_1\ l_1)$ and $(h_2\ k_2\ l_2)$,  pointing in the direction of the point $(au, bv, cw)$ is denoted by $[u\ v\ w]$. Where
        	\begin{equation}
			u = \left|
			\begin{array}{cc}
				k_1&l_1\\
		                k_2&l_2
			\end{array}\right|
            		\qquad
			v = \left|
			\begin{array}{cc}
				l_1&h_1\\
		                l_2&h_2
			\end{array}\right|
			\qquad
			w = \left|
			\begin{array}{cc}
				h_1&k_1\\
		                h_2&k_2
			\end{array}\right|
		\end{equation}
	}
    
	\begin{theorem}[Hauy's law of rational indices]\index{Hauy's law of rational indices}
		The Miller indices of every natural face of a crystal will always have rational proportions.
	\end{theorem}

\section{Symmetries}
	
	\newdef{Equivalent planes/axes}{
    		When applying certain symmetries to a plane or axis, it often occurs that we obtain a set of equivalent planes/axes. These equivalence classes are denoted respectively by $\{h\ k\ l\}$ and $<h\ k\ l>$.
	}
    
	\newprop{Rotational symmetry}{
    		Only $1, 2, 3, 4$ and $6$-fold rotational symmetries can occur. 
	}
    
\section{Crystal lattice}

	\begin{formula}
		For an orthogonal crystal lattice, the distance between planes of the family $(h\ k\ l)$ is given by:
        	\begin{equation}
			\label{maphy:d_hkl}
            		d_{hkl} = \stylefrac{1}{\sqrt{\left(\frac{h}{a}\right)^2 + \left(\frac{k}{b}\right)^2 + \left(\frac{l}{c}\right)^2}}
		\end{equation}
	\end{formula}
    
\subsection{Bravais lattice}
	
	\newdef{Bravais lattice}{\index{Bravais lattice}
    		A crystal lattice generated by a certain point group symmetry is called a Bravais lattice. There are 14 different Bravais lattices in 3 dimensions. These are the only possible ways to place (infinitely) many points in 3D space by applying symmetry operations to a point group.
	}

	\newdef{Wigner-Seitz cell}{\index{Wigner-Seitz cell}
    		The part of space consisting of all points closer to a given lattice point than to any other.
	}
    
\subsection{Reciprocal lattice}
	
	\newformula{Reciprocal basis vectors}{\index{reciprocal lattice}
    		The reciprocal lattice corresponding to a given Bravais lattice with primitive basis $\{\vector{a},\vector{b},\vector{c}\}$ is defined by the following reciprocal basis vectors
        	\begin{equation}
			\vector{a}^* = 2\pi\stylefrac{\vector{b}\times\vector{c}}{\vector{a}\cdot(\vector{b}\times\vector{c})}
		\end{equation}
		The vectors $\vector{b}^*$ and $\vector{c}^*$ are obtained by permutation of $(a,b,c)$. These vectors satisfy the relations
		\begin{align}
			\vector{a}\cdot\vector{a}^* &= 2\pi\nonumber\\
			\vector{b}\cdot\vector{b}^* &= 2\pi\\
			\vector{c}\cdot\vector{c}^* &= 2\pi\nonumber\\
		\end{align}
	}
	\newnot{Reciprocal lattice vector}{
	    	The reciprocal lattice vector $\vector{r}^{\ *}_{hkl}$ is defined as follows:
	        \begin{equation}
			\vector{r}^{\ *}_{hkl} = h\vector{a}^* + k\vector{b}^* + l\vector{c}^*
		\end{equation}
	}
    
	\begin{property}
		The reciprocal lattice vector $\vector{r}^*_{hkl}$ has the following properties:
        	\begin{itemize}
			\item $\vector{r}^{\ *}_{hkl}$ is perpendicular to the family of planes $(h\ k\ l)$ of the direct lattice.
			\item $||\vector{r}^{\ *}_{hkl}|| = \stylefrac{2\pi n}{d_{hkl}}$
		\end{itemize}
	\end{property}


\section{Diffraction}\index{diffraction}
\subsection{Constructive interference}

	\newformula{Laue conditions}{\index{Laue!conditions}
	    	Suppose that an incident beam makes angles $\alpha_0,\beta_0$ and $\gamma_0$ with the lattice axes. The diffracted beam making angles $\alpha,\beta$ and $\gamma$ with the axes will be observed if following conditions are satisfied:
    		\begin{align*}
			a(\cos\alpha - \cos\alpha_0) &= h\lambda\\
	                b(\cos\beta - \cos\beta_0) &= k\lambda\\
        	        c(\cos\gamma - \cos\gamma_0) &= l\lambda
	        \end{align*}
	        If these conditions have been met then we observe a diffracted beam of order $hkl$.
	}
	\begin{remark}
	    	Further conditions can be imposed on the angles, such as the pythagorean formula for orthogonal axes. This has the consequence that the only two possible ways to obtain a diffraction pattern are:
    		\begin{itemize}
			\item a fixed crystal and a polychromatic beam
			\item a rotating crystal and a monochromatic beam
		\end{itemize}
	\end{remark}
	
	\newformula{Vectorial Laue conditions}{
    		Let $\vec{k}_0, \vec{k}$ denote the wave vector of respectively the incident and diffracted beams. The Laue conditions can be reformulated in the following way:
    		\begin{equation}
        		\label{maphy:vectorial_laue_condition}
			\boxed{\vec{k} - \vec{k}_0 = \vec{r}^{\ *}_{hkl}}
		\end{equation}
	}

	\newformula{Bragg's law}{\index{Bragg's law}
    		Another equivalent formulation of the Laue conditions is given by following formula:
	    	\begin{equation}
			\label{maphy:braggs_law}
        		\boxed{2d_{hkl}\sin\theta = n\lambda}
		\end{equation}
	        where
        	\begin{flalign*}
			\qquad&\lambda:\text{wavelength of the incoming beam}&\\
        		&\theta:\text{the \textbf{Bragg angle}}&\\
		        &d_{hkl}:\text{distance between neighbouring planes}&
		\end{flalign*}
	}
    	\begin{remark}
    		The angle between the incident and diffracted beams is given by $2\theta$.
    	\end{remark}

	\begin{construct}[Ewald sphere]\index{Ewald sphere}
    		A simple construction to determine if Bragg difraction will occur is the Ewald sphere: Put the origin of the reciprocal lattice at the tip of the incident wave vector $\vector{k}_i$. Now construct a sphere with radius $\frac{2\pi}{\lambda}$ centered on the start of $\vector{k}_i$. All points on the sphere that coincide with a reciprocal lattice point satisfy the vectorial Laue condition \ref{maphy:vectorial_laue_condition}. Therefore Bragg diffraction will occur in the direction of all the intersections of the Ewald sphere and the reciprocal lattice.
	\end{construct}

\subsection{Intensity of diffracted beams}
	
	\newdef{Systematic extinctions}{\index{extinction}
    		Every particle in the motive emits its own waves. These waves will interfere and some will cancel out which leads to the absence of certain diffraction spots. These absences are called systematic extinctions.
    }
    
	\newdef{Atomic scattering factor}{\index{scattering!atomic scattering factor}
    		The waves produced by the individual electrons of an atom, which can have a different phase, can be combined into a resulting wave. The amplitude of this wave is called the atomic scattering factor.
	}
	\newdef{Structure factor}{\index{structure factor}
    		The waves coming from the individual atoms in the motive can also be combined, again taking into account the different phases, into a resulting wave. The amplitude of this wave is called the structure factor and it is given by:
        	\begin{equation}
        		\label{maphy:structure_factor}
        		F(hkl) = \sum_j\ f_j\exp\big[2\pi i(hx_j + ky_j + lz_j)\big]
		\end{equation}
	        where $f_j$ is the atomic scattering factor of the $j^{th}$ atom in the motive.
	}

	\begin{example}
		A useful example  of systematic extinctions is the structure factor of an FCC or BCC lattice for the following specific situations:
		
		\hspace{25pt} If $h+k+l$ is odd, then $F(hkl) = 0$ for a BCC lattice. If $h,k$ and $l$ are not all even or all odd then $F(hkl) = 0$ for an FCC lattice.
	\end{example}

	\newdef{Laue indices}{\index{Laue!indices}
    	Higher order diffractions can be rewritten as a first order diffraction in the following way:
        \begin{equation}
			2d_{nhnknl}\sin\theta = \lambda\qquad\text{with}\qquad d_{nhnknl} = \stylefrac{d_{hkl}}{n}
		\end{equation}
        Following from the interpretation of the Bragg law as diffraction being a reflection at the lattice plane $(h\ k\ l)$ we can introduce the (fictitious) plane with indices $(nh\ nk\ nl)$. These indices are called Laue indices.
    }
    \sremark{In contrast to Miller indices which cannot possess common factors, the Laue indices obviously can.}
        
\section{Alloys}
	
	\begin{theorem}[Hume-Rothery conditions]\index{Hume-Rothery conditions}\normalfont
	    	An element can be dissolved in a metal (forming a solid solution) if the following conditions are met:
	    	\begin{itemize}
	    		\item The difference between the atomic radii is $\leq 15\%$.
		        \item The crystal structures are the same.
	        	\item The elements have a similar electronegativity.
		        \item The valency is the same.
		\end{itemize}
	\end{theorem}
        
\section{Lattice defects}
	
	\newdef{Vacancy}{\index{vacancy}\index{Schottky defect|seealso{vacancy}}
		A lattice point where an atom is missing. Also called a \textbf{Schottky defect}.
	}
	\begin{formula}[Concentration of Schottky defects$^\dag$]
		Let $N$ denote the number of lattice points and $n$ the number of vacancies. The following relation gives the temperature dependence of Schottky defects:
		\begin{equation}
			\stylefrac{n}{n + N} = e^{-E_v/kT}
		\end{equation}
		where $T$ is the temperature and $E_v$ the energy needed to create a vacancy.
	\end{formula}
	\sremark{A similar relation holds for interstitials.}
	
	\newdef{Interstitial}{\index{interstitial}
		An atom placed at a position which is not a lattice point.
	}
	\newdef{Frenkel pair}{\index{Frenkel pair}
		An atom displaced from a lattice point to an interstitial location (hereby creating a vacancy-interstitial pair) is called a Frenkel defect.
	}
	\begin{formula}[Concentration of Frenkel pairs]
		Let $n_i$ denote the number of atoms displaced from the bulk of the lattice to any $N_i$ possible interstitial positions and thus creating $n_i$ vacancies. The following relation holds:
		\begin{equation}
			\stylefrac{n_i}{\sqrt{NN_i}} = e^{-E_{fr}/2kT}
		\end{equation}
		where $E_{fr}$ denotes the energy needed to create a Frenkel pair.
	\end{formula}
	
	\remark{In compounds the number of vacancies can be much higher than in mono-atomic lattices.}
	\remark{The existence of these defects creates the possibility of diffusion.}

\section{Electrical properties}
\subsection{Charge carriers}
	
	\newformula{Conductivity}{\index{conductivity}
    		Definition \ref{electricity:conductivity} can be modified to account for both positive and negative charge carriers: 
    		\begin{equation}
			\label{maphy:conductivity}
        		\sigma = n_nq_n\mu_n + n_pq_p\mu_p
		\end{equation}
    	}
	\sremark{The difference between the concentration of positive and negative charge carriers can differ by orders of magnitude across different materials. It can differ by up to 20 orders of magnitude.}

\subsection{Band structure}
	
	\newdef{Valence band}{\index{valence band}
    		The energy band corresponding to the outermost (partially) filled atomic orbital.
	}
	\newdef{Conduction band}{\index{conduction band}
	    	The first unfilled energy band.
	}
	\newdef{Band gap}{\index{band gap}
	    	The energy difference between the valence and conduction bands (if they do not overlap). It is the energy zone\footnotemark\ where no electron states can exist.
	    	\footnotetext{For a basic derivation see \cite{bransden}.}
	}
    
	\newdef{Fermi level}{\index{Fermi!level}
    		The energy level having a 50\% chance of being occupied at thermodynamic equilibrium.
	}
	\newformula{Fermi function}{\index{Fermi!function}
    		The following distribution gives the probability of a state with energy $E_i$ being occupied by an electron:
        	\begin{equation}
			\label{maphy:fermi_function}
        		\boxed{f(E_i) = \stylefrac{1}{e^{(E_i - E_f)/kT} + 1}}
		\end{equation}
        	where $E_f$ is the Fermi level as defined above.
	}
    
\subsection{Intrinsic semiconductors}

	\begin{formula}
		Let $n$ denote the charge carrier density as before. We find the following temperature dependence:
		\begin{equation}
			n \propto e^{-E_g/2kt}
		\end{equation}
		where $E_g$ is the band gap. This formula can be directly derived from the Fermi function by noting that for intrinsic semiconductors the Fermi level sits in the middle of the band gap, i.e. $E_c - E_f = E_g / 2$, and that for most semiconductors $E_g\gg kT$.
	\end{formula}
	
\subsection{Extrinsic semiconductors}

	\newdef{Doping}{
	    	Intentionally introducing impurities to modify the (electrical) properties.
	}
	\newdef{Acceptor}{
		Group III element added to create an excess of holes in the valence band. The resulting semiconductor is said to be a \textbf{p-type semiconductor}.
	}
	\newdef{Donor}{
    		Group IV element added to create an excess of electrons in the valence band. The resulting semiconductor is said to be an \textbf{n-type semiconductor}.
	}
    
\subsection{Ferroelectricity}
	
	Some materials can exhibit certain phase transitions between a paraelectric and ferroelectric state.
	
	Paraelectric materials have the property that the polarisation $\vec{P}$ and the electric field $\vec{E}$ are proportional. Ferroelectric materials have the property that they exhibit permanent polarization, even in the absence of an electric field. This permanent behaviour is the result of a symmetry breaking, i.e. the ions in the lattice have been shifted out of their 'central' positions and induce a permanent dipole moment.
	
    The temperature at which this phase transition occurs is called the \textbf{ferroelectric Curie temperature}. Above this temperature the material will behave as a paraelectric material.
    
   	\begin{remark}
   		Ferroelectricity can only occur in crystals with unit cells that do not have a center of symmetry. This would rule out the possiblity of having the asymmetry needed for the dipole moment.
   	\end{remark}
    

	\newdef{Saturation polarization}{\index{polarization}
	    	The maximum polarization obtained by a ferroelectric material. It it obtained when the domain formation also reaches a maximum. 
	}
	\newdef{Remanent polarization}{
    		The residual polarization of the material when the external electric field is turned off.
	}
	\newdef{Coercive field}{
    		The electric field needed to cancel out the remanent polarization.
	}
    
	\newdef{Piezoelectricity}{\index{piezoelectricity}
    		Materials that obtain a polarization when exposed to mechanical stress are called piezoelectric materials.
	}
	\begin{remark}
		All ferroelectric materials are piezoelectric, but the converse is not true. All crystals without a center of symmetry are piezoelectric. This property is however only a necessary (and not a sufficient) condition for ferroelectricity, as mentioned above.
	\end{remark}

	\begin{example}[Transducer]\index{transducer}
		A device that converts electrical to mechanical energy (and vice versa).
	\end{example}
    
\section{Magnetic properties}
	
	\begin{definition}[Diamagnetism]
    		In diamagnetic materials, the magnetization is oriented opposite to the applied field, so $B < H$. The susceptibility is small, negative and independent of the temperature.
	\end{definition}
	\begin{remark}
    		All materials exhibit a diamagnetic character.
	\end{remark}
        
	\begin{definition}[Paramagnetism]
    		The susceptibility is small, positive and inversely proportional to the temperature.
	\end{definition}
        
	\begin{definition}[Ferromagnetism]
    		Spontaneous magnetization can occur. The susceptibility is large and dependent on the applied field and temperature. Above a certain temperature, the \textbf{ferromagnetic Curie temperature}, the materials will behave as if they were only paramagnetic.
	\end{definition}
        
\subsection{Paramagnetism}
	
	\newformula{Curie's law}{\index{Curie's law}
    		If the interactions between the particles can be neglected, we obtain the following law:
    		\begin{equation}
			\label{magentism:curies_law}
        		\chi = \stylefrac{C}{T}
		\end{equation}
	        Materials that satisfy this law are called \textbf{ideal paramagnetics}.
	}
	\newformula{Curie-Weiss law}{\index{Weiss|seealso{Curie-Weiss}}\index{Curie!Curie-Weiss law}
    		If the interactions between particles cannot be neglected, we obtain the following law:
    		\begin{equation}
			\label{magentism:curie_weiss__law}
        		\chi = \stylefrac{C}{T-\theta}
		\end{equation}
	        where $\theta = CN_W$ with $N_W$ the \textbf{Weiss-constant}. This deviation of the Curie law is due to the intermolecular interactions that induce an internal magnetic field $H_m = N_WM$.
	}
	
	\begin{formula}[Brillouin function $B_J$]\index{Brillouin function}
		\begin{equation}
        		\label{magnetism:brillouin_function}
        		\boxed{B_J(y) = \stylefrac{2J + 1}{2J}\text{coth}\left(\stylefrac{2J + 1}{2J}y\right) - \stylefrac{1}{2J}\text{coth}\left(\stylefrac{y}{2J}\right)}
	        \end{equation}
        	where $y = \stylefrac{g\mu_BJB}{kT}$
	\end{formula}

	\begin{remark}
	    	Because coth$(y\rightarrow\infty)\approx1$ we have:
	        \begin{equation}
			\label{magnetism:absolute_saturation_magnetization}
		        \text{if }T\rightarrow0\quad\text{then}\quad M=Ng\mu_BJB_J(y\rightarrow\infty) = Ng\mu_BJ
		\end{equation}
	        This value is called the \textbf{absolute saturation magnetization}.
	\end{remark}
    
\subsection{Ferromagnetism}
	
	Ferromagnetics are materials that have strong internal interactions which lead to large scale (with respect to the lattice constant) parallel ordering of the atomic magnetic (dipole) moments. This leads to the spontaneous magnetization of the material and consequently a nonzero total dipole moment.
    
    \sremark{In reality, ferromagnetic materials do not always spontaneously possess a magnetic moment in the absence of an external field. When stimulated by a small external field, they will however display a magnetic moment, much larger than paramagnetic materials would.}
    
    \newdef{Domain}{\index{Weiss!domain}
    	The previous remark is explained by the existence of Weiss domains. These are spontaneously magnetized regions in a magnetic material. The total dipole moment is the sum of the moments of the individual domains. If not all the domains have a parallel orientation then the total dipole moment can be 0, a small external field is however sufficient to change the domain orientation and produce a large total magnetization.
    }
    \newdef{Bloch walls}{\index{Bloch!walls}
    	A wall between two magnetic domains.
    }
    
    \newdef{Ferromagnetic Curie temperature}{\index{Curie!ferromagnetic Curie temperature}
    	Above this this temperature the material loses its ferromagnetic properties and it becomes a paramagnetic material following the Curie-Weiss law.
    }
    
    \remark{For ferromagnetic (and ferrimagnetic) materials it is impossible to define a magnetic susceptibility as the magnetization is nonzero even in the absence of a magnetic field.\footnotemark\ Above the critical temperature (Curie/N\'eel) it is however possible to define a susceptiblity as the materials become paramagnetic in this region.
    }
    \footnotetext{This can be seen from equation \ref{magnetism:M}: $M = \chi H$. The susceptibility should be infinite.}
    
\subsection{Antiferromagnetism}
	When the domains in a magnetic material have an antiparallel ordering\footnotemark, the total dipole moment will be small. If the temperature rises, the thermal agitation however will disturb the orientation of the domains and the magnetic susceptibility will rise.
    \footnotetext{This will occur if it is energetically more favourable.}
    
    \newdef{N\'eel temperature}{\index{N\'eel!temperature}
    	At the N\'eel temperature, the susceptibility will reach a maximum. Above this temperature $(T>T_N)$ the material will become paramagnetic, satisfying the following formula:
        \begin{equation}
			\chi = \stylefrac{C}{T+\theta}
		\end{equation}
        This resembles a generalization of the Curie-Weiss law with a negative and therefore virtual critical temperature.
    }
    
\subsection{Ferrimagnetism}
	Materials that are not completely ferromagnetic nor antiferromagnetic, due to an unbalance between the sublattices, will have a nonzero dipole moment even in the absence of an external field. The magnitude of this moment will however be smaller than that of a ferromagnetic material. These materials are called ferrimagnetic materials.
    
    \newformula{N\'eel hyperbola}{\index{N\'eel!hyperbola}
    	Above the N\'eel temperature it is possible to define a susceptibility given by:
        \begin{equation}
			\stylefrac{1}{\chi} = \stylefrac{T}{C} - \stylefrac{1}{\chi_0} - \stylefrac{\sigma}{T - \theta'}
		\end{equation}
    }

\section{Mathematical description}
	\begin{theorem}[Neumann's principle]\index{Neumann!principle}
		The symmetry elements of the physical properties of a crystal should at least contain those of the point group of the crystal.
	\end{theorem}

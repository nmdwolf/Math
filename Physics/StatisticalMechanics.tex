\chapter{Statistical mechanics}

\section{Axioms}
    
    \begin{theorem}[Ergodic principle]\index{ergodic principle}
    	All microstates corresponding to the same macroscopic state are equally propable.
    \end{theorem}
	\begin{theorem}[Boltzmann formula]\index{Boltzmann!entropy}
    	The central axiom of statistical mechanics gives following formula for the entropy:
		\begin{equation}
			\label{statmech:boltzmann_formula}
            \boxed{S = k\ln\Omega(E, V, N, \alpha)}
		\end{equation}
        where $\Omega$ denotes the number of microstates corresponding to the system at energy $E$, volume $V$, and so on.
	\end{theorem}
    
	\section{Temperature}\index{temperature}
    	The temperature of a system in contact with a heat bath is defined as:
    	\begin{equation}
        	\label{statmech:temperature}
			\boxed{T = \left(\pderiv{E}{S}\right)_V}
		\end{equation}
    
\section{Maxwell-Boltzmann statistics}\index{Maxwell!Maxwell-Boltzmann statistics}\index{Boltzmann!Maxwell-Boltzmann statistics}

    	Consider a system of $N$ indistinguishable non-interacting particles. Let $\varepsilon_i$ be the energy associated with the $i$-th energy level with degeneracy $g_i$. The propability $p_i$ of finding a particle in the $i$-th energy level is given by:
    	\begin{equation}
			\boxed{p_i = \stylefrac{g_i e^{-\beta\varepsilon_i}}{Z}}
		\end{equation}
        where $Z$ is the single particle \textbf{partition function} defined as:
        \begin{equation}\index{partition function}
			\label{statmech:partition_function}
            \boxed{Z = \sum_i{g_ie^{-\beta\varepsilon_i}}}
		\end{equation}
        
\section{Grand canonical system}
    
        \newformula{Grand canonical partition function}{\index{partition function}
		\begin{equation}
			\mathcal{Z}_i = \sum_{\varepsilon_i}e^{\beta n_i(\mu - \varepsilon_i)}
		\end{equation}
        }
        \result{
        	In the case that $n_i \in \{0, 1\}$ this formula reduces to $\mathcal{Z}_i = e^{\beta\mu}Z_i$.
        }
        
        
    	\newdef{Fugacity}{\index{fugacity}
        	\begin{equation}
        		\label{statmech:fugacity}
                	z = e^{\mu N}
        	\end{equation}
        }
        
\section{Energy}

	\begin{theorem}[Virial theorem]\index{virial theorem}
		\begin{equation}
			\label{statmech:virial_theorem}
			\boxed{\langle T \rangle = -\frac{1}{2}\sum_k\langle \vector{r}_k\cdot\vector{F}_k \rangle}
		\end{equation}
	\end{theorem}
	\begin{result}
		For potentials of the form $V = ar^{-n}$ this becomes:
		\begin{equation}
			2\langle T \rangle = -n\langle V \rangle
		\end{equation}
	\end{result}
	
	\begin{theorem}[Equipartition theorem]\index{equipartition theorem}
		Let $x$ be any generalized coordinate (both position or momentum).
		\begin{equation}
			\boxed{\left\langle x^k\pderiv{H}{x^l} \right\rangle = \delta_{kl}k_bT}
		\end{equation}
	\end{theorem}
	\begin{result}
		For quadratic Hamiltonians this can be rewritten using Euler's theorem for homogeneous functions \ref{calculus:theorem:euler_homogeneous_functions} as:
		\begin{equation}
			\langle T \rangle = \frac{1}{2}k_bT
		\end{equation}
	\end{result}

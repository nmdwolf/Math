\chapter{Statistical Mechanics}

\section{Axioms}
    
	\begin{theorem}[Ergodic principle]\index{ergodic!principle}
    		All microstates corresponding to the same macroscopic state are equally propable.
	\end{theorem}
	
	\begin{theorem}[Boltzmann formula]\index{Boltzmann!entropy}
    		The central axiom of statistical mechanics gives following formula for the entropy:
		\begin{equation}
			\label{statmech:boltzmann_formula}
        		\boxed{S = k\ln\Omega(E, V, N, \alpha)}
		\end{equation}
	        where $\Omega$ denotes the number of microstates corresponding to the system with energy $E$, volume $V$, ...
	\end{theorem}
    
\section{Temperature}

    	\begin{formula}\index{temperature}
		The temperature of a system in contact with a heat bath is defined as:
		\begin{equation}
		    	\label{statmech:temperature}
			\boxed{T = \left(\pderiv{E}{S}\right)_V}
		\end{equation}
	\end{formula}

\section{Canonical ensemble}

	\newformula{Partition function}{\index{partition!function}\label{statmech:partition_function}
		The partition function for discrete systems is defined as:
		\begin{equation}
			\boxed{Z = \sum_i{g_ie^{-\beta\varepsilon_i}}}
		\end{equation}
		or for continuous systems:
		\begin{equation}
			Z(T) = \int \Omega(E, V, N)e^{-\beta E}dE
		\end{equation}
	}
	
	Consider a system of $N$ indistinguishable non-interacting particles. Let $\varepsilon_i$ be the energy associated with the $i^{th}$ energy level and let $g_i$ be its degeneracy. The probability $p_i$ of finding a particle in the $i^{th}$ energy level is given by:
    	\begin{equation}
		\boxed{p_i = \stylefrac{g_i e^{-\beta\varepsilon_i}}{Z}}
	\end{equation}
	
	\newdef{Helmholtz free energy}{\index{Helmholtz!free energy}
		A Legendre transform of the energy $E$ gives us:
		\begin{equation}
			F = -k_BT\ln Z = E - TS
		\end{equation}
	}

\section{Grand canonical ensemble}
    
        \newformula{Grand canonical partition function}{
        	The partition function of the $i^{th}$ energy level is defined as:
		\begin{equation}
			\mathcal{Z}_i = \sum_{n_k}e^{\beta n_k(\mu - \varepsilon_i)}
		\end{equation}
		The grand canonical partition function is then given by:
		\begin{equation}
			\boxed{\mathcal{Z} = \prod_i\mathcal{Z}_i = \sum_{n_k, \varepsilon_i}e^{\beta n_k(\mu - \varepsilon_i)}}
		\end{equation}
        }
        \remark{In the case of fermions, $n_i \in \{0, 1\}$, this formula reduces to $\mathcal{Z} = e^{\beta\mu}Z$.}
        
        
    	\newdef{Fugacity}{\index{fugacity}
        	\begin{equation}
        		\label{statmech:fugacity}
                	z = e^{\mu N}
        	\end{equation}
        }
        
\section{Energy}

	\begin{theorem}[Virial theorem]\index{virial theorem}
		\begin{equation}
			\label{statmech:virial_theorem}
			\boxed{\langle T \rangle = -\frac{1}{2}\sum_k\langle \vector{r}_k\cdot\vector{F}_k \rangle}
		\end{equation}
	\end{theorem}
	\begin{result}
		For potentials of the form $V = ar^{-n}$ this becomes:
		\begin{equation}
			2\langle T \rangle = -n\langle V \rangle
		\end{equation}
	\end{result}
	
	\begin{theorem}[Equipartition theorem]\index{equipartition theorem}
		Let $x$ be a generalized coordinate.
		\begin{equation}
			\boxed{\left\langle x^k\pderiv{H}{x^l} \right\rangle = k_bT\delta_{kl}}
		\end{equation}
	\end{theorem}
	\begin{result}
		For quadratic Hamiltonians this can be rewritten using Euler's theorem for homogeneous functions \ref{calculus:theorem:euler_homogeneous_functions} as:
		\begin{equation}
			\langle T \rangle = \frac{1}{2}k_bT
		\end{equation}
	\end{result}
	
\section{Black-body radiation}
	\begin{formula}[Planck's law]
		\begin{equation}
			\label{photon:plancks_law_frequency}
            \boxed{B_\nu(\nu, T) = \stylefrac{2h\nu^3}{c^2}\stylefrac{1}{e^{\frac{h\nu}{kt}} - 1}}
		\end{equation}
	\end{formula}
    
	\begin{formula}[Wien's displacement law]\index{Wien's displacement law}
		\begin{equation}
			\label{photon:wiens_displacement_law}
			\boxed{\lambda_{max}T = b}
		\end{equation}
		where $b = \num{2,8977729(17)E-3}\ $ Km is \textbf{Wien's displacement constant}.
	\end{formula}

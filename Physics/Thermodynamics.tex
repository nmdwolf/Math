\chapter{Thermodynamics}
\section{General definitions}

    \newdef{System}{\index{system}
        The part of space, and the objects contained in it, that we are interested in.
    }
    \newdef{Environment}{\index{environment}
        The complement of the system in space. More specifically this denotes the part of space aside of the system that has a potential influence on the system.
    }

    \newdef{Thermodynamic coordinates}{Macroscopical (i.e. it does not depend on any microscopic description) variables that describe the system. These are also called \textbf{state variables}.}
    \newdef{Intensive coordinate}{Coordinate that does not depend on the total amount of material (or equivalently on the system size). The opposite notion is called an \textbf{extensive coordinate}.}
    \newdef{Thermodynamic equilibrium}{A system in thermodynamic equilibrium is simultaneously in thermal, mechanical and chemical equilibrium. The system is fully described by a set of constant coordinates.}
    \begin{property}
        During thermodynamic equilibrium all intensive coordinates are uniform throughout the system.
    \end{property}

    \newdef{Isolated system}{An isolated system cannot interact with its environment (e.g. due to the presence of impenetrable walls).}
    \newdef{Diathermic wall}{A wall that allows heat transfer (and only heat transfer). This should be distinguished from an \textbf{adiabatic wall}, i.e. a wall that does not allow any transfer of heat.}
    \newdef{Heat bath}{\index{bath}
        A heat bath or \textbf{thermal reservoir} is a thermodynamic system (often part of the environment) for which the temperature remains constant under exchanging of heat (due to a virtually infinite heat capacity).
    }
    \newdef{Open system}{\index{open!system}
        A system that is allowed to interact with its environment.
    }
    \newdef{Quasistatic process}{A sequence of equilibrium states separated by infinitesimal changes.}
    \newdef{Path}{\index{path}
        The sequence of equilibrium states in a thermodynamic process is called its path.
    }

\section{Postulates}

    \begin{axiom}[Zeroth law]
        If two objects are in thermal equilibrium with a third object, they are also in thermal equilibrium with each other.
    \end{axiom}
    \begin{axiom}[First law]
        The change in internal energy is given by
        \begin{gather}
            \label{thermo:first_law}
            \Delta U = Q + W,
        \end{gather}
        or infinitesimally by
        \begin{gather}
            \label{thermo:first_law_differential}
            dU = \delta Q + \delta W
        \end{gather}
        where $W$ denotes the work done on the system and $Q$ denotes the heat that was extracted from the environment.
    \end{axiom}
    \sremark{The $\delta$ in the heat and work differentials implies that these are ''inexact'' differentials. This means that they are not the differential of functions of the thermodynamic coordinates alone. See section \ref{diff:section:forms} for more information on differential forms.}

    \begin{axiom}[Kelvin-Planck formulation of the second law]
        No machine can absorb an amount of heat and completely transform it into work.
    \end{axiom}
    \begin{axiom}[Clausius formulation of the second law]
        Heat cannot be passed from a cooler object to a warmer object without performing work.
    \end{axiom}

    \newformula{Clausius' inequality}{\index{Clausius!inequality}
        In differential form the inequality reads as
        \begin{gather}
            \label{thermo:clausius_inequality}
            \stylefrac{\delta Q}{T} \geq 0.
        \end{gather}
        ?? COMPLETE THIS STATEMENT (WHICH INEQUALITY?) ??
    }

    \begin{axiom}[Third law]
        No process can reach absolute zero through a finite sequence of operations.
    \end{axiom}

\section{Gases}
\subsection{Ideal gases}

    \begin{theorem}[Ideal gas law]\index{ideal gas constant}
        \begin{gather}
            \label{thermo:ideal_gas_law}
            PV = nRT
        \end{gather}
        where $R$ is the \textbf{ideal gas constant} $R\approx8.314 \frac{J}{K.\text{mol}}$.
    \end{theorem}
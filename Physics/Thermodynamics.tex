\chapter{Thermodynamics}
\section{General definitions}

    \newdef{System}{The part of space that we are interested in.}
    \newdef{Surroundings}{The complement of the system.}
    \newdef{Immediate surrounding}{The part of the surroundings that borders the system.}
    \newdef{Environment}{Everything except the immediate surroundings and the system.}

    \newdef{Thermodynamic coordinates}{Macroscopical variables that describe the system.}
    \newdef{Intensive coordinate}{Coordinate that does not depend on the total amount of material (or system size).}
    \newdef{Extensive coordinate}{Coordinate that does depend on the system size.}
    \newdef{Thermodynamic equilibrium}{A system in thermodynamic equilibrium is simultaneously in thermal, mechanical and chemical equilibrium. The system is also described by a set of constant coordinates.}
    \begin{property}
        During thermodynamic equilibrium all intensive coordinates are uniform throughout the system.
    \end{property}

    \newdef{Isolated system}{An isolated system cannot interact with its surroundings (e.g. due to the presence of impenetrable walls).}
    \newdef{Diathermic wall}{A wall that allows heat transfer.}
    \newdef{Adiabatic wall}{A wall that does not allow heat transfer.}
    \newdef{Open system}{A system that allows matter exchange.}
    \newdef{Closed system}{A system that does not allow matter exchange.}
    \newdef{Quasistatic process}{A sequence of equilibrium states separated by infinitesimal changes.}
    \newdef{Path}{The sequence of equilibrium states in a process is called the path.}

\section{Postulates}

    \begin{theorem}[Zeroth law]
        If two object are in thermal equilibrium with a third object, they are also in thermal equilibrium with each other.
    \end{theorem}
    \begin{theorem}[First law]
        The change in internal energy is given by
        \begin{gather}
            \label{thermo:first_law}
            U_f - U_i = W + Q
        \end{gather}
        or infinitesimally by
        \begin{gather}
            \label{thermo:first_law_differential}
            dU = \delta W + \delta Q.
        \end{gather}
    \end{theorem}
    \sremark{The $\delta$ in the heat and work differentials implies that these are ''inexact'' differentials. This means that they cannot be expressed as functions of the thermodynamic coordinates. See section \ref{diff:section:forms} for more information on differential forms.}

    \begin{theorem}[Kelvin-Planck formulation of the second law]
        No machine can absorb an amount of heat and completely transform it into work.
    \end{theorem}
    \begin{theorem}[Clausius formulation of the second law]
        Heat cannot be passed from a cooler object to a warmer object without performing work.
    \end{theorem}

    \newformula{Clausius' inequality}{\index{Clausius!inequality}
        In differential form, the inequality reads:
        \begin{gather}
            \label{thermo:clausius_inequality}
            \stylefrac{\delta Q}{T} \geq 0
        \end{gather}
        ?? COMPLETE THIS STATEMENT (WHICH INEQUALITY?) ??
    }

    \begin{theorem}[Third law]
        No process can reach absolute zero through a finite sequence of operations.
    \end{theorem}

\section{Gases}
\subsection{Ideal gases}

    \begin{theorem}[Ideal gas law]
        \begin{gather}
            \label{thermo:ideal_gas_law}
            PV = nRT
        \end{gather}
    \end{theorem}
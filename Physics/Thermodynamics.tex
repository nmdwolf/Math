\chapter{Thermodynamics}
	\section{General definitions}
    	\newdef{System}{The part of space that we are examining.}
        \newdef{Surroundings}{Everything outside the system.}
        \newdef{Immediate surrounding}{The part of the surroundings that 'lies' immediately next to the system.}
        \newdef{Environment}{Everything outside the immediate surroundings.}
        \newdef{Thermodynamic coordinates}{Macroscopical quantities that describe the system.}
        \newdef{Intensive coordinate}{Coordinate that does not depend on the total amount of material (or system size).}
        \newdef{Extensive coordinate}{Coordinate that does depend on the amount of material.}
		\newdef{Thermodynamic equilibrium}{A system in thermodynamic equilibrium is simultaneously in thermal, mechanical and chemical equilibrium. The system is also described by a certain set of constant coordinates.}
        \begin{theorem}
			During thermodynamic equilibrium, all intensive coordinates are uniform throughout the system.
		\end{theorem}
        \newdef{Isolated system}{An isolated system can't interact with its surroundings (due to the presence of impenetrable walls).
        }
        \newdef{Diathermic wall}{A diathermic wall is a wall that allows heat transfer.}
        \newdef{Adiabatic wall}{An adiabatic wall is a wall that does not allow heat transfer.}
        \newdef{Open system}{An open system is a system that allows matter exchange.}
        \newdef{Closed system}{A closed system is a system that does not allow matter exchange.}
        \newdef{Quasistatic process}{A quasistatic process is a sequence of equilibrium states separated by infinitesimal changes.}
        \newdef{Path}{The sequence of equilibrium states in a process is called the path.}

	\section{Postulates}
		\begin{theorem}[Zeroth law]
        	If two object are in thermal equilibrium with a third object then they are also in thermal equilibrium with eachother.
        \end{theorem}
        \begin{theorem}[First law]
        	\begin{equation}
				\label{thermo:first_law}
                U_f - U_i = W + Q
			\end{equation}
        	\begin{equation}
            	\label{thermo:first_law_differential}
                dU = \delta W + \delta Q
			\end{equation}
		\end{theorem}
        \sremark{The $\delta$ in the heat and work differentials implies that these are 'inexact' differentials. This means that they cannot be expressed as functions of the thermodynamic coordinates. More formally a differential form $dx$ is called inexact if the integral $\int dx$ is path dependent.}
        
        \begin{theorem}[Kelvin-Planck formulation of the second law]
        	No machine can absorb an amount of heat and completely transform it into work.
		\end{theorem}
        \begin{theorem}[Clausius formulation of the second law]
			Heat cannot be passed from a cooler object to a warmer object without performing work.
		\end{theorem}
        \newformula{Clausius' inequality}{\index{Clausius!inequality}
        	In differential form, the inequality reads:
        	\begin{equation}
        		\label{thermo:clausius_inequality}
                \stylefrac{\delta Q}{T} \geq 0
        	\end{equation}
        }
        
        \begin{theorem}[Third law]
        	No process can reach absolute zero in a finite sequence of operations.
		\end{theorem}
	\section{Gases}
    	\subsection{Ideal gases}
        	\begin{theorem}[Ideal gas law]
            	\begin{equation}
					\label{thermo:ideal_gas_law}
                    \boxed{PV = nRT}
				\end{equation}
            \end{theorem}
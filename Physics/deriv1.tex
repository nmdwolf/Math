\chapter{Derivations: Lagrangian formalism}
\section{d'Alembert's principle}\label{deriv:lagrange}
    In the following derivation we assume a constant mass.
    \begin{align}
        &\sum_k \left(\vector{F}_k - \dot{\vector{p}}_k\right)\dot{\vector{r}}_k = 0\nonumber\\
        \iff&\sum_k \left(\vector{F}_k - \dot{\vector{p}}_k\right)\cdot\left(\sum_l\pderiv{\vector{r}}{q_l}\dot{q_l}\right) = 0\nonumber\\
        \iff&\sum_l\left(\sum_k\vector{F}_k\cdot\pderiv{\vector{r}}{q_l} - \sum_km\ddot{\vector{r}}\cdot\pderiv{\vector{r}}{q_l}\right)\dot{q_l} = 0\nonumber\\
        \iff&\sum_l\left(Q_l - \sum_km\ddot{\vector{r}}\cdot\pderiv{\vector{r}}{q_l}\right)\dot{q_l} = 0\label{lagrange_deriv:deriv1}
    \end{align}
    \noindent Now we look at the following derivative:
    \begin{align}
        &\deriv{}{t}\left(\dot{\vector{r}}\cdot\pderiv{\vector{r}}{q_l}\right) = \ddot{\vector{r}}\cdot\pderiv{\vector{r}}{q_l} + \dot{\vector{r}}\cdot\deriv{}{t}\left(\pderiv{\vector{r}}{q_l}\right)\nonumber\\
        \iff&\ddot{\vector{r}}\cdot\pderiv{\vector{r}}{q_l} = \deriv{}{t}\left(\dot{\vector{r}}\cdot\pderiv{\vector{r}}{q_l}\right) - \dot{\vector{r}}\cdot\deriv{}{t}\left(\pderiv{\vector{r}}{q_l}\right)\nonumber\\
        \iff&\ddot{\vector{r}}\cdot\pderiv{\vector{r}}{q_l} = \deriv{}{t}\left(\dot{\vector{r}}\cdot{\color{red}\underbrace{\textcolor{black}{\pderiv{\vector{r}}{q_l}}}_A}\right) - \dot{\vector{r}}\cdot\left(\pderiv{\dot{\vector{r}}}{q_l}\right)\label{lagrange_deriv:deriv2}
    \end{align}
    To evaluate A we can take a look at another derivative:
    \begin{align}
        \pderiv{\dot{\vector{r}}}{\dot{q}_l} &= \pderiv{}{\dot{q}_l}\left(\sum_k\pderiv{r}{q_k}\dot{q}_k\right)\nonumber\\
        &=\sum_k\pderiv{r}{q_k}\delta_{kl}\nonumber\\
        &=\pderiv{\vector{r}}{q_l}\nonumber\\
        &=\textcolor{red}{A}\nonumber
    \end{align}
    Substituting this in formula \ref{lagrange_deriv:deriv2} gives:
    \begin{align}
        \ddot{\vector{r}}\cdot\pderiv{\vector{r}}{q_l} &= \deriv{}{t}\left(\dot{\vector{r}}\cdot\pderiv{\dot{\vector{r}}}{\dot{q}_l}\right) - \dot{\vector{r}}\cdot\left(\pderiv{\dot{\vector{r}}}{q_l}\right)\nonumber\\
        &=\deriv{}{t}\left(\stylefrac{1}{2}\pderiv{\dot{\vector{r}}^2}{\dot{q}_l}\right) - \stylefrac{1}{2}\pderiv{\dot{\vector{r}}^2}{q_l}\label{lagrange_deriv:deriv3}
    \end{align}
    If we multiply this by the mass $m$ and sum over all particles we get :
    \begin{align}
        \sum_km_k\ddot{\vector{r}}_k\cdot\pderiv{\vector{r}_k}{q_l}=\ &\deriv{}{t}\pderiv{}{\dot{q}_l}\left(\sum_k\stylefrac{1}{2}m\dot{\vector{r}}_k^2\right) - \pderiv{}{q_l}\left(\sum_k\stylefrac{1}{2}m\dot{\vector{r}}_k^2\right)\nonumber\\
        =\ &\deriv{}{t}\pderiv{T}{\dot{q}_l} - \pderiv{T}{q_l}\label{lagrange_deriv:deriv4}
    \end{align}
    Where we have denoted the total kinetic energy in the last line as $T$.\\
    Plugging this result into formula \ref{lagrange_deriv:deriv1} gives us:
    \begin{equation}
        \label{lagrange_deriv:deriv5}
        \sum_l\left(Q_l - \deriv{}{t}\pderiv{T}{\dot{q}_l} - \pderiv{T}{q_l}\right)\dot{q_l} = 0
    \end{equation}
    As all the $q_l$ are independent the following relation should hold for all $l$:
    \begin{align}
        &Q_l - \deriv{}{t}\left(\pderiv{T}{\dot{q}_l}\right) - \pderiv{T}{q_l} = 0\nonumber\\
        \iff&\boxed{\deriv{}{t}\left(\pderiv{T}{\dot{q}_l}\right) - \pderiv{T}{q_l} = Q_l}\label{lagrange_deriv:first_kind}
    \end{align}
    This last equation is known as a \textbf{Lagrange equation of the first kind}.\\
    If we have a system with only conservative forces acting on it, we can write the force on the $i$-th particle  as:
    \begin{equation}
        \label{lagrange_deriv:deriv7}
        F_i = -\nabla_iV
    \end{equation}
    With this in mind, lets take a look at the derivative of the potential $V$ with respect to the $l$-th generalized coordinate:
    \begin{equation}
        \label{lagrange_deriv:deriv8}
        \begin{aligned}
            \pderiv{V}{q_l} &= \sum_i\left(\nabla_iV\right)\cdot\pderiv{\vector{r}_i}{q_l}\\
            &=-Q_l
        \end{aligned}
    \end{equation}
    The differentiation of $V$ with respect to any generalized velocity $\dot{q}_l$ is trivially zero. This combined with the last formula \ref{lagrange_deriv:deriv8} and with formula \ref{lagrange_deriv:first_kind} gives:
    \begin{align}
        &\deriv{}{t}\left(\pderiv{T}{\dot{q}_l}\right) - \pderiv{T}{q_l} = Q_l\nonumber\\
        \iff&\deriv{}{t}\left(\pderiv{T}{\dot{q}_l}\right) - \pderiv{T}{q_l} = -\pderiv{V}{q_l} + \pderiv{V}{\dot{q}_l}\nonumber\\
        \iff&\deriv{}{t}\left(\pderiv{T}{\dot{q}_l} - \pderiv{V}{\dot{q}_l}\right) - \pderiv{}{q_l}\left(T - V\right) = 0\label{lagrange:deriv9}
    \end{align}
    If we introduce a new variable $L$, called the \textbf{Lagrangian}, we get the \textbf{Lagrangian equation of the second kind}:
    \begin{equation}
        \label{lagrange_deriv:second_kind}
        \boxed{\deriv{}{t}\left(\pderiv{L}{\dot{q}_l}\right) - \pderiv{L}{q_l} = 0}
    \end{equation}

\section{Hamilton's principle}
	In this part we start from the principle of least action. First we the define the \textbf{action} as following:
    \begin{equation}
		\label{lagrange_deriv:action_integral}
        \boxed{I = \int_{t_1}^{t_2}L\left(y(t), \dot{y}(t), t\right)dt}
	\end{equation}
    Then we require that this action is minimal for the physically acceptable path.
    To do this we define a family of paths:
    \begin{equation}
		\label{lagrange_deriv:family_of_paths}
        y(t, \alpha) = y(t) + \alpha\eta(t)
	\end{equation}
    Where $\eta(t)$ is an arbitrary function with the following boundary conditions:
    \begin{equation}
	    \left\{
        \begin{aligned}
        \eta(t_1) = 0\\
        \eta(t_2) = 0
        \end{aligned}
		\right.
	\end{equation}
    If we define the action integral over this family of paths, the integral \ref{lagrange_deriv:action_integral} becomes a function of $\alpha$:
    \begin{equation}
		\label{lagrange_deriv:action_integral_over_family}
        I(\alpha) = \int_{t_1}^{t_2}L\left(y(t,\alpha), \dot{y}(t, \alpha), t\right)dt
	\end{equation}
    Requiring that the action integral is stationary for $y(t)$ (thus $\alpha = 0$) is equivalent to:
    \begin{equation}
		\label{lagrange_deriv:stationary_condition}
        \left(\deriv{I}{\alpha}\right)_{\alpha=0} = 0
	\end{equation}
    This condition combined with formula \ref{lagrange_deriv:action_integral_over_family} gives us:
    \begin{equation}
    	\label{lagrange_deriv:derivative_of_integral}
        \deriv{I}{\alpha} = \int_{t_1}^{t_2}\deriv{}{\alpha}L\left(y(t,\alpha), \dot{y}(t, \alpha), t\right)dt
	\end{equation}
    As we evaluate this derivative in $\alpha = 0$ we can replace $y(t, \alpha)$ by $y(t)$ due to definition \ref{lagrange_deriv:family_of_paths}.
	\begin{align}
        \deriv{I}{\alpha}&=\int_{t_1}^{t_2}\left[\pderiv{L}{y}\pderiv{y}{\alpha} + \pderiv{L}{\dot{y}}\pderiv{\dot{y}}{\alpha}\right]dt\nonumber\\
        &=\int_{t_1}^{t_2}\left[\pderiv{L}{y}\eta(t) + \pderiv{L}{\dot{y}}\dot{\eta}(t)\right]dt
	\end{align}
    If we substitute $\pderiv{L}{\dot{y}} := h(t)$ and apply integration by parts to the second term in this integral, we get:
    \begin{align}
        \deriv{I}{\alpha}&=\int_{t_1}^{t_2}\left[\pderiv{L}{y}\eta(x) + h(t)\dot{\eta}(t)\right]dt\nonumber\\
        &=\int_{t_1}^{t_2}\left[\pderiv{L}{y}\eta(t) + h(t)\deriv{\eta}{t}\right]dt\nonumber\\
        &=\int_{t_1}^{t_2}\pderiv{L}{y}\eta(t)dt + \eta(t_2)h(t_2) - \eta(t_1)h(t_1) - \int_{t_1}^{t_2}\deriv{}{t}\left(\pderiv{L}{\dot{y}}\right)\eta(t)dt
	\end{align}
    Due to the initial conditions \ref{lagrange_deriv:stationary_condition} for the function $\eta(t)$, the two terms in the middle vanish and we obtain:
    \begin{equation}
		\label{lagrange_deriv:final_integral}
        \deriv{I}{\alpha}=\int_{t_1}^{t_2}\left[\pderiv{L}{y} - \deriv{}{t}\left(\pderiv{L}{\dot{y}}\right)\right]\eta(t)dt
	\end{equation}
    Furthermore, as the function $\eta(t)$ was arbitrary, the only possible way that this derivative can become zero is when the integrand is identically zero:
    \begin{equation}
		\label{lagrange_deriv:second_kind_with_hamilton}
        \boxed{\pderiv{L}{y} - \deriv{}{t}\left(\pderiv{L}{\dot{y}}\right) = 0}
	\end{equation}
    If we compare this result with formula \ref{lagrange_deriv:second_kind} we see that we can also obtain the \textbf{Lagrangian equations of the second kind} by starting from the principle of least action. (Where the variable $y$ represents the generalized coordinates $q_l$ and the variable $\dot{y}$ represents the generalized velocities $\dot{q}_l$)
    
    \begin{remark}
		Differential equations of the form
        \begin{equation}
			\label{lagrange_deriv:euler_lagrange_equation}
            \boxed{\pderiv{f}{y}(y, \dot{y}, x) = \deriv{}{x}\left(\pderiv{f}{\dot{y}}(y, \dot{y}, x)\right)}
		\end{equation}
        are known as \textbf{Euler-Lagrange equations}.
	\end{remark}

\section{Explanation for Noether's theorem \ref{qft:noethers_theorem}}\label{proof:noether}

	The general transformation rule for the Lagrangian is:
	\begin{equation}
		\label{noether_deriv:1}
		\mathcal{L}(x)\rightarrow\mathcal{L}(x) + \alpha\delta\mathcal{L}(x)
	\end{equation}
	To have a symmetry, i.e. keep the action invariant, the deformation factor has to be a 4-divergence:
	\begin{equation}
		\label{noether_deriv:2}
		\mathcal{L}(x)\rightarrow\mathcal{L}(x) + \alpha\partial_\mu\mathcal{J}^\mu(x)
	\end{equation}

	To obtain formula \ref{qft:conserved_current} we vary the Lagrangian explicitly:
	\begin{align*}
		\delta\mathcal{L} &= \pderiv{\mathcal{L}}{\phi}\delta\phi + \pderiv{\mathcal{L}}{(\partial_\mu\phi)}\delta(\partial_\mu\phi)\\
		&= \pderiv{\mathcal{L}}{\phi}\delta\phi + \partial_\mu\left(\pderiv{\mathcal{L}}{(\partial_\mu\phi)}\delta\phi\right) - \partial_\mu\left(\pderiv{\mathcal{L}}{(\partial_\mu\phi)}\right)\delta\phi\\
		&= \partial_\mu\left(\pderiv{\mathcal{L}}{(\partial_\mu\phi)}\delta\phi\right) + \left[\pderiv{\mathcal{L}}{\phi} - \pderiv{\mathcal{L}}{(\partial_\mu\phi)}\right]\delta\phi
	\end{align*}
	The second term vanishes due to the Euler-Lagrange equation \ref{lagrange_deriv:second_kind_with_hamilton}. Combining these formulas gives us:
	\begin{equation}
		\partial_\mu\left(\pderiv{\mathcal{L}}{(\partial_\mu\phi)}\delta\phi\right) - \partial_\mu\mathcal{J}^\mu(x) = 0
	\end{equation}
	From this equation we can conclude that the current
	\begin{equation}
		\boxed{j^\mu(x) = \pderiv{\mathcal{L}}{(\partial_\mu\phi)}\delta\phi - \mathcal{J}^\mu(x)}
	\end{equation}
	is conserved.


\chapter{Derivations: Optics and material physics}
\section{Optics}
	\subsection{Law of Lambert-Beer \ref{optics:lambert_beer}}
    From formula \ref{optics:dielectric_function_non_magnetic} we now that the complex refractive index can be written as \[\widetilde{n} = n+ik\]Where $k$ is called the \textbf{extinction coefficient}.
    \\From classical optics we also know that in a material the speed of light obeys the following relation: \[c = \widetilde{n}v\]Where we have used the complex refractive index. It readily follows that the wavenumber (sadly also given the letter $k$) can be written as:\[k = \stylefrac{\omega}{v} = \widetilde{n}\stylefrac{\omega}{c}\]
    From classical electromagnetism we know that a plane wave can be written as:\[E(x, t) = Re\left\{A\ \text{exp}\left[i(kx - \omega t + \phi)\right]\right\}\]
    So everything put together we get:\[E(x, t) = Re\left\{A\ \text{exp}\left[i\left((n+ik)\stylefrac{\omega}{c}x - \omega t + \phi\right)\right]\right\}\]
    or also:\[E(x, t) = Re\left\{A\ \text{exp}\left[in\stylefrac{\omega}{c}x\right]\cdot\text{exp}\left[-k\stylefrac{\omega}{c}x\right]\cdot\text{exp}\left[-i\omega t\right]\cdot\text{exp}\left[i\phi\right]\right\}\]
    We also know that the intensity is given by the following relation:\[I(x) = |E(x)|^2 = E^*(x)\cdot E(x)\]
    So only the second factor will remain. Dividing this by its value for $x=0$ we get:\[\stylefrac{I(x)}{I(0)} = \stylefrac{E(x)\cdot E^*(x)}{E(0)\cdot E^*(0)} = \text{exp}\left[-\stylefrac{2k\omega}{c}x\right] = \text{exp}[-\alpha x]\]
    Where $\alpha$ is the absorption coefficient as defined in formula \ref{optics:absorption_coefficient}.\qed

\chapter{Derivations: Classical and Statistical Mechanics}
\section{Moments of inertia}\label{deriv:inertia}\index{inertia}

In this section we will always use formula \ref{forces:moment_of_inertia} to calculate the moment of inertia.
\subsection{Disk}
	
	The volume of a (solid) disk is given by:
	\begin{equation}
		V_{disk} = \pi R^2d
	\end{equation}
	where $R$ is the radius and $d$ is the thickness. The mass density is then given by:
	\begin{equation}
		\rho = \frac{M}{\pi R^2d}
	\end{equation}
	Using cylindrical coordinates the moment of inertia then becomes:
	\begin{align}
		I &= \frac{M}{\pi R^2d}\int_0^{2\pi}d\varphi\int_0^ddz\int_0^Rr^3dr\\
		&= \frac{M}{\pi R^2d}2\pi d\frac{R^4}{4}\\
		&= \frac{1}{2}MR^2
	\end{align}
	
\subsection{Solid sphere}

	The volume of a solid sphere is given by:
	\begin{equation}
		V_{sphere} = \frac{4}{3}\pi R^3
	\end{equation}
	where $R$ is the radius. The mass density is then given by:
	\begin{equation}
		\rho = \frac{M}{\frac{4}{3}\pi R^3}
	\end{equation}
	We will use spherical coordinates to derive the moment of inertia, but we have to be carefull. The $r$ in formula \ref{forces:moment_of_inertia} is the distance between a point in the body and the axis of rotation. So it is not the same as the $r$ in spherical coordinates which is the distance between a point and the origin. However the relation between these two quantities is easily found using basic geometry to be:
	\begin{equation}
		r = r'\sin\theta
	\end{equation}
	where $r'$ is the spherical coordinate. Now we can calculate the moment of inertia as follows:
	\begin{align}
		I &= \frac{M}{\frac{4}{3}\pi R^3} \int_0^{2\pi}d\varphi\int_0^Rr'^{^4}dr'\int_0^\pi\sin^3\theta d\theta\\
		&= \frac{M}{\frac{4}{3}\pi R^3} 2\pi \frac{R^5}{5} \frac{4}{3}\\
		&= \frac{2}{5}MR^2
	\end{align}

\section{Schottky defects}\label{deriv:schottky_defects}\index{Schottky!defect}
	Let $E_v$ be the energy needed to remove a particle from its lattice point and move it to the surface. Furthermore we will neglect any surface effects and assume that the the energy $E_v$ is indepedent of the distance to the surface.
    
    The total energy of all vacancies is then given by $E = nE_v$. The number of possible microstates is
    \begin{equation}
    	\Omega = \stylefrac{(N+n)!}{n!N!}
	\end{equation}
	where we used the fact that the removal of $n$ particles creates $n$ more lattice points at the surface. Using Boltzmann's entropy formula \ref{statmech:boltzmann_formula} and Stirling's formula we obtain
    \begin{equation}
    	S(N, n) = k\ln\Omega = k\big[(N+n)\ln(N+n) -n\ln n - N\ln N \big]
    \end{equation}
    Using \ref{statmech:temperature} we can find the temperature:
    \begin{equation}
    	\stylefrac{1}{T} = \left(\pderiv{S}{E}\right)_{N, V} = \deriv{S}{n}\deriv{n}{E} = \frac{k}{E_v}\ln\frac{N+n}{n}
    \end{equation}
    which can be rewritten as
    \begin{equation}
    	\boxed{\stylefrac{n}{N + n} = \exp\left(-\frac{E_v}{kT}\right)}
    \end{equation}
    
    The density of Frenkel defects can be derived analogously.\index{Frenkel pair}

\chapter{Angular Momentum}
	In this chapter we consider the general angular momentum operator $\hat{J} = \left(\hat{J}_x, \hat{J}_y, \hat{J}_z\right)$. This operator works on the Hilbert space spanned by the eigenbasis $\{|j, m\rangle\}$.

\section{General operator}
    
	\begin{property}
    		The mutual eigenbasis of $\hat{J}^2$ and $\hat{J}_z$ is defined by the following two eigenvalue equations:
        	\begin{align}
        		\label{QM:angular_momentum:j}
        		\hat{J}^2|j, m\rangle &= j(j+1)\hbar^2|j, m\rangle\\
        		\label{QM:angular_momentum:m}
        		\hat{J}_z|j, m\rangle &= m\hbar|j, m\rangle
	        \end{align}
	\end{property}

	\begin{property}
    		The angular momentum operators generate a Lie algebra \ref{linalgebra:lie_algebra}. The Lie bracket is defined by following commutation relation:
        	\begin{equation}
        		\label{QM:angular_momentum:commutation}
        		\boxed{\left[\hat{J}_i, \hat{J}_j\right] = i\hbar\varepsilon_{ijk}\hat{J}_k}
        	\end{equation}
	\end{property}
    
	\newdef{Ladder operators\footnotemark}{\index{ladder operators}
	        \footnotetext{Also called the \textbf{creation} and \textbf{annihilation} operators (especially in quantum field theory).}
	    	The raising and lowering operators\footnote{These operators will only affect the $z$-projection, not the total angular momentum.} $\hat{J}_+$ and $\hat{J}_-$ are defined as:
	        \begin{equation}
	        	\hat{J}_+ = \hat{J}_x + i\hat{J}_y\qquad\text{and}\qquad\hat{J}_- = \hat{J}_x - i\hat{J}_y
	        \end{equation}
	}
	\begin{result}
	    	From the commutation relations of the angular momentum operators we can derive the commutation relations of the ladder operators:
	        \begin{equation}
	        	\left[\hat{J}_+, \hat{J}_-\right] = 2\hbar\hat{J}_z
	        \end{equation}
	\end{result}
    
	\begin{formula}
	    	The total angular momentum operator $\hat{J}^2$ can now be expressed in terms of $\hat{J}_z$ and the ladder operators using commutation relation \ref{QM:angular_momentum:commutation}:
	        \begin{equation}
	        	\hat{J}^2 = \hat{J}_+\hat{J}_- + \hat{J}_z^2 - \hbar\hat{J}_z
	        \end{equation}
	\end{formula}
        \begin{remark}[Casimir operator]\index{Casimir!invariant}
    		From the definition of $\hat{J}^2$ it follows that this operator is a Casimir invariant\footnote{See definition \ref{lie:casimir_invariant}.} in the algebra generated by the operators $\hat{J}_i$.
	\end{remark}
    
\section{Rotations}
\subsection{Infinitesimal rotation}
	
	\begin{formula}
		An infinitesimal rotation $\hat{R}(\delta\vector{\varphi})$ is given by the following formula:
	        \begin{equation}
        		\label{QM:angular_momentum:infinitesimal_rotation}
        		\boxed{\hat{R}(\delta\vector{\varphi}) = \mathbbm{1} - \frac{i}{\hbar}\vector{J}\cdot\delta\vector{\varphi}}
	        \end{equation}
        	A finite rotation can then be produced by applying this infinitesimal rotation repeatedly, which gives:
        	\begin{equation}
		        \label{QM:angular_momentum:finite_rotation}
        		\hat{R}(\vector{\varphi}) = \left(\mathbbm{1} - \frac{i}{\hbar}\vector{J}\cdot\frac{\vector{\varphi}}{n}\right)^n = \exp\left(-\frac{i}{\hbar}\vector{J}\cdot\vector{\varphi}\right)
	        \end{equation}
	\end{formula}
    
	\newformula{Matrix elements}{
	    	Applying a rotation over an angle $\varphi$ around the $z$-axis to a state $|j, m\rangle$ gives:
        	\begin{equation}
        		\hat{R}(\varphi\vector{e}_z)|j, m\rangle = \exp\left(-\frac{i}{\hbar}\hat{J}_z\varphi\right)|j, m\rangle = \exp\left(-\frac{i}{\hbar}m\varphi\right)|j, m\rangle
        	\end{equation}
        	Multiplying these states with a bra $\langle j', m'|$ and using the orthonormality of the eigenstates gives the matrix elements of the rotation operator:
        	\begin{equation}
        		\boxed{\hat{R}_{ij}(\varphi\vector{e}_z) = \exp\left(-\frac{i}{\hbar}m\varphi\right)\delta_{jj'}\delta_{mm'}}
        	\end{equation}
        	From the expression of the angular momentum operators and the rotation operator it is clear that a general rotation has no effect on the total angular momentum number $j$. This means that the rotation matrix will be a block diagonal matrix with respect to $j$. This amounts to the following reduction of the representation of the rotation group:
        	\begin{equation}
        		\boxed{\langle j, m'|\hat{R}(\varphi\vector{n})|j, m\rangle = \mathcal{D}^{(j)}_{m, m'}(\hat{R})}
        	\end{equation}
        	where the values $\mathcal{D}^{(j)}_{m, m'}(\hat{R})$ are the \textbf{Wigner D-functions}.
	}
	\begin{remark*}[Wigner D-functions]\index{Wigner!D-functions}
    		For every value of $j$ there are $(2j+1)$ values for $m$. The matrix $\mathcal{D}^{(j)}(\hat{R})$ is thus a $(2j+1)\times(2j+1)$-matrix
	\end{remark*}
    
\subsection{Spinor representation}

	\newdef{Pauli matrices}{\index{Pauli!matrices}
	    	\begin{equation}
	    		\label{QM:angular_momentum:pauli_matrices}
	    		\boxed{
			\sigma_x = \left(
			\begin{array}{cc}
		    		0&1\\
		        	1&0
			\end{array}
			\right)\qquad
			\sigma_y = \left(
			\begin{array}{cc}
		    		0&-i\\
		        	i&0
			\end{array}
			\right)\qquad
			\sigma_z = \left(
			\begin{array}{cc}
		    		1&0\\
		        	0&-1
			\end{array}
			\right)}
		\end{equation}
		From this definition it is clear that the Pauli matrices are Hermitian and unitary. Together with the $2\times2$ identity matrix\footnote{In the context of relativistic QM one often denotes the $2\times2$ identity matrix as $\sigma_0$.} they form a basis for the space of $2\times2$ Hermitian matrices.
	}

    \begin{formula}
    	In the spinor representation (J = $\frac{1}{2}$) the Wigner-D matrix reads:
    	\begin{equation}
    		\mathcal{D}^{(1/2)}(\varphi\vector{e}_z) = \left(
            \begin{array}{cc}
            	e^{-i/2 \varphi}&0\\
                0&e^{i/2\varphi}
            \end{array}
            \right)
    	\end{equation}
    \end{formula}
    
    
\section{Coupling of angular momenta}
\subsection{Total Hilbert space}
    
    Let $\mathcal{H}_i$ denote the Hilbert space of states belonging to the $i^{th}$ particle. The Hilbert space of the total system is given by the following tensor product:
    \[
    	\mathcal{H} = \mathcal{H}_1 \otimes ... \otimes \mathcal{H}_n
    \]
    Due to the tensor product definition above, the angular momentum operator $\hat{J}_i$ should now be interpreted as $\mathbbm{1}\otimes...\otimes\hat{J}_i\otimes...\otimes\mathbbm{1}$. This implies that the angular momentum operators $\hat{J}_{l\neq i}$ do not act on the space $\mathcal{H}_i$, so one can pull these operators through the tensor product:
    \[
    	\hat{J}_i|j_1\rangle\otimes...\otimes|j_n\rangle = |j_1\rangle\otimes...\otimes\hat{J}_i|j_i\rangle\otimes...\otimes|j_n\rangle
    \]
    The basis used above is called the \textbf{uncoupled basis}.
    
\subsection{Clebsch-Gordan series}

	Let $\vector{J}$ denote the total angular momentum defined as:
	\begin{equation}
	    	\vector{J} = \hat{J}_1 + \hat{J}_2
	\end{equation}
	With this operator we can define a \textbf{coupled} state $|\mathbf{J}, \mathbf{M}\rangle$ where $\mathbf{M}$ is the total magnetic quantum number which ranges from $-\mathbf{J}$ to $\mathbf{J}$.
    
	\newformula{Clebsch-Gordan coefficients}{
	        Because both bases (coupled and uncoupled) span the total Hilbert space $\mathcal{H}$ there exists a transformation between them. The transformation coefficients can be found by using the resolution of the identity:
	        \begin{equation}
		        \label{QM:angular_momentum:clebsch-gordan}
        		\boxed{|\mathbf{J}, \mathbf{M}\rangle = \sum_{m_1 = -j_1}^{j_1}\sum_{m_2 = -j_2}^{j_2} |j_1, j_2, m_1, m_2\rangle\langle j_1, j_2, m_1, m_2|\mathbf{J}, \mathbf{M}\rangle}
	        \end{equation}
        	These coefficients are called the Clebsch-Gordan coefficients.
	}
    
    \begin{property}
    	By acting with the operator $\hat{J}_z$ on both sides of equation \ref{QM:angular_momentum:clebsch-gordan} it is possible to proof that the CG coefficient are non-zero if and only if $\mathbf{M} = m_1 + m_2$.
    \end{property}

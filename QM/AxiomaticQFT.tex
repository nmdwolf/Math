\chapter{Axiomatic QFT}

\section{Algebraic QFT}
\subsection{Haag-Kastler axioms}

	\begin{axiom}[Local net of observables]\index{microcausality}\label{qft:microcausality}
		To every causally closed set\footnote{See definition \ref{relativity:causal_closure}.} $O$ one associates a $C^*$-algebra $\mathcal{A}(O)$. We require this assignement to satisfy the following conditions:
		\begin{itemize}
			\item Isotony: If $O_1\subset O_2$ then $\mathcal{A}(O_1)\subset\mathcal{A}(O_2)$.
			\item (Causal) locality\footnote{Also called \textbf{microcausality}.}: If $O_1$ and $O_2$ are spacelike separated then $[\mathcal{A}(O_1), \mathcal{A}(O_2)] = 0$ (graded commutator) within a larger algebra $\mathcal{A}(O)$ where $O_1, O_2\subset O$.
		\end{itemize}
	\end{axiom}
	
	\begin{axiom}[Haag duality]\index{Haag!duality}
		Let $\overline{O}$ denote the spacelike complement of $O$ and let $\mathcal{A}'$ denote the commutant of $\mathcal{A}$. Haag duality states that\footnote{Here it should be understood that $\mathcal{A}\left(\overline{O}\right)$ is the algebra generated by all algebras $\mathcal{1}(Q)$ where $Q$ ranges over the causally closed sets in $\overline{O}$.}
		\begin{equation}
			\mathcal{A}\left(\overline{O}\right)' = \mathcal{A}(O)
		\end{equation}
		for all causally closed sets $O$.
	\end{axiom}
	\remark{Haag duality is known to hold for all free theories and for some less trivial theories. However it is also known to fail in the case of symmetry breaking \cite{Roberts_haag}.}
	
\subsection{Weyl systems}

	\newdef{Weyl system}{\index{Weyl!system}
		Let $(L, \omega)$ be a symplectic vector space. Let $K$ be a complex vector space and let $W$ be a map from $L$ to the space of unitary operators on $K$. The pair $(K, W)$ is a Weyl system over $(L, \omega)$ if it satisfies:
		\begin{equation}
			W(z)W(z') = e^{\frac{i}{2}\omega(z, z')}W(z+z')
		\end{equation}
		for all $z, z'\in K$.
	}
	\newdef{Heisenberg system}{\index{Heisenberg!system}
		The generators\footnote{Their existence is guaranteed by Stone's theorem.} $\phi(z)$ of the map $t\mapsto W(tz)$ are said to form a Heisenberg system. These operators satisfy following properties:
		\begin{itemize}
			\item $\lambda\phi(z) = \phi(\lambda z)$ for all positive $\lambda$.
			\item $[\phi(z), \phi(z')] = -i\omega(z, z')$
			\item $\phi(z+z')$ is the closure\footnote{See definition \ref{operator:closure}.} of $\phi(z)+\phi(z')$
		\end{itemize}
	}
	
\section{Topological QFT}
\subsection{Atiyah-Segal axioms}

	\begin{axiom}[Atiyah-Segal]\index{Atiyah-Segal}
		\nomenclature[A_TQFT]{TQFT}{Topological quantum field theory}
		A $d$-dimensional topological quantum field theory (TQFT) is a symmetric monoidal functor $F:\textbf{Bord}_{d-1}^d\rightarrow\textbf{FinVect}$ satisfying following axioms:
		\begin{enumerate}
			\item Normalization: $F(\emptyset)=\mathbb{C}$
			\item Disjoint union: $F(M\sqcup M') = F(M)\otimes F(M')$
			\item Composition: If $N=M\cup M'$ where $\partial M$ and $\partial M$ have opposite orientation then: \[F(N) = F(M)\circ F(M')\]
			\item Invariance: If $f: M\rightarrow M'$ is a diffeomorphism rel boundary then $F\circ f = F$.
		\end{enumerate}
		where $M, M'$ are $d$-dimensional cobordisms between $(d-1)$-dimensional (closed) smooth manifolds.
	\end{axiom}
	
	\begin{example}[1D]
		In 1D one has the following possible smooth cobordisms and their associated TQFT operations:
		\begin{equation*}
			\begin{array}{l|l}
				\text{Point with orientation } + & \text{Vector space } V\\
				\text{Point with reversed orientation } - & \text{Dual space } V^*\\
				\text{Line between points} & \text{Linear map } f:V\rightarrow V\\
				\text{Cap between $\emptyset$ and points } +, - & \text{Coevaluation } \mathbb{C}\rightarrow V\otimes V^*\\
				\text{Cup between points $-, +$ and }\emptyset & \text{Evaluation } V^*\otimes V\rightarrow\mathbb{C}
			\end{array}
		\end{equation*}
		Essentially this gives us the structure of a (finite-dimensional) vector space with dual.
	\end{example}
	
	\begin{example}[2D]
		In 2D one can obtain a similar result by drawing all possible configurations, however the existence and combination of "\textit{pair of pants}"-diagrams gives a richer stucture. For 2D TQFT's the corresponding object is a (finite-dimensional) commutative and cocommutative Frobenius algebra.
	\end{example}
	
	In dimensions 3 and higher the definition above is intractable. To allow the construction to be generalized to higher dimensions one considers the following definition:
	\newdef{Extended TQFT}{
		A $d$-dimensional extended TQFT is a symmetric monoidal functor $F:\textbf{Bord}_1^d\rightarrow\textbf{FinVect}$ satisfying the Atiyah-Segal axioms where the invariance axiom is required only at the highest level of $k$-morphisms.
	}

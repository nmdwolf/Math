\chapter{Axiomatic QFT}

\section{Algebraic QFT}
\subsection{Haag-Kastler axioms}

	\begin{axiom}[Local net of observables]\index{microcausality}\label{qft:microcausality}
		To every causally closed set\footnote{See definition \ref{relativity:causal_closure}.} $O$ one associates a $C^*$-algebra $\mathcal{A}(O)$. We require this assignement to satisfy the following conditions:
		\begin{itemize}
			\item \textbf{Isotony}\footnote{This implies that local nets of observables are modelled by copresheafs $\mathbf{Mink}\rightarrow\mathbf{C^*Alg}$ such that morphisms in $\mathbf{Mink}$ are mapped to monomorphisms.}: If $O_1\subset O_2$ then $\mathcal{A}(O_1)\hookrightarrow\mathcal{A}(O_2)$.
			\item \textbf{(Causal) locality}\footnote{Also called \textbf{microcausality} or \textbf{Einstein causality}.}: If $O_1$ and $O_2$ are spacelike separated then $[\mathcal{A}(O_1), \mathcal{A}(O_2)] = 0$ (graded commutator) within a larger algebra $\mathcal{A}(O)$ where $O_1, O_2\subset O$.
		\end{itemize}
	\end{axiom}
	
	\begin{axiom}[Poincar\'e covariance]
		For all causally closed sets $O$ and Poincar\'e transformations $\Lambda$ there exists an isomorphism $\alpha^O_\Lambda:\mathcal{A}(O)\rightarrow\mathcal{A}(\Lambda O)$ such that the following conditions are satisfied:
		\begin{itemize}
			\item If $O_1\subset O_2$ then $\alpha_\Lambda\circ\iota_{O_1,O_2} = \iota_{\Lambda O_1, \Lambda O_2}\circ\alpha_\Lambda$
			\item The isomorphisms satisfy a composition rule: $\alpha^{\Lambda O}_{\Lambda'}\circ\alpha^O_\Lambda = \alpha^O_{\Lambda'\Lambda}$
		\end{itemize}
	\end{axiom}
	
	\begin{axiom}[Spectrum]
		For all spacetime regions $O$ one can construct a faithful $C^*$-algebra representation $\rho_O$ of $\mathcal{A}(O)$ on a fixed Hilbert space. The different representations should be compatible, i.e. if $O_1\subset O_2$ then the restriction of $\rho_{O_2}$ to $\mathcal{A}(O_1)$ should equal $\rho_{O_1}$. Furthermore all spacetime translations should be implemented unitarily:
		\begin{gather}
			U(a)\rho_O(c)U(a)^{-1} = \rho_{O+a}(\alpha^O_a(c))
		\end{gather}
		for all $c\in\mathcal{A}(O)$, where $U$ is a unitary representation of the translation group. In addition we require that the generators of the translation group have a spectrum that is contained in the future lightcone.
	\end{axiom}
	
	The following axiom is not part of the standard Haag-Kastler framework but can be added to introduce a notion of time evolution:
	\begin{axiom}[Time slice]
		Consider two spacetime regions $O_1, O_2$. If $O_1$ contains a Cauchy surface of $O_2$ then the morphism $\mathcal{A}(O_1\hookrightarrow O_2)$ of $C^*$-algebras is an isomorphism.
	\end{axiom}
	
	\begin{axiom}[Haag duality]\index{Haag!duality}
		Let $\overline{O}$ denote the spacelike complement of $O$ and let $\mathcal{A}'$ denote the commutant of $\mathcal{A}$. Haag duality states that\footnote{Here it should be understood that $\mathcal{A}\left(\overline{O}\right)$ is the algebra generated by all algebras $\mathcal{A}(Q)$ where $Q$ ranges over the causally closed sets in $\overline{O}$.}
		\begin{equation}
			\mathcal{A}\left(\overline{O}\right)' = \mathcal{A}(O)
		\end{equation}
		for all causally closed sets $O$.
	\end{axiom}
	\remark{Haag duality is known to hold for all free theories and for some less trivial theories. However it is also known to fail in the case of symmetry breaking \cite{Roberts_haag}.}
	
	To generalize the above axiom system to globally hyperbolic spacetimes we enter the realm of catgeory theory. We will follow the notation\footnote{This may cause confusion with other notations used in this text.} of \cite{cal_strobl} (??AND OTHERS??): Let $\textbf{Loc}$ be the category of globally hyperbolic spacetimes with orientation- and causal structure-preserving isometries. Let $\textbf{Obs}$ be the category of relevant algebras\footnote{Commutative algebras for classical physics and $C^*$-algebras for quantum theories.} together with suitable algebra morphisms. The assignement of algebras is then given by a functor $\func{\mathfrak{U}}{Loc}{Obs}$. The Haag-Kastler framework is recovered when we restrict $\textbf{Loc}$ to the subcategory of globally hyperbolic subsets of some manifold (with inclusions as morphisms).
	
\subsection{Weyl systems}

	\newdef{Weyl system}{\index{Weyl!system}
		Let $(L, \omega)$ be a symplectic vector space. Let $K$ be a complex vector space and let $W$ be a map from $L$ to the space of unitary operators on $K$. The pair $(K, W)$ is a Weyl system over $(L, \omega)$ if it satisfies:
		\begin{equation}
			W(z)W(z') = e^{\frac{i}{2}\omega(z, z')}W(z+z')
		\end{equation}
		for all $z, z'\in K$.
	}
	\remark{This is a generalization of the canonical commutation relations in their exponentiated (Weyl) form.}
	
	\newdef{Heisenberg system}{\index{Heisenberg!system}
		The generators\footnote{Their existence is guaranteed by Stone's theorem.} $\phi(z)$ of the map $t\mapsto W(tz)$ are said to form a Heisenberg system. These operators satisfy following properties:
		\begin{itemize}
			\item $\lambda\phi(z) = \phi(\lambda z)$ for all positive $\lambda$.
			\item $[\phi(z), \phi(z')] = -i\omega(z, z')$
			\item $\phi(z+z')$ is the closure\footnote{See definition \ref{operator:closure}.} of $\phi(z)+\phi(z')$
		\end{itemize}
	}
	
\section{Topological QFT}
\subsection{Atiyah-Segal axioms}

	\begin{axiom}[Atiyah-Segal]\index{Atiyah-Segal}
		\nomenclature[A_TQFT]{TQFT}{Topological quantum field theory}
		A $d$-dimensional topological quantum field theory (TQFT) is a symmetric monoidal functor $F:\textbf{Bord}_{d-1}^d\rightarrow\textbf{FinVect}$ satisfying following axioms:
		\begin{enumerate}
			\item Normalization: $F(\emptyset)=\mathbb{C}$
			\item Disjoint union: $F(M\sqcup M') = F(M)\otimes F(M')$
			\item Composition: If $N=M\cup M'$ where $\partial M$ and $\partial M$ have opposite orientation then: \[F(N) = F(M)\circ F(M')\]
			\item Invariance: If $f: M\rightarrow M'$ is a diffeomorphism rel boundary then $F\circ f = F$.
		\end{enumerate}
		where $M, M'$ are $d$-dimensional cobordisms between $(d-1)$-dimensional (closed) smooth manifolds.
	\end{axiom}
	
	\begin{example}[1D]
		In 1D one has the following possible smooth cobordisms and their associated TQFT operations:
		\begin{equation*}
			\begin{array}{l|l}
				\text{Point with orientation } + & \text{Vector space } V\\
				\text{Point with reversed orientation } - & \text{Dual space } V^*\\
				\text{Line between points} & \text{Linear map } f:V\rightarrow V\\
				\text{Cap between $\emptyset$ and points } +, - & \text{Coevaluation } \mathbb{C}\rightarrow V\otimes V^*\\
				\text{Cup between points $-, +$ and }\emptyset & \text{Evaluation } V^*\otimes V\rightarrow\mathbb{C}
			\end{array}
		\end{equation*}
		Essentially this gives us the structure of a (finite-dimensional) vector space with dual.
	\end{example}
	
	\begin{example}[2D]
		In 2D one can obtain a similar result by drawing all possible configurations, however the existence and combination of "\textit{pair of pants}"-diagrams gives a richer stucture. For 2D TQFT's the corresponding object is a (finite-dimensional) commutative and cocommutative Frobenius algebra.
	\end{example}
	
	In dimensions 3 and higher the definition above is intractable. To allow the construction to be generalized to higher dimensions one considers the following definition:
	\newdef{Extended TQFT}{
		A $d$-dimensional extended TQFT is a symmetric monoidal functor $F:\textbf{Bord}_1^d\rightarrow\textbf{FinVect}$ satisfying the Atiyah-Segal axioms where the invariance axiom is required only at the highest level of $k$-morphisms.
	}
	
\subsection{Open-closed TQFT}

	For a complete definition we refer to the original paper \cite{open_closed}.
	
	In the case of ''ordinary'' TQFT's one considers cobordisms between closed (smooth) manifolds, hence the relevant objects in this category are (smooth) manifolds with boundary. A generalization is obtained by relaxing the constraint on the cobordisms and allowing the notion of manifolds with corners (see section \ref{section:manifold_boundary}). For simplicity we will only consider the case of 2D TQFT's (as in the original definition).

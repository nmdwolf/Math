\chapter{Dirac equation}

\section{Dirac equation}

	\newdef{Dirac matrices}{\index{gamma matrices|see{Dirac matrices}}\index{Dirac!matrices}\index{Dirac!basis}\index{Weyl!basis}\index{chiral!basis|see{Weyl basis}}
		The time-like Dirac matrix $\gamma^0$ is defined as:
		\begin{equation}
			\gamma^0 = \left(
			\begin{array}{cc}
				\mathbbm{1}&0\\
				0&-\mathbbm{1}
			\end{array}
			\right)
		\end{equation}
		where $\mathbbm{1}$ is the 2-dimensional identity matrix. The space-like Dirac matrices $\gamma^k$, $k= 1, 2, 3$ are defined using the Pauli matrices $\sigma^k$:
		\begin{equation}
			\gamma^k = \left(
			\begin{array}{cc}
				0&\sigma^k\\
				-\sigma^k&0
			\end{array}
			\right)
		\end{equation}
		This form of the Dirac matrices fixes a basis called the \textbf{Dirac basis}. The \textbf{Weyl} or \textbf{chiral} basis is fixed by replacing the time-like matrix $\gamma^0$ by
		\begin{equation}
			\gamma^0 = \left(
			\begin{array}{cc}
				0&\mathbbm{1}\\
				\mathbbm{1}&0
			\end{array}
			\right)
		\end{equation}
	}
	\begin{property}
		The Dirac matrices satisfy
		\begin{equation}
			\gamma^\mu\gamma^\nu + \gamma^\nu\gamma^\mu = 2\eta^{\mu\nu}\mathbbm{1}
		\end{equation}
		This has the form of equation \ref{clifford:inner_product}. The Dirac matrices can thus be used as the generating set of a Clifford algebra (see defiinition \ref{clifford:clifford_algebra}).
	\end{property}
	
	\newnot{Feynman slash notation}{\index{Feynman!slash}
		Let $a = a_\mu x^\mu\in V$ be a general 4-vector. The Feynman slash $\slashed{a}$ is defined as follows:
		\begin{equation}
			\slashed{a} = \gamma^\mu a_\mu
		\end{equation}
		A more formal treatment of the Feynman slash notation shows us that it gives us a canonical map:
		\begin{equation}
			/ : V\rightarrow C\ell_V : a_\mu x^\mu\mapsto a_\mu\gamma^\mu
		\end{equation}
	}

	\newformula{Dirac equation}{\index{Dirac!equation}
		In covariant form the Dirac equation reads:
		\begin{equation}
			\boxed{(i\hbar\slashed\partial - mc)\psi = 0}
		\end{equation}
	}

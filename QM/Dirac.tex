\chapter{Dirac equation}

\section{Dirac matrices}

	\newdef{Dirac matrices}{\index{gamma matrices|see{Dirac matrices}}\index{Dirac!matrices}\index{Dirac!basis}\index{Weyl!basis}\index{chiral!basis|see{Weyl basis}}
		The time-like Dirac matrix $\gamma^0$ is defined as:
		\begin{equation}
			\gamma^0 = \left(
			\begin{array}{cc}
				\mathbbm{1}&0\\
				0&-\mathbbm{1}
			\end{array}
			\right)
		\end{equation}
		where $\mathbbm{1}$ is the 2-dimensional identity matrix. The space-like Dirac matrices $\gamma^k$, $k= 1, 2, 3$ are defined using the Pauli matrices\footnote{See definition \ref{QM:angular_momentum:pauli_matrices}.} $\sigma^k$:
		\begin{equation}
			\gamma^k = \left(
			\begin{array}{cc}
				0&\sigma^k\\
				-\sigma^k&0
			\end{array}
			\right)
		\end{equation}
		This form of the Dirac matrices fixes a basis called the \textbf{Dirac basis}. The \textbf{Weyl} or \textbf{chiral} basis is fixed by replacing the time-like matrix $\gamma^0$ by
		\begin{equation}
			\gamma^0 = \left(
			\begin{array}{cc}
				0&\mathbbm{1}\\
				\mathbbm{1}&0
			\end{array}
			\right)
		\end{equation}
	}
	\begin{property}\index{Dirac!algebra}
		The Dirac matrices satisfy following equality:
		\begin{equation}
			\{\gamma^\mu,\gamma^\nu\}_+ = 2\eta^{\mu\nu}\mathbbm{1}
		\end{equation}
		This has the form of equation \ref{clifford:inner_product}. The Dirac matrices can thus be used as the generating set of a Clifford algebra\footnote{See defiinition \ref{clifford:clifford_algebra}.}, called the \textbf{Dirac algebra}.
	\end{property}
	
	\newnot{Feynman slash notation}{\index{Feynman!slash}
		Let $a = a_\mu x^\mu\in V$ be a general 4-vector. The Feynman slash $\slashed{a}$ is defined as follows:
		\begin{equation}
			\slashed{a} = \gamma^\mu a_\mu
		\end{equation}
		A more formal treatment of the Feynman slash notation shows that it gives us a canonical map:
		\begin{equation}
			/ : V\rightarrow C\ell_V : a_\mu x^\mu\mapsto a_\mu\gamma^\mu
		\end{equation}
	}

\section{Spinors}
\subsection{Dirac equation}

	\newformula{Dirac equation}{\index{Dirac!equation}
		In covariant form the Dirac equation reads:
		\begin{equation}
			\boxed{(i\hbar\slashed\partial - mc)\psi = 0}
		\end{equation}
	}
	
	\newdef{Dirac adjoint}{\index{Dirac!adjoint}
		\begin{equation}
			\overline{\psi} = \psi^\dag\gamma^0
		\end{equation}
	}
	
	\newformula{Parity}{
		\begin{equation}
			\hat{P}(\psi) = \gamma^0\psi
		\end{equation}
	}
	
\subsection{Chiral spinors}

	We can also define a fifth matrix:
	\newdef{Chiral matrix}{\index{chiral!matrix}
		\begin{equation}
			\gamma^5 = i\gamma^0\gamma^1\gamma^2\gamma^3
		\end{equation}
	}

	\newdef{Chiral projection}{
		The chiral projections of a spinor $\psi$ are defined as follows:
		\begin{equation}
			\psi_L = \frac{1-\gamma^5}{2}\psi
		\end{equation}
		and
		\begin{equation}
			\psi_R = \frac{1+\gamma^5}{2}\psi
		\end{equation}
		Every spinor can then be written as:
		\begin{equation}
			\psi = \psi_L + \psi_R
		\end{equation}
	}

\chapter{Dirac Equation}

References for this chapter are \cite{supergravity}. (Note that one uses the mostly-pluses signature there.)  

\section{Dirac matrices}

	\newdef{Dirac matrices\footnotemark}{\index{gamma matrices|see{Dirac matrices}}\index{Dirac!matrices}\index{Dirac!basis}\index{Weyl!basis}\index{chiral!basis|see{Weyl basis}}
		\footnotetext{Often just called the \textbf{gamma matrices}.}
		There exist multiple different representations of the Clifford generators in signature $(1, 3)$. The first one is the \textbf{Dirac representation}. Here the time-like Dirac matrix $\gamma^0$ is defined as
		\begin{equation}
			\gamma^0 = \left(
			\begin{array}{cc}
				\mathbbm{1}_2&0\\
				0&-\mathbbm{1}_2
			\end{array}
			\right)
		\end{equation}
		The space-like Dirac matrices $\gamma^k$, $k= 1, 2, 3$ are defined using the Pauli matrices\footnote{See definition \ref{QM:angular_momentum:pauli_matrices}.} $\sigma^k$:
		\begin{equation}
			\gamma^k = \left(
			\begin{array}{cc}
				0&\sigma^k\\
				-\sigma^k&0
			\end{array}
			\right)
		\end{equation}
		
		The \textbf{Weyl} or \textbf{chiral} representation\footnote{This representation is widely used in advanced field theory and supergravity.} is defined by replacing the time-like matrix $\gamma^0$ by
		\begin{equation}
			\gamma^0 =
			\begin{pmatrix}
				0&\mathbbm{1}_2\\
				\mathbbm{1}_2&0
			\end{pmatrix}
		\end{equation}
		In signature $(3, 1)$ one obtains the Weyl repreesentation by defining $\sigma^\mu\equiv(\mathbbm{1}, \sigma_i)$ and $\overline{\sigma}^\mu \equiv \sigma_\mu$:
		\begin{equation}
			\gamma^\mu =
			\begin{pmatrix}
				0&\sigma_\mu\\
				\overline{\sigma}_\mu&0
			\end{pmatrix}
		\end{equation}
	}
	\remark{In the continuation of this paper we will adopt the Weyl representation.}
	
	\begin{property}\index{Dirac!algebra}
		The Dirac matrices satisfy following equality:
		\begin{equation}
			\label{dirac:clifford_relation}
			\{\gamma^\mu,\gamma^\nu\}_+ = 2\eta^{\mu\nu}\mathbbm{1}
		\end{equation}
		This has the form of equation \ref{clifford:inner_product}. The Dirac matrices can thus be used as the generating set of a Clifford algebra\footnote{See defiinition \ref{clifford:clifford_algebra}.}, called the \textbf{Dirac algebra}.
	\end{property}
	
	\newnot{Feynman slash notation}{\index{Feynman!slash}
		Let $a = a^\mu e_\mu\in V$ be a general 4-vector. The Feynman slash $\slashed{a}$ is defined as follows:
		\begin{equation}
			\slashed{a} = a^\mu\gamma_\mu
		\end{equation}
		In fact this is just a vector space morphism:
		\begin{equation}
			/ : V\rightarrow C\ell(V, \eta) : a^\mu e_\mu\mapsto a^\mu\gamma_\mu
		\end{equation}
	}

\section{Spinors}
\subsection{Dirac equation}

	\newformula{Dirac equation}{\index{Dirac!equation}
		In covariant form the Dirac equation reads:
		\begin{equation}
			\boxed{(i\hbar\slashed\partial - mc)\psi = 0}
		\end{equation}
	}
	
	\newdef{Dirac adjoint}{\index{Dirac!adjoint}
		\begin{equation}
			\overline{\psi} = \psi^\dag\gamma^0
		\end{equation}
		When working in the Weyl representation one should a factor $i$ to this definition.
	}
	\newdef{Majorana adjoint}{
		In the context of SUSY it is often convenient to work with a different adjoint spinor. Let $\mathcal{C}=i\gamma^3\gamma^1$ denote the charge conjugation operator. The Majorana adjoint is then defined by:
		\begin{equation}
			\overline{\psi} = \psi^tC
		\end{equation}
	}
	
	\newformula{Parity}{
		\begin{equation}
			\hat{P}(\psi) = \gamma^0\psi
		\end{equation}
	}
	
\subsection{Chiral spinors}

	In even dimensions one can define an additional matrix \footnote{In $d=4$ this matrix is often denoted by $\gamma_5$.} which also satisfies equation \ref{dirac:clifford_relation}:
	\newdef{Chiral matrix}{\index{chiral!matrix}
		\begin{equation}
			\gamma_\ast = (-i)^{m+1}\gamma_0\gamma_1...\gamma_{d-1}
		\end{equation}
		when working in $d=2m$ dimensions. In odd dimensions ($d=2m+1$) a generating set for the Clifford algebra can be obtained by taking the generating set in even dimension $d-1$ and adjoining the element $\pm\gamma_\ast$. This gives two inequivalent representations of the Clifford algebra (depending on the sign).
		
		Generally one can take the following representation for $\gamma_\ast$:
		\begin{equation}
			\gamma_\ast =
			\begin{pmatrix}
				\mathbbm{1}&0\\
				0&-\mathbbm{1}
			\end{pmatrix}
		\end{equation}
	}

	\newdef{Chiral projection}{
		The chiral projections of a spinor $\psi$ are defined as follows:
		\begin{equation}
			\psi_L = \frac{1+\gamma_\ast}{2}\psi
		\end{equation}
		and
		\begin{equation}
			\psi_R = \frac{1-\gamma_\ast}{2}\psi
		\end{equation}
		Every spinor can then be written as:
		\begin{equation}
			\psi = \psi_L + \psi_R
		\end{equation}
	}
	
\subsection{\texorpdfstring{Dirac algebra in $D=4$}{Dirac algebra in D=4}}

	For a lot of calculations, especially in quantum electrodynamics, one needs the properties of the gamma matrices. Therefor we will list here the most relevant relations in $D=3+1$:
	\begin{formula}[Trace algebra]
		\begin{align}
			\text{tr}(\gamma^\mu) = \text{tr}(\gamma^\mu\gamma^\nu\gamma^\rho) &= 0\\
			\text{tr}(\gamma^\mu\gamma^\nu) &= 4\eta^{\mu\nu}\\
			\text{tr}(\gamma^\mu\gamma^\nu\gamma^\kappa\gamma^\lambda) &= 4(\eta^{\mu\nu}\eta^{\kappa\lambda} - \eta^{\mu\kappa}\eta^{\nu\lambda} + \eta^{\mu\lambda}\eta^{\nu\kappa})\\
			\text{tr}(\gamma^5) = \text{tr}(\gamma^\mu\gamma^5) = \text{tr}(\gamma^\mu\gamma^\nu\gamma^5) = \text{tr}(\gamma^\mu\gamma^\nu\gamma^\rho\gamma^5)&= 0\\
			\text{tr}(\gamma^\mu\gamma^\nu\gamma^\kappa\gamma^\lambda\gamma^5) &= -4i\varepsilon^{\mu\nu\kappa\lambda}\\
			\text{tr}(\gamma^{\mu_1}\cdots\gamma^{\mu_k}) &= \text{tr}(\gamma^{\mu_k}\cdots\gamma^{\mu_1})
		\end{align}
	\end{formula}
	
	\begin{formula}[Contraction identities]
		\begin{align}
			\gamma^\mu\gamma_\mu &= 4\\
			\gamma^\mu\gamma^\nu\gamma_\mu &= -2\gamma^\nu\\
			\gamma^\mu\gamma^\nu\gamma^\rho\gamma_\mu &= 4\eta^{\nu\rho}\\
			\gamma^\mu\gamma^\nu\gamma^\kappa\gamma^\lambda\gamma_\mu &= -2\gamma^\lambda\gamma^\kappa\gamma^\nu\\
		\end{align}
	\end{formula}

\chapter{Mathematical formalism}
\section{Postulates}
	
    \subsection{Postulate 6: eigenfunction expansion}
    \newdef{Observable}{\index{observable}
    	An operator $\op{A}$ which possesses a complete set of eigenfunctions is called an observable.
	}
    
	\begin{formula}\index{eigenfunction!expansion}
    	Let $|\Psi\rangle$ be an arbitrary wavefunction representing the system. Let the set $\{|\psi_n\rangle\}$ be a complete set of eigenfunctions of an observable of the system. The wavefunction $|\Psi\rangle$ can then be expanded as a linear combination of those eigenfunctions:
		\begin{equation}
	        \label{qm_formalism:eigenfunction_expansion}
			\boxed{|\Psi\rangle = \sum_nc_n|\psi_n\rangle + \int c_a|\psi_a\rangle da}
		\end{equation}
        where the summation ranges over the discrete spectrum and the integral over the continuous spectrum.
	\end{formula}
    \begin{formula}[Closure relation]\index{closure}
    	For a complete set of discrete eigenfunctions the closure relation\footnotemark reads:
		\begin{equation}
	        \label{qm_formalism:closure}
				\sum_n|\psi_n\rangle\langle\psi_n| = \mathds{1}
		\end{equation}
        For a complete set of continuous eigenfunctions we have the following counterpart:
        \begin{equation}
	        \label{qm_formalism:closure_continuouos}
				\int|\psi_i\rangle\langle\psi_i|di = \mathds{1}
		\end{equation}
        For a mixed set of eigenfunctions a similar relation is obtained by summing over the discrete eigenfunctions and integrating over the continuous eigenfunctions.
	\end{formula}
    \footnotetext{This relation is also called the 'resolution of the identity'.}
    
    \sremark{To simplify the notation we will almost always use the notation of equation \ref{qm_formalism:closure} but implicitly integrate over possible continuous eigenfunctions.}

\section{Uncertainty relations}
	\newformula{Commutator}{\index{commutator}
    	Let $\op{A}, \op{B}$ be two operators. We define the commutator of $\op{A}$ and $\op{B}$ as follows:
    	\begin{equation}
			\label{qm_formalism:commutator}
            \boxed{\comm{A}{B} = \op{A}\op{B} - \op{B}\op{A}}
		\end{equation}
    }
    \begin{property}
        \begin{equation}
			\label{qm_formalism:commutator_left}
            \left[\op{A}\op{B}, \op{C}\right] = \op{A}\comm{B}{C} + \comm{A}{C}\op{B}
		\end{equation}
	\end{property}
    
    \newformula{Anticommutator}{\index{anticommutator}
    	Let $\op{A}, \op{B}$ be two operators. We define the anticommutator of $\op{A}$ and $\op{B}$ as follows:
    	\begin{equation}
			\label{qm_formalism:anticommutator}
            \boxed{\left\{\op{A},\op{B}\right\}_+ = \op{A}\op{B} + \op{B}\op{A}}
		\end{equation}
    }
    
    \newdef{Compatible observables}{\index{observable!compatibile observables}
    	Let $\op{A}, \op{B}$ be two observables. If there exists a complete set of functions $|\psi_n\rangle$ that are simultaneously eigenfunctions of $\op{A}$ and $\op{B}$, the two operators are called \textbf{compatible}.
	}

	\newformula{Uncertainty relation}{\index{uncertainty!relation}\index{Heisenberg!uncertainty relation}
    	Let $\op{A}, \op{B}$ be two observables. Let $\Delta A, \Delta B$ be the corresponding uncertainties. We have the following relation:
    	\begin{equation}
			\label{qm_formalism:uncertainty_relation}
            \boxed{\Delta A\Delta B = \stylefrac{1}{4}\left|\left\langle\left[\op{A}, \op{B}\right]\right\rangle\right|^2}
		\end{equation}
    }

\section{Matrix representation}
	\begin{formula}
    	The following formula gives the $(m,n)$-th element of the matrix representation of $\op{A}$ with respect to the basis orthonormal $\{\psi_n\}$.
		\begin{equation}
			\label{qm_formalism:matrix_entry}
            \boxed{A_{mn} = \langle\psi_m|\op{A}|\psi_n\rangle}
		\end{equation}
	\end{formula}
    \begin{remark}
		 The basis $\{\psi_n\}$ need not consist out of eigenfunctions of $\op{A}$.
	\end{remark}
    
\section{Slater determinants}

	\begin{theorem}[Symmetrization postulate]\index{symmetrization postulate}
    	A system of $n$ identical particles is described by a wave function $\Psi$ belonging to either $S^n(\mathcal{H})$ or $\Lambda^n(\mathcal{H})$, where $\mathcal{H}$ is the Hilbert space belonging to a single particle.
	\end{theorem}
    \begin{remark}
    	In ordinary quantum mechanics this is a postulate, but in quantum field theory this is a consequence of the spin-statistics theorem of Pauli.
    \end{remark}
    
    \begin{formula}
    	Let $\{\sigma\}$ be the set of all permutations of the sequence $(1, ..., n)$. Let $|\psi\rangle$ be the single-particle wave function. Fermionic systems are described by a wave function of the form
        \begin{equation}
        	|\Psi_F\rangle = \sum_{\sigma}|\psi_{\sigma(1)}\rangle\cdots|\psi_{\sigma(n)}\rangle
        \end{equation}
        Bosonic systems are described by a wave function
        \begin{equation}
        	|\Psi_B\rangle = \sum_{\sigma}\sgn(\sigma)|\psi_{\sigma(1)}\rangle\cdots|\psi_{\sigma(n)}\rangle
        \end{equation}
    \end{formula}

	\newdef{Slater determinant}{\index{Slater determinant}
    	Let $\{\phi_i(\vector{q}_i)\}_{i\leq N}$ be the set of wave functions (spin orbitals) describing a system of $N$ identical particles. The totally antisymmetric wave functions for the complete system is given by
        \begin{equation}
        	\label{qm_formalism:slater_determinant}
            \boxed{\psi(\vector{q}_1, ..., \vector{q}_N) = \frac{1}{\sqrt{N!}}\left|
            \begin{array}{ccc}
            	\phi_1(\vector{q}_1)&\cdots&\phi_N(\vector{q}_1)\\
                \vdots&&\vdots\\
                \phi_1(\vector{q}_N)&\cdots&\phi_N(\vector{q}_N)
            \end{array}
            \right|}
        \end{equation}
    }
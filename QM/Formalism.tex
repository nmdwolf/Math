\chapter{Mathematical formalism}

\section{Postulates}
\subsection{Postulate 6: eigenfunction expansion}

	\newdef{Observable}{\index{observable}
    		An operator $\hat{A}$ which possesses a complete set of eigenfunctions is called an observable.
	}
    
	\begin{formula}\index{eigenfunction!expansion}
    		Let $|\Psi\rangle$ be an arbitrary wavefunction representing the system. Let the set $\{|\psi_n\rangle\}$ be a complete set of eigenfunctions of an observable of the system. The wavefunction $|\Psi\rangle$ can then be expanded as a linear combination of those eigenfunctions:
		\begin{equation}
	        	\label{qm_formalism:eigenfunction_expansion}
			\boxed{|\Psi\rangle = \sum_nc_n|\psi_n\rangle + \int c_a|\psi_a\rangle da}
		\end{equation}
	        where the summation ranges over the discrete spectrum and the integral over the continuous spectrum.
	\end{formula}
   
	\begin{formula}[Closure relation]\index{closure!relation}
	    	For a complete set of discrete eigenfunctions the closure relation\footnotemark\ reads:
		\begin{equation}
		        \label{qm_formalism:closure}
			\sum_n|\psi_n\rangle\langle\psi_n| = \mathbbm{1}
		\end{equation}
	        For a complete set of continuous eigenfunctions we have the following counterpart:
	        \begin{equation}
		        \label{qm_formalism:closure_continuouos}
			\int|i\rangle\langle i|di = \mathbbm{1}
		\end{equation}
	        For a mixed set of eigenfunctions a similar relation is obtained by summing over the discrete eigenfunctions and integrating over the continuous eigenfunctions.
	        \footnotetext{This relation is also called the \textbf{resolution of the identity}.}
	\end{formula}
	\sremark{To simplify the notation we will almost always use the notation of equation \ref{qm_formalism:closure} but implicitly integrate over the continuous spectrum.}

\section{Uncertainty relations}
	
	\newdef{Commutator}{\index{commutator}
    		Let $\hat{A}, \hat{B}$ be two operators. We define the commutator of $\hat{A}$ and $\hat{B}$ as follows:
    		\begin{equation}
			\label{qm_formalism:commutator}
        		\boxed{\comm{A}{B} = \hat{A}\hat{B} - \hat{B}\hat{A}}
		\end{equation}
	}
	\begin{formula}
	        \begin{equation}
			\label{qm_formalism:commutator_left}
		        \left[\hat{A}\hat{B}, \hat{C}\right] = \hat{A}\comm{B}{C} + \comm{A}{C}\hat{B}
		\end{equation}
	\end{formula}
    
	\newdef{Anticommutator}{\index{anticommutator}
	    	Let $\hat{A}, \hat{B}$ be two operators. We define the anticommutator of $\hat{A}$ and $\hat{B}$ as follows:
	    	\begin{equation}
			\label{qm_formalism:anticommutator}
		        \boxed{\left\{\hat{A},\hat{B}\right\}_+ = \hat{A}\hat{B} + \hat{B}\hat{A}}
		\end{equation}
	}
    
	\newdef{Compatible observables}{\index{observable!compatibile observables}
	    	Let $\hat{A}, \hat{B}$ be two observables. If there exists a complete set of functions $|\psi_n\rangle$ that are eigenfunctions of both $\hat{A}$ and $\hat{B}$ then the two operators are said to be compatible.
	}

	\newformula{Heisenberg uncertainty relation}{\index{Heisenberg!uncertainty relation}
	    	Let $\hat{A}, \hat{B}$ be two observables. Let $\Delta A, \Delta B$ be the corresponding uncertainties.
	    	\begin{equation}
			\label{qm_formalism:uncertainty_relation}
		        \boxed{\Delta A\Delta B = \stylefrac{1}{4}\left|\left\langle\left[\hat{A}, \hat{B}\right]\right\rangle\right|^2}
		\end{equation}
	}

\section{Matrix representation}

	\begin{formula}
	    	The following formula gives the $(m,n)$-th element of the matrix representation of $\hat{A}$ with respect to the orthonormal basis $\{\psi_n\}$:
		\begin{equation}
			\label{qm_formalism:matrix_entry}
		        \boxed{A_{mn} = \langle\psi_m|\hat{A}|\psi_n\rangle}
		\end{equation}
	\end{formula}
	\begin{remark}
		 The basis $\{\psi_n\}$ need not consist out of eigenfunctions of $\hat{A}$.
	\end{remark}
    
\section{Slater determinants}

	\begin{theorem}[Symmetrization postulate]\index{symmetrization postulate}
	    	Let $\mathcal{H}$ be the Hilbert space belonging to a single particle. A system of $n$ identical particles is described by a wave function $\Psi$ belonging to either $S^n(\mathcal{H})$ or $\Lambda^n(\mathcal{H})$.
	\end{theorem}
	\begin{remark}
	    	In ordinary quantum mechanics this is a postulate, but in quantum field theory this is a consequence of the spin-statistics theorem of Pauli.
	\end{remark}
    
	\begin{formula}
	    	Let $\{\sigma\}$ be the set of all permutations of the sequence $(1, ..., n)$. Let $|\psi\rangle$ be the single-particle wave function. Fermionic systems are described by a wave function of the form
	        \begin{equation}
	        	|\Psi_F\rangle = \sum_{\sigma}|\psi_{\sigma(1)}\rangle\cdots|\psi_{\sigma(n)}\rangle
	        \end{equation}
	        Bosonic systems are described by a wave function of the form
	        \begin{equation}
	        	|\Psi_B\rangle = \sum_{\sigma}\sgn(\sigma)|\psi_{\sigma(1)}\rangle\cdots|\psi_{\sigma(n)}\rangle
	        \end{equation}
	\end{formula}

	\newdef{Slater determinant}{\index{Slater determinant}
	    	Let $\{\phi_i(\vector{q})\}_{i\leq N}$ be a set of wave functions (spin orbitals) describing a system of $N$ identical particles. The (totally antisymmetric) wave function of the system is given by:
	        \begin{equation}
	        	\label{qm_formalism:slater_determinant}
		        \boxed{\psi(\vector{q}_1, ..., \vector{q}_N) = \frac{1}{\sqrt{N!}}\det\left(
		        \begin{array}{ccc}
            			\phi_1(\vector{q}_1)&\cdots&\phi_N(\vector{q}_1)\\
                		\vdots&&\vdots\\
                		\phi_1(\vector{q}_N)&\cdots&\phi_N(\vector{q}_N)
		        \end{array}
		        \right)}
     		\end{equation}
	}
	
\section{Interaction picture}\index{interaction picture}\label{qm:interaction_pciture}

	Let $\hat{H} = \hat{H}_0 + \hat{V}(t)$ be the total Hamiltonian of a system where $\hat{V}(t)$ is the interaction Hamiltonian. Let $|\psi(t)\rangle$ and $\hat{O}$ be the state vector and operator in the Schr\"odinger picture.
	
	\begin{formula}
		In the interaction picture we define the state vector as follows:
		\begin{equation}
			|\psi(t)\rangle_I = e^{\frac{i}{\hbar}\hat{H}_0t}|\psi(t)\rangle
		\end{equation}
		From this it follows that the operators in the interaction picture are given by:
		\begin{equation}
			\hat{O}_I(t) = e^{\frac{i}{\hbar}\hat{H}_0t}\hat{O}e^{-\frac{i}{\hbar}\hat{H}_0t}
		\end{equation}
	\end{formula}
	\newformula{Schr\"odinger equation}{
		Using the previous definition the Schr\"odinger equation can be rewritten as follows:
		\begin{equation}
			i\hbar\deriv{}{t}|\psi(t)\rangle_I = \hat{V}_I(t)|\psi(t)\rangle_I
		\end{equation}
		The time-evolution of operators is given by:
		\begin{equation}
			\deriv{}{t}\hat{O}_I(t) = \frac{i}{\hbar}\left[\hat{H}_0, \hat{O}_I(t)\right]
		\end{equation}
	}

\chapter{Perturbation Theory}

\section{Rayleigh-Schr\"odinger theory}
\index{Schr\"odinger!Rayleigh-Schr\"odinger perturbation}

	The basic of assumptions of the Rayleigh-Schr\"odinger perturbation theory are that the perturbation Hamiltonian is time-independent and that the eigenfunctions of the unperturbed Hamiltonian $\hat{H}_0$ also form a complete set for the perturbed Hamiltonian.

	\begin{formula}
	    	The perturbed eigenfunctions and eigenvalues can be expanded in the following way, where we assume that $\lambda$ is a small perturbation parameter:
	        \begin{equation}
	        	|\psi_n\rangle = \sum_{i = 0}^{+\infty} \lambda^i |\psi_n^{(i)}\rangle
	        \end{equation}
	        \begin{equation}
	        	E_n = \sum_{i = 0}^{+\infty} \lambda^i E_n^{(i)}
	        \end{equation}
	        where $i$ denotes the order of the perturbation.
	\end{formula}

\section{Time-dependent perturbation theory}

	In this section we consider perturbed Hamiltoninians of the following form:
	\begin{equation}
		\hat{H}(t) = \hat{H}_0 + \lambda \hat{V}(t)
	\end{equation}

\subsection{Dyson series}

	\newformula{Tomonaga-Schwinger equation}{\index{Schwinger!Tomonaga-Schwinger equation}
		The evolution operator $\hat{U}(t)$ satisfies the following Schr\"odinger-type equation in the interaction picture\footnote{See section \ref{qm:interaction_picture}.}:
        	\begin{equation}
        		\label{QM:perturbation:tomonaga_schwinger_equation}
        		i\hbar\deriv{}{t}\hat{U}_I|\psi(0)\rangle_I = \hat{V}_I(t)\hat{U}_I|\psi(0)\rangle_I
        	\end{equation}
	}

	\newformula{Dyson series}{\index{Dyson!series}
	        Together with the initial value condition $\hat{U}_I(0) = \mathbbm{1}$ the Tomonaga-Schwinger equation becomes an initial value problem. A particular solution is given by:
	        \begin{equation}
	        	\hat{U}_I(t) = \mathbbm{1} - \stylefrac{i}{\hbar}\int_0^t\hat{V}_I(t')\hat{U}_I(t')dt'
	        \end{equation}
	        This solution can be iterated to obtain a series expansion of the evolution operator:
	        \begin{equation}
	        	\hat{U}(t) = 1 - \stylefrac{i}{\hbar}\int_0^t\hat{V}(t_1)dt_1 + \left(-\stylefrac{i}{\hbar}\right)^2\int_0^tdt_1\int_0^{t_1}dt_2\hat{V}(t_1)\hat{V}(t_2) + ...
	        \end{equation}
	        It is clear that the integrands obey a time-ordering. By introducing the \textbf{time-ordering operator} $\mathcal{T}$:
	        \begin{equation}
	        	\label{QM:time_ordering_operator}
		        \mathcal{T}\left(\hat{V}(t_1)\hat{V}(t_2)\right) = \left\{
	        	\begin{array}{ccc}
				\hat{V}(t_1)\hat{V}(t_2)&,&t_1 \geq t_2\\
                		\hat{V}(t_2)\hat{V}(t_1)&,&t_2 > t_1
		        \end{array}
        		\right.
	        \end{equation}
        	the integrals can be rewritten in a more symmetric form:
        	\begin{equation}\index{Dyson!series}
        		\hat{U}(t) = 1 - \stylefrac{i}{\hbar}\int_0^t\hat{V}(t_1)dt_1 + \frac{1}{2!}\left(-\stylefrac{i}{\hbar}\right)\int_0^tdt_1\int_0^{\textcolor{red}{t}}dt_2\mathcal{T}\left(\hat{V}(t_1)\hat{V}(t_2)\right) + ...
        	\end{equation}
        	or by comparing with the series expansion for exponential functions:
        	\begin{equation}
        		\label{QM:dyson_series}
        		\boxed{\hat{U}(t) = \mathcal{T}\left(e^{-\frac{i}{\hbar}\int_0^t\hat{V}(t')dt'}\right)}
        	\end{equation}
        	This expansion is called the \textbf{Dyson series}.
	}

\section{Variational method}\index{variational method}

	\newdef{Energy functional}{\index{energy}
	    	\begin{equation}
	    		\label{QM:perturbation:energy_functional}
		        E(\psi) = \stylefrac{\langle\psi|\hat{H}|\psi\rangle}{\langle\psi|\psi\rangle}
	    	\end{equation}
	}

	\begin{property}
    		The energy functional \ref{QM:perturbation:energy_functional} satisfies following inequality:
        	\begin{equation}
        		E(\psi) \geq E_0
        	\end{equation}
        	where $E_0$ is the ground state energy.
	\end{property}
    
	\begin{method}
	    	Assume that the trial function $|\psi\rangle$ depends on a set of parameters $\{c_i\}_{i\in I}$. The 'optimal' wave function (the one extremizing the energy functional) is found by solving the following system of equations:
	        \begin{equation}
	        	\pderiv{\psi}{c_i} = 0\qquad\qquad\forall i\in I
	        \end{equation}
	\end{method}
    
\section{Adiabatic approximation}
\subsection{Berry phase}\index{Berry!phase}

    	Consider a system for which the adiabatic approximation is valid. We then have a wavefunction of the form
        \begin{equation}
        	\psi(t) = C_a(t)\psi_a(t)\exp\left[-\frac{i}{\hbar}\int_{t_0}^tE_a(t')dt'\right]
        \end{equation}
    	It follows from the orthonormality of the eigenstates $\psi_k(t)$ that the coefficient $C_a(t)$ is just a phase factor, so we can write it as
        \begin{equation}
        	C_a(t) = e^{i\gamma_a(t)}
        \end{equation}
        Substituting this ansatz in the wavefunction and the Sch\"odinger equation gives a differential equation for the phase factor $\gamma_a(t)$. Integrating it gives:
        \begin{equation}
        	\label{QM:perturbation:berry:phase_factor}
        	\gamma_a(t) = i\int_{t_0}^t\left\langle\psi_a(t')\left|\pderiv{\psi_a(t')}{t'}\right\rangle\right. dt'
        \end{equation}
        Due to time evolution the wavefunction accumulates a phase through the coefficient $C_a(t)$ over the period $t_0-t_f$. This phase is called the \textbf{Berry phase}.
        
        Lets try to apply a phase transformation to remove the Berry phase:
        \begin{equation}
        	\label{QM:perturbation:berry:phase_transform}
        	\psi'_a(t) = \psi_a(t)e^{i\eta(t)}
        \end{equation}
        Entering this in equation \ref{QM:perturbation:berry:phase_factor} gives
        \begin{equation}
        	\bar\gamma'_a(t) = \bar\gamma_a(t) - \eta(t_f) + \eta(t_0)
        \end{equation}
        where the overhead bar denotes the integration between $t_0$ and $t_f$ in equation \ref{QM:perturbation:berry:phase_factor}. If the system is cyclic then $\psi_a(t_0) = \psi_a(t_f)$. Combining this with equation \ref{QM:perturbation:berry:phase_transform} gives us:
        \begin{equation}
        	\eta(t_f) - \eta(t_0) = 2k\pi\qquad\qquad k\in\mathbb{N}
        \end{equation}
        which implies that the Berry phase cannot be eliminated through a basis transformation and is thus an observable property of the system.
        
        \newdef{Berry connection}{\index{Berry!connection}
        	The quantity
        	\begin{equation}
        	    	\mathbf{A}(\vector{x}) = i\langle\psi_a(\vector{x})|\nabla_{\vector{x}}\psi_a(\vector{x})\rangle
        	\end{equation}
	        where $\nabla_{\vector{x}}$ denotes the gradient in phase space, is called the Berry connection (or Berry gauge potential). Applying Stokes' theorem to \ref{QM:perturbation:berry:phase_factor} gives us:
        	\begin{equation}
            		\bar\gamma_a = \int\boldsymbol{\mathcal{B}}\cdot d\vector{S}
	        \end{equation}
        	where $\boldsymbol{\mathcal{B}} = \nabla_{\vector{x}}\times\mathbf{A}(\vector{x})$ is called the \textbf{Berry curvature}. Although the Berry connection is gauge dependent, the Berry curvature is gauge invariant!
        }
        \remark{Using the language of differential geometry one immediately finds that the accumulated phase $\bar\gamma_a$ is simply the holonomy associated with the Berry connection along the considered trajectory.}

\chapter{Supersymmetry}

\section{Extensions of the Standard Model}

	\begin{theorem}[Coleman-Mandula]
		Consider a quantum field theory with the following constraints:
		\begin{enumerate}
			\item There exists a mass gap.
			\item For every mass $M$ there exist only finitely many particle species with mass $\leq M$.
			\item The two-point scattering amplitudes are nonvanishing at almost every energy.
			\item The (two-point) scattering amplitudes are analytic in the particle momenta.
		\end{enumerate}
		If the symmetry group\footnote{Here this means the symmetry group of the $S$-matrix.} contains a subgroup isomorphic to the Poincar\'e group\footnote{Or to be precise, its universal cover.} then it can be written as the direct product of the Poincar\'e group and an internal gauge group.
	\end{theorem}
	\sremark{In other words, it is impossible to combine the Poincar\'e group in a nontrivial way with the internal symmetry group.}
	
	Now the question arises if one can do better. That is, is there a nontrivial way to extend the symmetry group. A first possibility is given by conformal field theories. Here there exists no $S$-matrix and hence the above theorem is clearly not valid. However a second and more intricate possibility is given by supersymmetry. Here one does not work with an ordinary symmetry Lie algebra\footnote{See the original paper \cite{coleman_mandula} for why the algebra plays an essential role.} but with a Lie superalgebra. By allowing superstructures, hence fermionic symmetry generators, one can generalize the Coleman-Mandula theorem. The resulting theorem was proven by \textit{Sohnius, Lopusza\'nski and Haag}.

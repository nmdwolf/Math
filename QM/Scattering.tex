\chapter{Scattering theory}

\section{Cross section}

	\newformula{Differential cross section}{\index{cross section}
    		\begin{equation}
    			\label{QM:scattering:cross_section}
		        \deriv{\sigma}{\Omega} = \frac{N(\theta, \varphi)}{F}
	    	\end{equation}
        	where is the incoming particle flux and $N$ the detected flow rate\footnote{As $N$ is not defined as a rate per unit area (flux), the differential cross section has the dimension of area.}.
	}

\subsection{Fermi's golden rule}

	\newformula{Fermi's golden rule}{\index{Fermi!golden rule}
		The transition propability from state $i$ to state $f$ is given by:
		\begin{equation}
			\label{QM:scattering:fermi_golden_rule}
			\boxed{\Gamma_{i\rightarrow f} = \frac{2\pi}{\hbar}|\langle f|\hat{H}|i\rangle|^2\deriv{n}{E_f}}
		\end{equation}
	}


\section{Lippman-Schwinger equations}

	In this section we consider Hamiltonians of the following form: $\hat{H} = \hat{H}_0 + \hat{V}$ where $\hat{H}_0$ is the free Hamiltonian and $\hat{V}$ the scattering potential. We will also assume that both the total Hamiltonian and the free Hamiltonian have the same eigenvalues.

	\newformula{Lippman-Schwinger equation}{\index{Schwinger!Lippman-Schwinger equation}
    		\begin{equation}
    			|\psi^{(\pm)}\rangle = |\varphi\rangle + \stylefrac{1}{E - \hat{H}_0 \pm i\varepsilon}\hat{V}|\psi^{(\pm)}\rangle
    		\end{equation}
        	where $|\varphi\rangle$ is an eigenstate of the free Hamiltonian with the same energy as $|\psi\rangle$.
        }
	\begin{remark}
		The term $\pm i\varepsilon$ is added to the denominator as otherwise it would be singular. It has no real physical meaning.
	\end{remark}
    
	\newformula{Born series}{\index{Born!series}
    		If we rewrite the Lippman-Schwinger equation as$|\psi\rangle = |\varphi\rangle + \hat{G}_0\hat{V}|\psi\rangle$, were $\hat{G}_0$ is the Green's operator, then we can derive the following series expansion by iterating the equation:
        	\begin{equation}
        		\label{QM:cattering:born_series}
        	    	|\psi\rangle = |\varphi\rangle + \hat{G}_0\hat{V}|\varphi\rangle + \left(\hat{G}_0\hat{V}\right)^2|\varphi\rangle + ...
        	\end{equation}
	}
	\newformula{Born approximation}{\index{Born!approximation}
	    	If we cut off the Born series at the first order term in $\hat{V}$ then we obtain the Born approximation:
	        \begin{equation}
	        	\label{QM:scattering:born_approximation}
		        |\psi\rangle = |\varphi\rangle + \hat{G}_0\hat{V}|\varphi\rangle
	        \end{equation}
	}

\chapter{Yang-Mills Theory}

\section{Gauge invariance}

	Using the tools of differential geometry, as presented in chapters \ref{diff:chapter:bundles} and onward, we can introduce the general formulation of Yang-Mills gauge theories. Used references are \cite{principal_bundles}, \cite{sen_nash} and \cite{schuller}.

	Consider a general gauge group (Lie group) $SU(n)$, acting on a Hilbert bundle $\mathcal{H}$ of physical states over a base space (manifold) $M$. A general gauge transformation has the form
	\begin{equation}
		\label{qft:gauge_transformation}
		\psi'(x) = U(x)\psi(x)
	\end{equation}
	where $\psi, \psi':M\rightarrow\mathcal{H}$ are sections of the physical Hilbert bundle and $U:M\rightarrow G$ encodes the local behaviour of the gauge transformation.
	
	\begin{theorem}[Local gauge principle]
		The Lagrangian $\mathcal{L}[\psi]$, where $\psi(x)\in\mathcal{H}_x$, is invariant under the action of the gauge group $G$:
		\begin{equation}
			\mathcal{L}[U\psi] = \mathcal{L}[\psi]
		\end{equation}
	\end{theorem}
	
	Generally this gauge invariance can be achieved in the following way. Denote the Lie algebra corresponding to $G$ by $\mathfrak{g}$. Because the gauge transformation is local, the information on how it changes should be able to propagate through space. This is done by introducing a new field $B_\mu(x)$, called the \textbf{gauge field}. The most elegant formulation uses the following concept:
	\newdef{Covariant derivative}{\index{covariant!derivative}
		\begin{equation}
			\mathcal{D}_\mu = \partial_\mu + igB_\mu(x)
		\end{equation}
		where $B_\mu:M\rightarrow\mathfrak{g}$. Here we should also note that the explicit action of the covariant derivative depends on the chosen representation of $\mathfrak{g}$ on $\mathcal{H}$
	}
	
	So to achieve gauge invariance one should replace all derivatives by the covariant derivative. Now one could wonder if the covariant derivative itself satisfies the local gauge principle, i.e. $\mathcal{D}'\psi' = U\mathcal{D}\psi$. Lets write this out (from here on we will supress the coordinate dependence):
	\begin{align}
		U^{-1}\left(\pderiv{}{x^\mu} + igB_\mu'\right)\psi' &= U^{-1}\left(\pderiv{}{x^\mu} + igB_\mu'\right)U\psi\nonumber\\
		&= U^{-1}\pderiv{U}{x^\mu}\psi + \pderiv{\psi}{x^\mu} + igU^{-1}B_\mu'U\psi
	\end{align}
	This expression can only be equal to $\mathcal{D}\psi$ if
	\begin{equation}
		igB_\mu = U^{-1}\pderiv{U}{x^\mu} + igU^{-1}B_\mu'U
	\end{equation}
	which can be rewritten as
	\begin{equation}
		B_\mu' = UB_\mu U^{-1} - \frac{1}{ig}(\partial_\mu U)U^{-1}
	\end{equation}
	or in coordinate-independent form\footnote{See also equations \ref{diff:prin:local_compatibility} and \ref{diff:prin:mc_pullback}.}:
	\begin{equation}
		\boxed{\mathbf{B}' = U\mathbf{B}U^{-1} - \frac{1}{ig}dUU^{-1}}
	\end{equation}
	
	\begin{example}[QED]
		For quantum electrodynamics, which has U$(1)\cong S^1$ as its gauge group, we use the parametrization $U(x) = e^{ie\chi(x)}$ where $\chi:\mathbb{R}^n\rightarrow\mathbb{R}$. The relevant formulae then become:
		\begin{align}
			\partial_\mu &\longrightarrow \mathcal{D}_\mu = \partial_\mu + ieA_\mu\\
			A_\mu &\longrightarrow A_\mu' = A_\mu - \partial_\mu\chi
		\end{align}
		where $A_\mu$ is the classic electromagnetic potential.
	\end{example}
	
\section{Spontaneous symmetry breaking}

	\begin{theorem}[Goldstone]\index{Goldstone}
		Consider a QFT with Lie group $G$. Denote the generators of the corresponding Lie algebra by $\mathbf{X}_a$. Generators that do not destroy the vacuum\footnotemark, i.e. $\mathbf{X}_av\neq0$, correspond to massless scalar (Goldstone) bosons.
		\footnotetext{This corresponds to a transformation that leaves the vacuum invariant.}
	\end{theorem}

\chapter{Algebraic Geometry}

\section{Polynomials and Galois theory}
\subsection{Polynomials}\index{polynomial}

	\newdef{Polynomial ring}{
		Let $R$ be a (commutative) ring. The polynomial ring on the indeterminates $X=\{x_i\}_{i\in I}$ is the free commutative $R$-algebra on $X$.
	}
	
    	\newdef{Degree}{\index{degree}
        	The exponent of the highest order power in $x$. It is often denoted by $\deg(f)$.
		}
    	\newdef{Monic polynomial}{A polynomial for which the highest order term has coefficient 1.}

		\begin{theorem}[Fundamental theorem of algebra]\index{Fundamental theorem!of algebra}
        	\label{linalgebra:fundamental_theorem_of_algebra}
			Let $f(x) \in K[x]$ with $\deg(f) \geq 1$. Then $f(x)$ has at least $1$ root in $\mathbb{C}$.
		\end{theorem}
        \begin{result}
		If $f(x) \in \mathbb{C}[x]$ is a monic polynomial with $\deg(f)\geq1$, we can write:
		\begin{equation*}
			f(x) = \prod_{i=1}^k(x-a_i)^{n_i}
		\end{equation*}
		Where $a_1, ..., a_k\in\mathbb{C}$ and $n_1, ..., n_k\in\mathbb{N}$.
	\end{result}

	\newdef{Transcendental element}{\index{transcendental}\index{algebraic}
		Consider a base field $k$ and a field extension $L/k$. An element $x\in L$ for which there exists no non-trivial polynomial $p$ over $k$ such that $p(x) = 0$ is said to be transcendental. Otherwise it is said to be \textbf{algebraic}.
	}

	\newdef{Algebraic independence}{
		Consider a base field $k$ and a field extension $L/k$. A subset $S\subseteq L$ is said to be algebraically independent over $k$ if the elements of $S$ do not satisfy any non-trivial polynomial over $k$.
	}
	
\section{Schemes}
\subsection{Spectrum of a ring}

	\newdef{Spectrum}{\index{spectrum}\index{Zariski!topology}
		\nomenclature[S_Spec]{Spec$(R)$}{Spectrum of a commutative ring $R$.}
		Let $R$ be a commutative ring. The spectrum Spec$(R)$ is defined as the set of prime ideals of $R$. This set can be turned into a topological space by equipping it with the \textbf{Zariski topology}: Let $V_I$ be the set of prime ideals containing the ideal $I$. The collection of closed sets, inducing the Zariski topology, is given by $\{V_I\}_{I\text{ ideal of }R}$.
	}
	\remark{A basis for the above topology is given by the sets $D_f = \{I_p\not\ni f:f\in R, I_p \text{ is a prime ideal}\}$.}
	
	\begin{property}
		Spec$(R)$ is a compact $T_0$ space.
	\end{property}
	
	\newdef{Structure sheaf}{\index{structure!sheaf}
		Given a spectrum $X=$ Spec$(R)$, equipped with its Zariski topology, we can define a sheaf\footnotemark\ $\mathcal{O}_X$ by setting $\forall f\in R: \Gamma(D_f, \mathcal{O}_X) = R_f^*$, where $R_f^*$ is the localization of $R$ with respect to the monoid of powers of $f$.
		\footnotetext{In fact this is merely a \textit{B-sheaf} as it is only defined on the basis of the topology. However, every B-sheaf can be extended to a sheaf by taking the appropriate limits.}
	}
	
	\begin{property}
		The spectrum Spec$(R)$ together with its structure sheaf forms a ringed space.
	\end{property}

\subsection{Affine schemes}

	\newdef{Affine scheme}{\index{scheme}
		A ringed space, isomorphic to the spectrum Spec$(R)$ for some commutative ring $R$, is called an affine scheme.
	}
	
\subsection{Zariski tangent space}

	\begin{definition}[Zariski tangent space]\index{Zariski!tangent space}
		Consider a variety $X$ with structure sheaf $\mathcal{O}_X$. At every point $x\in X$ the ring $\mathcal{O}_{X, x}$ is a local ring and hence we obtain a maximal ideal $\mathfrak{m}_x$. The quotient $\mathfrak{m}_x/\mathfrak{m}_x^2$ is a vector space over the residue field $\mathcal{O}_{X, x}/\mathfrak{m}_x$. It is called the Zariski cotangent space at $x\in X$. Its algebraic dual is called the Zariski tangent space at $x\in X$.
	\end{definition}

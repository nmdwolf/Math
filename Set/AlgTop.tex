\chapter{Algebraic Topology}

References for this chapter are \cite{massey, merry_io}.

\section{Homotopy theory}
\subsection{Homotopy}

	\newdef{Retraction}{\index{retraction}
		Let $X$ be a topological space and let $A\subseteq X$ be a subspace. A continuous function $f:X\rightarrow A$ is called a retraction (and $A$ is called a \textbf{retract} of $X$) if it satisfies $f(a)=a$ for all $a\in A$.  
	}
	\newdef{Homotopy}{\index{homotopy}
		Let $f, g\in C(X,Y)$ where $X, Y$ are topological spaces. If there exists a continuous function $H:X\times [0, 1]\rightarrow Y$ such that $f(x) = H(x, 0)$ and $g(x) = H(x, 1)$ then $f$ and $g$ are said to be homotopic. This relation induces an equivalence relation on $C(X, Y)$.
	}
	\newdef{Deformation retraction}{
		Let $X$ be a topological space and let $A\subseteq X$ be a subspace. $A$ is called a deformation retract if there exists a homotopy between the identity map on $X$ and a retraction $f:X\rightarrow A$.
	}
	
	\newdef{Homotopy type}{\index{homotopy!equivalence}
		Let $X, Y$ be two topological spaces. $X$ and $Y$ are said to be homotopy equivalent, or of the same homotopy type, if there exist continuous functions $f:X\rightarrow Y$ and $g:Y\rightarrow X$ such that $f\circ g$ is homotopic to $\mathbbm{1}_Y$ and $g\circ f$ is homotopic to $\mathbbm{1}_X$. The maps $f, g$ are called \textbf{homotopy equivalences}.
	}
	\begin{property}\index{homeomorphism}
		Every homeomorphism is a homotopy equivalence.
	\end{property}
	
	\newdef{Null-homotopic}{
		A continuous function is null-homotopic if it is homotopic to a constant function.
	}
	\newdef{Contractible space}{\index{contractible}
		\label{topology:contractible_space}
		A topological space $X$ is said to be contractible if the identity map $\mathbbm{1}_X$ is null-homotopic. Equivalently, a space is called contractible if it is homotopy-equivalent to a point.
	}
	
\subsection{Homotopy groups}

	In this subsection we will always assume to be working with pointed spaces \ref{topology:pointed_space}. The base point will be denoted by $\ast$.
	
	\newdef{Loop space}{\index{loop}
		\nomenclature[S_zsymOmega]{$\Omega X$}{(Based) loop space on $X$.}
		\nomenclature[S_LX]{$LX$}{Free loop space on $X$.}
		The set of all \textbf{loops} in a pointed topological space $(X, \ast)$, i.e. all continuous maps\footnote{The mapping space is equipped with the compact-open topology.} $\delta:(S^1, t_0)\rightarrow (X, \ast)$ for which $\delta(t_0) = \ast$. This space is denoted by $\Omega X$. It can be equipped with a multiplication operation corresponding to the concatenation of loops\footnote{It should be noted that the speed at which the concatenated loops are traversed is doubled because the parameter $t$ should remain an element of $S^1\cong[0, 1]/_{0\sim1}$.}.
		
		When one drops the requirement of based loops, i.e. one considers the space of all continuous maps $S^1\rightarrow X$, the resulting space is called the \textbf{free loop space} on $X$. This object is denoted by $LX$.
	}
	
	\newdef{Fundamental group}{\index{group!fundamental}
		The fundamental group $\pi_1(X, x_0)$ is defined as the loop space of $(X, x_0)$ modulo homotopy. As the name implies the fundamental group can be given the structure of a multiplicative group where the operation is inherited from that of the loop space.
	}
	\begin{remark}
		In general, as the notation implies, the fundamental group depends on the base point $x_0$. However when the space $X$ is path-connected, the fundamental groups belonging to different base points are isomorphic. It follows that we can speak of "the" fundamental group in the case of path-connected spaces.
	\end{remark}
	
	\newdef{Fundamental groupoid}{\index{groupoid!fundamental}\index{Poincar\'e!groupoid}
		Let $X$ be a topological space. The fundamental groupoid\footnote{Sometimes called the \textbf{Poincar\'e groupoid}.} $\Pi_1(X)$ is the groupoid which has the points of $X$ as objects and the endpoint-preserving homotopy classes of continuous functions $f:S^1\rightarrow X$ as morphisms. The fundamental group $\pi_1(X, a)$ can be recovered as the automorphism group of $a\in\text{ob}(\Pi_1(X))$.
	}
	
	\begin{property}[Universal cover]\index{deck transformation}
		Consider a topological space $X$ and let $\widetilde{X}$ be its universal covering space. The group of deck transformations\footnote{See definition \ref{topology:deck_transformation}.} Aut$(\widetilde{X})$ is isomorphic to the fundamental group $\pi_1(X)$. Hence we obtain:
		\begin{gather}
			X\cong \widetilde{X}/\pi_1(X)
		\end{gather}
	\end{property}
	
	\newdef{Simply-connected space}{\index{simply-connected}\label{topology:simply_connected}
		A topological space is said to be simply-connected if it is path-connected and if the fundamental group is trivial.
	}
	
	The definition of a fundamental group can be generalized to arbitrary dimensions in the following way\footnote{Note however that we replace the interval $[0, 1]$ by the sphere $S^1$. This is non-restrictive as we can construct $S^n$ by identifying the boundary of $[0,1]^n$ with the basepoint $x_0$.}:
	\newdef{Homotopy group}{\index{homotopy!group}
		\nomenclature[S_zsymhos]{$\pi_n(X, x_0)$}{$n^{th}$ homotopy space over $X$ with basepoint $x_0$.}
		The homotopy group $\pi_n(X, x_0)$ is defined as the set of homotopy classes of continuous maps $f:S^n\rightarrow X$ based at $x_0\in X$. The set $\pi_0(X, x_0)$ is defined as the set of path-connected components of $X$.
	}
	
	\begin{property}
		For $n\geq1$ the sets $\pi_n(X, x_0)$ are groups.
	\end{property}
	\begin{property}\label{topology:abelian_homotopy_groups}
		For $n\geq2$  the homotopy groups $\pi_n(X, x_0)$ are Abelian.
	\end{property}

	\begin{property}
		If $X$ is path-connected, then the homotopy groups $\pi_n(X, x_0)$ and $\pi_n(X, x_1)$ are isomorphic for all $x_0, x_1\in X$ and all $n\in\mathbb{N}$.
	\end{property}
	\begin{property}
		Homeomorphic spaces have isomorphic homotopy groups $\pi_n$.
	\end{property}
	
	\begin{formula}
		Let $(X, x_0)$ and $(Y, y_0)$ be pointed topological spaces with homotopy groups $\pi_n(X, x_0)$ and $\pi_n(Y, y_0)$. The homotopy groups of their product is given by:
		\begin{gather}
			\pi_n(X\times Y, (x_0, y_0)) = \pi_n(X, x_0)\otimes\pi_n(Y, y_0)
		\end{gather}
		where $\otimes$ denotes the direct product of groups \ref{group:direct_product}.
	\end{formula}
	
	\newdef{$n$-connected space}{\index{connected}
		A topological space is said to be $n$-connected if its first $n$ homotopy groups are trivial.
	}

	\newdef{Weak homotopy equivalence}{\index{homotopy!equivalence}
		A continuous map which induces isomorphisms on all homotopy groups.
	}
	\begin{property}[Homotopy category]
		\nomenclature[S_hTop]{\textbf{hTop}}{Homotopy category}
		The homotopy category hTop has as objects the topological spaces and as morphisms the homotopy classes of continuous maps. It is immediately clear that there exists a functor $F:\textbf{Top}\rightarrow\textbf{hTop}$ that maps topological spaces to themselves and continuous maps to their homotopy classes.
		
		In fact the above definition is often to restrictive. Quillen gave a more general construction: The homotopy category (in the sense of Quillen) is obtained as the localization\footnote{See definition \ref{cat:localization}.} of \textbf{Top} with respect to the collection of weak homotopy equivalences.
	\end{property}
	
	\newdef{Eilenberg-MacLane space}{\index{Eilenberg-MacLane}
		Let $G$ be a group and choose a positive integer $n\in\mathbb{N}_0$. The Eilenberg-MacLane space $K(G, n)$ is a topological space with the following property:
		\begin{gather}
			\pi_i\Big(K(G, n)\Big)=\begin{cases}
				G&i=n\\
				0&i\neq n
			\end{cases}
		\end{gather}
		It follows from property \ref{topology:abelian_homotopy_groups} above that for $n>1$ the group $G$ has to be Abelian.
	}
	\begin{property}
		For every $G$ and $n$ the space $K(G, n)$ is unique up to weak homotopy equivalence.
	\end{property}
	\begin{property}
		The loop space $\Omega K(G, n)$ is homotopy equivalent to $K(G, n-1)$. 
	\end{property}

\subsection{CW complexes}

	\newdef{$n$-cell}{\index{cell}
		An open $n$-cell is a subset of a topological space homeomorphic to the $n$-dimensional open ball. A closed $n$-cell is the image of an $n$-dimensional closed ball under an attaching map\footnote{See definition \ref{topology:attaching_space}.}.
	}
	
	\newdef{CW complex}{\index{CW complex}\label{topology:cw_complex}
		A CW complex is a Hausdorff space $X$ together with a partition of $X$ in open cells satsifying following conditions:
		\begin{itemize}
			\item A subset of $X$ is closed if and only if it meets the closure of each cell in a closed et.
			\item For each open $n$-cell $C$ in the partition there exists an attaching map $f:\overline{B}_n\rightarrow X$ such that:
			\begin{itemize}
				\item $f|_{B_n}$ is homeomorphic to $C$.
				\item $f(\partial \overline{B}_n)$ is covered by a finite number of open cells in the partition, each having dimension smaller than $n$.
			\end{itemize}
		\end{itemize}
		where $\overline{B}_n$ denotes the closed $n$-dimensional ball.
	}
	\newdef{Regular CW complex}{
		A CW complex is called regular if for every open cell $C$ the attaching map $f$ is a homeomorphism onto the closure $\overline{C}$.
	}
	
	\begin{construct}\index{skeleton}
		Every CW complex can, up to isomorphism, be constructed inductively:
		
		First choose a discrete space $X_0$, i.e. a topological space equipped with the discrete topology. This space forms a 0-cell. Then we can add 1-cells $C_1$ using appropriate attaching maps $f:\partial\overline{B}_1\rightarrow X_0$. This way we obtain a 1-dimensional CW complex $X_1$. Inductively one obtains a sequence of nested $n$-dimensional CW complex $X_0\subset X_1\subset\cdots\subset X_n$.
		
		The spaces $X_i$ are also called \textbf{$i$-skeletons}.
	\end{construct}
	\remark{Infinite-dimensional CW complexes can be obtained by taking the direct limit\footnote{See definition \ref{direct_limit}.} of the sequence above.}
	
	\begin{theorem}[Whitehead]\index{Whitehead}
		A continuous map between CW-complexes is a homotopy equivalence if and only if it is a weak homotopy equivalence.
	\end{theorem}
	
\subsection{Fibrations}

	\newdef{Homotopy lifting property}{
		Consider a continuous map $\pi:E\rightarrow B$ between topological spaces. The map $\pi$ is said to have the homotopy lifting property with respect to a topological space $X$ if for every homotopy $f:X\times[0, 1]\rightarrow B$ and lifting $\widetilde{f}_0:X\rightarrow E$ of $f_0=f|_{X\times\{0\}}$ there exists a homotopy $\widetilde{f}:X\times[0, 1]$ lifting $f$ such that the following diagram commutes:
		\begin{figure}[ht!]
			\centering
			\begin{tikzpicture}
				\matrix (m) [matrix of math nodes,row sep=7em,column sep=7em, minimum width=2em, ampersand replacement=\&]{
					X \& E\\
					X\times[0, 1]\& B\\
				};
				\draw[->] (m-1-1) -- (m-1-2) node[pos=0.5, above]{$\widetilde{f}_0$};
				\draw[right hook ->] (m-1-1) -- (m-2-1) node[pos=0.5, left]{$X\times\{0\}$};
				\draw[->] (m-2-1) -- (m-2-2) node[pos=0.5, below]{$f$};
				\draw[->] (m-1-2) -- (m-2-2) node[pos=0.5, right]{$\pi$};
				\draw[dashed, ->] (m-2-1) -- (m-1-2) node[pos=0.5, above]{$\widetilde{f}$};
			\end{tikzpicture}
			\caption{Homotopy lifting property.}
			\label{fig:homotopy_lifting_property}
		\end{figure}
		
		where $\widetilde{f}$ denotes the lifting of $f$, i.e. $f = \pi\circ\widetilde{f}$.
	}
	
	\newdef{Hurewicz fibration}{\index{fibration}
		A map $\pi$ satisfying the homotopy lifting property with respect to every topological space $X$ is called a (Hurewicz) fibration.\footnote{If the homotopy lifting property only holds with respect to CW complexes (see definition \ref{topology:cw_complex}) then it is called a \textbf{Serre fibration}.}
	}
	\begin{property}
		Consider a fibration $\pi:E\rightarrow B$ with $B$ path-connected. All fibres, i.e. sets $\pi^{-1}(\{b\})$ where $b\in B$, are homotopy equivalent. Therefore a fibration is often denoted by the diagram $\prin{F}{E}{B}$.
	\end{property}
	
	We give two important examples:
	\begin{example}[Hopf fibration]\index{Hopf!fibration}
		The Hopf fibration is given by
		\begin{gather}
			\prin{S^1}{S^3}{S^2}
		\end{gather}
		\textit{Adam's theorem} states that this fibration can be generalized to higher dimensions as $\prin{S^n}{S^{2n+1}}{S^{2n}}$ only for $n\in\{0, 1, 3, 7\}$.
	\end{example}
	\begin{example}
		For all $n\in\mathbb{N}$ the following sequence forms a fibration:
		\begin{gather}
			\prin{\text{SO}(n)}{\text{SO}(n+1)}{S^n}
		\end{gather}
	\end{example}

\subsection{Rational homotopy theory}
	
	\newdef{Rational space}{\index{rational!space}
		A simply connected topological space $X$ such that the homotopy groups $\pi_n(X)$ are rational vector spaces.
	}
	
	\newdef{Rational homotopy equivalence}{\index{rational!homotopy equivalence}
		A continuous function $f:X\rightarrow Y$ for which the induced maps on rational homotopy groups
		\begin{gather}
			\pi_n(f)\otimes\mathbb{Q}:\pi_n(X)\otimes\mathbb{Q}\rightarrow\pi_n(Y)\otimes\mathbb{Q}
		\end{gather}
		are isomorphisms for all $n\in\mathbb{N}$.
	}
	
	\newdef{Rational homotopy category}{
		Consider the category \textbf{Top} of topological spaces. The rational homotopy category is obtained as the localization of \textbf{Top} with respect to the collection of rational homotopy equivalences.
	}

\section{Simplicial homology}\label{section:homology}
\subsection{Simplices}

	\newdef{Simplex}{\index{simplex}\index{vertex}\label{topology:simplex}
		A $k$-simplex $\sigma^k = [t_0, ..., t_k]$ is defined as the following set:
		\begin{gather}
			\sigma^k = \left\{\left.\sum_{i=0}^k\lambda_it_i\right|\sum_{i=0}^k\lambda_i = 1\text{ and }\lambda_i\geq0\right\}
		\end{gather}
		where the points (\textbf{vertices}) $t_i\in\mathbb{R}^n$ are \textbf{affinely independent}, i.e. the vectors $t_i-t_0$ are linearly independent. Equivalently, a simplicial $k$-simplex  is the convex hull of the $k+1$ vertices $\{t_0, ..., t_k\}$.
	}
	\begin{remark}[Barycentric coordinates]
		The coordinates $\lambda_i$ from previous definition are called barycentric coordinates. This follows from the fact that the point $\sum_{i=0}^k\lambda_it_i$ represents the barycenter of a gravitational system consisting of masses $\lambda_i$ placed at the points $t_i$.
	\end{remark}
	
	\newnot{Face}{\index{face}
		Consider a simplicial $k$-simplex $[v_0, ..., v_k]$. The face opposite to the vertex $v_i$ is the simplicial $(k-1)$-simplex $[v_0, ..., \hat{v}_i, ..., v_k]$ obtained by removing the vertex $v_i$.
	}
	
	\newdef{Simplicial complex}{\index{simplicial complex}
		A simplicial complex $\mathcal{K}$ is a set of simplices satisfying following conditions:
		\begin{itemize}
			\item If $\sigma$ is a simplex in $\mathcal{K}$ then so are its faces.
			\item If $\sigma_1, \sigma_2\in\mathcal{K}$ then either $\sigma_1\cap\sigma_2 = \emptyset$ or $\sigma_1\cap\sigma_2$ is a face of both $\sigma_1$ and $\sigma_2$.
		\end{itemize}
		A simplicial $k$-complex is a simplicial complex where every simplex has dimension at most $k$.
	}
	
	\newdef{Path-connectedness}{\index{path!connected}
		Let $\mathcal{K}$ be a simplicial complex. $\mathcal{K}$ is said to be path-connected if every two vertices in $\mathcal{K}$ are connected by edges in $\mathcal{K}$.
	}
	
	\newdef{Polyhedron}{\index{polyhedron}
		Let $\mathcal{K}$ be a simplicial complex. The polyhedron associated with $\mathcal{K}$ is the topological space constructed by equipping $\mathcal{K}$ with the Euclidean subspace topology.
	}
	
	
	\newdef{Triangulable spaces}{\index{triangulation}
		Let $X$ be a topological space and let $\mathcal{K}$ be a polyhedron. If there exists a homeomorphism $\varphi:\mathcal{K}\rightarrow X$ then we say that $X$ is triangulable and we call $\mathcal{K}$ a \textbf{triangulation} of $X$.
	}
	\begin{theorem}
		Let $\mathcal{K}$ be a path-connected polyhedron with basepoint $a_0$. Let $\mathcal{C}\subset\mathcal{K}$ be a contractible 1-dimensional subpolyhedron containing all vertices of $\mathcal{K}$. Let $G$ be the free group generated by the elements $g_{ij}$ corresponding to the ordered 1-simplices $[v_i,v_j]\in\mathcal{C}$.
		
		The group $G$ is isomorphic to the fundamental group $\pi_1(\mathcal{K}, a_0)$ if the generators $g_{ij}$ satisfy following two relations:
		\begin{itemize}
			\item $g_{ij}g_{jk} = g_{ik}$ for every ordered 2-simplex $[v_i,v_j,v_k]\in\mathcal{K}\backslash\mathcal{C}$
			\item $g_{ij} = e$ if $[v_i,v_j]\in\mathcal{C}$.
		\end{itemize}
	\end{theorem}
	\begin{result}
		From the theorem that homeomorphic spaces have the same homotopy groups it follows that the fundamental group of a triangulable space can be computed by looking at its triangulations.
	\end{result}
	
	\begin{remark}[Hauptvermutung]
		Although the homology invariants (which we will define below) do not depend on the choice of triangulation we should give a remark about the existence of non-equivalent triangulations. Before the construction of a counterexample it was believed, hence the terms \textit{Hauptvermutung} in German or \textit{main conjecture} in English, that every two triangulations of a topological space allowed a common refinement and hence where equivalent for many constructions. However it was shown that this conjecture is generally false, e.g. for topological manifolds in dimensions 5 and higher there exist an infinite number of non-equivalent triangulations. In dimensions below 4 it was proven by Rad\'o and Moise that the Hauptvermutung holds for topological manifolds (see theorem \ref{manifolds:rado_moise}).
	\end{remark}
	
	\newdef{$\Delta$-complex}{\index{$\Delta$-complex}
		Let $\{\sigma_\alpha:\Delta^n\rightarrow X\}$ be a collection of morphisms from simplicices to $X$ where the dimension $n$ may depend on the subscript $\alpha$. This collection forms a Delta complex (sometimes $\Delta$-complex) on $X$ if it satisfies the following conditions:
		\begin{itemize}
			\item The restrictions $\sigma_\alpha|_{\overset{\circ}{\Delta}\phantom{t}^n}$ are injective and every point in $X$ lies in the image of exactly one such restriction.
			\item The restriction of a morphism $\sigma_\alpha$ to the any one of the faces of $\Delta^n$ is equal to some other $\sigma_\beta$.
			\item A set in $X$ is open if and only if it is open in all the inverse images $\sigma_\alpha^{-1}$.
		\end{itemize}
		
		Similar to a CW-complex or cellular complex these conditions imply that every Delta complex can be constructed inductively from a (discrete) set of vertices by gluing and identifying edges.
	}

\subsection{Simplicial homology}	
	\newdef{Chain group}{\index{chain!group}\label{topology:chain_group}
		Let $\mathcal{K}$ be a simplicial $n$-complex. The $k^{th}$ chain group $C_k(\mathcal{K})$ is defined as the free Abelian group generated by the $k$-simplices in $\mathcal{K}$:
		\begin{gather}
			C_k(\mathcal{K}) = \left\{\left.\sum_ia_i\sigma_i\ \right|\sigma_i\text{ is a $k$-simplex in $\mathcal{K}$}\text{ and }a_i\in\mathbb{Z}\right\}
		\end{gather}
		For $k>n$ we define $C_k(\mathcal{K})$ to be $\{0\}$.
	}
	
	\newdef{Boundary operator}{\index{boundary}
		The boundary operator $\partial_k:C_k(\mathcal{K})\rightarrow C_{k-1}(\mathcal{K})$ is the group morphism defined by following properties:
		\begin{itemize}
			\item Linearity:
			\begin{gather}
				\partial_k\left(\sum_ia_i\sigma_i\right) = \sum_ia_i\partial_k\sigma_i
			\end{gather}
			\item For every oriented $k$-simplex $[v_0, ..., v_k]$:
			\begin{gather}
				\partial_k[v_0, ..., v_k] = \sum_{i=0}^k(-1)^i[v_0, ..., \hat{v}_i, ..., v_k]
			\end{gather}
			\item The boundary of every 0-chain is the identity 0.
		\end{itemize}
	}
	\remark{The alternating sum comes down the fact that we want the \textit{oriented} boundary.}
	
	\begin{property}
		The boundary operators satisfy following relation:
		\begin{gather}
			\label{topology:boundary_operator_relation}
			\partial_k\circ\partial_{k+1} = 0
		\end{gather}
		This property turns the system $(C_k, \partial_k)$ into a chain complex\footnote{See definition \ref{group:chain_complex}.}.
	\end{property}
	
	\newdef{Cycle group}{\index{cycle}
		The $k^{th}$ cycle group $Z_k(\mathcal{K})$ is defined as the set of $k$-chains $\sigma_k$ such that $\partial_k\sigma_k = 0$. These chains are called \textbf{cycles}.
	}
	\newdef{Boundary group}{
		The $k^{th}$ boundary group $B_k(\mathcal{K})$ is defined as the set of $k$-chains $\sigma_k$ for which there exists a $(k+1)$-chain $N$ such that $\partial_{k+1}N = \sigma_k$. These chains are called \textbf{boundaries}.
	}
	\newdef{Homology group}{\index{homology!simplicial}
		From property \ref{topology:boundary_operator_relation} it follows that $B_k(\mathcal{K})\subset Z_k(\mathcal{K})$ is a subgroup. We can thus define the $k^{th}$ homology group $H_k(\mathcal{K})$ as the following quotient group:
		\begin{gather}
			H_k(\mathcal{K}) = Z_k(\mathcal{K}) / B_k(\mathcal{K})
		\end{gather}
		
		Theorem \ref{group:theorem:free_group} tells us that we can write $H_k(\mathcal{K})$ as $G_k\oplus T_k$. Both of these groups tell us something about $\mathcal{K}$. The rank of $G_k$, denoted by $R_k(\mathcal{K})$, is equal to the number of $(k+1)$-dimensional holes in $\mathcal{K}$. The torsion subgroup $T_k$ tells us how the space $\mathcal{K}$ is twisted.
	}	
	
	\begin{property}
		If two topological spaces have the same homotopy type then they have isomorphic homology groups. It follows that homeomorphic spaces have isomorphic homology groups.
	\end{property}
	\begin{result}
		As was the case for the fundamental group, it follows from the definition of a triangulation that we can construct the homology groups for a given triangulable space by looking at its triangulations.
	\end{result}
	
	\newdef{Betti numbers}{\index{Betti number}
		The ranks $R_k(\mathcal{K})$ are called the Betti numbers of $\mathcal{K}$.
	}
	\begin{formula}[Euler characteristic]\index{Euler!characteristic}\index{Euler-Poincar\'e formula}
		The Euler characteristic of a triangulable space $X$ is defined as follows\footnote{This formula is sometimes called the \textit{Poincar\'e} or \textit{Euler-Poincar\'e} formula.}:
		\begin{gather}
			\boxed{\chi(X) = \sum_i(-1)^iR_i(X)}
		\end{gather}
	\end{formula}
	
	\begin{construct}
		The definition of homology groups can be generalized by letting the (formal) linear combinations used in the definition of the chain group (see \ref{topology:chain_group}) be of the following form:
		\begin{gather}
			c^k = \sum_ig_i\sigma_i^k
		\end{gather}
		where $G = \{g_i\}$ is an Abelian group and $\sigma_i^k$ are $k$-simplices. The $k^{th}$ homology group of $X$ with coefficients in $G$ is denoted by $H_k(X; G)$.
	\end{construct}
	\begin{property}
		When $G$ is a field, such as $\mathbb{Q}$ or $\mathbb{R}$, the torsion subgroups $T_k$ vanish. The relation between integral homology and homology with coefficients in a group is given by the \textit{Universal coefficient theorem}.
	\end{property}
	
	\newformula{K\"unneth formula}{\index{K\"unneth formula}
		Let $X, Y$ be two triangulable spaces. The homology groups of the Cartesian product $X\times Y$ with coefficients in a field $F$ is given by:
		\begin{gather}
			H_k(X\times Y; F) = \bigoplus_{k = i+j}H_i(X;F)\otimes H_j(Y;F)
		\end{gather}
	}
	\begin{remark*}
		When the requirement of $F$ being a field is relaxed to it merely being a group, the torsion subgroups have to be taken into account. This will not be done here.
	\end{remark*}
	
\subsection{Relative homology}

	In this section we use a simplicial complex $K$ and a subcomplex $L$.
	
	\newdef{Relative chain group}{\index{chain}
		The $k$-chain group of $K$ modulo $L$ is defined as the following quotient group:
		\begin{gather}
			C_k(K, L) = C_k(K) / C_k(L)
		 \end{gather}
	}
	\newdef{Relative boundary operator}{\index{boundary}
		The relative boundary operator $\overline\partial_k$ is defined as follows:
		\begin{gather}
			\overline\partial_k(c_k+C_k(L)) = \partial_k c_k + C_{k-1}(L)
		\end{gather}
		where $c_k\in C_k(K)$. This operator is a group morphism, just like the ordinary boundary operator $\partial_k$.
	}
	\newdef{Relative homology groups}{\index{homology!relative}
		The relative cycle and relative boundary groups are defined analogous to their ordinary counterparts. The relative homology groups are then defined as follows:
		\begin{gather}
			H_k(K, L) = \stylefrac{\ker(\overline\partial_k)}{\im(\overline\partial_{k+1})}
		\end{gather}
		Elements $h_k\in H_k(K, L)$ can thus be written as $h_k=z_k + C_k(L)$ where $z_k$ does not have to be a relative $k$-cycle but merely  a chain in $C_{k-1}(L)$.
	}
	\newdef{Homology sequence}{
		Using the relative homology groups we obtain following (long) exact sequence:
		\begin{gather}
			\cdots\rightarrow H_k(L)\xrightarrow{i_\ast}H_k(K)\xrightarrow{j_\ast}H_k(K, L)\xrightarrow{\partial_k}H_{k-1}(L)\rightarrow\cdots
		\end{gather}
		where $i_\ast$ and $j_\ast$ are the homology morphisms induced by the inclusions $i:L\rightarrow K$ and $j:K\rightarrow (K, L)$.
	}

	\begin{theorem}[Excision theorem]\index{excision theorem}\label{topology:theorem:excision}
		Let $U, V$ and $X$ be triangulable spaces such that $U\subset V\subset X$. If the closure $\overline{U}$ is contained in the interior $V\degree$ then:
		\begin{gather}
			H_k(X, V) = H_k(X\backslash U, V\backslash U)
		\end{gather}
	\end{theorem}
	
\subsection{Examples}

	\begin{example}
		Let $X$ be a contractible space.
		\begin{gather}
			H_k(X) = \begin{cases}
				\mathbb{Z}&k=0\\
				\{0\}&k>0
			\end{cases}
		\end{gather}
	\end{example}
	\begin{example}
		Let $P$ be a path-connected polyhedron (or path-connected triangulable space):
		\begin{gather}
			H_0(P) = \mathbb{Z}
		\end{gather}
		Furthermore, every point $p\in P$ determines a generator $\langle p \rangle\in H_0(P)$.
	\end{example}
	\begin{example}
		The homology groups of the $n$-sphere $S^n$ are given by:
		\begin{gather}
			H_k(S^n)=\begin{cases}
				\mathbb{Z}&k=0\text{ or }k=n\\
				\{0\}&\text{otherwise}
			\end{cases}
		\end{gather}
	\end{example}
	\newdef{Homology sphere}{\index{homology!sphere}
		A $n$-dimensional manifold having the same homology groups as the $n$-sphere.
	}
	\newdef{Degree}{\index{degree}
		From the example above we know that $H_n(S^n)=\mathbb{Z}$. Given a map $f:S^n\rightarrow S^n$ the induced map $f_*$ on homology is an endomorphism of $\mathbb{Z}$ and hence is of the form $f(x)=dx$ where $d\in\mathbb{Z}$. This coefficient is called the degree of $f$.
	}
	\begin{property}
		Two maps $f:S^n\rightarrow S^n$ have the same degree if and only if they are homotopic.
	\end{property}
	
	\begin{example}
		Consider a closed, connected and orientable manifold $M$ with $\dim(M) = n$.
		\begin{gather}
			H_n(M) = \mathbb{Z}
		\end{gather}
	\end{example}
	\begin{result}[Orientation]\index{orientation}\index{fundamental!class}
		A choice of orientation of $M$ coincides with a choice of generator for $H_n(M)$. This generator is called the \textbf{fundamental class}. In the case $M$ is disconnected, the fundamental class equals the direct sum of the generators of the connected components\footnote{Following the idea of the additivity axiom (see \ref{topology:eilenberg_steenrod_axioms}).}.
	\end{result}

\section{Singular homology}\label{section:singular_homology}
	
	\newdef{Singular simplex}{\index{simplex!singular}\index{simplex!standard}
		Consider the \textbf{standard} $k$-simplex $\Delta^k$:
		\begin{gather}
			\Delta^k = \left\{\left.(x_0, ..., x_k)\in\mathbb{R}^{k+1}\right\vert\sum_ix_i = 1 \text{ and }x_i\geq0\right\}
		\end{gather}
		A singular $k$-simplex in a topological space $X$ is defined as a continuous map $\sigma^k:\Delta^k\rightarrow X$.
	}
	\sremark{The name singular comes from the fact that the maps $\sigma^k$ need not be injective.}
	
	\newdef{Singular chain group}{\index{chain}
		The singular chain group $S_k(X)$ with coefficients in a group $G$ is defined as the set of formal linear combinations $\sum_ig_i\sigma_i^k$. The basis of this free group is in most cases infinite as there are multiple ways to map $\Delta^k$ to $X$.
	}
	
	Before continuing we first need to introduce an important concept in the context of simplicial complexes:
	\newdef{Face map}{\index{face!map}
		The face maps are morphisms $\varepsilon_i^k:\Delta^{k-1}\rightarrow\Delta^k$ that map $\Delta^{k-1}$ onto the i$^{th}$ face of $\Delta^k$. They are explicitly given by:
		\begin{gather}
			\varepsilon_i^k(s_0, ..., s_{k-1})= (s_0, ..., s_{i-1}, 0, s_i, ..., s_{k-1})
		\end{gather}
		Their defining property is the following relation:
		\begin{gather}
			\varepsilon_i^k\circ\varepsilon_j^{k-1} = \varepsilon_j^k\circ\varepsilon_{i-1}^{k-1}
		\end{gather}
		where $j\leq i$.
	}
	\sremark{Some authors (for example the authors at nLab) call these maps \textit{degeneracy maps} and call what in these notes are called \textit{degeneracy maps} face maps.}
	
	\newdef{Singular boundary operator}{\index{face!map}
		The singular boundary operator $\partial$ (we use the same notation as for simplicial boundary operators) is defined by its linear action on the singular chain group $S_k(X)$. It follows that we only have to know the action on the singular simplices $\sigma^k$.
		
		The action of the boundary operator on the singular simplex $\sigma^k$ is then given by:
		\begin{gather}
			\partial_k\sigma^k = \sum_{i=0}^k(-1)^i\sigma^k\circ\varepsilon^k_i
		\end{gather}
		where the $\varepsilon^k_i$ are the face maps defined above. The singular boundary operators satisfy the same relation as the the simplicial boundary operators:
		\begin{gather}
			\partial_{k-1}\circ\partial_k = 0
		\end{gather}
		and hence define a chain complex.
	}
	
	\newdef{Singular homology group}{\index{homology!singular}
		The singular homology groups are defined as follows:
		\begin{gather}
			H_k(X; G) = \stylefrac{\ker(\partial_k)}{\im(\partial_{k+1})}
		\end{gather}
	}
	
	\begin{theorem}
		For triangulable spaces the singular homology is equivalent to simplicial homology.
	\end{theorem}
	\begin{remark}
		When $X$ is not triangulable the previous theorem is not valid. The singular approach to homology is a more general construction, but it is often more difficult to compute the homology groups (even in the case of triangulable spaces).
	\end{remark}
	
	\begin{property}[Induced morphism]\index{pushforward}
		Consider a continuous map $f:X\rightarrow Y$ between topological spaces. This induces a map $f_k^S:S^k(X)\rightarrow S^k(Y)$ on the chain groups as follows:
		\begin{gather}
			f_k^S\left(\sum_\sigma c_\sigma\sigma\right) = \sum_\sigma c_\sigma f\circ\sigma
		\end{gather}
		This map takes cycle (resp. boundary) groups to (subgroups of) cycle (resp. boundary) groups and hence induces a morphism of homology groups\footnote{These induced morphisms are also called \textbf{pushforwards}.}:
		\begin{gather}
			f_\ast:H_k(X)\rightarrow H_k(Y):\langle h \rangle\mapsto \langle f_k^S(h) \rangle
		\end{gather}
	\end{property}
	\begin{result}
		$H_k$ is a functor Top $\rightarrow$ Ab that maps topological spaces to their homology groups and continuous maps $f$ to their pushforward $f_\ast$.
	\end{result}
	
	\begin{theorem}[Hurewicz]\index{Hurewicz}
		Let $X$ be path-connected. Let $[\cdot]$ and $\langle\cdot\rangle$ denote the equivalence classes in the homotopy and homology groups respectively. Then the map\footnote{Every path (and hence loop) is essentially a singular 1-cycle.} $h:\pi(X)\rightarrow H_1(X):[\gamma]\mapsto\langle\gamma\rangle$ defines a group morphism. Furthermore, this map induces an isomorphism $h':\pi(X)/[\pi(X), \pi(X)]\rightarrow H_1(X)$.
		
		More generally for every topological space $Y$ and every $n\in\mathbb{N}$ there exists a morphism $h_*:\pi_n(Y)\rightarrow H_n(Y)$. If $Y$ is $(n-1)$-connected then for every $k\leq n$ this morphism is in fact an isomorphism.
	\end{theorem}
	
	\begin{property}
		Let $X$ be a CW-complex. There exists an isomorphism $[X, K(G, n)]\rightarrow H^n(X; G)$ between the homotopy classes of maps $X\rightarrow K(G, n)$ and the $n^{th}$ singular cohomology of $X$ with coefficients in $G$.
		
		This morphism takes a map $f$ to the pullback $f^*\psi$ where $\psi$ is the inverse of the Hurewicz isomorphism $\pi_n(K)\rightarrow H_n(K;\mathbb{Z})$ and using the isomorphism $G=H^n(K;G)\cong\text{Hom}(H_n(K;\mathbb{Z}), G)$.
	\end{property}
	
\section{Axiomatic approach}

	\newdef{Eilenberg-Steenrod axioms}{\index{Eilenberg-Steenrod axioms}\label{topology:eilenberg_steenrod_axioms}
		All homology theories have a set of properties in common. By treating these properties as axioms we can construct homology theories as a sequence of functors $H_k: \text{Top}\times\text{Top}\rightarrow\text{Ab}$. The axioms are as follows:
		\begin{enumerate}
			\item \textbf{Homotopy}: If $f, g$ are homotopic maps then their induced homology maps are the same, i.e.\[f\cong g\implies H_k(f) = H_k(g), \forall k\in\mathbb{N}\]
			\item \textbf{Excision}\footnotemark: If $U\subset V\subset X$ and $\overline U\subset V\degree$ then $H_k(X, V) \cong H_k(X\backslash U, V\backslash U)$
			\item \textbf{Dimension}: If $X$ is a singleton then $H_k(X) = \{0\}$ for all $k\geq1$. The group $H_0(X)$ is called the \textbf{coefficient group} and gives the coefficients used in the linear combinations of the chain group.
			\item \textbf{Additivity}: If $X = \bigsqcup_i X_i$ then $H_k(X)\cong\bigoplus_i H_k(X_i)$
			\item \textbf{Exactness}: Each pair $(X, A)$, where $A\subset X$, induces a long exact sequence
			\begin{gather}
				\cdots\rightarrow H_k(A)\xrightarrow{i_\ast}H_k(X)\xrightarrow{j_\ast}H_k(X, A)\xrightarrow{\partial_k}H_{k-1}(A)\rightarrow\cdots
			\end{gather}
			where $i_\ast$ and $j_\ast$ are the homology morphisms induced by the inclusions $i:A\rightarrow X$ and $j:X\rightarrow (X, A)$.
		\end{enumerate}
		\footnotetext{See also theorem \ref{topology:theorem:excision}.}
	}
	\begin{remark}
		If the dimension axiom is removed from the set of axioms then we obtain a so-called \textit{extraordinary homology theory}.
	\end{remark}
	
\section{Equivariant cohomology}

	In this section we will consider topological spaces equipped with a continuous action of a topological group $G$. These spaces will be called topological $G$-spaces or just $G$-spaces.
	
	\newdef{Equivariant cohomology}{
		Let $X$ be a topological $G$-space for which the $G$-action is free. The equivariant cohomology of $X$ is defined as:
		\begin{gather}
			H_G^*(X) = H^*(X/G)
		\end{gather}
		where the $X/G$ is the orbit space with respect to the action of $G$ on $X$.
	}

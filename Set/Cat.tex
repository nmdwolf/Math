\chapter{Categories, Operads and Topoi}

\section{Category theory}
\subsection{Categories and subcategories}

	\newdef{Subcategory}{
		Let $C$ be a category. A subcategory $S$ of $C$ consists of a subcollection of objects ob$_S$ and a subcollection of morphisms hom$_S$ that satisfy following conditions:
		\begin{itemize}
			\item For every object in ob$_S$ the identity morphism is in hom$_S$.
			\item For every morphism in hom$_S$ both the source and target are in ob$_S$.
			\item For every pair of morphisms in hom$_S$ the composition is also in hom$_S$.
		\end{itemize}
		A subcategory is said to be \textbf{full} if for every two objects $X, Y\in\text{ob}_S:$ \[\text{hom}_S(X, Y) = \text{hom}_C(X, Y)\]
	}
	
	\newdef{Small category}{\index{small}
		A category $C$ is said to be small if both ob$(C)$ and hom$(C)$ are sets. A category $C$ is said to be locally small if for every two objects $X, Y\in\text{ob}(C)$ the collection of morphisms hom$_C(X, Y)$ is a set.
	}
	
	\newdef{Opposite categopy}{
		Let $C$ be a category. The opposite category $C^{op}$ is defined by reversing all arrows (morphisms) in $C$.
	}
	\begin{property}
		From the definition of the opposite category it easily follows that
		\begin{equation}
			(C^{op})^{op} = C
		\end{equation}
	\end{property}

\subsection{Functors}

	\newdef{Covariant functor}{\index{functor}
		Let $A, B$ be categories. A (covariant) functor $F$ is a map $A\rightarrow B$ satisfying following conditions:
		\begin{itemize}
			\item $F$ maps every object $X\in\text{ob}(A)$ to an object $F(X)\in\text{ob}(B)$.
			\item $F$ maps every morphism $\phi\in\text{hom}_A(X, Y)$ to a morphism $F(\phi)\in\text{hom}_B(F(X), F(Y))$.
			\item $F(\mathbbm{1}_X) = \mathbbm{1}_{F(X)}$
			\item $F(\phi\circ\psi) = F(\phi)\circ F(\psi)$
		\end{itemize}
	}
	\newdef{Contravariant functor}{
		Let $A, B$ be categories. A contravariant functor $F$ is a map $A\rightarrow B$ satisfying following conditions:
		\begin{itemize}
			\item $F$ maps every object $X\in\text{ob}(A)$ to an object $F(X)\in\text{ob}(B)$.
			\item $F$ maps every morphism $\phi\in\text{hom}_A(X, Y)$ to a morphism $F(\phi)\in\text{hom}_B(F(Y), F(X))$.
			\item $F(\mathbbm{1}_X) = \mathbbm{1}_{F(X)}$
			\item $F(\phi\circ\psi) = F(\psi)\circ F(\phi)$
		\end{itemize}
	}
	\remark{A contravariant functor can also be defined as a covariant functor in the opposite category.}
	
	\newdef{Faithful functor}{\index{faithful}
		A functor $F:C\rightarrow D$ is said to be faithful if the map hom$_C(X, Y)\rightarrow \text{hom}_D(F(X), F(Y))$ is injective for all objects $X, Y\in\text{ob}(C)$.
	}
	\newdef{Full functor}{\index{full}
		A functor $F:C\rightarrow D$ is said to be full if the map hom$_C(X, Y)\rightarrow \text{hom}_D(F(X), F(Y))$ is surjective for all objects $X, Y\in\text{ob}(C)$.
	}
	
	
	\begin{example}[hom-functor]
		Let $C$ be a locally small category. Every object $X\in\text{ob}(C)$ induces a functor $h^X:C\rightarrow\text{Set}$ defined as follows:
		\begin{itemize}
			\item $h^X$ maps every object $Y\in\text{ob}(C)$ to the set hom$(X, Y)$.
			\item For all $Y, Z\in\text{ob}(C)$, $h^X$ maps every morphism $f\in\text{hom}_C(Y, Z)$ to the morphism $f\circ-:\text{hom}_C(X, Y)\rightarrow\text{hom}_C(X, Z):g\mapsto f\circ g$.
		\end{itemize}
	\end{example}
	\remark{The contravariant hom-functor $h_X$ is defined by replacing hom$(X, -)$ by hom$(-, X)$.}
	
	\newdef{Natural transformation}{\index{natural!transformation}
		Let $F, G$ be two functors between the categories $C$ and $D$. A natural transformation $\psi$ from $F$ to $G$ consists of a collection of morphisms satisfying two conditions:
		\begin{itemize}
			\item For every object $X\in\text{ob}(C)$ there exists a morphism $\psi_X:F(X)\rightarrow G(X)$ in $\text{hom}(D)$. This morphism is called the component of $\psi$ at $X$.
			\item For every morphism $f\in\text{hom}_C(X, Y)$ we have $\psi_Y\circ F(f) = G(f)\circ\psi_X$.
		\end{itemize}
		It is often said that \textit{$\psi_X$ is natural in $X$}.
	}
	\begin{notation}
		A natural transformation $\psi$ from a functor $F$ to a functor $G$ is denoted by $\psi: F\Rightarrow G$.
	\end{notation}
	
	\newdef{Representable functor}{\index{representation}
		Let $C$ be a locally small category. A functor $F:C\rightarrow\text{Set}$ is said to be representable if there exists an object $X\in\text{ob}(C)$ such that $F$ is naturally isomorphic to $h^X$. The pair $(X, \psi)$, where $\psi$ is the natural isomorphism, is called a \textbf{representation} of $F$.
	}
	
	\newdef{Functor category}{\index{functor!category}
		Let $C$ be a small category and let $D$ be a category. The functors $F:C\rightarrow D$ form the objects of a category with the natural transformations as morphisms. This category is denoted by $[C, D]$ or $D^C$ analogous to \ref{set:function_set}.
	}
	
	\begin{theorem}[Yoneda's lemma]\index{Yoneda}
		Let $C$ be a locally small category and let $F:C\rightarrow\text{Set}$ be a functor. For every object $X\in\text{ob}(C)$ there exists a natural isomorphism\footnotemark\ between the set of natural transformations Nat$(h^X, F)$ and $F(X)$.
		\footnotetext{Here we used the fact that Nat$(h^-, -)$ can be seen as a functor from the product category Set$^C\times C$ to the category Set.}
	\end{theorem}
	\sremark{The image of a natural transformation $\psi\in\text{Nat}(h^X, F)$ is given by $\psi_X(\mathbbm{1}_X)$.}
	\begin{result}[Yoneda embedding]
		When $F$ is another hom-functor $h^Y$ we obtain the following result:
		\begin{equation}
			\text{Nat}(h^X, h^Y)\cong\text{hom}_C(Y, X)
		\end{equation}
		where one should pay attention to the right hand side where $Y$ appears in the \underline{first} argument.
		
		Let $\text{hom}(f, -)$ denote the natural transformation corresponding to the morphism $f\in\text{hom}_C(Y,X)$. The (contravariant) functor $h^-$ mapping every object $X\in\text{ob}(C)$ to its hom-functor hom$(A, -)$ and every morphism $f\in\text{hom}_C(Y, X)$ to the natural transformation hom$(f, -)$ can also be interpreted as a covariant functor $G:C^{op}\rightarrow\text{Set}^C$. This way we see that Yoneda's lemma gives us a fully faithful functor (i.e. an embedding) $h^-$ from the opposite category $C^{op}$ to the functor category Set$^C$.
		
		As usual all of this can be done for contravariant functors. This gives us an embedding $h_-:C\hookrightarrow\text{Set}^{C^{op}}$, called the Yoneda embedding. 
	\end{result}

\subsection{Initial and terminal objects}

	\newdef{Initial object}{
		An object $O$ in a category $C$ is called initial if for every other object $P$ there exists a unique morphism $\iota_{O, P}:O\rightarrow P$.
	}
	\newdef{Terminal object}{
		An object $O$ in a category $C$ is called terminal if for every other object $P$ there exists a unique morphism $\tau_{O, P}:P\rightarrow O$.
	}
	\begin{property}
		If an initial (resp. terminal) object exists, then it is unique (possibly up to isomorphism).
	\end{property}
	
	\newdef{Comma category}{\index{comma category}
		Consider three categories $A, B$ and $C$ together with functors $S:A\rightarrow C$ and $T:B\rightarrow C$. The comma category $S\downarrow T$ is defined as follows:
		\begin{itemize}
			\item The objects consist of all triples $(O_A, O_B, h)$ where $O_A\in\text{ob}(A), O_B\in\text{ob}(B)$ and $h\in\text{hom}(C):S(O_A)\rightarrow T(O_B)$.
			\item The morphisms between $(O_A, O_B, h)$ and $(O_A', O_B', h')$ are given by pairs $(f, g)\in\text{hom}_A(O_A, O_A')\times\text{hom}_B(O_B, O_B')$ that make the following diagram commute:
			\begin{figure}[ht!]
				\centering
				\begin{tikzpicture}
					\matrix (m) [matrix of math nodes,row sep=5em,column sep=5em, minimum width=2em, ampersand replacement=\&]{
						S(O_A) \& S(O_A')\\
						T(O_B) \& T(O_B')\\
					};
					
					\draw[->] (m-1-1) -- (m-2-1) node[pos=0.5, left]{$h$};
					\draw[->] (m-1-2) -- (m-2-2) node[pos=0.5, right]{$h'$};
					\draw[->] (m-1-1) -- (m-1-2) node[pos=0.5, above]{$S(f)$};
					\draw[->] (m-2-1) -- (m-2-2) node[pos=0.5, below]{$T(g)$};
				\end{tikzpicture}
				\caption{Comma category.}
				\label{fig:comma_category}
			\end{figure}
			\item Composition of morphisms is defined component-wise.
		\end{itemize}
	}
	

\section{Operad theory}
\subsection{Operads}

	\newdef{Plain operad\footnotemark}{\index{operad}\index{arity}
		\footnotetext{Also called a \textbf{non-symmetric operad} or \textbf{non-$\Sigma$ operad}.}
		Let $\mathcal{O} = \{P(n)\}_{n\in\mathbb{N}}$ be a sequence of sets, called \textbf{$n$-ary operations} ($n$ is the \textbf{arity}). The set $\mathcal{O}$ is called a plain operad if it satisfies following axioms:
		\begin{enumerate}
			\item $P(1)$ contains an identity element $\mathbbm{1}$.
			\item For all positive integers $n, k_1, ..., k_n$ there exists a composition
			\begin{equation}
				\circ:P(n)\times P(k_1)\times\cdots\times P(k_n)\rightarrow P(k_1+\cdots k_n):(\psi, \theta_1, ..., \theta_n)\mapsto \psi\circ(\theta_1, ..., \theta_n)
			\end{equation}
			that satisfies two additional axioms:
			\begin{itemize}
				\item Identity:
				\begin{equation}
					\theta\circ (\mathbbm{1}, ..., \mathbbm{1}) = \mathbbm{1}\circ\theta = \theta
				\end{equation}
				\item Associativity:
				\begin{align}
					\psi\circ\Big(\theta_1\circ&(\theta_{1, 1}, ..., \theta_{1, k_1}), ..., \theta_n\circ(\theta_{n, 1}, ..., \theta_{n, k_n})\Big)\nonumber\\
					&= \Big(\psi\circ(\theta_1, ..., \theta_n)\Big) \circ (\theta_{1,1}, ..., \theta_{1, k_1}, \theta_{2, 1}, ..., \theta_{n, k_n})
				\end{align}
			\end{itemize}
		\end{enumerate}
	}


\section{Topos theory}

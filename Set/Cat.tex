\chapter{Categories, Operads and Topoi}

\section{Category theory}
\subsection{Categories and subcategories}

	\newdef{Subcategory}{
		Let $C$ be a category. A subcategory $S$ of $C$ consists of a subcollection of objects ob$_S$ and a subcollection of morphisms hom$_S$ that satisfy following conditions:
		\begin{itemize}
			\item For every object in ob$_S$ the identity morphism is in hom$_S$.
			\item For every morphism in hom$_S$ both the source and target are in ob$_S$.
			\item For every pair of morphisms in hom$_S$ the composition is also in hom$_S$.
		\end{itemize}
		A subcategory is said to be \textbf{full} if for every two objects $X, Y\in\ob_S:$
		\begin{equation}
			\text{hom}_S(X, Y) = \text{hom}_C(X, Y)
		\end{equation}
	}
	
	\newdef{Small category}{\index{small}
		A category $C$ is said to be small if both ob$(C)$ and hom$(C)$ are sets. A category $C$ is said to be locally small if for every two objects $X, Y\in\ob(C)$ the collection of morphisms hom$_C(X, Y)$ is a set.
	}
	
	\newdef{Opposite categopy}{
		Let $C$ be a category. The opposite category $C^{op}$ is defined by reversing all arrows in $C$.
	}
	\begin{property}
		From the definition of the opposite category it easily follows that
		\begin{equation}
			(C^{op})^{op} = C
		\end{equation}
		i.e. $op$ is an involution.
	\end{property}
	
	\newdef{Enriched category}{\index{category!enriched}
	        Let $(M, \otimes, \mathbf{1})$ be a monoidal category. An enriched category over $M$, also called an $M$-category, consists of following elements:
	        \begin{itemize}
	                \item A collection of objects ob$(C)$.
	                \item For every pair of object $A, B\in\ob(C)$ there is an object $C(A, B)\in\ob(M)$ for which the following morphisms exist:
	                \begin{itemize}
	                        \item id$_A: \mathbf{1}\rightarrow C(A,A)$
	                        \item $\circ_{ABC}:C(B, C)\otimes C(A, B)\rightarrow C(A, C)$ replacing the usual composition  
	                \end{itemize}
	        \end{itemize}
	        The associativity and identity morphisms from ordinary categories are given by commutative diagrams of the id and $\circ$ morphisms together with the associators and unitors in $M$.
	}
	
	\newdef{Skeletal category}{\index{category!skeletal}
		A category is said to be skeletal if isomorphic objects are identical, i.e. every isomorphism is an identity morphism. The \textbf{skeleton} of a category is an equivalent skeletal category (often taken to be a subcategory).
	}
	
\subsection{Morphisms}

	\newdef{Monomorphism}{\index{monomorphism}
		Let $C$ be a category. A morphism $\mu\in\text{hom}_C(A, B)$ is called a monomorphism\footnote{Sometimes just a \textbf{mono} or a \textbf{monic} morphism.} if for every object $X\in\ob(C)$ and every two morphisms $\alpha_1, \alpha_2\in\text{hom}_C(X, A)$ such that $\mu\circ\alpha_1 = \mu\circ\alpha_2$ we can conclude that $\alpha_1=\alpha_2$.
	}
	\newdef{Epimorphism}{\index{epimorphism}
		Let $C$ be a category. A morphism $\varepsilon\in\text{hom}_C(A, B)$ is called an epimorphism\footnote{Sometimes just an \textbf{epi} or an \textbf{epic} morphism.} if for every object $X\in\ob(C)$ and every two morphisms $\alpha_1, \alpha_2\in\text{hom}_C(B, X)$ such that $\alpha_1\circ\varepsilon = \alpha_2\circ\varepsilon$ we can conclude that $\alpha_1=\alpha_2$.
	}
	
	\newdef{Injective object}{\index{injective!object}
		Let $C$ be an Abelian category. An object $I\in\ob(C)$ is said to be injective if for every $A, B\in\ob(C)$, monomorphism $f:A\rightarrow B$ and morphism $g:A\rightarrow I$ there exists a morphism $\phi:B\rightarrow I$ such that $\phi\circ f = g$.
		\begin{figure}[ht!]
			\centering
			\begin{tikzpicture}
				\matrix (m) [matrix of math nodes,row sep=7em,column sep=7em, minimum width=2em, ampersand replacement=\&]{
					A\&B\\
					I\&\\
				};
				\draw[right hook ->] (m-1-1) -- (m-1-2) node[pos=0.5, above]{$f$};
				\draw[->] (m-1-1) -- (m-2-1) node[pos=0.5, left]{$g$};
				\draw[dashed, ->] (m-1-2) -- (m-2-1) node[pos=0.5, below right]{$\phi$};
			\end{tikzpicture}
			\caption{Injective object $I$.}
			\label{fig:injective_object}
		\end{figure}
	}
	Dually one can construct:
	\newdef{Projective object}{\index{projective!object}
		Let $C$ be an Abelian category. An object $P\in\ob(C)$ is said to be projective if for every $A, B\in\ob(C)$, epimorphism $f:A\rightarrow B$ and morphism $g:P\rightarrow A$ there exists a morphism $\phi:P\rightarrow B$ such that $f\circ\phi = g$.
		\begin{figure}[ht!]
			\centering
			\begin{tikzpicture}
				\matrix (m) [matrix of math nodes,row sep=7em,column sep=7em, minimum width=2em, ampersand replacement=\&]{
					A\&B\\
					\&P\\
				};
				\draw[->>] (m-1-1) -- (m-1-2) node[pos=0.5, above]{$f$};
				\draw[->] (m-2-2) -- (m-1-2) node[pos=0.5, right]{$g$};
				\draw[dashed, ->] (m-2-2) -- (m-1-1) node[pos=0.5, below left]{$\phi$};
			\end{tikzpicture}
			\caption{Projective object $P$.}
			\label{fig:projective_object}
		\end{figure}
	}

\subsection{Functors}

	\newdef{Covariant functor}{\index{functor}
		Let $A, B$ be categories. A (covariant) functor $F$ is a map $A\rightarrow B$ satisfying following conditions:
		\begin{itemize}
			\item $F$ maps every object $X\in\ob(A)$ to an object $FX\in\ob(B)$.
			\item $F$ maps every morphism $\phi\in\text{hom}_A(X, Y)$ to a morphism $F(\phi)\in\text{hom}_B(FX, FY)$.
			\item $F(\mathbbm{1}_X) = \mathbbm{1}_{FX}$
			\item $F(\phi\circ\psi) = F(\phi)\circ F(\psi)$
		\end{itemize}
	}
	\newdef{Contravariant functor}{
		Let $A, B$ be categories. A contravariant functor $F$ is a map $A\rightarrow B$ satisfying following conditions:
		\begin{itemize}
			\item $F$ maps every object $X\in\ob(A)$ to an object $FX\in\ob(B)$.
			\item $F$ maps every morphism $\phi\in\text{hom}_A(X, Y)$ to a morphism $F(\phi)\in\text{hom}_B(FY, FX)$.
			\item $F(\mathbbm{1}_X) = \mathbbm{1}_{FX}$
			\item $F(\phi\circ\psi) = F(\psi)\circ F(\phi)$
		\end{itemize}
	}
	\remark{A contravariant functor can also be defined as a covariant functor from the opposite category.}
	
	\newdef{Faithful functor}{\index{faithful}
		A functor $F:C\rightarrow D$ is said to be faithful if the map hom$_C(X, Y)\rightarrow \text{hom}_D(FX, FY)$ is injective for all objects $X, Y\in\ob(C)$.
	}
	\newdef{Full functor}{\index{full}
		A functor $F:C\rightarrow D$ is said to be full if the map hom$_C(X, Y)\rightarrow \text{hom}_D(FX, FY)$ is surjective for all objects $X, Y\in\ob(C)$.
	}
	
	\begin{example}[hom-functor]
		Let $C$ be a locally small category. Every object $X\in\ob(C)$ induces a functor $h^X:C\rightarrow\text{Set}$ defined as follows:
		\begin{itemize}
			\item $h^X$ maps every object $Y\in\ob(C)$ to the set hom$_C(X, Y)$.
			\item For all $Y, Z\in\ob(C)$, $h^X$ maps every morphism $f\in\text{hom}_C(Y, Z)$ to the morphism $f\circ-:\text{hom}_C(X, Y)\rightarrow\text{hom}_C(X, Z):g\mapsto f\circ g$.
		\end{itemize}
	\end{example}
	\remark{The contravariant hom-functor $h_X$ is defined by replacing hom$(X, -)$ by hom$(-, X)$.}
	
	\newdef{Discrete fibration}{\index{fibration}
		Let $F:A\rightarrow B$ be a functor. $F$ is a discrete fibration if for every object $a\in\ob(A)$ and every morphism $f:b\rightarrow F(a)$ in $B$ there exists a unique morphism $g:c\rightarrow a$ in $A$ such that $F(g) = f$, where $b\in\ob(B), c\in\ob(A)$.
	}
	
	\newdef{Comma category}{\index{category!comma}
		Let $A, B, C$ be three categories and let $F:A\rightarrow C$ and $G:B\rightarrow C$ be two functors. The comma category $F\downarrow G$ is defined as follows:
		\begin{itemize}
			\item Objects are triples $(a, b, \gamma)$ where $a\in\ob(A), b\in\ob(B)$ and $\gamma:F(a)\rightarrow G(b)\in\text{hom}(C)$.
			\item Morphisms $(a, b, \gamma)\rightarrow(k, l, \sigma)$ are pairs $(f, g)$ where $f:a\rightarrow k\in\text{hom}(A)$ and $g:b\rightarrow l\in\text{hom}(B)$ such that $\sigma\circ F(f) = G(g)\circ\gamma$.
			\item Composition of morphisms is defined componentwise.
		\end{itemize}
	}
	\newdef{Slice category}{\index{category!slice}
		Let $C$ be a category and let $c\in\ob(C)$. The slice category $C/c$ of $C$ over $c$ is defined as follows:
		\begin{itemize}
			\item Objects are morphisms in $C$ with codomain $c$.
			\item Morphisms $f\rightarrow g$ are morphisms $h$ in $C$ such that $g\circ h = f$.
		\end{itemize}
	}
	
	\newdef{Sieve}{\index{sieve}
		Let $C$ be a small category. A sieve $S$ on $C$ is a fully faithfull discrete fibration $S\rightarrow C$.
		
		A sieve $S$ on an object $c\in C$ is a sieve in the slice category $C/c$. This means that $S$ is a subset of $\ob(C/c)$ that is closed under \textit{precomposition}, i.e. if $b\rightarrow c\in S$ and $a\rightarrow b\in\text{hom}(C)$ then the composition $a\rightarrow b\rightarrow c\in S$.
	}
	

\subsection{Natural transformations}
	\newdef{Natural transformation}{\index{natural!transformation}
		Let $F, G$ be two functors between the categories $C$ and $D$. A natural transformation $\psi$ from $F$ to $G$ consists of a collection of morphisms satisfying two conditions:
		\begin{itemize}
			\item For every object $X\in\ob(C)$ there exists a morphism $\psi_X:FX\rightarrow GX$ in $\text{hom}(D)$. This morphism is called the component of $\psi$ at $X$.
			\item For every morphism $f\in\text{hom}_C(X, Y)$ we have $\psi_Y\circ F(f) = G(f)\circ\psi_X$.
		\end{itemize}
		It is often said that \textbf{$\psi_X$ is natural in $X$}.
	}
	\begin{notation}
		A natural transformation $\psi$ from a functor $F$ to a functor $G$ is denoted by $\psi: F\Rightarrow G$.\footnote{This is in analogy with the notation for general 2-morphisms. See section \ref{cat:higher_category_theory} for more information.}
	\end{notation}
	
	\begin{example}
	        When considering representations as a functor $\rho:\text{Grp}\rightarrow\text{FinVect}$, we see that the intertwiners\footnote{See definition \ref{group:equivariant}.} play the rol of natural transformations.
	\end{example}
	\begin{example}[Functor category]\index{functor!category}
		Let $C$ be a small category and let $D$ be a category. The functors $F:C\rightarrow D$ form the objects of a category with the natural transformations as morphisms. This category is denoted by $[C, D]$ or $D^C$ (analogous to \ref{set:function_set}).
	\end{example}
	
	\newdef{Representable functor}{\index{functor!representable}
		Let $C$ be a locally small category. A functor $F:C\rightarrow\text{Set}$ is said to be representable if there exists an object $X\in\ob(C)$ such that $F$ is naturally isomorphic to $h^X$. The pair $(X, \psi)$, where $\psi$ is the natural isomorphism, is called a \textbf{representation} of $F$.
	}
	
	\begin{theorem}[Yoneda's lemma]\index{Yoneda}
		Let $C$ be a locally small category and let $F:C\rightarrow\emph{Set}$ be a functor. For every object $X\in\emph\ob(C)$ there exists a natural isomorphism\footnotemark\ between the set of natural transformations \emph{Nat}$(h^X, F)$ and $FX$.
		\footnotetext{Here we used the fact that Nat$(h^-, -)$ can be seen as a functor Set$^C\times C\rightarrow$ Set.}
	\end{theorem}
	\sremark{The image of a natural transformation $\psi\in\text{Nat}(h^X, F)$ is given by $\psi_X(\mathbbm{1}_X)$.}
	\begin{result}[Yoneda embedding]
		When $F$ is another hom-functor $h^Y$ we obtain the following result:
		\begin{equation}
			\text{Nat}(h^X, h^Y)\cong\text{hom}_C(Y, X)
		\end{equation}
		where one should pay attention to the right hand side where $Y$ appears in the \underline{first} argument.
		
		Let $\text{hom}(f, -)$ denote the natural transformation corresponding to the morphism $f\in\text{hom}_C(Y,X)$. The (contravariant) functor $h^-$ mapping every object $X\in\ob(C)$ to its hom-functor hom$(X, -)$ and every morphism $f\in\text{hom}_C(Y, X)$ to the natural transformation hom$(f, -)$ can also be interpreted as a covariant functor $G:C^{op}\rightarrow\text{Set}^C$. This way we see that Yoneda's lemma gives us a fully faithful functor (i.e. an embedding) $h^-$ from the opposite category $C^{op}$ to the functor category Set$^C$.
		
		As usual all of this can be done for contravariant functors. This gives us an embedding $h_-:C\hookrightarrow\text{Set}^{C^{op}}$, called the Yoneda embedding. 
	\end{result}

\subsection{Initial and terminal objects}

	\newdef{Initial object}{
		An object $O$ in a category $C$ is called initial if for every other object $P$ there exists a unique morphism $\iota_P:O\rightarrow P$.
	}
	\newdef{Terminal object}{
		An object $O$ in a category $C$ is called terminal if for every other object $P$ there exists a unique morphism $\tau_P:P\rightarrow O$. This object is sometimes denoted by $\mathbf{1}$.
	}
	\begin{property}
		If an initial (resp. terminal) object exists, then it is unique (possibly up to isomorphism).
	\end{property}
	
	\newdef{Zero object}{\index{zero!object}\label{cat:zero_object}
		An object which is both initial and terminal. The zero object is often denoted by $\mathbf{0}$.
	}
	\begin{property}[Zero morphism]
		From the definition of the zero object it follows that for any two objects $a, b$ there exists a unique morphism $0_{ab}:a\rightarrow b$.
	\end{property}

\subsection{Limits and products}

	\newdef{Diagram}{\index{diagram}
		Let $I, C$ be two categories. A diagram in $C$ with index category $I$ is a covariant functor $D:I\rightarrow C$.
	}
	
	\newdef{Cone}{\index{cone}
		Let $D:I\rightarrow C$ be a diagram. A cone from $a\in\ob(C)$ to $D$ consists of a family of morphisms $\psi_i:a\rightarrow Di, \forall i\in\ob(I)$ such that $\psi_{j} = Df\circ\psi_i$ for all morphisms $f:i\rightarrow j\in\text{hom}(I)$. (See figure \ref{fig:cone_component})
		
		\begin{figure}[ht!]
			\centering
			\begin{subfigure}[b]{0.49\textwidth}
				\centering
				\begin{tikzpicture}
					\matrix (m) [matrix of math nodes,row sep=4em,column sep=4em, minimum width=1em, ampersand replacement=\&]{
						\&a\&\\
						Di\vphantom{(}\&\&Df(i)\\
					};
					\draw[->] (m-1-2) -- (m-2-1) node[pos=0.5, above left]{$\psi_i$};
					\draw[->] (m-1-2) -- (m-2-3) node[pos=0.5, above right]{$\psi_{f(i)}$};
					\draw[->] (m-2-1) -- (m-2-3) node[pos=0.5, below]{$Df$};
				\end{tikzpicture}
				\caption{Component of cone over $D$.}
				\label{fig:cone_component}
			\end{subfigure}
			\begin{subfigure}[b]{0.49\textwidth}
				\centering
				\begin{tikzpicture}
					\matrix (m) [matrix of math nodes,row sep=4em,column sep=4em, minimum width=1em, ampersand replacement=\&]{
						a\vphantom{b}\&\&b\\
						\&Di\&\\
					};
					\draw[->] (m-1-1) -- (m-2-2) node[pos=0.5, below left]{$\psi_i$};
					\draw[->] (m-1-3) -- (m-2-2) node[pos=0.5, below right]{$\phi_i$};
					\draw[->] (m-1-1) -- (m-1-3) node[pos=0.5, above]{$f$};
				\end{tikzpicture}
				\caption{Morphism of cones.}
				\label{fig:cone_morphism}
			\end{subfigure}
			\label{fig:cone}
		\end{figure}
	}
	\begin{adefinition}\index{diagonal!functor}
		This definition can be reformulated by defining an additional functor\footnote{The notation $\Delta_a$ tells us that $\Delta:C\rightarrow [J,C]$ is the \textbf{diagonal functor}, i.e. $\Delta(c)$ is the constant functor from $I$ to $C$ with target object $c$.} $\Delta_a:I\rightarrow C$ which maps every element $i\in\ob(I)$ to $a$ and every morphism $g\in\text{hom}(I)$ to $\mathbbm{1}_a$. The morphisms $\psi_c$ can then be seen as the components of a natural transformation $\psi:\Delta_a\implies D$. Hence a cone $(a, \psi)$ is an element of $\text{hom}_{[I, C]}(\Delta_a, D)$.
	\end{adefinition}

	\newdef{Morphism of cones}{\index{morphism!of cones}
		Let $D:I\rightarrow C$ be a diagram and let $(a, \psi), (b, \phi)$ be cones to $D$. A morphism between these cones is a morphism of the apexes $f:a\rightarrow b$ such that the diagrams of the form \ref{fig:cone_morphism} commute for all $i\in\ob(I)$. The cones to $D$ together with these morphisms form a category Cone$(D)$, in fact this can easily be seen to be the comma category $(\Delta \downarrow D)$.
	}
	
	\newdef{Limit}{\index{limit}
		Consider a diagram $D:I\rightarrow C$. The limit $\lim D$ of this diagram, if it exists, is the terminal object of the category Cone$(D)$.
	}
	This definition gives us following universal property:
	\begin{uproperty}
		Let $D:I\rightarrow C$ be a diagram and let $\lim D$ be its limit. For every cone $(a, \psi)\in\text{Cone}(D)$ there exists a unique morphism $f:C\rightarrow\lim D$.
	\end{uproperty}
	
	\newdef{Equalizer}{\index{equalizer}
		Consider following diagram:\[X\overset{f}{\underset{g}{\rightrightarrows}} Y\] The limit of this diagram is called the equalizer of $f$ and $g$. Explicitly the equalizer is an object $E$ together with a morphism $e: E\rightarrow X$ such that $f\circ e = g\circ e$ and such that $E$ satisfies a universal property.
	}
	\begin{property}
		Consider an equalizer $(E, e)$. The morphism equalizing morphism $e$ is a monomorphism.
	\end{property}
	
	\newdef{Finitely complete category}{\index{finitely!complete}
		A category is said to be finitely complete if all equalizers and all finite products exist.
	}
	
	\newdef{Pullback\footnotemark}{\index{pullback}
		\footnotetext{Also called a \textbf{fibre product} or \textbf{Cartesian square}.}
		The pullback of two morphisms $f:a\rightarrow c$ and $g:b\rightarrow c$ is defined as the limit of following diagram:
		\begin{figure}[ht!]
			\centering
			\begin{tikzpicture}
				\matrix (m) [matrix of math nodes,row sep=4em,column sep=4em, minimum width=1em, ampersand replacement=\&]{
					a\&\&b\\
					\&c\&\\
				};
				\draw[->] (m-1-1) -- (m-2-2) node[pos=0.5, below left]{$f$};
				\draw[->] (m-1-3) -- (m-2-2) node[pos=0.5, below right]{$g$};
			\end{tikzpicture}
			\caption{Pullback of morphisms.}
			\label{fig:pullback_diagram}
		\end{figure}
	}
	\begin{notation}[Pullback]
		The pullback of two morphisms $f:a\rightarrow c$ and $g:b\rightarrow c$ is often denoted by $P = a\times_c b$.
	\end{notation}
	
	\begin{property}
		If a terminal object $\mathbf{1}$ exists then the pullback $a\times_{\mathbf{1}}b$ is equal to the product $a\times b$.
	\end{property}
	\newdef{Pushout}{\index{pushout}
		The dual notion of a pullback.
	}

\subsection{Abelian categories}

	\newdef{Preadditive category}{
		A (locally small) category enriched over Ab, i.e. every hom-set is an Abelian group and composition is bilinear.
	}
	
	\begin{property}\index{zero!object}
		Let $A$ be a preadditive category and let $x\in\ob(A)$. The following statements are equivalent:
		\begin{itemize}
			\item $x$ is an initial object.
			\item $x$ is a final object.
			\item $\mathbbm{1}_x$ = 0
		\end{itemize}
		Any initial/terminal object is hence a zero object\footnote{See definition \ref{cat:zero_object}.}.
	\end{property}
	\begin{property}
		In a preadditive category the finite products $\prod_{i\in J}x_i$ are isomorphic to the finite coproducts $\bigsqcup_{i\in J} x_i$ (these are called \textbf{direct sums}). If a product $x\times y$ exists then so does the coproduct $x\sqcup y$ and if the coproduct exists then so does the product.
	\end{property}
	
	\newdef{Additive category}{
		A preadditive category in which all finite products (and hence coproducts) exist.
	}
	
	In a preadditive category one can define the classical notions from algebra such as images and kernels:
	\newdef{Kernel}{\index{kernel}
		Let $f:x\rightarrow y$ be a morphism. A\footnote{Note the word \textit{a}. The kernel of a morphism is determined up to an isomoprhism.} kernel is a morphism $k:z\rightarrow x$ such that:
		\begin{itemize}
			\item $f\circ k = 0$
			\item Universal property: for every other morphism $k':z'\rightarrow x$ there exists a unique morphism $h:z'\rightarrow z$ such that $k\circ h = k'$.
		\end{itemize}
	}
	\begin{notation}[Kernel]
		If the kernel of $f:x\rightarrow y$ exists then it is denoted by Ker$(f)\rightarrow x$.
	\end{notation}
	
	\newdef{Cokernel}{
		Let $f:x\rightarrow y$ be a morphism. A cokernel is a morphism $p:y\rightarrow z$ such that:
		\begin{itemize}
			\item $p\circ f = 0$
			\item Universal property: for every other morphism $p':y\rightarrow z'$ there exists a unique morphism $h:z\rightarrow z'$ such that $h\circ p = p'$.
		\end{itemize}
	}
	\begin{notation}[Kernel]
		If the cokernel of $f:x\rightarrow y$ exists then it is denoted by $y\rightarrow\text{Coker}(f)$.
	\end{notation}
	\remark{The name and notation of the kernel\footnote{Similarly for the cokernel.} (in the categorical sense) is explained by remarking that by the Yoneda lemma the morphism \[\text{Ker}(f)\rightarrow x\] represents the functor \[F:z\mapsto\ker\Big(\hom_C(z, x)\rightarrow\hom_C(z, y)\Big)\]}

\subsection{Monoidal categories}

	\newdef{Monoidal category\footnotemark}{\index{category!monoidal}\index{category!tensor}\index{tensor!product}
		\footnotetext{Sometimes called a \textbf{tensor} category.}
		A category $C$ equipped with a bifunctor \[-\otimes -:C\times C\rightarrow C\] called the \textbf{tensor product} or \textbf{monoidal product}, together with a distinct object $\mathbf{1}$, called the \textbf{unit object}, and 3 natural isomorphisms
		\begin{itemize}
			\item \textbf{Associator}: $\alpha_{a, b, c}:(a\otimes b)\otimes c\cong a\otimes(b\otimes c)$
			\item \textbf{Left unitor}: $\lambda_a:\mathbf{1}\otimes a\cong a$
			\item \textbf{Right unitor}: $\rho_a: a\otimes\mathbf{1}\cong a$
		\end{itemize}
		such that the following diagrams commute:
		
		\begin{figure}[ht!]
			\centering
			\begin{tikzpicture}
				\matrix (m) [matrix of math nodes,row sep=4em,column sep=4em,minimum width=2em, ampersand replacement=\&]{
					(a\otimes\mathbf{1})\otimes b\&\&a\otimes(\mathbf{1}\otimes b)\\
					\&a\otimes b\&\\
				};
				
				\draw[->] (m-1-1) -- (m-1-3) node[pos=0.5, above]{$\alpha_{a, \mathbf{1}, b}$};
				\draw[->] (m-1-1) -- (m-2-2) node[pos=0.5, below left]{$\rho_a\otimes\mathbbm{1}$};
				\draw[->] (m-1-3) -- (m-2-2) node[pos=0.5, below right]{$\mathbbm{1}\otimes\lambda_b$};
			\end{tikzpicture}
			\caption{Triangle diagram.}
			\label{fig:triangle_diagram}
		\end{figure}
		\begin{figure}[ht!]
			\centering
			\begin{tikzpicture}
				\matrix (m) [matrix of math nodes,row sep=4em,column sep=0.2em,minimum width=1em, ampersand replacement=\&]{
					\&((a\otimes b)\otimes c)\otimes d\&\&(a\otimes (b\otimes c))\otimes d\&\\
					(a\otimes b)\otimes(c\otimes d)\&\&\&\&a\otimes((b\otimes c)\otimes d)\\
					\&\&a\otimes(b\otimes (c\otimes d))\&\&\\
				};
				
				\draw[->] (m-1-2) -- (m-1-4) node[pos=0.5, above]{$\alpha_{a, b, c}\otimes\mathbbm{1}$};
				\draw[->] (m-1-2) -- (m-2-1) node[pos=0.5, above left]{$\alpha_{a\otimes b, c, d}$};
				\draw[->] (m-2-1) -- (m-3-3) node[pos=0.5, below left]{$\alpha_{a, b, c\otimes d}$};
				\draw[->] (m-1-4) -- (m-2-5) node[pos=0.5, above right]{$\alpha_{a, b\otimes c, d}$};
				\draw[->] (m-2-5) -- (m-3-3) node[pos=0.5, below right]{$\mathbbm{1}\otimes\alpha_{b, c, d}$};
			\end{tikzpicture}
			\caption{Pentagon diagram.}
			\label{fig:pentagon_diagram}
		\end{figure}
	}
	\begin{theorem}[MacLane's coherence theorem]\index{coherence!theorem}
		Let $C, D$ be monoidal categories. Any two natural transformations $\eta, \varepsilon:F\implies G$ constructed solely from the associator $\alpha$ and the unitors $\lambda,\rho$ coincide.
	\end{theorem}

	\newdef{Closed monoidal category}{
		A monoidal category $(C, \otimes, \mathbf{1})$ is said to be closed monoidal if for every object $b\in\ob(C)$ there exists a a right adjoint to the tensor product functor $-\otimes b:C\rightarrow C$, i.e.:
		\begin{equation}
			\forall a, c\in\ob(C):\exists b\Rightarrow c\in\ob(C):\hom_C(a\otimes b, c)\cong\hom_C(a, b\Rightarrow c)
		\end{equation}
		where the isomorphism is natural in $a, c\in\ob(C)$.
	}
	\begin{example}
		Every Cartesian closed category is closed monoidal where the monoidal structure is given by the Cartesian product. The internal Hom functor is given by the exponential object.
	\end{example}

	\newdef{Monoidal functor}{\index{monoidal!functor}\index{coherence!maps}\index{structure!morphisms}
		Let $(C, \otimes, \mathbf{1}_C), (D, \circledast, \mathbf{1}_D)$ be two monoidal categories. A functor $F:C\rightarrow D$ is said to be monoidal if there exists:
		\begin{itemize}
			\item A natural isomorphism $\psi_{a, b}: F(a)\circledast F(b)\implies F(a\otimes b)$ such that the following diagram commutes:
			\begin{figure}[ht!]
				\centering
				\begin{tikzpicture}
					\matrix (m) [matrix of math nodes,row sep=4em,column sep=4em, minimum width=1em, ampersand replacement=\&]{
						(Fa\circledast Fb)\circledast Fc\&Fa\circledast(Fb\circledast Fc)\\
						F(a\otimes b)\circledast Fc\&Fa\circledast F(b\otimes c)\\
						F\Big((a\otimes b)\otimes c\Big)\&F\Big(a\otimes (b\otimes c)\Big)\\
					};
					\draw[->] (m-1-1) -- (m-1-2) node[pos=0.5, above]{$\alpha_D$};
					\draw[->] (m-3-1) -- (m-3-2) node[pos=0.5, below]{$F\alpha_C$};
					
					\draw[->] (m-1-1) -- (m-2-1) node[pos=0.5, left]{$\psi_{a, b}\circledast\mathbbm{1}$};
					\draw[->] (m-2-1) -- (m-3-1) node[pos=0.5, left]{$\psi_{a\otimes b, c}$};
					\draw[->] (m-1-2) -- (m-2-2) node[pos=0.5, right]{$\mathbbm{1}\circledast\psi_{b, c}$};
					\draw[->] (m-2-2) -- (m-3-2) node[pos=0.5, right]{$\psi_{a, b\otimes c}$};
				\end{tikzpicture}
			\end{figure}

			\item An isomorphism $\phi: \mathbf{1}_D\rightarrow F\mathbf{1}_C$.
		\end{itemize}
	}
	\remark{The morphisms $\psi_{a, b}$ and $\phi$ are also called \textbf{coherence maps} or \textbf{structure morphisms}.}
	
	\begin{property}
		For every monoidal functor $F$ there exists a canonical isomorphism $\phi:\mathbf{1}_D\rightarrow F\mathbf{1}_C$ defined by following commutative diagram:
		\begin{figure}[ht!]
			\centering
			\begin{tikzpicture}
				\matrix (m) [matrix of math nodes,row sep=4em,column sep=4em, minimum width=1em, ampersand replacement=\&]{
					\mathbf{1}_D\circledast F\mathbf{1}_C\&F\mathbf{1}_C\\
					F\mathbf{1}_C\circledast F\mathbf{1}_C\&F(\mathbf{1}_C\otimes\mathbf{1}_C)\\
				};
				\draw[->] (m-1-1) -- (m-1-2) node[pos=0.5, above]{$\lambda_D$};
				\draw[->] (m-2-1) -- (m-2-2) node[pos=0.5, below]{$\psi_{\mathbf{1}_C, \mathbf{1}_C}$};
			
				\draw[->] (m-1-1) -- (m-2-1) node[pos=0.5, left]{$\phi\circledast\mathbbm{1}$};
				\draw[->] (m-2-2) -- (m-1-2) node[pos=0.5, right]{$F\lambda_C$};
			\end{tikzpicture}
		\end{figure}
		
		This isomorphism also makes following two diagrams commute for all $X\in\ob(C)$:
		\begin{figure}[ht!]
			\centering
			\begin{subfigure}[b]{0.49\textwidth}
				\centering
				\begin{tikzpicture}
					\matrix (m) [matrix of math nodes,row sep=4em,column sep=4em, minimum width=1em, ampersand replacement=\&]{
						Fa\circledast\mathbf{1}_D\&Fa\circledast F\mathbf{1}_C\\
						Fa\&F(a\otimes\mathbf{1}_C)\\
					};
					\draw[->] (m-1-1) -- (m-1-2) node[pos=0.5, above]{$\mathbbm{1}\circledast\phi$};
					\draw[->] (m-2-2) -- (m-2-1) node[pos=0.5, below]{$F\rho_C$};
				
					\draw[->] (m-1-1) -- (m-2-1) node[pos=0.5, left]{$\rho_D$};
					\draw[->] (m-1-2) -- (m-2-2) node[pos=0.5, right]{$\psi_{a, \mathbf{1}_C}$};
				\end{tikzpicture}
			\end{subfigure}
			\begin{subfigure}[b]{0.49\textwidth}
				\centering
				\begin{tikzpicture}
					\matrix (m) [matrix of math nodes,row sep=4em,column sep=4em, minimum width=1em, ampersand replacement=\&]{
						\mathbf{1}_D\circledast Fb\&F\mathbf{1}_C\circledast Fb\\
						Fb\&F(\mathbf{1}_C\otimes b)\\
					};
					\draw[->] (m-1-1) -- (m-1-2) node[pos=0.5, above]{$\phi\circledast\mathbbm{1}$};
					\draw[->] (m-2-2) -- (m-2-1) node[pos=0.5, below]{$F\lambda_C$};
				
					\draw[->] (m-1-1) -- (m-2-1) node[pos=0.5, left]{$\lambda_D$};
					\draw[->] (m-1-2) -- (m-2-2) node[pos=0.5, right]{$\psi_{\mathbf{1}_C, b}$};
				\end{tikzpicture}
			\end{subfigure}
		\end{figure}
	\end{property}
	
	\newdef{Lax monoidal functor}{\index{lax!monoidal functor}
		A lax monoidal functor is defined as a monoidal functor for which the coherence maps are merely morphisms and not isomorphisms.
	}
	\begin{notation}
		A (lax) monoidal functor $F$ together with its coherence maps $\psi_{a, b}$ and $\phi$ is sometimes denoted by the triple $(F, \phi, \psi)$ or just $(F, \psi)$.
	\end{notation}
	
	\newdef{Monoidal natural transformation}{
		A natural transformation $\eta$ between (lax) monoidal functors $(F, \psi, \phi_F)$ and $(G, \sigma, \psi_G)$ is said to be (lax) monoidal if it makes the following diagrams commute:
		\begin{figure}[ht!]
			\centering
			\begin{subfigure}[b]{0.49\textwidth}
				\centering
				\begin{tikzpicture}
					\matrix (m) [matrix of math nodes,row sep=4em,column sep=4em, minimum width=1em, ampersand replacement=\&]{
						\&\mathbf{1}_D\&\\
						F\mathbf{1}_C\&\&G\mathbf{1}_C\\
					};
					\draw[->] (m-1-2) -- (m-2-1) node[pos=0.5, above left]{$\phi_F$};
					\draw[->] (m-1-2) -- (m-2-3) node[pos=0.5, above right]{$\phi_G$};
					\draw[->] (m-2-1) -- (m-2-3) node[pos=0.5, below]{$\eta_{\mathbf{1}_C}$};
				\end{tikzpicture}
			\end{subfigure}
			\begin{subfigure}[b]{0.49\textwidth}
				\centering
				\begin{tikzpicture}
					\matrix (m) [matrix of math nodes,row sep=4em,column sep=4em, minimum width=1em, ampersand replacement=\&]{
						Fa\circledast Fb\&F(a\otimes b)\\
						Ga\circledast Gb\&G(a\otimes b)\\
					};
					\draw[->] (m-1-1) -- (m-1-2) node[pos=0.5, above]{$\psi_{a, b}$};
					\draw[->] (m-2-1) -- (m-2-2) node[pos=0.5, below]{$\sigma_{a, b}$};
				
					\draw[->] (m-1-1) -- (m-2-1) node[pos=0.5, left]{$\eta_a\circledast\eta_b$};
					\draw[->] (m-1-2) -- (m-2-2) node[pos=0.5, right]{$\eta_{a\otimes b}$};
				\end{tikzpicture}
			\end{subfigure}
			\caption{Monoidal natural transformation.}
			\label{fig:monoidal_natural_transformation}
		\end{figure}
	}
	
	\newdef{Monoidal equivalence}{\index{monoidal!equivalence}
		Two monoidal categories $C, D$ are monoidally equivalent if there exist monoidal functors $F:C\rightarrow D$ and $G:D\rightarrow C$ such that there exist monoidal natural isomorphisms $\eta:\mathbbm{1}_C\implies G\circ F$ and $\varepsilon:F\circ G\implies\mathbbm{1}_D$.
	}

	\newdef{Strict monoidal category}{
		A monoidal category is called strict if the associator $\alpha$ and the unitors $\lambda,\rho$ are identity morphisms.
	}
	\begin{theorem}[MacLane's strictness theorem]
		Every monoidal category is monoidally equivalent to a strict monoidal category.
	\end{theorem}
	
	\newdef{Dual object}{\index{dual!object}
		Let $(C, \otimes, \mathbf{1})$ be a monoidal category and let $a\in\ob(C)$. A left dual $a^*$ of $a$ is an object in $C$ together with two morphisms $\eta:\mathbf{1}\rightarrow a\otimes a^*$ and $\varepsilon:a^*\otimes a\rightarrow\mathbf{1}$, called the \textbf{unit} and \textbf{counit} morphisms\footnote{Also called the \textbf{coevaluation} and {evaluation} morphisms.}, such that the following diagrams commute:
		\begin{figure}[ht!]
			\centering
			\begin{tikzpicture}
				\matrix (m) [matrix of math nodes,row sep=4em,column sep=2em, minimum width=1em, ampersand replacement=\&]{
					\&\&a\&\&\\
					\mathbf{1}\otimes a\&\&\&\&a\otimes\mathbf{1}\\
					\&(a\otimes a^*)\otimes a\&\&a\otimes(a^*\otimes a)\&\\
				};
				\draw[->] (m-2-1) -- (m-1-3) node[pos=0.5, above left]{$\lambda$};
				\draw[->] (m-2-1) -- (m-3-2) node[pos=0.5, below left]{$\eta\otimes\mathbbm{1}$};
				\draw[->] (m-3-2) -- (m-3-4) node[pos=0.5, below]{$\alpha$};
				\draw[->] (m-3-4) -- (m-2-5) node[pos=0.5, below right]{$\mathbbm{1}\otimes\varepsilon$};
				\draw[->] (m-2-5) -- (m-1-3) node[pos=0.5, above right]{$\rho$};
			\end{tikzpicture}
		\end{figure}
		\begin{figure}[ht!]
			\centering
			\begin{tikzpicture}
				\matrix (m) [matrix of math nodes,row sep=4em,column sep=2em, minimum width=1em, ampersand replacement=\&]{
					\&\&a^*\&\&\\
					a^*\otimes\mathbf{1}\&\&\&\&\mathbf{1}\otimes a^*\\
					\&a^*\otimes(a\otimes a^*)\&\&(a^*\otimes a)\otimes a^*\&\\
				};
				\draw[->] (m-2-1) -- (m-1-3) node[pos=0.5, above left]{$\rho$};
				\draw[->] (m-2-1) -- (m-3-2) node[pos=0.5, below left]{$\mathbbm{1}\otimes\eta$};
				\draw[->] (m-3-2) -- (m-3-4) node[pos=0.5, below]{$\alpha^{-1}$};
				\draw[->] (m-3-4) -- (m-2-5) node[pos=0.5, below right]{$\varepsilon\otimes\mathbbm{1}$};
				\draw[->] (m-2-5) -- (m-1-3) node[pos=0.5, above right]{$\lambda$};
			\end{tikzpicture}
		\end{figure}
		If the object $a^*$, together with the morphisms $\eta, \varepsilon$, exists then $a^*$ is called the \textbf{left dual}\footnote{Analogously, $a$ is called the \textbf{right dual} of $a^*$.} of $a$ and $a$ is said to be \textbf{dualizable}.
	}
	\begin{property}
		In a braided monoidal category the left and right duals of an object coincide.
	\end{property}
	
	\newdef{Rigid category\footnotemark}{\index{category!rigid}\index{category!autonomous}
		\footnotetext{Also called an \textbf{autonomous category}.}
		A monoidal category in which all duals exist. If only left (resp. right) duals exist then the category is said to be left (resp. right) rigid.
	}
	\newdef{Compact closed category}{\index{category!compact closed}
		A symmetric rigid category is also called a compact closed category.
	}

\subsection{Higher category theory}\label{cat:higher_category_theory}

	\newdef{$n$-Category}{\index{n-category}
		A (strict) $n$-category consists of:
		\begin{itemize}
			\item Objects, called 0-morphisms.
			\item 1-morphisms going between 0-morphisms.
			\item ...
			\item $n$-morphisms going between $(n-1)$-morphisms.
		\end{itemize}
		such that the composition of $k$-morphisms ($\forall k\leq n$) is associative and satisfies the unit laws as required in an ordinary category.
	}
	\begin{example}[2-Category]
		In a 2-category one can compose 2-morphisms in two different ways:
		\begin{enumerate}
			\item Horizontal composition:
			Consider the 2-morphisms $\alpha:f\implies g$ and $\beta:f'\implies g'$ where $f'\circ f, g'\circ g$ are well-defined. These 2-morphisms can be composed as: \[\beta\circ\alpha: f'\circ f\implies g'\circ g\]
			\item Vertical composition:
			Consider the 2-morphisms $\alpha:f\implies g$ and $\beta:g\implies h$ where $f, g$ and $h$ have the same domain and codomain. These 2-morphisms can be composed as: \[\beta\cdot\alpha: f\implies h\]
		\end{enumerate}
		Furthermore, the horizontal and vertical composition should satisfy an interchange law:
		\begin{equation}
			(\alpha\cdot\beta)\circ(\gamma\cdot\delta) = (\alpha\circ\gamma)\cdot(\beta\circ\delta)
		\end{equation}
	\end{example}
	
	\remark{$n$-Morphisms are also often called \textbf{$n$-cells}.}
	
	\begin{example}
		The classical example of a 1-category is Set, the classical example of a 2-category is Cat.
	\end{example}

\subsection{Groupoids}

	\newdef{Groupoid}{\index{groupoid}
		A groupoid $\mathcal{G}$ is a small category in which all morphisms are invertible.
	}
	
	\newdef{2-Groupoid}{
		An 2-groupoid is a 2-category in which all 1-morphisms are invertible and every 2-morphisms has a 'vertical' inverse. The 'horizontal' inverse can be constructed from the other inverses.
	}
	\newdef{2-Group}{
		A 2-group is defined as a 2-groupoid with only one object. From this it follows that the set of 1-morphisms forms a group and so does the set of 2-morphisms under horizontal composition. The 2-morphisms do not form a group under vertical composition\footnote{Because the sources/targets may not match up.}.
	}


\section{Operad theory}
\subsection{Operads}

	\newdef{Plain operad\footnotemark}{\index{operad}\index{arity}
		\footnotetext{Also called a \textbf{non-symmetric operad} or \textbf{non-$\Sigma$ operad}.}
		Let $\mathcal{O} = \{P(n)\}_{n\in\mathbb{N}}$ be a sequence of sets, called \textbf{$n$-ary operations} ($n$ is the \textbf{arity}). The set $\mathcal{O}$ is called a plain operad if it satisfies following axioms:
		\begin{enumerate}
			\item $P(1)$ contains an identity element $\mathbbm{1}$.
			\item For all positive integers $n, k_1, ..., k_n$ there exists a composition
			\begin{equation}
				\circ:P(n)\times P(k_1)\times\cdots\times P(k_n)\rightarrow P(k_1+\cdots k_n):(\psi, \theta_1, ..., \theta_n)\mapsto \psi\circ(\theta_1, ..., \theta_n)
			\end{equation}
			that satisfies two additional axioms:
			\begin{itemize}
				\item Identity:
				\begin{equation}
					\theta\circ (\mathbbm{1}, ..., \mathbbm{1}) = \mathbbm{1}\circ\theta = \theta
				\end{equation}
				\item Associativity:
				\begin{align}
					\psi\circ\Big(\theta_1\circ&(\theta_{1, 1}, ..., \theta_{1, k_1}), ..., \theta_n\circ(\theta_{n, 1}, ..., \theta_{n, k_n})\Big)\nonumber\\
					&= \Big(\psi\circ(\theta_1, ..., \theta_n)\Big) \circ (\theta_{1,1}, ..., \theta_{1, k_1}, \theta_{2, 1}, ..., \theta_{n, k_n})
				\end{align}
			\end{itemize}
		\end{enumerate}
	}

        \newdef{$O$-algebra}{\index{algebra!over an operad}
                An object $X$ is called an algebra over an operad $O$ if there exist morphisms $O(n)\times X^n\rightarrow X$ for every $n\in\mathbb{N}$ satisfying the usual composition and identity laws.
        }
        \begin{example}[Categorical $O$-algebra]
                An $O$-algebra in the category Cat.
        \end{example}

\section{Topos theory}

\chapter{Categories, Operads and Topoi}
\section{Category theory}



\section{Operad theory}
\subsection{Operads}

	\newdef{Plain operad\footnotemark}{\index{operad}\index{arity}
		\footnotetext{Also called a \textbf{non-symmetric operad} or \textbf{non-$\Sigma$ operad}.}
		Let $\mathcal{O} = \{P(n)\}_{n\in\mathbb{N}}$ be a sequence of sets, called \textbf{$n$-ary operations} ($n$ is the \textbf{arity}). The set $\mathcal{O}$ is called a plain operad if it satisfies following axioms:
		\begin{enumerate}
			\item $P(1)$ contains an identity element $\mathbbm{1}$.
			\item For all positive integers $n, k_1, ..., k_n$ there exists a composition
			\begin{equation}
				\circ:P(n)\times P(k_1)\times\cdots\times P(k_n)\rightarrow P(k_1+\cdots k_n):(\psi, \theta_1, ..., \theta_n)\mapsto \psi\circ(\theta_1, ..., \theta_n)
			\end{equation}
			that satisfies two additional axioms:
			\begin{itemize}
				\item Identity:
				\begin{equation}
					\theta\circ (\mathbbm{1}, ..., \mathbbm{1}) = \mathbbm{1}\circ\theta = \theta
				\end{equation}
				\item Associativity:
				\begin{align}
					\psi\circ\Big(\theta_1\circ&(\theta_{1, 1}, ..., \theta_{1, k_1}), ..., \theta_n\circ(\theta_{n, 1}, ..., \theta_{n, k_n})\Big)\nonumber\\
					&= \Big(\psi\circ(\theta_1, ..., \theta_n)\Big) \circ (\theta_{1,1}, ..., \theta_{1, k_1}, \theta_{2, 1}, ..., \theta_{n, k_n})
				\end{align}
			\end{itemize}
		\end{enumerate}
	}


\section{Topos theory}

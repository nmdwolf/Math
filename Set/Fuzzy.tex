\chapter{Fuzzy Set Theory}

    We start this chapter with a small organizational remark: Although the content of the current chapter fits better in the parts on general set theory and logic, it does use more advanced concepts from for example topology and category theory. Furthermore, the main application (for us) is the characterization of uncertainty in statistics and machine learning. For that reason we decided to add this chapter here.

    The main reference for the basics on fuzzy sets is the original paper \cite{fuzzy}. For the basics on (ordered) sets see section \ref{section:ordered_sets} in the beginning of this compendium.

\section{Fuzzy sets}

    \newdef{Fuzzy set}{\index{fuzzy!set}\index{empty}
        Consider a set $X$ (this set corresponds to the universe of discours in e.g. type theory or category theory). A fuzzy subset of $X$ is a function $A:X\rightarrow[0,1]$. One can interpret the value $A(x)$ at a point $x\in X$ as the grade of membership of $x$ in $A$. If the function $A$ only takes on values in $\{0, 1\}$, one obtains the indicator function of an ordinary subset.

        A fuzzy set is said to be \textbf{empty} if its defining function is identically zero.
    }
    \remark{In fact one can generalize this definition by replacing $[0,1]$ by a more general poset (with the necessary properties).}

    \newdef{Pullback}{\index{pullback}
        Consider two sets $X,Y$ and a fuzzy subset $A$ of $Y$. Given a function $f:X\rightarrow Y$ one can define the pullback $f^*A$ as usual:
        \begin{gather}
            f^*A(x) := A(f(x)).
        \end{gather}
    }

    The following definition is an immediate generalization of definition \ref{set:relation}:
    \newdef{Fuzzy relation}{\index{fuzzy!relation}
        A fuzzy subset of the product set $X\times X$. This definition can be extended to $n$-ary relation by considering fuzzy subsets of the $n$-fold product $X\times\cdots\times X$.

        The composition in \ref{set:relational_composition} can be extended through the following formula:
        \begin{gather}
            S\circ R(x, z) := \sup_{y\in X}\min\Big(R(x, y), S(y, z)\Big).
        \end{gather}
    }

    A more exotic construction for fuzzy sets is the following one (note that this only works if the codomain of fuzzy sets is $[0, 1]$):
    \newdef{Convex combination}{\index{convex}
        Consider three fuzzy sets $A,B,\Lambda$. The convex combination $C\equiv(A, B; \Lambda)$ is defined as follows in analogy to \ref{linalgebra:convex}:
        \begin{gather}
            C(x) := \Lambda(x)A(x) + (1-\Lambda(x))B(x).
        \end{gather}
    }
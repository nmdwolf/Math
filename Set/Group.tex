\chapter{Algebra}

\section{Groups}

    	\newdef{Semigroup}{\index{semigroup}
        	Let $G$ be a set equipped with a binary operation $\star$. $(G,\star)$ is a semigroup if it satisfies following axioms:
		\begin{enumerate}
			\item $G$ is closed under $\star$
        	        \item $\star$ is asssociative
		\end{enumerate}
        }
        
        \newdef{Monoid}{\index{monoid}
        	Let $M$ be a set equipped with a binary operation $\star$. $(M,\star)$ is a monoid if it satisfies following axioms:
        	\begin{enumerate}
			\item $M$ is closed under $\star$
			\item $\star$ is associative
			\item $M$ contains an identity element with respect to $\star$
		\end{enumerate}
        }
        
        \newdef{Group}{\index{group}
        	Let $G$ be a set equipped with a binary operation $\star$. $(G,\star)$ is a group if it satisfies following axioms:
		\begin{enumerate}
			\item $G$ is closed under $\star$
			\item $\star$ is asssociative
			\item $G$ has an identity element with respect to $\star$
			\item Every element in $G$ has an inverse element with respect to $\star$
		\end{enumerate}
        }
        
        \newdef{Commutative group}{\index{commutativity}\index{Abel!Abelian group}
        	\nomenclature[S]{$\text{Ab}$}{The category of Abelian groups.}
        	Let $(G, \star)$ be a group. If $\star$ is commutative, then $G$ is called a commutative group or \textbf{Abelian group}.
        }
        
        \begin{construct}[Grothendieck completion]\index{Grothendieck!completion}\label{group:grothendieck_completion}
		Let the couple $(A, \boxplus )$ be an Abelian monoid. From this monoid one can construct an Abelian group $G(A)$, called the Grothendieck completion of $A$, as the quotient of $A\times A$ by the equivalence relation
		\begin{equation}
			(a_1, a'_1)\sim (a_2, a'_2) \iff \exists c\in A: a_1 +\boxplus  a'_2 \boxplus + c = a'_1 \boxplus + a_2 \boxplus + c
		\end{equation}
		
		The identity element is given by the equivalence class of $(0, 0)$, which will be denoted by 0. By the definition of $G(A)$, this class contains all elements $\alpha\in\Delta_A$. From this last remark it follows that $[(a, b)] + [(b, a)] = 0$ which implies that the additive inverse of $[(a, b)]$ is given by $[(b, a)]$.
        \end{construct}
        
        \begin{example}
        	The Grothendieck completion of the natural numbers $G(\mathbb{N})$ coincides with the additive group of integers $\mathbb{Z}$. The positive integers are then given by the equivalence classes $[(n, 0)]$ and the negative integers are given by the classes $[(0, n)]$.
        \end{example}
        
        \begin{uproperty}
        	Let $G(A)$ be the Grothendieck completion of $A$. For every monoid morphism $m:A\rightarrow B$ between an Abelian monoid and an Abelian group, there exists a group morphism $\varphi:G(A)\rightarrow B$.
        \end{uproperty}

\subsection{Cosets}

        \newdef{Coset}{\index{coset}\index{normal!subgroup}\index{invariant!subgroup|see{normal subgroup}}\label{group:coset}
        	Let $G$ be a group and $H$ a subgroup of $G$. The left coset of $H$ with respect to $g\in G$ is defined as the set
            \begin{equation}
            	gH = \{gh: h\in H\}
            \end{equation}
            The right coset is analogously defined as $Hg$. If for all $g\in G$ the left and right cosets coincide then the subgroup $H$ is said to be a \textbf{normal subgroup}\footnotemark. The sets of left and right cosets are denoted by $G/H$ and $H\backslash G$ respectively.
            \footnotetext{Also called a \textbf{normal divisor} or \textbf{invariant subgroup}.}
        }
        
        \newadef{Normal subgroup}{
        	Let $G$ be a group with subgroup $H$. Consider the conjugate classes $gHg^{-1}$ for all $a\in G$. If all classes coincide with $H$ itself, then $H$ is called a normal subgroup.
        }
        
        \begin{notation}
        	Let $N$ be a normal subgroup of $G$. This is often denoted by $N\vartriangleleft G$.
        \end{notation}
        
        \newdef{Quotient group}{\index{quotient!group}\label{group:quotient_group}
        	Let $G$ be a group and $N$ a normal subgroup. The coset space\footnotemark\ $G/N$ can be turned into a group by equipping it with a product such that the product of $aN$ and $bN$ is $(aN)(bN)$. The fact that $N$ is a normal subgroup can be used to rewrite this as $(aN)(bN) = (ab)N$.
        	\footnotetext{Left and right cosets coincide for normal subgroups.}
        }
        
        \newdef{Center}{\index{center}
        	The center of a group is defined as follows:
            \begin{equation}
            	\label{group:center}
                Z(G) = \{z\in G: \forall g\in G , zg = gz\}
            \end{equation}
            This set is a normal subgroup of $G$.
        }
        
\subsection{Abelianization}

	\newdef{Commutator subgroup\footnotemark}{\index{commutator}
		\footnotetext{Also called the \textbf{derived subgroup}.}
		The commutator subgroup $[G, G]$ of $G$ is defined as the group generated by the elements \[[g, h] = g^{-1}h^{-1}gh\]
		for all $g, h\in G$. This group is a normal subgroup of $G$.
	}
	\begin{property}
		$G$ is Abelian if and only if $[G, G]$ is trivial.
	\end{property}
	
	\newdef{Abelianization}{\index{Abelianization}
		The quotient group $G/[G, G]$ is an Abelian group, called the Abelianization of $G$.
	}
	
	\begin{property}
		A quotient group $G/H$ is Abelian if and only if $[G, G]\leq H$.
	\end{property}

\subsection{Order}
        
        \newdef{Order of a group}{\index{order}
        	The number of elements in the group. It is denoted by $|G|$ or $\text{ord}(G)$.
        }
        \newdef{Order of an element}{
        	The order of an element $a\in G$ is the smallest integer $n$ such that
        	\begin{equation}
        		a^n = e
        	\end{equation}
        	where $e$ is the identity element of $G$.
        }
        
        \newdef{Torsion group}{\index{torsion}\label{group:torsion_group}
        	A torsion group is a group for which all element have finite order. The torsion set $\text{Tor}(G)$ of a group $G$ is the set of all elements $a\in G$ that have finite order. For Abelian groups, $\text{Tor}(G)$ is a subgroup.
        }
        
        \begin{theorem}[Lagrange]\index{Lagrange!theorem on finite groups}
        	Let $G$ be a finite group with subgroup $H$. Then $|H|$ is a divisor of $|G|$.
        \end{theorem}
        \begin{result}
        	The order of any element $g\in G$ is a divisor of $|G|$.
        \end{result}


\subsection{Symmetric and alternating groups}

	\newdef{Symmetric group}{
        	The symmetric group $S_n$ or $\text{Sym}_n$ of the set $V = \{1, 2, ..., n\}$ is defined as the set of all permutations of $V$. The number $n$ is called the \textbf{degree} of the symmetric group. The symmetric group $\text{Sym}(X)$ of a finite set $X$ is analogously defined.
        }
        
        \begin{theorem}[Cayley's theorem]\index{Cayley's theorem}
        	Every finite group is isomorphic to a subgroup of $S_n$ where $n=|G|$.
        \end{theorem}
        
        \newdef{Alternating group}{
        	The alternating group $A_n$ is the subgroup of $S_n$ containing all even permutations.
        }
        
        \newdef{Cycle}{\index{cycle}
        	A $k$-cycle is a permutation of the form $(a_1\ a_2\ ...\ a_k)$ sending $a_i$ to $a_{i+1}$ (and $a_k$ to $a_1$). A \textbf{cycle decomposition} of an arbitrary permutation is the decomposition into a product of disjoint cycles.
        }
        \begin{formula}
        	Let $\tau$ be a $k$-cycle. Then $\tau$ is $k$-cyclic (hence the name \textit{cycle}):
            	\begin{equation}
            		\tau^k = \mathbbm{1}_G
	        \end{equation}
        \end{formula}
        \begin{example}
        	Consider the set $\{1, 2, 3, 4, 5, 6\}$. The permutation $\sigma:x\mapsto x+2\ (\text{mod } 6)$ can be written using the cycle decomposition $\sigma = (1\ 3\ 5)(2\ 4\ 6)$.
        \end{example}
        
        \newdef{Transposition}{\index{transposition}
        	A permutation which exchanges two elements but lets the other ones unchanged.
        }
        
\subsection{Group presentations}

	\newdef{Relations}{\index{relation}
		Let $G$ be a group. If the product of a number of elements $g\in G$ is equal to the identity $e$ then this product is called a relation on $G$.
	}
	\newdef{Complete set of relations}{
		Let $H$ be a group generated by a subgroup $G$. Let $R$ be a set of relations on $G$. If $H$ is uniquely (up to an isomorphism) determined by $G$ and $R$ then the set of relations is said to be complete.
	}

	\newdef{Presentation}{\index{presentation}
		Let $H$ be a group generated by a subgroup $G$ and a complete set of relations $R$ on $G$. The pair $(G, R)$ is called a presentation of $H$.
		
		It is clear that every group can have many different presentations and that it is (very) difficult to tell if two groups are isomorphic by just looking at their presentations.
	}
	\begin{notation}
		The presentation of a group $G$ is often denoted by $\langle S|R \rangle$, where $S$ is the set of generators and $R$ the set of relations.
	\end{notation}

\subsection{Direct product}

	\newdef{Direct product}{\index{direct product! of groups}\label{group:direct_product}
		Let $G, H$ be two groups. The direct product $G\otimes H$ is defined as the set-theoretic Cartesian product $G\times H$ equipped with a binary operation $\cdot$ such that:
		\begin{equation}
			(g_1, h_1)\cdot(g_2, h_2) = (g_1g_2, h_1h_2)
		\end{equation}
		where the operations on the right hand side are the group operations in $G$ and $H$. The structure $G\otimes H = (G\times H, \cdot)$ forms a group.
	}
	\remark{This definition can be generalized to any number of groups, even infinity (where one has to replace the $n$-tuples by infinite Cartesian products).}
	
	\newdef{Weak direct product}{\index{direct sum!of groups}
		Consider the direct product of a number (finite or infinite) of groups. The subgroup consisting of all elements for which all components, except finitely many, are the identity is called the weak direct product or, in the case of Abelian groups, the \textbf{direct sum}.
	}
	\begin{notation}
		The direct sum is often denoted by $\oplus$, in accordance with the notation for vector spaces (and other algebraic structures).
	\end{notation}
	\remark{For a finite number of groups, the direct product and direct sum coincide.}
	
	\newdef{Inner semidirect product}{\index{split}
		Let $G$ be a group, $H$ a subgroup of $G$ and $N$ a normal subgroup of $G$. $G$ is said to be the inner semidirect product of $H$ and $N$, denoted by $N\rtimes H$, if it satifies the following equivalent statements:
		\begin{itemize}
			\item $G = NH$ where $N\cap H = \{e\}$.
			\item For every $g\in G$ there exist unique $n\in N, h\in H$ such that $g=nh$.
			\item For every $g\in G$ there exist unique $h\in H, n\in N$ such that $g=hn$.
			\item There exists a group morphism $\rho:G\rightarrow H$ which satisfies $\rho|_H = e$ and $\ker(\rho)=N$.
			\item The composition of the natural embedding $i:H\rightarrow G$ and the projection $\pi:G\rightarrow G/N$ is an isomorphism between $H$ and $G/N$.
		\end{itemize}
		$G$ is also said to \textbf{split} over $N$.
	}
	\begin{property}
		If both $H$ and $N$ are normal in the above definition, the inner semidirect product coincides with the direct product. For a \underline{finite} number of groups $\{G_i\}$ we see that the direct product is generated by the elements of the groups $G_i$.
		
		If the subgroups $H$ and $N$ have presentations $\langle S_H|R_H \rangle$ and $\langle S_N|R_N \rangle$ then the (inner) direct product is given by:
		\begin{equation}
			\label{group:direct_product_presentation}
			H\ast N = \langle S_H\cup S_N|R_H\cup R_N \cup R_C \rangle
		\end{equation}
		where $R_C$ is the set of relations that give the commutativity of $H$ and $N$.
	\end{property}
	
	\newdef{Outer semidirect product}{
		Let $G, H$ be two groups and let $\varphi:H\rightarrow\text{Aut}(G)$ be a group morphism. The outer semidirect product $G\rtimes_\varphi H$ is defined as the set-theoretic Cartesian product $G\times H$ equipped with a binary relation $\cdot$ such that:
		\begin{equation}
			(g_1, h_1)\cdot(g_2, h_2) = (g_1\varphi(h_1)(g_2), h_1h_2)
		\end{equation}
		The structure $(G\rtimes_\varphi H, \cdot)$ forms a group.
		
		By noting that the set $N = \{(g, e_H)|g\in G\}$ is a normal subgroup isomorphic to $G$ and that the set $B = \{(e_G, h)|h\in H\}$ is a subgroup isomorphic to $H$, we can also construct the outer semidirect product $G\rtimes_\varphi H$ as the inner semidirect product $N\rtimes B$.
	}
	
	\begin{remark}
		The direct product of groups is a special case of the outer semidirect product where the group morphism is given by the trivial map $\varphi:h\mapsto e_G$.
	\end{remark}
	
\subsection{Free product}

	\newdef{Free product}{\index{free!product}
		Consider two groups $G, H$. The free group $G\ast H$ is deifned as the set consisting of all words composed of all elements in $G$ and $H$ together with the concatenation (and reduction\footnote{Two elements of the same group, written next to eachother, are replaced by their product.}) as multiplication. Due to the reduction, every element in $G\ast H$ is of the form $g_1h_1g_2h_2...$.
	}
	\remark{For non-trivial groups the free product is always infinite.}
	
	\begin{property}
		If the groups $G$ and $H$ have presentations $\langle S_G|R_G \rangle$ and $\langle S_H|R_H \rangle$ then the free product is given by:
		\begin{equation}
			G\ast H = \langle S_G\cup S_H|R_G\cup R_H \rangle
		\end{equation}
		From \ref{group:direct_product_presentation} we see that the free product is a generalization of the direct product. 
	\end{property}
	
	\newdef{Free product with amalgamation}{\index{amalgamation}
		Consider the groups $F, G, H$ and two group morphisms $\phi:F\rightarrow G$ and $\psi:F\rightarrow H$. The free product with amalgamation $G\ast_F H$ is defined by adding the following set of relations to the presentation of the free product $G\ast H$:
		\begin{equation}
			\{\phi(f)\psi(f)^{-1} = e:f\in F\}
		\end{equation}
		Alternatively, the free product with amalgamation can be constructed as
		\begin{equation}
			G\ast_F H = (G\ast H) / N_F
		\end{equation}
		where $N_F$ is the normal subgroup generated by elements of the form $\phi(f)\psi(f)^{-1}$.
	}

\subsection{Free groups}

	\newdef{Free Abelian group}{\index{free!group}\index{basis}\index{rank}
		An abelian group $G$ with generators $\{g_i\}_{i\in I}$ is said to be freely generated if every element $g\in G$ can be uniquely written as a formal linear combination of the generators:
		\begin{equation}
			G = \left\{\left.\sum_ia_ig_i\right|a_i\in\mathbb{Z}\right\}
		\end{equation}
		The set of generators $\{g_i\}_{i\in I}$ is then called a \textbf{basis}\footnote{In analogy with the basis of a vector space.}\ of $G$. The number of elements in the basis is called the \textbf{rank} of $G$.
	}
	\begin{property}
		Consider a free group $G$. Let $H\subset G$ be a subgroup. Then $H$ is also free.
	\end{property}
	
	\begin{theorem}\index{torsion}\label{group:theorem:free_group}
		Let $G$ be a finitely generated Abelian group of rank $n$. This group can be constructed in two different ways:
		\begin{equation}
			G = F/H
		\end{equation}
		where both $F, H$ are free and finitely generated Abelian groups. The second decomposition is:
		\begin{equation}
			G = A\oplus T\qquad\text{where}\qquad T = Z_{h_1}\oplus\cdots\oplus Z_{h_m}
		\end{equation}
		where $A$ is a free and finitely generated group of rank $n-m$ and all $Z_{h_i}$ are cyclic groups of order $h_i$. The group $T$ is called the \textbf{torsion subgroup}\footnote{See also definition \ref{group:torsion_group}.}.
	\end{theorem}
	
	\begin{property}
		The rank $n-m$ and the numbers $h_i$ from previous theorem are unique.
	\end{property}

\subsection{Group actions}

        \newdef{Group morphism}{\index{morphism!of groups}
        	A group morphism $\Phi:G\rightarrow H$ is a map satisfying $\forall g, h \in G$
		\begin{equation}
            		\Phi(gh) = \Phi(g)\Phi(h)
	        \end{equation}
        }
        
        \newdef{Kernel}{\index{kernel}
        	The kernel of a group morphism $\Phi:G\rightarrow H$ is defined as the set
        	\begin{equation}
            		K = \{g\in G: \Phi(g) = \mathbbm{1}_H\}
	        \end{equation}
        }
        
        \begin{theorem}[First isomorphism theorem]\index{isomorphism!theorem}\label{group:theorem:first_isomorphism_theorem}
        	Let $G, H$ be a groups and let $\varphi:G\rightarrow H$ be a group morphism. If $\varphi$ is surjective than $G/\ker\varphi\cong H$.
        \end{theorem}
        
        \newdef{Group action}{\index{group!action}\label{group:group_action}
        	Let $G$ be a group. Let $V$ be a set. A map $\rho: G\times V \rightarrow V$ is called an action of $G$ on $V$ if it satisfies the following conditions:
        	\begin{itemize}
                	\item Identity: $\rho(\mathbbm{1}_G, v) = v$
                	\item Compatibility: $\rho(gh, v) = \rho(g, \rho(h, v))$
		\end{itemize}
		For all $g, h \in G$ and $v\in V$. The set V is called a (left) \textbf{G-space}.
        }
        \begin{remark}\label{group:permutation_remark}
        	A group action can alternatively be defined as a group morphism from $G$ to $\text{Sym}(V)$. It assigns a permutation of $V$ to every element $g\in G$.
	\end{remark}
        
        \begin{notation}
        	The action $\rho(g, v)$ is often denoted by $g\cdot v$ or even $gv$.
        \end{notation}
        
	\newdef{Orbit}{\index{orbit}
		The orbit of an element $x\in X$ with respect to a group $G$ is defined as the set:
		\begin{equation}
			\label{group:orbit}
			G\cdot x = \{g\cdot x|g\in G\}
		\end{equation}
		The relation $p\sim q \iff \exists g\in G: p = g\cdot q$ induces an equivalence relation for which the equivalence classes coincide with the orbits of $G$. The set of equivalence classes $X/\sim$ (sometimes denoted by $X/G$) is called the \textbf{orbit space}.
	}
	\newdef{Stabilizer}{\index{stabilizer}\index{isotropy group}
		The stabilizer group or \textbf{isotropy group} of an element $x\in X$ with respect to a group $G$ is defined as the set:
		\begin{equation}
			G_x = \{g\in G|g \cdot x = x\}
		\end{equation}
		This is a subgroup of $G$.
	}
	\begin{theorem}[Orbit-stabilizer theorem]
		Let $G$ be a group acting on a set $X$. Let $G_x$ be the stabilizer of some $x\in X$. The following relation holds:
		\begin{equation}
			|G\cdot x||G_x| = |G|
		\end{equation}
	\end{theorem}
	
	\newdef{Free action}{\index{free}\label{group:free_action}
		A group action is free if $g\cdot x = x$ implies $g = e$ for every $x\in X$. Equivalently, a group action is free if the stabilizer group of all elements is trivial.
	}
	\newdef{Faithful action}{\index{faithful!action}\index{effective!action|see{faithful}}\label{group:faithful_action}
		A group action is faithful or \textbf{effective} if the morphism $G\rightarrow\text{Sym}(X)$ is injective. Alternatively, a group action is faithful if for every two group elements $g, h\in G$ there exists an element $x\in X$ such that $g\cdot x\neq h\cdot x$.
	}
	
	\newdef{Transitive action}{\index{transitive!action}\label{group:transitive}
		A group action is transitive if for every two elements $x, y\in X$ there exists a group element $g\in G$ such that $g\cdot x = y$. Equivalently we can say that there is only one orbit.
	}
	\newdef{Homogeneous space}{\index{homogeneous!space}
		If the group action of a group $G$ on a $G$-space $X$ is transitive, then $X$ is said to be a homogeneous space.
	}

	\begin{property}[$\dag$]\label{group:transitive_action_property}
		Let $X$ be a set and let $G$ be a group such that the action of $G$ on $X$ is transitive. Then their exists a bijection $X\cong G/G_x$ where $G_x$ is the stabilizer of any element $x\in X$.
	\end{property}
	
	\newdef{Principal homogenous space}{\index{torsor}\label{group:torsor}
		If the group action of a group $G$ on a homogeneous space $X$ is also free, then $X$ is said to be a principal homogeneous space or \textbf{$G$-torsor}.
	}
        
        \newdef{G-module}{\index{module}
        	Let $G$ be a group. Let $M$ be a commutative group. $M$ equipped with a left group action $\varphi:G\times M\rightarrow M$ is a (left) G-module if $\varphi$ satisfies the following equation (distributivity):
		\begin{equation}
            		\label{group:g_module}
                	g\cdot(a+b) = g\cdot a + g\cdot b
	        \end{equation}
        	where $a, b\in M$ and $g\in G$.
        }
        \newdef{G-module morphism}{\index{morphism!of G-modules}\index{equivariant}\label{group:equivariant}
        	A G-module morphism is a map $f:V\rightarrow W$ satisfying
	        \begin{equation}
        	    	g\cdot f(v) = f(g\cdot v)
        	\end{equation}
        	where the $\cdot$ symbol represents the group action in $W$ and $V$ respectively. It is sometimes called a \textbf{G-map}, a \textbf{G-equivariant map} or an \textbf{intertwining map}.
        }
        
        \newdef{Crossed module}{\index{module!crossed}\label{group:crossed_module}
        	A crossed module is a quadruple $(G, H, t, \rho)$ where:
        	\begin{itemize}
        		\item $G$, $H$ are two groups.
        		\item $t:H\rightarrow G$ is a group morphisms.
        		\item $\alpha:G\rightarrow\text{Aut}(H)$, i.e. $G$ acts by automorphisms on $H$.
        	\end{itemize}
        	These structures satisfy two compatibility conditions:
        	\begin{itemize}
        		\item $t$ is $G$-equivariant:
        		\begin{equation}
        			t(\alpha(g)h) = gt(h)g^{-1}
        		\end{equation}
        		\item \textbf{Peiffer identity}:
        		\begin{equation}
        			\alpha(t(h))h' = hh'h^{-1}
        		\end{equation}
        	\end{itemize}
        }
        
\section{Rings}
	
	\newdef{Ring}{\index{ring}
		Let $R$ be a set equipped with two binary operations $+,\cdot$ (called addition and multiplication). $(R,+,\cdot)$ is a ring if it satisfies the following axioms:
    		\begin{enumerate}
			\item $(R,+)$ is a commutative group.
			\item $(R,\cdot)$ is a monoid.
			\item Multiplication is distributive with respect to addition.
		\end{enumerate}
	}
	
	\newdef{Unit}{\index{unit}
		An invertible element of ring $(R, +, \cdot)$. The set of units forms a group under multiplication.
	}
	
	\begin{construct}[Localization]\index{localization}
		Let $R$ be a commutative ring and let $S$ be a multiplicative monoid in $R$. We first define an equivalence relation $\sim$ on $R\times S$ in the following way:
		\begin{equation}
			(r_1, s_1)\sim(r_2, s_2) \iff \exists t\in S: t(r_1s_2 - r_2s_1) = 0
		\end{equation}
		
		The set $R^* = (R\times S)/\sim$, called the localization of $R$ with respect to $S$, can now be turned into a ring by defining an addition and a multiplication. By writing $(r, s)\in R^*$ as the formal fraction $\frac{r}{s}$ we obtain the familiar operations of fractions:
		\begin{itemize}
			\item $\displaystyle\frac{r_1}{s_1} + \frac{r_2}{s_2} = \frac{r_1s_2 + r_2s_1}{s_1s_2}$
			\item $\displaystyle\frac{r_1}{s_1}\cdot\frac{r_2}{s_2} = \frac{r_1r_2}{s_1s_2}$
		\end{itemize}
	\end{construct}
	\remark{The localization of $R$ with respect to the monoid $S$ can be interpreted as the ring obtained by collapsing $S$ into a single unit of $R$.}
	
	\begin{notation}
		The localization of $R$ with respect to $S$ is often denoted by $S^{-1}R$.
	\end{notation}

\subsection{Ideals}\index{ideal}

    	\newdef{Ideal}{\label{linalgebra:ideal}
    		Let $(R,+,\cdot)$ be a ring with $(R,+)$ its additive group. A subset $I\subseteq R$ is called an ideal\footnotemark\ of $R$ if it satisfies the following conditions:
        	\begin{enumerate}
			\item $(I,+)$ is a subgroup of $(R,+)$
                	\item $\forall n\in I, \forall r\in R:(n\cdot r), (r\cdot n)\in I$
		\end{enumerate}
		\footnotetext{More generally: two-sided ideal}
        }
        
        \newdef{Unit ideal}{Let $(R,+,\cdot)$ be a ring. $R$ itself is called the unit ideal.}
        \newdef{Proper ideal}{Let $(R,+,\cdot)$ be a ring. A subset $I\subset R$ is said to be a proper ideal if it is an ideal of $R$ and if it is not equal to $R$.}
        \newdef{Prime ideal}{
        	Let $(R,+,\cdot)$ be a ring. A proper ideal $I$ is a prime ideal if for any $a,b\in R$ the following relation holds:
        	\begin{equation}
        		ab\in I\implies \text a\in I \vee b\in I
        	\end{equation}
        }
        \newdef{Maximal ideal}{Let $(R,+,\cdot)$ be a ring. A proper ideal $I$ is said to be maximal if there exists no other proper ideal $T$ in R such that $I\subset T$.}
        \newdef{Minimal ideal}{A proper ideal is said to be minimal if it contains no other nonzero ideal.}
        
	\begin{construct}[Generating set of an ideal]\index{generating set! of an ideal}\label{group:generating_set_ideal}
		Let $R$ be a ring and let $X$ be a subset of $R$. The two-sided ideal generated by $X$ is defined as the intersection of all two-sided ideals containing $X$. An explicit construction is given by:
		\begin{equation}
			I = \left\{\left.\sum_{i=1}^n l_ix_ir_i\ \right\vert\ \forall n\in\mathbb{N}:\forall l_i, r_i\in R\text{ and } x_i\in X\right\}
		\end{equation}
		Left and right ideals are generated in a similar fashion.
	\end{construct}
	\begin{construct}[Extension]\index{extension}
        	Let $I$ be an ideal of a ring $R$ and let $\iota:R\rightarrow S$ be a ring morphism. The extension of $I$ with respect to $\iota$ is the ideal generated by the set $\iota(I)$.
        \end{construct}
	
	\newdef{Local ring}{\index{local!ring}\label{algebra:local_ring}
		A local ring is a ring for which a unique maximal left ideal exists.\footnote{This also implies that there exists a unique maximal right ideal and these ideals coincide.}
	}
	\begin{property}
		The localization of a ring $R$ with respect to a prime ideal $P$ is a local ring, where the maximal ideal is the extension of $P$ with respect to the ring morphism $\iota:R\rightarrow R^*$
	\end{property}
	
	\newdef{Residue field}{\index{residue!field}
		Consider a commutative unital ring $R$ and let $I$ be a maximal ideal. The quotient ring $R/I$ forms a field, called the residue field. 
	}

\subsection{Modules}
	
	\newdef{$R$-Module}{\index{module}
		Let $(R, +, \cdot)$ be a ring. A set $X$ is an $R$-module if it satisfies the same axioms as those of a vector space \ref{linalgebra:vector_space} but where the scalars are only elements of a ring instead of a field.
	}
	
	\begin{property}\label{algebra:module_basis}
		For a general $R$-module the existence of a basis is not guaranteed unless $R$ is a division ring. See construction \ref{linalgebra:hamel_basis} to see how this basis can be constructed.
	\end{property}
	\begin{result}
		As every field is in particular a division ring, the existence of a basis follows from the above property for $R$-modules.
	\end{result}
	
	\newdef{Free module}{\index{free!module}
		A module is said to be free if it admits a basis.
	}
	
	\newdef{Projective module}{\index{projective!module}
		A module $P$ is said to be projective if:
		\begin{equation}
			P\oplus M = F
		\end{equation}
		where $M$ is a module and $F$ is a free module.
	}

\subsection{Graded rings}
	
	\newdef{Graded ring}{\index{graded}\label{group:graded_ring}
		Let $R$ be a ring that can be written as the direct sum of Abelian groups $A_k$:
		\begin{equation}
			R = \bigoplus_{k\in\mathbb{N}}A_k
		\end{equation}
		If $R$ has the property that for every $i, j\in\mathbb{N}: A_i\star A_j\subseteq A_{i+j}$, where $\star$ is the ring multiplication, then $R$ is said to be a graded ring. The elements of the space $A_k$ are said to be \textbf{homogeneous of degree $k$}.
	}
	
	\newformula{Graded commutativity}{\index{commutativity!graded}
		Let $m = \deg v$ and let $n = \deg w$. If
		\begin{equation}
			\label{group:graded_commutativity}
			vw = (-1)^{mn}wv
		\end{equation}
		for all elements $v, w$ of the graded ring then it is said to be a graded-commutative ring.
	}

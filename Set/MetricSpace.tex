\chapter{Metric spaces}
\section{General definitions}
    	
    	\newdef{Metric}{\index{metric}\index{distance}\label{topology:metric}
        	A metric (or distance) on a set M is a map $d: M\times M\rightarrow\mathbb{R}^+$ that satisfies the following properties:
	        \begin{itemize}
			\item Non-degeneracy: $d(x,y) = 0 \iff x = y$
        	        \item Symmetry: $d(x,y) = d(y,x)$
        	        \item Triangle inequality: $d(x,z) \leq d(x,y) + d(y,z)$\quad $,\forall x,y,z\in M$
		\end{itemize}
        }
        \newdef{Metric space}{
        	A set $M$ equipped with a metric $d$ is called a metric space and is denoted by $(M,d)$.
        }
        
        \newdef{Diameter}{\index{diameter}
        	The diamater of a subset $U\subset M$ is defined as
            \begin{equation}
            	\text{diam}(U) = \sup_{x, y\in U}d(x, y)
            \end{equation}
        }
        \newdef{Bounded}{
        	A subset $U\subseteq M$ is bounded if $\text{diam}(U) < +\infty$.
        }
        
       	\begin{property}
        	Every metric space is a topological space\footnote{See next chapter.}.
        \end{property}
        Multiple topological notions can be reformulated in terms of a metric. The most important of them are given below:
        \newdef{Open ball}{\index{ball}
        	An open ball centered on a point $x_0\in M$ with radius $R>0$ is defined as the set:
		\begin{equation}
			\label{topology:open_ball}
	                \boxed{B(x_0,R) = \{x\in M : d(x,x_0) < R\}}
		\end{equation}
        }
        \newdef{Closed ball}{
        	The closed ball $\overline{B}(x_0,R)$ is defined as the union of the open ball $B(x_0,R)$ and its boundary, i.e. $\overline{B}(x_0,R) = \{x\in M:d(x,x_0) \leq R\}$.
	}
        
        \newdef{Interior point/neighbourhood}{\index{neighbourhood}\index{interior!point}
        	Let $N$ be a subset of $M$. A point $x\in N$ is said to be an interior point of $N$ if there exists an $R>0$ such that $B(x, R)\subset M$. Furthermore, $N$ is said to be a neighbourhood of $x$.
        }
        
        \newdef{Open set}{\index{open}
        	A subset $N\subset M$ is said to be open if every point $x\in N$ is an interior point of $N$.
        }
        \newdef{Closed set}{\index{closed}
        	A subset $V\subset M$ is said to be closed if its complement is open.
	}
        
        \newdef{Limit point}{\index{limit!point}
        	Let $S$ be a subset of $X$. A point $x\in X$ is called a limit point of $S$ if every neighbourhood of $x$ contains at least one point of $S$ different from $x$.
        }
        \newdef{Accumulation point}{
        	Let $x\in X$ be a limit point of $S$. Then $x$ is an accumulation point of $S$ if every open neighbourhood of $x$ contains infinitely many points of $S$. 
        }
        
        \newdef{Convergence}{\index{convergence}
        	A sequence $(x_n)_{n\in\mathbb{N}}:\mathbb{N}\rightarrow M$ in a metric space $(M, d)$ is said to be convergent to a point $a\in M$ if:
            \begin{equation}
				\label{topology:convergence}
                \forall\varepsilon>0:\exists N_0\in\mathbb{N}:\forall n\geq N_0:d(x_n,a)<\varepsilon
			\end{equation}
        }
        \newdef{Continuity}{\index{continuity}
        	Let $(M, d)$ and $(M',d')$ be two metric spaces. A function $f:M\rightarrow M'$ is said to be continuous at a point $a\in$ dom$(f)$ if:
            	\begin{equation}
			\label{metric:continuity}
                	\forall\varepsilon>0:\exists\delta_\varepsilon:\forall x\in\text{dom}(f):d(a,x)<\delta_\varepsilon\implies d'(f(a),f(x))<\varepsilon
		\end{equation}
        }
	\begin{property}
		Let $(M, d)$ be a metric space. The distance function $d:M\times M\rightarrow\mathbb{R}$ is a continuous function.
	\end{property}
	
	\newdef{Uniform continuity}{\index{uniform!continuity}
		Let $(M, d)$ and $(M',d')$ be two metric spaces. A function $f:M\rightarrow M'$ is said to be uniformly continuous if:
            	\begin{equation}
			\label{metric:uniform_continuity}
                	\forall\varepsilon>0:\exists\delta_\varepsilon:\forall x, y\in\text{dom}(f):d(x,y)<\delta_\varepsilon\implies d'(f(x),f(y))<\varepsilon
		\end{equation}
		This is clearly a stronger notion than that of continuity as the number $\varepsilon$ is equal for all points $y\in\text{dom}(f)$.
	}
        
\section{Examples of metrics}

	\newdef{Product space}{\index{product!space}
		Consider the cartesian product \[M = M_1\times M_2\times ... \times M_n\] with $\forall n:(M_n,d_n)$ a metric space. If equipped with the distance function $d(x,y) = \underset{1\leq i\leq n}{\max}\ d_i(x_i,y_i)$ this product is also a metric space. It is called the product metric space.
	}
	\begin{property}\index{projection}
		The projection associated with the set $M_j$ is defined as:
		\begin{equation}
			\label{metric:projection}
			\text{pr}_j:M\rightarrow M_j:(a_1,...,a_n)\mapsto a_j
		\end{equation}
		A sequence in a product metric space $M$ converges if and only if every component $(\text{pr}_j(x_m))_{m\in\mathbb{N}}$ converges in $(M_j, d_j)$.
	\end{property}

	\begin{example}[Supremum distance]
		Let $K\subset\mathbb{R}^n$ be a compact set. Denote the set of continuous functions $f:K\rightarrow\mathbb{C}$ by $\mathcal{C}(K,\mathbb{C})$. The following map defines a metric on $\mathcal{C}(K,\mathbb{C})$:
		\begin{equation}
			\label{topology:supremum_distance}
			d_\infty(f,g) = \sup_{x\in K}|f(x) - g(x)|
		\end{equation}
	\end{example}

	\begin{example}[p-metric]
		We can define following set of metrics on $\mathbb{R}^n$:
		\begin{equation}
			\label{topology:p_metric}
			\boxed{d_p(x,y) = \left(\sum_{i=1}^n|x_i-y_i|^p\right)^{^1/_p}}
		\end{equation}
	\end{example}
        \begin{example}[Chebyshev distance]\index{Chebyshev!distance}
		\begin{equation}
        	    	\label{topology:chebyshev_distance}
        	    	d_\infty(x,y) = \max_{1\leq i\leq n}|x_i - y_i|
		\end{equation}
            	It is also called the \textbf{maximum metric} or $L_\infty$ metric.
	\end{example}
        \begin{remark}
        	This metric is also an example of a product metric defined on the Euclidean product space $\mathbb{R}^n$. The notation $d_\infty$, which is also used for the supremum distance, can be justified if the space $\mathbb{R}^n$ is identified with the set of maps $\{1,...,n\}\rightarrow \mathbb{R}$ equipped with the supremum distance. Another justification is the following relation:
		\begin{equation}
			d_\infty(x,y) = \lim_{p\rightarrow\infty}\ d_p(x,y)
		\end{equation}
		which is also the origin of the name $L_\infty$ metric.
        \end{remark}

\section{Metrizable spaces}    
	
	\newdef{Metrizable space}{\index{metrizable}
		A topological space $X$ is metrizable if it is homeomorphic to a metric space $M$ or equivalently if there exists a metric function $d:X\times X\rightarrow \mathbb{R}$ such that it induces the topology on $X$.
	}
	\begin{theorem}[Urysohn's metrization theorem]\index{Urysohn!metrization theorem}
		Every second-countable $T_3$ space is metrizable.
	\end{theorem}

\section{Compactness in metric spaces}

	\begin{theorem}[Stone]\index{Stone!theorem on paracompactness}
		Every metric space is paracompact.
	\end{theorem}

	\newdef{Totally bounded}{\index{bounded!totally}
		A metric space $M$ is said to be totally bounded if it satisfies the following equivalent statements:
		\begin{itemize}
			\item For every $\varepsilon>0$ there exists a finite cover $\mathcal{F}$ of $M$ with $\forall F\in\mathcal{F}:\text{diam}(F)\leq\varepsilon$.
			\item For every $\varepsilon>0$ there exists a finite subset $E\subset M$ such that $M\subseteq\bigcup_{x\in E}B(x, \varepsilon)$.
		\end{itemize}
	}
	\begin{property}
		Every totally bounded set is bounded and every subset of a totally bounded set is also totally bounded. Furthermore, every totally bounded space is second-countable.
	\end{property}
	
	The following theorem is a generalization of the statement "\textit{a set is compact if and only if it is closed and bounded}" known from Euclidean space $\mathbb{R}^n$.
	\begin{theorem}
		For a metric space $M$ the following statements are equivalent:
		\begin{itemize}
			\item $M$ is compact.
			\item $M$ is sequentially compact.
			\item $M$ is complete and totally bounded.
		\end{itemize}
	\end{theorem}	
	
	\begin{theorem}[Heine-Cantor]\index{Heine!Heine-Cantor theorem}
		Let $M, M'$ be two metric spaces with $M$ being compact. Every continuous function $f:M\rightarrow M'$ is also uniformly continuous.
	\end{theorem}
	

	\newdef{Equicontinuity}{\index{equicontinuity}
		Let $X$ be a topological space and let $M$ be a metric space. A collection $\mathcal{F}$ of maps $X\rightarrow M$ is equicontinuous in $a\in X$ if for all neighbourhoods $U$ of $a$:
		\begin{equation}
			\label{topology:equicontinuity}
			(\forall f\in\mathcal{F})(\forall x\in U)(d(f(x), f(a)) \leq \varepsilon)
		\end{equation}
		for all $\varepsilon\geq 0$.
	}
	
	\begin{property}
		Let $I\subseteq\mathbb{R}$ be an open interval. Let $\mathcal{F}$ be a collection of differentiable functions such that $\{f'(t):f\in\mathcal{F}, t\in I\}$ is bounded. Then $\mathcal{F}$ is equicontinuous.
	\end{property}
	
	\begin{theorem}[Arzel\`a-Ascoli]\index{Arzel\`a-Ascoli}
		Let $K$ be a compact topological space and let $M$ be a complete metric space. The following statements are equivalent for any collection $\mathcal{F}\subseteq C(K, M)$:
		\begin{itemize}
			\item $\mathcal{F}$ is compact with respect to the supremum distance\footnotemark.
			\item $\mathcal{F}$ is equicontinuous, closed under uniform convergence and $\{f(x):f\in\mathcal{F}\}$ is totally bounded for every $x\in K$.
		\end{itemize}
	\end{theorem}
	\footnotetext{See formula \ref{topology:supremum_distance}.}

\section{Complete metric spaces}
	\newdef{Cauchy sequence}{\index{Cauchy!sequence}
		A sequence $(x_n)_{n\in\mathbb{N}}$ in a metric space $(M, d)$ is Cauchy (or has the Cauchy property) if
		\begin{equation}
			\label{topology:cauchy_sequence}
			(\forall\varepsilon>0)(\exists N\in\mathbb{N})(\forall m, n\geq N)(d(x_m, x_n) < \varepsilon)
		\end{equation}
	}
	\begin{property}\leavevmode
		\begin{itemize}
			\item Every closed subset of a complete metric space is complete.
			\item Every complete subset of a metric space is closed.
		\end{itemize}

	\end{property}
    
    \newprop{Cauchy criterion}{\index{Cauchy!criterion}
    	A metric space $(M, d)$ satisfies the Cauchy criterion if a sequence converges to a point $a\in M$  if and only if it is Cauchy.
    }
    \newdef{Completeness}{\index{completeness}
    	A metric space is complete if it satisfies the Cauchy criterion.
    }

\subsection{Injective metric spaces}

        \newdef{Metric retraction}{\index{retraction}
        	Let $(M, d)$ be a metric space. A function $f:X\rightarrow X$ is said to be a retraction of metric spaces if:
        	\begin{itemize}
        		\item $f$ is idempotent
        		\item $f$ is non-expansive, i.e. the following relation holds for all $x, y\in M$:
        			\begin{equation}
		        		d(f(x), f(y)) \leq d(x, y)
        			\end{equation}
        	\end{itemize}
        	The image of $f$ is called a (metric) retract of $M$.
        }
        
        \newdef{Injective metric space}{\index{injective}
		A metric space $M$ is said to be injective if whenever $M$ is isometric to a subspace $Y$ of a metric space $X$ then $Y$ is a metric retract of $X$.
	}
	
	\begin{property}
		Every injective metric space is complete.
	\end{property}
        
\subsection{Convex metric spaces}

	\newdef{Convex space}{\index{convex}
		A metric space $(M, d)$ is said to be convex if for every two points $x, y\in M$ there exists a third point $z\in M$ such that:
		\begin{equation}
			d(x, z) = d(x, y) + d(y, z)
		\end{equation}
	}
	\begin{property}
		A closed subset of Euclidean space is a convex metric space if and only if it is a convex set.
	\end{property}
	
	\newdef{Hyperconvex space}{\index{hyperconvex}
		A convex space for which the set of closed balls has the Helly property\footnote{See definition \ref{set:helly_family}.} is called a hyperconvex space.
	}
	
	\begin{theorem}[Aronszajn \& Panitchpakdi]
		A metric space is injective if and only if it is hyperconvex.	
	\end{theorem}
	
\section{CW complexes}

	\newdef{$n$-cell}{\index{cell}
		An open $n$-cell is a subset of a topological space homeomorphic to the $n$-dimensional open ball. A closed $n$-cell is the image of an $n$-dimensional closed ball under an attaching map\footnote{See definition \ref{topology:attaching_space}.}.
	}
	
	\newdef{CW complex}{\index{CW complex}\label{topology:cw_complex}
		A CW complex is a Hausdorff space $X$ together with a partition of $X$ in open cells satsifying following conditions:
		\begin{itemize}
			\item A subset of $X$ is closed if and only if it meets the closure of each cell in a closed et.
			\item For each open $n$-cell $C$ in the partition there exists an attaching map $f:\overline{B}_n\rightarrow X$ such that:
			\begin{itemize}
				\item $f|_{B_n}$ is homeomorphic to $C$.
				\item $f(\partial \overline{B}_n)$ is covered by a finite number of open cells in the partition, each having dimension smaller than $n$.
			\end{itemize}
		\end{itemize}
		where $\overline{B}_n$ denotes the closed $n$-dimensional ball.
	}
	\newdef{Regular CW complex}{
		A CW complex is called regular if for every open cell $C$ the attaching map $f$ is a homeomorphism onto the closure $\overline{C}$.
	}
	
	\begin{construct}\index{skeleton}
		Every CW complex can, up to isomorphism, be constructed inductively:
		
		First choose a discrete space $X_0$, i.e. a topological space equipped with the discrete topology. This space forms a 0-cell. Then we can add 1-cells $C_1$ using appropriate attaching maps $f:\partial\overline{B}_1\rightarrow X_0$. This way we obtain a 1-dimensional CW complex $X_1$. Inductively one obtains a sequence of nested $n$-dimensional CW complex $X_0\subset X_1\subset\cdots\subset X_n$.
		
		The spaces $X_i$ are also called \textbf{$i$-skeletons}.
	\end{construct}
	\remark{Infinite-dimensional CW complexes can be obtained by taking the direct limit\footnote{See definition \ref{direct_limit}.} of the sequence above.

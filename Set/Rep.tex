\chapter{Representation Theory}

\section{Group representations}
    	\newdef{Representation}{\index{representation}
    		A representation of a group $G$, acting on a vector space $V$, is a homomorphism $\rho: G\rightarrow GL(V)$ from $G$ itself to the automorphism group\footnote{See definition \ref{linalgebra:automorphism}.}\ of $V$. This is a specific case of a group action\footnote{See definition \ref{group:group_action}.}.
    	}

	\newdef{Subrepresentation}{
		A subrepresentation of a representation $V$ is a subspace of $V$ invariant under the action of the group $G$.
	}
        
	\begin{example}[Permutation representation]
		Consider a vector space $V$ equipped with a basis $\{e_i\}_{i\in I}$ with $|I| = n$. Let $G = S^n$ be the symmetric group of dimension $n$. Based on remark \ref{group:permutation_remark} we can consider the action of $G$ on the index set $I$. This representation is given by
		\begin{equation}
			\rho(g):\sum_{i\in I}v_ie_i\mapsto\sum_{i\in I}v_ie_{g\cdot i}
		\end{equation}
	\end{example}
        
        \begin{example}
        	Consider a representation $\rho$ on $V$. There exists a natural representation on the dual space $V^*$. The homomorphism $\rho^*:G\rightarrow GL(V^*)$ is given by:
            \begin{equation}
            	\rho^*(g) = \rho^T(g^{-1}): V^*\rightarrow V^*
            \end{equation}
            where $\rho^T$ is the transpose as defined in \ref{linalgebra:transpose}. This map satifies the following defining property:
            \begin{equation}
            	\Big\langle\rho^*(g)(v^*), \rho(g)(v)\Big\rangle = \langle v^*, v\rangle
            \end{equation}
            where $\langle\cdot,\cdot\rangle$ is the natural pairing of $V$ and its dual.
        \end{example}
        
        \begin{example}\index{tensor!product}
        	A representation $\rho$ which acts on spaces $V, W$ can also be extended to the tensor product  $V\otimes W$ in the following way:
            \begin{equation}
            	g(v\otimes w) = g(v)\otimes g(w)
            \end{equation}
        \end{example}

\section{Irreducible representations}

	\newdef{Irreducibility}{\index{irreducibility}
		A representation is said to be irreducible if there exist no proper non-zero subrepresentation.
	}
	
	\begin{example}[Standard representation]
		Consider the action of $\text{Sym}(n)$ on a vector space $V$. The line generated by $v_1+v_2+...+v_n$ is invariant under the permutation action of $\text{Sym}(n)$. It follows that the permutation representation (on finite-dimensional spaces) is never irreducible.
		
		The $(n-1)$-dimensional complementary subspace
		\begin{equation}
			W = \{a_1v_1 + a_2v_2 + ... + a_nv_n|a_1 + a_2 + ... + a_n = 0\}
		\end{equation}
		does form an irreducible representation when we restrict $\rho$ to $W$. It is called the standard representation of $S^n$.
	\end{example}
        
        \begin{theorem}[Schur's lemma]\index{Schur's lemma}
        	Let $V, W$ be two irreducible representations of a finite group $G$. Let $\varphi: V\rightarrow W$ be a $G$-module homomorphism. We then have:
            \begin{itemize}
	            \item $\varphi$ is an isomorphism or $\varphi = 0$
                \item If $V = W$ then $\varphi$ is constant, i.e. $\varphi$ is a scalar multiple of the identity map $\mathbbm{1}_V$.
            \end{itemize}
        \end{theorem}
        
        \begin{property}
        	If $W$ is a subrepresentation of $V$ then there exists an invariant complementary subspace $W'$ such that $V = W \oplus W'$.
            
            This space can be found as follows: Choose an arbitrary complement $U$ such that $V = W \oplus U$. From this we construct a projection map $\pi_0:V \rightarrow W$. Averaging over $G$ gives
            \begin{equation}
            	\pi(v) = \sum_{g\in G}g\circ\pi_0(g^{-1}v)
            \end{equation}
            which is a $G$-linear map $V\rightarrow W$. On $W$ it is given by the multiplication of $W$ by $|G|$. Its kernel is then an invariant subspace of $V$ under the action of $G$ and complementary to $W$.
        \end{property}
        \begin{property}
        	Let $G$ be a finite group. A representation $V$ can be uniquely decomposed as
            \begin{equation}
            	V = V_1^{\oplus a_1}\oplus\cdots\oplus V_k^{\oplus a_k}
            \end{equation}
            where all $V_k$'s are distinct irreducible representations.
        \end{property}
        
\section{Lie group representations}

	For more information on Lie groups (and Lie algebras which will be considered in the next section) see chapter \ref{chapter:lie}.
	
	\newdef{Adjoint representation}{\index{adjoint!representation of Lie groups}\label{lie:adjoint_representation}
		Let $G$ be a Lie group. Consider the conjugation map $\Psi_g:h\mapsto ghg^{-1}$. The adjoint representation of $G$ is defined by its differential:
		\begin{equation}
			\label{lie:adjoint_rep_of_group}
			\text{Ad}_g:=T_e\Psi_g:T_eG\rightarrow T_eG: X\mapsto gXg^{-1}
		\end{equation}
		It is a representation of $G$ on its own tangent space $T_eG\equiv\mathfrak{g}$.
	}
	
\section{Lie algebra representations}

       	\newformula{Adjoint representation on Lie algebras}{\index{adjoint!representation}
       		Using the fact that the adjoint representation of Lie groups\footnote{See definition \ref{lie:adjoint_representation}.} is smooth we can define the adjoint representation of Lie algebras as:
       		\begin{equation}
       			\text{ad}_g := T_e(\text{Ad}_g)
		\end{equation}
            	Explicitly, let $\mathfrak{g}$ be a Lie algebra. For every element $X\in\mathfrak{g}$ we define the Lie bracket as follows:
                \begin{equation}
                	[X, Y] := \text{ad}_X(Y)
                \end{equation}
	}
        \begin{property}
		Given the antisymmetry of the Lie bracket the Jacobi identity is equivalent to ad$:\mathfrak{g}\rightarrow$ Aut$(\mathfrak{g})$ being a Lie algebra homomorphism, i.e. ad$_{[X, Y]} = [$ad$_X, $ad$_Y]$.
	\end{property}
            
        Using the exponential map we can give property \ref{lie:prop_hom} an explicit form:
        \newformula{Induced homomorphism}{\index{homomorphism!induced}
            	Let $\phi:G\rightarrow H$ be a Lie group homomorphism\footnotemark\ with $G$ connected and simply-connected. This homomorphism induces a
 a Lie algebra homomorphism $\Phi:\mathfrak{g}\rightarrow\mathfrak{h}$ given by
 		\begin{equation}
 			\Phi(X) = \left.\deriv{}{t}\phi\left(e^{tX}\right)\right|_{t=0}
 		\end{equation}
                or equivalently
                \begin{equation}
                	\phi\left(e^{tX}\right) = e^{t\Phi(X)}
                \end{equation}
                \footnotetext{Continuity is needed to ensure that $\phi(e^{tX})$ is also a one-parameter subgroup (see \ref{group:OPS_composition}). However, this condition is always satisfied as it is inherent to the definition of a Lie group homomorphism.}
 	}
            
        \begin{example}
            	The homomorphism induced by $\text{Ad}:G\rightarrow H$ is precisely $\text{ad}:\mathfrak{g}\rightarrow\mathfrak{h}$. Informally we can thus say that the infinitesimal version of the similarity transformation is given by the commutator (in case of $G=$GL$_n$).
        \end{example}

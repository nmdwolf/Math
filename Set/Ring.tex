\section{Rings}\label{section:ring}

    \newdef{Ring}{\index{ring}
        Let $R$ be a set equipped with two binary operations $+,\cdot$ (called \textbf{addition} and \textbf{multiplication}). $(R,+,\cdot)$ is a ring if it satisfies the following axioms:
        \begin{enumerate}
            \item $(R,+)$ is an Abelian group.
            \item $(R,\cdot)$ is a monoid.
            \item Multiplication is distributive with respect to addition.
        \end{enumerate}
    }
    \newdef{Field}{\index{field}
        A ring $(R,+,\cdot)$ for which the monoid $(R\backslash\{1_+\},\cdot)$ is an Abelian group and $1_+\neq 1_\cdot$.
    }

    \newdef{Unit}{\index{unit}
        An invertible element of a ring $(R,+,\cdot)$. The set of units forms a group under multiplication.
    }

    \newdef{Integral domain}{\index{domain!integral}\label{algebra:integral_domain}
        A commutative ring $R$ in which the product of two nonzero elements is again nonzero.
    }

    \newdef{Reduced ring}{\index{reduced ring}
        A ring that contains no nonzero nilpotents.
    }

    \begin{construct}[Localization]\index{localization}
        Let $R$ be a commutative ring and let $S$ be a multiplicatively closed set in $R$. Define an equivalence relation $\sim$ on $R\times S$ in the following way:
        \begin{gather}
            (r_1,s_1)\sim(r_2,s_2) \iff \exists t\in S:t(r_1s_2 - r_2s_1) = 0.
        \end{gather}
        The set $S^{-1}R:=(R\times S)/\sim$, called the localization of $R$ with respect to $S$, can now be turned into a ring by defining an addition and a multiplication. By writing $(r,s)\in S^{-1}R$ as the formal fraction $\frac{r}{s}$, these operations are defined in analogy with the those of ordinary fractions:
        \begin{itemize}
            \item\textbf{Addition}: $\displaystyle\frac{r_1}{s_1} + \frac{r_2}{s_2} = \frac{r_1s_2 + r_2s_1}{s_1s_2}$,
            \item\textbf{Multiplication}: $\displaystyle\frac{r_1}{s_1}\cdot\frac{r_2}{s_2} = \frac{r_1r_2}{s_1s_2}$.
        \end{itemize}
    \end{construct}
    \remark{The localization of $R$ with respect to the set $S$ can be interpreted as the ring obtained by collapsing $S$ into a single unit of $R$.}

    \begin{notation}\label{algebra:localization_notation}
        For specific cases different notations are sometimes used. For example, choose an element $f\in R$ and let $R_f$ denote the localization of $R$ with respect to the set of powers of $f$, i.e. $S=\{f^n\mid n\in\mathbb{N}\}$. This is called the \textbf{localization at (the element)} $f$. Another example occurs when working with prime ideals. Let $P$ be a prime ideal (see the next section). It is not hard to show that the complement $R\backslash P$ is multiplicatively closed. The localization of $R$ with respect to this set is denoted by $R_P$ and is called the \textbf{localization at (the prime ideal)} $P$.
    \end{notation}

    \newdef{Valuation}{\index{valuation}
        Let $k$ be a field and let $\Gamma$ be a totally ordered\footnote{Definition \ref{set:total_order}.}, Abelian group. The group law and the order relation on $\Gamma$ can be extended to the union $\Gamma\cup\{\infty\}$ in the following way (the notation $\infty$ is only a convention):
        \begin{itemize}
            \item $g+\infty:=\infty+g:=\infty$ for all $g\in\Gamma$, and
            \item $g\leq\infty$ for all $g\in\Gamma$.
        \end{itemize}
        A valuation on $k$ (with values in $\Gamma$) is a map $\nu:k\rightarrow\Gamma\cup\{\infty\}$ such that:
        \begin{enumerate}
            \item $\nu(a) = \infty\iff a = 0$;
            \item $\nu(ab) = \nu(a) + \nu(b)$; and
            \item $\min(\nu(a),\nu(b))\leq\nu(a+b)$, where the equality holds if $\nu(a)\neq\nu(b)$.
        \end{enumerate}
    }

\subsection{Ideals}

    \newdef{Ideal}{\index{ideal}\label{algebra:ideal}
        Let $(R,+,\cdot)$ be a ring with $(R,+)$ its additive group. A subset $I\subseteq R$ is called a (two-sided) ideal of $R$ if it satisfies the following conditions:
        \begin{enumerate}
            \item $(I,+)$ is a subgroup of $(R,+)$.
            \item $\forall n\in I,\forall r\in R:n\cdot r,r\cdot n\in I$.
        \end{enumerate}
    }

    \newdef{Artinian ring}{\index{Artin!ring}
        A ring is said to be Artin(ian) if it satisfies the \textbf{descending chain condition} on ideals, i.e. if it contains no infinite descending chain \ref{set:chain} of ideals.
    }
    \newdef{Noetherian ring}{\index{Noether!ring}
        A ring is said to be Noether(ian) if it satisfies the \textbf{ascending chain condition} on ideals, i.e. if it contains no infinite ascending chain of ideals.
    }

    \newdef{Simple ring}{\index{simple!ring}
        A ring that has no nontrivial two-sided ideals. (Some authors require the ring to be Artinian.)
    }

    \newdef{Unit ideal}{
        A ring considered as an ideal of itself.
    }
    \newdef{Proper ideal}{
        An ideal that is not equal to the ring itself.
    }
    \newdef{Prime ideal}{
        Let $R$ be a ring. A proper ideal $I\subset R$ is a prime ideal if for any $a,b\in R$ the following relation holds:
        \begin{gather}
            ab\in I\implies a\in I\lor b\in I.
        \end{gather}
    }
    \newdef{Maximal ideal}{
        A proper ideal that is not contained in another proper ideal.
    }

    \begin{property}
        Every maximal ideal is prime.
    \end{property}

    \newdef{Jacobson radical}{\index{Jacobson radical}\label{algebra:jacobson_radical}
        The Jacobson radical of a ring $R$, often denoted by $J(R)$, is the ideal obtained as the intersection of all maximal left (or right) ideals. Equivalently, it is the intersection of the \textit{annihilators} of all simple, left (or right) $R$-modules.
    }

    \begin{construct}[Generating ideals]\index{ideal!generating set}\label{algebra:generating_set_ideal}
        Let $R$ be a ring and let $X$ be a subset of $R$. The two-sided ideal generated by $X$ is defined as the intersection of all two-sided ideals containing $X$. An explicit construction is given by
        \begin{gather}
            I = \left\{\sum_{i=1}^n l_ix_ir_i\,\middle\vert\,n\in\mathbb{N}, \forall i\leq n:l_i,r_i\in R\land x_i\in X\right\}.
        \end{gather}
        Left and right ideals are generated in a similar fashion.
    \end{construct}
    \begin{notation}
        If the ideal $I$ is generated by the elements $\{f_j\}_{j\in J}$ (for some index set $J$), it is often denoted by
        \begin{gather}
            I\equiv(f_1,f_2,\ldots).
        \end{gather}
    \end{notation}

    \begin{construct}[Extension]\index{extension!ideal}
        Let $I$ be an ideal of a ring $R$ and let $\iota:R\rightarrow S$ be a ring morphism. The extension of $I$ with respect to $\iota$ is the ideal generated by the set $\iota(I)$.
    \end{construct}

    \newdef{Principal ideal}{\index{ideal!principal}
        An ideal that is generated by a single element.
    }
    \newdef{Principal ideal domain}{\index{domain!principal ideal}
        An integral domain \ref{algebra:integral_domain} in which every ideal is principal.
    }

    \newdef{Local ring}{\index{local!ring}\label{algebra:local_ring}
        A ring for which a unique, maximal, left ideal exists. This also implies that there exists a unique, maximal, right ideal and that these ideals coincide.
    }
    \begin{property}[Characterization by invertible complements]\label{algebra:local_ring_invertible}
        A ring $R$ is local if and only if there exists a maximal ideal $M$ such that every element in the complement $R\backslash M$ is invertible.
    \end{property}

    \begin{property}[Prime localization]\label{algebra:localization_local_ring}
        The localization of a ring $R$ with respect to a prime ideal $P$ is a local ring, where the maximal ideal is given by the extension of $P$ with respect to the ring morphism $\iota:R\rightarrow R_P$. Equivalently, this says that the maximal ideal is given by $PR_P$.
    \end{property}

    \newdef{Residue field}{\index{residue!field}
        Consider a local ring $R$ and let $I$ be its maximal ideal. The quotient ring $R/I$ forms a field, called the residue field.
    }

\subsection{Modules}

    \newdef{$R$-module}{\index{module!over a ring}
        Let $(R,+,\cdot)$ be a ring. A set $X$ is said to be an $R$-module if it satisfies the same axioms as those of a vector space \ref{linalgebra:vector_space}, but where the scalars are only elements of a ring instead of a field.
    }
    \newdef{Free module}{\index{free!module}
        An $R$-module $M$ is said to be free if it admits a basis, i.e. there exists a set $\{x_i\}_{i\in I}$ (where $I$ can be infinite) such that:
        \begin{enumerate}
            \item every element $m\in M$ can be written as a linear combination $\sum_{j\in J}r_jx_j$, where $J\subseteq I$ is finite.
            \item the set $\{x_i\}_{i\in I}$ is linearly independent in the sense that
                \begin{gather}
                    \sum_{j\in J\subseteq I}r_jx_j=0\implies \forall j\in J:r_j=0.
                \end{gather}
        \end{enumerate}
    }
    \begin{example}[Dual numbers]\index{dual!numbers}
        Let $R$ be a ring. The $R$-algebra of dual numbers, often denoted by $R[\varepsilon]$, is defined as the free $R$-module with basis $\{1,\varepsilon\}$ subject to the relation $\varepsilon^2 = 0$.
    \end{example}

    \begin{property}[Division rings]\index{division ring}\label{algebra:module_basis}
        For a general $R$-module the existence of a basis is not guaranteed unless $R$ is a \textit{division ring}. (See Construction \ref{linalgebra:hamel_basis} for more information.)
    \end{property}
    \begin{result}
        Since every field is in particular a division ring, the existence of a basis follows from the above property for $R$-modules.
    \end{result}

    \newdef{Projective module}{\index{projective!module}
        A module $P$ is said to be projective if $P$ can be expressed as
        \begin{gather}
            P\oplus M = F,
        \end{gather}
        where $M$ is a module and $F$ is a free module, i.e. if $P$ is a direct sumand of a free module.
    }

\subsection{Semisimplicity}\index{semisimple!module}

    \newdef{Simple module}{\index{simple!module}
        A module over a ring is said to be simple if it contains no nontrivial submodules. A module is said to be \textbf{semisimple} if it admits a decomposition as a direct sum of simple modules. A ring is said to be semisimple if it is semisimple as a module over itself.
    }
    \begin{property}[Jacobson radical]
        A ring is semisimple if and only if it is Artinian and if its Jacobson radical \ref{algebra:jacobson_radical} vanishes.
    \end{property}

    \begin{theorem}[Artin-Wedderburn]\index{Artin-Wedderburn}\label{algebra:artin_wedderburn}
        Every semisimple ring is isomorphic to a direct sum of matrix rings over division rings $D_i$ with multiplicity $n_i$. Furthermore, the integers $D_i$ and $n_i$ are unique (up to a permutation of the indices).
    \end{theorem}

\section{Limits of algebraic structures}

    \newdef{Direct system}{\index{direct!system}
        Let $(I,\leq)$ be a directed set \ref{set:directed_set} and let $\{A_i\}_{i\in I}$ be a family of algebraic objects (groups, rings, ...). Consider a collection of morphisms $\{f_{ij}:A_i\rightarrow A_j\}_{i,j\in I}$ between these objects with the following properties:
        \begin{enumerate}
            \item for every $i\in I$: $f_{ii} = \mathbbm{1}_{A_i}$, and
            \item for every $i\leq j\leq k\in I$: $f_{ik} = f_{jk}\circ f_{ij}$.
        \end{enumerate}
        The pair $(A_i,f_{ij})$ is called a direct system (over $I$).
    }

    \newdef{Direct limit\footnotemark}{\index{direct!limit}\index{inductive!limit}\label{algebra:direct_limit}
        \footnotetext{Also called an \textbf{inductive limit}.}
        Consider a direct system $(A_i,f_{ij})$ over a directed set $I$. The direct limit $A$ of this direct system is defined as follows:
        \begin{gather}
            \varinjlim A_i := \left.\bigsqcup_{i\in I}A_i\right/\sim
        \end{gather}
        where the equivalence relation is given by $x\in A_i\sim y\in A_j\iff\exists k\in I: f_{ik}(x) = f_{jk}(y)$. Informally put: two elements are equivalent if they eventually become the same.

        The algebraic operations on $A$ are defined such that the inclusion maps $\phi_i:A_i\rightarrow A$ are morphisms.
    }

    \newdef{Inverse system}{\index{inverse!system}
        Let $(I,\leq)$ be a directed set \ref{set:directed_set} and let $\{A_i\}_{i\in I}$ be a family of algebraic objects (groups, rings, ...). Consider a collection of morphisms $\{f_{ij}:A_j\rightarrow A_i\}_{i,j\in I}$ between these objects with the following properties:
        \begin{enumerate}
            \item for every $i\in I$: $f_{ii} = \mathbbm{1}_{A_i}$, and
            \item for every $i\leq j\leq k\in I$: $f_{ik} = f_{ij}\circ f_{jk}$.
        \end{enumerate}
        The pair $(A_i,f_{ij})$ is called an inverse system (over $I$).
    }

    \newdef{Inverse limit\footnotemark}{\index{inverse!limit}\index{projective!limit}\label{algebra:inverse_limit}
        \footnotetext{Also called a \textbf{projective limit}.}
        Consider an inverse system $(A_i,f_{ij})$ over a directed set $I$. The inverse limit $A$ of this inverse system is defined as follows:
        \begin{gather}
            \varprojlim A_k := \left\{\vec{a}\in\prod_{i\in I}A_i\,\middle\vert\,a_i=f_{ij}(a_j), \forall i\leq j\right\}.
        \end{gather}
        For all $i\in I$ there exists a natural projection $\pi_i:\varprojlim A_k\rightarrow A_i$.
    }

    \begin{remark}
        The direct and inverse limit are each other's (categorical) dual. The former is a colimit while the latter is a limit in category theory.
    \end{remark}

\section{Galois theory}

    \newdef{Field extension}{\index{field!extension}\label{algebra:field_extension}
        Let $k$ be a field. A field extension of $k$ is a field $K$ such that $k\subset K$ and such that the operations of $k$ are the restrictions of those in $K$.
    }
    \begin{notation}
        A field extension $K$ of $k$ is often denoted by $K/k$.
    \end{notation}

    \newdef{Degree}{\index{degree!of extension}
        If $K+k$ is a field extension, then $K$ can be given the structure of a $k$-vector space \ref{linalgebra:vector_space}. The dimension of this vector space is called the degree of the extension $K$. It is often denoted by $[K:k]$.
    }

    ?? COMPLETE ??
\section{Rings}
	
	\newdef{Ring}{\index{ring}
		Let $R$ be a set equipped with two binary operations $+,\cdot$ (called addition and multiplication). $(R,+,\cdot)$ is a ring if it satisfies the following axioms:
    		\begin{enumerate}
			\item $(R,+)$ is a commutative group.
			\item $(R,\cdot)$ is a monoid.
			\item Multiplication is distributive with respect to addition.
		\end{enumerate}
	}
	
	\newdef{Unit}{\index{unit}
		An invertible element of ring $(R, +, \cdot)$. The set of units forms a group under multiplication.
	}
	
	\newdef{Integral domain}{\index{domain!integral}\label{alg:integral_domain}
		A commutative ring $R$ in which the product of two nonzero elements is again nonzero.
	}
	
	\begin{construct}[Localization]\index{localization}
		Let $R$ be a commutative ring and let $S$ be a multiplicative monoid in $R$. We first define an equivalence relation $\sim$ on $R\times S$ in the following way:
		\begin{equation}
			(r_1, s_1)\sim(r_2, s_2) \iff \exists t\in S: t(r_1s_2 - r_2s_1) = 0
		\end{equation}
		
		The set $R^* = (R\times S)/\sim$, called the localization of $R$ with respect to $S$, can now be turned into a ring by defining an addition and a multiplication. By writing $(r, s)\in R^*$ as the formal fraction $\frac{r}{s}$ we obtain the familiar operations of fractions:
		\begin{itemize}
			\item $\displaystyle\frac{r_1}{s_1} + \frac{r_2}{s_2} = \frac{r_1s_2 + r_2s_1}{s_1s_2}$
			\item $\displaystyle\frac{r_1}{s_1}\cdot\frac{r_2}{s_2} = \frac{r_1r_2}{s_1s_2}$
		\end{itemize}
	\end{construct}
	\remark{The localization of $R$ with respect to the monoid $S$ can be interpreted as the ring obtained by collapsing $S$ into a single unit of $R$.}
	
	\begin{notation}
		The localization of $R$ with respect to $S$ is often denoted by $S^{-1}R$. For specific cases different notations are sometimes used. For example choose an element $f\in R$, then $R_f$ denotes the localization of $R$ with respect to the set of powers of $f$, i.e. $S=\{f^n:n\in\mathbb{N}\}$. This is called the \textbf{localization at the element} $f$. Another example occurs when working with prime ideals: Let $P$ be a prime ideal, then it is not hard to show that $R\backslash P$ is multiplicatively closed. The localization of $R$ by this set is denoted by $R_P$ and called the \textbf{localization at the prime ideal} $P$.
	\end{notation}
	
	\newdef{Valuation}{\index{valuation}
		Let $K$ be a field and let $\Gamma$ be a totally ordered Abelian group\footnote{See definition \ref{group:total_order}.}. First we extend the group law of $\Gamma$ to the union $\Gamma\cup\{\infty\}$ in the following way:
		\begin{itemize}
			\item $g+\infty = \infty + g=\infty$ for all $g\in\Gamma$
			\item $g\leq\infty$ for all $g\in\Gamma$
		\end{itemize}
		A valuation of $K$ is a map $\nu:K\rightarrow\Gamma\cup\{\infty\}$ such that:
		\begin{enumerate}
			\item $\nu(a) = \infty \iff a = 0$
			\item $\nu(ab) = \nu(a) + \nu(b)$
			\item $\min(\nu(a), \nu(b))\leq\nu(a + b)$ where the equality holds if and only if $\nu(a)\neq\nu(b)$
		\end{enumerate}
	}

\subsection{Ideals}\index{ideal}

    	\newdef{Ideal}{\label{linalgebra:ideal}
    		Let $(R,+,\cdot)$ be a ring with $(R,+)$ its additive group. A subset $I\subseteq R$ is called an ideal\footnotemark\ of $R$ if it satisfies the following conditions:
        	\begin{enumerate}
			\item $(I,+)$ is a subgroup of $(R,+)$
                	\item $\forall n\in I, \forall r\in R:(n\cdot r), (r\cdot n)\in I$
		\end{enumerate}
		\footnotetext{More generally: two-sided ideal}
        }
        \newdef{Simple ring}{\index{simple!ring}
        	A ring is simple if it has no nontrivial two-sided ideals.\footnote{Some authors require the ring to be Artinian.}
        }
        
        \newdef{Unit ideal}{Let $(R,+,\cdot)$ be a ring. $R$ itself is called the unit ideal.}
        \newdef{Proper ideal}{Let $(R,+,\cdot)$ be a ring. A subset $I\subset R$ is said to be a proper ideal if it is an ideal of $R$ and if it is not equal to $R$.}
        \newdef{Prime ideal}{
        	Let $(R,+,\cdot)$ be a ring. A proper ideal $I$ is a prime ideal if for any $a,b\in R$ the following relation holds:
        	\begin{equation}
        		ab\in I\implies \text a\in I \vee b\in I
        	\end{equation}
        }
        \newdef{Maximal ideal}{Let $(R,+,\cdot)$ be a ring. A proper ideal $I$ is said to be maximal if there exists no other proper ideal $T$ in R such that $I\subset T$.}
        \newdef{Minimal ideal}{A proper ideal is said to be minimal if it contains no other nonzero ideal.}
        
        \begin{property}
        	Every maximal ideal is prime.
        \end{property}
        
        \newdef{Jacobson radical}{\index{Jacobson}
        	The Jacobson radical of a ring $R$, often denoted by $J(R)$, is the ideal obtained as the intersection of a maximal left (or right) ideals. Equivalently one can define it as the intersection of the \textit{annihilators} of all left (or right) simple $R$-modules.
        }
        
	\begin{construct}[Generating set of an ideal]\index{ideal!generating set}\label{group:generating_set_ideal}
		Let $R$ be a ring and let $X$ be a subset of $R$. The two-sided ideal generated by $X$ is defined as the intersection of all two-sided ideals containing $X$. An explicit construction is given by:
		\begin{equation}
			I = \left\{\left.\sum_{i=1}^n l_ix_ir_i\ \right\vert\ \forall n\in\mathbb{N}:\forall l_i, r_i\in R\text{ and } x_i\in X\right\}
		\end{equation}
		Left and right ideals are generated in a similar fashion.
	\end{construct}
	\begin{notation}
		Let the ideal $I$ be generated by the elements $\{f_j\}_{j\in J}$ (for some index set $J$). One in general uses the following notation:
		\begin{gather}
			I = (f_1, f_2, \ldots)
		\end{gather}
	\end{notation}
	
	\newdef{Principal ideal}{\index{ideal!principal}
		An ideal which is generated by a single element.
	}
	\newdef{Principal ideal domain}{\index{domain!principal ideal}
		An integral domain\footnote{See definition \ref{alg:integral_domain}.} in which every ideal is principal.
	}
	
	\begin{construct}[Extension]\index{extension!ideal}
        	Let $I$ be an ideal of a ring $R$ and let $\iota:R\rightarrow S$ be a ring morphism. The extension of $I$ with respect to $\iota$ is the ideal generated by the set $\iota(I)$.
        \end{construct}
	
	\newdef{Local ring}{\index{local!ring}\label{algebra:local_ring}
		A local ring is a ring for which a unique maximal left ideal exists.\footnote{This also implies that there exists a unique maximal right ideal and that these ideals coincide.}
	}
	\begin{property}
		\label{algebra:local_ring_invertible}
		A ring $R$ is local if and only if there exists a maximal ideal $M$ such that every element in $R\backslash M$ is invertible.
	\end{property}
	
	\begin{property}
		\label{algebra:localization_local_ring}
		The localization of a ring $R$ with respect to a prime ideal $P$ is a local ring, where the maximal ideal is the extension of $P$ with respect to the ring morphism $\iota:R\rightarrow R^*$. Equivalently this says that the maximal ideal is given by $PR_P$.
	\end{property}
	
	\newdef{Residue field}{\index{residue!field}
		Consider a local ring $R$ and let $I$ be its maximal ideal. The quotient ring $R/I$ forms a field, called the residue field.
	}

\subsection{Modules}
	
	\newdef{$R$-Module}{\index{module!over a ring}
		Let $(R, +, \cdot)$ be a ring. A set $X$ is an $R$-module if it satisfies the same axioms as those of a vector space \ref{linalgebra:vector_space} but where the scalars are only elements of a ring instead of a field.
	}
	
	\newdef{Free module}{\index{free!module}
		A module is said to be free if it admits a basis.
	}
	
	\begin{property}\label{algebra:module_basis}
		For a general $R$-module the existence of a basis is not guaranteed unless $R$ is a division ring. See construction \ref{linalgebra:hamel_basis} to see how this basis can be constructed.
	\end{property}
	\begin{result}
		As every field is in particular a division ring, the existence of a basis follows from the above property for $R$-modules.
	\end{result}
	
	\newdef{Projective module}{\index{projective!module}
		A module $P$ is said to be projective if:
		\begin{equation}
			P\oplus M = F
		\end{equation}
		where $M$ is a module and $F$ is a free module.
	}
	
	\begin{example}[Dual numbers]\index{dual!numbers}
		Let $R$ be a ring. The $R$-algebra of dual numbers over $R$, often denoted by $R[\varepsilon]$, is defined as the free $R$-module with basis ${1, \varepsilon}$ under the relation $\varepsilon^2 = 0$.
	\end{example}

\subsection{Semisimplicity}\index{semisimple!module}

	\newdef{Simple module}{\index{simple!module}
		A module over a ring is said to be simple if it contains no nontrivial submodules. A module is said to be \textbf{semisimple} if it admits a direct sum decomposition in terms of simple modules.
	}
	\newdef{Semisimple ring}{
		A ring is said to be semisimple if it is semisimple as a module over itself.
	}
	\begin{theorem}
		A ring is semisimple if and only if it is Artinian and if its Jacobson radical vanishes.
	\end{theorem}
	
	\begin{theorem}[Artin-Wedderburn]\index{Artin-Wedderburn}
		Any semisimple ring is isomorphic to a direct sum of matrix rings over division rings $D_i$ with multiplicity $n_i$. Both $D_i$ and $n_i$ are unique.\footnote{Up to a permutation of the indices.}
	\end{theorem}

\subsection{Graded rings}
	
	\newdef{Graded ring}{\index{graded}\label{group:graded_ring}
		Let $R$ be a ring that can be written as the direct sum of Abelian groups $A_k$:
		\begin{equation}
			R = \bigoplus_{k\in\mathbb{N}}A_k
		\end{equation}
		If $R$ has the property that for every $i, j\in\mathbb{N}: A_i\star A_j\subseteq A_{i+j}$, where $\star$ is the ring multiplication, then $R$ is said to be a graded ring. The elements of the space $A_k$ are said to be \textbf{homogeneous of degree $k$}.
	}
	
	\newformula{Graded commutativity}{\index{commutativity!graded}
		Let $m = \deg v$ and let $n = \deg w$. If
		\begin{equation}
			\label{group:graded_commutativity}
			vw = (-1)^{mn}wv
		\end{equation}
		for all elements $v, w$ of the graded ring then it is said to be a graded-commutative ring.
	}

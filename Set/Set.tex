\chapter{Set Theory}

    For a formal introduction to the underlying axioms (and generalizations) of set theory, see section \ref{set:section:axiomatization} at the end of this chapter.

\section{Collections}

    \newdef{Power set}{\index{power!set}\label{set:power_set}
        \nomenclature[S_P]{$P(S), 2^S$}{Power set of $S$.}
        Let $S$ be a set. The power set is defined as the set of all subsets of $S$ and is (often) denoted by $P(S)$ or $2^S$. The existence of this set is enforced by axiom \ref{set:power_set_axiom} (the \textit{axiom of power set}).
    }
    \result{All sets are elements of their power set: $S\subset P(S)$.}

    \newdef{Collection}{\index{collection}
        Let $A$ be a set. A collection of elements in A is a subset of $A$.
    }
    \newdef{Family}{\index{family}
        Let $A, I$ be two sets. A family of elements of $A$ (with \textbf{index set} $I$) is a function $f:I\rightarrow A$. A family with index set $I$ is often denoted by $(x_i)_{i\in I}$. In contrast to collections, a family can ''contain'' multiple copies of the same element.
    }

    \newdef{Helly family}{\index{Helly family}\label{set:helly_family}
        A Helly family of order $k$ is a pair $(X, F)$ with $F\subset P(X)$ such that for every finite $G\subset F$:
        \begin{gather}
            \bigcap_{V\in G}V = \emptyset\implies \exists H\subseteq G: \left(\bigcap_{V\in H}V = \emptyset\right) \land \Big(|H| \leq k\Big).
        \end{gather}
        A Helly family of order 2 is sometimes said to have the \textbf{Helly property}.
    }

    \newdef{Diagonal}{\index{diagonal}
        Let $S$ be a set. The diagonal of $S$ is defined as follows:
        \begin{gather}
            \Delta_S := \big\{(a, a)\in S\times S: a\in S\big\}.
        \end{gather}
    }

    \newdef{Cover}{\index{cover}\label{set:cover}
        A cover of $S$ is a collection of sets $\mathcal{F}\subseteq P(S)$ such that
        \begin{gather}
            \bigcup_{V\in\mathcal{F}}V = S.
        \end{gather}
    }

    \newdef{Partition}{\index{partition}
        A  partition of $X$ is a family of disjoint subsets $(A_i)_{i\in I} \subset P(X)$ such that $\bigcup_{i\in I}A_i = X$.
    }
    \newdef{Refinement}{\index{refinement}
        Let $P$ be a partition of $X$. A refinement $P'$ of $P$ is a collection of subsets such that every $A\in P$ can be written as a disjoint union of elements in $P'$. It follows that every refinement is also a partition.
    }

    \newdef{Filter}{\index{filter}
        Let $X$ be a partially ordered set. A family $\mathcal{F}\subseteq P(X)$ is a filter on $X$ if it satisfies following conditions:
        \begin{enumerate}
            \item \textbf{Empty set}: $\emptyset\not\in\mathcal{F}$
            \item \textbf{Closed under intersections}: $\forall A, B \in\mathcal{F}:A\cap B\in\mathcal{F}$
            \item \textbf{Closed under inclusion}: if $A\in\mathcal{F}$ and $A\subseteq B$ then $B\in\mathcal{F}$.
        \end{enumerate}
    }

    \newdef{Filtration}{\index{filtration}\label{set:filtration}
        Consider a set $A$ together with a collection of subsets $F_iA$ indexed by a totally ordered set $I$. The collection is said to be a filtration of $A$ if $i\leq j$ implies that $F_iA\subseteq F_jA$.

        A filtration is said to be \textbf{exhaustive} if $\bigcup_iF_iA=A$ and \textbf{separated} if $\bigcap_iF_iA=\emptyset$.
    }
    \newdef{Associated grading}{
        In the case where one can define quotient objects every filtration $\{F_iA\}_{i\in\mathbb{N}}$ of $A$ defines an associated graded object $\{G_iA := F_iA/F_{i-1}A\}$.
    }

\section{Set operations}

    \newdef{Symmetric difference}{\index{symmetric difference}\label{set:symmetric_difference}
        \begin{gather}
            A\Delta B := (A\backslash B)\cup(B\backslash A)
        \end{gather}
    }

    \newdef{Complement}{\index{complement}\label{set:complement}
        Let $\Omega$ be the universal set\footnote{See section \ref{section:universes}.}. Let $E\subseteq\Omega$. The complement of $E$ is defined as follows:
        \begin{gather}
            E^c := \Omega \backslash E.
        \end{gather}
    }

    \newformula{de Morgan's laws}{\index{de Morgan's laws}
        \begin{gather}
            \label{set:de_morgan_union}
            \left(\bigcup_i A_i\right)^c = \bigcap_i A_i^c
        \end{gather}
        \begin{gather}
            \label{set:de_morgan_intersection}
            \left(\bigcap_i A_i\right)^c = \bigcup_i A_i^c
    \end{gather}
    }

    \newdef{Converse relation}{\label{set:converse}\index{converse}
        Consider a \textit{relation} $R\subset X\times Y$ between two sets $X, Y$. The converse relation $R^t$ is defined as follows:
        \begin{gather}
            R^t := \big\{(y, x)\in Y\times X:(x, y)\in R\big\}.
        \end{gather}
    }
    \newdef{Composition of relations}{\label{set:relational_composition}\index{composition!of relations}
        Consider two relations $R\subset X\times Y$ and $S\subset Y\times Z$ between three sets $X, Y$ and $Z$. The composition $S\circ R$ is defined as follows:
        \begin{gather}
            S\circ R := \big\{(x, z)\in X\times Z:\exists y\in Y: (x, y)\in R\land (y, z)\in S\big\}.
        \end{gather}
    }

\section{Ordered sets}
\subsection{Posets}

    \newdef{Preordered set}{\index{preorder|see{order}}
        A preordered set is a set equipped with a reflexive and transitive binary relation.
    }
    \newdef{Partially ordered set}{\index{poset}\label{set:poset}
        A set $P$ equipped with a binary relation $\leq$ is called a partially ordered set (\textbf{poset}) if the following 3 axioms are fulfilled for all elements $a,b,c\in P$:
        \begin{enumerate}
            \item \textbf{Reflexivity}: $a\leq a$
            \item \textbf{Antisymmetry}: $a\leq b \land b\leq a\implies a = b$
            \item \textbf{Transitivity}: $a\leq b\land b\leq c\implies a\leq c$
        \end{enumerate}
        Equivalently, it is a preordered set for which the binary relation is also antisymmetric.
    }
    \newdef{Totally ordered set}{\index{order}\index{totality}\label{set:total_order}
        A poset $P$ with the property that for all $a,b\in P: a\leq b$ or $b\leq a$ is called a (nonstrict) totally ordered set. This property is called \textbf{totality}.
    }
    \newdef{Strict total order}{
        A nonstrict order $\leq$ has an associated strict order $<$ that satisfies $a<b \iff a\leq b\land a\neq b$.
    }

    \newdef{Maximal element}{
        An element $m$ of a poset $P$ is maximal if for every $p\in P:m\leq p\implies m=p$.
    }

    \newdef{Chain}{\index{chain}\label{set:chain}
        A totally ordered subset of a poset.
    }
    \begin{theorem}[Zorn's lemma\footnotemark]\index{Zorn}\label{set:zorns_lemma}
        \footnotetext{This theorem is equivalent to the \textit{axiom of choice}.}
        Let $(P, \leq)$ be a poset. If every chain in $P$ has an upper bound in $P$, then $P$ has a maximal element.
    \end{theorem}

    \newdef{Directed\footnotemark\ set}{\index{directed set}\index{filtered set|see{directed set}}
        \label{set:directed_set}
        \footnotetext{Sometimes called an \textbf{upward} directed set. \textbf{Downward} directed sets are analogously defined with a lower bound for every two elements. Directed sets are also sometimes called \textbf{filtered sets}.}
        A set $X$ equipped with a preorder $\leq$ with the additional property that every 2-element subset has an upper bound, i.e. for every two elements $a, b\in X$ there exists an element $c\in X$ such that $a\leq c\land b\leq c$.
    }
    \newdef{Net}{\index{net}\label{set:net}
        A net on a set $X$ is a subset of $X$ indexed by a directed set $I$.
    }

\subsection{Bounds}

    \newdef{Supremum}{\index{supremum}\label{set:supremum}
        The supremum $\sup(X)$ of a poset $X$ is the smallest upper bound of $X$.
    }
    \newdef{Infimum}{\index{infimum}\label{set:infimum}
        The infimum $\inf(X)$ of a poset $X$ is the greatest lower bound of $X$.
    }

    \newdef{Maximum}{\index{maximum}\label{set:maximum}
        If $\sup(X)\in X$ the supremum is called the maximum of $X$. This is denoted by $\max(X)$.
    }
    \newdef{Minimum}{\index{minimum}\label{set:minimum}
        If $\inf(X)\in X$ the supremum is called the minimum of $X$. This is denoted by $\min(X)$.
    }

\subsection{Lattices}

    \newdef{Semilattice}{\index{join}\index{meet}
        A poset $(P, \leq)$ for which every 2-element subset has a supremum (also called a \textbf{join}) in $P$ is called a join-semillatice. Similarly, a poset $(P, \leq)$ for which every 2-element subset has an infimum (also called a \textbf{meet}) in $P$ is called a meet-semilattice.
    }
    \begin{notation}
        The join of $\{a, b\}$ is denoted by $a\land b$. The meet of $\{a, b\}$ is denoted by $a\lor b$.
    \end{notation}
    \newdef{Lattice}{\index{lattice!set theory}
        A poset $(P, \leq)$ is called a lattice if it is both a join- and a meet-semilattice.
    }
    The above definition also allows for a purely algebraic formulation:
    \newadef{Lattice}{
        A lattice is an algebraic structure that admits operations $\land, \lor$ and constants $\top, \bot$ that satisfy the following axioms:
        \begin{enumerate}
            \item Both $\land$ and $\lor$ are idempotent, commutative and associative.
            \item They satisfy the \textbf{absorption laws}:
            \begin{gather}
                a\lor (a\land b) = a\qquad\qquad\qquad a\land (a\lor b) = a.
            \end{gather}
            \item $\top$ and $\bot$ are the respective identities of $\land$ and $\lor$.
        \end{enumerate}
        To go from this definition to the order-theoretic one we define the partial order \[a\leq b \iff a\land b=a.\] There exists an equivalent relation for the join.
    }

    \newdef{Bounded lattice}{
        A lattice $(P, \leq)$ is called bounded if it contains a greatest element (denoted by $\top$ or 1) and a smallest element (denoted by $\bot$ or 0) such that
        \begin{gather}
            \bot\leq x\leq\top
        \end{gather}
        for all $x\in P$. These elements are the identities for the join and meet operations:
        \begin{gather}
            x\land\top=x\qquad\qquad\qquad x\lor\bot=x.
        \end{gather}
    }

    \newdef{Frame}{\index{frame!set theory}\label{set:frame}
        A poset $(P, \leq)$ that admits all joins\footnote{When working with categories this has to be restricted to "all small joins", or equivalently, the index category should be a set.} and all finite limits and for which the \textbf{infinite distributivity law} is satisfied:
        \begin{gather}
            y\wedge\left(\bigvee_{i\in I}x_i\right) = \bigvee_{i\in I}\left(y\wedge x_i\right).
        \end{gather}
    }

    \newdef{Heyting algebra}{\index{Heyting!algebra}\index{complement}\label{set:heyting}
        A bounded lattice $H$ such that for every two elements $a, b\in H$ there exists a greatest element $x\in H$ for which
        \begin{gather}
            a\wedge x\leq b.
        \end{gather}
        This element is denoted by $a\rightarrow b$. The \textbf{pseudo-complement} $\neg a$ of an element $a\in H$ is then defined as $a\rightarrow\bot$.
    }
    \newdef{Boolean algebra}{\index{law of excluded middle}\index{Boolean!algebra}
        A Boolean algebra is a Heyting algebra in which the \textit{law of excluded middle} holds:
        \begin{gather}
            \forall x: \neg\neg x=x.
        \end{gather}
        This can be equivalently stated as
        \begin{gather}
            \forall x: x\lor\neg x=\top.
        \end{gather}
    }

\section{Algebra of sets}

    \newdef{Algebra of sets}{\index{algebra!of sets}\label{set:algebra_of_sets}
        A collection $\mathcal{F}\subset P(X)$ is a called an algebra over $X$ if it is closed under finite unions, finite intersections and complements. The pair $(X,\mathcal{F})$ is also called a \textbf{field of sets}.
    }

\subsection{\texorpdfstring{$\sigma$}{sigma}-algebra}

    \newdef{$\sigma$-algebra}{\index{$\sigma$!algebra}\label{set:sigma_algebra}
        A collection $\Sigma\subset P(X)$ is called a $\sigma$-algebra over a set $X$ if it satisfies the following 3 axioms:
        \begin{enumerate}
            \item \textbf{Total space}: $X\in\Sigma$
            \item \textbf{Closed under complements}: $\forall E\in\Sigma: E^c\in\Sigma$
            \item \textbf{Closed under countable unions}: $\forall\{E_i\}_{i=1}^n\subset\Sigma:\bigcup_{i=1}^nE_i\in\Sigma$
        \end{enumerate}
    }
    \begin{remark}
        Axioms $(2)$ and $(3)$ together with de Morgan's laws (equations \ref{set:de_morgan_union} and \ref{set:de_morgan_intersection}) imply that a $\sigma$-algebra is also closed under countable intersections.
    \end{remark}

    \begin{result}
        Every algebra of sets is a $\sigma$-algebra.
    \end{result}

    \begin{property}
        The intersection of a family of $\sigma$-algebras is again a $\sigma$-algebra.
    \end{property}

    \begin{definition}
        A $\sigma$-algebra $\mathcal{G}$ is said to be generated by a collection of sets $\mathcal{A}$ if
        \begin{gather}
            \label{set:generated_sigma_algebra}
            \mathcal{G} = \bigcap\{\mathcal{F}:\mathcal{F} \text{ is a } \sigma\text{-algebra that contains } \mathcal{A}\}.
        \end{gather}
        Equivalently it is the smallest $\sigma$-algebra containing $\mathcal{A}$.
    \end{definition}
    \begin{notation}\label{set:notation:generated_sigma_algebra}
        The $\sigma$-algebra generated by a collection of sets $\mathcal{A}$ is often denoted by $\mathcal{F}_\mathcal{A}$ or $\sigma(\mathcal{A})$.
    \end{notation}

    \begin{definition}\label{set:product_of_sigma_algebras}
        The product $\sigma$-algebra $\mathcal{F}$ can be defined in the following equivalently ways:
        \begin{itemize}
            \item $\mathcal{F}$ is generated by the collection
                \[\mathcal{C} = \{A_1\times \Omega_2:A_1\in\mathcal{F}_1\}\cup\{\Omega_1\times A_2:A_2\in\mathcal{F}_2\}.\]
            \item $\mathcal{F}$ is the smallest $\sigma$-algebra such that the following projections are measurable (see \ref{lebesgue:measurable_function}):
                \[\text{Pr}_1:\Omega\rightarrow\Omega_1:(\omega_1,\omega_2)\mapsto\omega_1\]
                \[\text{Pr}_2:\Omega\rightarrow\Omega_2:(\omega_1,\omega_2)\mapsto\omega_2.\]
            \item $\mathcal{F}$ is the smallest $\sigma$-algebra containing the products $A_1\times A_2$ for all $A_1\in\mathcal{F}_1, A_2\in\mathcal{F}_2$.
        \end{itemize}
    \end{definition}

\subsection{Monotone class}

    \newdef{Monotone class}{\index{monotone!class}
        Let $\mathcal{A}$ be a collection of sets. $\mathcal{A}$ is called a monotone class if it has the following two properties:
        \begin{enumerate}
            \item For every increasing sequence $A_1\subset A_2\subset\cdots$:\ \[\bigcup_{i=1}^{+\infty}A_i\in\mathcal{A}.\]
            \item For every decreasing sequence $A_1\supset A_2\supset\cdots$:\ \[\bigcap_{i=1}^{+\infty}A_i\in\mathcal{A}.\]
        \end{enumerate}
    }

    \begin{theorem}[Monotone class theorem]\label{set:theorem:monotone_class}
        Let $\mathcal{A}$ be an algebra of sets \ref{set:algebra_of_sets}. If $\mathcal{G}_\mathcal{A}$ is the smallest monotone class containing $\mathcal{A}$ then it coincides with the $\sigma$-algebra generated by $\mathcal{A}$.
    \end{theorem}

\section{Functions}
\subsection{Domain}

    \newdef{Domain}{\index{domain}
        Let $f:X\rightarrow Y$ be a function. The set $X$ (containing the arguments of $f$) is called the domain of $f$.
    }
    \begin{notation}
        The domain of $f$ is denoted by $\text{dom}(f)$.
    \end{notation}

    \newdef{Support}{\index{support}
        Let $f:X\rightarrow\mathbb{R}$ be a function with an arbitrary domain $X$. The support of $f$ is defined as the set of points where $f$ is non-zero.
    }
    \begin{notation}
        The support of $f$ is denoted by $\text{supp}(f)$.
    \end{notation}

    \begin{notation}\label{set:function_set}
        \nomenclature[S_YX]{$Y^X$}{Set of functions from a set $X$ to a set $Y$.}
        Let $X, Y$ be two sets. The set of functions $f:X\rightarrow Y$ is denoted by $Y^X$. (See also definition \ref{category:exponential_object} for a generalization.)
    \end{notation}

\subsection{Codomain}

    \newdef{Codomain}{\index{codomain}
        Let $f:X\rightarrow Y$ be a function. The set $Y$ is called the codomain of $f$.
    }
    \newdef{Image}{\index{image}
        Let $f:X\rightarrow Y$ be a function. The following subset of $Y$ is called the image of $f$:
        \begin{gather}
            \{y\in Y\ :\ \exists x\in X:f(x) = y\}.
        \end{gather}
    }
    \begin{notation}
        The image of a function $f$ is denoted by $\text{im}(f)$.
    \end{notation}
    \sremark{Some authors use these two notions interchangeably.}

    \newdef{Level set}{\index{level set}\label{set:level_set}
        Consider a function $f:X\rightarrow\mathbb{R}$. The following set is called the level set of $f$ at $c\in\mathbb{R}$:
        \begin{gather}
            L_c(f) := f^{-1}(c) \equiv \{x\in X:f(x) = c\}.
        \end{gather}
        For $X=\mathbb{R}^2$ the level sets are called \textbf{level curves} and for $X = \mathbb{R}^3$ they are called \textbf{level surfaces}.
    }

\subsection{Types of functions}

    \newdef{Injective}{\index{injective}\index{one-to-one|see{injective}}\label{set:injective}
        \nomenclature[O_inj]{$\hookrightarrow$}{injective function}
        A function $f:A\rightarrow B$ is called injective or \textbf{one-to-one} if the following condition is satisfied:
        \begin{gather}
            \forall a, a'\in A:f(a)=f(a')\implies a=a'.
        \end{gather}
    }
    \newnot{Injective map}{\[f:A\hookrightarrow B\]}

    \newdef{Surjective}{\index{surjective}\index{onto|see{surjective}}\label{set:surjective}
        \nomenclature[O_surj]{$\twoheadrightarrow$}{surjective function}
        A function $f:A\rightarrow B$ is called surjective or \textbf{onto} if the following condition is satisfied:
        \begin{gather}
            \forall b\in B, \exists a\in A:f(a) = b.
        \end{gather}
    }
    \newnot{Surjective map}{\[f:A\twoheadrightarrow B\]}

    \newnot{Isomorphic}{
        \nomenclature[O_isom]{$\cong$}{is isomorphic to}
        If two sets $X, Y$ are isomorphic we denote this by \[X\cong Y.\]
    }

\section{Partitions}
\subsection{Partition}

    \newdef{Composition}{\index{composition}
        Let $k,n\in\mathbb{N}$. A $k$-composition of $n$ is a $k$-tuple $(t_1,\ldots, t_k)$ such that $\sum_{i=1}^kt_k = n$.
    }
    \newdef{Partition}{\index{partition}
        Let $n\in\mathbb{N}$. A partition of $n$ is an ordered composition of $n$. Hence multiple different composition can determine the same partition.
    }

    \newdef{Young diagram\footnotemark}{\index{Young!diagram}\index{Ferrers diagram|see{Young diagram}}
        \footnotetext{Sometimes called a \textbf{Ferrers diagram}.}
        A Young diagram is a visual representation of the partition of an integer $n$. It is a left justified system of boxes, where every row corresponds to a part of the partition:
        \begin{figure}[!ht]
            \centering
            \ydiagram{5, 4, 4, 1}
            \caption{A Young diagram representing the partition $(5, 4, 4, 1)$ of 14.}
            \label{fig:young_diagram}
        \end{figure}
    }
    \newdef{Conjugate partition}{
        Let $\lambda$ be a partition of $n$ with associated Young diagram $\mathcal{D}$. The conjugate partition $\lambda'$ is obtained by reflecting $\mathcal{D}$ across its main diagonal.
    }
    \begin{example}
        Conjugating diagram \ref{fig:young_diagram} gives us diagram \ref{fig:young_diagram_conj} below. The associated partition is $(4, 3, 3, 3, 1)$.
        \begin{figure}[!ht]
            \centering
            \ydiagram{4, 3, 3, 3, 1}
            \caption{A Young diagram representing the partition $(4, 3, 3, 3, 1)$ of 14.}
            \label{fig:young_diagram_conj}
        \end{figure}
    \end{example}

    \newdef{Young tableau}{\index{Young!tableau}
        Consider a Young diagram of shape $\lambda$. A Young tableau of shape $\lambda$ is a filling of the corresponding Young diagram by the elements of a totally ordered set (with $n$ elements). This tableau is said to be \textbf{standard} if every row and every column is increasing.
    }

    \begin{formula}[Hook length formula]\index{hook length}
        The \textbf{hook} $H_{i,j}$ is defined as the part of a Young diagram given by the cell $(i,j)$ together with all cells below and to the right of $(i,j)$. Given a hook $H_{i,j}$ we define the hook length $h_{i,j}$ as the sum of all elements in $H_{i,j}$.

        The number of all possible standard Young tableaux of shape $\lambda$ (where $\lambda$ defines a partition of $n$) is given by the following formula:
        \begin{gather}
            f^\lambda = \frac{n!}{\prod_{(i,j)\in\lambda}h_{i,j}}.
        \end{gather}
    \end{formula}

    \newdef{Young tabloid}{
        A Young tabloid of shape $\lambda$ is defined as the equivalence class of Young tableaux which are connected by permuting the elements within a row. These are often drawn as in figure \ref{fig:young_tabloid}.
        \begin{figure}[!ht]
            \centering
            \ytableausetup{boxsize=normal,tabloids}\begin{ytableau}1&2&3&5&8\\ 4&6&9&10\\ 7&11&12&14\\ 15\end{ytableau}
            \caption{A Young tabloid associated to the Young diagram in figure \ref{fig:young_diagram}.}
            \label{fig:young_tabloid}
        \end{figure}
    }

\subsection{Superpartition}

    For the physical background of the notions introduced in this section, see chapter \ref{chapter:mathematical_formalism_qm}.

    \newdef{Superpartition}{\index{partition}\index{fermion}
        Let $m,n\in\mathbb{N}$. A superpartition in the $m$-\textit{fermion sector} is a sequence of integers of the following form:
        \begin{gather}
            \Lambda = (\Lambda_1,\ldots,\Lambda_m;\Lambda_{m+1},\ldots,\Lambda_n).
        \end{gather}
        where the first $m$ numbers are strictly ordered, i.e. $\Lambda_i>\Lambda_{i+1}$ for all $i< m$, and the last $n-m$ numbers form a normal partition.

        Both sequences, separated by a semicolon, form in fact distinct partitions themself. The first one represents the \textit{antisymmetric fermionic} sector (this explains the strict order) and the second one represents the \textit{symmetric bosonic} sector. This amounts to the following notation:\[\Lambda \equiv (\lambda^a;\lambda^s).\]
        The degree of the superpartition is given by $|\Lambda|=\sum_{i=1}^n\Lambda_i$.
    }
    \begin{notation}
        A superpartition of degree $n$ in the $m$-fermion sector is said to be a superpartition of $(n|m)$. To every superpartition $\Lambda$ we can also associate a unique partition $\Lambda^*$ by removing the semicolon and reordering the numbers such that they form a partition of $n$. The superpartition $\Lambda$ can then be represented by the Young diagram belonging to $\Lambda^*$ where the rows belonging to the fermionic sector are ended by a circle.
    \end{notation}

\section{Axiomatization}\label{set:section:axiomatization}
\subsection{ZFC}\nomenclature[A_ZFC]{ZFC}{Zermelo-Frenkel set theory with the axiom of choice.}

    The following set of axioms and axiom schemata gives a basis for axiomatic set theory that fixes a number of issues in naive set theory where one takes the notion of set for granted. This theory is called \textbf{Zermelo-Frenkel} set theory (ZF). When extended with the axiom of choice (see further) it is called ZFC, where the C stand for "choice".

    \begin{axiom}[Power set]\index{power!set}\label{set:power_set_axiom}
        \begin{gather}
            \forall x:\exists y: \forall z\big[z\in y\iff \forall w(w\in z\implies w\in x)\big]
        \end{gather}
        The set $y$ is called the power set $P(x)$ of $x$.
    \end{axiom}

    \begin{axiom}[Extensionality]\index{extensionality}
        \begin{gather}
            \forall x, y:\forall z\big[z\in x \iff z\in y\big]\implies x=y
        \end{gather}
        This axiom allows us to compare two sets based on their elements.
    \end{axiom}

    \begin{axiom}[Regularity\footnotemark]\index{regularity}
        \footnotetext{Also called the \textbf{axiom of foundation}.}
        \begin{gather}
            \forall x:\exists z\big[z\in x\big]\implies \exists a\big[a\in x \land \neg\exists b(b\in a \land b\in x)\big]
        \end{gather}
        This axiom says that for every non-empty set $x$ one can find an element $a\in x$ such that $x$ and $a$ are disjoint. Among other things this axiom implies that no set can contain itself.
    \end{axiom}

    The following axiom is technically not an axiom but an axiom schema, i.e. for every predicate $\varphi$ one obtains an axiom:
    \begin{axiom}[Specification]\index{specification}
        \begin{gather}
            \forall w_1,\ldots,w_n, A:\exists B:\forall x\big[x\in B\iff(x\in A\land \varphi(x, w_1,\ldots,w_n,A)\big]
        \end{gather}
        This axiom (schema) says that for every set $x$ one can build another set of elements in $x$ that satisfy a given predicate. By the axiom of extensionality this subset $B\subseteq A$ is unique.
    \end{axiom}

\subsection{Material set theory}

    ZF(C) is an instance of material set theory. Every element of a set is a set itself and hence has some kind of internal structure.

    \newdef{Pure set}{\index{pure!set}
        A set $U$ is pure if for every sequence $x_n\in x_{n-1} \in\cdots\in x_1\in U$ all the elements $x_i$ are also sets.
    }
    \newdef{Urelement\footnotemark}{\index{urelement}\index{atom}
        \footnotetext{Sometimes called an \textbf{atom}.}
        An object that is not a set.
    }

\subsection{Universes}\label{section:universes}

    ??TODO??

    To be able to talk about sets without running into problems, such as Russel's paradox, where one needs (or wants) to talk about the collection of all things satisfying a certain condition, one can introduce the concept of a universal set or universe. This set takes the place of the ''collection of things'' and all operations performed on its elements, i.e. the sets that we want to work with, act within this universe.

    \newdef{Grothendieck universe}{\index{Grothendieck!universe}
        A Grothendieck universe $U$ is a pure set satisfying the following axioms:
        \begin{enumerate}
            \item \textbf{Transitivity}: If $x\in U$ and $y\in x$ then $y\in U$.
            \item \textbf{Power set}: If $x\in U$ then $P(x)\in U$.
            \item \textbf{Pairing}: If $x, y\in U$ then $\{x, y\}\in U$.
            \item \textbf{Unions}: If $I\in U$ and $\{x_i\}_{i\in I} \subset U$ then $\bigcup_{i\in I}x_i\in U$.
        \end{enumerate}
    }

\subsection{Structural set theory}

    In contrast to material set theory, the fundamental notions in this theory are sets and the relations between them. An element of a set does not have any internal structure and so only becomes relevant if one specifies extra structure (or relations) on the sets. This implies that elements of sets are not sets themselves as this is a meaningless statement, since by default they lack internal structure. Even stronger, it is meaningless to compare two elements if one does not provide relations or extra structure on the sets.

    ?? COMPLETE ??

\subsection{ETCS}\nomenclature[A_ETCS]{ETCS}{Elementary Theory of the Category of Sets}

    ?? COMPLETE ??

    \sremark{ETCS is the abbreviation of ''Elementary Theory of the Category of Sets''.}

    \begin{axiom}
        The category of sets is a well-pointed (elementary) topos.
    \end{axiom}

\subsection{Real numbers}\index{real numbers}

    \begin{axiom}[Ordering]
        The set of real numbers is an ordered field $(\mathbb{R},+,\cdot,<)$.
    \end{axiom}
    \begin{axiom}[Dedekind completeness]\index{Dedekind!completeness}
        Every non-empty subset of $\mathbb{R}$ that is bounded above has a supremum.
    \end{axiom}

    \begin{axiom}
        The rational numbers form a subset of the real numbers: $\mathbb{Q}\subset\mathbb{R}$.
    \end{axiom}
    \sremark{There is only one way to extend the field of rational numbers to the field of reals such that it satisfies the two previous axioms. This means that for every two possible constructions, there exists a bijection between the two.}

    \newdef{Extended real line}{
        \begin{gather}
            \label{calculus:extended_real_line}
            \overline{\mathbb{R}} := \mathbb{R}\cup\{-\infty,\infty\} \equiv [-\infty, \infty]
        \end{gather}
    }

\chapter{Algebra}
\section{Groups}

    	\newdef{Semigroup}{\index{semigroup}
        	Let $G$ be a set equipped with a binary operation $\star$. $(G,\star)$ is a semigroup if it satisfies following axioms:
	\begin{enumerate}
				\item $G$ is closed under $\star$
                \item $\star$ is asssociative
	\end{enumerate}
        }
        
        \newdef{Monoid}{\index{monoid}
        	Let $M$ be a set equipped with a binary operation $\star$. $(M,\star)$ is a monoid if it satisfies following axioms:
            \begin{enumerate}
		\item $M$ is closed under $\star$
		\item $\star$ is associative
		\item $M$ contains an identity element with respect to $\star$
	\end{enumerate}
        }
        
        \newdef{Group}{\index{group}
        	Let $G$ be a set equipped with a binary operation $\star$. $(G,\star)$ is a group if it satisfies following axioms:
            \begin{enumerate}
		\item $G$ is closed under $\star$
		\item $\star$ is asssociative
		\item $G$ has an identity element with respect to $\star$
		\item Every element in $G$ has an inverse element with respect to $\star$
	\end{enumerate}
        }
        
        \newdef{Commutative group\footnotemark}{\index{commutativity}
        	Let $(G, \star)$ be a group. If $\star$ is commutative, then $G$ is called a commutative group.
        }
        \footnotetext{Also called an Abelian group.}\index{Abel!Abelian group}
        
        \newdef{Coset}{\index{coset}\index{normal}\label{group:coset}
        	Let $G$ be a group and $H$ a subgroup of $G$. The left coset of $H$ with respect to $g\in G$ is defined as the set
            \begin{equation}
            	gH = \{gh: h\in H\}
            \end{equation}
            The right coset is analogously defined as $Hg$. If for all $g\in G$ the left and right cosets coincide then the subgroup $H$ is said to be a \textbf{normal subgroup}. The sets of left and right cosets are denoted by $G/H$ and $G\backslash H$ respectively.
        }
        
        \newdef{Center}{\index{center}
        	The center of a group is defined as follows:
            \begin{equation}
            	\label{group:center}
                Z(G) = \{z\in G: \forall g\in G , zg = gz\}
            \end{equation}
            This set is a subgroup of $G$.
        }
        
        \newdef{Order of a group}{\index{order}
        	The number of elements in the group. It is denoted by $|G|$ or $\text{ord}(G)$.
        }
        \newdef{Order of an element}{
        	The order of an element $a\in G$ is the smallest integer $n$ such that
        	\begin{equation}
        		a^n = e
        	\end{equation}
        	where $e$ is the identity element of $G$.
        }
        \newdef{Torsion group}{\index{torsion}\label{group:torsion_group}
        	A torsion group is a group for which all element have finite order. The torsion $\text{Tor}(G)$ of a group $G$ is the set of all element $a\in G$ that have finite order. For Abelian groups, $\text{Tor}(G)$ is a subgroup.
        }
        
        \newdef{Quotient group}{\index{quotient!group}
        	\label{group:quotient_group}
        	Let $G$ be a group and $N$ a normal subgroup. The quotient group $G/N$ is defined as the set of left cosets of $N$ in $G$. This set can be turned into a group itself by equipping it with a product such that the product of $aN$ and $bN$ is $(aN)(bN)$. The fact that $N$ is a normal subgroup can be used to rewrite this as $(aN)(bN) = (ab)N$.
        }


\subsection{Symmetric and alternating groups}
		\newdef{Symmetric group}{
        	The symmetric group $S_n$ or $\text{Sym}_n$ of the set $V = \{1, 2, ..., n\}$ is defined as the set of all permutations of $V$.
        }
        \newdef{Alternating group}{
        	The alternating group $A_n$ is the subgroup of $S_n$ containing all even permutations.
        }
        
        \newdef{Cycle}{\index{cycle}
        	A $k$-cycle is a permutation of the form $(a_1, a_2, ..., a_k)$ sending $a_i$ to $a_{i+1}$ (and $a_k$ to $a_1$). A \textbf{cycle decomposition} of an arbitrary permutation is the decomposition into a product of disjoint cycles.
        }
        \begin{formula}
        	Let $\tau$ be a $k$-cycle. The following equality holds:
            \begin{equation}
            	\tau^k = \mathbbm{1}_G
            \end{equation}
        \end{formula}
        \begin{example}
        	Consider the set $\{1, 2, 3, 4, 5, 6\}$. The permutation $\sigma:x\mapsto x-2$ can be written as the cycle decomposition $\sigma = (1,3,5)(2,4,6)$.
        \end{example}
        
        \newdef{Transposition}{\index{transposition}
        	A permutation which exchanges two elements but lets the other ones unchanged.
        }

\subsection{Direct product}

	\newdef{Direct product}{\index{direct product! of groups}\label{group:direct_product}
		Let $G, H$ be two groups. The direct product $G\otimes H$ is defined as the set-theoretic Cartesian product $G\times H$ equipped with a binary operation $\cdot$ such that:
		\begin{equation}
			(g_1, h_1)\cdot(g_2, h_2) = (g_1g_2, h_1h_2)
		\end{equation}
		where the operations on the right hand side are the group operations in $G$ and $H$. The structure $G\otimes H = (G\times H, \cdot)$ is a group itself.
	}
	\begin{notation}
		When the groups are Abelian, the direct product is often denoted by $\oplus$.
	\end{notation}
	
	\newdef{Inner semidirect product}{\index{split}
		Let $G$ be a group, $H$ a subgroup of $G$ and $N$ a normal subgroup of $G$. $G$ is said to be the direct product of $H$ and $N$, denoted by $H\rtimes N$, if it satifies the following equivalent statements:
		\begin{itemize}
			\item $G = NH$ where $N\cap H = \{e\}$.
			\item For every $g\in G$ there exist unique $n\in N, h\in H$ such that $g=nh$.
			\item For every $g\in G$ there exist unique $h\in H, n\in N$ such that $g=hn$.
			\item There exists a group homomorphism $\rho:G\rightarrow H$ which satisfies $\rho|_H = e$ and $\ker(\rho)=N$.
			\item The composition of the natural embedding $i:H\rightarrow G$ and the projection $\pi:G\rightarrow G/N$ is a isomorphism between $H$ and $G/N$.
		\end{itemize}
		$G$ is also said to \textbf{split} over $N$.
	}
	\newdef{Outer semidirect product}{
		Let $G, H$ be two groups and let $\varphi:H\rightarrow\text{Aut}(G)$ be a group homomorphism. The outer semidirect product $G\rtimes_\varphi H$ is defined as the set-theoretic Cartesian product $G\times H$ equipped with a binary relation $\cdot$ such that:
		\begin{equation}
			(g_1, h_1)\cdot(g_2, h_2) = (g_1\varphi(h_1)(g_1), h_1h_2)
		\end{equation}
		The structure $(G\rtimes_\varphi H, \cdot)$ is a group itself.
		
		By noting that the set $N = \{(g, e_H)|g\in G\}$ is a normal subgroup isomorphic to $G$ and that the set $B = \{(e_G, h)|h\in H\}$ is a subgroup isomorphic to $H$ we can also construct the outer semidirect product $G\rtimes_\varphi H$ as the inner semidirect product $B\rtimes N$.
	}
	
	\begin{remark}
		The direct product of groups is a special case of the outer semidirect product where the group homomorphism is given by the trivial map $\varphi:h\mapsto e_G$.
	\end{remark}


\subsection{Free groups}

	\newdef{Free Abelian group}{\index{free!group}\index{basis}\index{rank}
		An abelian group $G$ with generators $\{g_i\}_{i\in I}$ is said to be freely generated if every element $g\in G$ can be uniquely written as a formal linear combination of the generators:
		\begin{equation}
			G = \left\{\left.\sum_ia_ig_i\right|a_i\in\mathbb{Z}\right\}
		\end{equation}
		The set of generators $\{g_i\}_{i\in I}$ is then called a \textbf{basis}\footnote{In analogy with the basis of a vector space.}\ of $G$. The number of elements in the basis is called the \textbf{rank} of $G$.
	}
	\begin{property}
		Consider a free group $G$. Let $H\subset G$ be a subgroup. Then $H$ is also free.
	\end{property}
	
	\begin{theorem}\label{group:theorem:free_group}
		Let $G$ be a finitely generated Abelian group of rank $n$, i.e. its basis has $n$ elements. This group can be constructed in two different ways:
		\begin{equation}
			G = F/H
		\end{equation}
		where both $F, H$ are freely and finitely generated Abelian groups. The second decomposition is:
		\begin{equation}
			G = A\oplus T\qquad\text{where}\qquad T = Z_{h_1}\oplus\cdots\oplus Z_{h_m}
		\end{equation}
		where $A$ is a freely and finitely generated group of rank $n-m$ and all $Z_{h_i}$ are cyclic groups of order $h_i$. The group $T$ is called the torsion subgroup\footnote{See also definition \ref{group:torsion_group}.}. The rank $n-m$ and the numbers $h_i$ are unique.
	\end{theorem}

\subsection{Group presentations}

	\newdef{Relations}{\index{relation}
		Let $G$ be a group. If the product of a number of elements $g\in G$ is equal to the identity $e$ then this product is called a relation on $G$.
	}
	\newdef{Complete set of relations}{
		Let $H$ be a group generated by a subgroup $G$. Let $R$ be a set of relations on $G$. If $H$ is uniquely (up to an isomorphism) determined by $G$ and $R$ then the set of relations is said to be complete.
	}

	\newdef{Presentation}{\index{presentation}
		Let $H$ be a group generated by a subgroup $G$ and a complete set of relations $R$ on $G$. The pair $(G, R)$ is called a presentation of $H$.
		
		It is clear that every group can have many different presentations and that it is (very) difficult to tell if two groups are isomorphic by just looking at their presentations.
	}

\subsection{Group actions}
        \newdef{Group homomorphism}{\index{homomorphism!of groups}
        	A group homomorphism $\Phi:G\rightarrow H$ is a map satisfying $\forall g, h \in G$
            \begin{equation}
            	\Phi(gh) = \Phi(g)\Phi(h)
            \end{equation}
        }
        
        \newdef{Kernel}{\index{kernel}
        	The kernel of a group homomorphism $\Phi:G\rightarrow H$ is defined as the set
            \begin{equation}
            	K = \{g\in G: \Phi(g) = \mathbbm{1}_H\}
            \end{equation}
        }
        \begin{theorem}[First isomorphism theorem]\index{isomorphism!theorem}\label{group:theorem:first_isomorphism_theorem}
        	Let $G, H$ be a groups and let $\varphi:G\rightarrow H$ be a group homomorphism. If $\varphi$ is surjective than $G/\ker\varphi\cong H$.
        \end{theorem}
        
        \newdef{Group action}{\index{group!action}\label{group:group_action}
            Let $G$ be a group. Let $V$ be a set. A map $\rho: G\times V \rightarrow V$ is called an action of $G$ on $V$ if it satisfies the following conditions:
            \begin{itemize}
                \item Identity: $\rho(\mathbbm{1}_G, v) = v$
                \item Compatibility: $\rho(gh, v) = \rho(g, \rho(h, v))$
            \end{itemize}
            For all $g, h \in G$ and $v\in V$. The set V is called a (left) \textbf{G-space}.
        }
        \remark{\label{group:permutation_remark}
        	A group action can alternatively be defined as a group homomorphism from $G$ to $\text{Sym}(V)$. It assigns a permutation of $V$ to every element $g\in G$.
	}
        \begin{notation}
        	The action $\rho(g, v)$ is often denoted by $g\cdot v$ or even $gv$.
        \end{notation}
        
	\newdef{Orbit}{\index{orbit}
		The orbit of an element $x\in X$ with respect to a group $G$ is defined as the set:
		\begin{equation}
			\label{group:orbit}
			x\cdot G = \{x\cdot g|g\in G\}
		\end{equation}
	}
	\newdef{Stabilizer}{\index{stabilizer}\index{isotropy group}
		The stabilizer group or \textbf{isotropy group} of an element $x\in X$ with respect to a group $G$ is defined as the set:
		\begin{equation}
			G_x = \{g\in G|g \cdot x = x\}
		\end{equation}
		This is a subgroup of $G$.
	}
	
	\newdef{Free action}{\index{free}\label{group:free_action}
		A group action is free if $g\cdot x = x$ implies $g = e$ for every $x\in X$.
	}
	\newdef{Faithful action\footnotemark}{\index{faithful!action}\label{group:faithful_action}
		\footnotetext{A faithful action is also called an \textbf{effective} action.}
		A group action is faithful if the homomorphism $G\rightarrow\text{Sym}(X)$ is injective. Alternatively, a group action is faithful if for every two group elements $g, h\in G$ there exists an element $x\in X$ such that $g\cdot x\neq h\cdot x$.
	}
	
	\newdef{Transitive action}{\index{transitive!action}\label{group:transitive}
		A group action is transitive if for every two elements $x, y\in X$ there exists a group element $g\in G$ such that $g\cdot x = y$. Equivalently we can say that there is only one orbit.
	}
	\newdef{Homogeneous space}{\index{homogeneous!space}
		If the group action of a group $G$ on a $G$-space $X$ is transitive, then $X$ is said to be a homogeneous space.
	}
	\begin{property}[$\dag$]\label{group:transitive_action_property}
		Let $X$ be a set and let $G$ be a group such that the action of $G$ on $X$ is transitive. Then their exists a bijection $X\cong G/G_x$ where $G_x$ is the stabilizer of any element $x\in X$.
	\end{property}
        
        \newdef{G-module}{\index{module}
        	Let $G$ be a group. Let $M$ be a commutative group. $M$ equipped with a left group action $\varphi:G\times M\rightarrow M$ is a (left) G-module if $\varphi$ satisfies the following equation (distributivity):
            \begin{equation}
            	\label{group:g_module}
                g\cdot(a+b) = g\cdot a + g\cdot b
            \end{equation}
            where $a, b\in M$ and $g\in G$.
        }
        \newdef{G-module homomorphism}{\index{homomorphism!of G-modules}\index{equivariant}\label{group:equivariant}
        	A G-module homomorphism is a map $f:V\rightarrow W$ satisfying
            \begin{equation}
            	g\cdot f(v) = f(g\cdot v)
            \end{equation}
            where the $\cdot$ symbol represents the group action in $W$ and $V$ respectively. It is sometimes called a \textbf{G-map}, a \textbf{G-equivariant map} or an \textbf{intertwining map}.
        }
        
\section{Rings}
	
	\newdef{Ring}{\index{ring}
		Let $R$ be a set equipped with two binary operations $+,\cdot$ (called addition and multiplication). $(R,+,\cdot)$ is a ring if it satisfies the following axioms:
    		\begin{enumerate}
			\item $(R,+)$ is a commutative group.
			\item $(R,\cdot)$ is a monoid.
			\item Multiplication is distributive with respect to addition.
		\end{enumerate}
	}
	
	\newdef{Unit}{\index{unit}
		An invertible element of ring $(R, +, \cdot)$. The set of units forms a group under multiplication.
	}

\subsection{Ideals}\index{ideal}

    	\newdef{Ideal}{\label{linalgebra:ideal}
    		Let $(R,+,\cdot)$ be a ring with $(R,+)$ its additive group. A subset $I\subseteq R$ is called an ideal\footnotemark\ of $R$ if it satisfies the following conditions:
        	\begin{enumerate}
			\item $(I,+)$ is a subgroup of $(R,+)$
                	\item $\forall n\in I, \forall r\in R:(n\cdot r), (r\cdot n)\in I$
		\end{enumerate}
		\footnotetext{More generally: two-sided ideal}
        }
        
        \newdef{Unit ideal}{Let $(R,+,\cdot)$ be a ring. $R$ itself is called the unit ideal.}
        \newdef{Proper ideal}{Let $(R,+,\cdot)$ be a ring. A subset $I\subset R$ is said to be a proper ideal if it is an ideal of $R$ and if it is not equal to $R$.}
        \newdef{Prime ideal}{Let $(R,+,\cdot)$ be a ring. A proper ideal $I$ is a prime ideal if for any $a,b\in R$ the following relation holds:
        	\[ab\in I\implies \text a\in I \vee b\in I\]
        }
        \newdef{Maximal ideal}{Let $(R,+,\cdot)$ be a ring. A proper ideal $I$ is said to be maximal if there exists no other proper ideal $T$ in R such that $I\subset T$.}
        \newdef{Minimal ideal}{A proper ideal is said to be minimal if it contains no other nonzero ideal.}
        
	\newdef{Generating set}{\index{generating set! of an ideal}\label{group:generating_set_ideal}
		Let $R$ be a ring and let $X$ be a subset of $R$. The two-sided ideal generated by $X$ is defined as the intersection of all two-sided ideals containing $X$. An explicit construction is given by:
		\begin{equation}
			I = \left\{\left.\sum_{i=1}^n l_ix_ir_i\ \right\vert\ \forall i\leq n: l_i, r_i\in R\text{ and } x_i\in X\right\}
		\end{equation}
		Left and right ideals are generated in a similar fashion.
	}
        
\subsection{Graded rings}
	
	\newdef{Graded ring}{\index{graded}\label{group:graded_ring}
		Let $R$ be a ring that can be written as the direct sum of Abelian groups $A_k$:
		\begin{equation}
			R = \bigoplus_{k\in\mathbb{N}}A_k
		\end{equation}
		If $R$ has the property that for every $i, j\in\mathbb{N}: A_i\star A_j\subseteq A_{i+j}$, where $\star$ is the ring multiplication, then $R$ is said to be a graded ring. The elements of the space $A_k$ are said to be \textbf{homogeneous of degree $k$}.
	}
	
	\newformula{Graded commutativity}{\index{commutativity!graded}
		Let $m = \deg v$ and let $n = \deg w$. If
		\begin{equation}
			\label{group:graded_commutativity}
			vw = (-1)^{mn}wv
		\end{equation}
		for all elements $v, w$ of the graded ring then it is said to be a graded-commutative ring.
	}
        

\section{Rings}\label{section:ring}

    \newdef{Ring}{\index{ring}
        Let $R$ be a set equipped with two binary operations $+,\cdot$ (called \textbf{addition} and \textbf{multiplication}). $(R,+,\cdot)$ is a ring if it satisfies the following axioms:
        \begin{enumerate}
            \item $(R,+)$ is an Abelian group.
            \item $(R,\cdot)$ is a monoid.
            \item Multiplication is distributive with respect to addition.
        \end{enumerate}
    }
    \newdef{Field}{\index{field}
        A ring $(R,+,\cdot)$ for which the monoid $(R\backslash\{1_+\},\cdot)$ is an Abelian group and $1_+\neq 1_\cdot$.
    }

    \newdef{Unit}{\index{unit}
        An invertible element of a ring $(R,+,\cdot)$. The set of units forms a group under multiplication.
    }

    \newdef{Integral domain}{\index{domain!integral}\label{algebra:integral_domain}
        A commutative ring $R$ in which the product of two nonzero elements is again nonzero.
    }

    \newdef{Reduced ring}{\index{ring!reduced}
        A ring that contains no nonzero nilpotents.
    }

    \begin{construct}[Localization]\index{localization}
        Let $R$ be a commutative ring and let $S$ be a multiplicatively closed set in $R$. Define an equivalence relation $\sim$ on $R\times S$ in the following way:
        \begin{gather}
            (r_1,s_1)\sim(r_2,s_2) \iff \exists t\in S:t(r_1s_2 - r_2s_1) = 0.
        \end{gather}
        The set $S^{-1}R:=(R\times S)/\sim$, called the localization of $R$ with respect to $S$, can now be turned into a ring by defining an addition and a multiplication. By writing $(r,s)\in S^{-1}R$ as the formal fraction $\frac{r}{s}$, these operations are defined in analogy with the those of ordinary fractions:
        \begin{itemize}
            \item\textbf{Addition}: $\displaystyle\frac{r_1}{s_1} + \frac{r_2}{s_2} = \frac{r_1s_2 + r_2s_1}{s_1s_2}$,
            \item\textbf{Multiplication}: $\displaystyle\frac{r_1}{s_1}\cdot\frac{r_2}{s_2} = \frac{r_1r_2}{s_1s_2}$.
        \end{itemize}
    \end{construct}
    \remark{The localization of $R$ with respect to the set $S$ can be interpreted as the ring obtained by collapsing $S$ into a single unit of $R$.}

    \begin{notation}\label{algebra:localization_notation}
        For specific cases different notations are sometimes used. For example, choose an element $f\in R$ and let $R_f$ denote the localization of $R$ with respect to the set of powers of $f$, i.e. $S=\{f^n\mid n\in\mathbb{N}\}$. This is called the \textbf{localization at (the element)} $f$. Another example occurs when working with prime ideals. Let $P$ be a prime ideal (see the next section). It is not hard to show that the complement $R\backslash P$ is multiplicatively closed. The localization of $R$ with respect to this set is denoted by $R_P$ and is called the \textbf{localization at (the prime ideal)} $P$.
    \end{notation}

    \newdef{Valuation}{\index{valuation}
        Let $k$ be a field and let $\Gamma$ be a totally ordered\footnote{Definition \ref{set:total_order}.}, Abelian group. The group law and the order relation on $\Gamma$ can be extended to the union $\Gamma\cup\{\infty\}$ in the following way (the notation $\infty$ is only a convention):
        \begin{itemize}
            \item $g+\infty:=\infty+g:=\infty$ for all $g\in\Gamma$, and
            \item $g\leq\infty$ for all $g\in\Gamma$.
        \end{itemize}
        A valuation on $k$ (with values in $\Gamma$) is a map $\nu:k\rightarrow\Gamma\cup\{\infty\}$ such that:
        \begin{enumerate}
            \item $\nu(a) = \infty\iff a = 0$;
            \item $\nu(ab) = \nu(a) + \nu(b)$; and
            \item $\min(\nu(a),\nu(b))\leq\nu(a+b)$, where the equality holds if $\nu(a)\neq\nu(b)$.
        \end{enumerate}
    }

\subsection{Ideals}

    \newdef{Ideal}{\index{ideal}\label{algebra:ideal}
        Let $(R,+,\cdot)$ be a ring with $(R,+)$ its additive group. A subset $I\subseteq R$ is called a (two-sided) ideal of $R$ if it satisfies the following conditions:
        \begin{enumerate}
            \item $(I,+)$ is a subgroup of $(R,+)$.
            \item $\forall n\in I,\forall r\in R:n\cdot r,r\cdot n\in I$.
        \end{enumerate}
    }

    \newdef{Artinian ring}{\index{Artin!ring}
        A ring is said to be Artin(ian) if it satisfies the \textbf{descending chain condition} on ideals, i.e. if it contains no infinite descending chain \ref{set:chain} of ideals.
    }
    \newdef{Noetherian ring}{\index{Noether!ring}
        A ring is said to be Noether(ian) if it satisfies the \textbf{ascending chain condition} on ideals, i.e. if it contains no infinite ascending chain of ideals.
    }

    \newdef{Simple ring}{\index{simple!ring}
        A ring that has no nontrivial two-sided ideals. (Some authors require the ring to be Artinian.)
    }

    \newdef{Unit ideal}{
        A ring considered as an ideal of itself.
    }
    \newdef{Proper ideal}{
        An ideal that is not equal to the ring itself.
    }
    \newdef{Prime ideal}{
        Let $R$ be a ring. A proper ideal $I\subset R$ is a prime ideal if for any $a,b\in R$ the following relation holds:
        \begin{gather}
            ab\in I\implies a\in I\lor b\in I.
        \end{gather}
    }
    \newdef{Maximal ideal}{
        A proper ideal that is not contained in another proper ideal.
    }

    \begin{property}
        Every maximal ideal is prime.
    \end{property}

    \newdef{Jacobson radical}{\index{Jacobson radical}\label{algebra:jacobson_radical}
        The Jacobson radical of a ring $R$, often denoted by $J(R)$, is the ideal obtained as the intersection of all maximal left (or right) ideals. Equivalently, it is the intersection of the \textit{annihilators} of all simple, left (or right) $R$-modules.
    }

    \begin{construct}[Generating ideals]\index{ideal!generating set}\label{algebra:generating_set_ideal}
        Let $R$ be a ring and let $X$ be a subset of $R$. The two-sided ideal generated by $X$ is defined as the intersection of all two-sided ideals containing $X$. An explicit construction is given by
        \begin{gather}
            I = \left\{\sum_{i=1}^n l_ix_ir_i\,\middle\vert\,n\in\mathbb{N}, \forall i\leq n:l_i,r_i\in R\land x_i\in X\right\}.
        \end{gather}
        Left and right ideals are generated in a similar fashion.
    \end{construct}
    \begin{notation}
        If the ideal $I$ is generated by the elements $\{f_j\}_{j\in J}$ (for some index set $J$), it is often denoted by
        \begin{gather}
            I\equiv(f_1,f_2,\ldots).
        \end{gather}
    \end{notation}

    \begin{construct}[Extension]\index{extension!ideal}
        Let $I$ be an ideal of a ring $R$ and let $\iota:R\rightarrow S$ be a ring morphism. The extension of $I$ with respect to $\iota$ is the ideal generated by the set $\iota(I)$.
    \end{construct}

    \newdef{Principal ideal}{\index{ideal!principal}
        An ideal that is generated by a single element.
    }
    \newdef{Principal ideal domain}{\index{domain!principal ideal}
        An integral domain \ref{algebra:integral_domain} in which every ideal is principal.
    }

    \newdef{Local ring}{\index{local!ring}\label{algebra:local_ring}
        A ring for which a unique, maximal, left ideal exists. This also implies that there exists a unique, maximal, right ideal and that these ideals coincide.
    }
    \begin{property}[Characterization by invertible complements]\label{algebra:local_ring_invertible}
        A ring $R$ is local if and only if there exists a maximal ideal $M$ such that every element in the complement $R\backslash M$ is invertible.
    \end{property}

    \begin{property}[Prime localization]\label{algebra:localization_local_ring}
        The localization of a ring $R$ with respect to a prime ideal $P$ is a local ring, where the maximal ideal is given by the extension of $P$ with respect to the ring morphism $\iota:R\rightarrow R_P$. Equivalently, this says that the maximal ideal is given by $PR_P$.
    \end{property}

    \newdef{Residue field}{\index{residue!field}
        Consider a local ring $R$ and let $I$ be its maximal ideal. The quotient ring $R/I$ forms a field, called the residue field.
    }

\subsection{Modules}

    \newdef{$R$-module}{\index{module!over a ring}\label{algebra:module}
        Let $(R,+,\cdot)$ be a ring. An Abelian group $(M,\oplus)$ is said to be a left $R$-module if there exists a left (monoid) action $\triangleright:(R,\cdot)\times M\rightarrow M$ that satisfies the following axioms:
        \begin{enumerate}
            \item\textbf{Left distributivity}: $r\triangleright(m\oplus n) = r\triangleright m\oplus r\triangleright n$ for all $r\in R$ and $m,n\in M$
            \item\textbf{Right distributivity}: $(r+s)\triangleright m = r\triangleright m \oplus s\triangleright m$ for all $r,s\in R$ and $m\in M$
        \end{enumerate}
        These conditions make sure that both the additive structure $(R,+)$ and the group structure $(M,\oplus)$ are compatible with the action of $(R,\cdot)$. Due to these compatibility conditions one can identify $\cdot\sim\triangleright$ and $+\sim\oplus$ without confusion.
    }
    \begin{remark}[\difficult{Categorical perspective}]
        The definition of a ring can be defined more concisely in categorical terms. Recall the definition of a algebra over a monad \ref{cat:algebra_monad}. Modules over a monoid object $A$ are defined as algebras over the monad $A\otimes-$. A ring $R$ is a monoid object in the category $\mathbf{Ab}$ of Abelian groups. So a module $M$ over $R$ consists of a morphism $\alpha:R\otimes M\rightarrow M$ satisfying the algebra axioms. The distributivity laws come for free since $\alpha$ is a morphism in $\mathbf{Ab}$ and, hence, is bilinear (in both arguments).
    \end{remark}
    \newadef{$R$-module}{
        The above two formulations can be restated similar to that of group modules \ref{group:module}. Consider the Abelian group $(M,\oplus)$. Its endomorphism set $\mathrm{End}(M,\oplus)$ can be given the structure of a ring where the addition is induced by that on $M$ and the multiplication is given by composition. A left $R$-module structure is then simply a ring morphism $R\rightarrow\mathrm{End}(M,\oplus)$.
    }

    \newdef{Free module}{\index{free!module}
        An $R$-module $M$ is said to be free if it admits a basis, i.e. there exists a set $\{x_i\}_{i\in I}$ (where $I$ can be infinite) such that:
        \begin{enumerate}
            \item every element $m\in M$ can be written as a linear combination $\sum_{j\in J}r_jx_j$, where $J\subseteq I$ is finite.
            \item the set $\{x_i\}_{i\in I}$ is linearly independent in the sense that
                \begin{gather}
                    \sum_{j\in J\subseteq I}r_jx_j=0\implies \forall j\in J:r_j=0.
                \end{gather}
        \end{enumerate}
    }
    \begin{example}[Dual numbers]\index{dual!numbers}
        Let $R$ be a ring. The $R$-algebra of dual numbers, often denoted by $R[\varepsilon]$, is defined as the free $R$-module with basis $\{1,\varepsilon\}$ subject to the relation $\varepsilon^2 = 0$.
    \end{example}

    \begin{property}[Division rings]\index{division ring}\label{algebra:module_basis}
        For a general $R$-module the existence of a basis is not guaranteed unless $R$ is a \textit{division ring}. (See Construction \ref{linalgebra:hamel_basis} for more information.)
    \end{property}
    \begin{result}
        Since every field is in particular a division ring, the existence of a basis follows from the above property for $R$-modules.
    \end{result}

    \newdef{Projective module}{\index{projective!module}
        A module $P$ is said to be projective if $P$ can be expressed as
        \begin{gather}
            P\oplus M = F,
        \end{gather}
        where $M$ is a module and $F$ is a free module, i.e. if $P$ is a direct sumand of a free module.
    }

\subsection{Semisimplicity}\index{semisimple!module}

    \newdef{Simple module}{\index{simple!module}
        A module over a ring is said to be simple if it contains no nontrivial submodules. A module is said to be \textbf{semisimple} if it admits a decomposition as a direct sum of simple modules. A ring is said to be semisimple if it is semisimple as a module over itself.
    }
    \begin{property}[Jacobson radical]
        A ring is semisimple if and only if it is Artinian and if its Jacobson radical \ref{algebra:jacobson_radical} vanishes.
    \end{property}

    \begin{theorem}[Artin-Wedderburn]\index{Artin-Wedderburn}\label{algebra:artin_wedderburn}
        Every semisimple ring is isomorphic to a direct sum of matrix rings over division rings $D_i$ with multiplicity $n_i$. Furthermore, the integers $D_i$ and $n_i$ are unique (up to a permutation of the indices).
    \end{theorem}

\section{Limits of algebraic structures}

    \newdef{Direct system}{\index{direct!system}
        Let $(I,\leq)$ be a directed set \ref{set:directed_set} and let $\{A_i\}_{i\in I}$ be a family of algebraic objects (groups, rings, ...). Consider a collection of morphisms $\{f_{ij}:A_i\rightarrow A_j\}_{i,j\in I}$ between these objects with the following properties:
        \begin{enumerate}
            \item for every $i\in I$: $f_{ii} = \mathbbm{1}_{A_i}$, and
            \item for every $i\leq j\leq k\in I$: $f_{ik} = f_{jk}\circ f_{ij}$.
        \end{enumerate}
        The pair $(A_i,f_{ij})$ is called a direct system (over $I$).
    }

    \newdef{Direct limit\footnotemark}{\index{direct!limit}\index{inductive!limit}\label{algebra:direct_limit}
        \footnotetext{Also called an \textbf{inductive limit}.}
        Consider a direct system $(A_i,f_{ij})$ over a directed set $I$. The direct limit $A$ of this direct system is defined as follows:
        \begin{gather}
            \varinjlim A_i := \left.\bigsqcup_{i\in I}A_i\right/\sim
        \end{gather}
        where the equivalence relation is given by $x\in A_i\sim y\in A_j\iff\exists k\in I: f_{ik}(x) = f_{jk}(y)$. Informally put: two elements are equivalent if they eventually become the same.

        The algebraic operations on $A$ are defined such that the inclusion maps $\phi_i:A_i\rightarrow A$ are morphisms.
    }

    \newdef{Inverse system}{\index{inverse!system}
        Let $(I,\leq)$ be a directed set \ref{set:directed_set} and let $\{A_i\}_{i\in I}$ be a family of algebraic objects (groups, rings, ...). Consider a collection of morphisms $\{f_{ij}:A_j\rightarrow A_i\}_{i,j\in I}$ between these objects with the following properties:
        \begin{enumerate}
            \item for every $i\in I$: $f_{ii} = \mathbbm{1}_{A_i}$, and
            \item for every $i\leq j\leq k\in I$: $f_{ik} = f_{ij}\circ f_{jk}$.
        \end{enumerate}
        The pair $(A_i,f_{ij})$ is called an inverse system (over $I$).
    }

    \newdef{Inverse limit\footnotemark}{\index{inverse!limit}\index{projective!limit}\label{algebra:inverse_limit}
        \footnotetext{Also called a \textbf{projective limit}.}
        Consider an inverse system $(A_i,f_{ij})$ over a directed set $I$. The inverse limit $A$ of this inverse system is defined as follows:
        \begin{gather}
            \varprojlim A_k := \left\{\vec{a}\in\prod_{i\in I}A_i\,\middle\vert\,a_i=f_{ij}(a_j), \forall i\leq j\right\}.
        \end{gather}
        For all $i\in I$ there exists a natural projection $\pi_i:\varprojlim A_k\rightarrow A_i$.
    }

    \begin{remark}
        The direct and inverse limit are each other's (categorical) dual. The former is a colimit while the latter is a limit in category theory.
    \end{remark}

\section{Galois theory}

    \newdef{Field extension}{\index{field!extension}\label{algebra:field_extension}
        Let $k$ be a field. A field extension of $k$ is a field $K$ such that $k\subset K$ and such that the operations of $k$ are the restrictions of those in $K$.
    }
    \begin{notation}
        A field extension $K$ of $k$ is often denoted by $K/k$.
    \end{notation}

    \newdef{Degree}{\index{degree!of extension}
        If $K+k$ is a field extension, then $K$ can be given the structure of a $k$-vector space \ref{linalgebra:vector_space}. The dimension of this vector space is called the degree of the extension $K$. It is often denoted by $[K:k]$.
    }

    ?? COMPLETE ??

\section{Limits of algebraic structures}

    \newdef{Direct system}{\index{direct!system}
        Let $(I, \leq)$ be a directed set\footnote{See definition \ref{set:directed_set}.} and let $\{A_i\}_{i\in I}$ be a family of algebraic objects (groups, rings, ...). Consider a collection of morphisms $\{f_{ij}:A_i\rightarrow A_j\}_{i,j\in I}$ between these objects with the following properties:
        \begin{itemize}
            \item For every $i\in I$: $f_{ii} = \mathbbm{1}_{A_i}$.
            \item For every $i\leq j\leq k\in I$: $f_{ik} = f_{jk}\circ f_{ij}$.
        \end{itemize}
        The pair $(A_i, f_{ij})$ is called a direct system (over $I$).
    }

    \newdef{Direct limit\footnotemark}{\index{direct!limit}\index{inductive limit|see{direct limit}}\label{direct_limit}
        \footnotetext{Also called an \textbf{inductive limit}.}
        Consider a direct system $(A_i, f_{ij})$ over a directed set $I$. The direct limit $A$ of this direct system is defined as follows:
        \begin{gather}
            \varinjlim A_i := \left.\bigsqcup_{i\in I}A_i\right/\sim
        \end{gather}
        where the equivalence relation is given by $x\in A_i\sim y\in A_j\iff\exists k\in I: f_{ik}(x) = f_{jk}(y)$. Informally put: two elements are equivalent if they eventually become the same.

        The algebraic operations on $A$ are defined such that the inclusion maps $\phi_i:A_i\rightarrow A$ are morphisms.
    }

    \newdef{Inverse system}{\index{inverse!system}
        Let $(I, \leq)$ be a directed set\footnote{See definition \ref{set:directed_set}.} and let $\{A_i\}_{i\in I}$ be a family of algebraic objects (groups, rings, ...). Consider a collection of morphisms $\{f_{ij}:A_j\rightarrow A_i\}_{i,j\in I}$ between these objects with the following properties:
        \begin{itemize}
            \item For every $i\in I$: $f_{ii} = \mathbbm{1}_{A_i}$.
            \item For every $i\leq j\leq k\in I$: $f_{ik} = f_{ij}\circ f_{jk}$.
        \end{itemize}
        The pair $(A_i, f_{ij})$ is called an inverse system (over $I$).
    }

    \newdef{Inverse limit\footnotemark}{\index{inverse!limit}\index{projective!limit|see{inverse limit}}\label{inverse_limit}
        \footnotetext{Also called a \textbf{projective limit}.}
        Consider an inverse system $(A_i, f_{ij})$ over a directed set $I$. The inverse limit $A$ of this inverse system is defined as follows:
        \begin{gather}
            \varprojlim A_k := \left\{\vec{a}\in\prod_{i\in I}A_i:a_i=f_{ij}(a_j), \forall i\leq j\right\}.
        \end{gather}
        For all $i\in I$ there exists a natural projection $\pi_i:\varprojlim A_k\rightarrow A_i$.
    }

    \begin{remark}
        The direct and inverse limit are each other's (categorical) dual. The former is a colimit while the latter is a limit in category theory.
    \end{remark}

\section{Galois theory}

    \newdef{Field extension}{\index{field!extension}
        Let $k$ be a field. A field extension of $k$ is a field $K$ such that $k\subset K$ and such that the operations of $k$ coincide with the ones of $K$ restricted to $k$.
    }
    \begin{notation}
        A field extension $K$ of $k$ is often denoted by $K/k$.
    \end{notation}

    \newdef{Degree}{\index{degree}
        If $L/K$ is a field extension then $L$ can be given a $K$-vector space structure. The dimension of this vector space is called the degree of the extension $L$. It is often denoted by $[L:K]$.
    }
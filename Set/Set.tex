\chapter{Set Theory}

    For a formal introduction to the underlying axioms (and generalizations) of set theory, see Section \ref{set:section:axiomatization} at the end of this chapter.

\section{Axiomatization}\label{set:section:axiomatization}
\subsection{ZFC}\nomenclature[A_ZFC]{ZFC}{Zermelo-Frenkel set theory with the axiom of choice}

    The following set of axioms and axiom schemata gives a basis for axiomatic set theory that fixes a number of issues in naive set theory where one takes the notion of set for granted. This theory is called \textbf{Zermelo-Frenkel} set theory (ZF). When extended with the axiom of choice (see further) it is called ZFC, where the C stand for "choice".

    \begin{axiom}[Power set]\index{power!set}\label{set:power_set_axiom}
        \begin{gather}
            \forall x:\exists y:\forall z\big[z\in y\iff\forall w(w\in z\implies w\in x)\big]
        \end{gather}
        The set $y$ is called the power set $P(x)$ of $x$.
    \end{axiom}

    \begin{axiom}[Extensionality]\index{extensionality}
        \begin{gather}
            \forall x,y:\forall z\big[z\in x\iff z\in y\big]\implies x=y
        \end{gather}
        This axiom allows us to compare two sets based on their elements.
    \end{axiom}

    \begin{axiom}[Regularity\footnotemark]\index{regularity}
        \footnotetext{Also called the \textbf{axiom of foundation}.}
        \begin{gather}
            \forall x:(\exists z\in x)\implies(\exists a\in x)\land\neg(\exists b\in a:b\in x)
        \end{gather}
        This axiom says that for every non-empty set $x$ one can find an element $a\in x$ such that $x$ and $a$ are disjoint. Among other things this axiom implies that no set can contain itself.
    \end{axiom}

    The following axiom is technically not an axiom but an axiom schema, i.e. for every predicate $\varphi$ one obtains an axiom:
    \begin{axiom}[Specification]\index{specification}
        \begin{gather}
            \forall w_1,\ldots,w_n,A:\exists B:\forall x\big(x\in B\iff(x\in A\land\varphi(x,w_1,\ldots,w_n,A)\big)
        \end{gather}
        This axiom (schema) says that for every set $x$ one can build another set of elements in $x$ that satisfy a given predicate. By the axiom of extensionality this subset $B\subseteq A$ is unique.
    \end{axiom}

\subsection{Material set theory}

    ZF(C) is an instance of material set theory. Every element of a set is itself a set and, hence, has some kind of internal structure.

    \newdef{Pure set}{\index{pure!set}
        A set $U$ such that for every sequence $x_n\in x_{n-1}\in\cdots\in x_1\in U$ all the elements $x_i$ are also sets.
    }
    \newdef{Urelement\footnotemark}{\index{urelement}\index{atom}
        \footnotetext{Sometimes called an \textbf{atom}.}
        An object that is not a set.
    }

\subsection{Universes}\label{section:universes}

    ?? TODO ??

    To be able to talk about sets without running into problems such as \textit{Russel's paradox}, where one needs (or wants) to talk about the collection of all things satisfying a certain condition, one can introduce the concept of a universal set or universe (of discourse). This set takes the place of the ``collection of things'' and all operations performed on its elements, i.e. the sets that one wants to work with, act within this universe.

    \newdef{Grothendieck universe}{\index{Grothendieck!universe}
        A Grothendieck universe $U$ is a pure set satisfying the following axioms:
        \begin{enumerate}
            \item\textbf{Transitivity}: If $x\in U$ and $y\in x$, then $y\in U$;
            \item\textbf{Power set}: If $x\in U$, then $P(x)\in U$;
            \item\textbf{Pairing}: If $x,y\in U$, then $\{x,y\}\in U$; and
            \item\textbf{Unions}: If $I\in U$ and $\{x_i\}_{i\in I}\subset U$, then $\bigcup_{i\in I}x_i\in U$.
        \end{enumerate}
    }

\subsection{Structural set theory}

    In contrast to material set theory, the fundamental notions in this theory are sets and the relations between them. An element of a set does not have any internal structure and only becomes relevant if one specifies extra structure (or relations) on the sets. This implies that elements of sets are not sets themselves. In fact this would be a meaningless statement since, by default, they lack internal structure. Even stronger, it is meaningless to compare two elements if one does not provide relations or extra structure on the sets.

    ?? COMPLETE ??

\subsection{\texorpdfstring{ETCS $\clubsuit$}{ETCS}}\nomenclature[A_ETCS]{ETCS}{Elementary Theory of the Category of Sets}

    ?? COMPLETE ??

    \sremark{ETCS is the abbreviation of ``Elementary Theory of the Category of Sets''.}

    \begin{axiom}
        The category of sets is a well-pointed (elementary) topos.
    \end{axiom}

\subsection{Real numbers}\index{real numbers}

    \begin{axiom}[Ordering]
        The set of real numbers is an ordered field $(\mathbb{R},+,\cdot,<)$.
    \end{axiom}
    \begin{axiom}[Dedekind completeness]\index{Dedekind!completeness}
        Every non-empty subset of $\mathbb{R}$ that is bounded above has a supremum.
    \end{axiom}

    \begin{axiom}
        The rational numbers form a subset of the real numbers: $\mathbb{Q}\subset\mathbb{R}$.
    \end{axiom}
    \sremark{There is only one way to extend the field of rational numbers to the field of reals such that it satisfies the previous axioms. This implies that for every two possible constructions, there exists a bijection between the two.}

    \newdef{Extended real line}{
        \begin{gather}
            \label{calculus:extended_real_line}
            \overline{\mathbb{R}} := \mathbb{R}\cup\{-\infty,\infty\} \equiv [-\infty, \infty]
        \end{gather}
    }

\section{Functions}
\subsection{Domain}

    \newdef{Domain}{\index{domain}
        Let $f:X\rightarrow Y$ be a function. The set $X$ is called the domain of $f$.
    }
    \begin{notation}
        \nomenclature[S_dom]{$\mathrm{dom}(f)$}{domain of a function $f$}
        The domain of $f$ is denoted by $\mathrm{dom}(f)$.
    \end{notation}

    \newdef{Support}{\index{support}
        Let $f:X\rightarrow\mathbb{R}$ be a function with an arbitrary domain $X$. The support of $f$ is defined as the set of points where $f$ is nonzero.
    }
    \begin{notation}
        \nomenclature[S_sup]{$\mathrm{supp}(f)$}{support of a function $f$}
        The support of $f$ is denoted by $\mathrm{supp}(f)$.
    \end{notation}

    \begin{notation}\label{set:function_set}
        \nomenclature[S_YX]{$Y^X$}{set of functions from a set $X$ to a set $Y$}
        Let $X,Y$ be two sets. The set of functions $f:X\rightarrow Y$ is denoted by $Y^X$ or $\mathrm{Map}(X,Y)$. (See also Definition \ref{category:exponential_object} for a generalization.)
    \end{notation}

\subsection{Codomain}

    \newdef{Codomain}{\index{codomain}
        Let $f:X\rightarrow Y$ be a function. The set $Y$ is called the codomain of $f$.
    }
    \newdef{Image}{\index{image}
        Let $f:X\rightarrow Y$ be a function. The following subset of $Y$ is called the image of $f$:
        \begin{gather}
            \{y\in Y\mid\exists x\in X:f(x) = y\}.
        \end{gather}
    }
    \begin{notation}
        \nomenclature[S_im]{$\mathrm{im}(f)$}{image of a function $f$}
        The image of a function $f$ is denoted by $\mathrm{im}(f)$.
    \end{notation}
    \sremark{Some authors use these two notions interchangeably.}

    \newdef{Level set}{\index{level set}\label{set:level_set}
        Consider a function $f:X\rightarrow\mathbb{R}$. The following set is called the level set of $f$ at $c\in\mathbb{R}$:
        \begin{gather}
            L_c(f) := f^{-1}(c) \equiv \{x\in X\mid f(x) = c\}.
        \end{gather}
        For $X=\mathbb{R}^2$ the level sets are called \textbf{level curves} and for $X = \mathbb{R}^3$ they are called \textbf{level surfaces}.
    }

\subsection{Functions}

    \newdef{Injective}{\index{injective}\index{one-to-one|see{injective}}\label{set:injective}
        \nomenclature[O_inj]{$\hookrightarrow$}{injective function}
        A function $f:A\rightarrow B$ is said to be injective or \textbf{one-to-one} if the following condition is satisfied:
        \begin{gather}
            \forall a,a'\in A:f(a)=f(a')\implies a=a'.
        \end{gather}
    }
    \newnot{Injective map}{\[f:A\hookrightarrow B\]}

    \newdef{Surjective}{\index{surjective}\index{onto|see{surjective}}\label{set:surjective}
        \nomenclature[O_surj]{$\twoheadrightarrow$}{surjective function}
        A function $f:A\rightarrow B$ is said to be surjective or \textbf{onto} if the following condition is satisfied:
        \begin{gather}
            \forall b\in B,\exists a\in A:f(a) = b.
        \end{gather}
    }
    \newnot{Surjective map}{\[f:A\twoheadrightarrow B\]}

    \newdef{Bijection}{\index{bijection}
        A function that has an inverse. Equivalently, a function that gives a one-to-one correspondence between the elements of the domain and those of the codomain.
    }
    \newnot{Isomorphic}{
        \nomenclature[O_isom]{$\cong$}{is isomorphic to}
        If two sets $X,Y$ are isomorphic, this is denoted by \[X\cong Y.\]
    }

    \begin{theorem}[Cantor-Bernstein-Schr\"oder]\index{Cantor-Bernstein-Schr\"oder}
        Consider two sets $A,B$. If there exist injections $A\hookrightarrow B$ and $B\hookrightarrow A$, there exists a bijection $A\cong B$.
    \end{theorem}

    \newdef{Involution}{\index{involution}\label{set:involution}
        A function $f:A\rightarrow A$ such that $f^2=\mathrm{id}_A$, i.e. $f$ is its own inverse. Every involution is in particular a bijection.
    }

\section{Collections}

    \newdef{Power set}{\index{power!set}\label{set:power_set}
        \nomenclature[S_P]{$P(S),2^S$}{power set of $S$}
        Let $S$ be a set. The power set is defined as the set of all subsets of $S$ and is (often) denoted by $P(S)$ or $2^S$. The existence of this set is enforced by Axiom \ref{set:power_set_axiom}, the \textit{axiom of power set}.
    }
    \result{All sets are elements of their power set: $S\in P(S)$.}

    \newdef{Collection}{\index{collection}
        Let $A$ be a set. A collection of elements in A is a subset of $A$.
    }
    \newdef{Family}{\index{family}
        Let $A, I$ be two sets. A family of elements of $A$ with \textbf{index set} $I$ is a function $f:I\rightarrow A$. A family with index set $I$ is often denoted by $(x_i)_{i\in I}$. In contrast to collections, a family can ``contain'' multiple copies of the same element.
    }

    \newdef{Helly family}{\index{Helly family}\label{set:helly_family}
        A Helly family of order $k$ is a pair $(X,F)$ with $F\subset P(X)$ such that for every finite $G\subset F$:
        \begin{gather}
            \bigcap_{V\in G}V = \emptyset\implies \exists H\subseteq G: \left(\bigcap_{V\in H}V = \emptyset\right) \land \Big(|H|\leq k\Big).
        \end{gather}
        A Helly family of order 2 is sometimes said to have the \textbf{Helly property}.
    }

    \newdef{Diagonal}{\index{diagonal}
        The diagonal of a set $S$ is defined as follows:
        \begin{gather}
            \Delta_S := \big\{(a,a)\in S\times S\mid a\in S\big\}.
        \end{gather}
    }

    \newdef{Cover}{\index{cover}\label{set:cover}
        A cover of a set $S$ is a collection of sets $\mathcal{F}\subseteq P(S)$ such that
        \begin{gather}
            \bigcup_{V\in\mathcal{F}}V=S.
        \end{gather}
    }

    \newdef{Partition}{\index{partition}
        A  partition of $X$ is a family of disjoint subsets $(A_i)_{i\in I}\subset P(X)$ such that $\bigcup_{i\in I}A_i=X$.
    }
    \newdef{Refinement}{\index{refinement}
        Let $P$ be a partition of $X$. A refinement $P'$ of $P$ is a collection of subsets such that every $A\in P$ can be written as a disjoint union of elements in $P'$. It follows that every refinement is also a partition.
    }

    \newdef{Filter}{\index{filter}
        Let $X$ be a partially ordered set. A family $\mathcal{F}\subseteq P(X)$ is a filter on $X$ if it satisfies the following conditions:
        \begin{enumerate}
            \item\textbf{Empty set}: $\emptyset\not\in\mathcal{F}$;
            \item\textbf{Closed under intersections}: $\forall A,B \in\mathcal{F}:A\cap B\in\mathcal{F}$; and
            \item\textbf{Closed under inclusion}: if $A\in\mathcal{F}$ and $A\subseteq B$, then $B\in\mathcal{F}$.
        \end{enumerate}
    }

    \newdef{Filtration}{\index{filtration}\label{set:filtration}
        Consider a set $A$ together with a collection of subsets $F_iA$ indexed by a totally ordered set $I$. The collection is said to be a filtration of $A$ if
        \begin{gather}
            i\leq j\implies F_iA\subseteq F_jA.
        \end{gather}
        A filtration is said to be \textbf{exhaustive} if $\bigcup_iF_iA=A$ and \textbf{separated} if $\bigcap_iF_iA=\emptyset$.
    }
    \newdef{Associated grading}{
        In the case where one can define quotient objects every filtration $\{F_iA\}_{i\in\mathbb{N}}$ of $A$ defines an associated graded object $\{G_iA := F_iA/F_{i-1}A\}$.
    }

\section{Set operations}

    \newdef{Symmetric difference}{\index{symmetric difference}\label{set:symmetric_difference}
        \begin{gather}
            A\Delta B := (A\backslash B)\cup(B\backslash A)
        \end{gather}
    }

    \newdef{Complement}{\index{complement}\label{set:complement}
        Let $\Omega$ be the universe of discours (Section \ref{section:universes}) and let $E\subseteq\Omega$. The complement of $E$ is defined as follows:
        \begin{gather}
            E^c := \Omega\backslash E.
        \end{gather}
    }

    \newformula{de Morgan's laws}{\index{de Morgan's laws}
        \begin{gather}
            \label{set:de_morgan_union}
            \left(\bigcup_i A_i\right)^c = \bigcap_i A_i^c
        \end{gather}
        \begin{gather}
            \label{set:de_morgan_intersection}
            \left(\bigcap_i A_i\right)^c = \bigcup_i A_i^c
        \end{gather}
    }

    \newdef{Relation}{\index{relation}\label{set:relation}
        A relation between sets $X$ and $Y$ is a subset of the Cartesian product $X\times Y$. A relation on $X$ is then simply a subset of $X\times X$. This definition can easily be extended to $n$-ary relations by working with subsets of $n$-fold products.
    }

    \newdef{Converse relation}{\index{converse}\label{set:converse}
        Consider a relation $R\subset X\times Y$ between two sets $X,Y$. The converse relation $R^t$ is defined as follows:
        \begin{gather}
            R^t := \big\{(y,x)\in Y\times X\,\big\vert\,(x,y)\in R\big\}.
        \end{gather}
    }
    \newdef{Composition of relations}{\index{composition!of relations}\label{set:relational_composition}
        Consider two relations $R\subset X\times Y$ and $S\subset Y\times Z$ between three sets $X,Y$ and $Z$. The composition $S\circ R$ is defined as follows:
        \begin{gather}
            S\circ R := \big\{(x,z)\in X\times Z\,\big\vert\,\exists y\in Y:(x,y)\in R\land (y,z)\in S\big\}.
        \end{gather}
    }

\section{Algebra of sets}

    \newdef{Algebra of sets}{\index{algebra!of sets}\label{set:algebra_of_sets}
        A collection $\mathcal{F}\subset P(X)$ is a called an algebra over $X$ if it is closed under finite unions, finite intersections and complements. The pair $(X,\mathcal{F})$ is also called a \textbf{field of sets}.
    }

    \newdef{$\sigma$-algebra}{\index{$\sigma$!algebra}\label{set:sigma_algebra}
        A collection $\Sigma\subset P(X)$ is called a $\sigma$-algebra over a set $X$ if it satisfies the following axioms:
        \begin{enumerate}
            \item\textbf{Total space}: $X\in\Sigma$,
            \item\textbf{Closed under complements}: $\forall E\in\Sigma: E^c\in\Sigma$, and
            \item\textbf{Closed under countable unions}: $\forall\{E_i\}_{i=1}^n\subset\Sigma:\bigcup_{i=1}^nE_i\in\Sigma$.
        \end{enumerate}
    }
    \begin{remark}
        Axioms $(2)$ and $(3)$ together with de Morgan's laws \eqref{set:de_morgan_union} and \eqref{set:de_morgan_intersection} imply that a $\sigma$-algebra is also closed under countable intersections.
    \end{remark}

    \begin{result}[Algebra of sets]
        Every algebra of sets is a $\sigma$-algebra.
    \end{result}

    \begin{property}[Intersections]
        The intersection of a family of $\sigma$-algebras is again a $\sigma$-algebra.
    \end{property}

    \begin{definition}[Generated $\sigma$-algebras]
        A $\sigma$-algebra $\mathcal{G}$ is said to be generated by a collection of sets $\mathcal{A}$ if
        \begin{gather}
            \label{set:generated_sigma_algebra}
            \mathcal{G} = \bigcap\{\mathcal{F}\mid\mathcal{F}\text{ is a } \sigma\text{-algebra that contains }\mathcal{A}\}.
        \end{gather}
        Equivalently it is the smallest $\sigma$-algebra containing $\mathcal{A}$.
    \end{definition}
    \begin{notation}\label{set:notation:generated_sigma_algebra}
        The $\sigma$-algebra generated by a collection of sets $\mathcal{A}$ is often denoted by $\mathcal{F}_\mathcal{A}$ or $\sigma(\mathcal{A})$.
    \end{notation}

    \begin{construct}[Product $\sigma$-algebras]\label{set:product_of_sigma_algebras}
        The product $\sigma$-algebra $\mathcal{F}$ can be defined in the following equivalently ways:
        \begin{itemize}
            \item $\mathcal{F}$ is generated by the collection
                \[\mathcal{C} = \{A_1\times \Omega_2\mid A_1\in\mathcal{F}_1\}\cup\{\Omega_1\times A_2\mid A_2\in\mathcal{F}_2\}.\]
            \item $\mathcal{F}$ is the smallest $\sigma$-algebra such that the following projections are measurable (see \ref{lebesgue:measurable_function}):
                \[\text{Pr}_1:\Omega\rightarrow\Omega_1:(\omega_1,\omega_2)\mapsto\omega_1\]
                \[\text{Pr}_2:\Omega\rightarrow\Omega_2:(\omega_1,\omega_2)\mapsto\omega_2.\]
            \item $\mathcal{F}$ is the smallest $\sigma$-algebra containing the products $A_1\times A_2$ for all $A_1\in\mathcal{F}_1, A_2\in\mathcal{F}_2$.
        \end{itemize}
    \end{construct}

    \newdef{Monotone class}{\index{monotone!class}
        Let $\mathcal{A}$ be a collection of sets. $\mathcal{A}$ is called a monotone class if it has the following two properties:
        \begin{enumerate}
            \item For every increasing sequence $A_1\subset A_2\subset\cdots$:\ \[\bigcup_{i=1}^{+\infty}A_i\in\mathcal{A}.\]
            \item For every decreasing sequence $A_1\supset A_2\supset\cdots$:\ \[\bigcap_{i=1}^{+\infty}A_i\in\mathcal{A}.\]
        \end{enumerate}
    }

    \begin{theorem}[Monotone class theorem]\label{set:monotone_class}
        Let $\mathcal{A}$ be an algebra of sets \ref{set:algebra_of_sets}. If $\mathcal{G}_\mathcal{A}$ is the smallest monotone class containing $\mathcal{A}$ then it coincides with the $\sigma$-algebra generated by $\mathcal{A}$.
    \end{theorem}

\section{Ordered sets}\index{order}\label{section:ordered_sets}
\subsection{Posets}

    \newdef{Preordered set}{\index{preorder|see{order}}
        A set equipped with a reflexive and transitive binary relation.
    }
    \newdef{Partially ordered set}{\index{poset}\index{order!partial}\label{set:poset}
        A set $P$ equipped with a binary relation $\leq$ is called a partially ordered set (or \textbf{poset}) if the following 3 axioms are fulfilled for all elements $a,b,c\in P$:
        \begin{enumerate}
            \item\textbf{Reflexivity}: $a\leq a$,
            \item\textbf{Antisymmetry}: $a\leq b \land b\leq a\implies a = b$, and
            \item\textbf{Transitivity}: $a\leq b\land b\leq c\implies a\leq c$.
        \end{enumerate}
        Equivalently, it is a preordered set for which the binary relation is also antisymmetric.
    }
    \newdef{Totally ordered set}{\index{order!total}\index{totality}\label{set:total_order}
        A poset $P$ with the property that for all $a,b\in P: a\leq b$ or $b\leq a$ is called a (nonstrict) totally ordered set. This property is called \textbf{totality}.
    }
    \newdef{Strict total order}{
        A nonstrict order $\leq$ has an associated strict order $<$ that satisfies $a<b \iff a\leq b\land a\neq b$.
    }

    \newdef{Linear order}{\index{order!linear}
        A binary relation $<$ on a set $P$ satisfying the following conditions for all $x,y,z\in P$:
        \begin{enumerate}
            \item\textbf{Irreflexivity}: $x\not<x$,
            \item\textbf{Asymmetry}: $x<y\implies y\not<x$,
            \item\textbf{Transitivity}: $x<y\land y<z\implies x<z$,
            \item\textbf{Comparison}: $x<z\implies x<y\lor y<z$, and
            \item\textbf{Connectedness}: $x\not<y\land y\not<x\implies x=y$.
        \end{enumerate}
    }
    \remark{By negation one can freely pass between linear orders and total orders. However, without the law of the excluded middle, there exists no bijection between these two.}

    \newdef{Maximal element}{
        An element $m$ of a poset $P$ such that for every $p\in P:m\leq p\implies m=p$.
    }
    \newdef{Chain}{\index{chain}\label{set:chain}
        A totally ordered subset of a poset.
    }
    \begin{theorem}[Zorn's lemma\footnotemark]\index{Zorn's lemma}\label{set:zorns_lemma}
        \footnotetext{This theorem is equivalent to the \textit{axiom of choice}.}
        Let $(P,\leq)$ be a poset. If every chain in $P$ has an upper bound in $P$, then $P$ has a maximal element.
    \end{theorem}

    \newdef{Directed\footnotemark\ set}{\index{directed set}\index{filtered set|see{directed set}}\label{set:directed_set}
        \footnotetext{Sometimes called a \textbf{filtered} set or \textbf{upward} directed set. \textbf{Downward} directed sets are analogously defined with a lower bound for every two elements.}
        A set $X$ equipped with a preorder $\leq$ with the additional property that every 2-element subset has an upper bound, i.e. for every two elements $a,b\in X$, there exists an element $c\in X$ such that $a\leq c\land b\leq c$.
    }
    \newdef{Net}{\index{net}\label{set:net}
        A net on a set $X$ is a subset of $X$ indexed by a directed set $I$.
    }

\subsection{Bounds}

    \newdef{Supremum}{\index{supremum}\label{set:supremum}
        The supremum $\sup(X)$ of a poset $X$ is the smallest upper bound of $X$.
    }
    \newdef{Infimum}{\index{infimum}\label{set:infimum}
        The infimum $\inf(X)$ of a poset $X$ is the greatest lower bound of $X$.
    }

    \newdef{Maximum}{\index{maximum}\label{set:maximum}
        If $\sup(X)\in X$, the supremum is called the maximum of $X$. This is denoted by $\max(X)$.
    }
    \newdef{Minimum}{\index{minimum}\label{set:minimum}
        If $\inf(X)\in X$, the supremum is called the minimum of $X$. This is denoted by $\min(X)$.
    }

\subsection{\difficult{Ordinals and cardinals}}

    \newdef{Well-ordering}{\index{well-order}\index{well-founded}
        A \textbf{well-founded} linear order, i.e. a linear order $<$ such that every nonempty subset has a minimal element.
    }

    \newdef{Ordinal number}{\index{ordinal}\label{set:ordinal}
        Consider the class of all well-ordered sets. An ordinal (number) is an isomorphism class of well-ordered sets. The class of ordinals is itself well-ordered by inclusion of ``initial segments''.

        However, this definition gives problems within the ZF(C) framework of set theory since these equivalence classes are proper classes and not sets. To overcome this problem one can use a different approach. By using a well-defined construction one can for every class select a particular representative and call this representative the ordinal (rank) of all well-ordered sets isomorphic to it.
    }
    The most-used such construction is that by \textit{Von Neumann}. For every well-ordered set $W$ there exists an isomorphism $W\rightarrow P(W)$ that maps an element to the set of all subsets bounded from above by it. By analogy the Von Neumann ordinals are inductively defined as those well-ordered sets containing all smaller ordinals:
    \newdef{Von Neumann ordinal}{\index{von Neumann!ordinal}
        A set that is strictly well-ordered by membership and such that every element is also a subset.

        The first few finite von Neumann ordinals are given as an example:
        \begin{itemize}
            \item $0 := \emptyset$,
            \item $1 := \{0\} = \{\emptyset\}$,
            \item $2 := \{0, 1\} = \{\emptyset, \{\emptyset\}\}$, and
            \item ...
        \end{itemize}
    }

    \newdef{Successor}{\index{successor}\index{limit!ordinal|see{ordinal}}
        Every ordinal number $\alpha$ has a \textbf{successor} $\alpha^+$ (using the Von Neumann definition this is simply $\alpha^+ := \alpha\cup\{\alpha\}$). An ordinal that is not the successor of another ordinal number is called a \textbf{limit ordinal}.
    }

    \begin{remark}\index{Burali-Forti paradox}
        The \textit{Burali-Forti paradox} is the statement that the class of all ordinals (and by extension the class of all well-ordered sets) is not a set.
    \end{remark}

    There also exist numbers representing the sizes of sets. These are called \textbf{cardinal numbers}. These ``numbers'' should satisfy the following conditions:
    \begin{itemize}
        \item Every set has a well-defined cardinality.
        \item Every cardinal number is the cardinality of some set.
        \item Bijective sets have the same cardinality.
    \end{itemize}
    Guided by these conditions one could naively use the following definition:
    \newdef{Cardinal number}{\index{cardinal}
        An isomorphism class of sets (under bijections).
    }
    However, similar to the problem encountered for ordinals above, these classes are not sets. To solve this, one can also use a similar trick and select a specific representative. For cardinals the following choice is made:
    \newdef{Cardinality}{
        The cardinality of a set is the smallest ordinal rank of any well-order on it, i.e. any ordinal number bijective to it.\footnote{The well-ordering theorem (if assumed) assures that this definition coincides with the naive one above.} The cardinal numbers inherit a well-ordering from the ordinal numbers.
    }

    \begin{remark}[Ordering]
        The Cantor-Bernstein-Schr\"oder theorem induces a partial ordering on cardinal numbers. However, without the axiom of choice this can never be a total ordering. This problem is also apparent in the above definition since the ordinal rank of sets is used and the well-orderability of all sets, i.e. the well-ordering theorem, is equivalent to the axiom of choice.
    \end{remark}

    Similar to ordinal numbers one can also define successors of cardinal numbers:
    \newdef{Successor}{\index{successor}
        Given a cardinal $\kappa$, one defines its successor $\kappa^+$ as the smallest cardinal larger than $\kappa$.
    }
    \sremark{It should be noted that the successor of a cardinal number is not necessarilly the same as its successor as an ordinal number (in fact this is only the case for the finite cardinals).}

    \newdef{Regular cardinal}{\index{regular!cardinal}\label{set:regular_cardinal}
        An infinite cardinal $\kappa$ such that there exist no set of cardinality $\kappa$ that is the union of less than $\kappa$ subsets of cardinality less than $\kappa$.
    }

\subsection{Lattices}

    \newdef{Semilattice}{\index{join}\index{meet}
        A poset $(P,\leq)$ for which every 2-element subset has a supremum (also called a \textbf{join}) in $P$ is called a join-semillatice. Similarly, a poset $(P,\leq)$ for which every 2-element subset has an infimum (also called a \textbf{meet}) in $P$ is called a meet-semilattice.
    }
    \begin{notation}
        The join of $\{a,b\}$ is denoted by $a\land b$. The meet of $\{a,b\}$ is denoted by $a\lor b$.
    \end{notation}
    \newdef{Lattice}{\index{lattice}
        A poset $(P,\leq)$ is called a lattice if it is both a join- and a meet-semilattice.
    }
    The above definition also allows for a purely algebraic formulation (in this case some authors might speak about \textbf{lattice-ordered sets}):
    \newadef{Lattice}{
        A lattice is an algebraic structure that admits operations $\land, \lor$ and constants $\top, \bot$ that satisfy the following axioms:
        \begin{enumerate}
            \item Both $\land$ and $\lor$ are idempotent, commutative and associative.
            \item The \textbf{absorption laws}:
            \begin{gather}
                a\lor (a\land b) = a\qquad\qquad\qquad a\land (a\lor b) = a.
            \end{gather}
            \item $\top$ and $\bot$ are the respective identities of $\land$ and $\lor$.
        \end{enumerate}
        To go from this definition to the order-theoretic one, define the partial order \[a\leq b\iff a\land b=a.\] There exists an equivalent relation for the join.
    }

    \newdef{Bounded lattice}{
        A lattice $(P,\leq)$ that contains a greatest element (denoted by $\top$ or 1) and a smallest element (denoted by $\bot$ or 0) such that
        \begin{gather}
            \bot\leq x\leq\top
        \end{gather}
        for all $x\in P$. These elements are the identities for the join and meet operations:
        \begin{gather}
            x\land\top=x\qquad\qquad\qquad x\lor\bot=x.
        \end{gather}
    }

    \newdef{Frame}{\index{frame!set theory}\label{set:frame}
        A complete lattice\footnote{When working with categories this has to be restricted to ``all small joins/meets'' or, equivalently, the index category should be a set.} $(P,\leq)$ for which the \textbf{infinite distributivity law} is satisfied:
        \begin{gather}
            y\wedge\left(\bigvee_{i\in I}x_i\right) = \bigvee_{i\in I}\left(y\wedge x_i\right).
        \end{gather}
    }

    \newdef{Heyting algebra}{\index{Heyting!algebra}\index{complement}\label{set:heyting}
        A bounded lattice $H$ such that for every two elements $a,b\in H$ there exists a greatest element $x\in H$ for which
        \begin{gather}
            a\wedge x\leq b.
        \end{gather}
        This element is denoted by $a\rightarrow b$. The \textbf{pseudo-complement} $\neg a$ of an element $a\in H$ is then defined as $a\rightarrow\bot$.
    }
    \newdef{Boolean algebra}{\index{law of excluded middle}\index{Boolean!algebra}
        A Boolean algebra $X$ is a Heyting algebra in which the \textit{law of excluded middle} holds:
        \begin{gather}
            \forall x\in X:\neg\neg x=x.
        \end{gather}
        This can be equivalently stated as
        \begin{gather}
            \forall x\in X:x\lor\neg x=\top.
        \end{gather}
    }

\section{Partitions}
\subsection{Partition}

    \newdef{Composition}{\index{composition}
        Let $k,n\in\mathbb{N}$. A $k$-composition of $n$ is a $k$-tuple $(t_1,\ldots, t_k)$ such that $\sum_{i=1}^kt_k = n$.
    }
    \newdef{Partition}{\index{partition}
        Let $n\in\mathbb{N}$. A partition of $n$ is an ordered composition of $n$. Hence multiple different composition can determine the same partition.
    }

    \newdef{Young diagram\footnotemark}{\index{Young!diagram}\index{Ferrers diagram|see{Young diagram}}
        \footnotetext{Sometimes called a \textbf{Ferrers diagram}.}
        A Young diagram is a visual representation of the partition of an integer $n$. It is a left justified system of boxes, where every row corresponds to a part of the partition:
        \begin{figure}[!ht]
            \centering
            \ydiagram{5, 4, 4, 1}
            \caption{A Young diagram representing the partition $(5,4,4,1)$ of 14.}
            \label{fig:young_diagram}
        \end{figure}
    }
    \newdef{Conjugate partition}{
        Let $\lambda$ be a partition of $n$ with associated Young diagram $\mathcal{D}$. The conjugate partition $\lambda'$ is obtained by reflecting $\mathcal{D}$ across its main diagonal.
    }
    \begin{example}
        Conjugating Diagram \ref{fig:young_diagram} gives Diagram \ref{fig:young_diagram_conj} below. The associated partition is $(4,3,3,3,1)$.
        \begin{figure}[!ht]
            \centering
            \ydiagram{4, 3, 3, 3, 1}
            \caption{A Young diagram representing the partition $(4,3,3,3,1)$ of 14.}
            \label{fig:young_diagram_conj}
        \end{figure}
    \end{example}

    \newdef{Young tableau}{\index{Young!tableau}
        Consider a Young diagram of shape $\lambda$. A Young tableau of shape $\lambda$ is a filling of the corresponding Young diagram by the elements of a totally ordered set (with $n$ elements). This tableau is said to be \textbf{standard} if every row and every column is increasing.
    }

    \begin{formula}[Hook length formula]\index{hook length}
        The \textbf{hook} $H_{i,j}$ is defined as the part of a Young diagram given by the cell $(i,j)$ together with all cells below and to the right of $(i,j)$. Given a hook $H_{i,j}$, define the hook length $h_{i,j}$ as the sum of all elements in $H_{i,j}$.

        The number of all possible standard Young tableaux of shape $\lambda$ (where $\lambda$ defines a partition of $n$) is given by the following formula:
        \begin{gather}
            f^\lambda = \frac{n!}{\prod_{(i,j)\in\lambda}h_{i,j}}.
        \end{gather}
    \end{formula}

    \newdef{Young tabloid}{
        A Young tabloid of shape $\lambda$ is defined as the equivalence class of Young tableaux which are connected by permuting the elements within a row. These are often drawn as in Figure \ref{fig:young_tabloid}.
        \begin{figure}[!ht]
            \centering
            \ytableausetup{boxsize=normal,tabloids}\begin{ytableau}1&2&3&5&8\\ 4&6&9&10\\ 7&11&12&14\\ 15\end{ytableau}
            \caption{A Young tabloid associated to the Young diagram in Figure \ref{fig:young_diagram}.}
            \label{fig:young_tabloid}
        \end{figure}
    }

\subsection{Superpartition}

    For the physical background of the notions introduced in this section, see Chapter \ref{chapter:mathematical_formalism_qm}.

    \newdef{Superpartition}{\index{partition}\index{fermion}
        Let $m,n\in\mathbb{N}$. A superpartition in the $m$-\textit{fermion sector} is a sequence of integers of the following form:
        \begin{gather}
            \Lambda = (\Lambda_1,\ldots,\Lambda_m;\Lambda_{m+1},\ldots,\Lambda_n),
        \end{gather}
        where the first $m$ numbers are strictly ordered, i.e. $\Lambda_i>\Lambda_{i+1}$ for all $i<m$, and the last $n-m$ numbers form a normal partition.

        Both sequences, separated by a semicolon, form in fact distinct partitions themself. The first one represents the \textit{antisymmetric fermionic} sector (this explains the strict order) and the second one represents the \textit{symmetric bosonic} sector. This amounts to the following notation:\[\Lambda\equiv(\lambda^a;\lambda^s).\]
        The degree of the superpartition is given by $|\Lambda|=\sum_{i=1}^n\Lambda_i$.
    }
    \begin{notation}
        A superpartition of degree $n$ in the $m$-fermion sector is said to be a superpartition of $(n|m)$. To every superpartition $\Lambda$ one can also associate a unique partition $\Lambda^*$ by removing the semicolon and reordering the numbers such that they form a partition of $n$. The superpartition $\Lambda$ can then be represented by the Young diagram belonging to $\Lambda^*$ where the rows belonging to the fermionic sector are ended by a circle.
    \end{notation}
\chapter{Set theory}
\section{Collections}

	\newdef{Power set}{\index{power!set}\label{set:power_set}
	    	Let $S$ be a set. The power set is defined as the set of all subsets of $S$ and is (often) denoted by $P(S)$ or $2^S$. The existence of this set is stated by the \textit{axiom of power set}.
	}
	\result{$S\subset P(S)$}

	\newdef{Collection}{\index{collection}
		Let $A$ be a set. A collection of elements in A is a subset of $A$.
	}
	\newdef{Family}{\index{family}
		Let $A$ be a set and let $I$ be another set, called the \textbf{index set}. A family of elements of $A$ is a map $f:I\rightarrow A$. A family with index set $I$ is often denoted by $(x_i)_{i\in I}$. In contrast to collections a family can 'contain' multiple copies of a single element.
	}
	
	\newdef{Helly family}{\index{Helly family}\label{set:helly_family}
		A Helly family of order $k$ is a pair $(X, F)$ with $F\subset 2^X$ such that for every finite $G\subset F$:
		\begin{equation}
			\bigcap_{V\in G}V = \emptyset\implies \exists H\subseteq G: \left(\bigcap_{V\in H}V = \emptyset\right) \land \Big(|H| \leq k\Big)
		\end{equation}
		A Helly family of order 2 is sometimes said to have the \textbf{Helly property}.
	}
	
	\newdef{Diagonal}{\index{diagonal}
		Let $S$ be a set. The diagonal of $S$ is defined as follows:
		\begin{equation}
			\Delta_S = \{(a, a)\in S\times S: a\in S\}
		\end{equation}
	}

	\newdef{Partition}{\index{partition}
    		A  partition of $X$ is a family of disjoint subsets $(A_i)_{i\in I} \subset X$ such that $\bigcup_{i\in I}A_i = X$.
	}
	\newdef{Refinement}{\index{refinement}
    		Let $P$ be a partition of $X$. A refinement $P'$ of $P$ is a finite collection of subsets such that every $A\in P$ can be written as a disjoint union of elements in $P'$. Hence $P'$ is also a partition.
	}
    
	\newdef{Cover}{\index{cover}
		A cover of $S$ is a collection of sets $\mathcal{F}\subseteq2^S$ such that
		\begin{equation}
			\label{set:cover}
			\bigcup_{V\in\mathcal{F}}V = S
		\end{equation}
	}

\section{Set operations}

    	\newdef{Symmetric difference}{\index{symmetric difference}
        	\begin{equation}
			\label{set:symmetric_difference}
	                A\Delta B = (A\backslash B)\cup(B\backslash A)
		\end{equation}
        }

    	\newdef{Complement}{\index{complement}
            	Let $\Omega$ be the universal set . Let $E\subseteq\Omega$. The complement of $E$ is defined as:
                \begin{equation}
			\label{set:complement}
	                E^c = \Omega \backslash E
		\end{equation}
	}

        \newformula{de Morgan's laws}{\index{de Morgan's laws}
        	\begin{equation}
        	    	\label{set:de_morgan_union}
			\left(\bigcup_i A_i\right)^c = \bigcap_i A_i^c
		\end{equation}
        	\begin{equation}
            		\label{set:de_morgan_intersection}
			\left(\bigcap_i A_i\right)^c = \bigcup_i A_i^c
		\end{equation}
        }

\section{Ordered sets}
\subsection{Posets}
	
	\newdef{Preordered set}{\index{preorder}
		A preordered set is a set equipped with a reflexive and transitive binary relation.
	}
	\newdef{Partially ordered set}{\index{poset}\label{set:poset}
		A set $P$ equipped with a binary relation $\leq$ is called a partially ordered set (\textbf{poset}) if the following 3 axioms are fulfilled for all elements $a,b,c\in P$:
		\begin{enumerate}
			\item Reflexivity: $a\leq a$
			\item Antisymmetry: $a\leq b \land b\leq a\implies a = b$
			\item Transitivity: $a\leq b\land b\leq c\implies a\leq c$
		\end{enumerate}
		It is a preordered set for which the binary relation is also anti-symmetric.
	}
	\newdef{Totally ordered set}{\index{order}\index{totality}
		A poset $P$ with the property that for all $a,b\in P: a\leq b$ or $b\leq a$ is called a (non-strict) totally ordered set. This property is called \textbf{totality}.
	}
	\newdef{Strict total order}{
		A non-strict order $\leq$ has an associated strict order $<$ that satisfies $a<b \iff a\leq b\land a\neq b$.
	}
	
	\newdef{Maximal element}{
		An element $m$ of a poset $P$ is maximal if for every $p\in P$, $m\leq p$ implies that $m=p$.
	}
        
	\newdef{Chain}{\index{chain}
		A totally ordered subset of a poset is called a chain.
	}
	\begin{theorem}[Zorn's lemma\footnotemark]\index{Zorn's lemma}\label{set:zorns_lemma}
		\footnotetext{This theorem is equivalent to the \textit{axiom of choice}.}
		Let $(P, \leq)$ be a poset. If every chain in $P$ has an upper bound in $P$, then $P$ has a maximal element.
	\end{theorem}

\subsection{Lattices}

	\newdef{Semilattice}{\index{join}\index{meet}
		A poset $(P, \leq)$ for which every 2-element subset has a supremum (also called a \textbf{join}) in $P$ is called a join-semillatice. Similarly, a poset $(P, \leq)$ for which every 2-element subset has an infimum (also called a \textbf{meet}) in $P$ is called a meet-semilattice.
	}
	\begin{notation}
		The join of $\{a, b\}$ is denoted by $a\land b$. The meet of $\{a, b\}$ is denoted by $a\lor b$.
	\end{notation}
	\newdef{Lattice}{\index{lattice!set theory}
		A poset $(P, \leq)$ is called a lattice if it is both a join- and a meet-semilattice.
	}

	\newdef{Directed\footnotemark\ set}{\index{directed set}\index{filtered set|see{directed set}}\label{set:directed_set}
		\footnotetext{Sometimes called an \textit{upward} directed set. Downward directed sets are analogously defined with a lower bound for every two elements. Directed sets are also sometimes called \textbf{filtered sets}.}
		A directed set is a set $X$ equipped with a preorder $\leq$ and with the additional property that every 2-element subset has an upper bound, i.e. for every two elements $a, b\in X$ there exists an element $c\in X$ such that $a\leq c\land b\leq c$.
	}
	
	\newdef{Net}{\index{net}\label{set:net}
		A net on a topological space $X$ is a subset of $X$ indexed by a directed set $I$.
	}

	\subsection{Bounded sets}
    	\newdef{Supremum}{\index{supremum}
        	\label{set:supremum}
        	The supremum $\sup(X)$ of a set $X$ is the smallest upper bound of $X$.
        }
        \newdef{Infimum}{\index{infimum}
        	\label{set:infimum}
        	The infimum $\inf(X)$ of a set $X$ is the greatest lower bound of $X$.
        }
        
        \newdef{Maximum}{\index{maximum}
        	\label{set:maximum}
        	If $\sup(X)\in X$ the supremum is called the maximum of $X$. This is denoted by $\max(X)$.
        }
        \newdef{Minimum}{\index{minimum}
        	\label{set:minimum}
        	If $\inf(X)\in X$ the supremum is called the minimum of $X$. This is denoted by $\min(X)$.
        }

\subsection{Real numbers}\index{real numbers}

	\newprop{First axiom}{
		The set of real numbers is an ordered field $(\mathbb{R},+,\cdot,<)$
	}
	\newprop{Completeness axiom\footnotemark}{
		\footnotetext{This form of the completeness axiom is also called the supremum property or the Dedekind completeness.}\index{Dedekind!completeness}
		Every non-empty subset of $\mathbb{R}$ that is bounded above has a supremum.
	}
        
        \begin{property}
			$\mathbb{Q}\subset\mathbb{R}$
		\end{property}
        \sremark{There is only one way to extend the field of rational numbers to the field of reals such that it satisfies the two previous axioms. This means that for every possible construction, their exists a bijection (isomorphism) between the two.}
        
        \newdef{Extended real line}{
        	\begin{equation}
		\label{calculus:extended_real_line}
		\overline{\mathbb{R}} = \mathbb{R}\cup\{-\infty,\infty\} = [-\infty, \infty]
	\end{equation}
        }

\subsection{Filter}
	\newdef{Filter}{\index{filter}
		Let $X$ be a partially ordered set. A family $\mathcal{F}\subseteq2^X$ is a filter on $X$ if it satisfies following conditions:
		\begin{enumerate}
			\item $\emptyset\not\in\mathcal{F}$
			\item $\forall A, B \in\mathcal{F}:A\cap B\in\mathcal{F}$
			\item If $A\in\mathcal{F}$ and $A\subseteq B$ then $B\in\mathcal{F}$
		\end{enumerate}
	}

\section{Algebra of sets}
	
	\newdef{Algebra of sets}{\index{algebra!of sets}\label{set:algebra_of_sets}
	    	A collection $\mathcal{F}$ of subsets of $X$ is a called an algebra over $X$ if it is closed under finite unions, finite intersections and complements. The pair $(X,\mathcal{F})$ is also called a \textbf{field of sets}.
	}
	
\subsection{\texorpdfstring{$\sigma$}{sigma}-algebra}
    
	\newdef{$\sigma$-algebra}{\index{$\sigma$-algebra}\label{set:sigma_algebra}
	    	A collection of sets $\Sigma$ is a $\sigma$-algebra over a set $X$ if it satisfies the following 3 axioms:
        	\begin{enumerate}
			\item $X\in\Sigma$
        		\item Closed under complements: $\forall E\in\Sigma: E^c\in\Sigma$
		        \item Closed under countable unions: $\forall\{E_i\}_{i=1}^n\subset\Sigma:\bigcup_{i=1}^nE_i\in\Sigma$
		\end{enumerate}
	}
	\begin{remark}
    		Axioms $(2)$ and $(3)$ together with de Morgan's laws\footnotemark\ imply that a $\sigma$-algebra is also closed under countable intersections.
    		\footnotetext{See equations \ref{set:de_morgan_union} and \ref{set:de_morgan_intersection}.}
	\end{remark}
	
	\begin{result}
		Every algebra of sets is also a $\sigma$-algebra.
	\end{result}
    
	\begin{property}
		The intersection of a family of $\sigma$-algebras is again a $\sigma$-algebra.
	\end{property}
    
	\begin{definition}
		A $\sigma$-algebra $\mathcal{G}$ is said to be generated by a collection of sets $\mathcal{A}$ if
        	\begin{equation}
			\label{set:generated_sigma_algebra}
        		\mathcal{G} = \bigcap\{\mathcal{F}:\mathcal{F} \text{ is a } \sigma\text{-algebra that contains } \mathcal{A}\}
		\end{equation}
	        It is the smallest $\sigma$-algebra containing $\mathcal{A}$.
	\end{definition}
	\begin{notation}\label{set:notation:generated_sigma_algebra}
		The $\sigma$-algebra generated by a collection of sets $\mathcal{A}$ is often denoted by $\mathcal{F}_\mathcal{A}$ or $\sigma(\mathcal{A})$.
	\end{notation}
    
	\newdef{Borel set}{\index{Borel!set}
    		Let $\mathcal{B}$ be the $\sigma$-algebra generated by all open\footnotemark\ sets $O\subset X$. The elements $B\in\mathcal{B}$ are called Borel sets.
    		\footnotetext{For $X=\mathbb{R}$ we find that open, closed and half-open (both types) intervals generate the same $\sigma$-algebra.}
	}

	\newdef{Product $\sigma$-algebra}{\label{set:product_of_sigma_algebras}
		The smallest $\sigma$-algebra containing the products $A_1\times A_2$ for all $A_1\in\mathcal{F}_1, A_2\in\mathcal{F}_2$ is called the product $\sigma$-algebra of $\mathcal{F}_1$ and $\mathcal{F}_2$.
	}
	\begin{notation}
		The product $\sigma$-algebra of $\mathcal{F}_1$ and $\mathcal{F}_2$ is denoted by $\mathcal{F}_1\times\mathcal{F}_2$.
	\end{notation}
    
	\begin{adefinition}
		The product $\sigma$-algebra $\mathcal{F}$ can also be equivalently defined in the following two ways:
        	\begin{enumerate}
			\item $\mathcal{F}$ is generated by the collection
	            		\[\mathcal{C} = \{A_1\times \Omega_2:A_1\in\mathcal{F}_1\}\cup\{\Omega_1\times A_2:A_2\in\mathcal{F}_2\}\]
		        \item $\mathcal{F}$ is the smallest $\sigma$-algebra such that the following projections are measurable (see \ref{lebesgue:measurable_function}):
			        \[\text{Pr}_1:\Omega\rightarrow\Omega_1:(\omega_1,\omega_2)\mapsto\omega_1\]
        	    		\[\text{Pr}_2:\Omega\rightarrow\Omega_2:(\omega_1,\omega_2)\mapsto\omega_2\]
		\end{enumerate}
	\end{adefinition}
    	\sremark{Previous definitions can easily be generalized to higher dimensions.}
    
\subsection{Monotone class}
	
	\newdef{Monotone class}{\index{monotone!class}
    		Let $\mathcal{A}$ be a collection of sets. $\mathcal{A}$ is called a monotone class if it has the following two properties:
        	\begin{itemize}
			\item For every increasing sequence $A_1\subset A_2\subset ...$\ :\[\bigcup_{i=1}^{+\infty}A_i\in\mathcal{A}\]
		        \item For every decreasing sequence $A_1\supset A_2\supset ...$\ :\[\bigcap_{i=1}^{+\infty}A_i\in\mathcal{A}\]
		\end{itemize}
	}

	\begin{theorem}[Monotone class theorem]\label{set:theorem:monotone_class}
		Let $\mathcal{A}$ be an algebra of sets \ref{set:algebra_of_sets}. If $\mathcal{G}_\mathcal{A}$ is the smallest monotone class containing $\mathcal{A}$ then it coincides with the $\sigma$-algebra generated by $\mathcal{A}$.
	\end{theorem}

\section{Functions}
\subsection{Domain}
	
    \newdef{Domain}{\index{domain}
    	Let $f:X\rightarrow Y$ be a function. The set $X$, containing the arguments of $f$, is called the domain of $f$.
    }
    \begin{notation}
		The domain of $f$ is denoted by $\text{dom}(f)$.
	\end{notation}
    
    \newdef{Support}{\index{support}
    	Let $f:X\rightarrow\mathbb{R}$ be a function with an arbitrary domain $X$. The support of $f$ is defined as the set of points where $f$ is non-zero.
    }
    \begin{notation}
		The support of $f$ is denoted by $\text{supp}(f)$
	\end{notation}
    \sremark{The support of a function is a subset of its domain.}
    
    \begin{notation}
    	Let $X, Y$ be two sets. The set of functions $\{f:X\rightarrow Y\}$ is often denoted by $X^Y$.
    \end{notation}
    
\subsection{Codomain}

	\newdef{Codomain}{\index{codomain}
    	Let $f:X\rightarrow Y$ be a function. The set $Y$, containing (at least) all the output values of $f$, is called the codomain of $f$.
    }
    \newdef{Image}{\index{image}
    	Let $f:X\rightarrow Y$ be a function. The following subset of $Y$ is called the image of $f$:
        	\[
            	\{y\in Y\ |\ \exists x\in X:f(x) = y\}
            \]
		It is denoted by $\text{im}(f)$.
    }
    
    \newdef{Level set}{\index{level set}
    	Let $f:X\rightarrow\mathbb{R}$ be a real-valued function and let $c\in\mathbb{R}$. The following set is called the level set of $f$:
        \begin{equation}
        	\label{set:level_set}
			L_c(f) = \{x\in X:f(x) = c\}
		\end{equation}
        For $X=\mathbb{R}^2$ the level set is called a \textbf{level curve} and for $X = \mathbb{R}^3$ it is called the \textbf{level surface}.
    }

\chapter{Algebra}
\section{Groups}

    	\newdef{Semigroup}{\index{semigroup}
        	Let $G$ be a set equipped with a binary operation $\star$. $(G,\star)$ is a semigroup if it satisfies following axioms:
	\begin{enumerate}
				\item $G$ is closed under $\star$
                \item $\star$ is asssociative
	\end{enumerate}
        }
        
        \newdef{Monoid}{\index{monoid}
        	Let $M$ be a set equipped with a binary operation $\star$. $(M,\star)$ is a monoid if it satisfies following axioms:
            \begin{enumerate}
		\item $M$ is closed under $\star$
		\item $\star$ is associative
		\item $M$ contains an identity element with respect to $\star$
	\end{enumerate}
        }
        
        \newdef{Group}{\index{group}
        	Let $G$ be a set equipped with a binary operation $\star$. $(G,\star)$ is a group if it satisfies following axioms:
            \begin{enumerate}
		\item $G$ is closed under $\star$
		\item $\star$ is asssociative
		\item $G$ has an identity element with respect to $\star$
		\item Every element in $G$ has an inverse element with respect to $\star$
	\end{enumerate}
        }
        
        \newdef{Commutative group\footnotemark}{\index{commutativity}
        	Let $(G, \star)$ be a group. If $\star$ is commutative, then $G$ is called a commutative group.
        }
        \footnotetext{Also called an Abelian group.}\index{Abel!Abelian group}
        
        \newdef{Coset}{\index{coset}\index{normal}\label{group:coset}
        	Let $G$ be a group and $H$ a subgroup of $G$. The left coset of $H$ with respect to $g\in G$ is defined as the set
            \begin{equation}
            	gH = \{gh: h\in H\}
            \end{equation}
            The right coset is analogously defined as $Hg$. If for all $g\in G$ the left and right cosets coincide then the subgroup $H$ is said to be a \textbf{normal subgroup}. The sets of left and right cosets are denoted by $G/H$ and $G\backslash H$ respectively.
        }
        
        \newdef{Center}{\index{center}
        	The center of a group is defined as follows:
            \begin{equation}
            	\label{group:center}
                Z(G) = \{z\in G: \forall g\in G , zg = gz\}
            \end{equation}
            This set is a subgroup of $G$.
        }
        
        \newdef{Order of a group}{\index{order}
        	The number of elements in the group. It is denoted by $|G|$ or $\text{ord}(G)$.
        }
        \newdef{Order of an element}{
        	The order of an element $a\in G$ is the smallest integer $n$ such that
        	\begin{equation}
        		a^n = e
        	\end{equation}
        	where $e$ is the identity element of $G$.
        }
        \newdef{Torsion group}{\index{torsion}\label{group:torsion_group}
        	A torsion group is a group for which all element have finite order. The torsion $\text{Tor}(G)$ of a group $G$ is the set of all element $a\in G$ that have finite order. For Abelian groups, $\text{Tor}(G)$ is a subgroup.
        }
        
        \newdef{Quotient group}{\index{quotient!group}
        	\label{group:quotient_group}
        	Let $G$ be a group and $N$ a normal subgroup. The quotient group $G/N$ is defined as the set of left cosets of $N$ in $G$. This set can be turned into a group itself by equipping it with a product such that the product of $aN$ and $bN$ is $(aN)(bN)$. The fact that $N$ is a normal subgroup can be used to rewrite this as $(aN)(bN) = (ab)N$.
        }


\subsection{Symmetric and alternating groups}
		\newdef{Symmetric group}{
        	The symmetric group $S_n$ or $\text{Sym}_n$ of the set $V = \{1, 2, ..., n\}$ is defined as the set of all permutations of $V$.
        }
        \newdef{Alternating group}{
        	The alternating group $A_n$ is the subgroup of $S_n$ containing all even permutations.
        }
        
        \newdef{Cycle}{\index{cycle}
        	A $k$-cycle is a permutation of the form $(a_1, a_2, ..., a_k)$ sending $a_i$ to $a_{i+1}$ (and $a_k$ to $a_1$). A \textbf{cycle decomposition} of an arbitrary permutation is the decomposition into a product of disjoint cycles.
        }
        \begin{formula}
        	Let $\tau$ be a $k$-cycle. The following equality holds:
            \begin{equation}
            	\tau^k = \mathbbm{1}_G
            \end{equation}
        \end{formula}
        \begin{example}
        	Consider the set $\{1, 2, 3, 4, 5, 6\}$. The permutation $\sigma:x\mapsto x-2$ can be written as the cycle decomposition $\sigma = (1,3,5)(2,4,6)$.
        \end{example}
        
        \newdef{Transposition}{\index{transposition}
        	A permutation which exchanges two elements but lets the other ones unchanged.
        }

\subsection{Direct product}

	\newdef{Direct product}{\index{direct product! of groups}\label{group:direct_product}
		Let $G, H$ be two groups. The direct product $G\otimes H$ is defined as the set-theoretic Cartesian product $G\times H$ equipped with a binary operation $\cdot$ such that:
		\begin{equation}
			(g_1, h_1)\cdot(g_2, h_2) = (g_1g_2, h_1h_2)
		\end{equation}
		where the operations on the right hand side are the group operations in $G$ and $H$. The structure $G\otimes H = (G\times H, \cdot)$ is a group itself.
	}
	\begin{notation}
		When the groups are Abelian, the direct product is often denoted by $\oplus$.
	\end{notation}
	
	\newdef{Inner semidirect product}{\index{split}
		Let $G$ be a group, $H$ a subgroup of $G$ and $N$ a normal subgroup of $G$. $G$ is said to be the direct product of $H$ and $N$, denoted by $H\rtimes N$, if it satifies the following equivalent statements:
		\begin{itemize}
			\item $G = NH$ where $N\cap H = \{e\}$.
			\item For every $g\in G$ there exist unique $n\in N, h\in H$ such that $g=nh$.
			\item For every $g\in G$ there exist unique $h\in H, n\in N$ such that $g=hn$.
			\item There exists a group homomorphism $\rho:G\rightarrow H$ which satisfies $\rho|_H = e$ and $\ker(\rho)=N$.
			\item The composition of the natural embedding $i:H\rightarrow G$ and the projection $\pi:G\rightarrow G/N$ is a isomorphism between $H$ and $G/N$.
		\end{itemize}
		$G$ is also said to \textbf{split} over $N$.
	}
	\newdef{Outer semidirect product}{
		Let $G, H$ be two groups and let $\varphi:H\rightarrow\text{Aut}(G)$ be a group homomorphism. The outer semidirect product $G\rtimes_\varphi H$ is defined as the set-theoretic Cartesian product $G\times H$ equipped with a binary relation $\cdot$ such that:
		\begin{equation}
			(g_1, h_1)\cdot(g_2, h_2) = (g_1\varphi(h_1)(g_1), h_1h_2)
		\end{equation}
		The structure $(G\rtimes_\varphi H, \cdot)$ is a group itself.
		
		By noting that the set $N = \{(g, e_H)|g\in G\}$ is a normal subgroup isomorphic to $G$ and that the set $B = \{(e_G, h)|h\in H\}$ is a subgroup isomorphic to $H$ we can also construct the outer semidirect product $G\rtimes_\varphi H$ as the inner semidirect product $B\rtimes N$.
	}
	
	\begin{remark}
		The direct product of groups is a special case of the outer semidirect product where the group homomorphism is given by the trivial map $\varphi:h\mapsto e_G$.
	\end{remark}


\subsection{Free groups}

	\newdef{Free Abelian group}{\index{free!group}\index{basis}\index{rank}
		An abelian group $G$ with generators $\{g_i\}_{i\in I}$ is said to be freely generated if every element $g\in G$ can be uniquely written as a formal linear combination of the generators:
		\begin{equation}
			G = \left\{\left.\sum_ia_ig_i\right|a_i\in\mathbb{Z}\right\}
		\end{equation}
		The set of generators $\{g_i\}_{i\in I}$ is then called a \textbf{basis}\footnote{In analogy with the basis of a vector space.}\ of $G$. The number of elements in the basis is called the \textbf{rank} of $G$.
	}
	\begin{property}
		Consider a free group $G$. Let $H\subset G$ be a subgroup. Then $H$ is also free.
	\end{property}
	
	\begin{theorem}\label{group:theorem:free_group}
		Let $G$ be a finitely generated Abelian group of rank $n$, i.e. its basis has $n$ elements. This group can be constructed in two different ways:
		\begin{equation}
			G = F/H
		\end{equation}
		where both $F, H$ are freely and finitely generated Abelian groups. The second decomposition is:
		\begin{equation}
			G = A\oplus T\qquad\text{where}\qquad T = Z_{h_1}\oplus\cdots\oplus Z_{h_m}
		\end{equation}
		where $A$ is a freely and finitely generated group of rank $n-m$ and all $Z_{h_i}$ are cyclic groups of order $h_i$. The group $T$ is called the torsion subgroup\footnote{See also definition \ref{group:torsion_group}.}. The rank $n-m$ and the numbers $h_i$ are unique.
	\end{theorem}

\subsection{Group presentations}

	\newdef{Relations}{\index{relation}
		Let $G$ be a group. If the product of a number of elements $g\in G$ is equal to the identity $e$ then this product is called a relation on $G$.
	}
	\newdef{Complete set of relations}{
		Let $H$ be a group generated by a subgroup $G$. Let $R$ be a set of relations on $G$. If $H$ is uniquely (up to an isomorphism) determined by $G$ and $R$ then the set of relations is said to be complete.
	}

	\newdef{Presentation}{\index{presentation}
		Let $H$ be a group generated by a subgroup $G$ and a complete set of relations $R$ on $G$. The pair $(G, R)$ is called a presentation of $H$.
		
		It is clear that every group can have many different presentations and that it is (very) difficult to tell if two groups are isomorphic by just looking at their presentations.
	}

\subsection{Group actions}
        \newdef{Group homomorphism}{\index{homomorphism!of groups}
        	A group homomorphism $\Phi:G\rightarrow H$ is a map satisfying $\forall g, h \in G$
            \begin{equation}
            	\Phi(gh) = \Phi(g)\Phi(h)
            \end{equation}
        }
        
        \newdef{Kernel}{\index{kernel}
        	The kernel of a group homomorphism $\Phi:G\rightarrow H$ is defined as the set
            \begin{equation}
            	K = \{g\in G: \Phi(g) = \mathbbm{1}_H\}
            \end{equation}
        }
        \begin{theorem}[First isomorphism theorem]\index{isomorphism!theorem}\label{group:theorem:first_isomorphism_theorem}
        	Let $G, H$ be a groups and let $\varphi:G\rightarrow H$ be a group homomorphism. If $\varphi$ is surjective than $G/\ker\varphi\cong H$.
        \end{theorem}
        
        \newdef{Group action}{\index{group!action}\label{group:group_action}
            Let $G$ be a group. Let $V$ be a set. A map $\rho: G\times V \rightarrow V$ is called an action of $G$ on $V$ if it satisfies the following conditions:
            \begin{itemize}
                \item Identity: $\rho(\mathbbm{1}_G, v) = v$
                \item Compatibility: $\rho(gh, v) = \rho(g, \rho(h, v))$
            \end{itemize}
            For all $g, h \in G$ and $v\in V$. The set V is called a (left) \textbf{G-space}.
        }
        \remark{\label{group:permutation_remark}
        	A group action can alternatively be defined as a group homomorphism from $G$ to $\text{Sym}(V)$. It assigns a permutation of $V$ to every element $g\in G$.
	}
        \begin{notation}
        	The action $\rho(g, v)$ is often denoted by $g\cdot v$ or even $gv$.
        \end{notation}
        
	\newdef{Orbit}{\index{orbit}
		The orbit of an element $x\in X$ with respect to a group $G$ is defined as the set:
		\begin{equation}
			\label{group:orbit}
			x\cdot G = \{x\cdot g|g\in G\}
		\end{equation}
	}
	\newdef{Stabilizer}{\index{stabilizer}\index{isotropy group}
		The stabilizer group or \textbf{isotropy group} of an element $x\in X$ with respect to a group $G$ is defined as the set:
		\begin{equation}
			G_x = \{g\in G|g \cdot x = x\}
		\end{equation}
		This is a subgroup of $G$.
	}
	
	\newdef{Free action}{\index{free}\label{group:free_action}
		A group action is free if $g\cdot x = x$ implies $g = e$ for every $x\in X$.
	}
	\newdef{Faithful action\footnotemark}{\index{faithful!action}\label{group:faithful_action}
		\footnotetext{A faithful action is also called an \textbf{effective} action.}
		A group action is faithful if the homomorphism $G\rightarrow\text{Sym}(X)$ is injective. Alternatively, a group action is faithful if for every two group elements $g, h\in G$ there exists an element $x\in X$ such that $g\cdot x\neq h\cdot x$.
	}
	
	\newdef{Transitive action}{\index{transitive!action}\label{group:transitive}
		A group action is transitive if for every two elements $x, y\in X$ there exists a group element $g\in G$ such that $g\cdot x = y$. Equivalently we can say that there is only one orbit.
	}
	\newdef{Homogeneous space}{\index{homogeneous!space}
		If the group action of a group $G$ on a $G$-space $X$ is transitive, then $X$ is said to be a homogeneous space.
	}
	\begin{property}[$\dag$]\label{group:transitive_action_property}
		Let $X$ be a set and let $G$ be a group such that the action of $G$ on $X$ is transitive. Then their exists a bijection $X\cong G/G_x$ where $G_x$ is the stabilizer of any element $x\in X$.
	\end{property}
        
        \newdef{G-module}{\index{module}
        	Let $G$ be a group. Let $M$ be a commutative group. $M$ equipped with a left group action $\varphi:G\times M\rightarrow M$ is a (left) G-module if $\varphi$ satisfies the following equation (distributivity):
            \begin{equation}
            	\label{group:g_module}
                g\cdot(a+b) = g\cdot a + g\cdot b
            \end{equation}
            where $a, b\in M$ and $g\in G$.
        }
        \newdef{G-module homomorphism}{\index{homomorphism!of G-modules}\index{equivariant}\label{group:equivariant}
        	A G-module homomorphism is a map $f:V\rightarrow W$ satisfying
            \begin{equation}
            	g\cdot f(v) = f(g\cdot v)
            \end{equation}
            where the $\cdot$ symbol represents the group action in $W$ and $V$ respectively. It is sometimes called a \textbf{G-map}, a \textbf{G-equivariant map} or an \textbf{intertwining map}.
        }
        
\section{Rings}
	
	\newdef{Ring}{\index{ring}
		Let $R$ be a set equipped with two binary operations $+,\cdot$ (called addition and multiplication). $(R,+,\cdot)$ is a ring if it satisfies the following axioms:
    		\begin{enumerate}
			\item $(R,+)$ is a commutative group.
			\item $(R,\cdot)$ is a monoid.
			\item Multiplication is distributive with respect to addition.
		\end{enumerate}
	}
	
	\newdef{Unit}{\index{unit}
		An invertible element of ring $(R, +, \cdot)$. The set of units forms a group under multiplication.
	}

\subsection{Ideals}\index{ideal}

    	\newdef{Ideal}{\label{linalgebra:ideal}
    		Let $(R,+,\cdot)$ be a ring with $(R,+)$ its additive group. A subset $I\subseteq R$ is called an ideal\footnotemark\ of $R$ if it satisfies the following conditions:
        	\begin{enumerate}
			\item $(I,+)$ is a subgroup of $(R,+)$
                	\item $\forall n\in I, \forall r\in R:(n\cdot r), (r\cdot n)\in I$
		\end{enumerate}
		\footnotetext{More generally: two-sided ideal}
        }
        
        \newdef{Unit ideal}{Let $(R,+,\cdot)$ be a ring. $R$ itself is called the unit ideal.}
        \newdef{Proper ideal}{Let $(R,+,\cdot)$ be a ring. A subset $I\subset R$ is said to be a proper ideal if it is an ideal of $R$ and if it is not equal to $R$.}
        \newdef{Prime ideal}{Let $(R,+,\cdot)$ be a ring. A proper ideal $I$ is a prime ideal if for any $a,b\in R$ the following relation holds:
        	\[ab\in I\implies \text a\in I \vee b\in I\]
        }
        \newdef{Maximal ideal}{Let $(R,+,\cdot)$ be a ring. A proper ideal $I$ is said to be maximal if there exists no other proper ideal $T$ in R such that $I\subset T$.}
        \newdef{Minimal ideal}{A proper ideal is said to be minimal if it contains no other nonzero ideal.}
        
	\newdef{Generating set}{\index{generating set! of an ideal}\label{group:generating_set_ideal}
		Let $R$ be a ring and let $X$ be a subset of $R$. The two-sided ideal generated by $X$ is defined as the intersection of all two-sided ideals containing $X$. An explicit construction is given by:
		\begin{equation}
			I = \left\{\left.\sum_{i=1}^n l_ix_ir_i\ \right\vert\ \forall i\leq n: l_i, r_i\in R\text{ and } x_i\in X\right\}
		\end{equation}
		Left and right ideals are generated in a similar fashion.
	}
        
\subsection{Graded rings}
	
	\newdef{Graded ring}{\index{graded}\label{group:graded_ring}
		Let $R$ be a ring that can be written as the direct sum of Abelian groups $A_k$:
		\begin{equation}
			R = \bigoplus_{k\in\mathbb{N}}A_k
		\end{equation}
		If $R$ has the property that for every $i, j\in\mathbb{N}: A_i\star A_j\subseteq A_{i+j}$, where $\star$ is the ring multiplication, then $R$ is said to be a graded ring. The elements of the space $A_k$ are said to be \textbf{homogeneous of degree $k$}.
	}
	
	\newformula{Graded commutativity}{\index{commutativity!graded}
		Let $m = \deg v$ and let $n = \deg w$. If
		\begin{equation}
			\label{group:graded_commutativity}
			vw = (-1)^{mn}wv
		\end{equation}
		for all elements $v, w$ of the graded ring then it is said to be a graded-commutative ring.
	}
        


\section{Other algebraic structures}

	\newdef{Chain complex}{\index{chain!complex}\index{boundary}\index{differential}\index{cycle}\label{group:chain_complex}
		Let $(A_k)_{k\in\mathbb{N}}$ be a sequence of algebraic structures together with a sequence $\{\partial_k:A_k\rightarrow A_{k-1}\}_{k\in\mathbb{N}}$ of morphisms, called the \textbf{boundary operators} or \textbf{differentials}, such that for all $k$:
		\begin{equation}
			\partial_k\circ\partial_{k+1} = 0
		\end{equation}
		This structure is called a chain complex\footnotemark. Elements in $\text{im}(\partial_k)$ are called \textbf{boundaries} and elements in $\text{ker}(\partial_k)$ are called \textbf{cycles}.
		\footnotetext{A \textbf{cochain complex} is constructed similarly. For this structure we consider an ascending order, i.e.: $\partial_k:A_k\rightarrow A_{k+1}$.}
	}

\subsection{Direct systems}

	\newdef{Direct system}{\index{direct!system}
		Let $(I, \leq)$ be a directed set\footnote{See definition \ref{set:directed_set}.}. Let $\{A_i\}_{i\in I}$ be a family of algebraic objects (groups, rings, ...) and let $\{f_{ij}:A_i\rightarrow A_j\}_{i,j\in I}$ be a set of homomorphisms with the following properties:
		\begin{itemize}
			\item For every $i\in I$: $f_{ii} = e_i$, where $e_i$ is the identity in $A_i$.
			\item For every $i\leq j\leq k\in I$: $f_{ik} = f_{jk}\circ f_{ij}$.
		\end{itemize}
		The pair $(A_i, f_{ij})$ is called a direct system over I.
	}

	\newdef{Direct limit}{\index{direct!limit}
		Consider a direct system $(A_i, f_{ij})$ over a (directed) set $I$. The direct limit $A$ of these direct systems is defined as follows:
		\begin{equation}
			\label{direct_limit}
			\varinjlim A_i = \left.\bigsqcup_{i\in I}A_i\right/\sim
		\end{equation}
		where the equivalence relation is given by $x\in A_i\sim y\in A_j\iff\exists k\in I: f_{ik}(x) = f_{jk}(y)$. Informally put: two elements are equivalent if they eventually become the same.
		
		The algebraic operations on $A$ are defined such that the inclusion maps $\phi_i:A_i\rightarrow A$ are morphisms.
	}

\subsection{Exact sequences}

	\newdef{Exact sequence}{\index{exact!sequence}
		Consider a sequence (finite or infinite) of algebraic structures and their corresponding homomorphisms:
		\begin{equation}
			A_0\xrightarrow{\Phi_1}A_1\xrightarrow{\Phi_2}\cdots\xrightarrow{\Phi_n}A_n
		\end{equation}
		The sequence is exact if for every $k\in\mathbb{N}: \text{im}(\Phi_k) = \text{ker}(\Phi_{k+1})$. This implies that $\Phi_{k+1}\circ\Phi_k = 0$ for all $h\in\mathbb{N}$.
	}
	
	\newdef{Short exact sequence}{
		A short exact sequence is an exact sequence of the form:
		\begin{equation}
			\label{short_exact_sequence}
			0\rightarrow A_0\xrightarrow{\Phi_1}A_1\xrightarrow{\Phi_2} A_3\rightarrow0
		\end{equation}
		A long exact sequence is an infinite exact sequence.
	}
	
	\begin{property}\index{epimorphism}\index{monomorphism}\index{bimorphism}
		Looking at some small examples we can derive some important constraints for certain exact sequences and especially for short exact sequences. Consider the sequence
		\[
			0\rightarrow A\xrightarrow{\Phi} B
		\]
		This sequence can only be exact if $\Phi$ is an injective homomorphism (\textbf{monomorphism}). This follows from the fact that the only element in the image of the map $0\rightarrow A$ is 0 because the map is a homomorphism. The kernel of $\Phi$ is thus trivial which implies that $\Phi$ is injective.
		
		Analogously, the sequence
		\[
			A\xrightarrow{\Psi}B\rightarrow0
		\]
		is exact if $\Psi$ is a surjective homomorphism (\textbf{epimorphism}). This follows from the fact that the kernel of the map $B\rightarrow0$ and thus the image of $\Psi$ is all of $B$ which implies that $\Psi$ is surjective.
		
		It follows that the sequence
		\[
			0\rightarrow A\xrightarrow{\Sigma}B\rightarrow0
		\]
		is exact if $\Sigma$ is a \textbf{bimorphism} (which is often an isomorphism).
	\end{property}

\section{Integers}
\subsection{Partition}
	\newdef{Composition}{\index{composition}
		Let $n\in\mathbb{N}$. A $k$-composition of $n$ is a $k$-tuple $(t_1, ..., t_k)$ such that $\sum_{i=1}^kt_k = n$.
	}
	\newdef{Partition}{\index{partition}
		Let $n\in\mathbb{N}$. A partition of $n$ is an ordered composition of $n$.
	}
	
	\newdef{Young diagram\footnotemark}{\index{Young!diagram}\index{Ferrers diagram}
		\footnotetext{Sometimes called a \textit{Ferrers} diagram.}
		A Young diagram is a visual representation of the partition of an integer $n$. It is a left justified system of boxes, where every row corresponds to a part of the partition.
		\begin{figure}[!ht]
			\centering
			\ydiagram{5, 4, 4, 1}
			\caption{A Young diagram representing the partition $(5, 4, 4, 1)$ of 14.}
			\label{fig:young_diagram}
		\end{figure}
	}
	\newdef{Conjugate partition}{
		Let $\lambda$ be a partition of $n$ with Young diagram $\mathcal{D}$. The conjugate partition $\lambda'$ is obtained by reflecting $\mathcal{D}$ across its main diagonal.
	}
	\begin{example}
		Using the diagram \ref{fig:young_diagram} we obtain the conjugate partition $(4, 3, 3, 3, 1)$ represented by
		\begin{figure}[!ht]
			\centering
			\ydiagram{4, 3, 3, 3, 1}
			\caption{A Young diagram representing the partition $(4, 3, 3, 3, 1)$ of 14.}
			\label{fig:young_diagram_conj}
		\end{figure}
	\end{example}
	
\subsection{Superpartition}

	\newdef{Superpartition}{
		Let $n\in\mathbb{N}$. A superpartition in the $m$-fermion sector is a sequence of integers of the following form:
		\begin{equation}
			\Lambda = (\Lambda_1, ..., \Lambda_m;\Lambda_{m+1}, ..., \Lambda_N)
		\end{equation}
		where the first $m$ numbers are strictly ordered, i.e. $\Lambda_i>\Lambda_{i+1}$ for all $i< m$, and the last $N-m$ numbers form a normal partition.
		
		Both sequences, separated by a semicolon, form in fact distinct partitions themself. The first one represents the antisymmetric fermionic sector (this explains the strict order) and the second one represents the symmetric bosonic sector. This amounts to the following notation:\[\Lambda = (\lambda^a;\lambda^s)\]
		The degree of the superpartition is given by $n\equiv|\Lambda|=\sum_{i=1}^N$.
	}
	\begin{notation}
		A superpartition of degree $n$ in the $m$-fermion sector is said to be a superpartition of $(n|m)$. To every superpartition $\Lambda$ we can also associate a unique partition $\Lambda^*$ by removing the semicolon and reordering the numbers such that they form a partition of $n$. The superpartition $\Lambda$ can then be represented by the Young diagram belonging to $\Lambda^*$ where the rows belonging to the fermionic sector are ended by a circle.
	\end{notation}

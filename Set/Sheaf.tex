\chapter{Sheaf Theory}\label{chapter:sheaf}

A reference for this chapter is \cite{brylinski}. For a background on topological spaces: see chapters \ref{chapter:topology} and onwards. For the concept of sheaf in category theory: see section \ref{section:grothendieck_topos}.

\section{Presheafs}

	\newdef{Presheaf}{\index{presheaf}\index{section}
		Let $(X, \tau)$ be a topological space. A presheaf over $X$ consists of an algebraic structure $\mathcal{F}(U)$ for every open set $U\in\tau$ and a morphism $\Phi^U_V:\mathcal{F}(U)\rightarrow \mathcal{F}(V)$ for every two open sets $U, V\in\tau$ with $V\subseteq U$ such that the following conditions are satisfied:
		\begin{enumerate}
			\item $\Phi^U_U = \text{Id}$
			\item If $W\subseteq V\subseteq U$ then $\Phi^U_W = \Phi^V_W\circ\Phi^U_V$.
		\end{enumerate}
		The set $\mathcal{F}(U)$ is called the set of \textbf{sections} over $U$ and the morphisms $\Phi^U_V$ are called the \textbf{restriction maps}.
	}
	
	\newdef{Morphism of presheaves}{\index{morphism!of presheaves}\index{germ}
		Let $\mathcal{F}, \mathcal{F}'$ be two presheaves over a space $X$. A morphism $\mathcal{F}\rightarrow \mathcal{F}'$ is a set of morphisms $\Psi_U:\mathcal{F}(U)\rightarrow \mathcal{F}'(U)$ that commute with the restriction maps $\Phi^U_V$.
	}
	
	\begin{adefinition}[Category theory]
		Using the language of category theory one can more easily introduce presheaves: Let $\mathbf{C}$ be a category and let $X$ be a topological space. A $\mathbf{C}$-valued presheaf on $X$ is a contravariant functor $\mathcal{F}:\text{Open}(X)\rightarrow\mathbf{C}$. A morphism of presheafs is accordingly a natural transformation between such functors. As such the category of presheaves on a topological space $X$ is in fact $\textbf{Set}^{\textbf{Open}(X)^{op}}$, the presheaf topos on $\textbf{Open}(X)$.
	\end{adefinition}
	
	\begin{example}[Constant presheaf]\label{sheaf:constant_presheaf}
		Let $S$ be any set. The constant presheaf over $X$ with target $S$ is defined by:\[\mathcal{F}(U) = S\] for every open set $U\subseteq X$.
	\end{example}

\section{Sheafs}
	
	\newdef{Sheaf}{\index{sheaf}\label{sheaf:def}
		Let $(X, \tau)$ be a topological space. A sheaf $\mathcal{O}_X$ over $X$ is a presheaf $\mathcal{F}$ satisfying the following additional conditions:
		\begin{enumerate}
			\item Locality\footnote{This is in fact a corollary of the second axiom.}: Let $\{U_i\in\tau\}$ be an open cover of $U\subseteq X$ and consider sections $s, t\in\mathcal{F}(U)$. If $\forall i: s|_{U_i}=t|_{U_i}$ then $s=t$.
			\item Gluing: Let $\{U_i\in\tau\}$ be an open cover of $U\subseteq X$ and let $\{s_i\in\mathcal{F}(U_i)\}$ be a collection of sections. If $\forall i, j: s_i|_{U_i\cap U_j} = s_j|_{U_i\cap U_j}$ then there exists a section $s\in\mathcal{F}(U)$ such that $\forall i: s|_{U_i} = s_i$.
		\end{enumerate}
	}
	\begin{property}
		Let $X$ be a topological space and let $\mathcal{F}$ be a presheaf over $X$. $\mathcal{F}$ is a sheaf over $X$ if for any open $U\subseteq X$ and every open cover $\{U_i\}_{i\in I}$ of $U$ the following diagram is an equalizer diagram:
		\begin{equation}
			\mathcal{F}(U)\rightarrow\prod_{i\in I}\mathcal{F}(U_i)\rightrightarrows\prod_{i, j\in I}\mathcal{F}(U_i\cap U_j)
		\end{equation}
		The two morphisms on the right are obtained by combining the projection morphisms of the direct product and the restriction morphisms $\Phi^{U_i}_{U_i\cap U_j}$ and $\Phi^{U_j}_{U_i\cap U_j}$.
	\end{property}
	
	\newdef{Stalk}{\index{stalk}\index{germ}
		Let $x\in X$ and consider the set of all neighbourhoods of $x$. This set can be turned into a directed set\footnote{See definition \ref{set:directed_set}.} by equipping it with the (partial) order relation \[U\subseteq V\implies U\geq V\] This turns the sheaf $\mathcal{F}$ over $X$ into a directed system. The stalk over $x$ is then defined as the following direct limit\footnote{See definition \ref{direct_limit}.}:
		\begin{equation}
			\mathcal{F}_x = \varinjlim_{U\ni x} \mathcal{F}(U)
		\end{equation}
		The equivalence class of a section $s\in\mathcal{F}(U)$ in $\mathcal{F}_x$ is called the \textbf{germ} of $s$ at $x$. Two sections belong to the same germ at $x$ if there exists a neighbourhood of $x$ on which they coincide.
	}
	\begin{notation}
		Similar to the notation of the restriction morphisms we denote the morphism that maps every section to its germ at $x$ by $\Phi^U_x$.
	\end{notation}
	
	\begin{property}
		Two subsheaves of a sheaf $\mathcal{G}$ over $X$ are equal if and only if their stalks are equal as subsets of $\mathcal{G}_x$ for all points $x\in X$.
	\end{property}
	
	\begin{construct}[Associated sheaf]
		Consider a presheaf $\mathcal{F}$ over a topological space $X$. From this presheaf one can construct a sheaf $\overline{\mathcal{F}}$, called the \textbf{sheafification} or associated sheaf of $\mathcal{F}$, in the following way:
		
		First we define a presheaf $\mathcal{G}$ such that:\footnote{Sections in this sheaf are said to be \textit{continuous}. This statement can be made formal using the concept of an \'etal\'e space. (See construction \ref{sheaf:etale_construction}.)}
		\begin{equation}
			\mathcal{G}(U) = \left\{\left.(s_x)_{x\in U}\in\prod_{x\in U}\mathcal{F}_x\right\vert\forall x\in U: \exists\text{ open }V\ni x, t\in\mathcal{F}(V): \forall v\in V: s_v = \Phi^V_v(t)\right\}
		\end{equation}
		The restriction maps $\rho^U_V$ are defined by:
		\begin{equation}
			\rho^U_V:(s_x)_{x\in U}\mapsto(s_x)_{x\in V}
		\end{equation}
		It is easily proven that this presheaf is in fact a sheaf and hence we obtain the sheafification of $\mathcal{F}$ by setting $\overline{\mathcal{F}} = \mathcal{G}$. This construction also gives a canonical morphism $\varphi:\mathcal{F}\rightarrow\overline{\mathcal{F}}$ defined by:
		\begin{equation}
			\varphi(s):U\rightarrow\prod_{x\in U}\mathcal{F}_x:x \mapsto s_x := \Phi^U_x(s)
		\end{equation}
		where $s\in\mathcal{F}(U)$ and $x\in U$.
	\end{construct}
	\sremark{We see that the sections in $\overline{\mathcal{F}}$ arise from (locally) resticting sections in $\mathcal{F}$ to the stalks $\mathcal{F}_x$.}
	
	\begin{uproperty}
		Let $\mathcal{F}$ be a presheaf over $X$ with associated sheaf $\overline{\mathcal{F}}$. For any sheaf $\mathcal{G}$ over $X$ and any morphism $\rho:\mathcal{F}\rightarrow\mathcal{G}$ there exists a unique morphism $\sigma:\overline{\mathcal{F}}\rightarrow\mathcal{G}$ such that $\rho = \sigma\circ\varphi$ where $\varphi$ is the canonical morphism $\mathcal{F}\rightarrow\overline{\mathcal{F}}$.
	\end{uproperty}
	
	\begin{property}
		Let $\mathcal{F}$ be a presheaf over $X$ with associated sheaf $\overline{\mathcal{F}}$. The morphism $\varphi:\mathcal{F}\rightarrow\overline{\mathcal{F}}$ induces an isomorphism $\varphi_x:\mathcal{F}_x\rightarrow\overline{\mathcal{F}}_x$ for all $x\in X$.
	\end{property}
	\begin{property}
		Let $\mathcal{F}$ be a sheaf over $X$ with associated sheaf $\overline{\mathcal{F}}$. The morphism $\varphi:\mathcal{F}\rightarrow\overline{\mathcal{F}}$ is an isomorphism.
	\end{property}
	
	There exists another, more topological, construction of the associated sheaf:
	\begin{construct}[\'Etal\'e spaces]\label{sheaf:etale_construction}
		Let $\mathcal{F}$ be a presheaf over $X$. Consider the disjoint union
		\begin{equation}
			\overline{\mathcal{F}} = \bigsqcup_{x\in X}\mathcal{F}_x
		\end{equation}
		Define for every local section $s\in\mathcal{F}(U)$ a function $\overline{s}:U\rightarrow\overline{\mathcal{F}}:x\mapsto s_x\in\mathcal{F}_x$ The union $\overline{\mathcal{F}}$ can be turned into an \'etal\'e space\footnote{See definition \ref{topology:etale_space}.} over $X$ by equipping it with the topology with basis
		\begin{equation}
			\{\overline{s}(U)\ |\ U\text{ open in }X, s\in\mathcal{F}(U)\}
		\end{equation}
		The projection map $\pi$ is given by $\pi:s_x\in\mathcal{F}_x\mapsto x$. The sheafification $\mathcal{F}^\ast$ is then given by the sheaf of sections of $\overline{\mathcal{F}}$.
	\end{construct}
	
	\begin{example}[Constant sheaf]
		Consider the constant presheaf over $X$ with target $S$ (see example \ref{sheaf:constant_presheaf}). The constant sheaf, denoted by $S_X$, is defined as the associated sheaf of this presheaf. The stalks over every point $x\in X$ can be identified with $S$. The continuous sections $S_X(U)$ are the locally constant functions $f:U\rightarrow S$.
	\end{example}
		
	\begin{notation}[Category of sheaves]
		\nomenclature[S_Sheaf]{$\mathbf{Sh}(X)$}{Category of sheafs over a topological space $X$.}
		Similar to the case of presheaves one can define a morphism of sheaves as a morphism commuting with the restriction maps.\footnote{When treating a (pre)sheaf as a functor, the (pre)sheaf morphism is a natural transformation.} The sheafs and sheaf morphisms over a space $X$ form a full subcategory of the category of presheafs, denoted by Sh$(X)$.
	\end{notation}
	
	\begin{property}
		The category of sheafs Sh$(X)$ is in fact an elementary topos, called the sheaf topos over $X$. (See property \ref{topoi:sheaf_topos}.)
	\end{property}
	
	\newdef{Global sections functor}{\index{section}
		Let $X$ be a topological space. The global sections functor $\Gamma(X, -)$ is defined as the functor $\Gamma(X, -):\text{Sh}(X)\rightarrow\text{Set}:\mathcal{F}\rightarrow\mathcal{F}(X)$.
	}
	\begin{property}
		The global sections functor is only left exact.
	\end{property}
	
\section{Resolutions and cohomology}

	In this section we will only use Abelian sheafs, i.e. sheafs with values in Ab, unless stated otherwise.

	\begin{property}
		Every Abelian sheaf admits an injective resolution.
	\end{property}
	\newdef{Sheaf cohomology group}{\index{cohomology!sheaf}
		Let $\mathcal{F}$ be a sheaf over $X$. Given an injective resolution\footnote{In fact this construction is independent of the chosen injective resolution.} $I$ of $\mathcal{F}$ one defines the sheaf cohomology groups of $\mathcal{F}$ over $X$ as the cohomology groups of the complex
		\begin{equation}
			\cdots\longrightarrow\Gamma(X, I^i)\longrightarrow\Gamma(X, I^{i+1})\longrightarrow\cdots
		\end{equation}
		The cohomology group $H^0(X, \mathcal{F})$ is equal to $\Gamma(X, \mathcal{F})$.
	}
	\newdef{Acyclic sheaf}{\index{sheaf!acyclic}
		A sheaf is called acyclic if its higher cohomology groups vanish.
	}
	\begin{theorem}[de Rham \& Weil]
		Let $\mathcal{F}$ be a sheaf. There exists an isomorphism between the sheaf cohomology groups defined above and the ones obtained by using an acyclic resolution of $\mathcal{F}$.
	\end{theorem}
	
	\newdef{Image and kernel}{
		Given a morphism of sheafs $\phi:\mathcal{F}\rightarrow\mathcal{G}$ over a space $X$ one can define the kernel/image presheafs which assign to every open subset $U\subseteq X$ the image/kernel of $\phi_U$. The kernel presheaf is already a sheaf and will be denoted by $\ker(\phi)$. The sheafification of the image presheaf will be denoted by $\im(\phi)$.
	}

	\newdef{Cohomology sheafs}{
		Let $\mathcal{F}^\bullet$ be a complex of sheafs over $X$. The cohomology sheafs $\underline{H}^i(X, \mathcal{F}^\bullet)$ assign to every open subset $U\subseteq X$ the quotient group $\ker(d^i_U)/\im(d^{i-1}_U)$.
	}
	
\section{Ringed spaces}

	\newdef{Ringed space}{\index{ringed space}
		A ringed space is a topological space $X$ equipped with a sheaf of rings $\mathcal{O}_X$.
	}
	\newdef{Locally ringed space}{
		A ringed space $(X, \mathcal{O}_X)$ is said to be locally ringed if the stalk over every point $x\in X$ is a local ring\footnote{See definition \ref{algebra:local_ring}.}.
	}

\section{Simplicial sets}

	\newdef{Simplex category}{\index{simplex!category}\index{face!map}\index{degeneracy!map}
		\nomenclature[S_simpcat]{$\Delta$}{The simplex category.}
		The simplex category $\Delta$ has as objects the posets of the form $[n] = \{0, \ldots, n\}$ and as morphisms the order-preserving maps.
	}
	\newdef{Simplicial set}{\index{simplicial set}
		The category $\mathbf{sSet}$ of simplicial sets is given by the presheaf category $\mathbf{Psh}(\Delta)$. The set $X_n:=X([n])$ is called the set of \textbf{$n$-simplices} in $X$.
	}
	\newdef{Simplicial object}{
		By internalizing the notion of a simplicial set in any category one obtains the definition of a simplicial object, i.e. a simplicial object in a category $\mathbf{C}$ is a $\mathbf{C}$-valued presheaf on $\Delta$. This way a simplicial set is just a simplicial object in \textbf{Set}.
	}
	\begin{property}\index{face!map}\index{degeneracy!map}
		All morphisms in the simplex category $\Delta$ are generated by two types:
		\begin{itemize}
			\item For every $n$ the unique map $\delta_{n, i}:[n-1]\rightarrow[n]$ which misses the $i^{th}$ element.
			\item For every $n$ the unique map $\sigma_{n, i}:[n+1]\rightarrow[n]$ which duplicates the $i^{th}$ element.
		\end{itemize}
		Under the action of a presheaf this gives the \textbf{face} and \textbf{degeneracy} maps $d_{n, i}$ and $s_{n, i}$. (If the index $n$ is clear, then it is often ommitted in the notation.)
		
		Their fundamental relations are called the \textbf{simplicial identities}:
		\begin{itemize}
			\item $d_i\circ d_j = d_{j-1}\circ d_i$ for $i<j$
			\item $d_i\circ s_j = s_{j-1}\circ d_i$ for $i<j$
			\item $d_i\circ s_j = \text{id}$ for $i=j$ or $i=j+1$
			\item $d_i\circ s_j = s_j\circ d_{i-1}$ for $i>j+1$
			\item $s_i\circ s_j = s_{j+1}\circ s_i$ for $i\leq j$
		\end{itemize}
	\end{property}
	
	\newdef{Standard simplex}{\index{simplex}
		For every $n$ we define the standard (simplicial) $n$-simplex $\Delta_n$ as the simplicial set given by $\Delta(-, [n])$.
	}
	\begin{property}
		By the Yoneda lemma there exists a natural bijection between the set of $n$-simplices of a simplicial set $X$ and the set of maps $\Delta_n\rightarrow X$.
	\end{property}
	
	\begin{construct}[Nerve]\index{nerve}
		\nomenclature[S_Nerve]{$N\mathbf{C}$}{The simplicial nerve of a small category $\mathbf{C}$.}
		To every small category \textbf{C} one can associate a simplicial set $N\mathbf{C}$ in the following way. The set $N\mathbf{C}_0$ is given by the set of objects in $\mathbf{C}$. The set $N\mathbf{C}_1$ is given by the set of morphisms in $\mathbf{C}$. Now for every two composable morphisms $f, g$ one obtains a canonical commuting triangle. Let $N\mathbf{C}_2$ be the set of all these triangles. The higher simplices are defined analogously.
		
		Equivalently one can define the (simplicial) nerve functor in the following way. Every poset $[n]$ as defined above admits a canonical category structure for which the order preserving maps give rise to the associated functors. This inclusion $\Delta\hookrightarrow\mathbf{Cat}$ then induces the following functor:
		\begin{gather}
			N:\textbf{Cat}\rightarrow[\Delta^{op}, \textbf{Set}]:\mathbf{C}\mapsto\hom_{\mathbf{Cat}}(-, \mathbf{C})
		\end{gather}
		Using this nerve functor we obtain that $N\mathbf{C}_k=\hom_{\mathbf{Cat}}([k], \mathbf{C})$, which is by definition equivalent to all strings of $k$ composable morphisms in $\mathbf{C}$.
	\end{construct}

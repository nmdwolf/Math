\chapter{Sheaf Theory}

\section{Presheafs}

	\newdef{Presheaf}{\index{presheaf}\index{section}
		Let $(X, \tau)$ be a topological space. A presheaf over $X$ consists of an algebraic structure $\mathcal{F}(U)$ for every open set $U\in\tau$ and a morphism $\Phi^U_V:\mathcal{F}(U)\rightarrow \mathcal{F}(V)$ for every two open sets $U, V\in\tau$ with $V\subseteq U$ such that the following conditions are satisfied:
		\begin{itemize}
			\item $\Phi^U_U = \text{Id}$
			\item If $W\subseteq V\subseteq U$ then $\Phi^U_W = \Phi^V_W\circ\Phi^U_V$.
		\end{itemize}
		The set $\mathcal{F}(U)$ is called the set of \textbf{sections} over $U$ and the morphisms $\Phi^U_V$ are called the \textbf{restriction maps}.
	}
	
	\newdef{Morphism of presheaves}{\index{morphism!of presheaves}\index{germ}
		Let $\mathcal{F}, \mathcal{F}'$ be two presheaves over a space $X$. A morphism $\mathcal{F}\rightarrow \mathcal{F}'$ is a set of morphisms $\Psi_U:\mathcal{F}(U)\rightarrow \mathcal{F}'(U)$ that commute with the restriction maps $\Phi^U_V$.
	}
	
	\begin{adefinition}[Category theory]
		Using the language of category theory one can more easily introduce presheaves: Let $\mathbf{C}$ be a category and let $X$ be a topological space. A $\mathbf{C}$-valued presheaf on $X$ is a contravariant functor $\mathcal{F}:\text{Open}(X)\rightarrow\mathbf{C}$.
	\end{adefinition}

\section{Sheafs}
	
	\newdef{Sheaf}{\index{sheaf}
		Let $(X, \tau)$ be a topological space. A sheaf $\mathcal{O}_X$ over $X$ is a presheaf $\mathcal{F}$ satisfying the following additional conditions:
		\begin{itemize}
			\item Locality: Let $\{U_i\in\tau\}$ be an open cover of $U\subseteq X$ and consider a section $s\in\mathcal{F}(U)$. If $\forall i, s|_{U_i}=0$ then $s=0$.
			\item Gluing: Let $\{U_i\in\tau\}$ be an open cover of $U\subseteq X$ and let $\{s_i\in\mathcal{F}(U_i)\}$ be a collection of sections. If $\forall i, j: s_i|_{U_i\cap U_j} = s_j|_{U_i\cap U_j}$ then there exists a section $s\in\mathcal{F}(U)$ such that $\forall i, s|_{U_i} = s_i$.
		\end{itemize}
	}
	\newdef{Stalk}{\index{stalk}\index{germ}
		Let $x\in X$ and consider the set of all neighbourhoods of $x$. This set can be turned into a directed set\footnote{See definition \ref{set:directed_set}.} by equipping it with the order relation $U\subset V\implies U\geq V$. This turns the sheaf $\mathcal{F}$ over $X$ into a directed system. The stalk over $x$ is then defined as the following direct limit\footnote{See definition \ref{direct_limit}.}:
		\begin{equation}
			\mathcal{F}_x = \varinjlim_{U\ni x} \mathcal{F}(U)
		\end{equation}
		The equivalence class of a section $s\in\mathcal{F}(U)$ in $\mathcal{F}_x$ is called the \textbf{germ} of $s$ at $x$.
	}
	
\section{Ringed spaces}

	\newdef{Ringed space}{\index{ringed space}
		A ringed space is a topological space $X$ equipped with a sheaf of rings $\mathcal{O}_X$.
	}
	\newdef{Locally ringed space}{
		A ringed space $(X, \mathcal{O}_X)$ is said to be locally ringed if the stalk over every point $x\in X$ is a local ring\footnote{See definition \ref{algebra:local_ring}.}.
	}

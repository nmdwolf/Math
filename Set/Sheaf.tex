\chapter{Sheaf Theory}\label{chapter:sheaf}

    A reference for this chapter is \cite{brylinski}. For a background on topological spaces: see chapters \ref{chapter:topology} and onwards. For the concept of sheaf in category theory: see section \ref{section:grothendieck_topos}.

\section{Presheaves}

    \newdef{Presheaf}{\index{presheaf}\index{section}
        Let $(X, \tau)$ be a topological space. A presheaf over $X$ consists of an algebraic structure $\mathcal{F}(U)$ for every open set $U\in\tau$ and a morphism $\Phi^U_V:\mathcal{F}(U)\rightarrow \mathcal{F}(V)$ for every two open sets $U, V\in\tau$ with $V\subseteq U$ such that the following conditions are satisfied:
        \begin{enumerate}
            \item $\Phi^U_U = \text{Id}$
            \item If $W\subseteq V\subseteq U$ then $\Phi^U_W = \Phi^V_W\circ\Phi^U_V$.
        \end{enumerate}
        The set $\mathcal{F}(U)$ is called the set of \textbf{sections} over $U$ and the morphisms $\Phi^U_V$ are called the \textbf{restriction maps}.
    }

    \newdef{Morphism of presheaves}{\index{morphism!of presheaves}\index{germ}
        Let $\mathcal{F}, \mathcal{F}'$ be two presheaves over a space $X$. A morphism $\mathcal{F}\rightarrow \mathcal{F}'$ is a set of morphisms $\Psi_U:\mathcal{F}(U)\rightarrow \mathcal{F}'(U)$ that commute with the restriction maps $\Phi^U_V$.
    }

    \begin{adefinition}[Category theory]
        Using the language of category theory one can more easily introduce presheaves: Let $\mathbf{C}$ be a category and let $X$ be a topological space. A $\mathbf{C}$-valued presheaf on $X$ is a contravariant functor $\mathcal{F}:\text{Open}(X)\rightarrow\mathbf{C}$. A morphism of presheaves is accordingly a natural transformation between such functors. As such the category of presheaves on a topological space $X$ is in fact $\textbf{Set}^{\textbf{Open}(X)^{op}}$, the presheaf topos on $\textbf{Open}(X)$.
    \end{adefinition}

    \begin{example}[Constant presheaf]\label{sheaf:constant_presheaf}
        Let $S$ be any set. The constant presheaf over $X$ with target $S$ is defined by
        \begin{gather}
            \mathcal{F}(U) := S
        \end{gather}
        for every open set $U\subseteq X$.
    \end{example}

\section{sheaves}

    \newdef{Sheaf}{\index{sheaf}\label{sheaf:def}
        Let $(X, \tau)$ be a topological space. A sheaf $\mathcal{O}_X$ over $X$ is a presheaf $\mathcal{F}$ satisfying the following additional conditions:
        \begin{enumerate}
            \item \textbf{Locality}\footnote{This is in fact a corollary of the second axiom.}: Let $\{U_i\}_{i\in I}\subset\tau$ be an open cover of $U\subseteq X$ and consider sections $s, t\in\mathcal{F}(U)$. If $\forall i\in I: s|_{U_i}=t|_{U_i}$ then $s=t$.
            \item \textbf{Gluing}: Let $\{U_i\}_{i\in I}\subset\tau$ be an open cover of $U\subseteq X$ and let $\{s_i\in\mathcal{F}(U_i)\}$ be a collection of sections. If $\forall i, j\in I: s_i|_{U_i\cap U_j} = s_j|_{U_i\cap U_j}$ then there exists a section $s\in\mathcal{F}(U)$ such that $\forall i\in I: s|_{U_i} = s_i$.
        \end{enumerate}
    }
    \begin{property}
        Let $X$ be a topological space and let $\mathcal{F}$ be a presheaf over $X$. $\mathcal{F}$ is a sheaf over $X$ if for any open $U\subseteq X$ and every open cover $\{U_i\}_{i\in I}$ of $U$ the following diagram is an equalizer diagram:
        \begin{equation}
            \mathcal{F}(U)\rightarrow\prod_{i\in I}\mathcal{F}(U_i)\rightrightarrows\prod_{i, j\in I}\mathcal{F}(U_i\cap U_j).
        \end{equation}
        The two morphisms on the right are obtained by combining the projection morphisms of the direct product and the restriction morphisms $\Phi^{U_i}_{U_i\cap U_j}$ and $\Phi^{U_j}_{U_i\cap U_j}$.
    \end{property}

    \newdef{Stalk}{\index{stalk}\index{germ}
        Let $x\in X$ and consider the set of all neighbourhoods of $x$. This set can be turned into a directed set\footnote{See definition \ref{set:directed_set}.} by equipping it with the (partial) order relation \[U\subseteq V\implies U\geq V.\] This turns the sheaf $\mathcal{F}$ over $X$ into a directed system. The stalk over $x$ is then defined as the following direct limit\footnote{See definition \ref{direct_limit}.}:
        \begin{equation}
            \mathcal{F}_x := \varinjlim_{U\ni x} \mathcal{F}(U).
        \end{equation}
        The equivalence class of a section $s\in\mathcal{F}(U)$ in $\mathcal{F}_x$ is called the \textbf{germ} of $s$ at $x$. Two sections belong to the same germ at $x$ if there exists a neighbourhood of $x$ on which they coincide.
    }
    \begin{notation}
        Similar to the notation of the restriction morphisms we denote the morphism that maps every section to its germ at $x$ by $\Phi^U_x$.
    \end{notation}

    \begin{property}
        Two subsheaves of a sheaf $\mathcal{G}$ over $X$ are equal if and only if their stalks are equal as subsets of $\mathcal{G}_x$ for all points $x\in X$.
    \end{property}

    \begin{construct}[Associated sheaf]
        Consider a presheaf $\mathcal{F}$ over a topological space $X$. From this presheaf one can construct a sheaf $\overline{\mathcal{F}}$, called the \textbf{sheafification} or associated sheaf of $\mathcal{F}$, in the following way:

        First we define a presheaf $\mathcal{G}$ such that\footnote{Sections in this sheaf are said to be \textit{continuous}. This statement can be made formal using the concept of an \'etal\'e space. (See construction \ref{sheaf:etale_construction}.)}
        \begin{equation}
            \mathcal{G}(U) := \left\{(s_x)_{x\in U}\in\prod_{x\in U}\mathcal{F}_x:\left(\forall x\in U: \exists\text{ open }V\ni x, t\in\mathcal{F}(V): \forall v\in V: s_v = \Phi^V_v(t)\right)\right\}.
        \end{equation}
        The restriction maps $\rho^U_V$ are defined by
        \begin{equation}
            \rho^U_V:(s_x)_{x\in U}\mapsto(s_x)_{x\in V}.
        \end{equation}
        It is easily proven that this presheaf is in fact a sheaf and hence we obtain the sheafification of $\mathcal{F}$ by setting $\overline{\mathcal{F}} = \mathcal{G}$. This construction also gives a canonical morphism $\varphi:\mathcal{F}\rightarrow\overline{\mathcal{F}}$ defined by
        \begin{equation}
            \varphi(s):U\rightarrow\prod_{x\in U}\mathcal{F}_x:x \mapsto s_x = \Phi^U_x(s)
        \end{equation}
        where $s\in\mathcal{F}(U)$ and $x\in U$.
    \end{construct}
    \sremark{We see that the sections of $\overline{\mathcal{F}}$ arise from (locally) resticting sections of $\mathcal{F}$ to the stalks $\mathcal{F}_x$.}

    \begin{uproperty}
        Let $\mathcal{F}$ be a presheaf over $X$ with associated sheaf $\overline{\mathcal{F}}$. For any sheaf $\mathcal{G}$ over $X$ and any morphism $\rho:\mathcal{F}\rightarrow\mathcal{G}$ there exists a unique morphism $\sigma:\overline{\mathcal{F}}\rightarrow\mathcal{G}$ such that $\rho = \sigma\circ\varphi$ where $\varphi$ is the canonical morphism $\mathcal{F}\rightarrow\overline{\mathcal{F}}$.
    \end{uproperty}

    \begin{property}
        Let $\mathcal{F}$ be a presheaf over $X$ with associated sheaf $\overline{\mathcal{F}}$. The morphism $\varphi:\mathcal{F}\rightarrow\overline{\mathcal{F}}$ induces an isomorphism $\varphi_x:\mathcal{F}_x\rightarrow\overline{\mathcal{F}}_x$ for all $x\in X$.
    \end{property}
    \begin{property}
        Let $\mathcal{F}$ be a sheaf over $X$ with associated sheaf $\overline{\mathcal{F}}$. The morphism $\varphi:\mathcal{F}\rightarrow\overline{\mathcal{F}}$ is an isomorphism.
    \end{property}

    There exists another, more topological, construction of the associated sheaf:
    \begin{construct}[\'Etal\'e spaces]\label{sheaf:etale_construction}
        Let $\mathcal{F}$ be a presheaf over $X$. Consider the disjoint union
        \begin{equation}
            \overline{\mathcal{F}} := \bigsqcup_{x\in X}\mathcal{F}_x.
        \end{equation}
        Define for every local section $s\in\mathcal{F}(U)$ a function $\overline{s}:U\rightarrow\overline{\mathcal{F}}:x\mapsto s_x\in\mathcal{F}_x$ The union $\overline{\mathcal{F}}$ can be turned into an \'etal\'e space\footnote{See definition \ref{topology:etale_space}.} over $X$ by equipping it with the topology with basis
        \begin{equation}
            \big\{\overline{s}(U):U\text{ open in }X, s\in\mathcal{F}(U)\big\}.
        \end{equation}
        The projection map $\pi$ is given by $\pi:s_x\in\mathcal{F}_x\mapsto x$. The sheafification $\mathcal{F}^\ast$ is then given by the sheaf of sections of $\overline{\mathcal{F}}$.
    \end{construct}

    \begin{example}[Constant sheaf]
        Consider the constant presheaf over $X$ with target $S$ (see example \ref{sheaf:constant_presheaf}). The constant sheaf, denoted by $S_X$, is defined as the associated sheaf of this presheaf. The stalks over every point $x\in X$ can be identified with $S$. The continuous sections $S_X(U)$ are the locally constant functions $f:U\rightarrow S$.
    \end{example}

    \begin{notation}[Category of sheaves]
        \nomenclature[S_Sheaf]{$\mathbf{Sh}(X)$}{Category of sheaves over a topological space $X$.}
        Similar to the case of presheaves one can define a morphism of sheaves as a morphism commuting with the restriction maps.\footnote{When treating a (pre)sheaf as a functor, the (pre)sheaf morphism is a natural transformation.} The sheaves and sheaf morphisms over a space $X$ form a full subcategory of the category of presheaves, denoted by $\mathbf{Sh}(X)$.
    \end{notation}

    \begin{property}
        The category of sheaves $\mathbf{Sh}(X)$ is in fact an elementary topos, called the sheaf topos over $X$. (See property \ref{topoi:sheaf_topos}.)
    \end{property}

    \newdef{Global sections functor}{\index{section}
        Let $X$ be a topological space. The global sections functor $\Gamma(X, -)$ is defined as the functor $\Gamma(X, -):\text{Sh}(X)\rightarrow\text{Set}:\mathcal{F}\rightarrow\mathcal{F}(X)$.
    }
    \begin{property}
        The global sections functor is only left exact.
    \end{property}

\section{Resolutions and cohomology}

    In this section we will only use Abelian sheaves, i.e. sheaves with values in $\mathbf{Ab}$, unless stated otherwise.

    \begin{property}
        Every Abelian sheaf admits an injective resolution.
    \end{property}
    \newdef{Sheaf cohomology group}{\index{cohomology!sheaf}
        Let $\mathcal{F}$ be a sheaf over $X$. Given an injective resolution\footnote{In fact this construction is independent of the chosen injective resolution.} $I$ of $\mathcal{F}$ one defines the sheaf cohomology groups of $\mathcal{F}$ over $X$ as the cohomology groups of the complex
        \begin{equation}
            \cdots\longrightarrow\Gamma(X, I^i)\longrightarrow\Gamma(X, I^{i+1})\longrightarrow\cdots.
        \end{equation}
        The cohomology group $H^0(X, \mathcal{F})$ is equal to $\Gamma(X, \mathcal{F})$.
    }
    \newdef{Acyclic sheaf}{\index{sheaf!acyclic}
        A sheaf is called acyclic if its higher cohomology groups vanish.
    }
    \begin{theorem}[de Rham \& Weil]
        Let $\mathcal{F}$ be a sheaf. There exists an isomorphism between the sheaf cohomology groups defined above and the ones obtained by using an acyclic resolution of $\mathcal{F}$.
    \end{theorem}

    \newdef{Image and kernel}{
        Given a morphism of sheaves $\phi:\mathcal{F}\rightarrow\mathcal{G}$ over a space $X$ one can define the kernel/image presheaves which assign to every open subset $U\subseteq X$ the image/kernel of $\phi_U$. The kernel presheaf is already a sheaf and will be denoted by $\ker(\phi)$. The sheafification of the image presheaf will be denoted by $\im(\phi)$.
    }

    \newdef{Cohomology sheaves}{
        Let $\mathcal{F}^\bullet$ be a complex of sheaves over $X$. The cohomology sheaves $\underline{H}^i(X, \mathcal{F}^\bullet)$ assign to every open subset $U\subseteq X$ the quotient group $\ker(d^i_U)/\im(d^{i-1}_U)$.
    }

\section{Ringed spaces}

    \newdef{Ringed space}{\index{ringed space}
        A ringed space is a topological space $X$ equipped with a sheaf of rings $\mathcal{O}_X$.
    }
    \newdef{Locally ringed space}{
        A ringed space $(X, \mathcal{O}_X)$ is said to be locally ringed if the stalk over every point $x\in X$ is a local ring\footnote{See definition \ref{algebra:local_ring}.}.
    }

\section{\difficult{Simplicial sets}}

    \newdef{Simplex category}{\index{simplex!category}\index{face!map}\index{degeneracy!map}
        \nomenclature[S_simpcat]{$\Delta$}{The simplex category.}
        The simplex category $\Delta$ has as objects the posets of the form $[n] = \{0, \ldots, n\}$ and as morphisms the order-preserving maps.
    }
    \newdef{Simplicial set}{\index{simplicial set}\label{sheaf:simplicial set}
        The category $\mathbf{sSet}$ of simplicial sets is given by the presheaf category $\mathbf{Psh}(\Delta)$. The set $X_n:=X([n])$ is called the set of \textbf{$n$-simplices} in $X$.
    }
    \newdef{Simplicial object}{
        By internalizing the notion of a simplicial set in any category one obtains the definition of a simplicial object, i.e. a simplicial object in a category $\mathbf{C}$ is a $\mathbf{C}$-valued presheaf on $\Delta$. This way a simplicial set is just a simplicial object in \textbf{Set}.
    }
    \remark{Note that the notion of \textbf{simplicial category} can mean two distinct things. In general it will mean a category enriched in $\mathbf{sSet}$. However, conform the above definition, it can also mean a simplicial object in the (2-)category $\mathbf{Cat}$. It can be shown that all simplicially enriched categories are a specific kind of degenerate simplicial object in $\mathbf{Cat}$.}

    \begin{property}\index{face!map}\index{degeneracy!map}
        All morphisms in the simplex category $\Delta$ are generated by two types:
        \begin{itemize}
            \item For every $n$, the unique map $\delta_{n, i}:[n-1]\rightarrow[n]$ which misses the $i^{th}$ element.
            \item For every $n$, the unique map $\sigma_{n, i}:[n+1]\rightarrow[n]$ which duplicates the $i^{th}$ element.
        \end{itemize}
        Under the action of a presheaf this gives the \textbf{face} and \textbf{degeneracy} maps $d_{n, i}$ and $s_{n, i}$. (If the index $n$ is clear, then it is often omitted in the notation.)

        Their fundamental relations are called the \textbf{simplicial identities}:
        \begin{itemize}
            \item $d_i\circ d_j = d_{j-1}\circ d_i$ for $i<j$
            \item $d_i\circ s_j = s_{j-1}\circ d_i$ for $i<j$
            \item $d_i\circ s_j = \text{id}$ for $i=j$ or $i=j+1$
            \item $d_i\circ s_j = s_j\circ d_{i-1}$ for $i>j+1$
            \item $s_i\circ s_j = s_{j+1}\circ s_i$ for $i\leq j$
        \end{itemize}
    \end{property}

    \newdef{Standard simplex}{\index{simplex}\label{sheaf:standard_simplex}
        For every $n$ we define the standard simplicial $n$-simplex $\Delta[n]$ as the Yoneda embedding $\Delta(-, [n])$. We can also define a functor $\func{\Delta_{top}}{\Delta}{Top}$ that maps $[n]$ to the standard topological $n$-simplex $\Delta^n$ (see definition \ref{top:standard_simplex}).
    }
    \begin{property}
        By the Yoneda lemma there exists a natural bijection between the set of $n$-simplices of a simplicial set $X$ and the set of maps $\Delta[n]\rightarrow X$.
    \end{property}

    \begin{construct}[Nerve and realization]
        Consider a general functor $\func{F}{S}{C}$ into a cocomplete category ($\mathbf{S}$ will often be a category of geometric shapes such as the simplex category $\Delta$ or the cube category $\mbox{\mancube}$). Every such functor induces an adjunction
        \begin{gather}
            \mathbf{C}\adj{|-|}{N}\mathbf{Psh}(\mathbf{S}).
        \end{gather}
        The \textbf{realization functor} $|-|$ is defined as the left Kan extension $\text{Lan}_{\mathcal{Y}}F$. The \textbf{nerve functor} is defined as the composition $\mathcal{Y}\circ F^{op}$. (Note that the Yoneda embedding in the definition of the nerve functor is the contravariant version.)

        Note that this definition also holds in the more general enriched setting, i.e. $\mathbf{Psh}(\mathbf{S})\equiv[\mathbf{S}^{op}, \mathcal{V}]$. If we furthermore assume that $\mathbf{C}$ is copowered over $\mathcal{V}$ then we can express the realization procedure as a coend:
        \begin{gather}
            |X| = \int^{s\in\mathbf{S}}Xs\cdot Fs.
        \end{gather}
    \end{construct}

    \begin{example}[Nerve]\index{nerve}
        \nomenclature[S_Nerve]{$N\mathbf{C}$}{The simplicial nerve of a small category $\mathbf{C}$.}
        To every small category \textbf{C} one can associate a simplicial set $N\mathbf{C}$ in the following way. The set $N\mathbf{C}_0$ is given by the set of objects in $\mathbf{C}$. The set $N\mathbf{C}_1$ is given by the set of morphisms in $\mathbf{C}$. Now, for every two composable morphisms $f, g$ one obtains a canonical commuting triangle. Let $N\mathbf{C}_2$ be the set of all these triangles. The higher simplices are defined analogously. Face maps act by composing morphisms or by dropping the exterior morphisms. Degeneracy maps act by inserting an identity morphism.

        Equivalently, one can define the (simplicial) nerve functor in the following way: Every poset $[n]$ as defined above admits a canonical category structure for which the order-preserving maps give rise to the associated functors. This inclusion $\Delta\hookrightarrow\mathbf{Cat}$ induces the functor
        \begin{gather}
            N:\textbf{Cat}\rightarrow[\Delta^{op}, \textbf{Set}]:\mathbf{C}\mapsto\mathbf{Cat}(-, \mathbf{C}).
        \end{gather}
        This way we obtain $N\mathbf{C}_k=\mathbf{Cat}([k], \mathbf{C})$. This object is by definition equivalent to the collection of all strings of $k$ composable morphisms in $\mathbf{C}$.
    \end{example}

    \begin{example}[Geometric realisation]\index{geometric!realisation}
        Consider a simplicial set $X$. From this object one can construct a topological space as follows: First one takes a point for every element in $X_0$. Then one glues 1-simplices between these points using the face maps. The higher simplices are attached analogously.

        For simplicial topological spaces\footnote{These include ordinary simplicial sets since every $n$-simplex $X_n$ can be endowed with the discrete topology.} one can easily give an explicit description:
        \begin{gather}
            |X| := \bigsqcup_{n\in\mathbb{N}}X_n\times\Delta^n / \sim
        \end{gather}
        where the equivalence relation identifies for all morphisms $f\in\text{hom}(\Delta)$ the points $(x, f_*y)$ and $(f^*x, y)$.\footnote{The morphisms $f^*, f_*$ are the ones induced by $X$ and $\Delta_{top}$.} For simplicial sets one can rewrite this as a functor tensor product\footnote{See definition \ref{cat:functor_tensor_product}.}:
        \begin{gather}
            |X| = X\otimes_{\Delta}\Delta_{top}
        \end{gather}
        where $\Delta_{top}$ is the embedding $\Delta\hookrightarrow\mathbf{Top}$.
    \end{example}
    \newdef{Singular set}{\index{singular!set}
        Given a topological space $X$ one can also define a simplicial set $\text{Sing}(X)$. Its components are defined as the set of morphisms from the standard (topological) $n$-simplex to $X$:
        \begin{gather}
            \text{Sing}(X)_n := \mathbf{Top}(\Delta^n, X).
        \end{gather}
        This is the object of relevance in the definition of singular (co)homology (see also the \textit{Dold-Kan correspondence}).
    }

    \begin{property}[Classifying space]\index{bar construction}\index{classifying space}
        For a (discrete) group $G$ one can construct two important objects: the delooping $\mathbf{B}G$ and the classifying space $BG$ (see definitions \ref{cat:group_delooping} and \ref{diff:prin:classifying_space} respectively). As their notations imply there exists some relation between these space: By first taking the nerve of $\mathbf{B}G$ and then going to its geometric realisation we obtain $BG$. In fact this method can be applied to any monoid $A$ to obtain the so-called (two-sided) \textit{bar construction}.
    \end{property}

\subsection{Kan complexes}

    \newdef{Horn}{\index{horn}
        Consider the standard simplex $\Delta[n]$. For all $n\geq1$ and $0\leq k\leq n$ we define the $(n,k)$-horn as the subsimplicial set obtained by removing the $k^{th}$ face from $\partial\Delta[n]$. When $k=0$ or $k=n$ the horn is called an \textbf{outer horn}.
    }

    \newdef{Kan fibration}{\index{Kan!fibration}
        A morphism of simplicial sets that has the right lifting property with respect to all horn inclusions $\Lambda^k[n]\hookrightarrow\Delta[n]$.
    }
    \newdef{Kan complex}{\index{Kan!complex}
        A simplicial set that has all horn fillers or equivalently a simplicial set for which the terminal morphism is a Kan fibration.
    }

    \begin{property}
        A simplicial set is the nerve of a (small) category if and only if all of its inner horns admit a unique filler. If we drop the restriction to inner horns then we obtain the condition for groupoids instead of categories.
    \end{property}

    By relaxing the above requirements we can generalize the notion of a category (due to \textit{Boardman} and \textit{Vogt}):
    \newdef{Quasicategory}{\index{quasicategory}
        A simplicial set that has (not necessarily unique) fillers for all inner horns.
    }
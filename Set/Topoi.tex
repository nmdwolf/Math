\chapter{Topos theory}

\section{Elementary topoi}

	\newdef{Subobject classifier}{\index{subobject!classifier}
		Consider a finitely complete category (in fact the existence of a terminal object suffices). A subobject classifier is a mono\footnote{The symbol for this morphism will become clear in subsection \ref{cat:internal_logic}.} $\texttt{true}:\mathbf{1}\hookrightarrow\Omega$ from the terminal object such that for every mono $\phi:a\hookrightarrow b$ there exists a unique morphism $\chi:b\rightarrow\Omega$ such that the following pullback square exists:
		\begin{figure}[ht!]
			\centering
			\begin{tikzcd}[ampersand replacement=\&, row sep=4em,column sep=4em, minimum width=2em]
				a \arrow[r, "!"] \arrow[d, hook, "\phi"'] \& \mathbf{1} \arrow[d, hook, "\texttt{true}"]\\
				b \arrow[r, "\chi!"] \& \Omega
			\end{tikzcd}
			\caption{Subobject classifier.}
			\label{fig:subobject_classifier}
		\end{figure}
	}
	\begin{adefinition}
		Consider a well-powered category \textbf{C}. The assignment of subobjects Sub$(a)$ to an object $a\in\ob{C}$ is a contravariant functor $\mathcal{S}:\mathbf{C}\rightarrow\mathbf{Set}$. A subobject classifier $\Omega$ is a representation of this functor, i.e. the following isomorphism is natural in $a$:
		\begin{gather}
			\mathcal{S}(a)\cong\text{hom}_{\mathbf{C}}(a, \Omega)
		\end{gather}
	\end{adefinition}
	
	\begin{example}
		The category $\mathbf{Set}$ has a subobject classifier, namely the 2-element set.
	\end{example}

	\newdef{Elementary topos}{\index{topos!elementary}
		An elementary topos is a Cartesian closed category which has a subobject classifier. Equivalently one can define an elementary topos as a finitely complete category in which all power objects exist.
		
		The power object $Pa$ of $a$ is related to the subobject classifier $\Omega$ through the exponential by the following relation:
		\begin{gather}
			Pa = \Omega^a
		\end{gather}
	}
	
	\begin{property}[Slice category]
		Let $\mathcal{C}$ be a topos. For every object $c\in\ob{C}$ the slice category $\mathbf{C}/c$ is also a topos. The subobject classifier is given by $\pi_2:\Omega\times c\rightarrow c$.
	\end{property}
	
	\begin{property}[Balanced]
		All monos in a topos are regular. Hence every mono arises as an equalizer and every epic equalizer is necessarily an isomorphism so it follows that every topos is balanced\footnote{See definition \ref{category:balanced}.}.
	\end{property}
	
	\begin{property}[Epi-mono factorization]\index{image}
		Every morphism $f:a\rightarrow b$ in a topos factorizes uniquely as an epi followed by a mono:
		\begin{gather}
			a\overset{e}{\twoheadrightarrow} c\overset{m}{\rightarrowtail} b
		\end{gather}
		The mono is called the \textbf{image} of $f$.
	\end{property}

\subsection{Topological sheafs}

	\begin{property}[Presheaf topos]\label{topoi:sheaf_topos}
		Consider the presheaf category\footnote{See chapter \ref{chapter:sheaf} for the application of sheafs in topology.} $\textbf{Psh}(X) = \widehat{\textbf{Open}(X)}$ over a topological space $X$. This category is an elementary topos where the subobject classifier $\Omega$ is defined as follows:
		\begin{gather}
			\Omega(U) = \{V:V\text{ is an open subset of }U\}
		\end{gather}
	\end{property}
	\remark{In fact the presheaf category $\mathbf{Psh}(\mathbf{C})$ for any (small) category $\mathbf{C}$ is an elementary topos. See property \ref{topoi:presheaf_topos} below.}
	
	\begin{construct}[Sheafs and \'etal\'e bundles]
		Let $X$ be a topological space. The functor \[I:\textbf{Open}(X)\rightarrow\textbf{Top}/X:U\mapsto(U\hookrightarrow X)\] induces the following adjunction:
		\begin{gather}
			\label{category:etale_adjunction}
			\textbf{Psh}(X) \adj{\Gamma}{E} \textbf{Top}/X
		\end{gather}
		The slice category on the right-hand side is equivalently the category of (topological) bundles\footnote{See chapter \ref{diff:chapter:bundles}.} over $X$. Both directions of the adjunction have a clear interpretation. The left adjoint assigns to every bundle its sheaf of (global) sections. The right adjoint assigns to every presheaf its bundle of germs.
		
		This adjunction restricts to an equivalence on the subcategories on which the unit and counit morphisms are isomorphisms. These subcategories are called the \textbf{sheaf} and \textbf{\'etal\'e bundle} categories respectively:
		\begin{gather}
			\textbf{Sh}(X) \cong \textbf{Et}(X)
		\end{gather}
		The left category is the well-knwon catgeory of sheafs on a topological space $X$. The right category is he full subcategory on local homeomorphisms, i.e. \'etal\'e spaces as defined in chapter 
	\end{construct}
	
	\begin{property}[Associated sheaf]
		The inclusion functor $\textbf{Sh}(X)\hookrightarrow\textbf{Psh}(X)$ admits a left adjoint. This is exactly the sheafification functor which assigns to every presheaf its associated sheaf. This functor is given by the composition $\Gamma\circ E$.\footnote{This amounts to construction \ref{sheaf:etale_construction}.}
		
		The fact that the counit of adjunction \ref{category:etale_adjunction} restricts to an isomorphism on the full subcategory $\textbf{Sh}(X)$ is equivalent to the fact that the sheafification of a sheaf $\Gamma$ is again $\Gamma$.
	\end{property}

	
\section{Internal logic}\label{cat:internal_logic}

	In this subsection we consider finitely complete categories which admit a subobject classifier (they don't have to be a topos).
	
	\newdef{Truth value}{\index{truth value}
		A global element of the subobject classifier, i.e. a morphism $\mathbf{1}\rightarrow\Omega$. The subobject classifier $\Omega$ is therefore sometimes called the \textbf{object of truth values}.
	}
	
	\begin{property}[Internal Heyting algebra]
		For all objects $X$ in an elementary topos, the poset of subobjects Sub$(X)$ has the structure of a Heyting algebra\footnote{See definition \ref{set:heyting}.}. Hence every topos canonically gives an external Heyting algebra, namely Sub$(\mathbf{1})$. Furthermore, every power object is an internal Heyting algebra. This in particular includes the subobject classifier $\Omega=P{\mathbf{1}}$.
	\end{property}

\section{Geometric morphisms}

	\newdef{Logical morphism}{\index{morphism!logical}
		Let $\mathcal{E}, \mathcal{F}$ be (elementary) toposes. A morphism $f:\mathcal{E}\rightarrow\mathcal{F}$ is called a logical morphism if it preserves finite limits, exponential objects and subobject classifiers.
	}

	\newdef{Geometric morphism}{\index{morphism!geometric}\index{direct!image}\index{inverse!image}
		Let $\mathcal{E}, \mathcal{F}$ be (elementary) toposes. A geometric morphism $f:\mathcal{E}\rightarrow\mathcal{F}$ consists of an adjunction \[\mathcal{E}\adj{f^*}{f_*}\mathcal{F}\] where the left adjoint preserves finite limits. The right adjoint $f_*$ is called the \textbf{direct image} part of $f$ and the left adjoint is called the \textbf{inverse image} part.
	}
	
	\begin{example}[Topological spaces]
		Every continuous map $f:X\rightarrow Y$ induces a geometric morphism
		\begin{gather}
			\textbf{Sh}(X)\adj{f^*}{f_*}\textbf{Sh}(Y)
		\end{gather}
		where the direct image functor $f_*$ is defined as follows:
		\begin{gather}
			f_*F(U) = F(f^{-1}U)
		\end{gather}
		for any sheaf $F\in\textbf{Sh}(X)$ and any open subset $U\in\textbf{Open}(Y)$.
	\end{example}
	
	By the previous example the global elements $\ast\rightarrow X$ of a topological space induce geometric morphisms of the form $\textbf{Sh}(\ast)\rightarrow\textbf{Sh}(X)$. By noting that $\textbf{Sh}(\ast)=\textbf{Set}$ we obtain the following generalization:
	\newdef{Point}{\index{point}
		A point of a topos $\mathcal{E}$ is a geometric morphism $\textbf{Set}\rightarrow\mathcal{E}$.
	}
	
	\newnot{Category of toposes}{
		\nomenclature[S_Topos]{$\mathbf{Topos}$}{The 2-category of (elementary) topoi and geometric morphisms.}
		The category of elementary toposes and geometric morphisms forms a (2-)category which we will denote by \textbf{Topos}.
	}

\section{Grothendieck topos}
	
	\newdef{Sieve}{\index{sieve}
		Let \textbf{C} be a small category. A sieve $S$ on \textbf{C} is a fully faithfull discrete fibration $S\hookrightarrow\textbf{C}$.
		
		A sieve $S$ on an object $c\in\textbf{C}$ is a sieve in the slice category $\textbf{C}/c$. This means that $S$ is a subset of $\text{ob}(\mathbf{C}/c)$ that is closed under \textit{precomposition}, i.e. if $b\rightarrow c\in S$ and $a\rightarrow b\in\text{hom}(\textbf{C})$ then the composition $a\rightarrow b\rightarrow c\in S$.
		
		All of this can be summarized by saying that a sieve on an object $c\in\ob{C}$ is a subfunctor of the hom-functor $\textbf{C}(-, c)$.
	}
	
	\begin{example}[Maximal sieve]
			Let $C$ be a category. The maximal sieve on $c\in\ob{C}$ is the collection of all morphisms $\{f\in\text{hom}(\textbf{C}):\cod(f) = c\}$ or equivalently all of $\text{ob}(\mathbf{C}/c)$.
	\end{example}
	\begin{example}[Pullback sieve]
		Consider a morphism $f:a\rightarrow b$. Given a sieve $S$ on $b$ one can construct the pullback sieve $f^*S$ on $a$ as the sieve of morphisms in $S$ which factor through $f$:
		\begin{gather}
			f^*S(a) = \{g:c\rightarrow a:f\circ g\in S(b)\}
		\end{gather}
	\end{example}
	
	\newprop{Presheaf topos}{\index{presheaf!topos}\label{topoi:presheaf_topos}
		Consider the presheaf category $\mathbf{Psh}(\mathbf{C})$ for an arbitrary (small) category $\mathbf{C}$. This category is in fact an elementary topos where the subobject classifier is defined on each object in the following way:
		\begin{equation}
			\underline{\Omega}(c) = \{S: S\text{ is a sieve on }c\}
		\end{equation}
		The action on a morphism $f:a\rightarrow b$ in $\mathbf{C}$ gives the morphism $\underline{\Omega}(f)$ which sends a sieve $S$ to its pullback sieve $f^*S$.
		
		The morphism $\texttt{true}:\underline{\mathbf{1}}\hookrightarrow\underline{\Omega}$ is defined as the natural transformation assigning to every object its maximal sieve. For every subobject $\underline{K}\hookrightarrow\underline{X}$ the characteristic morphism $\chi_K$ is defined as follows: Consider an object $c\in\ob{C}$ and element $x\in\underline{X}(c)$. The morphism $\chi_K|_c$ is then defined as follows:
		\begin{gather}
			\chi_K|_c(x) = \{f\in\mathbf{C}(d, c):\underline{X}(f)(x)\in\underline{K}(d)\}
		\end{gather}
	}

	The following definition is due to Giraud (the original definition used the notion of a \textit{cover}):
	\newdef{Grothendieck topology}{\index{Grothendieck!topology}\index{covering!sieve}\index{cover}
		A Grothendieck topology on a category is a function $J$ assigning to every object a collection of sieves satisfying the following conditions:
		\begin{itemize}
			\item (Identity\footnote{This condition can be rephrased in terms of isomorphisms: The sieve generated by any isomorphism is a covering sieve.}): For every object $c$ the maximal sieve $M_c$ is an element of $J(c)$.
			\item (Base change): If $S\in J(c)$ then $f^*S\in J(d)$ for every morphism $f:d\rightarrow c$.
			\item (Locality): Consider a sieve $S$ on $c$. If there exists a sieve $R\in J(c)$ such that for every morphism $(f:d\rightarrow c)\in R$ the pullback sieve $f^*S\in J(d)$ then $S\in J(c)$.
		\end{itemize}
		The sieves in $J$ are called (J-)\textbf{covering sieves}. A collection of morphisms with codomain $c\in\ob{C}$ is called a \textbf{cover}\footnote{Sometimes this term is also used to denote any collection of morphism with common codomain $c$, i.e. without reference to a covering sieve.} of $c$ if the sieve generated by thse morphisms is a covering sieve on $c$.
	}
	\begin{example}[Topological spaces]
		These conditions have the following interpretation in the case of topological coverings:
		\begin{itemize}
			\item The collection of all open subsets covers a space $U$.
			\item If $\{U_i\}_{i\in I}$ covers $U$ then $\{U_i\cap V\}_{i\in I}$ covers $U\cap V$.
			\item If $\{U_i\}_{i\in I}$ covers $U$ and if for every $i\in I$ the collection $\{U_{ij}\}_{j\in J_i}$ covers $U_i$ then $\{U_{ij}\}_{i\in I, j\in J_i}$ covers $U$.
		\end{itemize}
		
		The canonical Grothendieck topology on \textbf{Open}$(X)$ is given by the sieves $S=\{U_i\hookrightarrow U\}_{i\in I}$ where $\bigcup_{i\in I}U_i = U$. This topology is denoted by $J_{\text{Open}(X)}$.
	\end{example}

	\newdef{Site}{
		A (small) category equipped with a Grothendieck topology $J$.
	}
	
	\newdef{Matching family}{
		Consider a presheaf $F\in\widehat{\textbf{C}}$ together with a sieve $S$ on $c\in\ob{C}$. A matching family for $S$ with respect to $F$ is a natural transformation $\alpha:S\Rightarrow F$ between $S$, regarded as a subfunctor of hom$(-, c)$, and $F$.
		
		More explicitly it is an assignment of an element $x_f\in Fd$ to every morphism $(f:d\rightarrow c)\in S$ such that:
		\begin{gather}
			F(g)(x_f) = x_{f\circ g}
		\end{gather}
		for all morphisms $g:e\rightarrow d$. This can also be restated differently. A matching family for $S$ with respect to $F$ is a set of elements $\{x_f\}_{f\in S(c)}$ such that for all covering morphisms $f:d\rightarrow c, g:e\rightarrow c\in S(c)$ and all morphisms $f':z\rightarrow d, g': z\rightarrow e$ such that $f\circ f'=g\circ g'$ the following equations holds:
		\begin{gather}
			\label{topoi:matching_family}
			F(f')(x_f) = F(g')(x_g)
		\end{gather}
		
		Given such a matching family one calls an element $z\in Fc$ an \textbf{amalgamation} if it satisfies
		\begin{gather}
			F(f)(z) = x_f
		\end{gather}
		for all morphisms $f\in S(d)$.
	}
	\remark{If the base category has all pullbacks, for example if it is a topos on its own, then one can restrict the above commuting diagrams to the pullback diagrams of morphisms in the sieve $S$.}
	
	\newdef{Sheaf}{\index{sheaf}\index{presheaf!separated}
		\nomenclature[S_Shsite]{$\mathbf{Sh}(\mathbf{C}, J)$}{Category of $J$-sheaves on a site $(\mathbf{C}, J)$.}
		Consider a site $(\textbf{C}, J)$. A presheaf $F$ on $\textbf{C}$ is called a $J$-sheaf if every matching family for any covering sieve (on any object in $\mathbf{C}$) in $J$ admits a unique amalgamation\footnote{If there exists at most one amalgamation then the presheaf is said to be \textbf{separated}.}.
		
		From the natural transformation point of view this condition amounts to the existence of a unique extension of every natural transformation $\alpha:S\Rightarrow F$ along the inclusion $S\hookrightarrow\text{hom}(-, c)$ or equivalently of an isomorphism\footnote{If this map is merely a monomorphism then the presheaf is separated. (See also the previous footnote.)} $\iota_S:\text{Nat}(h_{\mathbf{C}}, F)\rightarrow\text{Nat}(S, F)$.
		
		The category $\mathbf{Sh}(\mathbf{C}, J)$ of $J$-sheaves on the site $(\mathbf{C}, J)$ is the full subcategory of $\widehat{\textbf{C}}$ on the presheaves which satisfy the above condition.
	}
	
	\begin{example}[Topological spaces]
		The usual category of sheaves $\textbf{Sh}(X)$ on a topological space $X$ is obtained as the category of sheaves on the site $(\textbf{Open}(X), J_{\text{open}(X)})$. Since the morphisms in the covering sieves are exactly the inclusion maps $U_i\hookrightarrow U$, the pullback of two such morphisms is given by the intersection $U_i\cap U_j$. Hence the condition for a matching family, as formulated in equation \ref{topoi:matching_family} above, gives the second part of definition \ref{sheaf:def}. The uniqueness of an amalgamation is equivalent to the first part of that definition.
	\end{example}
	
	\newdef{Grothendieck topos}{\index{Grothendieck!topos}
		A category\footnote{The fact that a Grothendieck topos is an elementary topos follows from the fact that \textbf{Set} is a topos.} equivalent to the category of sheaves on a (small) site.
	}
	
\subsection{Lawvere-Tierney topology}

	\newdef{Lawvere-Tierney topology}{\index{Lawvere-Tierney}
		As noted in section \ref{cat:internal_logic} on the internal logic of elementary topoi, the subobject classifier $\Omega$ has the structure of an internal Heyting algebra and in particular of a meet-semilattice (where the meet is given by the pullback of morphisms). This internal poset, viewed as an internal category, admits the construction of a closure operator\footnote{See definition \ref{cat:closure_operator}.} $j:\Omega\rightarrow\Omega$ satisfying the following conditions:
		\begin{gather}
			j\circ\land = \land\circ(j\times j)
		\end{gather}
		which states\footnote{This is not a trivial restatement.} that $j$ is (internally) order-preserving.
	}
	\begin{remark}
		The condition satisfied by the unit morphism in the definition of a closure operator can also be reformulated as follows in this context:
		\begin{gather}
			j\circ\texttt{true} = \texttt{true}
		\end{gather}
	\end{remark}
	The Lawvere-Tierney operator also induces a ''closure operator'' on all posets Sub$(X)$ in the topos. Given an object $X$ and a subobject $U\in\text{Sub}(X)$ one defines the closure $j_\ast(U)\in\text{Sub}(X)$ as the subobject classified by the characteristic map $j\circ\chi_U:X\rightarrow\Omega$.
	
	\newdef{Dense object}{\index{dense}
		Given a Lawvere-Tierney topology $j:\Omega\rightarrow\Omega$, a subobject $U\in\text{Sub}(X)$ is said to be dense (in $X$) if it satisfies $j_\ast(U) = X$.
	}
	\newdef{Sheaf}{\index{sheaf}
		Given a Lawvere-Tierney topology $j:\Omega\rightarrow\Omega$ on a topos $\mathcal{E}$, one calls an object $S\in\ob{E}$ a $j$-sheaf if for all dense morphisms $U\hookrightarrow X$ the induced map \[\mathcal{E}(X, S)\rightarrow\mathcal{E}(U, S)\] is a bijection.
	}
	
	\begin{property}
		For the presheaf topos on a small category $\mathbf{C}$ the Grothendieck topologies on $\mathbf{C}$ and Lawvere-Tierney topologies on $\mathbf{Psh}(\mathbf{C})$ are equivalent.
		\begin{proof}[Sketch of proof]
			Since a Grothendieck topology assigns to every object a collection of sieves, we find by property \ref{topoi:presheaf_topos} that $J(c)\subseteq\Omega_{\mathbf{Psh}}(c)$ for all $c\in\ob{C}$. By the base change condition of Grothendieck topologies this relation is natural in $c$ and hence $J$ is a subobject of $\Omega_{\mathbf{Psh}}$. We thus find a characteristic morphism $j:\Omega_{\mathbf{Psh}}\rightarrow\Omega_{\mathbf{Psh}}$ which can be proven (by the other conditions of Grothendieck topologies) to define a Lawvere-Tierney topology on $\mathbf{Psh}(\mathbf{C})$. Conversely, a Lawvere-Tierney topology is a morphism $j:\Omega\rightarrow\Omega$ and hence determines a unique subobject of $\Omega_{\mathbf{Psh}}$, i.e. a unique collection of sieves for every object $c\in\ob{C}$. From the conditions on Lawvere-Tierney topologies one can then prove that this collection satisfies the conditions of a Grothendieck topology.
		\end{proof}
	\end{property}
	\sremark{We can conclude that Lawvere-Tierney topologies generalize Grothendieck topologies, which determine a topology on presheaf topoi, to general topoi.}

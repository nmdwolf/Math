\chapter{Topos theory}

\section{Elementary topoi}

	\newdef{Subobject classifier}{\index{subobject!classifier}
		Consider a finitely complete category (in fact the existence of a terminal object suffices). A subobject classifier is a mono\footnote{The symbol for this morphism will become clear in subsection \ref{cat:internal_logic}.} $\text{true}:\mathbf{1}\hookrightarrow\Omega$ from the terminal object such that for every mono $\phi:a\hookrightarrow b$ there exists a unique morphism $\chi:b\rightarrow\Omega$ such that the following pullback square exists:
		\begin{figure}[ht!]
			\centering
			\begin{tikzcd}[ampersand replacement=\&, row sep=4em,column sep=4em, minimum width=2em]
				a \arrow[r, "!"] \arrow[d, hook, "\phi"'] \& \mathbf{1} \arrow[d, hook, "\text{true}"]\\
				b \arrow[r, "\chi!"] \& \Omega
			\end{tikzcd}
			\caption{Subobject classifier.}
			\label{fig:subobject_classifier}
		\end{figure}
	}
	\begin{adefinition}
		Consider a well-powered category $C$. The assignment of subobjects Sub$(a)$ to an object $a\in\ob{C}$ is a contravariant functor $\mathcal{S}:C\rightarrow\text{Set}$. A subobject classifier $\Omega$ is a representation of this functor, i.e. the following isomorphism is natural in $a$:
		\begin{equation}
			\mathcal{S}(a)\cong\text{hom}(a, \Omega)
		\end{equation}
	\end{adefinition}
	
	\begin{example}
		The category Set has a subobject classifier, namely the 2-element set.
	\end{example}
	
	\begin{property}
		Using Yoneda's lemma one obtains following statement: The subobject classifier $\Omega$ has a \textit{generic} element $\mathfrak{t}$ such that for every object $a\in\ob{C}$ and every element $b\in\text{Sub}(a)$ there exists a unique morphism $\chi:a\rightarrow\Omega$ such that $m = \chi^*\mathfrak{t}$.
	\end{property}

	\newdef{Elementary topos}{\index{topos!elementary}
		An elementary topos is a cartesian closed category which has a subobject classifier and in which all finite limits exist. Equivalently one can define an elementary topos as finitely complete category in which all power objects.
	}
	
\section{Internal logic}\label{cat:internal_logic}

	In this subsection we consider categories with all finite limits and a subobject classifier (so they don't have to be a topos).
	
	\newdef{Truth value}{\index{truth value}
		A global element of the subobject classifier, i.e. a morphism $\mathbf{1}\rightarrow\Omega$. The subobject classifier $\Omega$ is hence also sometimes called the \textbf{object of truth values}.
	}
	
	\begin{property}
		The subobject classifier is an internal poset.
	\end{property}
	
\section{Geometric morphisms}

	\newdef{Logical morphism}{\index{morphism!logical}
		Let $\mathcal{E}, \mathcal{F}$ be (elementary) toposes. A morphism $f:\mathcal{E}\rightarrow\mathcal{F}$ is called a logical morphism if it preserves finite limits, exponential objects and subobject classifiers.
	}

	\newdef{Geometric morphism}{\index{morphism!geometric}\index{direct!image}\index{inverse!image}
		Let $\mathcal{E}, \mathcal{F}$ be toposes. A morphism $f:\mathcal{E}\rightarrow\mathcal{F}$ is called a geometric morphism if the left adjoint in the induced adjunction \[\mathcal{E}\adj{f^*}{f_*}\mathcal{F}\] preserves finite limits. The right adjoint $f_*$ is called the \textbf{direct image} part of $f$ and the left adjoint is called the \textbf{inverse image} part.
	}
	
	\begin{example}[Topological spaces]
		Every continuous map $f:X\rightarrow Y$ induces a geometric morphism
		\begin{equation}
			\textbf{Sh}(X)\adj{f^*}{f_*}\textbf{Sh}(Y)
		\end{equation}
		where the direct image functor $f_*$ is defined as follows:
		\begin{equation}
			f_*F(U) = F(f^{-1}U)
		\end{equation}
		for any sheaf $F\in\textbf{Sh}(X)$ and any open subset $U\in\textbf{Open}(Y)$.
	\end{example}
	
	By the previous example the global elements $\ast\rightarrow X$ of a topological space induce geometric morphisms of the form $\textbf{Sh}(\ast)\rightarrow\textbf{Sh}(X)$. By noting that $\textbf{Sh}(\ast)=\textbf{Set}$ we obtain the following generalization:
	\newdef{Point}{\index{point}
		A point of a topos $\mathcal{E}$ is a geometric morphism $\textbf{Set}\rightarrow\mathcal{E}$.
	}
	
	\newnot{Category of toposes}{
		The category of elementary toposes and geometric morphisms forms a (2-)category which we will denote by \textbf{Topos}.
	}
	
\section{Sheaf topos}

	\begin{property}[Presheaf topos]
		Consider the presheaf category $\textbf{Psh}(X) = \widehat{\textbf{Open}(X)}$ over a topological space $X$. This category is an elementary topos where the subobject classifier $\Omega$ is defined as follows:
		\begin{equation}
			\Omega(U) = \{V:V\text{ is an open subset of }U\}
		\end{equation}
	\end{property}
	
	\begin{construct}[Sheafs and \'etal\'e bundles]
		Let $X$ be a topological space. The functor \[I:\textbf{Open}(X)\rightarrow\textbf{Top}/X:U\mapsto(U\hookrightarrow X)\] induces the following adjunction:
		\begin{equation}
			\label{category:etale_adjunction}
			\textbf{Psh}(X) \adj{\Gamma}{E} \textbf{Top}/X
		\end{equation}
		
		The slice category on the right-hand side is equivalently the category of (topological) bundles\footnote{See chapter \ref{diff:chapter:bundles}.} over $X$. Both directions of the adjunction have a clear interpretation. The left adjoint assigns to every bundle its sheaf of (global) sections. The right adjoint assigns to every presheaf its bundle of germs.
		
		This adjunction restricts to an equivalence on the subcategories on which the unit and counit morphisms are isomorphisms. These subcategories are called the \textbf{sheaf} and \textbf{\'etal\'e bundle} categories respectively:
		\begin{equation}
			\textbf{Sh}(X) \cong \textbf{Et}(X)
		\end{equation}
	\end{construct}
	
	\begin{property}[Associated sheaf]
		The inclusion functor $\textbf{Sh}(X)\hookrightarrow\textbf{Psh}(X)$ admits a left adjoint. This is exactly the sheafification functor which assigns to every presheaf its associated sheaf. This functor is given by the composition $\Gamma\circ E$.\footnote{This amounts to construction \ref{sheaf:etale_construction}.}
		
		The fact that the counit of adjunction \ref{category:etale_adjunction} restricts to an isomorphism on the full subcategory $\textbf{Sh}(X)$ is equivalent to the fact that the sheafification of a sheaf $\Gamma$ is again $\Gamma$.
	\end{property}

\section{Grothendieck topos}

	\newdef{Discrete fibration}{\index{fibration}
		Let $\func{F}{A}{B}$ be a functor. $F$ is a discrete fibration if for every object $A\in\ob{A}$ and every morphism $f:B\rightarrow FA$ in $\textbf{B}$ there exists a unique morphism $g:C\rightarrow A$ in $\textbf{A}$ such that $F(g) = f$, where $B\in\ob{B}, C\in\ob{A}$.
	}
	
	\newdef{Sieve}{\index{sieve}
		Let \textbf{C} be a small category. A sieve $S$ on \textbf{C} is a fully faithfull discrete fibration $S\hookrightarrow\textbf{C}$.
		
		A sieve $S$ on an object $c\in\textbf{C}$ is a sieve in the slice category $\textbf{C}/c$. This means that $S$ is a subset of $\ob{C}/c$ that is closed under \textit{precomposition}, i.e. if $b\rightarrow c\in S$ and $a\rightarrow b\in\text{hom}(\textbf{C})$ then the composition $a\rightarrow b\rightarrow c\in S$.
		
		All of this can be summarized by saying that a sieve on an object $c\in\ob{C}$ is a subfunctor of the hom-functor $\textbf{C}(-, c)$.
	}
	
	\begin{example}[Maximal sieve]
			Let $C$ be a category. The maximal sieve on $c\in\ob{C}$ is the collection of all morphisms $\{f\in\text{hom}(\textbf{C}):\cod(f) = c\}$.
	\end{example}

	The following definition is due to Giraud (the original definition used covers).
	\newdef{Grothendieck topology}{\index{Grothendieck!topology}\index{covering!sieve}
		A Grothendieck topology on a category is a function $J$ assigning to every object a collection of sieves satisfying the following conditions:
		\begin{itemize}
			\item For every object $c$ the maximal sieve $M_c$ is an element of $J(c)$.
			\item If $S\in J(c)$ then $f^*S\in J(d)$ for every morphism $(f:d\rightarrow c)\in S$.
			\item Consider a sieve $S$ on $c$. If there exists a sieve $R\in J(c)$ such that for every morphism $(f:d\rightarrow c)\in R$ the pullback sieve $f^*S\in J(d)$ then $S\in J(c)$.
		\end{itemize}
		The sieves in $J$ are called (J-)\textbf{covering sieves}. 
	}
	\begin{example}[Topological spaces]
		These conditions have the following interpretation in the case of topological coverings:
		\begin{itemize}
			\item The collection of all open subsets covers a space $U$.
			\item If $\{U_i\}_{i\in I}$ covers $U$ then $\{U_i\cap V\}_{i\in I}$ covers $U\cap V$.
			\item If $\{U_i\}_{i\in I}$ covers $U$ and if for every $i\in I$ the collection $\{U_{ij}\}_{j\in J_i}$ covers $U_i$ then $\{U_{ij}\}_{i\in I, j\in J_i}$ covers $U$.
		\end{itemize}
		
		The canonical Grothendieck topology on \textbf{Open}$(X)$ is given by the sieves $S=\{U_i\hookrightarrow U\}_{i\in I}$ where $\bigcup_{i\in I}U_i = U$. This topology is denoted by $J_{\text{Open}(X)}$.
	\end{example}

	\newdef{Site}{
		A (small) category equipped with a Grothendieck topology $J$.
	}
	
	\newdef{Matching family}{
		Consider a presheaf $F\in\text{ob}(\widehat{\textbf{C}})$ together with a sieve $S$ on $c\in\ob{C}$. A matching family for $S$ with respect to $F$ is a natural transformation $\alpha:S\implies F$ between $S$ regarded as a subfunctor of hom$(-, c)$ and $F$.
		
		More explicitly it is as assignment of an element $x_f\in Fd$ to every morphism $(f:d\rightarrow c)\in S$ such that:
		\begin{equation}
			Fg(x_f) = x_{f\circ g}
		\end{equation}
		for all morphisms $g:e\rightarrow d$.
		
		Given such a matching family one calls an element $z\in Fc$ an \textbf{amalgamation} if it satisfies
		\begin{equation}
			Ff(z) = x_f
		\end{equation}
		for all morphisms $f\in S$.
	}
	\newdef{Sheaf}{\index{sheaf}
		Consider a site $(\textbf{C}, J)$. A presheaf $F$ on $\textbf{C}$ is called a $J$-sheaf if every matching family for any covering sieve (on any object in $\mathbf{C}$) in $J$ admits a unique amalgamation.
		
		From the natural transformation point of view this condition amounts to the existence of a unique extension of $\alpha:S\implies F$ along the inclusion $S\hookrightarrow\text{hom}(-, c)$.
		
		The category $\textbf{Sh}(\mathbf{C}, J)$ of $J$-sheaves on the site $(\mathbf{C}, J)$ is the full subcategory of $\widehat{\textbf{C}}$ on the presheaves which satisfy the above condition.
	}
	
	\begin{example}[Topological spaces]
		The usual category of sheaves $\textbf{Sh}(X)$ on a topological space $X$ is obtained as the category of sheaves on the site $(\textbf{Open}(X), J_{\text{open}(X)})$.
	\end{example}
	
	\newadef{Grothendieck topos}{\index{Grothendieck!topos}
		A category\footnote{The fact that a Grothendieck topos is indeed an elementary topos follows from the fact that \textbf{Set} is a topos.} equivalent to a category of sheaves on a (small) site.
	}
	
	\begin{property}[Balanced]
		All monics and epics in a Grothendiek topos are regular. Hence property \ref{category:regular_iso} implies that every bimorphism is in fact an isomorphism and it follows that every Grothendieck topos is balanced\footnote{See definition \ref{category:balanced}.}.
	\end{property}
	
	\begin{property}[Epi-mono factorization]\index{image}
		Every morphism $f:a\rightarrow b$ in a Grothendieck topos factorizes uniquely as an epi followed by a mono:
		\begin{equation}
			a\overset{e}{\twoheadrightarrow} c\overset{m}{\rightarrowtail} b
		\end{equation}
		The mono is called the \textbf{image} of $f$.
	\end{property}

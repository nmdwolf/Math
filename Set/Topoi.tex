\chapter{\difficult{Topos theory}}

    The main reference for this chapter is \cite{johnstone}. Other useful references include \cite{caramello}. For an introduction to stacks and descent see \cite{vistoli}.

\section{Elementary topoi}

    \newdef{Subobject classifier}{\index{subobject!classifier}
        Consider a finitely complete category (in fact the existence of a terminal object suffices). A subobject classifier is a mono\footnote{The symbol for this morphism will become clear in subsection \ref{cat:internal_logic}.} $\texttt{true}:1\hookrightarrow\Omega$ from the terminal object such that for every mono $\phi:a\hookrightarrow b$ there exists a unique morphism $\chi:b\rightarrow\Omega$ such that the following pullback square exists:
        \begin{figure}[ht!]
            \centering
            \begin{tikzpicture}
                \node (A) at (0, 0) {$a$};
                \node (B) at (0, -2) {$b$};
                \node (1) at (2, 0) {$1$};
                \node (O) at (2, -2) {$\Omega$};
                \draw[->] (A) -- node[above]{$!$} (1);
                \draw[right hook->] (A) -- node[left]{$\phi$} (B);
                \draw[right hook->] (1) -- node[right]{$\texttt{true}$} (O);
                \draw[->] (B) -- node[above]{$\chi!$} (O);
            \end{tikzpicture}
            \caption{Subobject classifier.}
            \label{fig:subobject_classifier}
        \end{figure}
    }
    \begin{adefinition}
        Consider a well-powered category $\mathbf{C}$. The assignment of subobjects $\text{Sub}(a)$ to an object $a\in\ob{C}$ is a contravariant functor $\text{Sub}:\mathbf{C}\rightarrow\mathbf{Set}$. A subobject classifier $\Omega$ is a representation of this functor, i.e. the following isomorphism is natural in $a$:
        \begin{gather}
            \text{Sub}(a)\cong\mathbf{C}(a, \Omega).
        \end{gather}
    \end{adefinition}

    \begin{example}
        The category $\mathbf{Set}$ has a subobject classifier, namely the 2-element set.
    \end{example}

    \newdef{Elementary topos}{\index{topos!elementary}
        An elementary topos is a finitely complete Cartesian closed category containing a subobject classifier. Equivalently, one can define an elementary topos as a finitely complete category in which all power objects exist.

        The power object $Pa$ of $a$ is related to the subobject classifier $\Omega$ by the following relation:
        \begin{gather}
            Pa = \Omega^a.
        \end{gather}
    }
    \begin{remark}[Finite colimits]
        The original definition by \textit{Lawvere} also required the existence of finite colimits. However, it can be proven that finite cocompleteness follows from the other axioms.
    \end{remark}

    \begin{theorem}[Fundamental theorem of topos theory]\index{fundamental theorem!of topos theory}
        Let $\mathcal{C}$ be a topos. For every object $c\in\ob{C}$ the slice category $\mathbf{C}/c$ is also a topos. The subobject classifier is given by $\pi_2:\Omega\times c\rightarrow c$.
    \end{theorem}

    \begin{property}[Balanced]
        All monos in a topos are regular. Hence, every mono arises as an equalizer and every epic equalizer is necessarily an isomorphism. It follows that every topos is balanced (see definition \ref{category:balanced}).
    \end{property}

    \begin{property}[Epi-mono factorization]\index{image}
        Every morphism $f:a\rightarrow b$ in a topos factorizes uniquely as an epi followed by a mono:
        \begin{gather}
            a\overset{e}{\twoheadrightarrow} c\overset{m}{\rightarrowtail} b.
        \end{gather}
        The mono is called the \textbf{image} of $f$.
    \end{property}

\section{Internal logic}\label{cat:internal_logic}

    In this subsection we consider finitely complete categories which admit a subobject classifier (they don't have to be a topos).

    \newdef{Truth value}{\index{truth value}
        A global element of the subobject classifier, i.e. a morphism $1\rightarrow\Omega$. The subobject classifier $\Omega$ is therefore sometimes called the \textbf{object of truth values}.
    }

    \begin{property}[Internal Heyting algebra]
        For all objects $x$ in an elementary topos, the poset of subobjects $\text{Sub}(x)$ has the structure of a Heyting algebra \ref{set:heyting}. Hence every topos canonically gives an external Heyting algebra, namely $\text{Sub}(1)$. Furthermore, every power object is an internal Heyting algebra. This in particular includes the subobject classifier $\Omega=P1$.
    \end{property}

\section{Geometric morphisms}

    \newdef{Base change}{\index{base change}
        Consider a category $\mathbf{C}$ with pullbacks. For every morphism $f:a\rightarrow b$ one can define a functor $f^*:\mathbf{C}/b\rightarrow\mathbf{C}/a$. This functor acts by pullback along $f$.
    }

    \newdef{Logical morphism}{\index{morphism!logical}
        Let $\mathcal{E}, \mathcal{F}$ be (elementary) topoi. A morphism $f:\mathcal{E}\rightarrow\mathcal{F}$ is called a logical morphism if it preserves finite limits, exponential objects and subobject classifiers.
    }
    \begin{property}
        If a logical morphism has a left adjoint then it also has a right adjoint.
    \end{property}

    \newdef{Geometric morphism}{\index{morphism!geometric}\index{direct!image}\index{inverse!image}
        Let $\mathcal{E}, \mathcal{F}$ be (elementary) topoi. A geometric morphism $f:\mathcal{E}\rightarrow\mathcal{F}$ consists of an adjunction \[\mathcal{E}\adj{f^*}{f_*}\mathcal{F}\] where the left adjoint preserves finite limits, i.e. is left exact. The right adjoint $f_*$ is called the \textbf{direct image} part of $f$ and the left adjoint is called the \textbf{inverse image} part. If $f^*$ itself has a left adjoint, then $f$ is said to be \textbf{essential}.
    }

    \newdef{Geometric embedding}{\index{embedding}
        A geometric morphism for which the direct image part is fully faithful.
    }
    \begin{property}[Characterization of geometric embeddings]\label{topos:characterization_embedding}
        Let $f:\mathcal{E}\rightarrow\mathcal{F}$ be a geometric embedding and let $W\subset\text{hom}(\mathcal{E})$ be the collection of morphisms that are mapped to isomorphisms under $f^*$. $\mathcal{F}$ is equivalent to the full subcategory of $\mathcal{E}$ on $W$-local objects. It is also equivalent to the localization\footnote{See definition \ref{cat:localization}.} $\mathcal{E}[W^{-1}]$ at $W$.
    \end{property}

    \begin{property}[Base change]
        The base change functors on a topos are logical and admit a left adjoint (this is just the postcomposition functor). This implies that these functors can be refined to essential geometric morphisms.
    \end{property}

    \begin{example}[Topological spaces]\label{topos:topological_spaces}
        Every continuous map $f:X\rightarrow Y$ induces a geometric morphism
        \begin{gather}
            \mathbf{Sh}(X)\adj{f^*}{f_*}\mathbf{Sh}(Y)
        \end{gather}
        where the direct image functor $f_*$ is defined as follows:
        \begin{gather}
            f_*F(U) := F(f^{-1}U)
        \end{gather}
        for any sheaf $F\in\mathbf{Sh}(X)$ and any open subset $U\in\mathbf{Open}(Y)$. The inverse image functor $f^*$ is defined using the equivalence between sheaves on topological spaces and \'etal\'e bundles as noted above. Consider a sheaf $E\in\mathbf{Sh}(Y)$ as a bundle $\pi:E\rightarrow Y$. The inverse image of $E$ along a continuous function $f:X\rightarrow Y$ is just the pullback of $\pi$ and $f$.
    \end{example}

    By the previous example the global elements $\ast\rightarrow X$ of a topological space induce geometric morphisms of the form $\mathbf{Sh}(\ast)\rightarrow\mathbf{Sh}(X)$. By noting that $\mathbf{Sh}(\ast)=\mathbf{Set}$ we obtain the following generalization:
    \newdef{Point}{\index{point}
        A point of a topos $\mathcal{E}$ is a geometric morphism $\mathbf{Set}\rightarrow\mathcal{E}$.
    }

    \newnot{Category of topoi}{
        \nomenclature[S_Topos]{$\mathbf{Topos}$}{The 2-category of (elementary) topoi and geometric morphisms.}
        The category of elementary topoi and geometric morphisms is a 2-category. We will denote this category by $\mathbf{Topos}$.

        In fact to obtain the structure of a 2-category we need to define an appropriate notion of 2-morphism. Because a geometric morphism consists of an adjunction one can consider two distinct conventions, namely one can choose the 2-morphisms in $\mathbf{Topos}$ to be natural transformations $f^*\Rightarrow g^*$ (with associated transformations $g_*\Rightarrow f_*$) or one can choose them to be natural transformations $f_*\Rightarrow g_*$ (and associated transformations $g^*\Rightarrow f^*$). We will follow \cite{johnstone} and use the inverse image convention, i.e. a 2-morphism $f\Rightarrow g$ consists of natural transformations $f^*\Rightarrow g^*$ and $g_*\Rightarrow f_*$.
    }

\section{Grothendieck topos}\label{section:grothendieck_topos}

    \newdef{Sieve}{\index{sieve}
        Let $\mathbf{C}$ be a small category. A sieve $S$ on $\mathbf{C}$ is a fully faithfull discrete fibration $S\hookrightarrow\mathbf{C}$.

        A sieve $S$ on an object $c\in\mathbf{C}$ is a sieve in the slice category $\mathbf{C}/c$. This means that $S$ is a subset of $\text{ob}(\mathbf{C}/c)$ that is closed under precomposition, i.e. if $b\rightarrow c\in S$ and $a\rightarrow b\in\text{hom}(\mathbf{C})$ then the composition $a\rightarrow b\rightarrow c\in S$.

        All of this can be summarized by saying that a sieve on an object $c\in\ob{C}$ is a subfunctor of the hom-functor $\mathbf{C}(-, c)$.
    }

    \begin{example}[Maximal sieve]
        Let $\mathbf{C}$ be a category. The maximal sieve on $c\in\ob{C}$ is the collection of all morphisms $\{f\in\text{hom}(\mathbf{C}):\cod(f) = c\}$ or equivalently all of $\text{ob}(\mathbf{C}/c)$.
    \end{example}
    \begin{example}[Pullback sieve]
        Consider a morphism $f:a\rightarrow b$. Given a sieve $S$ on $b$ one can construct the pullback sieve $f^*S$ on $a$ as the sieve of morphisms in $S$ which factor through $f$:
        \begin{gather}
            f^*S(a) = \big\{(g:c\rightarrow a):f\circ g\in S(b)\big\}.
        \end{gather}
    \end{example}

    \newprop{Presheaf topos}{\index{presheaf!topos}\label{topoi:presheaf_topos}
        Consider the presheaf category $\mathbf{Psh}(\mathbf{C})$ for an arbitrary (small) category $\mathbf{C}$. This category is in fact an elementary topos where the subobject classifier is defined on each object in the following way:
        \begin{gather}
            \underline{\Omega}(c) := \{S: S\text{ is a sieve on }c\}.
        \end{gather}
        The action on a morphism $f:a\rightarrow b$ in $\mathbf{C}$ gives the morphism $\underline{\Omega}(f)$ which sends a sieve $S$ to its pullback sieve $f^*S$.

        The morphism $\texttt{true}:\underline{\mathbf{1}}\hookrightarrow\underline{\Omega}$ is defined as the natural transformation assigning to every object its maximal sieve. For every subobject $\underline{K}\hookrightarrow\underline{X}$ the characteristic morphism $\chi_K$ is defined as follows: Consider an object $c\in\ob{C}$ and element $x\in\underline{X}(c)$. The component $\chi_K|_c$ is then given by
        \begin{gather}
            \chi_K|_c(x) = \{f\in\mathbf{C}(d, c):\underline{X}(f)(x)\in\underline{K}(d)\}.
        \end{gather}
    }

    The following definition is due to Giraud (the original definition used the notion of a \textit{cover}, see at the end of this section):
    \newdef{Grothendieck topology}{\index{Grothendieck!topology}\index{covering!sieve}\index{cover}
        A Grothendieck topology on a category is a function $J$ assigning to every object a collection of sieves satisfying the following conditions:
        \begin{itemize}
            \item \textbf{Identity}\footnote{This condition can be rephrased in terms of isomorphisms: Sieves generated by an isomorphism are covering sieves. The name itself stems from the fact that the maximal sieve is generated from the identity morphism.}: For every object $c$ the maximal sieve $M_c$ is an element of $J(c)$.
            \item \textbf{Base change}: If $S\in J(c)$ then $f^*S\in J(d)$ for every morphism $f:d\rightarrow c$.
            \item \textbf{Locality}: Consider a sieve $S$ on $c$. If there exists a sieve $R\in J(c)$ such that for every morphism $(f:d\rightarrow c)\in R$ the pullback sieve $f^*S\in J(d)$, then $S\in J(c)$.
        \end{itemize}
        The sieves in $J$ are called ($J$-)\textbf{covering sieves}. A collection of morphisms with codomain $c\in\ob{C}$ is called a \textbf{cover}\footnote{Sometimes this term is also used to denote any collection of morphism with common codomain $c$, i.e. without reference to a covering sieve.} of $c$ if the sieve generated by thse morphisms is a covering sieve on $c$.
    }
    \begin{example}[Topological spaces]
        These conditions have the following interpretation in the case of topological coverings:
        \begin{itemize}
            \item The collection of all open subsets covers a space $U$.
            \item If $\{U_i\}_{i\in I}$ covers $U$ then $\{U_i\cap V\}_{i\in I}$ covers $U\cap V$.
            \item If $\{U_i\}_{i\in I}$ covers $U$ and if for every $i\in I$ the collection $\{U_{ij}\}_{j\in J_i}$ covers $U_i$ then $\{U_{ij}\}_{i\in I, j\in J_i}$ covers $U$.
        \end{itemize}

        The canonical Grothendieck topology on $\mathbf{Open}(X)$ is given by the sieves $S=\{U_i\hookrightarrow U\}_{i\in I}$ where $\bigcup_{i\in I}U_i = U$. This topology is denoted by $J_{\mathbf{Open}(X)}$.
    \end{example}

    \newdef{Site}{\index{site}
        A (small) category equipped with a Grothendieck topology $J$.
    }

    \newdef{Matching family}{\index{matching!family}\label{topoi:matching_family}
        Consider a presheaf $F\in\mathbf{Psh(C)}$ together with a sieve $S$ on $c\in\ob{C}$. A matching family for $S$ with respect to $F$ is a natural transformation $\alpha:S\Rightarrow F$ between $S$, regarded as a subfunctor of $\mathbf{C}(-, c)$, and $F$.

        More explicitly it is an assignment of an element $x_f\in Fd$ to every morphism $(f:d\rightarrow c)\in S$ such that
        \begin{gather}
            F(g)(x_f) = x_{f\circ g}
        \end{gather}
        for all morphisms $g:e\rightarrow d$. Equivalently, a matching family for $S$ with respect to $F$ is a set of elements $\{x_f\}_{f\in S}$ such that for all covering morphisms $f:d\rightarrow c, g:e\rightarrow c\in S$ and all morphisms $f':z\rightarrow d, g': z\rightarrow e$ such that $f\circ f'=g\circ g'$ the following equations holds:
        \begin{gather}
            \label{topoi:matching_family_condition}
            F(f')(x_f) = F(g')(x_g).
        \end{gather}

        Given such a matching family one calls an element $z\in Fc$ an \textbf{amalgamation} if it satisfies
        \begin{gather}
            F(f)(z) = x_f
        \end{gather}
        for all morphisms $f\in S(d)$. The existence of such an element can also be stated in terms of natural transformations: Consider the obvious inclusion $\iota_S$ of $S$ into the the hom-functor $\mathbf{C}(-,c)$. Every morphism with codomain $c$ can be obtained from the identity morphism by precomposition and hence a natural transformation $\mathbf{C}(-,c)\Rightarrow F$ is determined by its action on the identity morphisms $\mathbbm{1}_c$. The existence of an amalgamation is thus equivalent to the existence of an extension of $S$ along $\iota_S$.
    }
    \remark{If the base category has all pullbacks, for example if it is a topos on its own, then one can restrict the above commuting diagrams to the pullback diagrams of morphisms in the sieve $S$.}

    \newdef{Sheaf}{\index{sheaf}\index{presheaf!separated}\label{topoi:sheaf}
        \nomenclature[S_Shsite]{$\mathbf{Sh}(\mathbf{C}, J)$}{Category of $J$-sheaves on a site $(\mathbf{C}, J)$.}
        Consider a site $(\mathbf{C}, J)$. A presheaf $F$ on $\mathbf{C}$ is called a $J$-sheaf if every matching family for any covering sieve (on any object in $\mathbf{C}$) in $J$ admits a unique amalgamation\footnote{If there exists at most one amalgamation then the presheaf is said to be \textbf{separated}.} or equivalently if all sieves admit a unique extension to representable presheafs.

        The category $\mathbf{Sh}(\mathbf{C}, J)$ of $J$-sheaves on the site $(\mathbf{C}, J)$ is the full subcategory of $\widehat{\mathbf{C}}$ on the presheaves which satisfy the above condition.
    }
    We can also restate this definition in terms of local objects \ref{cat:local_object}:
    \newadef{Sheaf}{\index{descent}\label{topoi:local_object_sheaf}
        By definition every covering sieve admits a morphism into the Yoneda embedding: $\eta:S\hookrightarrow\mathcal{Y}c$. If we denote the collection of all these morphisms by $\mathcal{S}$, then a presheaf is a sheaf if and only if it is $\mathcal{S}$-local, i.e. if the following morphism is an isomorphism for all $\eta\in\mathcal{S}$:
        \begin{gather}
            Fc\cong\mathbf{Psh}(\mathcal{Y}c, F)\xrightarrow{\mathbf{Psh}(\eta, F)}\mathbf{Psh}(S, F).
        \end{gather}
        This is also called the \textbf{descent condition} of ordinary sheafs. In this context the collection of matching families $\text{Match}(S, P)$ for a sieve $S$ with respect to a presheaf $F$ is often called the \textbf{descent object} of $S$ with repsect to $P$.
    }

    \begin{example}[Topological spaces]
        The usual category of sheaves $\mathbf{Sh}(X)$ on a topological space $X$ is obtained as the category of sheaves on the site $(\mathbf{Open}(X), J_{\text{open}(X)})$. Since the morphisms in the covering sieves are exactly the inclusion maps $U_i\hookrightarrow U$, the pullback of two such morphisms is given by the intersection $U_i\cap U_j$. Hence the condition for a matching family, as formulated in equation \ref{topoi:matching_family} above, gives the second part of definition \ref{sheaf:def}. The uniqueness of an amalgamation is equivalent to the first part of that definition.
    \end{example}

    \begin{example}[Canonical topology]\index{topology!canonical}
        The canonical topology on a category is the largest Grothendieck topology for which all representable presheafs are sheafs. A subcanonical topology is then defined as a subtopology of the canonical one, i.e. any Grothendieck topology for which all representable presheafs are sheafs.
    \end{example}
    \begin{example}[Minimal and maximal topologies]
        The minimal Grothendieck topology on a category is the one for which only the maximal sieves are covering sieves. In this topology all presheafs are sheafs. The maximal Grothendieck topology is the one for which all sieves are covering sieves. In this topology only the terminal element of the associated presheaf category is a sheaf.
    \end{example}

    \newdef{Grothendieck topos}{\index{Grothendieck|seealso{topos}}\index{topos!Grothendieck}
        A category\footnote{The fact that a Grothendieck topos is an elementary topos follows from the fact that $\mathbf{Set}$ is a topos.} equivalent to the category of sheaves on a (small) site. This site is often called the \textbf{site of definition} for the given topos.
    }

    \begin{property}
        For every Grothendieck topos there exists a site of definition for which the Grothendieck topology is (sub)canonical.
    \end{property}

    \begin{construct}[Sheafification]\index{sheafification}
        Given a presheaf $\mathcal{F}$ we can construct a sheaf $\overline{\mathcal{F}}$ along the same lines of construction \ref{sheaf:colimit_construction}.
    \end{construct}

    \newdef{Global sections functor}{\index{global!sections}
        Every Grothendieck topos $\mathcal{E}$ admits a geometric morphism to $\mathbf{Set}$ where the left adjoint assigns to an object $X$ its set of global elements:
        \begin{gather}
            \Gamma:\mathcal{E}\rightarrow\mathbf{Set}:X\mapsto\mathcal{E}(1,X).
        \end{gather}
        When $\mathcal{E}$ is the sheaf topos over a topological space, this is exactly the global sections functor \ref{sheaf:global_sections_functor}. The left adjoint assigns to every set $S$ the copower $S\cdot1\equiv\bigsqcup_{s\in S}1$. When $\mathcal{E}$ is a sheaf topos, this adjoint is exactly the constant sheaf functor.
    }

    A different approach for defining sheaf topoi is through an embedding of sheafs into presheafs.
    \newdef{Local isomorphism}{\index{local!isomorphism}
        A system of local isomorphisms in $\mathbf{Psh}(\mathbf{C})$ is a class of morphisms in $\mathbf{Psh}(\mathbf{C})$ forming a system of weak equivalences \ref{cat:weak_equivalence} closed under pullbacks along morphisms out of representable presheafs.
    }
    \begin{property}[Local isomorphisms and Grothendieck topologies]
        A system of local isos induces a \textit{system of local epis} in the following way: $f:X\rightarrow Y$ is a local epi if $\im(f)\rightarrow Y$ is a local iso. A Grothendieck topology is defined by declaring a presheaf $F\in\mathbf{Psh}(\mathbf{C})$ to be a covering sieve at $X\in\ob{C}$ if $F\hookrightarrow\mathcal{Y}X$ is a local epi.
    \end{property}

    \newadef{Sheaf topos}{\index{topos}
        A category $\mathbf{Sh}(\mathbf{C})$ equipped with a geometric embedding into $\mathbf{Psh}(\mathbf{C})$.

        \begin{proof}[Proof of equivalence]
            By property \ref{topos:characterization_embedding} such a category is equivalent to the full subcategory on $S$-local presheaves for some system of local isomorphisms $S$ and therefore also to a sheaf topos in the sense of Grothendieck by the property above.
        \end{proof}
    }
    \begin{remark}[Descent condition]
        This is essentially a restatement of the descent condition \ref{topoi:local_object_sheaf}: Covering sieves, regarded as subfunctors, are in particular local isomorphisms. Stability of sieves under pullback together with the co-Yoneda lemma \ref{cat:ninja_yoneda}, which says that every presheaf is a colimit of representables, then generate the full collection of local isomorphisms.
    \end{remark}

    As a last point we also introduce the weaker notion of coverages:
    \newdef{Coverage}{\index{coverage}\label{topoi:coverage}
        Let $\mathbf{C}$ be a category. A coverage on $\mathbf{C}$ is a map assigning to every object $c\in\ob{C}$ a collection of families $\{f:d\rightarrow c\}\subset\text{hom}(\mathbf{C})$ satisfying the following condition: If $\{f:d\rightarrow c\}$ is a \textbf{covering family} on $c$, then for every morphism $g:c'\rightarrow c$ there exists a covering family $\{f':d'\rightarrow c'\}$ on $c'$ such that every composite $g\circ f'$ factors through some $f$.
    }
    It should be clear that every coverage generates a sieve (the smallest sieve containing the coverage). Furthermore, although coverages are weaker and easier to handle, they are in fact equivalent for the purpose of sheaf theory:
    \begin{property}
        Consider a covering family $C$ and let $S_C$ be the sieve it generates. A presheaf is a sheaf for $C$ if and only if it is a sheaf for $S_C$.
    \end{property}

\subsection{Topological sheaves}

    See chapter \ref{chapter:sheaf} for the application of sheaves to topology.

    \begin{property}[Presheaf topos]\label{topoi:sheaf_topos}
        Consider the presheaf category $\mathbf{Psh}(X) = \widehat{\mathbf{Open}(X)}$ over a topological space $X$. This category is an elementary topos where the subobject classifier $\Omega$ is defined as follows:
        \begin{gather}
            \Omega(U) := \{V:V\text{ is an open subset of }U\}.
        \end{gather}
    \end{property}
    \remark{In fact the presheaf category $\mathbf{Psh}(\mathbf{C})$ for any (small) category $\mathbf{C}$ is an elementary topos. See property \ref{topoi:presheaf_topos} below.}

    \begin{construct}[Sheaves and \'etal\'e bundles]\label{topos:etale_adjunction}
        Let $X$ be a topological space. The functor \[I:\mathbf{Open}(X)\rightarrow\mathbf{Top}/X:U\mapsto(U\hookrightarrow X)\] induces the following adjunction:
        \begin{gather}
            \mathbf{Top}/X\adj{E}{\Gamma}\mathbf{Psh}(X).
        \end{gather}
        The slice category on the right-hand side is equivalently the category of (topological) bundles\footnote{See chapter \ref{chapter:bundles}.} over $X$. Both directions of the adjunction have a clear interpretation. The right adjoint assigns to every bundle its sheaf of local sections. The left adjoint assigns to every presheaf its bundle of germs.

        By restricting to the subcategories on which this adjunction becomes an adjoint equivalence we obtain the \textbf{\'etal\'e space} and \textbf{sheaf} categories respectively:
        \begin{gather}
            \mathbf{Et}(X)\cong\mathbf{Sh}(X).
        \end{gather}
        The category on the right is the category of sheaves on a topological space $X$. The category on the left is the full subcategory on local homeomorphisms, i.e. \'etal\'e spaces as defined in chapter \ref{chapter:sheaf}.
    \end{construct}

    \begin{property}[Associated sheaf]
        The inclusion functor $\mathbf{Sh}(X)\hookrightarrow\mathbf{Psh}(X)$ admits a left adjoint. This is exactly the sheafification functor which assigns to every presheaf its associated sheaf. This functor is given by the composition $\Gamma\circ E$.\footnote{This amounts to construction \ref{sheaf:etale_construction}.}

        The fact that the counit of adjunction \ref{topos:etale_adjunction} restricts to an isomorphism on the full subcategory $\mathbf{Sh}(X)$ is equivalent to the fact that the sheafification of a sheaf $\Gamma$ is again $\Gamma$.
    \end{property}

    \newdef{Petit and gros topoi}{\index{topos!petit \& gros}
        Consider a topological space $X$ together with its category of opens $\mathbf{Op}(X)$. The petit topos over $X$ is defined as the usual sheaf topos $\mathbf{Sh}(X)$. It represents $X$ as some kind of generalized space. (By construction \ref{topos:etale_adjunction} the objects in a small topos are the \'etale spaces over a given base space.) However, we can also build a topos whose objects are generalized spaces. To this end we choose a site $S$ of ''probes'' and call the sheaf topos $\mathbf{Sh}(S)$ a gros topos. (See section \ref{section:smooth_spaces} for more information.)
    }

    \begin{property}[Localic reflection]\index{localic reflection}
        Mapping a topological space to its sheaf of continuous sections defines a functor $\func{\mathbf{Sh}}{Top}{Topos}$ by example \ref{topos:topological_spaces}. When restricted to the full subcategory of sober spaces \ref{topology:sober_space} this functor becomes fully faithful. When generalizing to sober locales, we even obtain a reflective inclusion \ref{cat:reflective_inclusion}.

        This property states that we lose no information when regarding (sober) topological spaces as sheaf topoi. This also explains the name ''petit topos''.
    \end{property}

\subsection{Lawvere-Tierney topology}

    \newdef{Lawvere-Tierney topology}{\index{Lawvere-Tierney}
        As noted in section \ref{cat:internal_logic} on the internal logic of elementary topoi, the subobject classifier $\Omega$ has the structure of an internal Heyting algebra and in particular that of a meet-semilattice (where the meet is given by the pullback of morphisms). This internal poset, viewed as an internal category, admits the construction of a closure operator\footnote{See definition \ref{cat:closure_operator}.} $j:\Omega\rightarrow\Omega$ satisfying the following condition:
        \begin{gather}
            j\circ\land = \land\circ(j\times j).
        \end{gather}
        This condition states\footnote{This is not a trivial statement.} that $j$ is (internally) order-preserving.
    }
    \begin{remark}
        The condition satisfied by the unit morphism in the definition of a closure operator can also be reformulated as follows in this context:
        \begin{gather}
            j\circ\texttt{true} = \texttt{true}.
        \end{gather}
    \end{remark}
    The Lawvere-Tierney operator also induces a ''closure operator'' on all posets Sub$(X)$ in the topos. Given an object $X$ and a subobject $U\in\text{Sub}(X)$ one defines the closure $j_\ast(U)\in\text{Sub}(X)$ as the subobject classified by the characteristic map $j\circ\chi_U:X\rightarrow\Omega$.

    \newdef{Dense object}{\index{dense}
        Given a Lawvere-Tierney topology $j:\Omega\rightarrow\Omega$, a subobject $U\in\text{Sub}(X)$ is said to be dense (in $X$) if it satisfies $j_\ast(U) = X$.
    }
    \newadef{Sheaf}{\index{sheaf}
        Given a Lawvere-Tierney topology $j:\Omega\rightarrow\Omega$ on a topos $\mathcal{E}$, one calls an object $S\in\ob{E}$ a $j$-sheaf if for all dense morphisms $U\hookrightarrow X$ the induced map \[\mathcal{E}(X, S)\rightarrow\mathcal{E}(U, S)\] is a bijection.
    }

    \begin{property}
        For the presheaf topos on a small category $\mathbf{C}$, the Grothendieck topologies on $\mathbf{C}$ and Lawvere-Tierney topologies on $\mathbf{Psh}(\mathbf{C})$ are equivalent.
        \begin{proof}[Sketch of proof]
            Since a Grothendieck topology assigns to every object a collection of sieves, we find by property \ref{topoi:presheaf_topos} that $J(c)\subseteq\Omega_{\mathbf{Psh}}(c)$ for all $c\in\ob{C}$. By the base change condition of Grothendieck topologies this relation is natural in $c$ and hence $J$ is a subobject of $\Omega_{\mathbf{Psh}}$. We thus find a characteristic morphism $j:\Omega_{\mathbf{Psh}}\rightarrow\Omega_{\mathbf{Psh}}$ which can be proven (by the other conditions of Grothendieck topologies) to define a Lawvere-Tierney topology on $\mathbf{Psh}(\mathbf{C})$. Conversely, a Lawvere-Tierney topology is a morphism $j:\Omega\rightarrow\Omega$ and hence determines a unique subobject of $\Omega_{\mathbf{Psh}}$, i.e. a unique collection of sieves for every object $c\in\ob{C}$. From the conditions on Lawvere-Tierney topologies one can then prove that this collection satisfies the conditions of a Grothendieck topology.
        \end{proof}
    \end{property}
    \sremark{We can conclude that Lawvere-Tierney topologies generalize Grothendieck topologies from presheaf topoi to general topoi.}

\section{Stacks}\index{stack}

    An important subject, especially in the context of gauge theories in physics, is that of groupoid-valued (pre)sheafs. We first generalize sites to 2-categories:
    \newdef{2-coverage}{\index{coverage}\index{site}
        Virtually the same as an ordinary coverage \ref{topoi:coverage}. However, instead of exact factorization, we only require factorization up to an isomorphism. A 2-category equipped with a 2-coverage is called a \textbf{2-site}.
    }
    As for 1-sites, every coverage generates a unique sieve. It is the full subcategory on those morphisms that factor through a covering map in the given coverage (again up to isomorphism).

    \newdef{2-presheaf}{\index{presheaf}\index{prestack}
        Consider a 2-category $\mathbf{C}$. A 2-presheaf on $\mathbf{C}$ is a pseudofunctor $\cfunc{F}{C}{Cat}$. When $\mathbf{C}$ is the categorification of a 1-category, i.e. it has discrete Hom-categories, we often speak of \textbf{prestacks}.
    }
    As in the case of ordinary categories (see definition \ref{topoi:local_object_sheaf})we can define 2-sheafs through a descent condition:
    \newdef{2-sheaf}{\index{sheaf}\label{topoi:2_sheaf}
        A 2-presheaf $\cfunc{F}{C}{Cat}$ on a 2-site $(\mathbf{C},J)$ is said to be a 2-sheaf with respect to $J$ if for all sieves $S\in J$ the following functor is an equivalence:
        \begin{gather}
            Fc\cong\mathbf{Psh}_2(\mathcal{Y}c, F)\rightarrow\mathbf{Psh}_2(S, F)
        \end{gather}
        where the fist equivalence is just the 2-Yoneda lemma.
    }
    \begin{remark}
        It should be noted that 2-(pre)sheafs can also be defined on ordinary (1-)sites. Sieves, regarded as subfunctors of the Yoneda embedding, take values in $\mathbf{Set}$. By composing these with the embedding $\mathbf{Set}\hookrightarrow\mathbf{Cat}$ of sets as (discrete) categories we obtain 2-presheafs (in fact 2-subfunctors of the 2-Yoneda embedding). Often 2-sheafs over 1-sites are called \textbf{stacks} (although this terminology is also used for general 2-sites).
    \end{remark}

    \newdef{Prestack of groupoids}{
        Consider a category $\mathbf{C}$. A prestack of groupoids is a $\mathbf{Grpd}$-valued prestack on $\mathbf{C}$.

        The category of (groupoid-valued) prestacks becomes $\mathbf{Grpd}$-enriched if we take the Hom-object between two prestacks $F,G$ to consist of the following data:
        \begin{itemize}
            \item The objects are natural transformations $\alpha:F\Rightarrow G$ (note that the components are themselves functors).
            \item The morphisms $\mathfrak{m}$ are ''strict modifications'' (see also definition \ref{cat:modification}) in the sense that they map objects in $\mathbf{C}$ to natural transformations satisfying the whiskering condition
            \begin{gather}
                \mathbbm{1}_{Ff}\cdot\mathfrak{m}_b = \mathfrak{m}_a\cdot\mathbbm{1}_{Gf}.
            \end{gather}
        \end{itemize}
    }

    For ordinary sites and presheafs we defined descent in terms of matching families. Since we are now taking values in a 2-category, the matching families are a bit more complex. However, this structure is already familiar in differential geometry and algebraic topology where it is known under the name of the \textit{\v{C}ech nerve}:
    \newdef{\v{C}ech groupoid}{\index{Cech!groupoid}
        Consider a site $(\mathbf{C},J)$. To every covering family $\mathcal{U}:=\{f_i:c_i\rightarrow c\}$ we assign an internal groupoid in presheafs $C(\mathcal{U})$ consisting of the following data:
        \begin{itemize}
            \item $\text{ob}(C(\mathcal{U})) := \bigsqcup_i\mathcal{Y}c_i$, and
            \item $\text{hom}(C(\mathcal{U})) := \bigsqcup_{i,j}\mathcal{Y}c_i\times_{\mathcal{Y}c}\mathcal{Y}c_j$.
        \end{itemize}
        This is equivalent to the ($\mathbf{Grpd}$-valued) presheaf that assigns to every object $d\in\ob{C}$ the groupoid consisting of the following data: Its objects are pairs $(i, g_i:d\rightarrow c_i)$ such that $c_i\in\mathcal{U}$ and Hom-objects between two such pairs consist of a unique arrow if and only if
        \begin{gather}
            f_i\circ g_i = f_j\circ g_j.
        \end{gather}
    }
    If we compare this last equation with the condition for matching families in definition \ref{topoi:matching_family} we could presume that the \v{C}ech groupoid is related to matching families. This intuition is indeed correct:
    \begin{property}[Matching families]\label{topoi:cech_matching_families}
        Any ordinary presheaf $F$ can be considered to be $\mathbf{Grpd}$-valued by the embedding $\mathbf{Set}\hookrightarrow\mathbf{Grpd}$. For any covering family $\mathcal{U}$ there exists an isomorphism
        \begin{gather}
            \cfunccat{C}{Grpd}(C(\mathcal{U}), F)\cong\text{Match}(\mathcal{U}, F).
        \end{gather}
        Note that the left-hand side is an ordinary set since $F$ is $\mathbf{Set}$-valued. Because the \v{C}ech groupoid (co)represents a descent object it is sometimes called a \textbf{codescent object}.
    \end{property}
    It is exactly this (co)descent property of the \v{C}ech groupoid that will be used in chapter \ref{chapter:hdg} to define (higher) smooth groupoids.

    People with some experience in algebraic topology will also notice that the \v{C}ech groupoid only contains the first degrees of the \v{C}ech complex. The full \v{C}ech complex can be obtained from the following construction:
    \newdef{\v{C}ech nerve}{\index{Cech!nerve}
        Consider a morphism $f:d\rightarrow c$ in a category $\mathbf{C}$. The \v{C}ech nerve $C_\bullet(U)$ is the \textit{simplicial object} (see definition \ref{model:simplicial_object}) that is defined as the $(k+1)$-fold pullback of $f$ with itself in degree $k$. For a covering family $\mathcal{U}:=\{f_i:c_i\rightarrow c\}$ we define its \v{C}ech nerve as $C_\bullet(\mathcal{U}):=C_\bullet(\bigsqcup_ic_i\rightarrow c)$.
    }
    For $\infty$-sheafs we will use the full \v{C}ech nerve, however for 2-sheafs and in particular stacks, we will only use its 3-coskeleton. This extra information will encode the \textit{cocycle condition} well-known for example in the study of fibre bundles (see equation \ref{diff:G_cocycle_condition}).

\subsection{Stacks on a 1-site}

    In this section we will use the notion of fibred categories and their equivalence to $\mathbf{Cat}$-valued pseudofunctors (see section \ref{section:fibred_categories}).

    \newdef{Descent datum}{\index{descent}
        Consider a category $\mathbf{C}$ with a covering family $\mathcal{U}:=\{f_i:c_i\rightarrow c\}$ and a fibred $\mathbf{C}$-category $F$. The projections associated to the pullback $c_i\cap c_j:=c_i\times_cc_j$ will be denoted by $\pi_{1,2}$ (and analogously for iterated pullbacks) and their (Catesian) pullback functors (induced by the cleavage on $\mathbf{C}$) will be denoted by $\pi^*$. A descent datum for $\mathcal{U}$ with respect to $F$ is a pair $(\{x_i\},\{f_{ij}\})$ where $\{x_i\}$ is a matching family for $\mathcal{U}$ with respect to $F$ and every $f_{ij}$ is an isomorphism $\pi_1^*x_i\cong \pi_2^*x_j$. This data is required to satisify the following \textbf{cocycle condition}:
        \begin{gather}
            \pi_{13}^*f_{ik} = \pi_{12}^*f_{ij}\circ\pi_{23}^*f_{jk}.
        \end{gather}
        Morphisms $(\{x_i\},\{f_{ij}\})\rightarrow(\{y_i\},\{g_{ij}\})$ between descent data are families of morphisms $\{\phi_i:x_i\rightarrow y_i\}$ that satisfy
        \begin{gather}
            \pi_1^*\phi_i\circ f_{ij} = g_{ij}\circ\pi_2^*\phi_j.
        \end{gather}
        We will denote the category of descent data for $\mathcal{U}$ with respect to $F$ by $\text{Descent}(\mathcal{U}, F)$
    }
    \begin{construct}
        Consider an object $\xi$ in $Fc$. From this object we construct a descent datum as follows: The objects $x_i$ are the pullbacks $f_i^*\xi$ and the isomorphisms $f_{ij}:\pi_2^*f_i^*\xi\cong\pi_1^*f_j^*\xi$ are obtained from the fact that both these objects are (Cartesian) pullbacks of the same morphisms. Arrows in $Fc$ induce morphisms of descent data by (Cartesian) pullbacks along the covering maps.

        This construction defines a functor $Fc\rightarrow\text{Descent}(\mathcal{U}, F)$. (It can be shown that all of this is independent of a choice of cleavage up to equivalence.)
    \end{construct}

    \newdef{Stack}{\index{prestack}\index{stack}
        Consider a fibred category $F$ over a site $(\mathbf{C},J)$.
        \begin{itemize}
            \item $F$ is called a \textbf{separated prestack} if for each covering family $\mathcal{U}$ on $c\in\ob{C}$ the functor $Fc\rightarrow\text{Descent}(\mathcal{U}, F)$ is fully faithful.
            \item $F$ is called a \textbf{stack} if for each covering family $\mathcal{U}$ on $c\in\ob{C}$ the functor $Fc\rightarrow\text{Descent}(\mathcal{U}, F)$ is an equivalence.
        \end{itemize}
        This becomes a generalization of the descent condition \ref{topoi:local_object_sheaf} after observing that $\text{Descent}(\mathcal{U}, F)\cong\hom_{\mathbf{C}}(S(\mathcal{U}), F)$ where $S(\mathcal{U})$ is the sieve generated by $\mathcal{U}$, regarded as a fibred category.
    }

    A more conceptual (although completely equivalent) generalization from (1-)sheafs to 2-sheafs can be obtained by starting from property \ref{topoi:cech_matching_families}. There it was shown that matching families for (1-)presheafs can be obtained as natural transformations from the \v{C}ech groupoid.
    \begin{property}[Descent data and \v{C}ech nerve]
        Here $C(\mathcal{U})$ will denote the 3-coskeleton of the \v{C}ech nerve $C_\bullet(\mathcal{U})$. Pseudonatural transformations $C(\mathcal{U})\rightarrow F$ can be shown to be equivalent to tuples $(x, \{x_i\}, \{x_{ij}\}, \{x_{ijk}\})$ where $x_i\in Fc_i$ that fit into cubes lying in the image of $C_2(\mathcal{U})$ in which all edges consist of Cartesian morphisms. Arrows between such cubes are given by arrows between the vertices that make the ''obvious'' diagrams commute.

        By comparing these cubes to the above definition of descent data we obtain the following equivalence:
        \begin{gather}
            \text{Descent}(\mathcal{U}, F)\cong\cfunccat{C}{Cat}(C(\mathcal{U}), F).
        \end{gather}

        ?? FINISH THIS ??
    \end{property}

    \begin{remark}[1-sheafs]
        Although most of the above looks very abstract and complex compared to ordinary sheafs, it is not really so. In fact, if we restrict to pseudofunctors of the form $\mathbf{C}^{op}\rightarrow\mathbf{Set}$, where we use the embedding $\mathbf{Set}\hookrightarrow\mathbf{Cat}$ to view sets as discrete categories, we obtain ordinary sheafs as a subcategory of stacks. For example: By the equivalence between pseudofunctors and Grothendieck fibrations, we know that the Cartesian pullbacks $f^*$ are in fact just the images of morphism $f$ under the pseudofunctor $F$. This way the condition $\pi_1^*x_i\cong\pi^*_2x_j$ can be rewritten as $Ff'_i(x_i)=Ff'_j(x_j)$, which is nothing but the matching family condition \ref{topoi:matching_family_condition}.
    \end{remark}

\section{Higher topoi}

    In this section we generalize the notion of topos from ordinary category theory to higher category theory. In particular we will consider $\infty$-sheafs. This will requires us to use a suitable foundation for $\infty$-category theory. For this we will use the language of (simplicial) model categories as introduced in chapter \ref{chapter:model_theory}.

    \newdef{\texorpdfstring{$\infty$-groupoid}{Infinity-groupoid}}{\index{groupoid}
        Objects of the full simplicial subcategory of $\mathbf{sSet}_{Quillen}$ on Kan complexes. From property \ref{model:horn_filler} we immediately see how this generalizes the definition of ordinary groupoids: For groupoids we need unique horn fillers (composition in ordinary categories is unique), while for $\infty$-groupoids we allow this up to higher coherence.
    }
    \newdef{\texorpdfstring{$(\infty,1)$-category}{(Infinity,1)-category}}{\index{category}
        An $\infty\mathbf{Grpd}$-enriched category, or equivalently, a simplicially enriched category for which all hom-objects are Kan complexes. The functor category between $(\infty,1)$-categories is defined through the (simplicial) nerve and realization functors \ref{model:nerve}:
        \begin{gather}
            \funccat{C}{D} := |\mathbf{sSet}(N\mathbf{C}, N\mathbf{D})|.
        \end{gather}
    }

    The most straightforward definition of an $\infty$-sheaf generalizes definition \ref{topoi:local_object_sheaf}:
    \newdef{\texorpdfstring{$\infty$-sheaf}{Infinity-sheaf}}{\index{sheaf}
        Consider an $\infty$-site $(\mathbf{C},J)$ and let $S$ denote the collection of monomorphism in $\mathbf{Psh}_\infty(\mathbf{C})$ induced by the covering sieves. An $\infty$-presheaf on $\mathbf{C}$ is called a $J$-sheaf if it is $S$-local. A presheaf with values in an $\infty$-category $\mathbf{D}$ is called a sheaf if the representable $\mathbf{D}(d, F-)$ is a $J$-sheaf for all $d\in\ob{D}$.
    }
    \newdef{\texorpdfstring{$\infty$-stack}{Infinity-stack}}{\index{stack}
        An $(\infty,1)$-sheaf taking values in $\infty\mathbf{Grpd}$.
    }

    ?? PERHAPS MOVE infinity-CATEGORY STUFF TO CHAPTER ''MODEL THEORY'' ??

\section{Cohesion}

    In this section we will often talk about (Grothendieck) topoi \textbf{over} some base topos $\mathcal{S}$, i.e. topoi equipped with a geometric morphism to $\mathcal{S}$.

    \newdef{Local topos}{\index{local!topos}
        Consider a topos $\mathcal{E}$ over a base topos $\mathcal{S}$. $\mathcal{E}$ is said to be ($\mathcal{S}$-)local if the geometric morphism $(f^*\dashv f_*):\mathcal{E}\leftrightarrows\mathcal{S}$ admits a right adjoint $f^!$ such that one of the following equivalent statements:
        \begin{itemize}
            \item $f^!$ is fully faithful.
            \item $f^*$ is fully faithful.
            \item $f^!$ is an $\mathcal{S}$-indexed functor \ref{cat:indexed_category}.
            \item $f^!$ is Cartesian closed \ref{cat:cartesian_closed_functor}.
        \end{itemize}
        If we take $\mathcal{S}=\mathbf{Set}$, the conditions are automatically satisfied since all functors are $\mathbf{Set}$-indexed.

        The left and right adjoints are sometimes also called the \textbf{discrete} and \textbf{codiscrete object functors} $\text{Disc}$ and $\text{coDisc}$ (in fact this terminology is applied more generally when $\mathcal{E}$ is just a category). If these functors exist, we say $\mathcal{E}$ has \textbf{(co)discrete objects}. This terminology derives from the case of the forgetful functor $\func{\Gamma}{Top}{Set}$ where the (fully faithful) left and right adjoints equip a set with the (co)discrete topology.
    }
    \begin{property}
        A topos is local if and only if $1$ is tiny \ref{cat:tiny}.
    \end{property}

    \newdef{Locally-connected topos}{\index{locally-connected}
        An object in a category is said to be \textbf{connected} if its representable functor preserves finite coproducts. A topos is said to be \textbf{locally connected} if all objects can be written as coproducts of connected objects. This defines a functor
        \begin{gather}
            \Pi_0:\mathcal{E}\rightarrow\mathbf{Set}:\bigsqcup_{i\in I}X_i\mapsto I
        \end{gather}
        left adjoint to the constant sheaf functor (which is itself left adjoint to the global section functor). This functor is suitably called the \textbf{connected component functor}.
    }
    \begin{property}[Locally connected geometric morphisms]
        A topos is locally connected if and only if its global section geometric morphism is essential (and the left adjoint is an indexed functor, but this is again automatic over $\mathcal{S}=\mathbf{Set}$). More generally we say a topos over some base topos $\mathcal{S}$ is \textbf{locally connected} if its associated geometric morphism is essential and the left adjoint is $\mathcal{S}$-indexed.
    \end{property}

    \newdef{Connected topos}{\index{connected}
        A topos over a base topos is said to be \textbf{connected} if the inverse image part of the associated geometric morphism is fully faithful. For sheaf topoi over a topological space $X$ this is exactly the requirement for $X$ being connected.

        For locally connected topoi this amounts to the property that the left adjoint in its adjoint triple preserves the terminal object. Furthermore, a locally connected topos is said to be \textbf{strongly connected} if the left adjoint in its adjoint triple preserves finite products (in particular turning it into a connected topos).
    }
    \begin{property}
        Every local topos is connected.
    \end{property}

    \newdef{Cohesive topos}{\index{topos!cohesive}
        A local strongly connected topos. This implies the existence of an adjoint quadruple $(\Pi_0,\text{Disc},\Gamma,\text{coDisc})$ where both $\text{Disc}$ and $\text{coDisc}$ are fully faithful.
    }

    \begin{property}[Cohesive modalities]
        The adjoint quadruple on a cohesive topos induces an adjoint triple of modalities, i.e. idempotent (co)monads (see section \ref{section:modal_type_theory} for a formal introduction):
        \begin{gather}
            (\smallint\dashv\flat\dashv\sharp):=(\text{Disc}\circ\Pi_0\dashv\text{Disc}\circ\Gamma\dashv\text{coDisc}\circ\Gamma).
        \end{gather}
        These are respectively called the \textbf{shape}, \textbf{flat} and \textbf{sharp} modalities.
    \end{property}

    ?? COMPLETE (e.g. work by Schreiber) ??
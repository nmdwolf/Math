\chapter{General Topology}\label{chapter:topology}
\section{Topological spaces}

    \newdef{Topology}{\index{topology}
        \nomenclature[S_Top]{$\mathbf{Top}$}{Category of topological spaces.}
        Let $\Omega$ be a set and consider a collection of subsets $\tau\subseteq 2^\Omega$. The set $\tau$ is a topology on $\Omega$ if it satisfies following axioms:
        \begin{enumerate}
            \item $\emptyset\in\tau$ and $\Omega\in\tau$
            \item $\forall\ \mathcal{F}\subseteq\tau: \bigcup_{V\in\mathcal{F}}V \in \tau$
            \item $\forall\ U, V\in\tau: U\cap V\in\tau$.
        \end{enumerate}
        Furthermore, we call the elements of $\tau$ \textbf{open sets} and the couple $(\Omega, \tau)$ a \textbf{topological space}. The \textbf{closed sets} are defined as the sets which have an open complement.
    }

    \begin{property}[Category of opens $\clubsuit$]
        \nomenclature[S_Open]{$\mathbf{Open}(X)$}{Category of open subsets of a topological space $X$.}
        Consider a topological space $(X, \tau)$. Let $U\subseteq V\in\tau$. The topology (as a set) $\tau$ together with the collection of inclusion maps $U\hookrightarrow V$ forms a (small) category $\mathbf{Open}(X)$.
    \end{property}

    \newdef{Pointed topological space}{\label{topology:pointed_space}\index{pointed!topological space}
        Let $x_0\in X$. The triple $(X, \tau, x_0)$ is called a pointed topological space with base point $x_0$.
    }

    \begin{example}[Relative topology\footnotemark]\label{topology:relative_topology}
        \footnotetext{Sometimes called the \textbf{subspace topology}.}
        Let $(X, \tau_X)$ be a topological space and $Y$ a subset of $X$. We can turn $Y$ into a topological space by equipping it with the following topology:
        \begin{gather}
            \tau_\text{rel} := \big\{U_i\cap Y:U_i\in \tau_X\big\}.
        \end{gather}
    \end{example}
    \begin{example}[Discrete topology]\index{discrete!topology}
        The discrete topology is the topology such that every subset is open (and thus also closed).
    \end{example}

    \newdef{Interior}{\index{interior}
        \nomenclature[O_Int]{$X^\circ$}{The interior of a set $X$.}
        Consider a subset $X$ of a topological space $\Omega$. The interior $X^\circ$ is defined as the union of all open subsets of $X$. Elements of the interior are called \textbf{interior points} of $X$.
    }
    \newdef{Closure}{\index{closure}
        \nomenclature[O_Clos]{$\overline{X}$}{The closure of a set $X$.}
        Consider a subset $X$ of a topological space $\Omega$. The closure $\overline{X}$ of $X$ is defined as the intersection of all closed sets containing $X$.
    }
    \newdef{Boundary}{\index{boundary}
        \nomenclature[O_Bound]{$\partial X$}{The boundary of a set $X$.}
        Consider a subset $X$ of a topological space $\Omega$. The boundary $\partial X$ of $X$ is defined as $\overline{X}\backslash X^\circ$.
    }

    \newdef{Borel set}{\index{Borel!set}\label{topology:borel_set}
        Let $\mathcal{B}$ be the $\sigma$-algebra\footnote{See definition \ref{set:sigma_algebra}.} generated by all open\footnote{For $X=\mathbb{R}$ we find that open, closed and half-open (both types) intervals generate the same $\sigma$-algebra.} sets $O\subset X$. The elements $B\in\mathcal{B}$ are called Borel sets.
    }

    \newdef{Topological group}{\index{group!topological}
        A topological group is a group $G$ equipped with a topology such that both the multiplication and inversion map are continuous.
    }

\subsection{Neighbourhoods}

    \newdef{Neighbourhood}{\index{neighbourhood}
        A set $V\subseteq\Omega$ is a neighbourhood of a point $a\in\Omega$ if there exists an open set $U$ such that $a\in U\subseteq V$.
    }

    Although the following two notions are often treated as synonyms in the literature we have chosen to give them separate meanings:
    \newdef{Limit point}{\index{limit!point}
        Let $S$ be a subset of $X$. A point $x\in X$ is called a limit point of $S$ if every neighbourhood of $x$ contains at least one point of $S$ different from $x$.
    }
    \newdef{Accumulation point\footnotemark}{\index{accumulation point|seealso{limit point}}
        \footnotetext{Sometimes called a \textbf{cluster point}.}
        Let $x\in X$ be a limit point of $S$. Then $x$ is an accumulation point of $S$ if every open neighbourhood of $x$ contains infinitely many points of $S$.
    }
    A subtly different notion is the following one:
    \newdef{Adherent point}{\index{adherent point}
        Let $S$ be a subset of $X$. A point $x\in X$ is called an adherent point of $S$ if every neighbourhood of $x$ contains at least one point of $S$. Hence every limit point of $S$ is an adherent point. A point $x$ is an adherent point of $S$ if and only if it is an element of the closure $\overline{S}$.
    }

    \newdef{Basis}{\index{basis}
        Let $\mathcal{B}\subseteq\tau$ be a family of open sets. The family $\mathcal{B}$ is a basis for the topological space $(\Omega, \tau)$ if every $U\in\tau$ can be written as
        \begin{gather}
            U = \bigcup_{V\in\mathcal{F}}V
        \end{gather}
        where $\mathcal{F}\subseteq\mathcal{B}$.
    }
    \newdef{Local basis}{
        Let $\mathcal{B}_x$ be a family of open neighbourhoods of a point $x\in\Omega$. $\mathcal{B}_x$ is a local basis of $x$ if every neighbourhood of $x$ contains at least one element in $\mathcal{B}_x$.
    }

    \newdef{First-countability}{\index{countable!countable space}
        A topological space $(\Omega, \tau)$ is first-countable if for every point $x\in\Omega$ there exists a countable local basis.
    }
    \begin{property}[Decreasing basis]
        Let $x\in\Omega$. If there exists a countable local basis for $x$ then there also exists a countable decreasing local basis for $x$.
    \end{property}

    \newdef{Second-countability}{
        A topological space $(\Omega, \tau)$ is second-countable if there exists a countable (global) basis.
    }

    \begin{property}
        Let $X$ be a topological space. The closure of a subset $V$ is given by:
        \begin{gather}
            \label{topology:closure}
            \overline{V} = \{x\in X:\exists \text{ a net } (x_\lambda)_{\lambda\in I} \text{ in } X:x_\lambda\rightarrow x\}.
        \end{gather}
        This implies that the topology on $X$ is completely determined by the convergence of nets\footnote{See definition \ref{set:net}.}.
    \end{property}
    \begin{result}
        In first-countable spaces we only have to consider the convergence of sequences.
    \end{result}

    \newdef{Germ}{\index{germ}\label{topology:germ}
        Let $X$ be a topological space and let $Y$ be a set. Consider two functions $f, g: X\rightarrow Y$. If there exists a neighbourhood $U$ of a point $x\in X$ such that \[f(u) = g(u)\qquad\qquad\forall u\in U\] then this property defines an equivalence relation denoted by $f\sim_x g$ and the equivalence classes are called germs.
    }

    \begin{property}
        Let the set $Y$ in the previous definition be the set of reals $\mathbb{R}$. Then the germs at a point $p\in X$ satisfy the following closure/linearity relations:
        \begin{itemize}
            \item $[f] + [g] = [f+g]$
            \item $\lambda[f] = [\lambda f]$
            \item $[f][g] = [fg]$
        \end{itemize}
        where $[f], [g]$ are two germs at $p$ and $\lambda\in\mathbb{R}$ is a scalar.
    \end{property}

\subsection{Separation axioms}\index{separation axioms}

    \newdef{Irreducible}{\index{irreducible!space}
        A topological space is said to be irreducible if it is not the union of two proper closed subsets or, equivalently, if the intersection of two nonempty open subsets is again nonempty.
    }

    \newdef{$T_0$-space\footnotemark}{\index{Kolmogorov!topology}
        \footnotetext{$T_0$ spaces are also said to carry the \textbf{Kolmogorov topology}.}
        A topological space is $T_0$ if for every two distinct points $x, y$ at least one of them has a neighbourhood not containing the other. The points are said to be topologically distinguishable.
    }

    \newdef{$T_1$-space\footnotemark}{\index{Fr\'echet!topology}
        \footnotetext{$T_1$ spaces are also said to carry the \textbf{Fr\'echet topology} (not to be confused with Fr\'echet spaces in functional analysis).}
        A topological space is $T_1$ if for every two distinct points $x,y$ there exists neighbourhood $U, V$ of $x$ and $y$ respectively such that $y\not\in U$ and $x\not\in V$. The points are said to be separated.
    }

    \newdef{Hausdorff space}{\index{Hausdorff!space}\label{topology:hausdorff}
        A topological space $X$ is a Hausdorff space or $T_2$-space if it satisfies the following condition:
        \begin{gather}
            (\forall x, y \in X)(\exists \text{ neighbourhoods }U, V)(x\in U, y\in V, U\cap V=\emptyset).
        \end{gather}
        The points are said to be \textbf{separated by neighbourhoods}. It can be shown that this definition is equivalent to requiring that the diagonal $\Delta_X$ is closed in the product space $X\times X$.
    }
    \begin{property}
        Every singleton (and thus also every finite set) is closed in a Hausdorff space.
    \end{property}

    \newdef{Urysohn space\footnotemark}{\index{Urysohn!space}
        \footnotetext{Sometimes called a $T_{2\nicefrac{1}{2}}$-space.}
        A topological space is an Urysohn space if every two distinct points are separated by closed neighbourhoods.
    }

    \newdef{Regular space}{\index{regular}\label{topology:regular}
        A topological space is said to be regular if for every closed subset $F$ and every point $x\not\in F$ there exist disjoint open subsets $U, V$ such that $x\in U$ and $F\subset V$.
    }
    \newdef{$T_3$-space}{
        A space that is both regular and $T_0$ is $T_3$.
    }

    \newdef{Normal space}{\index{normal}\label{topology:normal}
        A topological space is said to be normal if every two closed subsets have disjoint neighbourhoods.
    }
    \newdef{$T_4$-space}{
        A space that is both normal and $T_1$ is $T_4$.
    }

    \begin{property}
        A space satisfying the separation axiom $T_k$ also satisfies all separation axioms $T_{i\leq k}$.
    \end{property}

\section{Morphisms}
\subsection{Convergence}

    \newdef{Convergence}{\index{convergence}
        A sequence $(x_n)_{n\in\mathbb{N}}$ in $X$ is said to converge to a point $a\in X$ if
        \begin{gather}
            (\forall \text{ neighbourhoods } U \text{ of } a)(\exists N > 0)(\forall n>N)(x_n\in U).
        \end{gather}
    }

    \begin{property}
        Every subsequence of a converging sequence converges to the same point\footnote{This limit does not have to be unique. See the next property for more information.}.
    \end{property}
    \begin{property}\label{topology:theorem:hausdorff_limit}
        Let $X$ be a Hausdorff space. The limit of a converging sequence in $X$ is unique.
    \end{property}

\subsection{Continuity}

    \newdef{Continuity}{\index{continuity}
        \nomenclature[S_Cont]{$C(X, Y)$}{Set of all continuous functions between two topological spaces $X, Y$.}
        A function $f:X\rightarrow Y$ is continuous if the inverse image $f^{-1}(U)$ of every open set $U$ is also open. The set of all continuous functions between two topological spaces $X, Y$ is often denoted by $C(X, Y)$.
    }

    \newdef{Initial topology}{\index{topology!initial}\label{topology:initial_topology}
        Consider a family of maps $\{f_i:X\rightarrow Y_i\}_{i\in I}$ between topological spaces. The initial topology on $X$ with respect to these maps is the coarsest topology on $X$ for which all maps $f_i$ are continuous.
    }
    \newdef{Final topology}{\index{topology!final}
        Consider a family of maps $\{f_i:Y_i\rightarrow X\}_{i\in I}$ between topological spaces. The final topology on $X$ with respect to these maps is the finest topology on $X$ for which all maps $f_i$ are continuous.
    }

    \begin{property}[Continuity]
        Let $X$ be a first-countable space. Consider a function $f:X\rightarrow Y$. The following statements are equivalent:
        \begin{itemize}
            \item $f$ is continuous.
            \item The sequence $(f(x_n))_{n\in\mathbb{N}}$ converges to $f(a)\in Y$ whenever the sequence $(x_n)_{n\in\mathbb{N}}$ converges to $a\in X$.
        \end{itemize}
    \end{property}
    \begin{result}
       If the space $Y$ in the previous theorem is Hausdorff then the limit $f(a)$ does not need to be known since it is unique by property \ref{topology:theorem:hausdorff_limit} above.
    \end{result}
    \begin{remark}
        If the space $X$ is not first-countable, we have to consider the convergence of nets \ref{set:net}.
    \end{remark}

    \begin{theorem}[Urysohn's lemma]\index{Urysohn!lemma}\label{topology:urysohns_lemma}
        A topological space X is normal\footnote{See definition \ref{topology:normal}.} if and only if every two closed disjoint subsets $A, B\subset X$ can be separated by a continuous function $f:X\rightarrow [0, 1]$ i.e.
        \begin{gather}
            f(a) = 0, \forall a\in A\qquad\qquad f(b) = 1, \forall b\in B.
        \end{gather}
    \end{theorem}

    \begin{theorem}[Tietze extension theorem]\index{Tietze extension theorem}
        Let $X$ be a normal space and let $A\subset X$ be a closed subset. Consider a continuous function $f:A\rightarrow\mathbb{R}$. There exists a continuous function $F:X\rightarrow\mathbb{R}$ such that $\forall a\in A: F(a) = f(a)$. Furthermore, if the function $f$ is bounded then $F$ can be chosen to be bounded by the same number.
    \end{theorem}
    \sremark{The Tietze extension theorem is equivalent to Urysohn's lemma.}

\subsection{Homeomorphisms}

    \newdef{Homeomorphism}{\index{homeomorphism}
        A map $f$ is called a homeomorphism if both $f$ and $f^{-1}$ are continuous and bijective.
    }
    \newdef{Diffeomorphism}{\index{diffeomorphism}\label{topology:diffeomorphism}
        A homeomorphism, differentiable\footnote{For this concept one technically needs the structure of a differentiable manifold (see chapter \ref{chapter:manifolds}).} of class $C^k$, is called a $C^k$-diffeomorphism.
    }

    \newdef{Embedding}{\index{embedding}\label{topology:embedding}
        A continuous map is an embedding if it is a homeomorphism onto its image.
    }
    \newdef{Local homeomorphism}{\index{local!homeomorphism}\index{\'etale map}
        A continuous map $f:X\rightarrow Y$ is a local homeomorphism if for every point $x\in X$ there exists an open neighbourhood $U$ such that $f(U)$ is open and such that $f_U$ is an embedding. Local homeomorphisms are also said to be \textbf{\'etale}.
    }

    \newdef{Covering space}{\index{covering!space}\label{topology:covering_space}
        Consider two topological spaces $X, C$ and a continuous surjection $p:C\rightarrow X$, called the \textbf{covering map}. $C$ is said to be a covering space of $X$ if for all points $x\in X$ there exists a neighbourhood $U$ of $x$ such that $p^{-1}(U)$ can be written as a disjoint union $\bigsqcup_i C_i$ of open sets in $C$ such that every set $C_i$ is mapped homeomorphically onto $U$. The neighbourhoods $U$ are sometimes said to be \textbf{evenly covered}.
    }
    \begin{notation}
        Because the covering map $p:C\rightarrow M$ is surjective, the space $M$ can be left implicit. Therefore covering spaces are often just denoted by the couple $(C, p)$.
    \end{notation}

    \newdef{Covering transformation}{\label{topology:covering_transformation}
        Consider two covering spaces $(C, p)$ and $(C', p')$. A continuous function $f:C\rightarrow C'$ is called a covering transformation if $p'\circ f=p$.
    }

    \newdef{Deck transformation}{\index{deck transformation}\label{topology:deck_transformation}
        Let $p:C\rightarrow X$ be a covering map of $X$. The automorphism group of $(C,p)$ in the category of covering spaces (over $M$) is given by all homeomorphisms $\varphi$ satisfying $p\circ\varphi=p$. These automorphisms are called deck transformations.
    }

    \newdef{\'Etal\'e space}{\index{etale!space}\index{stalk}\label{topology:etale_space}
        Let $X$ be a topological space. A topological space $Y$ is an \'etal\'e space over $X$ if there exists a continuous surjection $\pi:Y\rightarrow X$ such that $\pi$ is a local homeomorphism. The preimage $\pi^{-1}(x)$ of a point $x\in X$ is called the \textbf{stalk} of $\pi$ over $x$.
    }
    \begin{example}
        Every covering space is an \'etal\'e space.
    \end{example}

    \newdef{Pseudogroup}{\index{pseudo!group}\label{topology:pseudogroup}
        Let $X$ be a topological space. A pseudogroup is a collection $\mathcal{G}$ of homeomorphisms $\phi:U\rightarrow V$ between open subsets of $X$ such that:
        \begin{itemize}
            \item $\mathbbm{1}_U\in\mathcal{G}$ for all open $U\subseteq X$.
            \item If $\phi\in\mathcal{G}$ then $\phi^{-1}\in\mathcal{G}$.
            \item If $V\subset U$ is open then $\phi|_V\in\mathcal{G}$.
            \item If $U=\bigcup_{i\in I}U_i$ and $\phi|_{U_i}:U_i\rightarrow V$ is an element of $\mathcal{G}$ for all $i\in I$ then $\phi\in\mathcal{G}$.
            \item If $\phi:U\rightarrow V$ and $\psi:U'\rightarrow V'$ are elements of $\mathcal{G}$ and $V\cap U'\neq\emptyset$ then $\psi\circ\phi|_{\phi^{-1}(V\cap U')}\in\mathcal{G}$.
        \end{itemize}
    }

\section{Constructions}

    \begin{construct}[Product topology]\index{product!topology}\index{Tychonoff!topology}\label{topology:tychonoff_topology}
        First we consider the case where the index set $I$ is finite. The product space $X = \prod_{i\in I}X_i$ can be turned into a topological space by equipping it with the topology generated by the following basis:
        \begin{gather}
            \mathcal{B} := \left\{\prod_{i\in I}U_i:U_i\in\tau_i\right\}.
        \end{gather}
        In the general case (countably infinite and uncountable index sets) the topology can be defined using the canonical projections $\pi_i:X\rightarrow X_i$. The general product topology (\textbf{Tychonoff topology}) is the coarsest (finest) topology such that all projections $\pi_i$ are continuous.
    \end{construct}

    \begin{construct}[Disjoint union]\index{disjoint union}\label{topology:disjoint_union}
        Let $\{X_i\}_{i\in I}$ be a family of topological spaces. Now consider the disjoint union
        \begin{gather}
            X = \bigsqcup_{i\in I} X_i
        \end{gather}
        together with the canonical inclusion maps $\phi_i:X_i\rightarrow X:x_i\mapsto(i, x_i)$. We can turn $X$ into a topological space by equipping it with the following topology:
        \begin{gather}
            \tau_X = \big\{U\subseteq X:(\forall i\in I:\phi_i^{-1}(U)\text{ is open in }X_i)\big\}.
        \end{gather}
    \end{construct}

    \begin{construct}[Quotient space]\index{quotient!space}
        Consider a topological space $X$ and a subset $A\subseteq X$. The quotient $X/A$ is defined as the set of points $X\backslash A$ together with a single point obtained by identifying all points in $A$. This canonically turns the quotient space into a pointed space.

        For the degenerate case where $A=\emptyset$ we can also apply the above definition. However, this has the awkward effect that it adjoins a new point to the space $X$ instead of a collapsing it:
        \begin{gather}
            X/\emptyset = X\sqcup\ast.
        \end{gather}
        Let $\pi$ be the canonical projection $X\rightarrow X/A$. The quotient space can be turned into a topological space by equipping it with the following topology:
        \begin{gather}
            \label{topology:quotient_space}
            \tau_q := \big\{U\subseteq X/A:\pi^{-1}(U)\text{ is open in }X\big\}.
        \end{gather}
    \end{construct}

    \begin{construct}[Wedge sum]\index{wedge!sum}
        Consider two pointed spaces $(X, x_0), (Y, y_0)$. The wedge sum $X\vee Y$ is defined as the quotient of the disjoint union $X\sqcup Y$ obtained by identifying the basepoints $x_0\sim y_0$.
    \end{construct}
    \newdef{Smash product}{\index{smash product}
        Consider two pointed topological spaces $(X, x_0), (Y, y_0)$. The smash product $X\wedge Y$ is defined as the following quotient:
        \begin{gather}
            X\wedge Y := (X\times Y)/(X\vee Y)
        \end{gather}
        where $X\vee Y$ sits inside the product as the union of $X\times\{y_0\}$ and $\{x_0\}\times Y$.
    }

    \begin{construct}[Suspension]\index{suspension}\label{topology:suspension}
        Let $X$ be a topological space. The suspension of $X$ is defined as the following quotient space:
        \begin{gather}
            SX := (X\times [0, 1])/\big\{(x, 0) \sim (y, 0)\text{ and }(x, 1) \sim (y, 1):x, y\in X\big\}.
        \end{gather}
        By the remark at the end of the previous definition we see that the suspension of the empty set is in fact not empty, but equal to the two-point space $S^0$.

        An often more interesting construction is the \textbf{reduced suspension} $\Sigma X$. This obtained by taking the ordinary suspension $SX$ of a pointed space $(X, x_0)$ and identifying all copies of $x_0$:
        \begin{gather}
            \Sigma X := SX/(x_0\times[0,1]).
        \end{gather}
        An equivalent definition of the reduced suspension can be given in terms of the smash product:
        \begin{gather}
            \Sigma X = X\wedge S^1.
        \end{gather}
    \end{construct}
    \begin{example}[Spheres]\label{topology:sphere_suspension}
        Up to homeomorphisms the spheres are related by (reduced) suspensions:
        \begin{gather}
            \Sigma S^n \cong S^{n+1}.
        \end{gather}
        If we identify the empty set with the $(-1)$-sphere then the remark from previous definition lets us continue this relation to the case $n=-1$.
    \end{example}

    \begin{construct}[Attaching space]\index{attaching space}\label{topology:attaching_space}
        Let $X, Y$ be two topological spaces and consider a subspace $A\subseteq Y$. For every continuous map $f:A\rightarrow X$, called the \textbf{attaching map}, we can construct the attaching space\footnote{Sometimes called the \textbf{adjunction space}.} $X\cup_f Y$ in the following way:
        \begin{gather}
            X\cup_f Y := (X\sqcup Y)/\{A\sim f(A)\}.
        \end{gather}
    \end{construct}

    \begin{construct}[Join]\index{join}
        Let $\{A_i\}_{i\leq n}$ be a collection of topological spaces. The join, denoted by $A=A_1\circ\cdots\circ A_n$, is defined as follows: Every point of $A$ is defined by an $n$-tuple of non-negative numbers $\{t_i\}_{i\leq n}$ satisfying\footnote{Hence an element of an $n$-simplex. (See definition \ref{topology:simplex}.)} $\sum_it_i=1$ and for each index $i$ such that $t_i\neq 0$ a point $a_i\in A_i$. This point in $A$ is then denoted by $t_1a_1\oplus\cdots\oplus t_na_n$.

        In the case of two spaces one has a more intuitive (but equivalent) construction: Let $A, B$ be two topological spaces. The join $A\circ B$ is defined as the quotient space $(A\times B\times [0, 1])/\sim$ where the relation $\sim$ is defined as follows:
        \begin{itemize}
            \item For all $a\in A$ and $b, b'\in B$: $(a, b, 0)\sim(a, b', 0)$.
            \item For all $a, a'\in A$ and $b\in B$: $(a, b, 1)\sim(a', b, 1)$.
        \end{itemize}
        This can be interpreted as collapsing one end of the cylinder $(A\times B)\times[0, 1]$ to $A$ and the other end to $B$.
    \end{construct}
    \begin{property}
        The join induces a monoidal structure on the category \textbf{Top} where the tensor unit is given by the empty space $\emptyset$.
    \end{property}

    \newdef{Mapping cylinder}{\index{mapping cylinder}
        Let $f:X\rightarrow Y$ be a continuous function. The mapping cylinder $M_f$ is defined as follows:
        \begin{gather}
            M_f := \left([0, 1]\times X\bigsqcup Y\right)/\sim_f.
        \end{gather}
        where the equivalence relation $\sim_f$ is generated by the relations $(0, x)\sim f(x)$. From this definition it follows that the ''top'' of the cylinder is homeomorphic to $X$ and the "base" is homeomorphic to $f(X)\subseteq Y$.
    }

\section{Connected spaces}

    \newdef{Connected space}{\index{connected}\label{topology:connected}
        A topological space $X$ is connected if it cannot be written as the disjoint union of two non-empty open sets. Equivalently, $X$ is connected if the only clopen sets are $X$ and $\emptyset$.
    }

    \begin{property}
        Let $X$ be a connected space. Let $f$ be a function on $X$. If $f$ is locally constant, i.e. for every $x\in X$ there exists a neighbourhood U on which $f$ is constant, then $f$ is constant on all of $X$.
    \end{property}

    \begin{theorem}[Intermediate value theorem]\index{intermediate value theorem}\label{topology:theorem:intermediate_value_theorem}
        Let $X$ be a connected space. Let $f:X\rightarrow\mathbb{R}$ be a continuous function. If $a, b\in f(X)$ then for every $c\in ]a, b[$ we have that $c\in f(X)$.
    \end{theorem}

    \newdef{Path-connected space\protect\footnotemark}{\index{arc!connected|see{path-connected}}\index{path!connected}
        \footnotetext{A similar notion is that of \textbf{arcwise-connectedness} where the function $\varphi$ is required to be a homeomorphism.}
        Let $X$ be a topological space. If for every two points $x, y\in X$ there exists a continuous function $\varphi:[0, 1]\rightarrow X$ (i.e. a \textbf{path}) such that $\varphi(0)=x$ and $\varphi(1)=y$ then the space is said to be path-connected.
    }

    \begin{property}
        Every path-connected space is connected. The converse does not hold. There exists however the following (stronger) relation: A connected and locally path-connected space is path-connected.
    \end{property}

    \begin{remark}
        The notions of connectedness and path-connectedness define equivalence relations on the space $X$. The equivalence classes are closed in $X$ and form a cover of $X$. The set of path components of $X$ is often denoted by $\pi_0(X)$.\footnote{This follows from the relation with homotopy groups (see section \ref{section:homotopy}).}
    \end{remark}

\section{Compact spaces}
\subsection{Compactness}

    \newdef{Sequentially compact}{
        A topological space is sequentially compact if every sequence\footnote{The sequence itself does not have to converge.} has a convergent subsequence.
    }

    \newdef{Finite intersection property}{\index{finite!intersection property}
        A family $\mathcal{F}\subseteq2^X$ of subsets has the finite intersection property\footnote{The family is then called a FIP-family.} (FIP) if every finite subfamily has a nonzero intersection
        \begin{gather}
            \bigcap_{i\in I}V_i \neq \emptyset
        \end{gather}
        for all finite index sets $I$.
    }

    \newdef{Locally finite cover}{
        An open cover of a topological space $X$ is said to be locally finite if every $x\in X$ has a neighbourhood that intersects only finitely many sets in the cover of $X$.
    }

    \begin{property}
        A first-countable space is sequentially compact if and only if every countable open cover has a finite subcover.
    \end{property}

    \newdef{Lindel\"of space}{\index{Lindel\"of!space}
        A space for which every open cover has a countable subcover.
    }
    \begin{property}
        Every second-countable space is also a Lindel\"of space.
    \end{property}

    \newdef{Compact space}{\index{compact}
        A topological space $X$ is compact if every open cover of $X$ has a finite subcover.
    }

    \begin{theorem}[Heine-Borel\footnotemark]\index{Heine-Borel}
        \footnotetext{Also Borel-Lebesgue.}
        If a topological space $X$ is sequentially compact and second-countable then every open cover has a finite subcover. This implies that $X$ is compact.
    \end{theorem}
    \begin{theorem}[Heine-Borel on real numbers]
            A subset of $\mathbb{R}^n$ is compact if and only if it is closed and bounded.
    \end{theorem}

    \begin{theorem}[Tychonoff's theorem]\index{Tychonoff!compactness theorem}
        Any product\footnote{Finite, countably infinite or even uncountably infinite.} of compact topological spaces is again compact when equipped with the (Tychonoff) product topology \ref{topology:tychonoff_topology}.
    \end{theorem}

    \newdef{Relatively compact}{\label{topology:relatively_compact}
        A topological space is called relatively compact if its closure is compact.
    }

    \newdef{Locally compact}{
        A topological space is locally compact if every point $x\in X$ has a compact neighbourhood.
    }

    \begin{theorem}[Dini]\index{Dini}
        Let $(X, \tau)$ be a compact space. Let $(f_n)_{n\in\mathbb{N}}$ be a monotone sequence of continuous functions $f_n:X\rightarrow\mathbb{R}$. If $(f_n)\rightarrow f$ pointwise to a continuous function $f$ then the convergence is uniform.
    \end{theorem}

    \newdef{Paracompact space}{\index{paracompactness}\label{topology:paracompact}
        A topological space is paracompact if every open cover has a locally finite open refinement.
    }

    \begin{property}
        A paracompact Hausdorff space is normal.
    \end{property}

    \newdef{$\omega$-boundedness}{
        Let $X$ be a topological space. $X$ is said to be $\omega$-bounded if the closure of every countable subset is compact.
    }

    \newdef{Partition of unity}{\index{partition!of unity}\label{topology:partition_of_unity}
        Let $\{\varphi_i: X\rightarrow [0, 1]\}_i$ be a collection of continuous functions such that for every $x\in X$ the following conditions hold:
        \begin{itemize}
            \item For every neighbourhood $U$ of $x$, the set $\{f_i:\text{supp}f_i\cap U \neq \emptyset\}$ is finite.
            \item $\sum_if_i = 1$
        \end{itemize}
    }
    \begin{definition}[Subordinate]
        Consider an open cover $\{V_i\}_{i\in I}$ of $X$, indexed by a set $I$. If there exists a partition of unity, also indexed by $I$, such that $\text{supp}(\varphi_i)\subseteq U_i$, then this partition of unity is said to be \textbf{subordinate} to the open cover.
    \end{definition}

    \begin{property}\label{topology:paracompact_partition_unity}
        A paracompact space is Hausdorff if and only if it admits a partition of unity subordinate to any open cover.
    \end{property}

    \newdef{Numerable open cover}{\index{numerable}
        An open cover $\{U_i\}_{i\in I}$ of a space $X$ is said to be numerable if $X$ admits a partition of unity subordinate to $\{U_i\}_{i\in I}$.
    }

\subsection{Compactifications}

    \newdef{Dense}{\index{dense}
        A subset $V\subseteq X$ is dense in a topological space $X$ if $\overline{V} = X$.
    }
    \newdef{Separable space}{\index{separable!space}
        \label{topology:separable}
        A topological space is separable if it contains a countable dense subset.
    }
    \begin{property}
        Every second-countable space is separable.
    \end{property}

    \newdef{Compactification}{\index{compactification}
        A compact topological space $(X', \tau')$ is a compactification of a topological space $(X, \tau)$ if $X$ is a dense subspace of $X'$.
    }

    \begin{example}
        Standard examples of compactifications are the extended real line $\mathbb{R} \cup \{-\infty, +\infty\}$ and the extended complex plane $\mathbb{C}\cup\{\infty\}$ for the real line and the complex plane respectively.
    \end{example}
    \begin{remark*}
        It is important to note that compactifications are not necessarily unique.
    \end{remark*}

    \newdef{One-point compactification}{\index{Alexandrov compactification}\label{topology:alexandrov_compactification}
        Let $X$ be a Hausdorff space. A one-point compactification or \textbf{Alexandrov compactification} is a compactification $X'$ such that $X'\setminus X$ is a singleton.
    }
    \begin{example}[Real line]
        The classic example of a (one-point) compactification is that of the real line. If we adjoin the points $\pm\infty$ and identify them, then we obtain $S^1$. In general we can obtain the $n$-dimensional sphere $S^n$ as the one-point compactification of $\mathbb{R}^n$. This can be regarded as an \textit{inverse stereographic projection}.
    \end{example}

\section{Uniform spaces}

    \newdef{Uniform structure}{
        Consider a set $X$. A uniform structure on $X$ consists of a collection $\mathfrak{U}$ of subsets $U\subseteq X\times X$ that satisfy the following properties:
        \begin{enumerate}
            \item If $U\in\mathfrak{U}$ and $U\subset V$ then $V\in\mathfrak{U}$.
            \item If $U, V\in\mathfrak{U}$ then $U\cap V\in\mathfrak{U}$.
            \item If $U\in\mathfrak{U}$ then $\Delta_X\subset U$.
            \item If $U\in\mathfrak{U}$ then there exists $V\in\mathfrak{U}$ such that $V\circ V=U$.
            \item If $U\in\mathfrak{U}$ then $U^t\in\mathfrak{U}$.
        \end{enumerate}
        where the'' transpose'' $U^t$ denotes the converse of $U$ (see definition \ref{set:converse}) and where the composition $\circ$ is the relational composition of $V$ and $V$ (see definition \ref{set:relational_composition}). The elements of the uniformity $\mathfrak{U}$ are called \textbf{entourages}. If $(x, y)\in U$ for some entourage $U\in\mathfrak{U}$ then $x$ and $y$ are said to be \textbf{$U$-close}.
    }
    \remark{The first three conditions imply that a uniform structure is in particular a filter.}

\section{\texorpdfstring{Locales $\clubsuit$}{Locales}}

    \begin{property}
        Consider the poset $\textbf{Open}(X)$ of opens of a topological space $X$. This set is closed under finite intersections (limits) and arbitrary unions (colimits). Furthermore, arbitrary unions distribute over finite intersections:
        \begin{gather}
            V\cap\left(\bigcup_{i\in I}U_i\right) = \bigcup_{i\in I}\left(V\cap U_i\right).
        \end{gather}
        This implies that the poset $\textbf{Open}(X)$ is a frame\footnote{See definition \ref{set:frame}.}.
    \end{property}

    \newdef{Locale}{\index{locale}
        The previous property can be used to generalize the notion of topological space to include \textit{pointless spaces}. Let \textbf{Frame} denote the category of frames together with frame homomorphisms. The category of locales is defined as the opposite category: \[\textbf{Loc} := \textbf{Frame}^{op}.\]
    }
    \begin{construct}[From locale to topological space]
        There exists an adjunction \[\textbf{Loc}\adj{\iota}{\text{Point}}\textbf{Top}\] where the right adjoint is defined as follows:

        \indent Let $L$ be a locale. For a topological space the points are given by continuous maps $\ast\rightarrow X$ and hence by frame morphisms $\textbf{Open}(X)\rightarrow\Omega_{\text{Frame}}=\{0, 1\}$. Generalizing this to locales one defines the set of points of $L$ as the $\Omega_{\text{Loc}}$-elements: \[\text{Point}(L) := \textbf{Loc}(\Omega_{\text{Loc}}, L).\] This set can be given a topology by declaring for every $U\in L$ the set $\{p\in\text{Point}(L) : p^{-1}(U) = 1\}$ to be open.
    \end{construct}
\chapter{General Topology}
\section{Topological spaces}

	\newdef{Topology}{\index{topology}
		\nomenclature[S_Top]{$\text{Top}$}{Category of topological spaces.}
    		Let $\Omega$ be a set. Let $\tau\subseteq 2^\Omega$. The set $\tau$ is a topology on $\Omega$ if it satisfies following axioms:
	        \begin{enumerate}
        		\item $\emptyset\in\tau$ and $\Omega\in\tau$
        		\item $\forall\ \mathcal{F}\subseteq\tau: \bigcup_{V\in\mathcal{F}}V \in \tau$
		        \item $\forall\ U, V\in\tau: U\cap V\in\tau$
	        \end{enumerate}
        	Furthermore we call the elements of $\tau$ open sets and the couple $(\Omega, \tau)$ a topological space.
	}
	\sremark{On topological spaces the open sets are thus defined by axioms.}
	
	\begin{property}
		\nomenclature[S_Open]{$\text{Open}(X)$}{Category of open subsets of a topological space $X$.}
		Consider a topological space $(X, \tau)$. Let $U\subseteq V\in\tau$. The inclusion maps $U\hookrightarrow V$ are morphisms. The set of these morphisms together with the topology $\tau$ form a (small) category $\text{Open}(X)$.
	\end{property}

	\newdef{Relative topology\footnotemark}{
		\footnotetext{Sometimes called the \textbf{subspace topology}.}
		Let $(X, \tau_X)$ be a topological space and $Y$ a subset of $X$. We can turn $Y$ into a topological space by equipping it with the following topology:
		\begin{equation}
			\label{topology:relative_topology}
			\tau_\text{rel} = \{U_i\cap Y:U_i\in \tau_X\}
		\end{equation}
	}
	
	\newdef{Disjoint union}{\index{disjoint union}\label{topology:disjoint_union}
		Let $\{X_i\}_{i\in I}$ be a family of topological spaces. Now consider the disjoint union
		\begin{equation}
			X = \bigsqcup_{i\in I} X_i
		\end{equation}
		together with the canonical inclusion maps $\phi_i:X_i\rightarrow X:x_i\mapsto(x_i, i)$. We can turn $X$ into a topological space by equipping it with the following topology:
		\begin{equation}
			\tau_X = \{U\subseteq X| \forall i\in I:\phi_i^{-1}(U)\text{ is open in }X_i\}
		\end{equation}
	}

	\newdef{Quotient space}{\index{quotient!space}
		Let $X$ be a topological space and let $\sim$ be an equivalence relation defined on $X$. The set $X/_\sim$ can be turned into a topological space by equipping it with the following topology:
		\begin{equation}
			\label{topology:quotient_space}
			\tau_\sim = \{U\subseteq X/_\sim|\pi^{-1}(U)\text{ is open in }X\}
		\end{equation}
		where $\pi$ is the canonical surjective map from $X$ to $X/_\sim$.
	}

	\begin{example}[Discrete topology]\index{discrete!topology}
		The discrete topology is the topology such that every subset is open (and thus also closed).
	\end{example}
	
	\begin{example}[Product topology]\index{product!topology}\index{Tychonoff!topology}\label{topology:tychonoff_topology}
		First consider the case where the index set $I$ is finite. The product space $X = \prod_{i\in I}X_i$ can be turned into a topological space by equipping it with the topology generated by the following basis:
		\begin{equation}
			\mathcal{B} = \left\{\left.\prod_{i\in I}U_i\ \right|U_i\in\tau_i\right\}
		\end{equation}
		For general cases (countably infinite and uncountable index sets) the topology can be defined using the canonical projections $\pi_i:X\rightarrow X_i$. The general product topology (\textbf{Tychonoff topology}) is the coarsest (finest) topology such that all projections $\pi_i$ are continuous.
	\end{example}
	
	\newdef{Topological group}{\index{group!topological}
		A topological group is a group $G$ equipped with a topology such that both the multiplication and inversion map are continuous.
	}
	
	
	\newdef{Pointed topological space}{\label{topology:pointed_space}
		Let $x_0\in X$. The triple $(X, \tau, x_0)$ is called a pointed topological space with base point $x_0$.
	}
	\begin{construct}[Suspension]\index{suspension}
		Let $X$ be a topological space. The suspension of $X$ is defined as the following quotient space:
		\begin{equation}
			\label{topology:suspension}
			SX = 	(X\times [0, 1])/\{(x, 0) \sim (y, 0)\text{ and }(x, 1) \sim (y, 1)|x, y\in X\}
		\end{equation}
	\end{construct}
	
	\begin{construct}[Attaching space]\index{attaching space}\label{topology:attaching_space}
		Let $X, Y$ be two topological spaces and consider a subspace $A\subseteq Y$. For every continuous map $f:A\rightarrow X$, called the \textbf{attaching map}, we can construct the attaching space\footnote{Sometimes called the \textbf{adjunction space}.} $X\cup_f Y$ in the following way:
		\begin{equation}
			X\cup_f Y = (X\sqcup Y)/\{A\sim f(A)\}
		\end{equation}
	\end{construct}
	
	\begin{construct}[Join]\index{join}
		Let $\{A_i\}_{i\leq n}$ be a collection of topological spaces. The join, denoted by $A=A_1\circ\cdots\circ A_n$, is defined as follows: Every point of $A$ is defined by an $n$-tuple of non-negative numbers $\{t_i\}_{i\leq n}$ satisfying\footnote{Hence an element of an $n$-simplex. (See definition \ref{topology:simplex}.)} $\sum_it_i=1$ and for each index $i$ such that $t_i\neq 0$ a point $a_i\in A_i$. This point in $A$ is then denoted by $t_1a_1\oplus\cdots\oplus t_na_n$.
		
		In the case of two spaces one has a more intuitive (but equivalent) construction: Let $A, B$ be two topological spaces. The join $A\circ B$ is defined as the quotient space $(A\times B\times [0, 1])/\sim$ where the relation $\sim$ is defined as follows:
		\begin{itemize}
			\item For all $a\in A$ and $b, b'\in B$: $(a, b, 0)\sim(a, b', 0)$
			\item For all $a, a'\in A$ and $b\in B$: $(a, b, 1)\sim(a', b, 1)$
		\end{itemize}
		which can be viewed as collapsing one end of the cilinder $(A\times B)\times[0, 1]$ to $A$ and the other end to $B$.
	\end{construct}
	\begin{property}
		The join induces a monoidal structure on the category \textbf{Top} where the tensor unit is given by the empty space $\emptyset$.
	\end{property}
    
\section{Neighbourhoods}
\subsection{Neighbourhoods}

	\newdef{Neighbourhood}{\index{neighbourhood}
		A set $V\subseteq\Omega$ is a neighbourhood of a point $a\in\Omega$ if there exists an open set $U\in\tau$ such that $a\in U\subseteq V$.
	}
    
	\newdef{Basis}{\index{basis}
	    	Let $\mathcal{B}\subseteq\tau$ be a family of open sets. The family $\mathcal{B}$ is a basis for the topological space $(\Omega, \tau)$ if every $U\in\tau$ can be written as:
	        \begin{equation}
	        	U = \bigcup_{V\in\mathcal{F}}V
	        \end{equation}
	        where $\mathcal{F}\subseteq\mathcal{B}$.
	}
	\newdef{Local basis}{
	    	Let $\mathcal{B}_x$ be a family of open neighbourhoods of a point $x\in\Omega$. $\mathcal{B}_x$ is a local basis of $x$ if every neighbourhood of $x$ contains at least one element in $\mathcal{B}_x$.
	}
    
	\newdef{First-countability}{\index{countable!countable space}
	    	A topological space $(\Omega, \tau)$ is first-countable if for every point $x\in\Omega$ there exists a countable local basis.
	}
	\begin{property}[Decreasing basis]
		Let $x\in\Omega$. If there exists a countable local basis for $x$ then there also exists a countable decreasing local basis for $x$.
	\end{property}
    
	\newdef{Second-countability}{
    		A topological space $(\Omega, \tau)$ is second-countable if there exists a countable global basis.
	}
    
	\begin{property}
	    	Let $X$ be a topological space. The closure of a subset $V$ is given by:
	    	\begin{equation}
	    		\label{topology:closure}
	    		\overline{V} = \{x\in X| \exists \text{ a net } (x_\lambda)_{\lambda\in I} \text{ in } X:x_\lambda\rightarrow x\}
	    	\end{equation}
	    	This implies that the topology on $X$ is completely determined by the convergence of nets\footnote{See definition \ref{set:net}.}.
	\end{property}
	\begin{result}
	    	In first-countable spaces we only have to consider the convergence of sequences.
	\end{result}
    
	\newdef{Germ}{\index{germ}\label{topology:germ}
		Let $X$ be a topological space and let $Y$ be a set. Consider two functions $f, g: X\rightarrow Y$. If there exists a neighbourhood $U$ of a point $x\in X$ such that
		\[f(u) = g(u)\qquad\qquad\forall u\in U\]
		then this property defines an equivalence relation denoted by $f\sim_x g$ and the equivalence classes are called \textbf{germs}.
	}
	
	\begin{property}
		Let the set $Y$ in the previous definition be the set of reals $\mathbb{R}$. Then the germs at a point $p\in X$ satisfy following closure/linearity relations:
		\begin{itemize}
			\item $[f] + [g] = [f+g]$
			\item $\lambda[f] = [\lambda f]$
			\item $[f][g] = [fg]$
		\end{itemize}
		where $[f], [g]$ are two germs at $p$ and $\lambda\in\mathbb{R}$ is a scalar.
	\end{property}
   
\subsection{Separation axioms}\index{separation axioms}

	\newdef{$T_0$ axiom\footnotemark}{\index{Kolmogorov!topology}
		\footnotetext{$T_0$ spaces are also said to carry the \textbf{Kolmogorov topology}.}
		A topological space is $T_0$ if for every two distinct points $x, y$ at least one of them has a neighbourhood not containing the other. The points are said to be topologically distinguishable.
	}
	
	\newdef{$T_1$ axiom\footnotemark}{\index{Fr\'echet!topology}
		\footnotetext{$T_1$ spaces are also said to carry the \textbf{Fr\'echet topology}.}
		A topological space is $T_1$ if for every two distinct points $x, y$ there exists a neighbourhood $U$ of $x$ such that $y\not\in U$. The points are said to be separated.
	}

	\newdef{Hausdorff space}{\index{Hausdorff!space}
		A topological space is a Hausdorff space or $T_2$ space if it satisfies the following axiom:
		\begin{equation}
			(\forall x, y \in\Omega)(\exists \text{ neighbourhoods }U, V)(x\in U, y\in V, U\cap V=\emptyset)
		\end{equation}
		This axiom is called the \textbf{Hausdorff separation axiom} or $T_2$ axiom. The points are said to be separated by neighbourhoods.
	}
    	\begin{property}
		Every singleton (and thus also every finite set) is closed in a Hausdorff space.
	\end{property}
	
	\newdef{Urysohn space\footnotemark}{\index{Urysohn!space}
		\footnotetext{Sometimes called a $T_{2\nicefrac{1}{2}}$ space.}
		A topological space is an Urysohn space if every two distcinct points are separated by closed neighbourhoods.
	}

	\newdef{Regular space}{\index{regular}\label{topology:regular}
		A topological space is said to be regular if for every closed subset $F$ and every point $x\not\in F$ there exist disjoint open subsets $U, V$ such that $x\in U$ and $F\subset V$.
	}
	\newdef{$T_3$ axiom}{
		A space that is both regular and $T_0$ is $T_3$.
	}

	\newdef{Normal space}{\index{normal}\label{topology:normal}
		A topological space is said to be normal if every two closed subsets have disjoint neighbourhoods.
	}
	\newdef{$T_4$ axiom}{
		A space that is both normal and $T_1$ is $T_4$.
	}
	
	\begin{property}
		A space satisfying the separation axiom $T_k$ also satisfies all separation axioms $T_{i\leq k}$.
	\end{property}
    
\section{Morphisms}
\subsection{Convergence}

	\newdef{Convergence}{\index{convergence}
	    	A sequence $(x_n)_{n\in\mathbb{N}}$ in $X$ is said to converge to a point $a\in X$ if:
	        \begin{equation}
	        	(\forall \text{ neighbourhoods } U \text{ of } a)(\exists N > 0)(\forall n>N)(x_n\in U)
	        \end{equation}
	}

	\begin{property}
	    	Every subsequence of a converging sequence converges to the same point\footnote{This limit does not have to be unique. See the next property for more information.}.
	\end{property}
	\begin{property}\label{topology:theorem:hausdorff_limit}
	    	Let $X$ be a Hausdorff space. The limit of a converging sequence in $X$ is unique.
	\end{property}

\subsection{Continuity}
    
	\newdef{Continuity}{\index{continuity}
    		A function $f:X\rightarrow Y$ is continuous if the inverse image $f^{-1}(U)$ of every open set $U$ is also open.
	}
	\begin{theorem}
    		Let $X$ be a first-countable space. Consider a function $f:X\rightarrow Y$. The following statements are equivalent:
        	\begin{itemize}
        		\item $f$ is continuous
        		\item The sequence $(f(x_n))_{n\in\mathbb{N}}$ converges to $f(a)\in Y$ whenever the sequence $(x_n)_{n\in\mathbb{N}}$ converges to $a\in X$.
	        \end{itemize}
	\end{theorem}
	\begin{result}
	   	If the space $Y$ in the previous theorem is Hausdorff then the limit $f(a)$ does not need to be known because the limit is unique (see \ref{topology:theorem:hausdorff_limit}).
	\end{result}
	\begin{remark}
    		If the space $X$ is not first-countable, we have to consider the convergence of nets \ref{set:net}.
	\end{remark}
    
	\begin{theorem}[Urysohn's lemma]\index{Urysohn!lemma}
    		A topological space X is normal\footnotemark\ if and only if every two closed disjoint subsets $A, B\subset X$ can be separated by a continuous function $f:X\rightarrow [0, 1]$ i.e.
    		\footnotetext{See definition \ref{topology:normal}.}
	    	\begin{equation}
    			\label{topology:urysohns_lemma}
    			f(a) = 0, \forall a\in A\qquad\qquad f(b) = 1, \forall b\in B
    		\end{equation}
	\end{theorem}
	
	\begin{theorem}[Tietze extension theorem]\index{Tietze extension theorem}
		Let $X$ be a normal space and let $A\subset X$ be a closed subset. Consider a continuous function $f:A\rightarrow\mathbb{R}$. There exists a continuous function $F:X\rightarrow\mathbb{R}$ such that $\forall a\in A: F(a) = f(a)$. Furthermore, if the function $f$ is bounded then $F$ can be chosen to be bounded by the same number.
	\end{theorem}
	\sremark{The Tietze extension theorem is equivalent to Urysohn's lemma.}
    
\subsection{Homeomorphisms}
	
	\newdef{Homeomorphism}{\index{homeomorphism}
		A map $f$ is called a homeomorphism if both $f$ and $f^{-1}$ are continuous and bijective.
	}
	\newdef{Diffeomorphism}{\index{diffeomorphism}\label{topology:diffeomorphism}
		A homeomorphism, differentiable of class $C^k$, is called a $C^k$-diffeomorphism.
	}
	
	\newdef{Embedding}{\index{embedding}\label{topology:embedding}
		A continuous map is an embedding if it is a homeomorphism onto its image.
	}
	\newdef{Local homeomorphism}{\index{local!homeomorphism}
		A continuous map $f:X\rightarrow Y$ is a local homeomorphism if for every point $x\in X$ there exists an open neighbourhood $U$ such that $f(U)$ is open and such that $f_U$ is an embedding.
	}
	
	\newdef{Mapping cylinder}{\index{mapping cylinder}
		Let $f:X\rightarrow Y$ be a continuous function. The mapping cylinder $M_f$ is defined as follows:
		\begin{equation}
			M_f = \left([0, 1]\times X\bigsqcup Y\right)/\sim_f
		\end{equation}
		where the equivalence relation $\sim_f$ is generated by the relations $(0, x)\sim f(x)$. From this definition it follows that the "top" of the cylinder is homeomorphic to $X$ and the "base" is homeomorphic to $f(X)\subseteq Y$.
	}
	
	\newdef{Covering space}{\index{covering space}\label{topology:covering_space}
		Consider two topological spaces $X, C$ and a continuous surjective map $\phi:C\rightarrow X$, called the \textbf{covering map}. $C$ is said to be a covering space of $X$ if for all points $x\in X$ there exists a neighbourhood $U$ of $x$ such that $\phi^{-1}(U)$ can be written as a disjoint union $\bigsqcup_i C_i$ of open sets in $C$ such that every set $C_i$ is mapped homeomorphically onto $U$. The neighbourhoods $U$ are said to be \textbf{evenly covered}.
	}
	
	\newdef{Universal covering space}{
		A covering space $C$ is said to be universal if it is simply-connected\footnote{See definition \ref{topology:simply_connected}.}.
	}
	\begin{uproperty}
		Let $X$ be a topological space and let $C_X$ be the universal covering space of $X$, every other covering space $C$ of $X$ is also covered by $C_X$.
	\end{uproperty}
	
	\newdef{Deck transformation}{\index{deck transformation}\label{topology:deck_transformation}
		Let $p:C\rightarrow X$ be a covering map of $X$. The group of deck transformations\footnote{In fact this group forms the automorphism group of $(C, p)$ in the category of covering spaces of $X$.} of $(C, p)$ is given by all homeomorphisms $\varphi$ satisfying $p\circ \varphi = p$.
	}
	
	\newdef{\'Etal\'e space}{\index{etale!space}\index{stalk}\label{topology:etale_space}
		Let $X$ be a topological space. A topological space $Y$ is an \'etal\'e space over $X$ if there exists a continuous surjective map $\pi:Y\rightarrow X$ such that $\pi$ is a local homeomorphism. The preimage $\pi^{-1}(x)$ of a point $x\in X$ is called the \textbf{stalk} of $\pi$ over $x$.
	}
	\begin{property}
		Every covering space is an \'etal\'e space.
	\end{property}
	\newdef{Section}{\index{section!of \'etal\'e spaces}
		A section of an \'etal\'e space $\pi:Y\rightarrow X$ over an open set $U\subseteq X$ is a continuous map $f$ such that $\pi\circ f = \mathbbm{1}_U$.
	}
	
	\newdef{Pseudogroup}{\index{pseudo!group}\label{topology:pseudogroup}
		Let $X$ be a topological space. A collection $\mathcal{G}$ of homeomorphisms $\phi:U\subseteq M\rightarrow M$ such that:
		\begin{itemize}
			\item $\mathbbm{1}_M\in\mathcal{G}$
			\item If $\phi\in\mathcal{G}$ then $\phi^{-1}\in\mathcal{G}$
			\item If $V\subset U$ is open then $\phi|_V\in\mathcal{G}$
			\item If $U=\bigcup_{i\in I}U_i$ and $\phi_i:U_i\rightarrow$ is an element of $\mathcal{G}$ for all $i\in I$ then $\phi\in\mathcal{G}$
			\item If $\phi:U\rightarrow V$ and $\psi:U'\rightarrow V'$ are elements of $\mathcal{G}$ and $V\cap U'\neq\emptyset$ then $\psi\circ\phi|_{\phi^{-1}(V\cap U')}\in\mathcal{G}$
		\end{itemize}
	}
	
\section{Connected spaces}
	
	\newdef{Connected space}{\index{connected}\label{topology:connected}
		A topological space $X$ is connected if it cannot be written as the disjoint union of two non-empty open sets. Equivalently, $X$ is connected if the only clopen sets are $X$ and $\emptyset$.
	}
	
	\begin{property}
		Let $X$ be a connected space. Let $f$ be a function on $X$. If $f$ is locally constant, i.e. for every $x\in X$ there exists a neighbourhood U on which $f$ is constant, then $f$ is constant on all of $X$.
	\end{property}
	
	\begin{theorem}[Intermediate value theorem]\index{intermediate value theorem}\label{topology:theorem:intermediate_value_theorem}
		Let $X$ be a connected space. Let $f:X\rightarrow\mathbb{R}$ be a continuous function. If $a, b\in f(X)$ then for every $c\in ]a, b[$ we have that $c\in f(X)$.
	\end{theorem}

	\newdef{Path-connected space\protect\footnotemark}{\index{arc}\index{path!connected}
		\footnotetext{A similar notion is that of \textbf{arcwise-connectedness} where the function $\varphi$ is required to be a homeomorphism.}
		Let $X$ be a topological space. If for every two points $x, y\in X$ there exists a continuous function $\varphi:[0, 1]\rightarrow X$ (i.e. a \textbf{path}) such that $\varphi(0)=x$ and $\varphi(1)=y$ then the space is said to be path-connected.
	}
	
	\begin{property}
		Every path-connected space is connected.
	\end{property}
	The converse does not hold. There exists however the following (stronger) relation:
	\begin{property}
		A connected and locally path-connected space is path-connected.
	\end{property}
	
	\begin{remark}
		The notions of connectedness and path-connectedness define equivalence relations on the space $X$. The equivalence classes are closed in $X$ and form a cover of $X$.
	\end{remark}

\section{Compact spaces}
\subsection{Compactness}
	
	\newdef{Sequentially compact}{
    		A topological space is sequentially compact if every sequence\footnote{The sequence itself does not have to converge.} has a convergent subsequence.
	}
    
	\newdef{Finite intersection property}{\index{finite intersection property}
    		A family $\mathcal{F}\subseteq2^X$ of subsets has the finite intersection property\footnote{The family is then called a FIP-family.} if every finite subfamily has a non-zero intersection:
        	\begin{equation}
        		\bigcap_{i\in I}V_i \neq \emptyset
        	\end{equation}
        	for all finite index sets $I$.
	}

	\newdef{Locally finite cover}{
		An open cover of a topological space $X$ is said to be locally finite if every $x\in X$ has a neighbourhood that intersects only finitely many sets in the cover of $X$.
	}
    
	\begin{property}
		A first-countable space is sequentially compact if and only if every countable open cover has a finite subcover.
	\end{property}
    
	\newdef{Lindel\"of space}{\index{Lindel\"of!space}
		A space for which every open cover has a countable subcover.
	}
	\begin{property}
		Every second-countable space is also a Lindel\"of space.
	\end{property}
    
	\newdef{Compact space}{\index{compact}
		A topological space $X$ is compact if every open cover of $X$ has a finite subcover.
	}
    
	\begin{theorem}[Heine-Borel\footnotemark]\index{Heine-Borel}
	    	\footnotetext{Also Borel-Lebesgue.}
	    	If a topological space $X$ is sequentially compact and second-countable then every open cover has a finite subcover. This implies that $X$ is compact.
	\end{theorem}
	\begin{theorem}[Heine-Borel on real numbers]
    		A subset of $\mathbb{R}^n$ is compact if and only if it is closed and bounded.
	\end{theorem}
	
	\begin{theorem}[Tychonoff's theorem]\index{Tychonoff!theorem (compactness)}
		Any product\footnote{Finite, countably infinite or even uncountably infinite.} of compact topological spaces is again compact when equipped with the (Tychonoff) product topology \ref{topology:tychonoff_topology}.
	\end{theorem}
	
	\newdef{Relatively compact}{\label{topology:relatively_compact}
		A topological space is called relatively compact if its closure is compact.
	}

	\newdef{Locally compact}{
    		A topological space is locally compact if every point $x\in X$ has a compact neighbourhood.
	}
    
	\begin{theorem}[Dini]\index{Dini}
    		Let $(X, \tau)$ be a compact space. Let $(f_n)_{n\in\mathbb{N}}$ be an increasing sequence of continuous functions $f_n:X\rightarrow\mathbb{R}$. If $(f_n)_n\rightarrow f$ pointwise to a continuous function $f$ then the convergence is uniform.
	\end{theorem}
    
	\newdef{Paracompact space}{\index{paracompactness}\label{topology:paracompact}
		A topological space is paracompact if every open cover has a locally finite open refinement.
	}
	
	\begin{property}
		A paracompact Hausdorff space is normal.
	\end{property}

	\newdef{$\omega$-boundedness}{
		Let $X$ be a topological space. $X$ is said to be $\omega$-bounded if the closure of every countable subset is compact.
	}
 
	\newdef{Partition of unity}{\index{partition!of unity}\label{topology:partition_of_unity}
		Let $\{\varphi_i: X\rightarrow [0, 1]\}_i$ be a collection of continuous functions such that for every $x\in X$:
		\begin{itemize}
			\item For every neighbourhood $U$ of $x$, the set $\{f_i:\text{supp}f_i\cap U \neq \emptyset\}$ is finite.
			\item $\sum_if_i = 1$
		\end{itemize}
	}
	\begin{definition}[Subordinate]
		Consider an open cover $\{V_i\}_{i\in I}$ of $X$, indexed by a set $I$. If there exists a partition of unity, also indexed by $I$, such that $\text{supp}(\varphi_i)\subseteq U_i$, then this partition of unity is said to be \textbf{subordinate} to the open cover.
	\end{definition}
	
	\begin{property}\label{topology:paracompact_partition_unity}
		A paracompact space is Hausdorff if and only if it admits a partition of unity subordinate to any open cover.
	\end{property}
	
	\newdef{Numerable open cover}{\index{numerable}
		An open cover $\{U_i\}_{i\in I}$ of a space $X$ is said to be numerable if $X$ admits a partition of unity subordinate to $\{U_i\}_{i\in I}$.
	}

\subsection{Compactifications}

	\newdef{Dense}{\index{dense}
		A subset $V\subseteq X$ is dense in a topological space $X$ if $\overline{V} = X$.
	}
	\newdef{Separable space}{\index{separable}
		\label{topology:separable}
		A topological space is separable if it contains a countable dense subset.
	}
	\begin{property}
		Every second-countable space is separable.
	\end{property}
    
	\newdef{Compactification}{\index{compactification}
	    	A compact topological space $(X', \tau')$ is a compactification of a topological space $(X, \tau)$ if $X$ is a dense subspace of $X'$.
	}
    
	\begin{example}
		Standard examples of compactifications are the extended real line $\mathbb{R} \cup \{-\infty, +\infty\}$ and the extended complex plane $\mathbb{C}\cup\{\infty\}$ for the real line and the complex plane respectively.
	\end{example}
	\begin{remark*}
		It is important to note that compactifications are not necessarily unique.
	\end{remark*}
    
	\newdef{One-point compactification}{\index{Alexandrov compactification}\label{topology:alexandrov_compactification}
		Let $X$ be a Hausdorff space. A one-point compactification or \textbf{Alexandrov compactification} is a compactification $X'$ such that $X'\setminus X$ is a singleton.
	}
    
\section{Locales}

	\begin{property}
		Consider the poset $\textbf{Open}(X)$ of opens of a topological space $X$. This set is closed under finite intersections (limits) and arbitrary unions (colimits). Furthermore, arbitrary unions distribute over finite intersections:
		\begin{equation}
			V\cap\left(\bigcup_{i\in I}U_i\right) = \bigcup_{i\in I}\left(V\cap U_i\right)
		\end{equation}
		This implies that the poset $\textbf{Open}(X)$ is a frame\footnote{See definition \ref{set:frame}.}.
	\end{property}
	
	\newdef{Locale}{\index{locale}
		The previous property can be used to generalize the notion of topological space to include \textit{pointless spaces}. Let \textbf{Frame} denote the category of frames together with frame homomorphisms. The category of locales is defined as the opposite category: \[\textbf{Loc} = \textbf{Frame}^{op}\]
	}
	\begin{construct}[From locale to topological space]
		There exists an adjunction \[\textbf{Loc}\adj{\iota}{\text{Point}}\textbf{Top}\] where the right adjoint is defined as follows:
		
		Let $L$ be a locale. For a topological space the points are given by continuous maps $\ast\rightarrow X$ and hence by frame morphisms $\textbf{Open}(X)\rightarrow\Omega_{\text{Frame}}=\{0, 1\}$. Generalizing this to locales one defines the set of points of $L$ as the $\Omega_{\text{Loc}}$-elements: \[\text{Point}(L) = \textbf{Loc}(\Omega_{\text{Loc}}, L)\] This set can be given a topology by declaring for every $U\in L$ the set $\{p\in\text{Point}(L) : p^{-1}(U) = 1\}$ to be open.
	\end{construct}

\chapter{Statistics}

    In this chapter, most definitions and formulas will be based on either a standard calculus approach or a data-driven approach. For a measure-theory based approach see chapter \ref{chapter:probability}.

\section{Data samples}
\subsection{Moment}

    \newformula{$r^{th}$ sample  moment}{\index{moment}
        \begin{gather}
            \label{statistics:sample_moment}
            \overline{x^r} := \stylefrac{1}{N}\sum_{i=1}^Nx_i^r
        \end{gather}
    }
    \newformula{$r^{th}$ central sample moment}{
        \begin{gather}
            \label{statistics:central_sample_moment}
            m_r := \stylefrac{1}{N}\sum_{i=1}^N(x_i-\overline{x})^r
        \end{gather}
    }

\subsection{Mean}\index{mean}

    \newdef{Arithmetic mean}{The arithmetic mean is used to average out differences between measurements. It is equal to the $1^{st}$ sample moment:
        \begin{gather}
            \label{statistics:arithmetic_mean}
            \overline{x} := \stylefrac{1}{N}\sum_{i=1}^Nx_i.
        \end{gather}
    }
    \newdef{Weighted mean}{\label{statistics:weighted_mean}
        Let $f:\mathbb{R}\rightarrow\mathbb{R}^+$ be a weight function. The weighted mean is given by:
        \begin{gather}
            \overline{x} := \stylefrac{\sum_if(x_i)x_i}{\sum_if(x_i)}
        \end{gather}
    }
    \begin{result}
        If the data has been grouped in bins, the weight function is given by the number of elements in each bin. Knowing this the (binned) mean becomes:
        \begin{gather}
            \label{statistics:binned_arithmetic_mean}
            \overline{x} = \stylefrac{1}{N}\sum_{i=1}n_ix_i
        \end{gather}
    \end{result}
    \remark{In the above definitions the measurements $x_i$ can be replaced by function values $f(x_i)$ to calculate the mean of the function $f(x)$.}
    \remark{It is also important to notice that $\overline{f}(x) \neq f(\overline{x})$. The equality only holds for linear functions.}

    \newdef{Geometric mean}{
        Let $\{x_i\}$ be a positive data set\footnote{A negative data set is also allowed. The real condition is that all values should have the same sign.}. The geometric mean is used to average out \textit{normalised} measurements, i.e. ratios with respect to a reference value.
        \begin{gather}
            \label{statistics:geometric_mean}
            g := \left(\prod_{i=1}^Nx_i\right)^{1/N}
        \end{gather}
        The following relation exists between the arithmetic and geometic mean:
        \begin{gather}
            \ln g = \overline{\ln x}
        \end{gather}
    }

    \newdef{Harmonic mean}{
        \begin{gather}
            \label{statistics:harmonic_mean}
            h := \left(\stylefrac{1}{N}\sum_{i=1}^Nx_i^{-1}\right)^{-1}
        \end{gather}
        The following relation exists between the arithmetic and harmonic mean:
        \begin{gather}
            \frac{1}{h} = \overline{x^{-1}}
        \end{gather}
    }

    \begin{property}
        Let $\{x_i\}$ be a positive data set.
        \begin{gather}
            h\leq g\leq\overline{x}
        \end{gather}
        The equalities only hold when all $x_i$ are equal.
    \end{property}

    \newdef{Mode}{\index{mode}
        The most occurring value in a dataset.
    }
    \newdef{Median}{\index{median}
        The median of dataset is the value $x_i$ such that half of the values is greater than $x_i$ and the other half is smaller than $x_i$.
    }

\subsection{Dispersion}

    \newdef{Range}{\index{range}
        The simplest indicator for statistical dispersion:
        \begin{gather}
            R := x_{\max} - x_{\min}.
        \end{gather}
        It is however very sensitive for extreme values.
    }

    \newdef{Mean absolute difference}{
        \begin{gather}
            \text{MD} := \stylefrac{1}{N}\sum_{i=1}^N|x_i - \overline{x}|
        \end{gather}
    }

    \newdef{Sample variance}{\index{variance}\label{statistics:sample_variance}
        \begin{gather}
            V(X) := \stylefrac{1}{N}\sum_{i=1}^N(x_i-\overline{x})^2
        \end{gather}
    }
    \begin{formula}
        The variance can also be rewritten in the following way:
        \begin{gather}
            \label{statistics:variance_without_sum}
            V(X) = \overline{x^2} - \overline{x}^2.
        \end{gather}
    \end{formula}
    \begin{remark}\index{Bessel!correction}
        A better estimator for the variance of a sample is given by the following formula which incorporates the \textbf{Bessel corection}:
        \begin{gather}
            \label{statistics:bessel_correction}
            \hat{s} := \stylefrac{1}{N-1}\sum_{i=1}^N(x_i - \overline{x})^2.
        \end{gather}
        See remark \ref{statistics:variance_bessel_correction} for more information.
    \end{remark}

    \newdef{Skewness}{\index{skewness}\label{statistics:skewness}
        The skewness $\gamma$ describes the asymmetry of a distribution. It is defined as the proportionality constant relating the third central moment $m_3$ and the standard deviation $\sigma$:
        \begin{gather}
            m_3 = \gamma\sigma^3.
        \end{gather}
        A positive skewness indicates a tail to the right or alternatively a median smaller than $\overline{x}$. A negative skewness indicates a median larger than $\overline{x}$.
    }
    \newdef{Pearson's mode skewness}{\index{Pearson!skewness|see{skewness}}\label{statistics:pearsons_skewness}
        \begin{gather}
            \gamma_P := \stylefrac{\overline{x} - \text{mode}}{\sigma}
        \end{gather}
    }

    \newdef{Kurtosis}{\index{kurtosis}\label{statistics:kurtosis}
        The kurtosis $c$ is an indicator for the "tailedness". It is defined as the proportionality constant relating the fourth central moment $m_4$ and the standard deviation $\sigma$:
        \begin{gather}
            m_4 = c\sigma^4.
        \end{gather}
    }
    \newdef{Excess kurtosis}{
        The excess kurtosis is defined as $c-3$. This fixes the excess kurtosis of all univariate normal distributions at 0. A positive excess is an indicator for long "fat" tails, a negative excess indicates short "thin" tails.
    }

    \newdef{Percentile}{\index{percentile}
        The $p^{th}$ percentile $c_p$ is defined as the value that is larger than $p\%$ of the measurements. The median is the $50^{th}$ percentile.
    }

    \newdef{Interquartile range}{
        The interquartile range is the difference between the upper and lower quartile ($75^{th}$ and $25^{th}$ percentile respectively).
    }

    \newdef{FWHM}{\index{FWHM}
        \nomenclature[A_FWHM]{FWHM}{Full width at half maximum}
        The \textbf{Full Width at Half Maximum} is the difference between the two values of the independent variable where the dependent variable is half of its maximum.
    }
    \begin{property}
        For Gaussian distributions the following relation exists between the FWHM and the standard deviation $\sigma$:
        \begin{gather}
            \text{FWHM} = 2.35\sigma.
        \end{gather}
    \end{property}

\subsection{Multivariate datasets}

    When working with bivariate (or even multivariate) distributions it is useful to describe the relationship between the different random variables. The following two definitions are often used.

    \newdef{Covariance}{\index{covariance}
        Let $X, Y$ be two random variables. The covariance of $X$ and $Y$ is defined as follows:
        \begin{gather}
            \label{statistics:covariance}
            \text{cov}(X,Y) := \stylefrac{1}{N}\sum_{i=1}^N(x_i-\overline{x})(y_i - \overline{y}) = \overline{xy} - \overline{x}\ \overline{y}.
        \end{gather}
        The covariance of $X$ and $Y$ is often denoted by $\sigma_{XY}$.
    }
    \begin{formula}
        The covariance and standard deviation are related by the following equality:
        \begin{gather}
            \sigma_X^2 = \sigma_{XX}.
        \end{gather}
    \end{formula}

    \newdef{Correlation coefficent}{\index{correlation}
        \begin{gather}
            \label{statistics:correlation_coefficient}
            \rho_{XY} := \stylefrac{\text{cov}(X,Y)}{\sigma_X\sigma_Y}
        \end{gather}
        The correlation coefficient is bounded to the interval $[-1,1]$. It should be noted that its magnitude is only an indicator for the linear dependence.
    }
    \begin{remark}
        For multivariate distributions the above definitions can be generalized using matrices:
        \begin{align}
            \label{statistics:covariance_matrix}
            V_{ij} &= \text{cov}(x_{(i)}, x_{(j)})\\
            \label{statistics:correlation_matrix}
            \rho_{ij} &= \rho_{(i)(j)}
        \end{align}
        where $\text{cov}(x_{(i)}, x_{(j)})$ and $\rho_{(i)(j)}$ are defined using equations \ref{statistics:covariance} and \ref{statistics:correlation_coefficient}. The following equality is generally valid:
        \begin{gather}
            \label{statistics:general_variance_formula}
            V_{ij} = \rho_{ij}\sigma_i\sigma_j.
        \end{gather}
    \end{remark}

\section{Law of large numbers}

    \begin{theorem}[Law of large numbers]\index{law of large numbers}\label{statistics:theorem:large_numbers}
        If the size $N$ of a sample tends towards infinity, then the observed frequencies tend towards the theoretical propabilities.
    \end{theorem}
    \begin{result}[Frequentist probability\footnotemark]
        \footnotetext{Also called the \textbf{empirical probability}.}
        \begin{gather}
            \label{statistics:frequentist_probability}
            P(X) := \lim_{n\rightarrow\infty}\stylefrac{f_n(X)}{n}
        \end{gather}
    \end{result}

\section{Probability densities}

    \begin{remark*}
        In the following sections and subsections, all distributions will be taken to be continuous. The formulas can be modified for use with discrete distributions by replacing the integral with a summation.
    \end{remark*}

    \newdef{Probability density function}{\index{probability!density function}
        Let $X$ be a random variable and $P(X)$ the associated probability distribution. The PDF $f(X)$ is defined as follows:
        \begin{gather}
        \label{statistics:pdf}
            P(x_1\leq X \leq x_2) := \int_{x_1}^{x_2}f(X)dX.
        \end{gather}
        An alternative definition\footnote{A more formal definition uses measure theory and in particular the Radon-Nikodym derivative from chapter \ref{chapter:lebesgue}.} is the following:
        \begin{gather}
        \label{statistics:pdf_derivative}
            f(X) := \lim_{\delta x\rightarrow 0} \stylefrac{P(x\leq X\leq x+\delta x)}{\delta x}.
        \end{gather}
    }

    \newdef{Cumulative distribution function}{\index{cumulative distribution function}
        Let $X$ be a random variable and $f(X)$ the associated PDF The cumulative distribution function $F(X)$ is defined as follows:
        \begin{gather}
            \label{statistics:cdf}
            F(x) := \int_{-\infty}^xf(X)dX.
        \end{gather}
    }
    \begin{theorem}
        Let $X$ be a random variable. Let $P(X)$ and $F(X)$ be the associated PDF and c.d.f. Using standard calculus the following equality can be proven:
        \begin{gather}
            P(x_1\leq X\leq x_2) = F(x_2) - F(x_1).
        \end{gather}
    \end{theorem}
    \begin{theorem}
        $F(X)$ is continuous if and only if $P_X(\{y\}) = 0$ for every $y\in\mathbb{R}$.
    \end{theorem}

    \begin{remark}[Normalisation]\index{normalisation}
        \begin{gather}
            \label{statistics:normalisation}
            F(\infty) = 1
        \end{gather}
    \end{remark}

    \begin{formula}
        The $p^{th}$ percentile $c_p$ can be computed as follows\footnote{This is clear from the definition of a percentile, as this implies that $F(c_p) = p$.}:
        \begin{gather}
            c_p = F^{-1}(p).
        \end{gather}
    \end{formula}

    \newdef{Parametric family}{\index{parametric family}
        A family of probability densities indexed by one or more parameters $\theta$.
    }

    \begin{formula}\index{random variable}
        Let $X$ be random variable with associated PDF $f(X)$ and let $a(X)$ be a function of $X$. The random variable $A = a(X)$ has an associated PDF $g(A)$. If the function $a(x)$ can be inverted, then $g(A)$ can be computed as follows:
        \begin{gather}
            \label{statistics:function_of_random_variable}
            g(a) = f(x(a))\left|\deriv{x}{a}\right|.
        \end{gather}
    \end{formula}

\subsection{Multivariate distributions}

    \sremark{In this section all defintions and thereoms will be given for bivariate distributions, but can be easily generalized to more random variables.}

    \newdef{Joint density}{\index{probability!density function}
        Let $X, Y$ be two random variables. The joint PDF $f_{XY}(x, y)$ is defined as follows:
        \begin{gather}
            \label{statistics:joint_pdf}
            f_{XY}(x, y)dxdy :=
            \begin{cases}
                f_x(x\in[x, x+dx])\\
                f_y(y\in[y, y+dy]).
            \end{cases}
        \end{gather}
    }
    \remark{Because $f_{XY}$ is a probability density, the normalisation condition \ref{statistics:normalisation} should be fulfilled.}

    \newdef{Conditional density}{\index{probability!density function}
        The conditional PDF of $X$ when $Y$ has the value y is given by the following formula:
        \begin{gather}
            \label{statistics:conditional_pdf}
            g(x|y) = \stylefrac{f_{XY}(x,y)}{f_Y(y)}
        \end{gather}
        where we should pay attention to the remark made when we defined \ref{prob:formal_conditional}.
    }
    \result{If $X$ and $Y$ are independent, then by remark \ref{prob:remark_independence} the marginal p.d.f is equal to the conditional PDF.}

    \begin{theorem}[Bayes' theorem]\index{Bayes}
        The conditional PDF can be computed without prior knowledge of the joint PDF:
        \begin{gather}
            \label{statistics:theorem:bayes}
            g(x|y) = \stylefrac{h(y|x)f_X(x)}{f_Y(y)}.
        \end{gather}
    \end{theorem}
    \sremark{This theorem is the statistical (random variable) analogue of theorem \ref{prob:theorem:bayes}.}

    \begin{formula}
        Let $Z = XY$ with $X, Y$ two independent random variables. The distribution $f(z)$ is given by
        \begin{gather}
            f(z) = \int_{-\infty}^{+\infty}g(x)h(z/x)\frac{dx}{|x|} = \int_{-\infty}^{+\infty}g(z/y)h(y)\frac{dy}{|y|}.
        \end{gather}
    \end{formula}
    \begin{result}
        Taking the Mellin transform \ref{transforms:mellin} of both the positive and negative part of the above integrand (to be able to handle the absolute value) gives the following relation:
        \begin{gather}
            \mathcal{M}\{f\} = \mathcal{M}\{g\}\mathcal{M}\{h\}.
        \end{gather}
    \end{result}
    \begin{formula}
        Let $Z = X + Y$ with $X, Y$ two independent random variables. The distribution $f(z)$ is given by the convolution of $g(x)$ and $h(y)$:
        \begin{gather}
            f(z) = \int_{-\infty}^{+\infty}g(x)h(z-x)dx = \int_{-\infty}^{+\infty}g(z-y)h(y)dy.
        \end{gather}
    \end{formula}

\subsection{Important distributions}

    \newformula{Uniform distribution}{\index{distribution!uniform}
        \begin{gather}\label{statistics:uniform_distr}
            P(x;a,b) :=
            \begin{cases}
                \stylefrac{1}{b-a}&a\leq x\leq b\\
                0&\text{elsewhere}
            \end{cases}
        \end{gather}
        \begin{gather}
            E(x) = \stylefrac{a+b}{2}
        \end{gather}
        \begin{gather}
            V(x) = \stylefrac{(b-a)^2}{12}
        \end{gather}
    }

    \begin{formula}[Normal distribution]\index{distribution!normal}\index{Gauss!distribution}
        \begin{gather}
            \label{statistics:normal_distr}
            \mathcal{G}(x;\mu, \sigma) := \stylefrac{1}{\sqrt{2\pi}\sigma}e^{-\stylefrac{(x-\mu)^2}{2\sigma^2}}
        \end{gather}
        This distribution is also called a \textbf{Gaussian distribution}.
    \end{formula}

    \newformula{Standard normal distribution}{
        \begin{gather}
            \label{statistics:standard_normal_distr}
            \mathcal{N}(z) := \stylefrac{1}{\sqrt{2\pi}}e^{-\stylefrac{z^2}{2}}
        \end{gather}
    }
    \remark{Every Gaussian distribution can be rewritten as a standard normal distribution by setting $Z = \stylefrac{X-\mu}{\sigma}$.}
    \remark{The CDF of the standard normal distribution is given by the error function: $F(z) = \text{Erf}(z)$.}

    \begin{formula}[Exponential distribution]\index{distribution!exponential}
        \begin{gather}
            \label{statistics:exponential_distr}
            P(x;\tau) := \stylefrac{1}{\tau}e^{-\stylefrac{x}{\tau}}
        \end{gather}
        \begin{gather}
            E(x) = \tau
        \end{gather}
        \begin{gather}
            V(x) = \tau^2
        \end{gather}
    \end{formula}
    \begin{property}\index{memory}\label{statistics:theorem:memoryless_exponential_distribution}
        The exponential distribution is \textbf{memoryless}:
        \begin{gather}
            P(X>x_1+x_2|X>x_2) = P(X>x_1)
        \end{gather}
    \end{property}

    \newformula{Bernoulli distribution}{\index{distribution!Bernoulli}\index{Bernoulli|seealso{distribution}}
        A random variable that can only take 2 possible values is described by a Bernoulli distribution. When the possible values are 0 and 1, with respective chances $p$ and $1-p$, the distribution is given by
        \begin{gather}
            \label{statistics:bernoulli_distr}
            P(x;p) = p^x(1-p)^{1-x}.
        \end{gather}
        \begin{gather}
            E(x) = p
        \end{gather}
        \begin{gather}
            V(x) = p(1-p)
        \end{gather}
    }

    \newformula{Binomial distribution}{\index{distribution!binomial}
        A process with $n$ identical independent trials, all Bernoulli processes $P(x; p)$, is described by a binomial distribution:
        \begin{gather}
            \label{statistics:binomial_distr}
            P(r;p,n) = p^r(1-p)^{n-r}\stylefrac{n!}{r!(n-r)!}.
        \end{gather}
        \begin{gather}
            E(r) = np
        \end{gather}
        \begin{gather}
            V(r) = np(1-p)
        \end{gather}
    }

    \begin{formula}[Poisson distribution]\index{distribution!Poisson}
        A process with known possible outcomes but an unknown number of events is described by a Poisson distribution $P(r;\lambda)$ with $\lambda$ the average expected number of events.
        \begin{gather}
            \label{statistics:poisson_distr}
            P(r;\lambda) = \stylefrac{e^{-\lambda}\lambda^r}{r!}.
        \end{gather}
        \begin{gather}
            E(r) = \lambda
        \end{gather}
        \begin{gather}
            V(r) = \lambda
        \end{gather}
    \end{formula}
    \begin{property}
        If two Poisson processes $P(r;\lambda_a)$ and $P(r;\lambda_b)$ occur simultaneously and if there is no distinction between the two, then the probability of $r$ events is also described by a Poisson distribution with average $\lambda_a+\lambda_b$.
    \end{property}
    \result{The number of events coming from $A$ is given by a binomial distribution $P(r_a;\Lambda_a, r)$ where $\Lambda_a = \stylefrac{\lambda_a}{\lambda_a + \lambda_b}$.}
    \begin{remark}
        For large values of $\lambda$ ($\lambda\rightarrow\infty$), the Poisson distribution $P(r;\lambda)$ can be approximated by a Gaussian distribution $\mathcal{G}(x;\lambda,\sqrt{\lambda})$.
    \end{remark}

    \begin{formula}[$\chi^2$ distribution]\index{distribution!$\chi^2$}
        The sum of $k$ squared independent (standard) normally distributed random variables $Y_i$ defines the random variable:
        \begin{gather}
            \chi^2_k := \sum_{i=1}^kY_i^2
        \end{gather}
        where $k$ is said to be the number of \textbf{degrees of freedom}.
        \begin{gather}
            \label{statistics:chi_squared_distr}
            P(\chi^2;n) = \stylefrac{\chi^{n-2}e^{-\stylefrac{\chi^2}{2}}}{2^{\frac{n}{2}}\Gamma\left(\stylefrac{n}{2}\right)}
        \end{gather}
    \end{formula}
    \remark{Due to the CLT the $\chi^2$ distribution approximates a Guassian distribution for large $k$: $P(\chi^2;k)\xrightarrow{k>30}\mathcal{G}(\sqrt{2\chi^2};\sqrt{2k-1},1)$

    \newformula{Student-t distribution}{\index{distribution!Student-t}
        The Student-t distribution describes a sample with estimated standard deviation $\hat{\sigma}$:
        \begin{gather}
            \label{statistics:student_t_distr}
            P(t;n) = \stylefrac{\Gamma\left(\frac{n+1}{2}\right)}{\sqrt{n\pi}\ \Gamma\left(\frac{n}{2}\right)\left(1 + \frac{t^2}{n}\right)^{\frac{n+1}{2}}}
        \end{gather}
        where
        \begin{gather}
            t := \stylefrac{(x-\mu) / \sigma}{\hat{\sigma}/\sigma} = \stylefrac{z}{\sqrt{\chi^2/n}}.
        \end{gather}
    }
    \sremark{The significance of a difference between the sample mean $\overline{x}$ and the true mean $\mu$ is smaller due to the (extra) uncertainty of the estimated standard deviation.}

    \newformula{Cauchy distribution\footnotemark}{\index{Cauchy!distribution}\index{Breit-Wigner|see{Cauchy distribution}}
        \footnotetext{Also known (especially in particle physics) as the \textbf{Breit-Wigner} distribution.}
        The general form $f(x; x_0, \gamma)$ is given by
        \begin{gather}
            \label{stat:cauchy_distribution}
            f(x; x_0, \gamma) := \frac{1}{\pi}\frac{\gamma}{(x - x_0)^2 + \gamma^2}.
        \end{gather}
        The characteristic function \ref{prob:characteristic_function} is given by
        \begin{gather}
            E\left[e^{itx}\right] = e^{ix_0t - \gamma|t|}.
        \end{gather}
    }
    \begin{property}
        Both the mean and variance of the Cauchy distribution are undefined.
    \end{property}

\section{Central limit theorem (CLT)}

    \begin{theorem}[Central limit theorem]\index{central limit theorem}\label{statistics:theorem:CLT}
        A sum of $n$ independent random variables $X_i$ has the following properties:
        \begin{itemize}
            \item $\mu = \sum_i\mu_i$
            \item $V(X) = \sum_iV_i$
            \item The sum will be approximately (!!) normally distributed.
        \end{itemize}
    \end{theorem}
    \begin{remark}
        If the random variables are not independent, property 2 will not be fulfilled.
    \end{remark}
    \remark{The sum of Gaussians will be exactly Gaussian.}

\section{Errors}

    \newdef{Systematic error}{
        Errors that always have the same influence (e.g. they shift all values in the same way), that are not necessarilly independent of eachother and that cannot be directly inferred from the measurements.
    }

\subsection{Different measurement types}

    When performing a sequence of measurements $x_i$ with different variances $\sigma_i^2$, it is impossible to use the arithmetic mean \ref{statistics:arithmetic_mean} in a meaningful way because the measurements are not of the same type. Therefore it is also impossible to apply the CLT \ref{statistics:theorem:CLT}.

    These problems can be resolved by the using the weighted mean \ref{statistics:weighted_mean}:
    \begin{gather}
        \overline{x} := \stylefrac{\sum_i\frac{x_i}{\sigma_i^2}}{\sum_i\frac{1}{\sigma_i^2}}.
    \end{gather}
    The variation of the weighted mean is given by
    \begin{gather}
        \label{statistics:weighted_mean_variance}
        V(\overline{x}) := \stylefrac{1}{\sum_i\sigma_i^{-2}}.
    \end{gather}

\subsection{Propagation of errors}

    \begin{formula}\index{variance}
        Let $X$ be random variable with variance $V(x)$. The variance of a linear function $f(X) = aX + b$ is given by
        \begin{gather}
            \label{statistics:variance_linear_function}
            V(f) = a^2V(x).
        \end{gather}
    \end{formula}
    \begin{formula}
        Let $X$ be random variable with \textbf{small} (!!) variance $V(x)$. The variance of a general function $f(X)$ is given by
        \begin{gather}
            \label{statistics:variance_general_function}
            V(f) \approx \left(\deriv{f}{x}\right)^2V(x).
        \end{gather}
    \end{formula}
    \begin{result}
        The correlation coefficient $\rho$ (\ref{statistics:correlation_coefficient}) of a random variable $X$ and a \textbf{linear} function of $X$ is independent of $\sigma_x$ and is always equal to $\pm1$.
    \end{result}

    \newformula{Law of error propagation}{\index{propagation}
        Let $\vec{X}$ be a set of random variables with \textbf{small} variances. The variance of a general function $f(\vec{X})$ is given by
        \begin{gather}
            \label{statistics:error_propagation}
            V(f) = \sum_p\left(\pderiv{f}{X_{(p)}}\right)^2V(X_{(p)}) + \sum_p\sum_{q\neq p}\left(\pderiv{f}{X_{(p)}}\right)\left(\pderiv{f}{X_{(q)}}\right)\text{cov}(X_{(p)}, X_{(q)}).
        \end{gather}
    }

    \newdef{Fractional error}{\index{fractional error}
        Let $X, Y$ be two independent random variables. The standard deviation of $f(x, y) = xy$ is given by the fractional error:
        \begin{gather}
            \label{statistics:fractional_error}
            \left(\stylefrac{\sigma_f}{f}\right)^2 = \left(\stylefrac{\sigma_x}{x}\right)^2 + \left(\stylefrac{\sigma_y}{y}\right)^2.
        \end{gather}
    }
    \remark{The fractional error of quantity is equal to the fractional error of the reciprocal of that quantity.}

    \begin{property}
        Let $X$ be a random variable. The error of the logarithm of $X$ is equal to the fractional error of $X$.
    \end{property}

    \newformula{Covariance of functions}{\index{covariance}
        \begin{gather}
            \label{statistics:covariance_functions}
            \text{cov}(f_1, f_2) = \sum_p\sum_q\left(\pderiv{f_1}{X_{(p)}}\right)\left(\pderiv{f_2}{X_{(q)}}\right)\text{cov}(X_{(p)}, X_{(q)})
        \end{gather}
    }
    \begin{result}
        Let $\vector{f} = \{f_1,\ldots,f_k\}$. The covariance matrix $V_f$ of the $k$ functions is given by
        \begin{gather}
            V_f = GV_XG^T
        \end{gather}
        where $G$ is the Jacobian matrix of $\vector{f}$.
    \end{result}

\section{Estimators}

    \newdef{Estimator}{\index{estimator}
        An estimator is a procedure that, given a sample, produces a numerical value for a property of the parent population.
    }

\subsection{General properties}

    \newdef{Consistency}{\index{consistency}
        An estimator $\hat{a}$ is said to be consistent if
        \begin{gather}
            \label{statistics:consistency}
            \lim_{N\rightarrow\infty}\hat{a} = a.
        \end{gather}
    }
    \newdef{Unbiased estimator}{\label{statistics:unbiased_estimator}
        An estimator $\hat{a}$ is said to be unbiased if
        \begin{gather}
            \langle\hat{a}\rangle = a.
        \end{gather}
    }

    \newdef{Bias}{\index{bias}
        The bias of an estimator is defined as follows:
        \begin{gather}
            \label{statistics:bias}
            B(\hat{a}) := |\langle\hat{a}\rangle - a|.
        \end{gather}
    }

    \newdef{Mean squared error}{
        The mean squared error (MSE) of an estimator is defined as follows:
        \begin{gather}
            \label{statistics:mean_squared_error}
            \Upsilon(\hat{a}) := B(\hat{a})^2 + V(\hat{a}).
        \end{gather}
    }
    \remark{If an estimator is unbiased, the MSE is equal to the variance of the estimator.}

\subsection{Fundamental estimators}

    \begin{property}[Unbiased mean]
        The sample mean \ref{statistics:arithmetic_mean} is an unbiased estimator for the population mean:
        \begin{gather}
            \langle\overline{x}\rangle = \mu.
        \end{gather}
        Furthermore, it is also consistent. Both of these properties are a consequence of the CLT.
    \end{property}
    \begin{formula}[Standard error of the mean]\index{standard!error}
        Using the Bienaym\'e formula \ref{prob:bienayme} one can show that the standard error, i.e. the standard deviation of the sample mean, is given by the following formula:
        \begin{gather}
            \label{statistics:standard_error}
            V(\overline{x}) = \stylefrac{\sigma^2}{N}.
        \end{gather}
    \end{formula}

    \newformula{Variance estimator for known mean}{\index{variance!estimator}
        If the true mean $\mu$ is known then a consistent and unbiased estimator for the variance is given by:
        \begin{gather}
            \widehat{V[x]} = \stylefrac{1}{N}\sum_{i=1}^N(x_i-\mu)^2.
        \end{gather}
    }
    \newformula{Variance estimator for unknown mean}{\index{Bessel!correction}\label{statistics:variance_bessel_correction}
        If the true mean $\mu$ is unknown and the sample mean has been used to estimate $\mu$, then a consistent and unbiased estimator is given by\footnote{The modified factor $\stylefrac{1}{N-1}$ is called the Bessel correction. It corrects the bias of the estimator given by the sample variance \ref{statistics:sample_variance}. The consistency, however, is guaranteed by the CLT.}:
        \begin{gather}
            s^2 = \stylefrac{1}{N-1}\sum_{i=1}^N(x_i-\overline{x})^2.
        \end{gather}
    }

\subsection{Estimation error}

    \newformula{Variance of estimator of variance}{
        \begin{gather}
            V\left(\widehat{V[x]}\right) =  \stylefrac{(N-1)^2}{N^3}<(x - <x>)^4> - \stylefrac{(N-1)(N-3)}{N^3}<(x - <x>)^2>^2
        \end{gather}
    }
    \newformula{Variance of estimator of standard deviation}{
        \begin{gather}
            V\left(\widehat{\sigma}\right) = \stylefrac{1}{4\sigma^2}V\left(\widehat{V[x]}\right)
        \end{gather}
    }
    \begin{remark}
        The previous result is a little odd, as one has to know the true standard deviation to compute the variance of the estimator. This problem can be solved in two ways. Or a value (hopefully close to the real one) inferred from the sample is used as an estimator or a theoretical one is used in the design phase of an experiment to see what the possible outcomes are.
    \end{remark}

\subsection{Likelihood function}

    \newdef{Likelihood}{\index{likelihood}\label{likelihood}
        The likelihood $\mathcal{L}(a;\vec{x})$ is the probability to find a set of measurements $\vector{x} = \{x_1,\ldots,x_N\}$ given a distribution $P(X;a)$:
        \begin{gather}
            \label{statistics:likelihood}
            \mathcal{L}(a;\vector{x}) = \prod_{i=1}^NP(x_i;a).
        \end{gather}
    }
    \newdef{Log-likelihood}{
        \begin{gather}
            \label{statistics:log_likelihood}
            \log\mathcal{L}(a;\vector{x}) = \sum_i\ln P(x_i;a)
        \end{gather}
    }

    \begin{property}
        The expectation value of an estimator $\hat{a}$ is given by
        \begin{gather}
            \langle\hat{a}\rangle = \int \hat{a}\mathcal{L}(\hat{a};X)dX.
        \end{gather}
    \end{property}

    \begin{theorem}[Cramer-Rao bound]\index{Cramer-Rao}\index{minimum!variance bound}
        The variance of an \textbf{unbiased} estimator has a lower bound: the Cramer-Rao bound or \textbf{minimum variance bound (MVB)}:
        \begin{gather}
            \label{statistics:theorem:minimum_variance_bound}
            V(\hat{a})\geq\stylefrac{1}{\left<\left(\deriv{\ln\mathcal{L}}{a}\right)^2\right>}.
        \end{gather}
        For a biased estimator with bias $b$ the MVB takes on the following form:
        \begin{gather}
            \label{statistics:theorem:biased_minimum_variance_bound}
            V(\hat{a})\geq\stylefrac{\left(1+\deriv{b}{a}\right)^2}{\left<\left(\deriv{\ln\mathcal{L}}{a}\right)^2\right>}.
        \end{gather}
    \end{theorem}
    \begin{remark}
        \begin{gather}
            \left<\left(\deriv{\ln\mathcal{L}}{a}\right)^2\right> = -\left<\mderiv{2}{\ln\mathcal{L}}{a}\right>
        \end{gather}
    \end{remark}

    \newdef{Fisher information}{\index{Fisher!information}
        \begin{gather}
            \label{statistics:fisher_information}
            I_X(a) := \left<\left(\deriv{\ln\mathcal{L}}{a}\right)^2\right> = N\int\left(\deriv{\ln P}{a}\right)^2PdX
        \end{gather}
        Using this definition we can rewrite the Cramer-Rao inequality as follows:
        \begin{gather}
            V(\hat{a})\geq I_X(a).
        \end{gather}
    }

    \newdef{Finite-sample efficiency}{\index{efficient}
        An unbiased estimator is said to be (finite-sample) efficient if it saturates the Cramer-Rao bound. In general the \textbf{efficiency} of (unbiased) estimators is defined through the Cramer-Rao bound as follows:
        \begin{gather}
            e(\hat{a}) := \frac{I_X(a)^{-1}}{V(\hat{a})}.
        \end{gather}
    }

\subsection{Maximum likelihood estimation}

    From definition \ref{likelihood} it follows that the estimator $\hat{a}_{MLH}$ that makes the given measurements most probable is the value of $a$ for which the likelihood function is maximal. It is therefore not the most probable estimator.

    Using Bayes' theorem we find $P(a;x) = P(x;a)\frac{P(a)}{P(x)}$. The prior probability $P(x)$ is fixed since the values $x_i$ are given by our measurement and hence does not vary and the probability $P(a)$ is generally assumed to be a uniform distribution if there is no prior knowledge about $a$. It follows that $P(a;x)$ and $P(x;a)$ are proportional and hence the logarithms of these functions differ only by an additive constant. This leads us to following method for finding an estimator $\hat{a}$:

    \newmethod{Maximum likelihood estimator}{\index{likelihood!estimator}
        The maximum likelihood estimator $\hat{a}$ is obtained by solving the following equation:
        \begin{gather}
            \label{statistics:maximum_likelihood_estimator}
            \left.\deriv{\ln\mathcal{L}}{a}\right|_{a=\hat{a}} = 0.
        \end{gather}
    }
    \remark{MLH estimators are mostly consistent but often biased.}
    \begin{property}
        MLH estimators are invariant under parameter transformations.
    \end{property}
    \result{The invariance implies that the two estimators $\hat{a}$ and $\widehat{f(a)}$ cannot both be unbiased at the same time.}

    \begin{property}
        Asymptotically every consistent estimator becomes unbiased and efficient.
    \end{property}

\subsection{Least squares estimation}

    To fit a (parametric) function $y = f(x;a)$ to a set of 2 variables $(x, y)$, where the $x$ values are exact and the $y$ values have an uncertainty $\sigma_i$, we can use the following method:
    \newmethod{Least squares}{$ $\index{least squares}
        \begin{enumerate}
            \item For every event $(x_i, y_i)$ define the residual $d_i := y_i - f(x_i;a)$.
            \item Determine the $\chi^2$-statistic (analytically):
                \begin{gather}
                    \chi^2 := \sum_i\stylefrac{d_i^2}{f_i}
                \end{gather}
                where $f_i = f(x_i;a)$.
            \item Find the most probable value of $\hat{a}$ by solving the equation \[\deriv{\chi^2}{a} = 0.\]
        \end{enumerate}
    }
    \begin{property}
        The optimal (minimal) $\chi^2$ is (asymptotically) distributed according to a $\chi^2$-distribution \ref{statistics:chi_squared_distr} $P(\chi^2;n)$. The number of degrees of freedom $n$ is equal to the number of events $N$ minus the number of fitted parameters $k$. (See more in section \ref{statistics:section:chi_squared_test}.)
    \end{property}

    \newformula{Linear fit}{\index{linear!fit}
        When all uncertainties $\sigma_i$ are equal, the slope $\hat{m}$ and intercept $\hat{c}$ are given by the following formulas:
        \begin{align}
            \label{statistics:least_squares_slope}
            \hat{m} &= \stylefrac{\overline{xy} - \overline{x}\ \overline{y}}{\overline{x^2} - \overline{x}^2} = \stylefrac{\text{cov}(x, y)}{V(x)}\\
            \label{statistics:least_squares_intercept}
            \hat{c} &= \overline{y} - \hat{m}\overline{x} = \stylefrac{\overline{x^2} - \overline{x}\ \overline{y}}{\overline{x^2} - \overline{x}^2}.
        \end{align}
    }
    \remark{The equation $\overline{y} = \hat{c} + \hat{m}\overline{x}$ means that the linear fit passes through the center of mass $(\overline{x}, \overline{y})$.}

    \newformula{Errors of linear fit}{
        \begin{align}
            \label{statistics:least_squares_slope_variance}
            V(\hat{m}) &= \stylefrac{1}{N(\overline{x^2} - \overline{x}^2)}\sigma^2\\&\nonumber\\
            \label{statistics:least_squares_intercept_variance}
            V(\hat{c}) &= \stylefrac{\overline{x^2}}{N(\overline{x^2} - \overline{x}^2)}\sigma^2\\&\nonumber\\
            \label{statistics:least_squares_linear_fit_covariance}
            \text{cov}(\hat{m}, \hat{c}) &= \stylefrac{-\overline{x}}{N(\overline{x^2} - \overline{x}^2)}\sigma^2
        \end{align}
    }
    \begin{remark}
        When the uncertainties $\sigma_i$ are different, the arithmetic means have to be replaced by weighted means and the quantity $\sigma^2$ has to be replaced by its weighted variant:
        \begin{gather}
            \overline{\sigma^2} = \stylefrac{\sum\sigma_i^2/\sigma_i^2}{\sum1/\sigma_i^2} = \stylefrac{N}{\sum\sigma_i^{-2}}.
        \end{gather}
    \end{remark}

    The least squares method is very useful to fit data that has been grouped in bins (histograms):
    \newmethod{Binned least squares}{$ $
        \begin{enumerate}
            \item $N$ i.i.d. events with distributions $P(X;a)$ divided in $N_B$ intervals where the interval $j$ is centered on the value $x_j$, has a width $W_j$ and contains $n_j$ events.
            \item The ideally expected number of events in the $j^{th}$ interval: $f_j = NW_jP(x_j;a)$.
            \item The real number of events has a Poisson distribution: $\overline{n}_j = \sigma_j^2 = f_j$.
            \item Define the binned $\chi^2$ as \[\chi^2 := \displaystyle\sum_i^{N_B}\stylefrac{(n_i - f_i)^2}{f_i^2}.\]
        \end{enumerate}
    }

\section{Confidence intervals}\index{confidence}

    The true value of a parameter $\varepsilon$ can never be known exactly. However, it is possible to construct an interval $I$ in which this value should lie with a certain confidence $C$.
    \begin{example}
        Let $X$ be a random variable with distribution $\mathcal{N}(x;\mu,\sigma)$. The measurement $x$ lies in the interval $[\mu - 1.96\sigma, \mu+1.96\sigma]$ with 95\% \underline{probability}. The true value $\mu$ lies in the interval $[x - 2\sigma, x+2\sigma]$ with 95\% \underline{confidence}.
    \end{example}
    \begin{remark*}
        In the previous example there are some Bayesian assumptions: all possible values (left or right side of peak) are given the same probability due to the Gaussian distribution. If one removes this symmetry condition a more careful approach is required. Furthermore, the apparent symmetry between the uncertainty and confidence levels is only valid for Gaussian distributions.
    \end{remark*}

\subsection{Interval types}

    \newdef{Two-sided confidence interval}{
        \begin{gather}
            \label{statistics:two_sided_interval}
            P(x_-\leq X\leq x_+) = \int_{x_-}^{x_+}P(x)dx = C
        \end{gather}
        There are three possible (often used) two-sided intervals:
        \begin{itemize}
            \item \textbf{symmetric interval}: $x_+ - \mu = \mu - x_-$
            \item \textbf{shortest interval}: $|x_+ - x_-|$ is minimal
            \item \textbf{central interval}: $\int_{-\infty}^{x_-}P(x)dx = \int_{x_+}^\infty P(x)dx = \stylefrac{1-C}{2}$
        \end{itemize}
        The central interval is the most widely used confidence interval.
    }
    \remark{For Gaussian distributions these three definitions are equivalent.}

    \newdef{One-sided confidence interval}{
        \begin{align}
            \label{statistics:one_sided_interval1}
            P(x \geq x_-) &= \int_{x_-}^{+\infty}P(x)dx = C\\
            \label{statistics:one_sided_interval2}
            P(x \leq x_+) &= \int_{-\infty}^{x_+}P(x)dx = C
        \end{align}
    }
    \remark{For a discrete distribution it is often impossible to find integers $x_{\pm}$ such that the real value lies with exact confidence $C$ in the interval $[x_-, x_+]$.}

    \newdef{Discrete central confidence interval}{
        \begin{align}
            \label{statistics:central_discreteInterval_lower_bound}
            x_- &= \max_\theta\left[\sum_{x=0}^{\theta - 1}P(x;X)\right]\leq\stylefrac{1-C}{2}\\
            \label{statistics:central_discreteInterval_upper_bound}
            x_+ &= \min_\theta\left[\sum_{x=\theta + 1}^{+\infty}P(x;X)\right]\leq\stylefrac{1-C}{2}
        \end{align}
    }

\subsection{General construction}

    For every value of the true parameter $X$ it is possible to construct a confidence interval. This leads to the construction of two functions $x_-(X)$ and $x_+(X)$. The 2D diagram obtained by plotting $x_-(X)$ and $x_+(X)$ with the $x$-axis horizontally and $X$-axis vertically is called the \textbf{confidence region}.
    \begin{method}
        Let $x_0$ be a point estimate of the parameter $X$. From the confidence region it is possible to infere a confidence interval $[X_-(x), X_+(x)]$, where the upper limit $X_+$ is not the limit such that there is only a $\stylefrac{1-C}{2}$ chance of having a true parameter $X\geq X_+$, but the limit such that if the true parameter $X\geq X_+$ then there is a chance of $\stylefrac{1-C}{2}$ to have a measurement $x_0$ or smaller.
    \end{method}

\subsection{Interval for a sample mean}

    \newformula{Interval with known variance}{
        If the sample size is large enough, the real distribution is unimportant, because the CLT ensures a Gaussian distribution of the sample mean $\overline{X}$. The $\alpha$-level confidence interval such that $P(-z_{\alpha/2} < Z < z_{\alpha/2})$ with $Z = \stylefrac{\overline{X} - \mu}{\sigma/\sqrt{N}}$ is given by
        \begin{gather}
            \left[\overline{X} - z_{\alpha/2}\stylefrac{\sigma}{\sqrt{N}},\overline{X} + z_{\alpha/2}\stylefrac{\sigma}{\sqrt{N}}\right].
        \end{gather}
    }
    \remark{If the sample size is not sufficiently large, the measured quantity must follow a normal distribution.}

    \newformula{Interval with unknown variance}{
        To account for the uncertainty of the estimated standard deviation $\hat{\sigma}$, the student-t distribution \ref{statistics:student_t_distr} is used instead of a Gaussian distribution to describe the sample mean $\overline{X}$. The $\alpha$-level confidence interval is given by
        \begin{gather}
            \left[\overline{X} - t_{\alpha/2;(n-1)}\stylefrac{s}{\sqrt{N}},\overline{X} + t_{\alpha/2;(n-1)}\stylefrac{s}{\sqrt{N}}\right].
        \end{gather}
        where $s$ is the estimated standard deviation \ref{statistics:variance_bessel_correction}.
    }

    \newformula{Wilson score interval}{\index{Wilson score interval}
        For a sufficiently large sample, a sample proportion $\hat{P}$ is approximately Gaussian distributed with expectation value $\pi$ and variance $\frac{\pi(\pi-1)}{N}$. The $\alpha$-level confidence interval is given by
        \begin{gather}
            \left[\stylefrac{(2N\hat{P} + z^2_{\alpha/2}) - z_{\alpha/2}\sqrt{z^2_{\alpha/2} + 4N\hat{P}(1 - \hat{P})}}{2(N + z^2_{\alpha/2})},\stylefrac{(2N\hat{P} + z^2_{\alpha/2}) + z_{\alpha/2}\sqrt{z^2_{\alpha/2} + 4N\hat{P}(1 - \hat{P})}}{2(N + z^2_{\alpha/2})}\right].
        \end{gather}
    }
    \sremark{The expectation value and variance are these of a binomial distribution \ref{statistics:binomial_distr} with $r = X/N$.}

\section{Hypothesis testing}\index{hypothesis}

    \newdef{Simple hypothesis}{
        A hypothesis is called simple if the distribution is fully specified.
    }
    \newdef{Composite hypothesis}{
        A hypothesis is called composite if the distribution is given relative to some parameter values.
    }

\subsection{Testing}

    \newdef{Type I error}{\index{error}
        Rejecting a true null hypothesis.
    }
    \newdef{Type II error}{
        Accepting a false null hypothesis.
    }

    \newdef{Significance}{\index{significance}
        The probability of making a type I error:
        \begin{gather}
            \label{statistics:significance}
            \alpha = \int P_{H_0}(x)dx.
        \end{gather}
    }
    \begin{property}
        Let $a_1 > a_2$. An $a_2$-level test is also significant at the $a_1$-level.
    \end{property}
    \remark{For discrete distributions it is not always possible to achieve an exact level of significance.}
    \sremark{Type I errors occur occasionally. They cannot be prevented, they should however be controlled.}

    \newdef{Power}{\index{power}
        The probability of not making a type II error:
        \begin{gather}
            \label{statistics:power}
            \beta = \int P_{H_1}(x)dx\qquad\rightarrow\qquad\text{power: }1-\beta.
        \end{gather}
    }

    \begin{remark}
        A good test is a test with a small significance and a large power. The propabilities $P_{H_0}(x)$ and $P_{H_1}(x)$ should be as different as possible.
    \end{remark}

    \begin{theorem}[Neyman-Pearson test]\index{Neyman-Pearson test}
        The following test is the most powerful test at significance level $\alpha$ for a threshold $\eta$:

        The null hypothesis $H_0$ is rejected in favour of the alternative hypothesis $H_1$ if the likelihood ratio $\Lambda$ satisfies the following condition:
        \begin{gather}\index{likelihood!ratio}
            \label{statistics:likelihood_ratio}
            \Lambda(x) = \stylefrac{L(x|H_0)}{L(x|H_1)} \leq \eta
        \end{gather}
        where $P(\Lambda(x)\leq\eta|H_0) = \alpha $
    \end{theorem}
    \sremark{In some references the reciprocal of $\Lambda(x)$ is used as the definition of the likelihood ratio.}

\section{Comparison tests}

    \newdef{McNemar test}{\index{test!McNemar}
        Consider two models constructed for a given data set. Construct the contingency table describing the number of true positives and true negatives for both models:
        \begin{gather}
            \begin{array}{c||c|c}
                &\text{TP (model 1)}&\text{TN (model 1)}\\
                \hline
                \text{TP (model 2)}&a&b\\
                \hline
                \text{TN (model 2)}&c&d
            \end{array}
        \end{gather}
        The null hypothesis of the McNemar test is that there is no significant difference between the predictive power of the model, i.e. $ p_a+p_c = p_a+p_b \land p_b+p_d = p_c+p_d$ where $p_\alpha$ indicates the proportion of the variable $\alpha$. By noting that the diagonal values are redundant in this description one can write the hypotheses more concisely: \[H_0: b=c\]\[H_1: b\neq c.\] The test statistic is the McNemar chi-squared statistic: $\chi^2 = \frac{(b-c)^2}{b+c}$. When the values of $b$ and $c$ are large enough ($>25$) one can approximate this distribution by an ordinary $\chi^2$-distribution with 1 degree of freedom.
    }
    \begin{remark}[Edwards' correction]\index{Edwards' correction}
        It is common to apply a continuity correction (similar to the \textit{Yates-correction} for the ordinary chi-squared test):
        \begin{gather}
           \chi^2 := \frac{(|b-c|-1)^2}{b+c}.
        \end{gather}
        This follows from the fact that for small $b, c$ the exact $p$-values should be compared with a binomial test which compares $b$ to $b+c$ (note the factor 2):
        \begin{gather}
           p = 2\sum_{i=b}^{b+c}\binom{b+c}{i}0.5^i(1 - 0.5)^{b+c-i}.
        \end{gather}
    \end{remark}

    \newdef{Wilcoxon signed-rank test}{\index{test!Wilcoxon}
        Consider a paired data sample, i.e. two dependent data samples for which the $i^{th}$ entries are paired together. This test checks if the population means are different. The test statistic is defined as follows:

        \qquad First one calculates the differences $d_i$ and ranks their absolute values (ties are assigned an average rank). Then one calculates the sums of the ranks $R_+, R_-$ for positive differences and negative differences and take smallest of these: $T=\min(R_+, R_-)$. For small data samples ($n<25$) one can look for tables listing critical values for $T$. For larger data samples one can (approximately) use a standard normal distribution with statistic \[z := \frac{T-\frac{1}{4}n(n+1)}{\sqrt{\frac{1}{24}n(n+1)(2n+1)}}.\]
    }
    \begin{remark}
        The major benefit of this test over a signed $t$-test is that the Wilcoxon test does not require the data samples to be drawn from a normal distribution. However in the case where the assumptions for a paired $t$-test are met, the $t$-test is more powerful.
    \end{remark}

    \newdef{Family-wise error}{\index{error}
        Given a collection of hypothesis tests, the family-wise error is defined as the probability of making at least one type-I error.
    }

    \newdef{Friedman test}{\index{test!Friedman}
        Consider $k$ models tested on $N$ data sets. For every data set one ranks the models according to decreasing performance. For every $i\leq k$ one defines the average rank $R_i=\frac{1}{N}\sum_{j\leq N}r^j_i$ where $r^j_i$ is the rank of the $i^{th}$ model on the $j^{th}$ data set. Under the null hypothesis (all models perform equally well) the average ranks should be the same for all models.

        The Friedman statistic
        \begin{gather}
            \chi^2_F := \frac{12N}{k(k+1)}\left(\sum_{i\leq k}R_i^2 - \frac{k(k+1)^2}{4}\right)
        \end{gather}
        follows a $\chi^2$-distribution with $k-1$ degrees of freedom when $N>10$ and $k>5$. For smaller values of these parameters one can look up the exact critical values in the literature.
    }
    \begin{remark}
        It was shown that the original Friedman test is rather conservative and that a better statistic is
        \begin{gather}
            F := \frac{(N-1)\chi^2_F}{N(k+1)-\chi^2_F}.
        \end{gather}
        This follows a $F$-distribution with $k-1$ and $(N-1)(k-1)$ degrees of freedom. As a further remark we mention that the (nonparametric) Friedman test is weaker than (parametric) repeated measures ANOVA whenever the assumptions for the latter hold (similar to the case of the Wilcoxon signed-rank test).
    \end{remark}

\subsection{Post-hoc tests}

    After successfully using one of the multi-model tests from the previous section to reject the null hypothesis of equal performance one is often interested in exactly which model outperforms the others. For this one can use one of the following pairwise tests:

    \newdef{Nemenyi test}{\index{test!Nemenyi}
        Consider the average ranks $R_i$ from the Friedman test. As a test statistic one uses
        \begin{gather}
            z := \frac{R_i - R_j}{\sqrt{\frac{k(k+1)}{6N}}}
        \end{gather}
        where $k$ is the number of models and $N$ is the number of data sets. The exact critical values can either be found in the literature or one can approximately use a normal distribution.
    }

    \newdef{Bonferroni-Dunn test}{\index{test!Bonferonni-Dunn}
        If all one wants to do is see if a particular model performs better than a given baseline model than the Nemenyi test is too conservative since it corrects for $k(k-1)/2$ model comparisons instead of $k-1$. Therefore it is better to use a general method to control the family-wise error for multiple measurements. The Bonferroni-Dunn test modifies the Nemenyi test by performing a Bonferroni correction with $n-1$ degrees of freedom.
    }

    A more powerful test is given by the following strategy:
    \newdef{Holm test}{\index{test!Holm}
        Consider the $p$-values of the Nemenyi test. Instead of comparing all values to a single Bonferroni-corrected significance, one can use a so-called ''step-down'' method. First we order the $p$-values in ascending order and compare the smallest one to $\frac{\alpha}{k-1}$. If this value is significant, i.e. the hypothesis that the associated models perform equally well is rejected, then one compares $p_2$ to $\frac{\alpha}{k-2}$ and so on until one finds a hypothesis that cannot be rejected. All remaining hypotheses are retained as well.
    }

    \begin{remark}
        It is possible that the post hoc test fails to report a significant difference even though the Friedman test rejected the null hypothesis. This is a consequence of the lower power of the post hoc tests.
    \end{remark}

\section{Goodness of fit}\index{goodness of fit}

    Let $f(x;\theta)$ be the fitted function obtained using $N$ measurements.

    \newdef{Akaike information criterion}{\index{Akaike information criterion}
        \nomenclature[A_Ak]{AIC}{Akaike information criterion}
        Consider a model $f(x, \theta)$ with $k$ parameters fitted to a given data sample and let $\mathcal{L}_0$ be the maximum of the associated likelihood function. The Akaike information criterion is defined a follows:
        \begin{gather}
            \text{AIC} := 2k - 2\ln(\mathcal{L}_0).
        \end{gather}
        From this definition it is immediately clear that the AIC rewards goodness-of-fit but penalizes overfitting due to the first term.

        This criterion is often useful when trying to select the best model/parameters to describe a certain data set. However it should be noted that it is not an absolute measure of quality.
    }

\subsection{\texorpdfstring{$\chi^2$}{Chi squared}-test}\index{$\chi^2$-test}\label{statistics:section:chi_squared_test}

    \begin{property}
        If there are $N - n$ fitted parameters we have:
        \begin{gather}
            \label{statistics:gof:chance_for_chi_square}
            \int_{\chi^2}^\infty f\left(\chi^2|n\right)d\chi^2 \approx 1\implies
            \begin{cases}
                \circ\text{ good fit}\\
                \circ\text{ errors were overestimated}\\
                \circ\text{ selected measurements}\\
                \circ\text{ lucky shot}
            \end{cases}
        \end{gather}
    \end{property}
    \begin{property}[Reduced chi-squared]
        The reduced chi-squared statistic is defined as follows:
        \begin{gather}
            \chi^2_{\text{red}} := \chi^2/n
        \end{gather}
        where $n$ is the number of degrees of freedom. Depending on the value of this statistic one can draw the following conclusions (under the right assumptions):
        \begin{itemize}
            \item $\chi^2_{\text{red}} \gg 1$: poor modelling
            \item $\chi^2_{\text{red}} > 1$: bad modelling or underestimation of the uncertainties
            \item $\chi^2_{\text{red}} \approx 1$: good fit
            \item $\chi^2_{\text{red}} < 1$: (improbable) overestimation of the uncertainties
        \end{itemize}
    \end{property}

\subsection{Runs test}\index{runs test}

    A good $\chi^2$-test does not mean that the fit is good. As mentioned in property \ref{statistics:gof:chance_for_chi_square} it is possible that the errors were overestimated. Another condition for a good fit is that the data points vary around the fit, i.e. there are no long sequences of points that lie above/underneath the fit. (It is a result of the ''randomness'' of a data sample.) This condition is tested with a runs test \ref{statistics:runs_distribution}.

    \begin{remark}
        The $\chi^2$-test and runs test are complementary. The $\chi^2$-test only takes the absolute value of the differences between the fit and data points into account, the runs test only takes the signs of the differences into account.
    \end{remark}

    \newformula{Runs distribution}{\index{distribution!runs}
        Let $N_+$ and $N_-$ denote the number of points above and below the fit. Under the hypothesis that all points were independently drawn from the same distribution the number of runs is distributed as follows (approximately Gaussian):
        \begin{gather}
            \label{statistics:runs_distribution}
            \begin{aligned}
                P(r_{even}) &= 2\stylefrac{C^{N_+ - 1}_{\frac{r}{2} - 1}C^{N_- - 1}_{\frac{r}{2} - 1}}{C^N_{N_+}}\\
                P(r_{odd}) &= \stylefrac{C^{N_+ - 1}_{\frac{r - 3}{2}}C^{N_- - 1}_{\frac{r - 1}{2}} + C^{N_- - 1}_{\frac{r - 3}{2}}C^{N_+ - 1}_{\frac{r - 1}{2}}}{C^N_{N_+}}
            \end{aligned}
        \end{gather}
        where $C^n_k$ is the binomial coefficient $\binom{n}{k}$. The first two moments of this distribution are given by the following formulas:
        \begin{align}
            E(r) &= 1 + 2\stylefrac{N_+ N_-}{N}\\
            V(r) &= 2\stylefrac{N_+ N_-}{N}\stylefrac{2N_+ N_- - N}{N(N-1)}
        \end{align}
    }
    \remark{For $r > 10\text{ a }15$ the runs distribution approximates a Gaussian distribution.}

\subsection{Kolmogorov test}\index{test!Kolmogorov}

    \newdef{Empirical distribution function}{\index{distribution!empirical}\label{statistics:empirical_distribution}
        The empirical probability distribution function is defined by placing delta masses at the measurements:
        \begin{gather}
            P_n := \frac{1}{n}\sum_{i=1}^n\delta_{x_i}.
        \end{gather}
        The associated cumulative distribution is then given by
        \begin{gather}
            \label{statistics:empirical_distribution_function}
            F_n(x) := \frac{1}{n}\sum_{i=1}^n\mathbbm{1}_{]-\infty,x]}(x_i)
        \end{gather}
        where $\mathbbm{1}_A(x)$ is the indicator function \ref{lebesgue:indicator_function}.
    }
    \newdef{Kolmogorov-Smirnov statistic}{
        Let $F(x)$ be a given cumulative distribution function. The $n^{th}$ Kolmogorov-Smirnov statistic is defined as follows:
        \begin{gather}
            \label{statistics:kolmogorov_smirnov_statistic}
            D_n := \sup_x|F_n(x) - F(x)|.
        \end{gather}
    }

    \newdef{Kolmogorov distribution}{\index{distribution!Kolmogorov}
        \begin{gather}
            \label{statistics:kolmogorov_distribution_cumulative}
            P(x\geq K) := 1 - 2\sum_{i=1}^{+\infty}(-1)^{i-1}e^{-2i^2x^2} = \stylefrac{\sqrt{2\pi}}{x}\sum_{i=1}^{+\infty}e^{-(2i-1)^2\pi^2/(8x^2)}
        \end{gather}
    }

    \newprop{Kolmogorov-Smirnov test}{\index{test!Kolmogorov-Smirnov}\index{Kolmogorov-Smirnov|see{test}}
        Let the null hypothesis $H_0$ state that a given data sample is described by a continuous distribution $P(x)$ with cumulative distribution function $F(x)$. The null hypothesis is rejected at significance level $\alpha$ if
        \begin{gather}
            D_n\sqrt{n} > K_{\alpha}
        \end{gather}
        where $K_{\alpha}$ is defined by the Kolmogorov distribution: $P(K\leq K_{\alpha}) = 1-\alpha$.
    }
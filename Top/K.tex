\chapter{\texorpdfstring{$K$-theory $\clubsuit$}{K-theory}}\index{K-theory}

    In this chapter all topological (base) spaces are supposed to be both compact and Hausdorff (unless stated otherwise). This ensures that the complex of $K$-theories satisfies the Eilenberg-Steenrod axioms \ref{topology:eilenberg_steenrod_axioms}. In general we will work over an arbitrary field $k$. If necessary we will specialize to a specific field such as $\mathbb{R}$ or $\mathbb{C}$.

    The main reference for this chapter is \cite{karoubi}.

\section{Preliminaries}

    \newdef{Stable general linear group}{\index{stable!group}\label{k:stable_group}
        Let $R$ be a (unital) ring. For every two integers $m<n$ there exists a canonical inclusion $\text{GL}(m, R)\hookrightarrow\text{GL}(n, R)$ through extension (direct sum) by $\mathbbm{1}_{n-m}$. This allows us to define the stable general linear group (or infinite general linear group) as a direct limit:
        \begin{gather}
            \text{GL}(R) := \varinjlim_{n\in\mathbb{N}}\text{GL}(n, R).
        \end{gather}
    }
    \remark{A similar construction leads to the stable orthogonal and stable unitary groups $\text{O}$ and $\text{U}$.\footnote{One sometimes uses the notations $\text{O}(\infty)$ and $\text{U}(\infty)$ (when $R=\mathbb{R}$ and $R=\mathbb{C}$ respectively).}

\section{\texorpdfstring{Topological $K$-theory}{Topological K-theory}}

    In this section we will work over a general field $k$. Certain statements will be given for the specific cases $k=\mathbb{R},\mathbb{C}$, since these are the most relevant for us.

\subsection{Introduction (degree 0)}

    \newdef{$K$-theory}{
        Let $\text{Vect}(X)/\sim$ be the set of isomorphism classes of finite-dimensional vector bundles over a base space $X$. Because this set is well-behaved with respect to Whitney sums, the structure $(\text{Vect}(X)/\sim, \oplus)$ forms an Abelian monoid. The Grothendieck completion \ref{group:grothendieck_completion} of this monoid is called the (real) $K$-theory of $X$.
    }

    \begin{notation}
        \nomenclature[S_K]{$K^0(X)$}{$K$-theory over a (compact Hausdorff) space $X$.}
        The $K$-theory of a space $X$ is denoted by $K^0(X)$.
    \end{notation}

    \begin{example}[Point]\label{k:point}
        Let $\ast$ be the one-point space. The $K$-theory $K^0(\ast)$ is isomorphic to the additive group of integers $(\mathbb{Z}, +)$.
    \end{example}

    \newdef{Virtual vector bundle}{\index{virtual!vector bundle}\index{vector!bundle}\label{k:virtual_bundle}
        The elements of $K^0(X)$ are pairs $([E],[E'])$ that can formally be written as a difference $[E]-[E']$ of vector bundles. Such pairs are called virtual (vector) bundles.
    }
    \newdef{Virtual rank}{\index{virtual!rank}\index{rank!of vector bundle}
        The virtual rank of the virtual bundle $([E],[E'])$ is defined as follows:
        \begin{gather}
            \text{rk}([E],[E']) := \text{rk}(E) - \text{rk}(E').
        \end{gather}
    }

    \begin{property}
        Property \ref{diff:hausdorff} implies that every virtual bundle is of the form $[E]-[X\times\mathbb{R}^n]$ for some vector bundle $E$ and integer $n\in\mathbb{N}$.
    \end{property}

    \newdef{Reduced $K$-theory}{
        Let $(X,x_0)$ be a pointed space. The inclusion $\{x_0\}\hookrightarrow X$ induces a group morphism $\rho:K^0(X)\rightarrow K^0(x_0)$ given by the restriction of virtual bundles to the basepoint $x_0$. The reduced $K$-theory $\widetilde{K}^0(X)$ is given by $\ker(\rho)$.
    }

    \begin{adefinition}
        One can define the reduced $K$-theory $\widetilde{K}(X)$ equivalently as follows: Consider the stable isomorphism classes\footnote{See definition \ref{diff:stable_isomorphism}.} of vector bundles over $X$. Under Whitney sums these define a commutative group $(\text{Vect}(X)/\sim_{stable}, \oplus)$ which is (naturally) isomorphic to $\widetilde{K}^0(X)$.
    \end{adefinition}

    The following construction is very similar to \ref{clifford:functor_k_group}. In fact it is the one obtained for the Banach functor that restricts vector bundles to subspaces.
    \newdef{Relative $K$-theory}{\label{k:relative}
        Consider a space $X$ and a closed subspace $Y$. Let $\mathscr{V}(X, Y)$ denote the set of triples $(E,E',f)$ where $E,E'$ are vector bundles over $X$ and $f$ is an isomorphism between the restrictions $E|_Y$ and $E'|_Y$. Elements in $\mathscr{V}(X, Y)$ are said to be isomorphic if there exist isomorphisms of vector bundles that make the ''obvious'' diagram commute. The sum of such triples is defined elementwise. Let $\mathscr{E}(X, Y)$ denote the subset of $\mathscr{V}(X, Y)$ consisting of triples $(F,F',g)$ where $F=F'$ and $g$ is homotopic to $\mathbbm{1}_{F|_Y}$ in $\text{Aut}(F|_Y)$.

        The relative $K$-theory $K^0(X, Y)$ is defined as the quotient of $\mathscr{V}(X, Y)$ by the following equivalence relation:
        \begin{gather}
            x\sim x' \iff \exists e,e'\in\mathscr{E}(X, Y):x+e\cong_{\mathscr{V}} x'+e'.
        \end{gather}
        So elements of relative $K$-theory are pairs of vector bundles over $X$ that coincide on the subspace $Y$ modulo a relation akin to that from the Grothendieck construction. It should also be clear that choosing $Y=\emptyset$ gives exactly the Grothendieck construction and hence $K^0(X, \emptyset) \equiv K^0(X)$.
    }
    \begin{adefinition}
        Consider a space $X$ with a closed subspace $Y$.
        \begin{gather}
            K^0(X, Y) := \ker\left(K^0(X)\rightarrow K^0(Y)\right).
        \end{gather}
    \end{adefinition}
    \begin{property}[Excision]\index{excision}\label{k:excision}
        Consider a space $X$ together with a closed subspace $Y$. The relative $K$-theory is related to the reduced $K$-theory as follows:
        \begin{gather}
            K^0(X, Y) \cong \widetilde{K}^0(X/Y).
        \end{gather}
    \end{property}

\subsection{Classification}

    \begin{property}[Classifying space of orthogonal group]\index{classifying!space}
        Here we consider the classifying space (see definition \ref{diff:classifying_space}) of the orthogonal group $\text{O}(n)$. Recall the Grassmannian $\text{Gr}(n, \mathbb{R}^N)$ of $n$-dimensional subspaces in $\mathbb{R}^k$. There exists a canonical inclusion of Grassmannians:
        \begin{gather}
            \iota_k:\text{Gr}(n, \mathbb{R}^k)\hookrightarrow \text{Gr}(n, \mathbb{R}^{k+1}):W\mapsto W.
        \end{gather}
        By taking the direct limit of these inclusions, one obtains the infinite Grassmannian:
        \begin{gather}
            B\text{O}(n) := \varinjlim_{k\in\mathbb{N}}\text{Gr}(n, \mathbb{R}^k).
        \end{gather}
        As the name implies, it can be shown that this is the classifying space of $\text{O}(n)$.

        There also exists a canonical inclusion of Grassmannians
        \begin{gather}
            \iota_{n,k}:\text{Gr}(n, \mathbb{R}^k)\hookrightarrow \text{Gr}(n+1, \mathbb{R}^{k+1}):W\mapsto W\oplus\text{span}\{e_{k+1}\}.
        \end{gather}
        This in turn induces an inclusion $B\text{O}(n)\hookrightarrow B\text{O}(n+1)$ of classifying spaces. The direct limit over this system of inclusions is denoted by $B\text{O}$, it is the classifying space of the stable orthogonal group $\text{O}$.
    \end{property}
    \begin{remark*}
        A similar construction allows us to construct the classifying space $B\text{U}(n)$ by starting from complex Grassmannians.
    \end{remark*}
    \begin{remark}\label{k:kuiper_remark}
        It should be noted that neither $B\text{O}$ nor $B\text{U}$ can be expressed as classifying spaces of a group over some infinite-dimensional Hilbert space. This follows from \textit{Kuiper's theorem} which states that such groups are contractible and hence have vanishing homotopy groups (which does not hold for our classifying spaces).
    \end{remark}

    \begin{property}[Homotopy classification]
        For all spaces $X$ the $K$-theory can be represented as follows:
        \begin{gather}
            K^0(X) = [X, B\text{GL}(k)\times\mathbb{Z}].
        \end{gather}
        When specializing to $k=\mathbb{R},\mathbb{C}$ this becomes
        \begin{gather}
            K^0_{\mathbb{R}}(X) = [X, B\text{O}\times\mathbb{Z}]
        \end{gather}
        and
        \begin{gather}
            K^0_{\mathbb{C}}(X) = [X, B\text{U}\times\mathbb{Z}].
        \end{gather}
        due to homotopy invariance. Reduced $K$-theory can be obtained by considering basepoint-preserving homotopies. For connected spaces this is equivalent to $[X, B\text{GL}]$.
    \end{property}

    \begin{remark}[Noncompact spaces]
        If we consider noncompact spaces one can still use either the Grothendieck construction or the representable definition for topological $K$-theory. However, these will not coincide anymore although there does exist an injection from the Grothendieck $K$-theory to the representable $K$-theory.
    \end{remark}

    The following theorem should be compared to remark \ref{k:kuiper_remark} above (in fact this theorem can be proven through Kuiper's theorem):
    \begin{theorem}[Atiyah-J\"anich]\index{Atiyah-J\"anich}\label{k:atiyah_janich}
        The space of Fredholm operators\footnote{See definition \ref{banach:fredholm}.} on a separable and infinite-dimensional Hilbert space forms a classifying space for $K$-theory.
    \end{theorem}

\subsection{Negative degree}

    \newdef{$K^{-1}$}{
        For every field $k$ we define (again, this can also be seen as a property when using a different definition) the relative $K$-functor $K^{-1}$ as follows:
        \begin{gather}
            K^{-1}(X, Y) := [X/Y, \text{GL}(k)]_*
        \end{gather}
        where the asterisk denotes the fact that we consider basepoint-preserving homotopies. We can obtain $K^{-1}(X)$ by considering $Y=\emptyset$ and recalling relation \ref{topology:empty_quotient}:
        \begin{gather}
            K^{-1}(X) := [X, \text{GL}(k)].
        \end{gather}
        For $k=\mathbb{R},\mathbb{C}$ one can use homotopy invariance to obtain
        \begin{align}
            K^{-1}_{\mathbb{R}}(X) &:= [X, \text{O}]\\
            K^{-1}_{\mathbb{C}}(X) &:= [X, \text{U}].
        \end{align}
    }

    To define lower degree groups it will be useful to extend $K$-theory to locally compact spaces:
    \newdef{$K$-theory of locally compact spaces}{\label{k:locally_compact}
        Let $X$ be a locally compact space and denote its one-point compactification \ref{topology:alexandrov_compactification} by $\widehat{X}$. We define the groups $K^0(X)$ and $K^{-1}(X)$ as follows:
        \begin{align}
            K^0(X) &:= \text{ker}\left(K^0(\widehat{X})\rightarrow K^0(\{\infty\})\right)\label{k:locally_compact_formula}\\
            K^{-1}(X) &:= \text{ker}\left(K^{-1}(\widehat{X})\rightarrow K^{-1}(\{\infty\})\right)
        \end{align}
        So we define the $K$-theory of a locally compact space as the reduced $K$-theory of its one-point compactification.
    }
    \begin{result}[Relative $K$-theory and complements]
        Consider a space $X$ with a closed subspace $Y$. We can identify $(X/Y)\backslash\{y_0\}$ with $X\backslash Y$ and hence we obtain
        \begin{gather}
            K^0(X\backslash Y) = \widetilde{K}^0(X/Y).
        \end{gather}
        When combined with the excision property this gives a result akin of ordinary (singular) cohomology where the relative cocycles were those defined on the complement $X\backslash Y$:
        \begin{gather}
            \label{k:relative_and_complement}
            K^0(X, Y) \cong K^0(X\backslash Y).
        \end{gather}
    \end{result}

    \newdef{$K^{-n}$}{\label{k:lower_groups}
        We can generally define lower relative $K$-groups as follows:
        \begin{gather}
            \label{k:lower_relative}
            K^{-n}(X, Y) := K^0((X\backslash Y)\times\mathbb{R}^n).
        \end{gather}
        By taking $Y=\emptyset$ we obtain the groups $K^{-n}(X)$ as $K^0$-groups of trivial line bundles:
        \begin{gather}
            K^{-n}(X) := K^0(X\times\mathbb{R}^n).
        \end{gather}
        Before relating this to reduced $K$-theory we first warn the reader about a possible confusion. Homotopy invariance of $K$-theory would seem to imply that the above definition is senseless, since $X\cong X\times\mathbb{R}^n$ in the homotopy category. However, $X\times\mathbb{R}^n$ is not compact (even if $X$ is) and hence we should work with definition \ref{k:locally_compact}.

        It can be shown through a series of homeomorphisms that the above definition is equivalent to the following one:
        \begin{gather}
            \label{k:relative_reduced}
            K^{-n}(X, Y) \cong \widetilde{K}^0(\Sigma^n(X/Y))
        \end{gather}
        where $\Sigma$ denotes the reduced suspension functor \ref{topology:suspension}. As such, the reduced suspension functor gives us a way to move down in the tower of $K$-groups.
    }

\subsection{Bott periodicity}\index{Bott!periodicity}

    \newdef{Cup product}{\index{cup product}
        We first generalize the Whitney sum and tensor product constructions to vector bundles over different base spaces. Let $E,E'$ be vector bundles over the base spaces $B,B'$. Consider the projection maps $\pi:B\times B'\rightarrow B$ and $\pi:B\times B'\rightarrow B'$. The exterior sum $E\oplus E'\rightarrow B\times B'$ is defined as the Whitney sum $\pi^*(E)\oplus\pi'^*(E')$. Analogously we define the exterior product bundle as the tensor product $\pi^*(E)\otimes\pi'^*(E')$. Fibrewise, this is just the ordinary direct sum and tensor product of vector spaces.

        The exterior product induces a bilinear map on $K$-theory as follows: From definition \ref{k:virtual_bundle} we know that every element $x\in K^0(X)$ can be written as formal difference $[E]-[E']$ of vector bundles over $X$. Using this decomposition we define the cup product $x\cup y$ through the following formula:
        \begin{gather}
            ([E]-[E'])\cup([F]-[F']) := [E\otimes F] + [E'\otimes F'] - [E\otimes F'] - [E'\otimes F].
        \end{gather}

        We can now extend this definition to locally compact spaces. Every element of $K^0(Y)$, for $Y$ locally compact, defines an element in $K^0(\widehat{Y})$ and for such elements the cup product was defined above. By restricting to $Y\times Y'$ one can then obtain an element of $K^0(Y\times Y')$:
        \begin{gather}
            K^0(Y)\times K^0(Y')\xrightarrow{\iota\times\iota'}K^0(\widehat{Y})\times K^0(\widehat{Y'})\xrightarrow{\cup}K^0(\widehat{Y}\times\widehat{Y'})\xrightarrow{\text{res}} K^0(Y\times Y')
        \end{gather}
        where $\iota,\iota'$ are the inclusions induced by \ref{k:locally_compact_formula}.
    }
    \begin{property}[Ring structure]
        By precomposing with the diagonal morphism $K^0(X\times X)\rightarrow K^0(X)$ we can endow $K^0(X)$ with a commutative ring structure. (At least over commutative fields such as $\mathbb{R},\mathbb{C}$.)
    \end{property}

    \begin{property}
        If we recall definition \ref{k:lower_groups} we immediately see that the cup product on $K^0$ also defines a bilinear operation $K^{-m}(X)\times K^{-n}(X)\rightarrow K^{-m-n}(X)$. Furthermore, as above, this induces a multiplicative structure on the complex $K^\bullet(X):=\bigoplus_{n=0}^\infty K^{-n}(X)$. This multiplication can be shown to endow the $K$-complex with the structure of a graded-commutative algebra \ref{linalgebra:graded_commutative}.

        By using the isomorphism \ref{k:relative_and_complement} we can also extend the cup product to an operation on relative $K$-theory:
        \begin{gather}
            K^{-m}(X, Y)\rightarrow K^{-n}(X', Y')\rightarrow K^{-m-n}(X\times X', X\times Y'\cup X'\times Y).
        \end{gather}
    \end{property}

    \begin{notation}[Bott element]\index{Bott!element}
        Let us consider the complex relative $K$-group \[K^0_{\mathbb{C}}(D^2, S^1)\cong\widetilde{K}^0_{\mathbb{C}}(S^2)\cong K^0_{\mathbb{C}}(\mathbb{R}^2)\] of the unit disk with respect to its boundary. By $\beta$ we will denote the element represented by the triple $\left(D^2\times\mathbb{C}^2, D^2\times\mathbb{C}^2, \alpha:(x,v)\mapsto xv\right)$.
    \end{notation}
    \begin{theorem}[Complex Bott periodicity]
        The cup product with the Bott element $\beta$ gives an isomorphism $K^{-n}_{\mathbb{C}}(X, Y)\cong K^{-n}_{\mathbb{C}}(X\times D^2, X\times S^1\cup Y\times D^2)\cong K^{-n-2}_{\mathbb{C}}(X, Y)$. This also implies that cupping with $\beta$ gives an isomorphism $K^0_{\mathbb{C}}(X)\cong K^0_{\mathbb{C}}(X\times\mathbb{R}^2)$.
    \end{theorem}
    \begin{result}
        Applying Bott periodicity to the case $X=\ast, Y=\emptyset$ and comparing to example \ref{k:point}, we obtain $K^0_{\mathbb{C}}(D^2, S^1)\cong\mathbb{Z}$. We also conclude that $\beta$ is a generator of $K^0_{\mathbb{C}}(D^2, S^1)$.
    \end{result}
    \begin{result}[Spheres]
        Bott periodicity and equation \ref{k:relative_reduced} also allow us to compute the reduced $K$-theory of spheres:
        \begin{gather}
            \widetilde{K}^0_{\mathbb{C}}(S^n) =
            \begin{cases}
                0&n\text{ odd}\\
                \mathbb{Z}&n\text{ even}.
            \end{cases}
        \end{gather}
        For $n$ even we can see that the generator is given by $\beta^{n/2}$.
    \end{result}

    \begin{property}[Homotopy groups of unitary group]\label{k:homotopy_group_U}
        Property \ref{diff:vector_bundles_over_sphere} can be generalized to the stable linear group to obtain an isomorphism $\widetilde{K}^0_k(S^n)\cong\pi_{n-1}(\text{GL}(k))$. By specializing to $k=\mathbb{C}$, recalling that $\text{GL}(m,\mathbb{C})$ deformation retracts onto $\text{U}(m)$ and applying Bott periodicity, we obtain that the homotopy groups of the stable unitary group are mod 2-periodic. Furthermore, by using the fibration
        \begin{gather}
            \text{U}(n)\rightarrow\text{U}(n+1)\rightarrow S^{2n+1},
        \end{gather}
        in particular its induced long exact sequence, one can show that for $n>i/2$ the homotopy groups $\pi_i(\text{U}(n))$ satisfy the same periodic relation.
    \end{property}

    \begin{theorem}[Real Bott periodicity]
        The cup product with the real Bott element, i.e. the generator of $K^{-8}_{\mathbb{R}}(\ast)\cong\mathbb{Z}$, gives an isomorphism
        \begin{gather}
            K^{-n}_{\mathbb{R}}(X, Y)\rightarrow K^{-n-8}_{\mathbb{R}}(X, Y).
        \end{gather}
    \end{theorem}

    \begin{theorem}[Weak Bott periodicity]
        The following spaces are homotopy equivalent:\footnote{For an extensive list see \cite{karoubi}.}
        \begin{align}
            \text{\emph{GL}}(\mathbb{R})&\sim\Omega^8\text{\emph{GL}}(\mathbb{R})\\
            \text{\emph{O}}&\sim\Omega^8\text{\emph{O}}\nonumber\\
            \text{\emph{U}}&\sim\Omega(\mathbb{Z}\times B\text{\emph{U}})\\
            \mathbb{Z}\times B\text{\emph{U}}&\sim\Omega\text{\emph{U}}.
        \end{align}
    \end{theorem}
    \begin{result}
        Through Eckmann-Hilton duality this implies the periodicity in the homotopy groups of the stable orthogonal and unitary groups (cf. property \ref{k:homotopy_group_U}).
    \end{result}
    \begin{property}
        One can also relate real and quaternionic $K$-theory through the following homotopy equivalences:
        \begin{align}
            \mathbb{Z}\times B\text{GL}(\mathbb{R})&\sim\Omega^4(B\text{GL}(\mathbb{H}))\\
            \mathbb{Z}\times B\text{GL}(\mathbb{H})&\sim\Omega^4(B\text{GL}(\mathbb{R})).
        \end{align}
    \end{property}

    \begin{remark}[Positive degree]
        Bott periodicity allows to define $K$-groups in positive degree.
    \end{remark}

\subsection{Clifford modules}\index{Clifford!module}

    One can restate the above sections in terms of \textbf{Clifford modules} (also called \textbf{Clifford module bundles}), i.e. vector bundles for which the fibres carry a representation of a Clifford algebra. We will mainly use the content of section \ref{section:clifford_bott}.

    From definition \ref{clifford:functor_k_group} and example \ref{clifford:k00} it should be clear that what we called $K^0(X)$ is in fact equivalent to $K^{0,0}(\mathbf{C})$ for $\mathbf{C}=\text{Vect}(X)$. In a similar vein one can prove that $K^{-1}(X)$ is equivalent to $K^{1,0}(\text{Vect}(X))$.

    This relation is in fact generalizable to all values for $p,q$ (writing $K^{p,q}(X)$ for $K^{p,q}(\text{Vect}(X))$):
    \begin{property}
        The $K^0(X)$-modules $K^{p,q}(X)$ and $K^{q-p}(X)$ are isomorphic.
    \end{property}

    Property \ref{clifford:bott_periodicity_category} then implies the Bott periodicity for the groups $K^{q-p}(X)$. For complex $K$-theory one can use remark \ref{clifford:complex_bott_periodicity}. The most important takeaway for this section is that one can rephrase $K$-theory in terms of Clifford modules and canonically induced functors between them.

\subsection{Cohomology theory}

    \begin{property}[Excision]\index{excision}
        It can be shown that the excision property \ref{k:excision} holds at every degree $n\in\mathbb{Z}$:
        \begin{gather}
            K^n(X/Y, \{y_0\})\cong K^n(X, Y).
        \end{gather}
        More generally this can be stated as
        \begin{gather}
            K^n(X\backslash U, Y\backslash U)\cong K^n(X, Y).
        \end{gather}
        where $\overline{U}\subset\mathring{Y}$.
    \end{property}
    \begin{property}[Homotopy invariance]
        Homotopic maps induce equal morphisms in $K$-theory at every degree. In particular this implies that homotopy equivalences induce isomorphisms in $K$-theory.
    \end{property}

    \begin{remark}[Generalized cohomology]
        The above properties imply that the complex of $K$-groups satisfies the Eilenberg-Steenrod axioms \ref{topology:eilenberg_steenrod_axioms} except for the dimension axiom. As such it is a generalized cohomology theory.
    \end{remark}

\subsection{Applications}

    In this section we list some applications to mathematics. ?? REFER TO PHYSICS ??

    \begin{property}[Degree]\index{degree!of map}
        Recall the definition \ref{topology:degree} of degree from algebraic topology. In the same way one can assign to every continuous function $f:S^n\rightarrow S^n$ a degree through its induced action on $K$-theory. In fact, the topological degree and the $K$-theoretic degree coincide.

        One can also extend this to ''multidegrees''. For example in the case of bidegree we consider continuous functions $\mu:S^n\times S^n\rightarrow S^n$. The bidegree $(p,q)$ of $\mu$ is defined as the pair of degrees of the maps $x\mapsto\mu(x,x_0)$ and $y\mapsto(x_0,y)$ for a fixed basepoint $x_0$.
    \end{property}

    \newdef{$H$-space}{\index{H-space}
        A sphere $S^n$ is said to be an $H$-space if it admits a continuous function $\mu:S^n\times S^n\rightarrow S^n$ of bidegree $(1,1)$, i.e. it admits such a function for which the pointwise maps are homotopic to the identity.\footnote{This second formulation can also be used for other topological spaces, e.g. the $H$-structure on loop groups \ref{topology:h_structure}.}
    }

    \begin{property}[Puppe sequence]\index{Puppe sequence}
        The mapping cone $C_f$ from definition \ref{topology:mapping_cylinder} fits in an exact sequence:
        \begin{gather}
            X\longrightarrow Y\longrightarrow C_f\longrightarrow \Sigma X\longrightarrow \Sigma Y.
        \end{gather}
        Reduced $K$-theory maps this to an induced exact sequence (whenever the reduced $K$-theory is defined for $X,Y$):
        \begin{gather}
            \widetilde{K}^0(\Sigma X)\longrightarrow\widetilde{K}^0(\Sigma Y)\longrightarrow \widetilde{K}^0(C_f)\longrightarrow\widetilde{K}^0(X)\longrightarrow\widetilde{K}^0(Y).
        \end{gather}
    \end{property}
    The Puppe sequence in $K$-theory allows us to prove an important theorem in the case of mappings of spheres:
    \newdef{Hopf invariant}{\index{Hopf!invariant}
         The Puppe sequence for $f:S^{2n-1}\rightarrow S^n$ with $n$ even implies that $\widetilde{K}^0(C_f)\cong\mathbb{Z}\oplus\mathbb{Z}$ where the generators are induced by the Bott elements of the spheres. The relation $\beta_{2n}=\beta_n\cup\beta_n$ of Bott elements gives an induced relation $a^2=\lambda b$ of generators in $\widetilde{K}^0(C_f)$. The integer $\lambda$ is called the Hopf invariant of $f$.\footnote{The choice of generator $a$ corresponding to $\beta_n$ is not relevant due to the relations $b^2=ab=0$ obtained by dimensional arguments.}
     }
     One can show that every ''multiplication'' map $\mu: S^{n-1}\times S^{n-1}\rightarrow S^{n-1}$ of bidegree $(p,q)$ induces a map $S^{2n-1}\rightarrow S^n$ of Hopf invariant $pq$.

     \begin{theorem}[Atiyah-Adams]\index{Atiyah-Adams}
         Let $n$ be even. If a continuous function $S^{2n-1}\rightarrow S^n$ has odd Hopf invariant, then $n=2,4$ or 8.
     \end{theorem}
     \begin{result}
         A sphere $S^{n-1}$ can only admit a multiplication map of bidegree $(1,1)$, or equivalently, admit an $H$-structure, if $n=2,4$ or 8.\footnote{The case $S^0$ can be proven in a different way.}
     \end{result}

    \begin{theorem}[Atiyah-Hirzebruch]\index{Atiyah-Hirzebruch}
        Let $X$ be a compact Hausdorff space. The Chern character induces the following isomorphisms:
        \begin{gather}
            K^0_{\mathbb{C}}(X)\otimes\mathbb{Q}\cong\bigoplus_{i\in\mathbb{Z}}H^{2i}(X;\mathbb{Q})\\
            K^1_{\mathbb{C}}(X)\otimes\mathbb{Q}\cong\bigoplus_{i\in\mathbb{Z}}H^{2i+1}(X;\mathbb{Q}).
        \end{gather}
        For noncompact spaces one needs to work with rational $K$-theory.
    \end{theorem}

\section{\texorpdfstring{Algebraic $K$-theory}{Algebraic K-theory}}
\subsection{Determinant}

    Over noncommutative rings $R$ the determinant of a matrix is not as easily defined as over commutative rings such as field. For example in the $2\times2$ case one could choose either $ad-bc$ or $da-bc$ (or any other permutation), there exists no canonical choice. To fix this we take a look at the most important properties of the determinant map:
    \begin{itemize}
        \item It is invariant under elementary row/column operations (see item 3 of property \ref{linalgebra:determinant_properties}).
        \item It is invariant under augmentation by the identity, i.e. under the transformation $A\mapsto A\oplus\mathbbm{1}$.
    \end{itemize}
    To implement the second property we will have to move from the finite-dimensional general linear groups $\text{GL}(n, R)$ to their stable version from definition \ref{k:stable_group}. On this group one can then define an equivalence relation by saying that two matrices are equivalent if they belong to the same coset with respect to the subgroup $E(R)$ generated by the elementary matrices \ref{linalgebra:elementary_matrix}. It can also be shown that $E(R)$ is equal to the commutator subgroup $[\text{GL}(R), \text{GL}(R)]$.

    The determinant map is then abstractly defined as the quotient map from the following definition:
    \newdef{$K_1$}{\index{determinant}
        The first algebraic $K$-group of a ring $R$ is defined as the Abelianization of its stable general linear group:
        \begin{gather}
            K_1(R) := \text{GL}(R)/[\text{GL}(R),\text{GL}(R)].
        \end{gather}
        The quotient map $\pi:\text{GL}(R)\rightarrow K_1(R)$ is called the \textbf{determinant map}.
    }

    To obtain lower $K$-groups we will define a ''suspension functor'':
    \newdef{Suspension}{\index{suspension}
        Let $R$ be a ring. By $\text{Mat}(R)$ we now denote the infinite matrix ring over $R$, i.e. the set of matrices with a finite number of nonzero entries in each row and column. This ring contains an ideal $\text{Mat}_{\text{fin}}(R)$ generated by all matrices that are zero outside a block of finite size. The suspension of $R$ is then defined as follows:
        \begin{gather}
            \Sigma R := \text{Mat}(R)/\text{Mat}_{\text{fin}}(R).
        \end{gather}
    }
    \newdef{Lower $K$-groups}{
        For all integers $n\geq1$ one defines the $K$-groups as follows:
        \begin{gather}
            H_n(R) := K_{1-n}(\Sigma^nR).
        \end{gather}
    }

    \begin{example}[$K_0$]
        It can be shown that $K_0(R)$ corresponds to the Grothendieck group associated to the monoid of finitely-generated projective $R$-modules. The relation to its topological counterpart is given by the Serre-Swan theorem \ref{diff:serre_swan}.
    \end{example}
\chapter{General Topology}\label{chapter:topology}
\section{Topological spaces}

    \newdef{Topology}{\index{topology}
        \nomenclature[S_Top]{$\mathbf{Top}$}{category of topological spaces}
        Let $X$ be a set and consider a collection of subsets $\tau\subseteq 2^X$. The set $\tau$ is a topology on $X$ if it satisfies the following axioms:
        \begin{enumerate}
            \item $\emptyset\in\tau$ and $X\in\tau$,
            \item $\forall\,\mathcal{F}\subseteq\tau: \bigcup_{V\in\mathcal{F}}V \in \tau$, and
            \item $\forall\,U,V\in\tau: U\cap V\in\tau$.
        \end{enumerate}
        The elements of $\tau$ are called \textbf{open sets} and the couple $(X,\tau)$ is called a \textbf{topological space}. The \textbf{closed sets} are defined as the sets that have an open complement. Because complements are uniquely defined, one could just as well define a topology in terms of closed subsets.
    }

    \begin{property}[\difficult{Category of opens}]
        \nomenclature[S_Open]{$\mathbf{Open}(X)$}{category of open subsets of a topological space $X$}
        Consider a topological space $(X,\tau)$ and let $U\subseteq V\in\tau$. The topology $\tau$ together with the collection of inclusion maps $U\hookrightarrow V$ forms a poset and, by extension, a small category $\mathbf{Open}(X)$.
    \end{property}

    \newdef{Pointed topological space}{\label{topology:pointed_space}\index{pointed!topological space}
        Let $x_0\in X$ be any element of a topological space. The triple $(X,\tau,x_0)$ is called a pointed topological space with base point $x_0$.
    }

    \begin{example}[Relative topology\footnotemark]\label{topology:relative_topology}
        \footnotetext{Sometimes called the \textbf{subspace topology}.}
        Any subset $Y$ of a topological space $(X,\tau_X)$ can be turned into a topological space by equipping it with the following topology:
        \begin{gather}
            \tau_\text{rel} := \{U_i\cap Y\mid U_i\in \tau_X\}.
        \end{gather}
    \end{example}
    \begin{example}[Discrete topology]\index{discrete!topology}
        The topology in which every subset is open (and thus also closed).
    \end{example}
    \begin{example}[Indiscrete topology]
        The topology in which only the empty set and the space itself are open.
    \end{example}

    \newdef{Interior}{\index{interior}
        \nomenclature[O_zint]{$X^\circ,\overset{\circ}{X}$}{interior of a topological space $X$}
        The interior $Y^\circ$ of a subset $Y$ of a topological space $X$ is defined as the union of all open subsets of $Y$. Elements of the interior are called \textbf{interior points} of $Y$.
    }
    \newdef{Closure}{\index{closure}
        \nomenclature[O_zclos]{$\overline{X}$}{closure of a topological space $X$}
        The closure $\overline{Y}$ of a subset $Y$ of a topological space $X$ is defined as the intersection of all closed sets containing $Y$.
    }
    \newdef{Boundary}{\index{boundary}
        \nomenclature[O_zbound]{$\partial X$}{boundary of a topological space $X$}
        The boundary $\partial Y$ of a subset $Y$ of a topological space $X$ is defined as $\overline{Y}\backslash Y^\circ$.
    }

    \newdef{Borel set}{\index{Borel!set}\label{topology:borel_set}
        Let $\mathcal{B}$ be the $\sigma$-algebra \ref{set:sigma_algebra} generated by all open subsets of a topological space. The elements $B\in\mathcal{B}$ are called Borel sets.
    }
    \begin{property}[Real line]
        For $\mathbb{R}$, the open, closed and half-open (both types) intervals all generate the same $\sigma$-algebra and, accordingly, the same Borel sets.
    \end{property}

    \newdef{Topological group}{\index{group!topological}
        A group equipped with a topology such that both the multiplication and inversion morphisms are continuous.
    }

\subsection{Neighbourhoods}

    \newdef{Neighbourhood}{\index{neighbourhood}
        A set $N\subseteq X$ is a neighbourhood of a point $x\in X$ if there exists an open set $U$ such that $x\in U\subseteq N$.
    }

    Although the following two notions are often treated as synonyms in the literature, they can be given a separate meaning:
    \newdef{Limit point}{\index{limit!point}
        Let $Y$ be a subset of $X$. A point $x\in X$ is called a limit point of $Y$ if every neighbourhood of $x$ contains at least one point of $Y$ different from $x$.
    }
    By relaxing the last part of this definition, a slightly different notion is obtained:
    \newdef{Adherent point}{\index{adherent point}
        Let $Y$ be a subset of $X$. A point $x\in X$ is called an adherent point of $Y$ if every neighbourhood of $x$ contains at least one point of $S$. A point $x$ is an adherent point of $Y$ if and only if it is an element of the closure $\overline{Y}$.
    }

    \newdef{Accumulation point\footnotemark}{\index{accumulation point|seealso{limit point}}
        \footnotetext{Sometimes called a \textbf{cluster point}.}
        Let $x\in X$ be a limit point of $Y$. It is called an accumulation point of $Y$ if every open neighbourhood of $x$ contains infinitely many points of $Y$.
    }

    \newdef{Basis}{\index{basis}
        A collection $\mathcal{B}\subseteq\tau$ of open subsets of a topological space $(X,\tau)$ is a basis for $(X,\tau)$ if every $U\in\tau$ can be written as
        \begin{gather}
            U = \bigcup_{V\in\mathcal{F}}V,
        \end{gather}
        where $\mathcal{F}\subseteq\mathcal{B}$.
    }
    \newdef{Local basis}{
        A collection $\mathcal{B}_x$ of open neighbourhoods of a point $x\in X$ is a local basis of $x$ if every neighbourhood of $x$ contains at least one element in $\mathcal{B}_x$.
    }

    \newdef{First-countable space}{\index{countability axiom}
        A topological space $(X,\tau)$ for which for every point $x\in X$ there exists a countable local basis.
    }
    \begin{property}[Decreasing basis]
        Let $x\in X$. If there exists a countable local basis for $x$, there also exists a countable decreasing local basis for $x$.
    \end{property}

    \newdef{Second-countable space}{
        A topological space $(X,\tau)$ for which there exists a countable (global) basis.
    }

    \begin{property}[Closure]
        Let $X$ be a topological space. The closure of a subset $Y\subseteq X$ is given by
        \begin{gather}
            \label{topology:closure}
            \overline{Y} = \{x\in X\mid\exists\text{ a net }\net{x}\text{ in } X:x_\alpha\longrightarrow x\}.
        \end{gather}
        This implies that the topology on $X$ is completely determined by the convergence of nets \ref{set:net}.
    \end{property}
    \newdef{Fr\'echet-Urysohn space}{\index{Fr\'echet-Urysohn space}\index{sequential!space}
        A topological space for which the closure of every subset is equal to its sequential closure, i.e. the subset obtained as in \eqref{topology:closure}, but with nets replaced by sequences.

        Fr\'echet-Urysohn spaces form an important subclass of \textit{sequential spaces}, i.e. topological spaces where the topology is uniquely determined by the convergence of sequences (a subset of a sequential space is closed if and only if every convergent sequence converges to a point in the set).
    }
    The following property is of great practical importance:
    \begin{property}\label{topology:first_countable_sequential}
        Every first-countable space is Fr\'echet-Urysohn and, therefore, only convergent sequences have to be considerd in these spaces.
    \end{property}

    \newdef{Germ}{\index{germ}\label{topology:germ}
        Let $X$ be a topological space and let $Y$ be a set. Consider two functions $f,g:X\rightarrow Y$. If there exists a neighbourhood $N$ of a point $x\in X$ such that \[f(u) = g(u)\qquad\qquad\forall u\in N,\] this property defines an equivalence relation denoted by $f\sim_x g$ and the equivalence classes are called germs.
    }

\subsection{Separation axioms}\index{separation axioms}

    \newdef{Irreducible}{\index{irreducible!space}
        A topological space is said to be irreducible if it is not the union of two proper closed subsets or, equivalently, if the intersection of two nonempty open subsets is again nonempty.
    }

    \newdef{$T_0$-space}{\index{distinguishable}\index{Kolmogorov!topology}
        A topological space such that for every two distinct points at least one of them has a neighbourhood not containing the other. The points are said to be \textbf{topologically distinguishable}. $T_0$-spaces are also said to carry a \textbf{Kolmogorov topology}.
    }

    \newdef{$T_1$-space}{\index{separated}\index{Fr\'echet!topology}
        A topological space such that for every two distinct points $x,y$ there exists neighbourhood $N,N'$ of $x$ and $y$ respectively such that $y\not\in N$ and $x\not\in N'$. The points are said to be \textbf{separated}. $T_1$-spaces are also said to carry a \textbf{Fr\'echet topology} (not to be confused with Fr\'echet spaces from functional analysis).
    }

    \newdef{Hausdorff space}{\index{Hausdorff!space}\label{topology:hausdorff}
        A topological space $X$ is a Hausdorff space or $T_2$-space if it satisfies the following condition:
        \begin{gather}
            \forall x,y\in X:\exists\text{ neighbourhoods }N\ni x,N'\ni y:N\cap N'=\emptyset.
        \end{gather}
        The points are said to be \textbf{separated by neighbourhoods}. It can be shown that this definition is equivalent to requiring that the diagonal $\Delta_X$ is closed in the product space $X\times X$.
    }
    \begin{property}[Closed points]
        Every singleton and, by extension, every finite subset is closed in a Hausdorff space.
    \end{property}

    \newdef{Urysohn space}{\index{Urysohn!space}
        A topological space is an Urysohn space or $T_{2\nicefrac{1}{2}}$-space if every two distinct points are separated by closed neighbourhoods.
    }

    \newdef{Regular space}{\index{regular}\label{topology:regular}
        A topological space such that for every closed subset $V$ and every point $x\not\in V$ there exist disjoint open subsets $U,U'$ such that $x\in U$ and $V\subset U'$.
    }
    \newdef{$T_3$-space}{
        A space that is both regular and $T_0$.
    }

    \newdef{Normal space}{\index{normal}\label{topology:normal}
        A topological space such that every two closed subsets have disjoint neighbourhoods.
    }
    \newdef{$T_4$-space}{
        A space that is both normal and $T_1$.
    }

    \begin{property}[Nesting of axioms]
        A space satisfying the separation axiom $T_k$ also satisfies all separation axioms $T_{i\leq k}$.
    \end{property}

\subsection{Convergence}

    \newdef{Convergence}{\index{convergence}
        A sequence $\seq{x}$ in $X$ is said to converge to a point $x\in X$ if
        \begin{gather}
            \forall \text{ neighbourhoods } U \text{ of } x(\exists N\in\mathbb{N}_0(\forall n>N:x_n\in U)).
        \end{gather}
    }
    The ``limit'' of a convergent sequence does not have to be unique:
    \begin{property}[Uniqueness]\label{topology:hausdorff_limit}
        The limit of a converging sequence in a Hausdorff space is unique.
    \end{property}

    \begin{property}[Subsequences]
        Every subsequence of a converging sequence converges to the same point.
    \end{property}

\section{Morphisms}
\subsection{Continuity}

    \newdef{Continuity}{\index{continuity}
        \nomenclature[S_Cont]{$C(X,Y)$}{set of continuous functions between two topological spaces $X$ and $Y$}
        A function between topological spaces is said to be continuous if the inverse image of every open set is also open. The set of all continuous functions between two topological spaces $X,Y$ is often denoted by $C(X,Y)$.
    }

    \newdef{Initial topology}{\index{topology!initial}\label{topology:initial_topology}
        Consider a collection of functions $\{f_i:X\rightarrow Y_i\}_{i\in I}$ between topological spaces. The initial topology on $X$ with respect to this family is the coarsest topology on $X$ for which all maps $f_i$ are continuous.
    }
    \newdef{Final topology}{\index{topology!final}
        Consider a collection of functions $\{f_i:Y_i\rightarrow X\}_{i\in I}$ between topological spaces. The final topology on $X$ with respect to this family is the finest topology on $X$ for which all maps $f_i$ are continuous.
    }

    \begin{property}[Continuity]
        Consider a function $f:X\rightarrow Y$ of topologial spaces, where $X$ is first-countable. The following statements are equivalent:
        \begin{itemize}
            \item $f$ is continuous.
            \item The sequence $(f(x_n))_{n\in\mathbb{N}}$ converges to $f(a)\in Y$ whenever the sequence $\seq{x}$ converges to $a\in X$.
        \end{itemize}
    \end{property}
    \begin{result}
       If the space $Y$ in the previous theorem is Hausdorff, the limit $f(a)$ does not need to be known since it is unique by Property \ref{topology:hausdorff_limit} above.
    \end{result}
    \begin{remark}
        If the space $X$ is not first-countable, one has to consider the convergence of nets \ref{set:net}.
    \end{remark}

    \begin{theorem}[Urysohn's lemma]\index{Urysohn!lemma}\label{topology:urysohns_lemma}
        A topological space X is normal \ref{topology:normal} if and only if every two closed disjoint subsets $A, B\subset X$ can be separated by a continuous function $f:X\rightarrow [0,1]$, i.e. $\forall a\in A,b\in B$ there exists a continuous function $f:X\rightarrow [0,1]$ such that
        \begin{gather}
            f(a) = 0\qquad\text{and}\qquad f(b) = 1.
        \end{gather}
    \end{theorem}
    The following, seemingly different, theorem is actually equivalent to Urysohn's lemma:
    \begin{theorem}[Tietze extension theorem]\index{Tietze extension theorem}
        Consider a continuous function $f:V\rightarrow\mathbb{R}$, where $V$ is a closed subset of normal space $X$. There exists a continuous function $F:X\rightarrow\mathbb{R}$ such that $\forall x\in V:F(x) = f(x)$. Furthermore, if the function $f$ is bounded, then $F$ can be chosen to be bounded by the same number.
    \end{theorem}

\subsection{Homeomorphisms}

    \newdef{Homeomorphism}{\index{homeomorphism}
        A function $f$ such that both $f$ and $f^{-1}$ are continuous and bijective.
    }

    \newdef{Embedding}{\index{embedding}\label{topology:embedding}
        A continuous function that is a homeomorphism onto its image.
    }
    \newdef{Local homeomorphism}{\index{local!homeomorphism}\index{\'etale morphisms}\label{topology:etale_morphism}
        A continuous function $f:X\rightarrow Y$ is a local homeomorphism if for every point $x\in X$ there exists an open neighbourhood $U$ such that $f(U)$ is open and such that $f|_U$ is an embedding. Local homeomorphisms are also called \textbf{\'etale morphisms}.
    }

    \newdef{Covering space}{\index{covering!space}\label{topology:covering_space}
        Consider two topological spaces $X,C$ and a continuous surjection $p:C\rightarrow X$. $C$ is said to be a covering space of $X$ (and $p$ is called a \textbf{covering map}) if for all points $x\in X$ there exists an open neighbourhood $U$ of $x$ such that $p^{-1}(U)$ can be written as a disjoint union $\bigsqcup_iC_i$ of open sets in $C$ where every set $C_i$ is mapped homeomorphically onto $U$. The neighbourhoods $U$ are sometimes said to be \textbf{evenly covered}.
    }
    \begin{notation}
        Because the covering map $p:C\rightarrow M$ is surjective, the space $M$ can be left implicit. Therefore, covering spaces are often just denoted by the couple $(C,p)$.
    \end{notation}

    \newdef{Covering transformation}{\label{topology:covering_transformation}
        Consider two covering spaces $(C,p)$ and $(C',p')$. A continuous function $f:C\rightarrow C'$ is called a covering transformation if $p'\circ f=p$.
    }

    \newdef{Deck transformation}{\index{deck transformation}\label{topology:deck_transformation}
        Let $p:C\rightarrow X$ be a covering map. The automorphism group of $(C,p)$ in the category of covering spaces (over $X$) is given by all homeomorphisms $\varphi$ satisfying $p\circ\varphi=p$. These automorphisms are called deck transformations.
    }

    \newdef{\'Etal\'e space}{\index{etale!space}\index{stalk}\label{topology:etale_space}
        Let $X$ be a topological space. A topological space $Y$ is called an \'etal\'e space over $X$ if there exists a continuous surjection $\pi:Y\rightarrow X$ such that $\pi$ is a local homeomorphism. The preimage $\pi^{-1}(x)$ of a point $x\in X$ is called the \textbf{stalk} of $\pi$ over $x$.
    }
    \begin{example}
        Every covering space is an \'etal\'e space.
    \end{example}

    \newdef{\difficult{Pseudogroup}}{\index{pseudo!group}\label{topology:pseudogroup}
        Let $X$ be a topological space. A pseudogroup is a collection $\mathcal{G}$ of homeomorphisms $\phi:U\rightarrow V$ between open subsets of $X$ such that:
        \begin{enumerate}
            \item $\mathbbm{1}_U\in\mathcal{G}$ for all open $U\subseteq X$.
            \item If $\phi\in\mathcal{G}$, then $\phi^{-1}\in\mathcal{G}$.
            \item If $V\subset U$ is open, then $\phi|_V\in\mathcal{G}$.
            \item If $U=\bigcup_{i\in I}U_i$ and $\phi|_{U_i}:U_i\rightarrow V$ is an element of $\mathcal{G}$ for all $i\in I$, then $\phi\in\mathcal{G}$.
            \item If $\phi:U\rightarrow V$ and $\psi:U'\rightarrow V'$ are elements of $\mathcal{G}$ and $V\cap U'\neq\emptyset$, then $\psi\circ\phi|_{\phi^{-1}(V\cap U')}\in\mathcal{G}$.
        \end{enumerate}
    }

\section{Constructions}

    \begin{construct}[Product topology]\index{product!topology}\index{Tychonoff!topology}\label{topology:tychonoff_topology}
        First, consider the case with only a finite number of spaces $\{X_i\}_{i\in I}$. The Cartesian product $X:=\prod_{i\in I}X_i$ can be turned into a topological space by equipping it with the topology generated by the following basis:
        \begin{gather}
            \mathcal{B} := \left\{\prod_{i\in I}U_i\,\middle\vert\,U_i\in\tau_i\right\}.
        \end{gather}
        In the general case the topology can be defined using the canonical projections $\pi_i:X\rightarrow X_i$. The general product topology, called the \textbf{Tychonoff topology}, is the initial topology with respect to the projections $\pi_i$.
    \end{construct}

    \begin{construct}[Disjoint union]\index{disjoint union}\label{topology:disjoint_union}
        Let $\{X_i\}_{i\in I}$ be a family of topological spaces and consider the disjoint union
        \begin{gather}
            X := \bigsqcup_{i\in I}X_i
        \end{gather}
        together with the canonical inclusion maps $\phi_i:X_i\rightarrow X:x_i\mapsto(i,x_i)$. The set $X$ can be turned into a topological space by equipping it with the following topology:
        \begin{gather}
            \tau_X := \big\{U\subseteq X\,\big\vert\,\forall i\in I:\phi_i^{-1}(U)\text{ is open in }X_i\big\}.
        \end{gather}
    \end{construct}

    \begin{construct}[Quotient space]\index{quotient!space}
        Consider a topological space $X$ and a subset $Y\subseteq X$. The quotient $X/Y$ is defined as the set $X\backslash Y\sqcup\{\ast\}$ where the point $\ast$ can be regarded as the result of identifying all points in $Y$. This canonically turns the quotient space into a pointed space.

        Let $\pi$ be the canonical projection $X\rightarrow X/Y$. The quotient space can be turned into a topological space by equipping it with the following topology:
        \begin{gather}
            \label{topology:quotient_space}
            \tau_q := \big\{U\subseteq X/Y\,\big\vert\,\pi^{-1}(U)\text{ is open in }X\big\}.
        \end{gather}
    \end{construct}
    \begin{remark}[Degenerate quotient]
        For the degenerate case $Y=\emptyset$ one can also apply the above definition. However, this has the awkward effect that it adjoins a new point to the space $X$ instead of a collapsing it:
        \begin{gather}
            \label{topology:empty_quotient}
            X/\emptyset = X\sqcup\ast.
        \end{gather}
    \end{remark}

    \begin{construct}[Wedge sum]\index{wedge!sum}
        Consider two pointed spaces $(X,x_0),(Y,y_0)$. The wedge sum $X\vee Y$ is defined as the quotient of the disjoint union $X\sqcup Y$ obtained by identifying the basepoints $x_0\sim y_0$.
    \end{construct}
    \newdef{Smash product}{\index{smash product}
        Consider two pointed topological spaces $(X,x_0),(Y,y_0)$. The smash product $X\wedge Y$ is defined as the quotient
        \begin{gather}
            X\wedge Y := (X\times Y)/(X\vee Y),
        \end{gather}
        where $X\vee Y$ sits inside the product as the union of $X\times\{y_0\}$ and $\{x_0\}\times Y$.
    }

    \begin{construct}[Suspension]\index{suspension}\label{topology:suspension}
        Let $X$ be a topological space. The suspension of $X$ is defined as the following quotient space:
        \begin{gather}
            SX := (X\times [0,1])/\big\{(x,0)\sim (y,0)\text{ and }(x,1)\sim (y,1)\,\big\vert\,x,y\in X\big\}.
        \end{gather}
        By the remark about degenerate quotients the suspension of the empty set is in fact not empty, but equal to the two-point space $S^0$.

        An often more interesting construction is the \textbf{reduced suspension} $\Sigma X$. This is obtained by taking the ordinary suspension $SX$ of a pointed space $(X,x_0)$ and identifying all copies of $x_0$:
        \begin{gather}
            \Sigma X := SX/(x_0\times[0,1]).
        \end{gather}
        An equivalent definition of the reduced suspension can be given in terms of the smash product:
        \begin{gather}
            \Sigma X = X\wedge S^1.
        \end{gather}
    \end{construct}
    \begin{example}[Spheres]\label{topology:sphere_suspension}
        Up to homeomorphisms the spheres are related by (reduced) suspensions:
        \begin{gather}
            SS^n\cong S^{n+1}\cong\Sigma S^n.
        \end{gather}
        If one identifies the empty set with the $(-1)$-sphere, this relation can be continued to the case $n=-1$.
    \end{example}

    \begin{construct}[Attaching space]\index{attaching space}\label{topology:attaching_space}
        Let $X,Y$ be two topological spaces and consider a subspace $A\subseteq X$. For every continuous function $f:A\rightarrow Y$, called the \textbf{attaching map}, one can construct the attaching space (or \textbf{adjunction space}) $X\cup_f Y$ in the following way:
        \begin{gather}
            X\cup_f Y := (X\sqcup Y)/\{A\sim f(A)\}.
        \end{gather}
        In categorical terms it is the pushout \ref{cat:pushout} in $\mathbf{Top}$ of the inclusion $\iota:A\hookrightarrow X$ along $f:A\rightarrow Y$.
    \end{construct}

    \begin{construct}[Join]\index{join}\label{topology:join}
        Let $\{A_i\}_{i\leq n}$ be a finite collection of topological spaces. The join, denoted by $A=A_1\circ\cdots\circ A_n$, is defined as follows. Every point of $A$ is defined by the following data:
        \begin{enumerate}
            \item an element of the standard $n$-simplex \ref{topology:simplex}, i.e. an $n$-tuple of nonnegative numbers $\{t_i\}_{i\leq n}$ satisfying $\sum_it_i=1$;
            \item for each index $i$ such that $t_i\neq 0$, a point $a_i\in A_i$.
        \end{enumerate}
        This point in $A$ is denoted by $t_1a_1\oplus\cdots\oplus t_na_n$.

        In the case of two spaces there exists a more intuitive construction. Let $A,B$ be two topological spaces. The join $A\circ B$ is equal to the quotient space $(A\times B\times[0,1])/\sim$, where the relation $\sim$ is defined as follows:
        \begin{itemize}
            \item For all $a\in A$ and $b,b'\in B$: $(a,b,0)\sim(a,b',0)$.
            \item For all $a,a'\in A$ and $b\in B$: $(a,b,1)\sim(a',b,1)$.
        \end{itemize}
        This can be interpreted as collapsing one end of the cylinder $(A\times B)\times[0,1]$ to $A$ and the other end to $B$.
    \end{construct}
    \begin{property}[\difficult{Monoidal structure}]
        The join induces a monoidal structure on the category \textbf{Top} where the tensor unit is given by the empty space $\emptyset$.
    \end{property}

\section{Connected spaces}

    \newdef{Connected space}{\index{connected}\label{topology:connected}
        A topological space that cannot be written as the disjoint union of two non-empty open sets. Equivalently, a space is connected if the only clopen sets are the empty set and the space itself.
    }

    \begin{property}[Locally constant implies constant]
        Let $X$ be a connected space and let $f$ be a function on $X$. If $f$ is locally constant, i.e. for every $x\in X$ there exists a neighbourhood U on which $f$ is constant, then $f$ is constant on all of $X$.
    \end{property}

    \begin{theorem}[Intermediate value theorem]\index{intermediate value theorem}\label{topology:intermediate_value_theorem}
        Let $X$ be a connected space and let $f:X\rightarrow\mathbb{R}$ be a continuous function. If $a,b\in f(X)$, then for every $c\in\ ]a,b[\ :c\in f(X)$.
    \end{theorem}

    \newdef{Path-connected space\protect\footnotemark}{\index{arc!connected|see{path-connected}}\index{path!connected}
        \footnotetext{A similar notion is that of \textbf{arcwise-connectedness} where the function $\varphi$ is required to be a homeomorphism.}
        Let $X$ be a topological space. If for every two points $x,y\in X$ there exists a continuous function $\varphi:[0, 1]\rightarrow X$ (i.e. a \textbf{path}) such that $\varphi(0)=x$ and $\varphi(1)=y$, then the space is said to be path-connected.
    }

    \begin{property}[Path-connected implies connected]
        Every path-connected space is connected. The converse does not hold. A connected and locally path-connected space is path-connected.
    \end{property}

    \begin{remark}[Connected components]\label{topology:connected_components}
        (Path-)connectedness defines an equivalence relation on the space $X$. The equivalence classes are closed in $X$ and form a cover of $X$. The set of path components of $X$ is often denoted by $\pi_0(X)$.
    \end{remark}

\section{Compact spaces}\label{section:compact}
\subsection{Compactness}

    \newdef{Sequentially compact space}{
        A topological space in which every sequence has a convergent subsequence (the sequence itself does not have to be convergent).
    }

    \newdef{Finite intersection property}{\index{finite!intersection property}
        A collection $\mathcal{F}\subseteq2^X$ of subsets has the finite intersection property (FIP) if
        \begin{gather}
            \bigcap_{V\in\mathcal{F}'}V\neq\emptyset
        \end{gather}
        for all finite $\mathcal{F}'\subset\mathcal{F}$.
    }

    \newdef{Locally finite cover}{
        An open cover of a topological space $X$ is said to be locally finite if every $x\in X$ has a neighbourhood that intersects only finitely many sets in the given cover.
    }

    \begin{property}[First-countable spaces]
        A first-countable space is sequentially compact if and only if every countable open cover has a finite subcover.
    \end{property}

    \newdef{Lindel\"of space}{\index{Lindel\"of!space}
        A space for which every open cover has a countable subcover.
    }
    \begin{property}
        Every second-countable space is a Lindel\"of space.
    \end{property}

    \newdef{Compact space}{\index{compact}
        A topological space for which every open cover of has a finite subcover.
    }

    \begin{theorem}[Heine-Borel\footnotemark]\index{Heine-Borel}\index{Borel-Lebesgue}\label{topology:heine_borel}
        \footnotetext{Also called the \textbf{Borel-Lebesgue} theorem.}
        If a topological space $X$ is sequentially compact and second-countable, every open cover has a finite subcover and, therefore, $X$ is compact.
    \end{theorem}
    \begin{result}[Real numbers]
        A subset of $\mathbb{R}^n$ is compact if and only if it is closed and bounded.
    \end{result}

    \begin{theorem}[Tychonoff's theorem]\index{Tychonoff!compactness theorem}
        Any product of compact topological spaces is compact under the (Tychonoff) product topology \ref{topology:tychonoff_topology}.
    \end{theorem}

    \newdef{Relatively compact space}{\label{topology:relatively_compact}
        A topological space for which its closure is compact.
    }

    \newdef{Locally compact space}{
        A topological space in which every point has a compact neighbourhood.
    }

    \begin{theorem}[Dini]\index{Dini}
        Let $(X,\tau)$ be a compact space and let $\seq{f}$ be a monotone sequence of continuous functions $f_n:X\rightarrow\mathbb{R}$. If $f_n\longrightarrow f$ pointwise to a continuous function $f$, the convergence is uniform.
    \end{theorem}

    \newdef{$\omega$-bounded space}{
        A topological space in which the closure of every countable subset is compact.
    }

    \newdef{Paracompact space}{\index{paracompactness}\label{topology:paracompact}
        A topological space for which every open cover has a locally finite open refinement.
    }

    \begin{property}
        Every paracompact Hausdorff space is normal.
    \end{property}

    \newdef{Partition of unity}{\index{partition!of unity}\label{topology:partition_of_unity}
        A collection $\{f_i:X\rightarrow[0,1]\}_{i\in I}$ of continuous functions such that for every $x\in X$ the following conditions hold:
        \begin{enumerate}
            \item\textbf{Locally finite}: For every neighbourhood $U$ of $x$, the set $\{f_i\mid\text{supp}f_i\cap U \neq \emptyset\}$ is finite.
            \item\textbf{Normalization}: $\sum_if_i = 1$.
        \end{enumerate}
        Consider an open cover $\{V_i\}_{i\in I}$ of $X$. If there exists a partition of unity, also indexed by $I$, such that $\text{supp}(\varphi_i)\subseteq U_i$, then this partition of unity is said to be \textbf{subordinate} to the given cover.
    }

    \begin{property}[Hausdorff spaces]\label{topology:paracompact_partition_unity}
        A paracompact space is Hausdorff if and only if it admits a partition of unity subordinate to any open cover.
    \end{property}

    \newdef{Numerable open cover}{\index{numerable}
        An open cover of a topological space is said to be numerable if the space admits a partition of unity subordinate to the given cover.
    }

    \newdef{Compact-open topology}{\index{topology!compact-open}
        Consider the mapping space $C(X,Y)$ between two topological spaces. This space is often endowed with a topology generated by the subbasis of subsets of the form
        \begin{gather}
            U^K := \{f:X\rightarrow Y\mid K\text{ compact}, U\text{ open and } f(K)\subseteq U\}.
        \end{gather}
    }
    \begin{property}[Internal hom]\index{mapping!space}\label{topology:internal_hom}
        Consider two topological spaces $X,Y$ with $X$ locally compact and equip the mapping space $C(X,Y)$ with the compact-open topology. The following relation is satisfied for all topological spaces $Z$:
        \begin{gather}
            C(Z\times X,Y)\cong C(Z,C(X,Y)),
        \end{gather}
        i.e. the mapping space $C(X,Y)$ is an internal hom \ref{cat:internal_hom} in the category $\mathbf{Top}$ and, because the topological product is the product in $\mathbf{Top}$, $C(X,Y)$ is even an exponential object \ref{cat:exponential_object}. For this reason the mapping spaces $C(X,Y)$ are also sometimes denoted by $Y^X$.
    \end{property}

\subsection{Compactifications}

    \newdef{Dense}{\index{dense}
        A subset $V\subseteq X$ is said to be dense in a topological space $X$ if $\overline{V}=X$.
    }
    \newdef{Separable space}{\index{separable!space}\label{topology:separable}
        A topological space that contains a countable, dense subset.
    }
    \begin{property}
        Every second-countable space is separable.
    \end{property}

    \newdef{Compactification}{\index{compactification}
        A compact topological space $(X',\tau')$ is a compactification of a topological space $(X,\tau)$ if $X$ is a dense subspace of $X'$.
    }

    \begin{example}
        Standard examples of compactifications are the extended real line $\mathbb{R}\cup\{-\infty,+\infty\}$ and the extended complex plane $\mathbb{C}\cup\{\infty\}$ for the real line and the complex plane, respectively.
    \end{example}
    \begin{remark*}
        It is important to note that compactifications are not necessarily unique.
    \end{remark*}

    \newdef{One-point compactification}{\index{Alexandrov compactification}\label{topology:alexandrov_compactification}
        Let $X$ be a Hausdorff space. A one-point compactification or \textbf{Alexandrov compactification} is a compactification $\widehat{X}$ such that $\widehat{X}\setminus X$ is a singleton.
    }
    \begin{example}[Real line]
        The classic example of a (one-point) compactification is that of the real line. By adjoining the points $\pm\infty$ and identifying them, the circle $S^1$ is obtained. In general one can obtain the $n$-dimensional sphere $S^n$ as the one-point compactification of $\mathbb{R}^n$. This can be regarded as an \textit{inverse stereographic projection}.
    \end{example}

\section{Uniform spaces}

    \newdef{Uniform structure}{
        Consider a set $X$. A uniform structure on $X$ consists of a collection $\mathfrak{U}$ of subsets $U\subseteq X\times X$ that satisfy the following properties:
        \begin{enumerate}
            \item If $U\in\mathfrak{U}$ and $U\subset V$, then $V\in\mathfrak{U}$.
            \item If $U, V\in\mathfrak{U}$, then $U\cap V\in\mathfrak{U}$.
            \item If $U\in\mathfrak{U}$, then $\Delta_X\subset U$.
            \item If $U\in\mathfrak{U}$, there exists $V\in\mathfrak{U}$ such that $V\circ V=U$.
            \item If $U\in\mathfrak{U}$, then $U^t\in\mathfrak{U}$.
        \end{enumerate}
        The ``transpose'' $U^t$ denotes the converse \ref{set:converse} of $U$ and the composition $\circ$ is the relational composition \ref{set:relational_composition} of $V$ and $V$. The elements of the uniformity $\mathfrak{U}$ are called \textbf{entourages}. If $(x,y)\in U$ for some entourage $U\in\mathfrak{U}$, then $x$ and $y$ are said to be \textbf{$U$-close}.
    }
    \remark{The first three conditions imply that a uniform structure is in particular a filter.}

    ?? COMPLETE (Bourbaki) ??

\section{\texorpdfstring{Locales $\clubsuit$}{Locales}}

    \begin{property}[Opens form a frame]
        Consider the poset $\mathbf{Open}(X)$ of opens of a topological space $X$. This set is closed under finite intersections (limits) and arbitrary unions (colimits). Furthermore, arbitrary unions distribute over finite intersections:
        \begin{gather}
            V\cap\left(\bigcup_{i\in I}U_i\right) = \bigcup_{i\in I}\left(V\cap U_i\right).
        \end{gather}
        This implies that the poset $\mathbf{Open}(X)$ is a frame \ref{set:frame}.
    \end{property}

    \newdef{Locale}{\index{locale}
        The previous property can be used to generalize the notion of topological spaces to include ``pointless spaces''. Let \textbf{Frame} denote the category of frames together with frame homomorphisms. The category of locales is defined as the opposite category: \[\mathbf{Loc} := \mathbf{Frame}^{op}.\]
    }
    \begin{construct}[From locale to topological space]
        There exists an adjunction \[\mathbf{Loc}\adj{\iota}{\mathrm{Point}}\mathbf{Top},\] where the right adjoint is defined as follows:
        \begin{quote}
            Let $L$ be a locale. For a topological space the points are given by continuous functions $\ast\rightarrow X$ and, hence, by frame morphisms $\mathbf{Open}(X)\rightarrow1\equiv\Omega_{\mathrm{Frame}}=\{0, 1\}$. Generalizing this to locales, one defines the set of points of $L$ as the $\Omega_\mathrm{Loc}$-elements: \[\mathrm{Point}(L) := \mathbf{Loc}(1,L).\] This set can be given a topology by declaring for every $U\in L$ the set $\{p\in\mathrm{Point}(L)\mid p^{-1}(U) = 1\}$ to be open.
        \end{quote}
    \end{construct}

    \newdef{Sober space}{\index{sober}\label{topology:sober_space}
        A topological space $X$ such that the map $X\rightarrow\mathrm{Point}(X)$ is a homeomorphism, i.e. the points of $X$ are precisely determined by its frame of opens. Equivalently, a topological space such that every irreducible closed subset is the closure of a unique point. Important examples are Hausdorff spaces.
    }

    ?? COMPLETE ??
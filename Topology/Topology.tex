\chapter{Topology}
\section{Topological spaces}

	\newdef{Topology}{\index{topology}
    	Let $\Omega$ be a set. Let $\tau\subseteq 2^\Omega$. The set $\tau$ is a topology on $\Omega$ if it satisfies following axioms:
        \begin{enumerate}
        	\item $\emptyset\in\tau$ and $\Omega\in\tau$
            \item $\forall\ \mathcal{F}\subseteq\tau: \bigcup_{V\in\mathcal{F}}V \in \tau$
            \item $\forall\ U, V\in\tau: U\cap V\in\tau$
        \end{enumerate}
        Furthermore we call the elements of $\tau$ open sets and the couple $(\Omega, \tau)$ a topological space.
    }
    \sremark{On topological spaces the open sets are thus defined by axioms.}

	\newdef{Relative topology}{
		Let $(X, \tau_X)$ be a topological space and $Y$ a subset of $X$. We can turn $Y$ into a topological space by equipping it with the following topology, called the relative topology:
		\eq{
			\label{topology:relative_topology}
			\tau_\text{rel} = \{U_i\cap Y:U_i\in \tau_X\}
		}
	}
	
	\newdef{Disjoint union}{\index{disjoint union}\label{topology:disjoint_union}
		Let $\{X_i\}_{i\in I}$ be a family of topological spaces. Now consider the disjoint union
		\begin{equation}
			X = \bigsqcup_{i\in I} X_i
		\end{equation}
		together with the canonical inclusion maps $\phi_i:X_i\rightarrow X:x_i\mapsto(x_i, i)$. We can turn $X$ into a topological space by equipping it with the following topology:
		\begin{equation}
			\tau_X = \{U\subseteq X| \forall i\in I:\phi_i^{-1}(U)\text{ is open in }X_i\}
		\end{equation}
	}

	\newdef{Quotient space}{\index{quotient!space}
		Let $X$ be a topological space and let $\sim$ be an equivalence relation defined on $X$. The set $X/_\sim$ can be turned into a topological space by equipping it with the following topology:
		\begin{equation}
			\label{topology:quotient_space}
			\tau_\sim = \{U\subseteq X/_\sim|\pi^{-1}(U)\text{ is open in }X\}
		\end{equation}
		where $\pi$ is the canonical surjective map from $X$ to $X/_\sim$.
	}

	\begin{example}[Discrete topology]\index{discrete!topology}
		The discrete topology is the topology such that every subset is open (and thus also closed).
	\end{example}
	\begin{example}[Product topology]\index{product!topology}\index{Tychonoff!product topology}
		First consider the case where the index set $I$ is finite. The product space $X = \prod_{i\in I}X_i$ can be turned into a topological space by equipping it with the topology generated by the following basis:
		\[
			\mathcal{B} = \left\{\left.\prod_{i\in I}U_i\ \right|U_i\in\tau_i\right\}
		\]
		For general cases (countably infinite and uncountable index sets) the topology can be defined using the canonical projections $\pi_i:X\rightarrow X_i$. The general product topology (also called Tychonoff topology) is the coarsest (finest) topology such that all projections $\pi_i$ are continuous.
	\end{example}
	
	\newdef{Topological group}{\index{group!topology}
		A topological group is a group $G$ equipped with a topology such that both the multiplication and inversion map are continuous.
	}
	
	
	\newdef{Pointed topological space}{
		Let $x_0\in X$. The triple $(X, \tau, x_0)$ is called a pointed topological space with base point $x_0$.
	}
	\newdef{Suspension}{\index{suspension}
		Let $X$ be a topological space. The suspension of $X$ is defined as the following quotient space:
		\begin{equation}
			\label{topology:suspension}
			SX = 	(X\times [0, 1])/\{(x, 0) \sim (y, 0)\text{ and }(x, 1) \sim (y, 1)|x, y\in X\}
		\end{equation}
	}
    
\section{Neighbourhoods}

	\newdef{Neighbourhood}{\index{neighbourhood}
		A set $V\subseteq\Omega$ is a neighbourhood of a point $a\in\Omega$ if there exists an open set $U\in\tau$ such that $a\in U\subseteq V$.
	}
    
    \newdef{Basis}{\index{basis}
    	Let $\mathcal{B}\subseteq\tau$ be a family of open sets. The family $\mathcal{B}$ is a basis for the topological space $(\Omega, \tau)$ if every $U\in\tau$ can be written as:
        \begin{equation}
        	U = \bigcup_{V\in\mathcal{F}}V
        \end{equation}
        where $\mathcal{F}\subseteq\mathcal{B}$.
    }
    \newdef{Local basis}{
    	Let $\mathcal{B}_x$ be a family of open neighbourhoods of a point $x\in\Omega$. $\mathcal{B}_x$ is a local basis of $x$ if every neighbourhood of $x$ contains at least one element in $\mathcal{B}_x$.
    }
    
    \newdef{First-countability}{\index{countable!countable space}
    	A topological space $(\Omega, \tau)$ is first-countable if for every point $x\in\Omega$ there exists a countable local basis.
    }
	\begin{property}[Decreasing basis]
		Let $x\in\Omega$. If there exists a countable local basis for $x$ then there also exists a countable decreasing local basis for $x$.
	\end{property}
    
    \newdef{Second-countability}{
    	A topological space $(\Omega, \tau)$ is second-countable if there exists a countable global basis.
    }
    
    \begin{property}
    	Let $X$ be a topological space. The closure of a subset $V$ is given by:
    	\eq{
    		\label{topology:closure}
    		\overline{V} = \{x\in X| \exists \text{ a net } (x_\lambda)_{\lambda\in I} \text{ in } X:x_\lambda\rightarrow x\}
    	}
    	This implies that the topology on $X$ is completely determined by the convergence of nets\footnote{See definition \ref{set:net}.}.
    \end{property}
    \begin{result}
    	In first-countable spaces we only have to consider the convergence of sequences.
    \end{result}
    
	\newdef{Germ}{\index{germ}\label{topology:germ}
		Let $X$ be a topological space and let $Y$ be a set. Consider two functions $f, g: X\rightarrow Y$. If there exists a neighbourhood $U$ of a point $x\in X$ such that
		\[f(u) = g(u)\qquad\qquad\forall u\in U\]
		then this property defines an equivalence relation denoted by $f\sim_x g$ and the equivalence classes are called \textbf{germs}.
	}
	
	\begin{property}
		Let the set $Y$ in the previous definition be the set of reals $\mathbb{R}$. Then the germs at a point $p\in X$ satisfy following closure/linearity relations:
		\begin{itemize}
			\item $[f] + [g] = [f+g]$
			\item $\lambda[f] = [\lambda f]$
			\item $[f][g] = [fg]$
		\end{itemize}
		where $[f], [g]$ are two germs at $p$ and $\lambda\in\mathbb{R}$ is a scalar.
	\end{property}
   
\subsection{Separation axioms}

	\newdef{Hausdorff space}{\index{Hausdorff!space}\index{separation axiom}
		A topological space is a Hausdorff space or $T_2$ space if it satisfies the following axiom:
		\begin{equation}
			(\forall x, y \in\Omega)(\exists \text{ neighbourhoods }U, V)(x\in U, y\in V, U\cap V=\emptyset)
		\end{equation}
		This axiom is called the \textbf{separation axiom of Hausdorff}.
	}
    	\begin{property}
		Every singleton (and thus also every finite set) is closed in a Hausdorff space.
	\end{property}

	\newdef{Regular space}{\index{regular}\label{topology:regular}
		A topological space is said to be regular if for every closed subset $F$ and every point $x\not\in F$ there exist disjoint open subsets $U, V$ such that $x\in U$ and $F\subset V$.
	}
	\begin{notation}
		A space that is both regular and Hausdorff is a $T_3$ space.
	\end{notation}

	\newdef{Normal space}{\index{normal}\label{topology:normal}
		A topological space is said to be normal if every two closed subsets have disjoint neighbourhoods.
	}
	\begin{notation}
		A space that is both normal and Hausdorff is a $T_4$ space.
	\end{notation}
    
\section{Convergence and continuity}

	\newdef{Convergence}{\index{convergence}
    	A sequence $(x_n)_{n\in\mathbb{N}}$ in $X$ is said to converge to a point $a\in X$ if:
        \begin{equation}
        	(\forall \text{ neighbourhoods } V \text{ of } a)(\exists\ N > 0)(\forall\ n>N)(x_n\in V)
        \end{equation}
    }
    \begin{property}
    	Every subsequence of a converging sequence converges to the same point\footnote{This limit does not have to be unique. See the next property for more information.}.
    \end{property}
    \begin{property}
    	\label{topology:theorem:hausdorff_limit}
    	Let $X$ be a Hausdorff space. The limit of a converging sequence in $X$ is unique.
    \end{property}
    
	\newdef{Continuity}{\index{continuity}
    		A function $f:X\rightarrow Y$ is continuous if the inverse image $f^{-1}(U)$ of every open set $U$ is also open.
	}
	\begin{theorem}
    		Let $X$ be a first-countable space. Consider a function $f:X\rightarrow Y$. The following statements are equivalent:
        	\begin{itemize}
        		\item $f$ is continuous
        		\item The sequence $(f(x_n))_{n\in\mathbb{N}}$ converges to $f(a)\in Y$ whenever the sequence $(x_n)_{n\in\mathbb{N}}$ converges to $a\in X$.
	        \end{itemize}
	\end{theorem}
    \begin{result}
    	If the space $Y$ in the previous theorem is Hausdorff then the limit $f(a)$ does not need to be known because the limit is unique (see \ref{topology:theorem:hausdorff_limit}).
    \end{result}
    \begin{remark}
    	If the space $X$ is not first-countable, we have to consider the convergence of nets \ref{set:net}.
    \end{remark}
    
	\begin{theorem}[Urysohn's lemma]\index{Urysohn!lemma}
    		A topological space X is normal\footnotemark\ if and only if every two closed disjoint subsets $A, B\subset X$ can be separated by a continuous function $f:X\rightarrow [0, 1]$ i.e.
	    	\begin{equation}
    			\label{topology:urysohns_lemma}
    			f(a) = 0, \forall a\in A\qquad\qquad f(b) = 1, \forall b\in B
    		\end{equation}
    		\footnotetext{See definition \ref{topology:normal}.}
	\end{theorem}
    
\subsection{Homeomorphisms}
	
	\newdef{Homeomorphism}{\index{homeomorphism}
		A function $f$ is called a homeomorphism if both $f$ and $f^{-1}$ are continuous and bijective.
	}
	\newdef{Embedding}{\index{embedding}
		\label{topology:embedding}
		A function is an embedding if it is homeomorphic onto its image.
	}
	
\section{Connectedness}
	
	\newdef{Connected space}{\index{connected}\label{topology:connected}
		A topological space $X$ is connected if it cannot be written as the disjoint union of non-empty open sets. Equivalently, $X$ is connected if the only clopen sets are $X$ and $\emptyset$.
	}
	
	\begin{property}
		Let $X$ be a connected space. Let $f$ be a function on $X$. If $f$ is locally constant, i.e. for every $x\in X$ there exists a neighbourhood U on which $f$ is constant, then $f$ is constant on all of $X$.
	\end{property}
	
	\begin{theorem}[Intermediate value theorem]\index{intermediate value theorem}
		Let $X$ be a connected space. Let $f:X\rightarrow\mathbb{R}$ be a continuous function. If $a, b\in f(X)$ then for every $c\in ]a, b[$ we have that $c\in f(X)$.
	\end{theorem}

	\newdef{Path-connected space\protect\footnotemark}{\index{arc}
		\footnotetext{A similar notion is that of \textbf{arcwise-connectedness} where the function $\varphi$ is required to be a homeomorphism.}
		Let $X$ be a topological space. If for every two points $x, y\in X$ there exists a continuous function $\varphi:[0, 1]\rightarrow X$ (i.e. a \textbf{path}) such that $\varphi(0)=x$ and $\varphi(1)=y$ then the space is said to be path-connected.
	}
	
	\begin{property}
		Every path-connected space is connected.
	\end{property}
	The converse of previous property does not hold. There exists however the following (stronger) relation.
	\begin{property}
		A connected and locally path-connected space is path-connected.
	\end{property}
	
	\begin{remark}
		The notions of connectedness and path-connectedness define equivalence relations on the space $X$. The equivalence classes are closed in $X$ and form a cover of $X$.
	\end{remark}

\section{Compactness}
	
	\newdef{Sequentially compact}{
    		A topological space is sequentially compact if every sequence\footnotemark\ has a convergent subsequence.
    		\footnotetext{The sequence itself does not have to converge.}
	}
    
	\newdef{Finite intersection property}{\index{finite intersection property}
    		A family $\mathcal{F}\subseteq2^X$ of subsets has the finite intersection property\footnotemark\ if every finite subfamily has a non-zero intersection:
        	\begin{equation}
        		\bigcap_{i\in I}V_i \neq \emptyset
        	\end{equation}
        	for all finite index sets $I$.
        	\footnotetext{The family is then called a FIP-family.}
	}

	\newdef{Locally finite cover}{
		An open cover of a topological space $X$ is said to be locally finite if every $x\in X$ has a neighbourhood that intersects only finitely many sets in the cover of $X$.
	}
    
    \begin{property}
    	A first-countable space is sequentially compact if and only if every countable open cover has a finite subcover.
    \end{property}
    
    \newdef{Lindel\"of space}{\index{Lindel\"of!space}
    	A space for which every open cover has a countable subcover.
    }
    \begin{property}
    	Every second-countable space is also a Lindel\"of space.
    \end{property}
    
    \newdef{Compact space}{\index{compactness}
    	A topological space $X$ is compact if every open cover of $X$ has a finite subcover.
    }
    
	\begin{theorem}[Heine-Borel\footnotemark]\index{Heine-Borel}
	    	\footnotetext{Also Borel-Lebesgue.}
	    	If a topological space $X$ is sequentially compact and second-countable then every open cover has a finite subcover. This implies that $X$ is compact.
	\end{theorem}
	\begin{theorem}[Heine-Borel on $\mathbb{R}^n$]
    		A subset of $\mathbb{R}^n$ is compact if and only if it is closed and bounded.
	\end{theorem}

	\newdef{Locally compact}{
    		A topological space is locally compact if every point $x\in X$ has a compact neighbourhood.
	}
    
	\begin{theorem}[Dini's theorem]\index{Dini!theorem}
    		Let $(X, \tau)$ be a compact space. Let $(f_n)_{n\in\mathbb{N}}$ be an increasing sequence of continuous functions $f_n:X\rightarrow\mathbb{R}$. If $(f_n)_n\rightarrow f$ pointwise to a continuous function $f$ then the convergence is uniform.
	\end{theorem}
    
	\newdef{Paracompact space}{\label{topology:paracompact}
		A topological space is paracompact if every open cover has a locally finite open refinement.
	}

	\newprop{$\omega$-boundedness}{
		Let $X$ be a topological space. $X$ is said to be $\omega$-bounded if the closure of every countable subset is compact.
	}
 
	\newdef{Partition of unity}{\index{partition!of unity}\label{topology:partition_of_unity}
		Let $\{\varphi_i: X\rightarrow [0, 1]\}_i$ be a collection of continuous functions such that for every $x\in X$:
		\begin{itemize}
			\item For every neighbourhood $V$ of $x$, the set $\{f_i:\text{supp}f_i\cap U \neq \emptyset\}$ is finite.
			\item $\sum_if_i = 0$
		\end{itemize}
	}
	\begin{definition}
		Consider an open cover $\{V_i\}_{i\in I}$ of $X$, indexed by a set $I$. If there exists a partition of unity, also indexed by $I$, such that $\text{supp}(\varphi_i)\subseteq U_i$, then this partition of unity is said to be \textbf{subordinate} to the open cover.
	\end{definition}

\subsection{Compactifications}
	\newdef{Dense}{\index{dense}
		A subset $V\subseteq X$ is dense in a topological space $X$ if $\overline{V} = X$.
	}
	\newdef{Separable space}{\index{separable}
		\label{topology:separable}
		A topological space is separable if it contains a countable dense subset.
	}
	\begin{property}
		Every second-countable space is separable.
	\end{property}
    
    \newdef{Compactification}{\index{compactification}
    	A compact topological space $(X', \tau')$ is a compactification of a topological space $(X, \tau)$ if $X$ is a dense subspace of $X'$.
    }
    
    \begin{example}
    	Standard examples of compactifications are the extended real line $\mathbb{R} \cup \{-\infty, +\infty\}$ and the extended complex plane $\mathbb{C}\cup\{\infty\}$ for the real line and the complex plane respectively.
    \end{example}
    \begin{remark*}
    	It is important to note that compactifications are not unique.
    \end{remark*}
    
    \newdef{One-point compactification}{\index{Alexandrov compactification}\label{topology:alexandrov_compactification}
    	Let $X$ be a Hausdorff space. A one-point compactification or \textbf{Alexandrov compactification} is a compactification $X'$ such that $X'\setminus X$ is a singleton.
    }



    
\section{Homotopy theory}

	\newdef{Homotopy}{\index{homotopy}
		Let $f, g: X\rightarrow Y$ be continuous functions between topological spaces. If there exists a continuous function $H:X\times [0, 1]\rightarrow Y$ such that $f(x) = H(x, 0)$ and $g(x) = H(x, 1)$ then $f$ and $g$ are said to be homotopic. This relation also induces an equivalence relation on $\mathcal{C}(X, Y)$.
	}
	
	\newdef{Homotopy type}{
		Let $X, Y$ be two topological spaces. $X$ and $Y$ are said to be homotopy equivalent, or of the same homotopy type, if there exist functions $f:X\rightarrow Y$ and $g:Y\rightarrow X$ such that $f\circ g$ is homotopic to $\mathbbm{1}_Y$ and $g\circ f$ is homotopic to $\mathbbm{1}_X$. The maps $f, g$ are called \textbf{homotopy equivalences}.
	}
	\begin{property}\index{homeomorphism}
		Every homeomorphism is a homotopy equivalence.
	\end{property}
	
	\newdef{Null-homotopic}{
		A continuous function is null-homotopic if it is homotopic to a constant function.
	}
	\newdef{Contractible space}{\index{contractible}
		\label{topology:contractible_space}
		A topological space $X$ is said to be contractible if the identity map $\mathbbm{1}_X$ is null-homotopic. Equivalently, the space is homotopy-equivalent to a point.
	}
	
\subsection{Fundamental group}

	In this subsection we will always assume to be working with pointed spaces. The base point will be denoted by $x_0$.
	
	\newdef{Loop space}{
		The space of all based loops in $X$. It is denoted by $\Omega X$. We can define a multiplication operation on $\Omega X$ corresponding to the concatenation of loops\footnote{It should be noted that the speed at which the concatenated loops are traversed is doubled because the parameter $t$ should remain an element of $[0, 1]$.}.
	}
	
	\newdef{Fundamental group}{
		The fundamental group $\pi_1(X, x_0)$ is defined as the loop space modulo homotopy. In general, as the notation implies, the fundamental group depends on the base point $x_0$. However when the space $X$ is path-connected, the fundamental groups belonging to different base points are isomorphic. It follows that we can speak of "the" fundamental group in the case of path-connected spaces.
		
		As the name implies the fundamental group can be given the structure of a multiplicative group, where the operation is inherited from that of the loop space.
	}
	
	This definition can be generalized to arbitrary dimensions in the following way\footnote{Note however that we replace the interval $[0, 1]$ by the sphere $S^1$. This is nonrestrictive as we can construct $S^n$ by mapping (identifying) the boundary of $[0,1]^n$ to the basepoint $x_0$.}:
	\newdef{Homotopy group}{\index{homotopy!group}
		The homotopy group $\pi_n(X, x_0)$ is defined as the set of homotopy classes of continuous maps $f:S^n\rightarrow X$ based at $x_0\in X$. The set $\pi_0(X, x_0)$ is defined as the set of path-connected components of $X$.
	}
	
	\begin{property}
		For $n\geq1$ the sets $\pi_n(X, x_0)$ are groups.
	\end{property}
		\begin{property}
		For $n\geq2$  the homotopy groups $\pi_n(X, x_0)$ are abelian.
	\end{property}
	
	\begin{property}
		If $X$ is path-connected, then the homotopy groups $\pi_n(X, x_0)$ and $\pi_n(X, x_1)$ are isomorphic for all $x_0, x_1\in X$ and all $n\in\mathbb{N}$.
	\end{property}
	\begin{property}
		Homeomorphic spaces have the same homotopy groups $\pi_n$.
	\end{property}
	
	\begin{formula}
		Let $(X, x_0)$ and $(Y, y_0)$ be pointed topological spaces with homotopy groups $\pi_n(X, x_0)$ and $\pi_n(Y, y_0)$. The homotopy groups of their product is given by:
		\begin{equation}
			\pi_n(X\times Y, (x_0, y_0)) = \pi_n(X, x_0)\otimes\pi_n(Y, y_0)
		\end{equation}
		where $\otimes$ denotes the direct product of groups \ref{group:direct_product}.
	\end{formula}

\section{Homology and Cohomology}\label{section:homology}
\subsection{Simplexes}
	\newdef{Simplex}{\index{simplex}
		A $k$-simplex $\sigma^k$ is defined as the following set:
		\begin{equation}
			\sigma^k = \left\{\left.\sum_{i=0}^k\lambda_it_i\right|\sum_{i=0}^k\lambda_i = 1\text{ and }\lambda_i\geq0\right\}
		\end{equation}
		where the points (vertices) $t_i$ are linearly independent, i.e. the vectors $t_i-t_0$ are linearly independent. Equivalently it is the convex hull of the $k+1$ vertices $\{t_0, ..., t_k\}$.
	}
	\newdef{Barycentric coordinates}{
		The coordinates $\lambda_i$ from previous defintion are called barycentric coordinates. This comes from the fact that the point $\sum_{i=0}^k\lambda_it_i$ represents the barycenter of a gravitational system consisting of masses $\lambda_i$ placed at the points $t_i$.
	}
	
	\newdef{Simplicial complex}{\index{simplicial complex}
		A simplicial complex $\mathcal{K}$ is a set of simplexes satisfying following conditions:
		\begin{itemize}
			\item If $\sigma$ is a simplex in $\mathcal{K}$ then so are its faces.
			\item If $\sigma_1, \sigma_2\in\mathcal{K}$ then we have $\sigma_1\cap\sigma_2 = \emptyset$ or $\sigma_1\cap\sigma_2$ is a face of both $\sigma_1$ and $\sigma_2$.
		\end{itemize}
		A simplicial $k$-complex is a simplicial complex where every simplex has dimension at most $k$.
	}
	
	\newdef{Polyhedron}{\index{polyhedron}
		Let $\mathcal{K}$ be a simplicial complex. The polyhedron associated with $\mathcal{K}$ is the topological spaces constructed by equipping $\mathcal{K}$ with the Euclidean subspace topology.
	}
	
	\newdef{Path-connectedness}{\index{path-connected}
		Let $\mathcal{K}$ be a simplicial complex. $\mathcal{K}$ is said to be path-connected if every two vertices in $\mathcal{K}$ are connected by edges in $\mathcal{K}$.
	}
	
	
	\newdef{Triangulable spaces}{\index{triangulation}
		Let $X$ be a topological space and let $\mathcal{K}$ be a polyhedron. If there exists a homeomorphism $\varphi:\mathcal{K}\rightarrow X$ then we say that $X$ is \textbf{triangulable} and we call $\mathcal{K}$ a \textbf{triangulation} of $X$.
	}
	\begin{theorem}
		Let $\mathcal{K}$ be a path-connected polyhedron with basepoint $a_0$. Let $\mathcal{C}\subset\mathcal{K}$ be a contractible 1-dimensional subpolyhedron containing all vertexes of $\mathcal{K}$. Let $G$ be the free group generated by the elements $g_{ij}$ corresponding to the ordered 1-simplexes $[v_i,v_j]\in\mathcal{C}$. If the generators $g_{ij}$ satisfy following two relations:
		\begin{itemize}
			\item $g_{ij}g_{jk} = g_{ik}$ for every ordered 2-simplex $[v_i,v_j,v_k]\in\mathcal{K}\backslash\mathcal{C}$
			\item $g_{ij} = e$ if $[v_i,v_j]\in\mathcal{C}$.
		\end{itemize}
		then the group $G$ is isomorphic to fundamental group $\pi_1(\mathcal{K}, a_0)$.
	\end{theorem}
	\begin{result}
		From the theorem that homeomorphic spaces have the same homotopy groups it follows that the fundamental group of a triangulable space can be computed by looking at its triangulations.
	\end{result}

\subsection{Simplicial homology}	
	\newdef{Chain group}{\index{chain!group}\label{topology:chain_group}
		Let $\mathcal{K}$ be a simplicial $n$-complex. The $k$-th chain group $C_k(\mathcal{K})$ is defined as the free Abelian (additive) group generated by the $k$-simplexes in $\mathcal{K}$:
		\begin{equation}
			C_k(\mathcal{K}) = \left\{\left.\sum_ia_i\sigma_i\ \right|\sigma_i\text{ is a $k$-simplex in $\mathcal{K}$}\text{ and }a_i\in\mathbb{Z}\right\}
		\end{equation}
		For $k>n$ we define $C_k(\mathcal{K})$ to be $\{0\}$.
	}
	
	\newdef{Boundary operator}{\index{boundary}
		The boundary operator $\partial_k:C_k(\mathcal{K})\rightarrow C_{k-1}(\mathcal{K})$ is the homomorphism defined by following properties:
		\begin{itemize}
			\item $\partial_k$ is linear, i.e. $\partial_k(\sum_ia_i\sigma_i) = \sum_ia_i\partial_k(\sigma_i)$
			\item For every oriented $k$-simplex $[v_0, ..., v_k]$ we have that
			\begin{equation}
				\partial_k[v_0, ..., v_k] = \sum_{i=0}^k(-1)^i[v_0, ..., \hat{v}_i, ..., v_k]
			\end{equation}
			\item The boundary of every 0-chain is 0.
		\end{itemize}
		where $[v_0, ..., \hat{v}_i, ..., v_k]$ denotes the $(k-1)$-simplex obtained by removing the vertex $v_i$.
	}
	\begin{property}\index{chain!complex}
		The boundary operators satisfy following relation:
		\begin{equation}
			\label{topology:boundary_operator_relation}
			\partial_k\circ\partial_{k-1} = 0
		\end{equation}
		This property turns the system $(C_k, \partial_k)$ into a so-called \textbf{chain complex}.
	\end{property}
	
	\newdef{Cycle group}{\index{cycle}
		The $k^{th}$ cycle group $Z_k(\mathcal{K})$ is defined as the set of $k$-chains $\sigma_k$ such that $\partial_k\sigma_k = 0$. These chains are also called \textbf{cycles}.
	}
	\newdef{Boundary group}{
		The $k^{th}$ boundary group $B_k(\mathcal{K})$ is defined as the set of $k$-chains $\sigma_k$ for which there exists a $(k+1)$-chain $N$ such that $\partial_{k+1}N = \sigma_k$. These chains are called \textbf{boundaries}.
	}
	\newdef{Homology group}{\index{homology!simplicial}
		From property \ref{topology:boundary_operator_relation} it follows that $B_k(\mathcal{K})$ is a subgroup of $Z_k(\mathcal{K})$. We can thus define the $k^{th}$ homology group $H_k(\mathcal{K})$ as the following quotient:
		\begin{equation}
			H_k(\mathcal{K}) = Z_k(\mathcal{K}) / B_k(\mathcal{K})
		\end{equation}
		Theorem \ref{group:theorem:free_group} tells us that we can write $H_k(\mathcal{K})$ as $G_k\oplus T_k$. Both of these groups tell us something about $\mathcal{K}$. The rank of $G_k$, denoted by $R_k(\mathcal{K})$, is equal to the number of $(k+1)$-dimensional holes in $\mathcal{K}$. The torsion subgroup $T_k$ tells us how the space $\mathcal{K}$ is twisted.
	}	
	
	\begin{property}
		If two topological spaces are of the same homotopy type then they have isomorphic homology groups. It follows that homeomorphic spaces have isomorphic homology groups.
	\end{property}
	\begin{result}
		It follows from the definition of a triangulation that we can (easily) construct the homology groups for a given triangulable space by looking at one of its triangulations.
	\end{result}
	
	\newdef{Betti numbers}{\index{Betti number}
		The numbers $R_k(\mathcal{X})$ from the definition of homology groups are called the Betti numbers of $\mathcal{X}$.
	}
	\begin{formula}[Euler characteristic]\index{Euler!characteristic}\index{Euler-Poincar\'e formula}
		The Euler characteristic of a triangulable space $X$ is defined as follows\footnote{This formula is sometimes called the \textit{Euler-Poincar\'e} or \textit{Poincar\'e} formula.}:
		\begin{equation}
			\boxed{\chi(X) = \sum_i(-1)^iR_i(X)}
		\end{equation}
	\end{formula}
	
	\begin{definition}
		The definition of homology groups can be generalized by letting the (formal) linear combinations used in the definition of the chain group (see \ref{topology:chain_group}) be of the following form:
		\begin{equation}
			c^k = \sum_ig_i\sigma_i^k
		\end{equation}
		where $G = \{g_i\}$ is an Abelian group and $\sigma_i^k$ are $k$-simplexes. The $k^{th}$ homology group of $X$ with coefficients in $G$ is denoted by $H_k(X; G)$. In case of $G$ being a field, such as $\mathbb{Q}$ or $\mathbb{R}$, the torsion subgroups $T_k$ vanish. The relation between integral homology and homology with coefficients in a group (or field) is given by the \textit{Universal coefficient theorem}.
	\end{definition}
	\newformula{K\"unneth formula}{\index{K\"unneth formula}
		Let $X, Y$ be two triangulable spaces. The homology groups of the Cartesian product $X\times Y$ with coefficients in a field $F$ is given by:
		\begin{equation}
			H_k(X\times Y; F) = \bigoplus_{k = i+j}H_i(X;F)\otimes H_j(Y;F)
		\end{equation}
		When $F$ is replaced by the set of integers the torsion subgroups have to be taken into account. This will not be done here.
	}
	
\subsection{Relative homology}

	In this section we use a simplicial complex $K$ and a subcomplex $L$.
	
	\newdef{Relative chain group}{\index{chain}
		The $k$-chain group of $K$ modulo $L$ is defined as the following quotient group:
		\begin{equation}
			C_k(K, L) = C_k(K) / C_k(L)
		 \end{equation}
	}
	\newdef{Relative boundary operator}{\index{boundary}
		The relative boundary operator $\overline\partial_k$ is defined as follows:
		\begin{equation}
			\overline\partial_k(c_k+C_k(L)) = \partial_k c_k + C_{k-1}(L)
		\end{equation}
		where $c_k\in C_k(K)$. This operator is, just like the ordinary boundary operator $\partial_k$, a homomorphism.
	}
	\newdef{Relative homology groups}{\index{homology!relative}
		The relative cycle and relative boundary groups are defined analogous to their ordinary counterparts. The relative homology groups are then defined as follows:
		\begin{equation}
			H_k(K, L) = \stylefrac{\ker\overline\partial_k}{\im\overline\partial_{k+1}}
		\end{equation}
		Elements $h_k\in H_k(K, L)$ can thus be written as $h_k=z_k + C_p(L)$ where $z_k$ does not have to be a relative $k$-cycle. We merely require that $\partial_kz_k$ is a chain in $C_{k-1}(L)$.
	}
	\newdef{Homology sequence}{
		Using the relative homology groups we obtain following (long) exact sequence:
		\begin{equation}
			\cdots\rightarrow H_k(L)\xrightarrow{i_\ast}H_k(K)\xrightarrow{j_\ast}H_k(K, L)\xrightarrow{\partial_k}H_{k-1}(L)\rightarrow\cdots
		\end{equation}
		where $i_\ast$ and $j_\ast$ are the homology homomorphisms induced by the inclusions $i:L\rightarrow K$ and $j:K\rightarrow (K, L)$.
	}

	\begin{theorem}[Excision theorem]\index{excision theorem}\label{topology:theorem:excision}
		Let $X, U, V$ be a triangulable spaces such that $U\subset V\subset X$. If the closure $\overline{U}$ is contained in the interior $V\degree$ then:
		\begin{equation}
			H_k(X, V) = H_k(X\backslash U, V\backslash U)
		\end{equation}
	\end{theorem}

\subsection{Singular homology}
	
	\newdef{Singular simplex}{\index{simplex!singular}\index{simplex!standard}
		Consider the \textbf{standard} $k$-simplex $\Delta^k$:
		\begin{equation}
			\Delta^k = \{(x_0, ..., x_k)\in\mathbb{R}^{k+1}|\sum_ix_i = 1 \text{ and }x_i\geq0\}
		\end{equation}
		A singular $k$-simplex in a topological space $X$ is defined as a continuous map $\sigma^k:\Delta^k\rightarrow X$. The name singular comes from the fact that the maps $\sigma^k$ need not be invertible.
	}
	
	\newdef{Singular chain group}{\index{chain}
		The singular chain group $S_k(X)$ with coefficients in a group $G$ is defined as the set of formal linear combinations $\sum_ig_i\sigma_i^k$. The basis of this freely generated group is in most cases infinite as there are multiple ways to map $\Delta^k$ to $X$.
	}
	\newdef{Singular boundary operator}{
		The singular boundary operator $\partial$ (we use the same notation as for simplicial boundary operators) is defined by its linear action of the singular chain group $S_k(X)$. It follows that we only have to know the action on the singular simplexes $\sigma^k$.
		
		We first introduce the notation $[v_0, ..., v_k] := [\sigma^k(e_0), ..., \sigma^k(e_k)]$ where $e_i$ is the $i^{th}$ vertex of the standard simplex $\Delta^k$. The action on the singular simplex $\sigma^k$ is then given by:
		\begin{equation}
			\partial\sigma^k = \sum_i^k(-1)^k[v_0, ..., \hat{v}_i, ..., v_k]
		\end{equation}
		The singular boundary operators satisfy the same relation as in the simplicial case:
		\begin{equation}
			\partial_k\circ\partial_{k-1} = 0
		\end{equation}
	}
	\newdef{Singular homology group}{\index{homology!singular}
		The singular homology groups are defined as follows:
		\begin{equation}
			H_k(X; G) = \stylefrac{\ker\partial_k}{\im\partial_{k+1}}
		\end{equation}
	}
	
	\begin{theorem}
		Let $X$ be a triangulable space. The $k^{th}$ singular homology group of $X$ is isomorphic to the $k^{th}$ simplicial homology group of $X$.
	\end{theorem}
	\begin{remark}
		When $X$ is not triangulable the previous theorem is not valid. The singular approach to homology is thus a more general construction, but it is often more difficult to compute the homology groups (even in the case of triangulable spaces).
	\end{remark}

\subsection{Examples}

	\begin{example}
		Let $X$ be a contractible space. We then find that:
		\begin{equation}
			H_k(X) = \begin{cases}
				\mathbb{Z}&k=0\\
				\{0\}&k>0
			\end{cases}
		\end{equation}
	\end{example}
	\begin{example}
		Let $P$ be a connected polyhedron. We then find that:
		\begin{equation}
			H_0(P) = \mathbb{Z}
		\end{equation}
	\end{example}
	\begin{example}
		The homology groups of the $n$-sphere $S^n$ are given by:
		\begin{equation}
			H_k(S^n)=\begin{cases}
				\mathbb{Z}&k=0\text{ or }k=n\\
				\{0\}&\text{otherwise}
			\end{cases}
		\end{equation}
	\end{example}

\subsection{Axiomatic approach}

	\newdef{Eilenberg-Steenrod axioms}{\index{Eilenberg-Steenrod axioms}
		All homology theories have the following properties in common. By treating these properties as axioms we can construct homology theories as a sequence of functors $H_k$. The axioms are as follows:
		\begin{enumerate}
			\item \textbf{Homotopy}: If $f, g$ are homotopic maps then their induced homology maps are the same.
			\item \textbf{Excision}\footnotemark: If $U\subset V\subset X$ and $\overline U\subset V\degree$ then $H_k(X, V) \cong H_k(X\backslash U, V\backslash U)$
			\item \textbf{Dimension}: If $X$ is a singleton then $H_k(X) = \{0\}$ for all $k\geq1$.
			\item \textbf{Additivity}: If $X = \bigsqcup_i X_i$ then $H_k(X)\cong\bigoplus_i H_k(X_i)$
			\item \textbf{Exactness}: Each pair $(X, A)$, where $A\subset X$, induces a long exact sequence
			\begin{equation}
				\cdots\rightarrow H_k(A)\xrightarrow{i_\ast}H_k(X)\xrightarrow{j_\ast}H_k(X, A)\xrightarrow{\partial_k}H_{k-1}(A)\rightarrow\cdots
			\end{equation}
			where $i_\ast$ and $j_\ast$ are the homology homomorphisms induced by the inclusions $i:A\rightarrow X$ and $j:X\rightarrow (X, A)$.
		\end{enumerate}
		\footnotetext{See also theorem \ref{topology:theorem:excision}.}
		Let $X$ be a singleton. The group $H_0(X)$ is called the \textbf{coefficient group} and gives the coefficients used in the construction of the free Abelian chain groups $C_k$.
	}
	\begin{remark}
		If the dimension axiom is removed from the set of axioms, then we obtain a so-called \textit{extraordinary homology theory}.
	\end{remark}

\section{Sheaf theory}
\subsection{Sheafs}
	
	\newdef{Sheaf}{\index{sheaf}
		Let $X$ be a topological space. A sheaf over $X$ is a tuple $(S, X, \pi)$, where $S$ is a topological space and $\pi:S\rightarrow X$ a continuous surjection, such that the following two conditions are satisfied:
		\begin{itemize}
			\item For every point $s\in S$ there exists a neighbourhood $U$ such that $\pi|_U$ is homeomorphism onto some open neighbourhood of $\pi(s)\in X$. This map induces the discrete topology on $S$
			\item For every $x\in X$, the set $\pi^{-1}(x)$ is an algebraic structure such that the corresponding algebraic operation is continuous.
		\end{itemize}
	}
	\newdef{Stalk}{\index{stalk}
		The preimage $\pi^{-1}(x)$ is called the stalk over $x$ and is often denoted by $S_x$.
	}
	\newdef{Homomorphism of sheaves}{\index{homomorphism!of sheaves}
		Let $S, S'$ be two sheaves over the same space with projections $\pi$ and $\pi'$. A homomorphism of sheaves is a map $\Phi$ satisfying the following conditions:
		\begin{itemize}
			\item $\Phi:S\rightarrow S'$ is continuous.
			\item $\pi = \pi'\circ\Phi$, i.e. $\Phi$ maps stalks in $S$ to corresponding stalks in $S'$.
			\item For each $x\in X$, the restriction $\Phi|_x:S_x\rightarrow S'_x$ is a homomorphism of the algebraic structures corresponding to the stalks.
		\end{itemize}
	}
	
\subsection{Presheafes}

	\newdef{Presheaf}{\index{presheaf}
		Let $X$ be a topological space. A presheaf over $X$ consists of an algebraic structure $S_U$ for every open set $U\subseteq X$ and a homomorphism $\Phi^U_V:S_U\rightarrow S_V$ for every two open sets $U, V\subseteq X$ with $V\subseteq U$ such that the following conditions are satisfied:
		\begin{itemize}
			\item If $U=\emptyset$ then $S_U = 0$, where 0 is the zero object in the category corresponding to the algebraic structure of $S_U$.
			\item $\Phi^U_U = \mathbbm{1}_X$
			\item If $W\subseteq V\subseteq U$ then $\Phi^U_W = \Phi^V_W\circ\Phi^U_V$.
		\end{itemize}
	}
	
	\newdef{Homomorphism of presheaves}{\index{homomorphism!of presheaves}\index{germ}
		Let $S, S'$ be two presheaves. A homomorphism $S\rightarrow S'$ is a set of homomorphisms $\Psi_U:S_U\rightarrow S_U'$ that commute with the maps $\Phi^U_V$.
	}
	
	\begin{construct}\label{sheaf:associated_construction}
		For every presheaf over $X$ we can construct a sheaf $(S, X, \pi)$. For every $x\in X$ we set the stalk $S_x$ to be the direct limit \ref{direct_limit} of the direct system $(S_U, \Phi^U_V)$. The set $S$ is then defined as the union of all sets $S_x$ and $\pi$ maps every element of $S_x\subset S$ to $x$.
		
		The topology on $S$ is defined by means of the following basis. For every $U\in X$ and every element $f\in S_U$ we construct a subset $f_U\subset S$ given by $\{f_x\in S_x:x\in U\}$ where $f_x$ is called the \textbf{germ}\footnote{This is a generalization of definition \ref{topology:germ}.} of $f$ at $x$. The basis for our topology is then given by the set $\{f_U:U\subset X, f\in S_U\}$.
	\end{construct}
	
\subsection{Sections}{
	\newdef{Section}{\index{section}
		A section of a sheaf $(S, X, \pi)$ over an open set $U$ is a continuous map $s:U\rightarrow S$ such that $\pi\circ s = \mathbbm{1}_U$. The set of all sections carry the same algebraic structure as $S$.
	}
	\begin{remark*}
		A \textbf{global} section is a section $s:X\rightarrow S$.
	\end{remark*}
}

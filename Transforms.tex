\chapter{Integral transforms}
\section{Fourier series}
   
	\newdef{Dirichlet kernel}{\index{Dirichlet!kernel}
   		The Dirichlet kernel is the collection of functions of the form
        \begin{equation}
        	\label{transforms:dirichlet_kernel}
            D_n(x) = \stylefrac{1}{2\pi}\sum_{k=-n}^ne^{ikx}
        \end{equation}
	}
    \newformula{Sieve property}{\index{sieve}
    	If $f\in C^1[-\pi, \pi]$ then
        \begin{equation}
        	\lim_{n\rightarrow+\infty}\int_{-\pi}^\pi f(x)D_n(x)dx = 0
        \end{equation}
    }
    \begin{formula}
    	For $2\pi$-periodic functions, the $n$-th degree Fourier approximation is given by following convolution:
    	\begin{equation}
    		s_n(x) = \sum_{k=-n}^n\widetilde{f}(k)e^{ikx} = (D_n \ast f)(x)
    	\end{equation}
    \end{formula}
    
    \begin{theorem}[Convergence of the Fourier series]
    	Let $f:\mathbb{R}\rightarrow\mathbb{R}$ be a $2\pi$-periodic function. If $f(x)$ is piecewise $C^1$ on $[-\pi, \pi]$ the the limit $\lim_{n\rightarrow+\infty}(D_n\ast f)(x)$ converges to $\frac{f(x+) + f(x-)}{2}$ for all $x\in\mathbb{R}$.
    \end{theorem}
	\newformula{Generalized Fourier series}{\index{Fourier!series}
    	Let $f(x)\in \mathcal{L}^2[-l, l]$ be a $2l$-periodic function. This function can be approximated by the following series:
        \begin{equation}
            \label{transforms:fourier_series}
            \boxed{f(x) = \sum_{n = -\infty}^{+\infty} \left(\frac{1}{2l}\int_{-l}^le^{-i\frac{n\pi x'}{l}}f(x')dx'\right) e^{i\frac{n\pi x}{l}}}
        \end{equation}
	}
    
    \begin{formula}[Fourier coefficients]
		As seen in the general formula, the Fourier coefficient $\widetilde{f}(n)$ can be calculated by taking the inner product \ref{hilbert:inner_product} of $f(x)$ and the $n$-th eigenfunction $e_n$:
        \begin{equation}
			\label{transforms:fourier_coefficients}
            \widetilde{f}(n) = \left\langle e_n|f\right\rangle = \int_{-l}^le_n^*(x)f(x)dx \qquad\text{with}\qquad e_n = \sqrt\frac{1}{2l}e^{i\frac{n\pi x}{l}}
		\end{equation}
	\end{formula}
    
    \newdef{Periodic extension}{\index{periodic extension}
    	Let $f(x)$ be piecewise $C^1$ on $[-L, L]$. The periodic extension $f^L(x)$ is defined by repeating the restriction of $f(x)$ to $[-L, L]$ every $2L$. The \textbf{normalized periodic extension} is defined as
        \begin{equation}
        	f^{L, \nu}(x) = \stylefrac{f^L(x+) + f^L(x-)}{2}
        \end{equation}
    }
    \begin{theorem}
    	If $f:\mathbb{R}\rightarrow\mathbb{R}$ is piecewise $C^1$ on $[-L, L]$ then the Fourier series approximation of $f(x)$ converges to $f^{L, \nu}(x)$ for all $x\in\mathbb{R}$.
    \end{theorem}

\section{Fourier transform}\index{Fourier!transform}
	The Fourier series can be used to expand a $2l$-periodic function as an infinite series of exponentials. For expanding a non-periodic function we need the Fourier integral: 
	\begin{equation}
		\label{transforms:fourier}
        \boxed{\mathcal{F}(\omega) = \frac{1}{\sqrt{2\pi}} \int_{-\infty}^{\infty}f(t)e^{-i\omega t}dt}
	\end{equation}
    
    \begin{equation}
		\label{transforms:inverse_fourier}
        f(t) = \mathcal{F}^{-1}(t) = \frac{1}{\sqrt{2\pi}} \Xint-_{-\infty}^{\infty}\mathcal{F}(\omega)e^{i\omega t}d\omega
	\end{equation}
    
    Equation \ref{transforms:fourier} is called the (forward) Fourier transform of $f(t)$ and equation \ref{transforms:inverse_fourier} is called the inverse Fourier transform.
    
    \begin{notation}
		The Fourier transform of a function $f(t)$, as seen in equation \ref{transforms:fourier}, is often denoted by $\widetilde{f}(\omega)$.
	\end{notation}
    
    \begin{theorem}[Convergence of the Fourier integral]
    	If $f:\mathbb{R}\rightarrow\mathbb{R}$ is Lipschitz continuous (see \ref{calculus:lipschitz_continuity}) and if $\int_{-\infty}^{+\infty}|f(x)|dx$ is convergent then the Fourier integral converges to $f(x)$ for all $x\in\mathbb{R}$.
    \end{theorem}
    \begin{theorem}[Fourier inversion theorem]
    	If both $f(t), \mathcal{F}(\omega)\in\mathcal{L}^1(\mathbb{R})$ are continuous then the Cauchy principal value in \ref{transforms:inverse_fourier} can be replaced by a normal integral.
    \end{theorem}
    \begin{remark}
    	Schwartz functions (see \ref{distribution:schwartz_space}) are continuous elements of $\mathcal{L}^1(\mathbb{R})$ and as such the Fourier inversion theorem also holds for these functions. This is interesting because checking the conditions for Schwartz functions is often easier then checking the more general conditions of the theorem.
    \end{remark}
    
    \begin{property}
    	From the Riemann-Lebesgue lemma \ref{lebesgue:riemann_lebesue_lemma} it follows that
        \begin{equation}
        	\mathcal{F}(\omega)\rightarrow0 \qquad\text{if}\qquad |\omega|\rightarrow0
        \end{equation}
    \end{property}
    
    \newprop{Parceval's theorem}{\index{Parceval!theorem}
    	Let $(f, \widetilde{f})$ and $(g,\widetilde{g})$ be two Fourier transform pairs.
        \begin{equation}
			\label{transforms:parcevals_theorem}
            \int_{-\infty}^{+\infty}f(x)g(x)dx = \int_{-\infty}^{+\infty}\widetilde{f}(k)\widetilde{g}(k)dk
		\end{equation}
    }
    \begin{result}[Plancherel theorem]\index{Plancherel theorem}
    	The integral of the square (of the modulus) of a Fourier transform is equal to the integral of the square (of the modulus) of the original function:
    	\begin{equation}
    		\label{transforms:plancherel_theorem}
            \int_{-\infty}^{+\infty}|f(x)|^2dx = \int_{-\infty}^{+\infty}|\widetilde{f}(k)|^2dk
    	\end{equation}
    \end{result}
    
    \subsection{Convolution}
        \newformula{Convolution}{\index{convolution}
            \begin{equation}
                \label{transforms:convolution}
                \boxed{(f \ast g)(t) = \frac{1}{\sqrt{2\pi}} \int_{-\infty}^{\infty}f(\tau)g(t - \tau)d\tau}
            \end{equation}
        }
        
        \begin{property}[Commutativity]
			\begin{equation}
				f \ast g = g \ast f
	        \end{equation}
		\end{property}
        
        \begin{theorem}[Convolution Theorem]
			\begin{equation}
				\widetilde{f \ast g} = \widetilde{g} \widetilde{f}
	        \end{equation}
		\end{theorem}

\section{Laplace transform}
	\newformula{Laplace transform}{\index{Laplace!transform}
        \begin{equation}
            \label{transforms:laplace}
            \mathcal{L}\{F(t)\}_{(s)} = \int_{0}^{\infty}f(t)e^{-st}dt
        \end{equation}
    }
    
    \newformula{Bromwich integral}{\index{Bromwich!integral}
        \begin{equation}
            \label{transforms:inverse_laplace}
            f(t) = \frac{1}{2\pi i} \int_{\gamma - i\infty}^{\gamma + i\infty}\mathcal{L}\{F(t)\}_{(s)}e^{st}ds
        \end{equation}
    }
    
    \begin{notation}
		The Laplace transform as defined in equation \ref{transforms:laplace} is sometimes denoted by $f(s)$ .
	\end{notation}
    
\section{Mellin transform}
	\newformula{Mellin transform}{\index{Mellin!transform}
	    	\begin{equation}
			\label{transforms:mellin}
	    		\mathcal{M}\{f(x)\}(s) = \int_0^{+\infty}x^{s-1}f(x)dx
	    	\end{equation}
	}
	\newformula{Inverse Mellin transform}{
		\begin{equation}
		    \label{transforms:inverse_mellin}
		    f(x) = \frac{1}{2\pi i} \int_{\gamma - i\infty}^{\gamma + i\infty}\mathcal{M}\{f(x)\}_{(s)}x^{-s}ds
		\end{equation}
	}
    
\section{Integral representations}
	\newformula{Heaviside step function}{\index{Heaviside!step function}
		\begin{equation}
			\theta(x) = \frac{1}{2\pi i}\int_{-\infty}^\infty\frac{e^{ikx}}{x - i\varepsilon}dk
		\end{equation}
	}
	\newformula{Dirac delta function}{\index{Dirac!delta function}
		\begin{equation}
			\delta^{(n)}(\vector{x}) = \frac{1}{(2\pi)^n}\int_{-\infty}^\infty e^{i\vector{k}\cdot\vector{x}}d^nk
		\end{equation}
	}

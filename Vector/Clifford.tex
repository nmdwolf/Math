\chapter{Clifford Algebra}\label{chapter:clifford}

    The main references for this chapter are \cite{AMP1, AMP2, gallier_clifford}. One should note that there are various conventions for the different structures that arise in the study of Clifford algebras and their representations. Even the references we give do not agree on the conventions they adopt.

    In general we will also assume that all metrics (and quadratic forms) are nondegenerate. A part of the theory can also be extended to the degenerate case, but we will not need this. See \cite{gallier_clifford} for more information.

\section{Clifford algebra}

    \newdef{Clifford algebra}{\index{Clifford!algebra}\label{clifford:clifford_algebra}
        Consider a unital associative algebra $V$ together with a quadratic form $Q:V\rightarrow K$. The Clifford algebra over $V$ associated to $Q$ is the free algebra generated by $V$ under the following relation:
        \begin{gather}
            \label{clifford:condition}
            v\cdot v = Q(v)1
        \end{gather}
        where $1$ is the unit element in $V$. This condition implies that the square of a vector is a scalar.
    }
    \begin{notation}
        The Clifford algebra corresponding to $V$ and $Q$ is often denoted by $C\ell(V, Q)$.
    \end{notation}

    \begin{construct}
        The previous definition can be given an explicit construction. First we construct the tensor algebra of $V$:
        \begin{gather}
            T(V) = \bigoplus_{k\in\mathbb{N}}V^{\otimes k}.
        \end{gather}
        Then, we construct a two-sided ideal $I$ of $V$ generated\footnote{See definition \ref{group:generating_set_ideal}.} by $\{v\otimes v - Q(v)1_V\ :\ v\in V\}$. The Clifford algebra $C\ell(V, Q)$ can then be constructed as the quotient algebra $T(V)/I$.
    \end{construct}

    \begin{remark}
        Looking at definition \ref{tensor:adef_exterior_algebra} we see that the exterior algebra $\Lambda^*(V)$ coincides with the Clifford algebra $C\ell(V, 0)$. If $Q\neq0$ then the two algebras are still isomorphic as vector spaces (if\footnote{This condition will often come back in this chapter.} $\text{char}(V)\neq2$).
    \end{remark}

    \begin{property}[Dimension]
        If $V$ has dimension $n$, then $C\ell(V, Q)$ has dimension $2^n$.
    \end{property}

    \begin{example}
        The classic example of a Clifford algebra is given by a vector space with Lorentzian signature $(p, q)$, i.e. a vector space with a semidefinite form $g(\cdot, \cdot)$ admitting a basis $\{e_i\}_{i\leq p+q}$ such that
        \begin{gather}
            \begin{cases}
                g(e_i, e_i) = 1 & 1\leq i\leq p\\
                g(e_i, e_i) = -1 & p<i\leq p+q.
            \end{cases}
        \end{gather}
        The Clifford algebra $C\ell_{p,q}(K)$ or $K_{p, q}$ is then defined as the Clifford algebra generated under the relation $v\cdot v = -g(v, v)1$. In physics this convention would correspond to the ''mostly pluses''-convention, which is mainly adopted in general relativity.
    \end{example}

\section{Geometric algebra}

    \newdef{Geometric algebra}{\index{geometric!algebra}
        Let $V$ be a vector space equipped with a symmetric bilinear form $g:V\times V\rightarrow K$. The geometric algebra (GA) over $V$ is defined as the Clifford algebra $C\ell(V, g)$. Here we implicitly used the classic relation $Q(v) = g(v,v)$ since we used quadratic forms in definition \ref{clifford:clifford_algebra}. This identification is unique as long as $\text{char}(V)\neq2$.
    }

    \newdef{Inner and exterior product}{\index{inner!product}\index{exterior!product}
        Analogous to the inner product in linear algebra and the wedge product in exterior algebra one can define a(n) (a)symmetric product on the geometric algebra.

        First of all we note that the product $ab$ of two vectors $a$ and $b$ can be written as the sum of a symmetric and an antisymmetric part:
        \begin{gather}
            \label{clifford:geometric_product}
            ab = \stylefrac{1}{2}(ab + ba) + \stylefrac{1}{2}(ab - ba).
        \end{gather}
        We can then define the inner product as the symmetric part:
        \begin{gather}
            \label{clifford:inner_product}
            a\cdot b := \stylefrac{1}{2}(ab + ba) = \stylefrac{1}{2}\left((a+b)^2 - a^2 - b^2\right) = g(a, b).
        \end{gather}
        Analogously we define the exterior (outer) product as the antisymmetric part:
        \begin{gather}
            \label{clifford:exterior_product}
            a\wedge b := \stylefrac{1}{2}(ab - ba).
        \end{gather}
        These definitions allow us the rewrite formula \ref{clifford:geometric_product} as follows:
        \begin{gather}
            ab = a\cdot b + a\wedge b.
        \end{gather}
    }
    \begin{remark*}
        Looking at the last equality in the definition of the inner product \ref{clifford:inner_product} we see that condition \ref{clifford:condition} is indeed satisfied when $a=b$.
    \end{remark*}

    \newdef{Multivector}{\index{blade}\index{multi!vector}\index{pseudo!scalar}\index{pseudo!vector}
        Any element of the GA over $V$ is called a multivector. The simple multivectors of grade $k$, i.e. elements of the form $v_1v_2...v_k$ with $v_i\in V$ for all $i$, are called \textbf{$k$-blades}. (This should again remind the reader of the content of section \ref{tensor:section:wedge_product}.) Sums of multivectors of different grades are called \textbf{mixed} multivectors\footnote{Although important, these elements do not represent a geometric structure.}.

        Let $n=\dim(V)$. Multivectors of grade $n$ are also called \textbf{pseudoscalars} and multivectors of grade $n-1$ are also called \textbf{pseudovectors}.
    }

    \newdef{Grade projection operator}{\index{projection}
        Let $a$ be a general multivector. The grade (projection) operator $\langle\cdot\rangle_k:\mathcal{G}\rightarrow\mathcal{G}_k$ is defined as the projection of $a$ on the $k$-vector part of $a$.
    }

    Using these projection operators we can extend the inner and exterior product to the complete GA as follows:
    \begin{formula}
        Let $A, B$ be two multivectors of respectively grades $m$ and $n$. Their inner product is defined as
        \begin{gather}
            A\cdot B := \langle AB \rangle_{|m-n|}
        \end{gather}
        and their exterior product is defined as
        \begin{gather}
            A\wedge B := \langle AB \rangle_{m+n}.
        \end{gather}
        An explicit calculation for $A\in\mathcal{G}_1, B\in\mathcal{G}_k$ gives us:
        \begin{align}
            A\cdot B &= \frac{1}{2}\left(AB - (-1)^kBA\right)\\
            A\wedge B &= \frac{1}{2}\left(AB + (-1)^kBA\right).
        \end{align}
    \end{formula}

\section{Classification of Clifford algebras}

    \begin{formula}[Dimensional reduction]
        \begin{gather}
            \mathbb{R}_{p+1, q+1}\cong\mathbb{R}_{p, q}\otimes M_2(\mathbb{R})
        \end{gather}
    \end{formula}
    \begin{formula}
        \begin{gather}
            \mathbb{R}_{p+1, q}\cong\mathbb{R}_{q+1, p}
        \end{gather}
    \end{formula}
    \begin{formula}
        \begin{gather}
            \mathbb{R}_{p, q+2}\cong\mathbb{R}_{q, p}\otimes\mathbb{H}
        \end{gather}
    \end{formula}

    The following theorem has deep implications in K-theory. It is also (through K-theory) related to the \textit{tenfold way} of Altland \& Zirnbauer in condensed matter physics
    \begin{theorem}[Bott periodicity]\index{Bott!periodicity}
        The classification of (real) Clifford algebras is periodic modulo 8:
        \begin{gather}
            \mathbb{R}_{p, q+8}\cong\mathbb{R}_{p+8, q}\cong\mathbb{R}_{p, q}\otimes M_{16}(\mathbb{R}).
        \end{gather}
        For complex Clifford algebras one has a similar statement, but with periodicity 2.
    \end{theorem}

\section{Pin group}
\subsection{Clifford group}

    \newdef{Transposition}{\index{transposition}
        Let $\{e_i\}_{i\leq n}$ be a basis for $V$. On the tensor algebra $T(V)$ there exists an antiautomorphism $v^t$ that reverses the order of the basis vectors:
        \begin{gather}
            \label{clifford:transposition}
            \cdot^t:e_i\otimes e_j\otimes\cdots\otimes e_k\mapsto e_k\otimes\cdots\otimes e_j\otimes e_i.
        \end{gather}
        Because the ideal in the definition of a Clifford algebra is invariant under this map, it induces an antiautomorphism, called the transposition or \textbf{reversal}, on $C\ell(V)$.
    }

    \newdef{Main involution}{\index{involution}\index{inversion}
        Let $V_0, V_1$ be respectively the grade 0 and 1 components of the Clifford algebra $C\ell(V, Q)$. Consider the following operator:
        \begin{gather}
            \hat{v} =
            \begin{cases}
                v\qquad\qquad&v\in V_0\\
                -v\qquad\qquad&v\in V_1
            \end{cases}
        \end{gather}
        This operator can be generalized to all of $C\ell(V, Q)$ using linearity. The resulting operator is called the main involution or \textbf{inversion} on $C\ell(V, Q)$. It turns the Clifford algebra into a superalgebra\footnote{See definition \ref{linalgebra:superalgebra}.}.
    }

    \newformula{Twisted conjugation}{
        Let $v\in V$ be a vector and let $s\in C\ell(V, Q)$ be a unit of the Clifford algebra over $V$, i.e. $Q(s) \neq 0$. The twisted conjugation of $v$ by $s$ is given by the following map:
        \begin{gather}
            \chi: C\ell(V, Q)\rightarrow\text{Aut}(C\ell(V, Q))\qquad\text{with}\qquad\chi(s)v = sv\hat{s}^{-1}.
        \end{gather}
    }
    \newdef{Clifford group\footnotemark}{\index{Clifford!group}\index{Lipschitz!group}
        \footnotetext{Sometimes called the \textbf{Lipschitz group}.}
        The Clifford group $\Gamma(V, Q)$ is defined as follows:
        \begin{gather}
            \Gamma(V, Q) = \left\{s\in C\ell_{hom}(V, Q): s\text{ is invertible and }v\in V \implies sv\hat{s}^{-1} \in V\right\}
        \end{gather}
        Because the units of $C\ell(V, Q)$ form a group, $\Gamma(V, Q)$ also forms a group.
    }
    \begin{property}
        When restricting to the units of $C\ell(V)$ that belong to $V$ itself, the twisted conjugation is given by a Householder transformation\footnote{See definition \ref{linalgebra:householder_transformation}.}.
    \end{property}

    \begin{property}
        Let us now restrict to the case where $V$ is finite-dimensional and $Q$ is nondegenerate. If we interpret the condition $\chi_s(v)\in V$ as stating the existence of a linear transformation\footnote{Here we use the isomorphism between the degree-1 subspace of $C\ell(V)$ and $V$ itself.} $L\in\text{End}(V)$ such that
        \begin{gather}
            se_i\hat{s}^{-1} = L^j_ie_j
        \end{gather}
        then we see that $L$ preserves the norm on $V$ and accordingly that the map $s\mapsto L$ defines a surjective homomorphism\footnote{In char$(K)\neq2$, the surjectiveness of the map $\chi$ follows from the \textit{Cartan-Dieudonn\'e theorem}. For characteristic 2 one prove that the surjectiveness holds using different methods.}
        \begin{gather}
            \rho:\Gamma(V, Q)\rightarrow\text{O}(V, Q): s\mapsto L.
        \end{gather}
        Being a group morphism to a matrix group acting on $V$, it defines a representation called the \textbf{vector(ial) representation}. Furthermore, from the first isomorphism theorem \ref{group:theorem:first_isomorphism_theorem} it follows that $O(V, Q)$ is isomorphic to $\Gamma(V, Q)/\ker\chi$ where $\ker\chi = \mathbb{R}_0$. This isomorphism also implies\footnote{Again using the \textit{Cartan-Dieudonn\'e theorem}, valid only when char$(K)\neq2$.} that the Clifford group is given by the set of finite products of invertible elements $v\in V$:\footnote{This is more or less the Cartan-Dieudonn\'e theorem.}
        \begin{gather}
            \Gamma(V, Q) = \left\{\prod_i^n s_i : s_i\text{ invertible in }V, n\in\mathbb{N}\right\}.
        \end{gather}
    \end{property}
    \begin{result}
        By noting that pure rotations can be decomposed as an even number of reflections we find that
        \begin{gather}
            \Gamma^+(V, Q)/\mathbb{R}_0\cong\text{SO}(V, Q)
        \end{gather}
        where $\Gamma^+$ is the intersection of the even Clifford algebra and the Clifford group.
    \end{result}

    \begin{remark}
        As we noted in the beginning of this chapter, there is a variety of different conventions in use. One of the important distinctions is the definition (or choice) of conjugation map $\chi$. Atiyah, Bott and Shapiro have introduced the twisted conjugation map that we used for the definition of the Clifford group. Before them, the common choice was the ordinary conjugation map\footnote{The notation ad$_s$ comes from the fact that this map resembles the adjoint action of a group.}
        \begin{gather}
            \text{ad}_s:v\mapsto svs^{-1}.
        \end{gather}
        Although the difference between these maps seems to be rather subtle, the implications are important. If we would have chosen the conjugation $\text{ad}$ for our definition of the Clifford group, we would only have found a surjective homomorphism in the case of $\dim(V)$ being odd. Moreover, the action by a degree-1 element would not be given by a Householder transformation anymore, but instead it would be the negative of this operation. This distinction is in particular important for the next section.
    \end{remark}

\subsection{Pin and Spin}\label{clifford:section:spin}

    \newformula{Spinor norm}{\index{spinor!norm}
        On $\Gamma(V, Q)$ one can define the spinor norm:\footnote{This map can be generalized to the full Clifford algebra, but then the image will not just be the underlying field anymore.}
        \begin{gather}
            \mathcal{N}(x):\Gamma(V, Q)\rightarrow K^\times:x\mapsto x^tx
        \end{gather}
        where $x^t$ is the transposition \ref{clifford:transposition}. On $V$, $\mathcal{N}$ coincides with the norm induced by $Q$.
    }
    \newdef{Pin and spin groups}{\index{pin group}\index{spin!group}
        \nomenclature[S_Pin]{$\text{Pin}(V)$}{Pin group of the Clifford algebra $C\ell(V, Q)$.}
        Using the spinor norm $\mathcal{N}$ we can now define the pin and spins groups as follows:
        \begin{gather}
            \text{Pin}(V) := \{s\in\Gamma(V, Q): \mathcal{N}(s) = \pm1\}
        \end{gather}
        and
        \begin{gather}
            \text{Spin}(V) := \text{Pin}(V)\cap\Gamma^+(V, Q).
        \end{gather}
    }
    \begin{remark}
        In the literature one can sometimes find the following alternative definition of the spinor norm:
        \begin{gather}
            \mathcal{N}(x) := \hat{x}^tx.
        \end{gather}
    \end{remark}

    \begin{adefinition}
        The Pin group can also be defined as the set of elements in $\Gamma(V, Q)$ that can be written as a product of unit Clifford vectors (here by unit we mean unit norm and not just invertible as before). The Spin group is then defined as the elements that can be written as the product of an even number of unit Clifford vectors.
    \end{adefinition}

    \begin{property}\label{clifford:pin_group}
        The Pin group satisfies the following isomorphism:
        \begin{gather}
            \text{Pin}(V, Q)/\mathbb{Z}_2 \cong O(V, Q)
        \end{gather}
        and analogously for the Spin group and SO$(V, Q)$. These relations imply that the Pin and Spin groups form a double cover\footnote{A covering group is a topological group that is also a covering space. See definition \ref{topology:covering_space} for more information about the latter.} of respectively the orthogonal and special orthogonal groups.
    \end{property}

    \newdef{Spinor}{\index{spinor}\index{spin!representation}\index{Weyl!spinor}\index{Dirac!spinor}\index{gamma matrix}
        Consider a vector space $V$ equipped with a (faithful) representation of the group Spin$(m,n)$. This representation is called the \textbf{spin(or) representation}. Elements of $V$ are called spinors.

        More precisely, if we consider the complex Clifford algebra $C\ell_{m,n}(\mathbb{C})$ then we can have two possibilities: either $m+n$ is even or $m+n$ is odd. In the even case ($m+n=2k$) one can prove (using the Artin-Wedderburn theorem \ref{algebra:artin_wedderburn}) that the algebra is isomorphic to the matrix algebra $M(2^k, \mathbb{C})$. In the odd case ($m+n=2k+1$) the algebra is isomorphic to the direct sum $M(2^k, \mathbb{C})\oplus M(2^k, \mathbb{C})$.

        Inside these matrix algebras one can find a set of elements satisfying the Clifford relation \ref{clifford:condition} and thereby generating the Clifford algebra (the so-called \textbf{gamma matrices}). The real algebra generated by these elements is isomorphic to the real Clifford algebra $C\ell_{m,n}(\mathbb{R})$.\footnote{Note however that these matrices themselves will still be complex-valued.} The fundamental representation of this real algebra is often called the \textbf{Dirac representation}. If $m+n$ is even then the representation splits into two irreducible representations called the \textbf{Weyl} or \textbf{half-spin(or) representations}.
    }

    \begin{example}
        The following table gives some group isomorphisms for the spin group in $\dim n$:
        \begin{gather*}
            \begin{array}{c|c}
                n&\text{Spin}(n)\\
                \hline
                1&\text{O}(1)\\
                2&\text{U}(1)\\
                3&\text{SU}(2)\\
                4&\text{SU}(2)\times\text{SU}(2)
            \end{array}
        \end{gather*}
        For quadratic forms of signature $(p, q)$ we find the following table:
        \begin{gather*}
            \begin{array}{c|c}
                (1, n)&\text{Spin}(1, n)\\
                \hline
                (1,1)&\text{GL}(1, \mathbb{R})\\
                (1,2)&\text{SL}(1, \mathbb{R})\\
                (1,3)&\text{SL}(2, \mathbb{C})
            \end{array}
        \end{gather*}
    \end{example}

    \begin{formula}
        Consider the basis of $\mathfrak{su}(2)$ given by the Pauli matrices \ref{QM:angular_momentum:pauli_matrices}. An explicit (double) covering map $\rho:\text{Spin}(3)\cong\text{SU}(2)\rightarrow\text{SO}(3)$ is given by:
        \begin{gather}
            \rho:U\mapsto\frac{1}{2}\text{tr}(U\sigma_i U^\dag\sigma^j).
        \end{gather}
    \end{formula}

    \begin{property}
        For all $m, n\in\mathbb{N}$ we have an isomorphism
        \begin{gather}
            \text{Spin}(m, n)\cong\text{Spin}(n, m).
        \end{gather}
    \end{property}
    \begin{remark}
        Note that the above isomorphism only holds for the $\text{Spin}$-groups and not for the associated $\text{Pin}$-groups. This could have major consequences in physics. In general physicists freely switch between a $(1,3)$- and $(3,1)$-signature because all particles are assumed to be spinors (and not pinors). However, some pinors can only occur for one specific signature and this way it might be possible to detect the signature of the universe.
    \end{remark}
\chapter{Clifford Algebra}

\section{Clifford algebra}

	\newdef{Clifford algebra}{\index{Clifford!algebra}\label{clifford:clifford_algebra}
		Let $V$ be unital associative algebra. The Clifford algebra over $V$ with quadratic form $Q:V\rightarrow K$ is the free algebra generated by $V$ under the following condition:
		\begin{equation}
			\label{clifford:condition}
			v\cdot v = Q(v)1
		\end{equation}
		where $1$ is the unit element in $V$. This condition implies that the square of a vector is a scalar.
	}
	\begin{notation}
		The Clifford algebra corresponding to $V$ and $Q$ is denoted by $C\ell(V, Q)$.
	\end{notation}
	
	\begin{construct}
		The previous definition can be given an explicit construction. First we construct the tensor algebra of $V$:
		\begin{equation}
			T(V) = \bigoplus_{k\in\mathbb{N}}V^{\otimes k}
		\end{equation}
		Then we construct a two-sided ideal $I$ of $V$ generated\footnotemark\ by $\{v\otimes v - Q(v)1_V\ |\ v\in V\}$. The Clifford algebra $C\ell(V, Q)$ can then be constructed as the quotient algebra $T(V)/I$.
		\footnotetext{See definition \ref{group:generating_set_ideal}.}
	\end{construct}
	
	\begin{remark}
		Looking at definition \ref{tensor:adef_exterior_algebra} we see that the exterior algebra $\Lambda^*(V)$ coincides with the Clifford algebra $C\ell(V, 0)$. If $Q\neq0$ then the two algebras are still linearly isomorphic when $\text{char}(V)\neq2$.
	\end{remark}
	
	\begin{property}[Dimension]
		If $V$ has dimension $n$ then $C\ell(V, Q)$ has dimension $2^n$.	
	\end{property}
	
\section{Geometric algebra}

	\newdef{Geometric algebra}{\index{geometric algebra}
		Let $V$ be a vector space equipped with a symmetric bilinear form $g:V\times V\rightarrow K$. The geometric algebra (GA) over $V$ is defined as the Clifford algebra $C\ell(V, g)$. If $\text{char}(V)\neq2$ then the bilinear form uniquely determines a quadratic form $Q:v\mapsto g(v, v)$ as required in definition \ref{clifford:clifford_algebra}.
	}
	
	\newdef{Inner and exterior product}{\index{inner!product}\index{exterior!product}
		Analogous to the inner product in linear algebra and the wedge product in exterior algebras one can define an (a)symmetric product on the geometric algebra.
		
		First of all we note that the product $ab$ of two vectors $a$ and $b$ can be written as the sum of a symmetric and an antisymmetric part:
		\begin{equation}
			\label{clifford:geometric_product}
			ab = \stylefrac{1}{2}(ab + ba) + \stylefrac{1}{2}(ab - ba)
		\end{equation}
		We can then define the inner product as the symmetric part:
		\begin{equation}
			\label{clifford:inner_product}
			a\cdot b := \stylefrac{1}{2}(ab + ba) = \stylefrac{1}{2}\left((a+b)^2 - a^2 - b^2\right) = g(a, b)
		\end{equation}
		Analogously we define the exterior (outer) product as the antisymmetric part:
		\begin{equation}
			\label{clifford:exterior_product}
			a\wedge b := \stylefrac{1}{2}(ab - ba)
		\end{equation}
		
		These definitions allow us the rewrite formula \ref{clifford:geometric_product} as:
		\begin{equation}
			\boxed{ab = a\cdot b + a\wedge b}
		\end{equation}
	}
	\begin{remark*}
		Looking at the last equality in the definition of the inner product \ref{clifford:inner_product} we see that condition \ref{clifford:condition} is satisfied when $a=b$.
	\end{remark*}
	
	\begin{example}[Exterior algebra]
		When $g$ is fully degenerate, i.e. $g(v, v) = 0$ for all $v\in V$, the inner product is identically zero for all vectors and the geometric algebra coincides with exterior algebra\footnote{See definition \ref{tensor:exterior_algebra}.} over $V$. For general forms $g$ the exterior algebra is a subalgebra of the GA.
	\end{example}
	
	\newdef{Multivector}{\index{blade}\index{multi!vector}\index{pseudo!scalar}\index{pseudo!vector}
		Any element of the GA over $V$ is called a multivector. The simple multivectors of grade $k$, i.e. elements of the form $v_1v_2...v_k$ with $v_i\in V$ for all $i$, are called $k$-blades. This generalizes the remark underneath \ref{tensor:not:antysimmetric_space}. Sums of multivectors of different grades are called mixed multivectors\footnote{These elements do not readily represent a geometric structure.}.
		
		Let $n=\dim(V)$. Multivectors of grade $n$ are also called \textbf{pseudoscalars} and multivectors of grade $n-1$ are also called \textbf{pseudovectors}.
	}
	
	\newdef{Grade projection operator}{\index{projection}
		Let $a$ be a general multivector. The grade (projection) operator $\langle\cdot\rangle_k:\mathcal{G}\rightarrow\mathcal{G}_k$ is defined as the projection of $a$ on the $k$-vector part of $a$.
	}
	
	Using the grade operators we can extend the inner and exterior product to the complete GA as follows.
	\begin{formula}
		Let $A, B$ be two multivectors of respectively grade $m$ and $n$. Their inner product is defined as:
		\begin{equation}
			A\cdot B = \langle AB \rangle_{|m-n|}
		\end{equation}
		Their exterior product is defined as:
		\begin{equation}
			A\wedge B = \langle AB \rangle_{m+n}
		\end{equation}
	\end{formula}
	
\section{Pin group}
\subsection{Clifford group}

	\newdef{Transposition}{\index{transposition}
		Let $\{e_i\}_{i\leq n}$ be a basis for $V$. On the tensor algebra $T(V)$ there exists an anti-automorphism $v^t$ that reverses the order of the basis vectors:
		\begin{equation}
			\label{clifford:transposition}
			\cdot^t:e_i\otimes e_j\otimes\cdots\otimes e_k\mapsto e_k\otimes\cdots\otimes e_j\otimes e_i
		\end{equation}
		Because the ideal in the definition of a Clifford algebra is invariant under this map, it induces an anti-automorphism, called the transposition or \textbf{reversal}, on $C\ell(V)$.
	}

	\newdef{Main involution}{\index{involution}\index{inversion}
		Let $V_0, V_1$ be respectively the grade 0 and 1 components of the Clifford algebra $C\ell(V, Q)$. Consider the following operator:
		\begin{equation}
			\hat{v} = \begin{cases}
				v\qquad\qquad&v\in V_0\\
				-v\qquad\qquad&v\in V_1
			\end{cases}
		\end{equation}
		This operation can be generalized to all of $C\ell(V, Q)$ using linearity. The resulting operator is called the main involution or \textbf{inversion} on $C\ell(V, Q)$. Furthermore it turns the Clifford algebra into a superalgebra\footnote{See definition \ref{linalgebra:superalgebra}.}.
	}
	
	\newformula{Twisted conjugation}{
		Let $v\in V$ be a vector and let $s\in C\ell(V, Q)$ be a unit of the Clifford algebra over $V$, i.e. $Q(s) \neq 0$. The twisted conjugation of $v$ by $s$ is given by the map:
		\begin{equation}
			\chi: C\ell(V, Q)\times V: \chi(s)v = sv\hat{s}^{-1}
		\end{equation}
		This map clearly preserves the norm on $V$ and hence $\chi(s)$ is an element of $O(V, Q)$ for all units $s$.
	}
	\newdef{Clifford group}{\index{Clifford!group}
		The Clifford group $\Gamma(V, Q)$ is defined as follows:
		\begin{equation}
			\Gamma(V, Q) = \{s\in C\ell_{hom}(V, Q): sv\hat{s}^{-1} \in V, v\in V\}
		\end{equation}
		where the subscript $hom$ indicates that we only look at the homogeneous Clifford vectors. Because the units of $C\ell(V)$ form a group, $\Gamma(V, Q)$ also forms a group.
	}
	\begin{property}
		When looking at the units of $C\ell(V)$ that belong to $V$ itself, the twisted conjugation is given by a Householder transformation\footnote{See definition \ref{linalgebra:householder_transformation}.}.
	\end{property}
	
	\begin{property}
		If $V$ is finite-dimensional and $Q$ non-degenerate, the map
		\begin{equation}
			\chi:\Gamma(V, Q)\rightarrow O(V, Q): s\mapsto\chi(s)
		\end{equation}
		defines a representation\footnote{In char$(K)\neq2$, the surjectiveness of the map $\chi$ follows from the \textit{Cartan-Dieudonn\'e theorem}. But even in characteristic 2, the surjectiveness holds.} called the \textbf{vectorial representation}. Furthermore, from the first isomorphism theorem \ref{group:theorem:first_isomorphism_theorem} it follows that $O(V, Q)$ is isomorphic to $\Gamma(V, Q)/\ker\chi$ where $\ker\chi = \mathbb{R}_0$. This isomorphism also implies\footnote{Again using the \textit{Cartan-Dieudonn\'e theorem}, valid only when char$(K)\neq2$.} that the Clifford group is given by the set of finite products of invertible elements $v\in V$:
		\begin{equation}
			\Gamma(V, Q) = \left\{\prod_i^n s_i : s_i\text{ invertible in }V, n\in\mathbb{N}\right\}
		\end{equation}
	\end{property}
	\begin{result}
		By nothing that pure rotations can be decomposed as an even number of reflections we find that:
		\begin{equation}
			\Gamma^+(V, Q)/\mathbb{R}_0 \cong \text{SO}(V, Q)
		\end{equation}
		where $\Gamma^+$ is the intersection of the even Clifford algebra and the Clifford group.
	\end{result}

\subsection{Pin and Spin}

	\newformula{Spinor norm}{
		On $\Gamma(V, Q)$ one can define the spinor norm:
		\begin{equation}
			\mathcal{N}(x):\Gamma(V, Q)\rightarrow K^\times:x\mapsto xx^t
		\end{equation}
		where $x^t$ is the transposition \ref{clifford:transposition}. On $V$, $\mathcal{N}$ coincides with the norm induced by $Q$.
	}
	\newdef{Pin and spin groups}{\index{pin group}\index{spin!group}
		\nomenclature[S_Pin]{$\text{Pin}(V)$}{Pin group of the Clifford algebra $C\ell(V, Q)$.}
		Using the spinor norm $\mathcal{N}$ we can now define the pin and spins groups as follows:
		\begin{equation}
			\text{Pin}(V) = \{s\in\Gamma(V, Q): \mathcal{N}(s) = 1\}
		\end{equation}
		and
		\begin{equation}
			\text{Spin}(V) = \text{Pin}(V)\cap\Gamma^+(V, Q)
		\end{equation}
	}
	\begin{remark}
		The Pin group can also be defined as the set of elements in $\Gamma(V, Q)$ that can be written as a product of unit Clifford vectors. The Spin group is then defined as the elements that can be written as the product of an even number of unit Clifford vectors.
	\end{remark}
	
	\begin{property}
		The Pin group satisfies following isomorphicity relation:
		\begin{equation}
			\text{Pin}(V, Q)/\mathbb{Z}_2 \cong O(V, Q)
		\end{equation}
		and analogously for the Spin group and SO$(V, Q)$. These relations also imply that the Pin and Spin groups form a double cover\footnotemark\ of respectively the orthogonal and special orthogonal groups.
		\footnotetext{A covering group is a topological group that is also a covering space. See definition \ref{topology:covering_space} for more information about the latter.}
	\end{property}
	
	\newdef{Spinor}{\index{spinor}\index{spin!representation}
		Consider a vector space $V$ equipped with a representation of the group Spin$(n)$, called the \textbf{spin representation}. Elements of $V$ are called spinors.
	}
	
	\begin{example}
		The following table gives some group isomorphisms for the spin group in $\dim n$:
		\begin{equation*}
			\begin{array}{c|c}
				n&\text{Spin}(n)\\
				\hline
				1&\text{O}(1)\\
				2&\text{U}(1)\\
				3&\text{SU}(2)\\
				4&\text{SU}(2)\times\text{SU}(2)
			\end{array}
		\end{equation*}
		For quadratic forms of signature $(p, q)$ we find the following table:
		\begin{equation*}
			\begin{array}{c|c}
				(1, n)&\text{Spin}(1, n)\\
				\hline
				(1,1)&\text{GL}(1, \mathbb{R})\\
				(1,2)&\text{SL}(1, \mathbb{R})\\
				(1,3)&\text{SL}(2, \mathbb{C})
			\end{array}
		\end{equation*}
	\end{example}
	
	\begin{formula}
		Consider the basis of $\mathfrak{su}(2)$ given by the Pauli matrices \ref{QM:angular_momentum:pauli_matrices}. An explicit (double) covering map $\rho:\text{Spin}(3)\cong\text{SU}(2)\rightarrow\text{SO}(3)$ is given by:
		\begin{equation}
			\rho:U\mapsto\frac{1}{2}\text{tr}(U\sigma_i U^\dag\sigma^j)e_j\otimes\varepsilon^i
		\end{equation}
		where $\varepsilon^k$ is the dual\footnote{See equation \ref{linalgebra:dual_basis_2}.} of the basis vector $e_k$.
	\end{formula}

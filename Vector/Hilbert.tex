\chapter{Banach spaces and Hilbert spaces}

	In this chapter the term "linear operator", which is normally reserved for maps of the form $f:V\rightarrow V$, is used instead of "linear map". This was done to keep the vocabulary in track with that of the standard literature on Banach spaces and operator spaces.
	
	For a revision of inner product spaces see section \ref{linalgebra:innerproduct}.

\section{Banach spaces}

	\newdef{Norm}{\index{norm}
    		Let $V$ be a $K$-vector space. A function $||\vec{v}||:V\rightarrow[0,+\infty[$ is called a norm if it satisfies following conditions:
		\begin{itemize}
			\item \textbf{Non-degeneracy:} $||\vec{v}|| = 0 \iff \vec{v} = 0$
			\item \textbf{Homogeneity:} $||a\vec{v}|| = |a|||\vec{v}||$ for all scalars $a\in K$
			\item \textbf{Triangle equality (subadditivity):} $||\vec{v} + \vec{w}|| \leq ||\vec{v}|| + ||\vec{w}||$
		\end{itemize}
	}
	\remark{
		A norm $||\cdot||$ clearly induces a metric\footnotemark\ by setting $d(x,y) = ||x-y||$.
	}
	\footnotetext{See definition \ref{topology:metric}.}
    
	\newdef{Normed vector space}{
    		A $K$-vector space equipped with a norm $||\cdot||$.
	}
	\newdef{Banach space}{\index{Banach!space}\label{linalgebra:banach_space}
	    	A normed vector space that is complete\footnotemark\ with respect to the norm $||\cdot||$.
		\footnotetext{See condition \ref{topology:cauchy_sequence}.}
	}
	
	\newdef{Reflexive space}{
		A Banach space $V$ for which its dual coincides with the dual of its dual, i.e. $V^* = (V^*)^*$.
	}
	\begin{property}
		Every finite-dimensional Banach spaces is reflexive. This follows from property \ref{linalgebra:dual_space_dimension}.
	\end{property}
	
	\begin{property}
		Let $(x_n)$ be a Cauchy sequence in a normed space $V$. Then $(||x_n||)$ is a convergent sequence in $\mathbb{R}$. This implies that every Cauchy sequence in a normed space is bounded.
	\end{property}
	
	\begin{property}
		The topological (continuous) dual of a Banach space is also a Banach space.
	\end{property}

\subsection{Bounded operators}
	
	\newdef{Bounded operator}{\index{bounded operator}
		Let $L:V\rightarrow W$ be a linear operator between two Banach spaces. The operator is said to be bounded if there exists a scalar $M$ that satisfies the following condition:
		\begin{equation}
			\label{operator:bounded_operator}
			\boxed{\forall v\in V:||Lv||_W \leq M||v||_V}
		\end{equation}
	}
	\begin{notation}
		\nomenclature[S]{$\mathcal{B}(V, W)$}{Space of bounded continuous maps from the set $X$ to the set $Y$.}
		The space of bounded linear operators from $V$ to $W$ is denoted by $\mathcal{B}(V, W)$.
	\end{notation}
	\begin{property}
		If $V$ is a Banach space then $\mathcal{B}(V)$ is also a Banach space.
	\end{property}

	\newdef{Operator norm}{\index{operator!norm}
    		The operator norm of $L$ is defined as follows:
    		\begin{equation}
    			||L||_{op} = \inf\{M|M\text{ satisfies condition \ref{operator:bounded_operator}}\}
    		\end{equation}
    		As the name suggests it is a norm on $\mathcal{B}(V, W)$. The topology induced by this norm is called the norm topology.
    		
    		Equivalent definitions of the operator norm are:
    		\begin{equation}
    			||L||_{op} = \sup_{||x||\leq1}||L(x)|| = \sup_{||x||=1}||L(x)|| = \sup_{x\neq0}\stylefrac{||L(x)||}{||x||}
    		\end{equation}
	}
	
	Following property reduces the problem of continuity to that of boundedness.
	\begin{property}
		Let $f\in\mathcal{L}(V, W)$. Following statements are equivalent:
		\begin{itemize}
			\item $f$ is bounded.
			\item $f$ is continuous at 0.
			\item $f$ is continuous on $V$.
			\item $f$ is uniformly continuous.
			\item $f$ maps bounded sets to bounded sets.
		\end{itemize}
	\end{property}

	\begin{property}
		Let $A$ be a bounded linear operator with eigenvalue $\lambda$. We then have:
		\begin{equation}
			|\lambda|\leq||A||_{op}
		\end{equation}
	\end{property}
	\begin{property}
		Let $A$ be a bounded linear operator. Let $A^\dag$ denote its adjoint\footnotemark. Then $A^\dag$ is bounded and $||A||_{op} = ||A^\dag||_{op}$.
		\footnotetext{See definition \ref{linalgebra:adjoint_operator}.}
	\end{property}

\subsection{Theorems}

	\begin{property}
		Let $X$ be a general TVR. Every linear map $\varphi:\mathbb{K}^n\rightarrow X$ is continuous.
	\end{property}
	\begin{property}
		Let $X$ be a finite-dimensional normed vector space. Every linear bijection $\varphi:\mathbb{K}^n\rightarrow X$ is a homeomorphism.
	\end{property}
	\begin{result}
		Two finite-dimensional normed vector spaces with the same dimension are homeomorphic. It follows that all metrics on a finite-dimensional normed vector space are equivalent.
	\end{result}

	\begin{theorem}[Open mapping theorem\footnotemark]\index{open mapping theorem}
		\footnotetext{Sometimes called the \textit{Banach-Schauder} theorem.}
		Let $f:V\rightarrow W$ be a continuous linear operator between two Banach spaces. If $f$ is surjective then it also open.
	\end{theorem}

\subsection{Spectrum}

	\newdef{Resolvent set}{\index{resolvent set}
		Let $A$ be a bounded linear operator on a normed space $V$. The resolvent set $\rho(A)$ consists of all scalar $\lambda\in\mathbb{C}$ such that $(A-\lambda\mathbbm{1})^{-1}$ is a bounded linear operator, called the resolvent of $A$, on a dense subset of $V$. These scalars $\lambda$ are called \textbf{regular values} of $A$.
	}
	\newdef{Spectrum}{\index{spectrum}
		The set of scalars $\mu\not\in\rho(A)$ is called the spectrum of $A$.
	}
	
	\begin{remark}
		It is obvious from the definition of an eigenvalue that every eigenvalue of $A$ belongs to the spectrum of $A$. The converse however is not true.
	\end{remark}
	
	\newdef{Point spectrum}{
		The set of scalars $\mu\in\mathbb{C}$ for which the resolvent of $A$ fails to be injective is called the point spectrum of $A$. This set contains exactly the eigenvalues of $A$.
	}
	\newdef{Continuous spectrum}{
		The set of scalars $\mu\in\mathbb{C}$ for which the resolvent of $A$ fails to be surjective but for which the range of the resolvent is dense in $V$ is called the continuous spectrum of $A$. The scalars for which the range is not dense is called the \textbf{residual spectrum} $\sigma_r(A)$.
	}
	\newdef{Compression spectrum}{
		The set of scalars $\mu\in\mathbb{C}$ for which the resolvent of $A$ fails to have a dense range in $V$ is called the compression spectrum $\sigma(A)$. It follows that $\sigma_r(A)\subseteq\sigma(A)$.
	}

\section{Hilbert space}

	\newdef{Hilbert space}{\index{Hilbert!space}\label{hilbert:hilbert_space}
		\nomenclature[S]{$\mathcal{H}$}{Hilbert space}
		A vector space that is both a Banach space and an inner product space (where the norm is induced by the inner product).
	}

	\begin{example}
		Let $f, g \in \mathcal{L}^2([a,b], \mathbb{C})$, the inner product of $f$ and $g$ is defined as:
		\begin{equation}
			\label{hilbert:inner_product}
		        \boxed{\langle f|g\rangle = \int_a^bf^*(x)\overline{g(x)}dx}
		\end{equation}
	\end{example}
	\remark{
		See section \ref{lebesgue:section:hilbert_space} for a more formal treatment of this subject.
	}
    
	\begin{formula}
		It is also possible to define an inner product with respect to a weight function $\phi(x)$:
		\begin{equation}
			\label{hilbert:weighted_inner_product}
			\int_a^bf^*(x)g(x)\phi(x)dx
		\end{equation}
		Using this formula it is possible to define orthogonality with respect to a weight function.
	\end{formula}
    
\subsection{Inner products and norms}
	\begin{formula}
		Let $V$ be an inner product space. A norm on $V$ can be induced by the inner product in the following way:
		\begin{equation}
			\label{linalgebra:inner_product:norm}
			||v||^2 = \langle v|v \rangle
		\end{equation}
		However not every norm induces an inner product. Only norms that satisfy the parallellogram law \ref{linalgebra:parallellogram_law} induce an inner product. This inner product can be recovered through the polarization identity \ref{linalgebra:polarization_identity} (see below).
	\end{formula}
	
	\begin{property}[Cauchy-Schwarz inequality]\index{Cauchy!Cauchy-Schwarz inequality}\label{linalgebra:theorem:cauchy_schwarz}
		\begin{equation}
			\boxed{|\langle v|w\rangle| \leq ||v||\ ||w||}
		\end{equation}
		where the equality holds if and only if $v$ and $w$ are linearly dependent.
	\end{property}
	\begin{result}
		The Cauchy-Schwarz inequality can be used to prove the triangle inequality. Together with the properties of an inner product this implies that an inner product space is also a normed space.
	\end{result}
	
	\begin{formula}[Parallellogram law]\index{parallellogram law}
		\begin{equation}
			\label{linalgebra:parallellogram_law}
			||v+w||^2 + ||v-w||^2 = 2(||v||^2 + ||w||^2)
		\end{equation}
	\end{formula}
	\begin{formula}[Polarization identity]\index{polarization identity}
		\begin{equation}
			\label{linalgebra:polarization_identity}
			4 \langle v|w \rangle = ||v+w||^2 - ||v-w||^2 + i\left(||v+iw||^2 - ||v-iw||^2\right)
		\end{equation}
	\end{formula}
	\begin{formula}[Pythagorean theorem]\index{Pythagorean theorem}
		In an inner product space the triangle equality reduces to the well-known Pythagorean theorem for orthogonal vectors $v, w$:
		\begin{equation}
			\label{linalgebra:pythagorean_theorem}
			||v+w||^2 = ||v||^2 + ||w||^2
		\end{equation}
		This formula can be extended to any set of orthogonal vectors $x_1, ..., x_n$:
		\begin{equation}
			\boxed{\left\lVert\sum_{i=1}^nx_i\right\rVert^2 = \sum_{i=1}^n||x_i||^2}
		\end{equation}
	\end{formula}

\subsection{Generalized Fourier series}

	\begin{property}[Bessel's inequality]\index{Bessel!inequality}
		First of all we have following general equality for orthonormal vectors $x_1, ..., x_n$ and complex scalars $a_1, ..., a_n$:
		\begin{equation}
			\left\lVert x - \sum_{i=1}^n a_ix_i\right\rVert^2 = ||x||^2 - \sum_{i=1}^n|\langle x, x_i\rangle|^2 + \sum_{i=1}^n|\langle x, x_i\rangle - a_i|^2
		\end{equation}
		This expression becomes minimal for $a_i = \langle x, x_i\rangle$ (last term becomes 0). This leads to Bessel's inequality:
		\begin{equation}
			\label{norm:bessels_inequality}
			\boxed{\sum_{i=1}^n|\langle x, x_i\rangle|^2 \leq ||x||^2}
		\end{equation}
	\end{property}
	\begin{result}\index{Fourier!generalized series}
		The sum in \ref{norm:bessels_inequality} is bounded for all $n$, so the series $\sum_{i=1}^{+\infty}$ converges for all $x$. This implies that the sequences $(\langle x, x_n\rangle)$ belongs to the space $l^2$ of square-summable sequences.
	\end{result}
	
	This result does however not imply that the generalized Fourier series $\sum_{i=1}^{+\infty}\langle x, x_i\rangle x_i$ converges to $x$. The following theorem gives a necessary and sufficient condition for the convergence.
	\begin{theorem}
		Let $\mathcal{H}$ be a Hilbert space. Let $(x_n)$ be an orthonormal sequence in $\mathcal{H}$ and let $(a_n)$ be a sequence in $\mathbb{C}$. The expansion $\sum_{i=1}^{+\infty}a_ix_i$ converges in $\mathcal{H}$ if and only if $(a_n)\in l^2$. Furthermore the expansion satisfies following equality:
		\begin{equation}
			\left\lVert\sum_{i=1}^{+\infty}a_ix_i\right\rVert^2 = \sum_{i=1}^{+\infty}|a_i|^2
		\end{equation}
		As we noted the sequence $(\langle x, x_n\rangle)$ belongs to $l^2$ so the generalized Fourier series converges of $x\in\mathcal{H}$ converges in $\mathcal{H}$.
	\end{theorem}
	\begin{remark}
		Although the convergence of the generalized Fourier series of $x\in\mathcal{H}$ can be established using previous theorem, it does not follow that the expansion converges to $x$ itself. We can merely say that the Fourier expansion is the best approximation of $x$ with respect to the norm on $\mathcal{H}$.
	\end{remark}
    
\subsection{Complete sets}

	\newdef{Complete set}{\index{complete set}
		Let $\{e_i\}_{i\in I}$ be a set (possibly a sequence) of orthonormal vectors in an inner product space $V$. This set is said to be complete if every vector $x\in V$ can be expressed as follows:
		\begin{equation}
			x = \sum_{i\in I}\langle x, x_i\rangle x_i
		\end{equation}
		This implies that a complete set is a basis for the vector space.
	}
	Another characterization is the following.
	\begin{adefinition}
		A complete set of orthonormal vectors is a set $S\subset V$ such that we cannot add another vector $w$ to it satisfying:
		\begin{equation}
			\forall v_i\in S: \langle v_i, w\rangle = 0\qquad\land\qquad  w\neq0
		\end{equation}
	\end{adefinition}
	
	\begin{property}
		For complete sequences $(x_n)$ the inequality of Bessel \ref{norm:bessels_inequality} becomes an equality. Furthermore, the generalized Fourier series with respect to the complete sequence is unique.
	\end{property}
	
	Using previous property we can prove the following theorem due to Parceval.
	\begin{theorem}[Parceval]\index{Parceval}
		Let $(x_n)$ be a complete sequence in a Hilbert space $\mathcal{H}$. Every vector $x\in\mathcal{H}$ has a unique Fourier series representation $\sum_{i=1}^{+\infty}a_ix_i$ where the Fourier coefficients $(a_i)$ belong to $l^2$ and the inequality of Bessel is an equality.
		
		Conversely if the inequality of Bessel becomes an equality for every $x\in\mathcal{H}$ then the sequence $(x_n)$ is complete.
	\end{theorem}
	
	\begin{property}
		A sequence $(x_n)$ in a Hilbert space $\mathcal{H}$ is complete if and only if $\langle x, x_i\rangle = 0$ for all $x_i$ implies that $x=0$.
	\end{property}

\subsection{Orthogonality and projections}

	The basic notions on orthogonality in inner product space can be found in section \ref{linalgebra:section:orthogonality}.

	\begin{property}
		Let $S$ be a subset (not necessarily a subspace) of a Hilbert space $\mathcal{H}$. The orthogonal complement $S^\perp$ is closed in $\mathcal{H}$.
	\end{property}
	\begin{result}
		The previous property implies that the orthogonal complemement of some arbitrary subset of a Hilbert space is a Hilbert space itself.
	\end{result}
	
	\begin{theorem}[Projection theorem]
		\label{linalgebra:theorem:projection_theorem}
		Let $H$ be a Hilbert space and $K\leq H$ a complete subspace. For every $h\in H$ there exists a unique $h'\in K$ such that $h-h'$ is orthogonal to every $k\in K$, i.e $h-h'\in K^\perp$.
	\end{theorem}
	\remark{
		An equivalent definition for the unique $h'\in K$ is $||h-h'|| = \inf\{||h-k||:k\in K\}$.
	}
	\begin{result}
		It follows that given a complete (or closed) subspace $S$ the Hilbert space $\mathcal{H}$ can be decomposed as $\mathcal{H} = S\oplus S^\perp$.
	\end{result}

\subsection{Separable Hilbert spaces}

	The definition of separable spaces in the sense of point-set topology is given in \ref{topology:separable}. An equivalent definition for Hilbert spaces is the following.
	\newadef{Separable Hilbert space}{
		A Hilbert space is separable if it contains a complete sequence of orthonormal vectors.
	}
	\begin{result}
		Using the Gram-Schmidt method it follows from previous definition that every finite-dimensional Hilbert space is separable.
	\end{result}

	The following theorem shows that (up to an isomorphism) there are only 2 distinct types of separable Hilbert spaces.
	\begin{theorem}
		Let $\mathcal{H}$ be separable. If $\mathcal{H}$ is finite-dimensional with dimension $n$ then it is isometrically isomorphic to $\mathbb{C}^n$. If $\mathcal{H}$ is infinite-dimensional then it is isometrically isomorphic to $l^2$.
	\end{theorem}
	\begin{property}
		Every orthogonal subset of a separable Hilbert space is countable.
	\end{property}

\subsection{Compact operators}

	\newdef{Compact operator}{\index{compact!operator}
		Let $A$ be a linear operator on a Hilbert space $\mathcal{H}$. $A$ is said to be compact if for every sequence $(x_n)$ in $\mathcal{H}$ the sequence $(A[x_n])$ has a convergent subsequence.
	}
	
	\begin{property}
		Every compact operator on a Hilbert space is bounded.
	\end{property}
	\begin{result}
		Every linear operator on a finite-dimensional Hilbert space is bounded.
	\end{result}

\subsection{Linear functionals}

	\begin{property}
		Let $f$ be a continuous linear functional. Then $\dim(\ker f)^\perp$ is 0 or 1 where the former case only arises when $f\equiv0$.
	\end{property}

	\begin{theorem}[Riesz' representation theorem]\index{Riesz'!representation theorem}
    		Let $\mathcal{H}$ be a Hilbert space. For every continuous linear functional $\rho:\mathcal{H}\rightarrow\mathbb{R}$ there exists a unique element $x_0\in\mathcal{H}$ such that
    		\begin{equation}
	    		\rho(h) = \langle h, x_0 \rangle
    		\end{equation}
	    	for all $h\in\mathcal{H}$. This implies that $\mathcal{H}$ and $\mathcal{H}^*$ are isometrically isomorphic. Furthermore the operator norm of $\rho$ is equal to the norm of $x_0$.
	\end{theorem}
	\begin{remark}
    		This theorem justifies the bra-ket notation used in quantum mechanics where one associates to every ket $|\psi\rangle\in\mathcal{H}$ a bra $\langle\psi|\in\mathcal{H}^*$.
	\end{remark}

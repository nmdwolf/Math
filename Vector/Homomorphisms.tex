\subsection{Homomorphisms}

	\newdef{Homomorphism space}{\index{morphism!of vector spaces}
		\nomenclature[S]{$\hom(V, W)$}{The set of morphisms from a set $V$ to a set $W$.}
    		Let $V,W$ be two K-vector spaces. The set of all linear maps between $V$ and $W$ is called the homomorphism space of $V$ to $W$, or shorter: the 'hom-space' of $V$ to $W$.
	    	\begin{equation}
        		\label{linalgebra:hom_space}
        		\hom_K(V, W) = \{f:V\rightarrow W\ |\ \text{f is a linear map}\}
		\end{equation}
	}
	\begin{theorem}\label{linalgebra:hom_dimension}
	    	If $\ V,W$ are two finite-dimensional K-vector spaces we have:
        	\begin{equation}
        		\dim\left(\hom_K(V,W)\right) =\dim(V)\cdot\dim(W)
		\end{equation}
	\end{theorem}
        
	\newdef{Endomorphism ring}{
		\nomenclature[S]{$\text{End}(V)$}{The ring of endomorphisms on a set $V$.}
		The space $\hom_K(V, V)$ with the composition as multiplication forms a ring, the endomorphism ring. It is denoted as $\text{End}_K(V)$ or $\text{End}(V)$.
	}
	\begin{property}
		The endomormphism ring $\text{End}(V)$ forms a Lie algebra\footnote{See also \ref{lie:end_as_lie_algebra}.} when equipped with the commutator $[A, B] = A\circ B - B\circ A$.
	\end{property}
	
	\begin{property}[Jordan-Chevalley decomposition]\index{Jordan!decomposition}\index{semisimple!operator}\index{nilpotent}\label{linalgebra:jordan_chevalley}
		Every endomorphism $A$ can be decomposed as follows:
		\begin{equation}
			A = A_{ss} + A_n
		\end{equation}
		where
		\begin{itemize}
			\item $A_{ss}$ is \textbf{semisimple}, i.e. for every the invariant subspace of $A_{ss}$ there exists an invariant complementary subspace.
			\item $A_n$ is \textbf{nilpotent}, i.e. $\exists k\in\mathbb{N}: A_n^k = 0$.
		\end{itemize}
		Furthermore, this decomposition is unique and the endomorphisms $A_{ss}, A_n$ can be written as polynomials in $A$.
	\end{property}
    
	\newdef{Minimal polynomial}{\index{minimal polynomial}
	    	Let $f\in\text{End}(V)$ and $V$ a finite-dimensional K-vector space. The monic polynomial $\mu_f(x)$ of the lowest order such that $\mu_f(f)=0$ is called the minimal polynomial of $f$.
	}
	\begin{property}\label{linalgebra:minimal_polynomial_divisor}
		Let $f\in\text{End}(V)$. Let $\mu_f(x)$ be the minimal polynomial of $f$. Let $\varphi(x)\in K[x]$. If $\varphi(f) = 0$, then the minimal polynomial $\mu_f(x)$ divides $\varphi(x)$. 
	\end{property}
    
\subsection{Dual space}

	\newdef{Dual space}{\index{dual!space}
	    	Let $V$ be a K-vector space. The dual space $V^*$ of $V$ is the following vector space:
	    	\begin{equation}
			\label{linalgebra:dual_space}
		        V^*:=\text{Hom}_K(V, K)=\{f:V\rightarrow K\ :\ f\text{ is a linear map}\}
		\end{equation}
	}
	\newdef{Linear form}{\index{linear!form}
	    	The elements of $V^*$ are called \textit{linear forms}.
	}
	\begin{property}\label{linalgebra:dual_space_dimension}
		From theorem \ref{linalgebra:hom_dimension} it follows that $\dim(V^*) = \dim(V)$.
	\end{property}
	\begin{remark}
		If $V$ is infinite-dimensional, theorem \ref{linalgebra:dual_space_dimension} is not valid. In the infinite-dimensional case we \textbf{always} have $|V^*|>|V|$ (where we now use the cardinality instead of the dimension).
	\end{remark}
    
	\newdef{Dual basis}{
    		Let $\mathcal{B} = \{e_1, e_2, ..., e_n\}$ be a basis for a finite-dimensional K-vector space $V$. We can define a basis $\mathcal{B}^* = \{\varepsilon_1, \varepsilon_2, ..., \varepsilon_n\}$ for $V^*$, called the dual basis of $\mathcal{B}$, as follows:
    		\begin{equation}
			\label{linalgebra:dual_basis}
        		\boxed{\varepsilon_i:V\rightarrow K:\sum_{j=1}^na_ie_i\mapsto a_i}
		\end{equation}
		The relation between the basis and dual basis can also be written as:
		\begin{equation}
			\label{linalgebra:dual_basis_2}
			\varepsilon^i(e_j) = \delta^i_j
		\end{equation}
	}
    
	\newdef{Dual map}{\index{dual!map}\label{linalgebra:transpose}
		Let $f:V\rightarrow W$ be a linear map. The linear map $f^*:W^*\rightarrow V^*:\varphi\rightarrow\varphi\circ f$ is called the dual map or \textbf{transpose} of $f$.
	}
	\newnot{Transpose}{\index{transpose}
		When $V=W$ the dual map $f^*$ is often denoted by $f^T$.
	}
    
    \newdef{Natural pairing}{\index{natural!pairing}
    	The natural pairing of $V$ and its dual $V^*$ is defined as the following bilinear map:
        \begin{equation}
        	\label{linalgebra:natural_pairing}
            \langle v, v^*\rangle = v^*(v)
        \end{equation}
    }

\subsection{Convex functions}
	
	\newdef{Convex function}{\index{convex!function}
		Let $X$ be a convex subset of $V$. A function $f:X\rightarrow \mathbb{R}$ is convex if for all $x, y\in X$ and $t\in[0, 1]$:
		\begin{equation}
			f(tx + (1-t)y) \leq tf(x) + (1-t)f(y)
		\end{equation}
	}
	\begin{remark}
		For a concave function we have to turn the inequality around.
	\end{remark}
	\begin{result}
		A linear map $f:X\rightarrow\mathbb{R}$ is both convex and concave.
	\end{result}
	
	\begin{theorem}[Karamata's inequality]\index{Karamata's inequality}
		Let $I\subset\mathbb{R}$ be an interval and let $f:I\rightarrow\mathbb{R}$ be a convex function. If $(x_1, ..., x_n)$ is a tuple that majorizes $(y_1, ..., y_n)$, i.e. $\forall k\leq n$
		\begin{align}
			\sum_{i=1}^nx_i &= \sum_{i=1}^ny_i\\
			x_{(1)} + ... + x_{(k)}&\geq y_{(1)} + ... + y_{(k)}
		\end{align}
		where $x_{(i)}$ denotes the ordering\footnote{In decreasing order: $x_{(1)}\geq...\geq x_{(n)}$.} of the tuple $(x_1, ..., x_n)$. Then
		\begin{equation}
			\sum_{i=1}^nf(x_i)\geq \sum_{i=1}^nf(y_i)
		\end{equation}
	\end{theorem}

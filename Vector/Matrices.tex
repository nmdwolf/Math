\section{Matrices}
	
	\begin{notation}\label{linalgebra:matrix_set}
	        The set of all $m\times n$-matrices defined over the field K is denoted by $M_{m,n}(K)$. If $m=n$, the set is denoted by $M_n(K)$.
	\end{notation}
	
	\begin{property}[Dimension]\index{dimension}\label{linalgebra:dimension_of_matrix_space}
		The dimension of $M_{m,n}(K)$ is $mn$.
	\end{property}
    
	\newdef{Trace}{\index{trace}\label{linalgebra:trace}
	        Let $A = (a_{ij})\in M_n(K)$. We define the trace of $A$ as follows:
    		\begin{equation}
			\boxed{\operatorname{tr}(A) = \sum_{i=1}^na_{ii}}
		\end{equation}
	}
	\begin{property}\label{linalgebra:trace_commutative}
		Let $A, B\in M_n(K)$. We have the following properties of the trace:
	        \begin{enumerate}
        		\item $\text{tr}:M_n(K)\rightarrow K$ is a linear map
			\item $\text{tr}(AB) = \text{tr}(BA)$
			\item $\text{tr}(AB) \neq \text{tr}(A)\text{tr}(B)$
        		\item $\text{tr}(A^T) = \text{tr}(A)$
		\end{enumerate}
	\end{property}
    
	\newformula{Hilbert-Schmidt norm\footnotemark}{\index{Frobenius!norm}\index{Hilbert-Schmidt norm}
    		\footnotetext{Also called the \textbf{Frobenius norm}.}
    		The Hilbert-Schmidt matrix norm is given by following formula:
	        \begin{equation}
        		\label{linalgebra:hilbert_schmidt_norm}
        		||A||^2_{HS} = \sum_{i, j}|A_{ij}|^2 = \text{tr}(A^\dag A)
	        \end{equation}
        	If one identifies $M_{n}(\mathbb{C})$ with $\mathbb{C}^{2n}$ then this norm equals the standard Hermitian norm.
	}

	\newformula{Hadamard product}{\index{Hadamard!product}
    		The Hadamard product of two matrices $A, B\in M_{m\times n}(K)$ is defined as the entry-wise product:
        	\begin{equation}
        		(A\circ B)_{ij} = A_{ij}B_{ij}
        	\end{equation}
	}
    
	\newdef{General linear group}{\index{general linear group}\label{linalgebra:GL_matrices}
		\nomenclature[S_GLn]{GL$_n(K)$}{General linear group: group of all invertible $n$-dimensional matrices over the field $K$.}
	        The set of invertible matrices is called the general linear group and is denoted by GL$_n(K)$.
	}
	\begin{property}
		For all $A\in\text{GL}_n(K)$ we have:
	        \begin{itemize}
			\item $A^T\in\text{GL}_n(K)$
		        \item $\left(A^T\right)^{-1}=\left(A^{-1}\right)^T$
		\end{itemize}
	\end{property}
    
	\begin{property}\label{linalgebra:dim_columns_rows}
		Let $A\in M_{m,n}(K)$. Denote the set of columns of $A$ as $\{A_1, A_2, ..., A_n\}$ and the set of rows of $A$ as $\{R_1, R_2, ..., R_m\}$. The set of columns is a subspace of $K^m$ and the set of rows is a subspace of $K^n$. Furthermore we have:
		\[
			\dim(\text{span}(A_1, ..., A_n)) = \dim(\text{span}(R_1, ..., R_m))
		\]
	\end{property}
	
	\newdef{Rank of a matrix}{\index{rank}
		We can define the rank of matrix $A\in M_{m,n}(K)$ as follows:
		\begin{equation}
			\label{linalgebra:matrix_rank}
        			\text{rk}(A) := \dim(\text{span}(A_1, ..., A_n)) \overset{\ref{linalgebra:dim_columns_rows}}{=} \dim(\text{span}(R_1, ..., R_m))
		\end{equation}
	}
    
	\begin{property}\label{linalgebra:rank_properties}
	        The rank of a matrix has the folowing properties:
        	\begin{enumerate}
			\item Let $A\in M_{m,n}(K)$ and $B\in M_{n,r}(K)$. We have $\text{rk}(AB)\leq\text{rk}(A)$ and $\text{rk}(AB)\leq\text{rk}(A)$.
        		\item Let $A\in\text{GL}_n(K)$ and $B\in M_{n,r}(K)$. We have $\text{rk}(AB)=\text{rk}(B)$.
		        \item Let $A\in\text{GL}_n(K)$ and $B\in M_{r,n}(K)$. We have $\text{rk}(BA)=\text{rk}(B)$.
		\end{enumerate}
	\end{property}
	\begin{property}\label{linalgebra:dim_matrix_left_multiplication}
	        Let $A\in M_{m,n}(K)$. First define the following linear map:
        	\begin{equation}
			\label{linalgebra:matrix_left_multiplication}
        		\boxed{L_A:K^n\rightarrow K^m:v\mapsto Av}
		\end{equation}
	        This map has the following properties:
        	\begin{enumerate}
        		\item $\text{im}(L_A) = \text{span}(A_1, ..., A_n)$
			\item $\dim(\text{im}(L_A))=\text{rk}(A)$
		\end{enumerate}
	\end{property}
	\sremark{The second property is a direct consequence of the first one and definition \ref{linalgebra:matrix_rank}.}
    
\subsection{System of equations}

	\begin{theorem}\label{linalgebra:matrix_and_equations}
	        Let $AX=w$ with $A\in M_{m,n}(K)$, $w\in K^m$ and $X\in K^n$ be a system of $m$ equations in $n$ variables. Let $L_A$ be the linear map as defined in equation \ref{linalgebra:matrix_left_multiplication}. We then have the following properties:
        	\begin{enumerate}
			\item The system is false if and only if $w\not\in\text{im}(L_A)$.
        		\item If the system is not false, the solution set is an affine space. If $v_0\in K^n$ is a solution, then the solution set is given by: $L_A^{-1}(w)=v_0+\text{ker}(L_A)$.
		        \item If the system is homogeneous ($AX=0$), then the solution set is equal to $\text{ker}(L_A)$.
		\end{enumerate}
	\end{theorem}
	\begin{theorem}[Uniqueness]\label{linalgebra:rank_unique_solution}
	        Let $AX=w$ with $A\in M_n(K)$ be a system of $n$ equations in $n$ variables. If $\text{rk}(A)=n$, then the system has a unique solution.
	\end{theorem}
    
\subsection{Coordinates and matrix representations}

        \newdef{Coordinate vector}{\label{linalgebra:coordinate_vector}
        	Let $\mathcal{B} = \{b_1, ..., b_n\}$ be a basis of $V$. Let $v\in V$ such that $v=\sum_{i=1}^n\lambda_ib_i$. We define the coordinate vector of $v$ with respect to $\mathcal{B}$ as $(\lambda_1, ..., \lambda_n)^T$. The $\lambda_i$'s are called the \textbf{coordinates} of $v$ with respect to $\mathcal{B}$.
        }
        \newdef{Coordinate isomorphism}{\label{linalgebra:coordinate_isomorphism}
        	With the previous definition in mind we can define the coordinate isomorphism of $v$ with respect to $\mathcal{B}$ as follows:
        	\begin{equation}
			\boxed{\beta:V\rightarrow K^n:\sum_{i=1}^n\lambda_ib_i\mapsto(\lambda_1, ..., \lambda_n)^T}
		\end{equation}
        }
        
        \newdef{Matrix representation}{\label{linalgebra:matrix_representation}
        	Let $V$ be an $n$-dimensional K-vector space and $W$ an $m$-dimensional K-vector space. Let $f:V\rightarrow W$ be a linear map. Let $\mathcal{B}=\{b_1, ..., b_n\}, \mathcal{C} = \{c_1, ..., c_m\}$ be a basis for $V$, respectively $W$. The matrix representation of $f$ with respect to $\mathcal{B}$ and $\mathcal{C}$ can be derived as follows: For every $j\in\{1, ..., n\}$ we can write $f(b_j) = \sum_{i=1}^ma_{ij}c_i$, so with this in mind we can define the matrix $(a_{ij})\in M_{m,n}(K)$ as the matrix represenation of $f$.
        }
        \newnot{Matrix representation of a linear map}{
        	The matrix representation of $f$ with respect to $\mathcal{B}$ and $\mathcal{C}$ is denoted by $A_{f, \mathcal{B}, \mathcal{C}}$.
        }
        
        \newmethod{Construction of a matrix representation}{\label{linalgebra:method:matrix_representation}
        	From definition \ref{linalgebra:matrix_representation} we can see that $j$-th column of $A_{f, \mathcal{B}, \mathcal{C}}$ coincides with the coordinate vector of $f(b_j)$ with respect to $\mathcal{C}$. We use this relation to construct $A_{f, \mathcal{B}, \mathcal{C}}$ by writing for every $j\in\{1, ..., n\}$ the coordinate vector of $f(b_j)$ in the $j$-th column.
        }
        
        \begin{theorem}\label{linalgebra:theorem:matrix_representation}
		Let $(\lambda_1, ..., \lambda_n)^T$ be the coordinate vector of $v\in V$ with respect to $\mathcal{B}$. Let $(\mu_1, ..., \mu_m)^T$ be the coordinate vector of $f(v)$ with respect to $\mathcal{C}$. Then the following relation holds:
            	\begin{equation}
			\left(
			\begin{array}{c}
				\mu_1\\
				\vdots\\
				\mu_m
			\end{array}\right)
	                = A_{f, \mathcal{B}, \mathcal{C}}
        	        \left(\begin{array}{c}
				\lambda_1\\
				\vdots\\
				\lambda_n
			\end{array}\right)
		\end{equation}
	\end{theorem}
        
        \begin{theorem}\label{linalgebra:theorem:map_matrix_link}
		For every matrix $A\in M_{m,n}(K)$ there exists a linear map $f:V\rightarrow W$ such that $A_{f, \mathcal{B}, \mathcal{C}} = A$.
	\end{theorem}
        On the other hand we also have the following theorem:
        \begin{theorem}
		Let $f:K^n\rightarrow K^m$ be a linear map. There exists a matrix $A\in M_{m,n}(K)$ such that $f=L_A$.
	\end{theorem}
        \begin{theorem}
		Let $\beta$ and $\gamma$ be the coordinate isomorphisms with respect to respectively $\mathcal{B}$ and $\mathcal{C}$. From theorem \ref{linalgebra:theorem:matrix_representation} it follows that:
        	\begin{equation}
			\gamma(f(v)) = A_f\cdot\beta(v)
		\end{equation}
        	or alternatively
        	\begin{equation}
			\gamma\circ f = L_{A_f}\circ\beta
		\end{equation}
	\end{theorem}
        
        \begin{theorem}\label{linalgebra:theorem:matrix_composition_hom}
	        The map $\text{Hom}_K(V,W)\rightarrow M_{m,n}(K):f\mapsto A_f$ is an isomorphism and for every $f\in\text{Hom}_K(V,W)$ and $g\in \text{Hom}_K(W,U)$ we have:
		\begin{equation}
			A_{g\circ f} = A_gA_f
		\end{equation}
	\end{theorem}
	
	\begin{theorem}\label{linalgebra:theorem:matrix_composition_end}
	        The map $\text{End}_K(V)\rightarrow M_n(K):f\mapsto A_{f, \mathcal{B}, \mathcal{B}}$ is an isomorphism and for every $f,g\in\text{End}_K(V)$ we have:
        	\begin{equation}
			A_{g\circ f} = A_gA_f
		\end{equation}
	\end{theorem}
	
        \begin{theorem}\label{linalgebra:matrix_invertable_map}
	        Let $f\in\text{End}_K(V)$. Let $A_f$ be the corresponding matrix representation. The linear map $f$ is invertible if and only if $A_f$ is invertible. Furthermore, if $A_f$ is invertible, we have that \[\left(A_f\right)^{-1} = A_{f^{-1}}\] In other words, the following map is an isomorphism\footnotemark:
	        \begin{equation}
	        	\text{GL}_K(V)\rightarrow\text{GL}_n(K):f\mapsto A_f
	        \end{equation}
	        \footnotetext{Follows from theorem \ref{linalgebra:theorem:matrix_composition_end}.}
	\end{theorem}
        \remark{The sets GL$_K(V)$ and GL$_N(K)$ are groups. So the previous theorem states that the map $f\mapsto A_f$ is a group isomorphism.}
        
        \begin{theorem}
		Let $V = K^n$. Let $f\in V^*$. From construction \ref{linalgebra:method:matrix_representation} it follows that $A_f = (f(e_1), ..., f(e_n))\in M_{1,n}(K)$ with respect to the standard basis of $V$. This combined with theorem \ref{linalgebra:theorem:matrix_representation} gives:
	        \begin{equation}
			f(\lambda_1, ..., \lambda_n)^T = (f(e_1), ..., f(e_n))(\lambda_1, ..., \lambda_n)^T = \sum_{i=1}^nf(e_i)\lambda_i
		\end{equation}
        	or alternatively with $\{\varepsilon_1, ..., \varepsilon_n\}$ the dual basis to the standard basis of $V$:
	        \begin{equation}
        	    	\label{linalgebra:map_in_function_of_dual_basis}
			\boxed{f = \sum_{i=1}^nf(e_i)\varepsilon_i}
		\end{equation}
	\end{theorem}
        
        \begin{theorem}
		Let $f:V\rightarrow W$ be a linear map. Let $f^*:W^*\rightarrow V^*$ be the corresponding dual map. If $A_f$ is the matrix representation of $f$ with respect to $\mathcal{B}$ and $\mathcal{C}$, then the transpose $A_f^T$ is the matrix representation of $f^*$ with respect to the dual basis of $\mathcal{C}$ and the dual basis of $\mathcal{B}$.
	\end{theorem}
        
\subsection{Coordinate transforms}
        
        \newdef{Transition matrix}{\label{linalgebra:transition_matrix}
        	Let $\mathcal{B} = \{b_1, ..., b_n\}$ and $\mathcal{B}' = \{b_1', ..., b_n'\}$ be two bases of $V$. Every element of $\mathcal{B}'$ can be written as a linear combination of elements in $\mathcal{B}$:
        	\begin{equation}
        		b_j' = q_{1j}b_1 + ... + q_{nj}b_n
        	\end{equation}
	        The matrix $Q = (q_{ij})\in M_n(K)$ is called the transition matrix from the 'old' basis $\mathcal{B}$ to the 'new' basis $\mathcal{B}'$.}
        
        \begin{theorem}\label{linalgebra:theorem:transition_matrix}
		Let $\mathcal{B}, \mathcal{B}'$ be two basis of $V$. Let $Q$ be the transition matrix from $\mathcal{B}$ to $\mathcal{B}'$. We find the following statements:
	        \begin{enumerate}
			\item Let $\mathcal{C}$ be an arbitrary basis of $V$ with $\gamma$ the corresponding coordinate isomorphism. Define the following matrices:
        	        	\[
        	        		B=(\gamma(b_1), ..., \gamma(b_n))\quad\text{and}\quad B'=(\gamma(b_1'), ..., \gamma(b_n'))
        	        	\]
		                Then $BQ = B'$.
			\item $Q\in\text{GL}_n(K)$ and $Q^{-1}$ is the transition matrix from $\mathcal{B}'$ to $\mathcal{B}$.
                	\item Let $v\in V$ with $(\lambda_1, ..., \lambda_n)^T$ the coordinate vector with respect to $\mathcal{B}$ and $(\lambda_1', ..., \lambda_n')^T$ the coordinate vector with respect to $\mathcal{B}'$. Then:
                	\[
		                Q\left(
				\begin{array}{c}
					\lambda_1'\\
                			\vdots\\
			        	\lambda_n'
				\end{array}
                        	\right)
                        	=
		                \left(
                		\begin{array}{c}
					\lambda_1\\
                			\vdots\\
			        	\lambda_n
				\end{array}
                        	\right)
				\quad\text{and}\quad
                        	\left(
                        	\begin{array}{c}
					\lambda_1'\\
                        		\vdots\\
			        	\lambda_n'
				\end{array}
		                \right)
                	        =
                	        Q^{-1}\left(
                	        \begin{array}{c}
					\lambda_1\\
                		        \vdots\\
			                \lambda_n
				\end{array}
                        	\right)
			\]
		\end{enumerate}
	\end{theorem}
        
	\begin{theorem}\label{linalgebra:theorem:transition_matrix_representation}
        	Let $V,W$ be two finite-dimensional K-vector spaces. Let $\mathcal{B}, \mathcal{B}'$ be two bases of $V$ and $\mathcal{C}, \mathcal{C}'$ two bases of $W$. Let $Q, P$ be the transition matrices from $\mathcal{B}$ to $\mathcal{B}'$ and from $\mathcal{C}$ to $\mathcal{C}'$ respectively. Let $A=A_{f, \mathcal{B}, \mathcal{C}}$ and $A' = A_{f, \mathcal{B}', \mathcal{C}'}$. Then:
	        \begin{equation}
        	    	A' = P^{-1}AQ
        	\end{equation}
	\end{theorem}
        \begin{result}
		Let $f\in \text{End}_K(V)$ and let $Q$ be the transition matrix. From theorem \ref{linalgebra:theorem:transition_matrix_representation} it follows that:
            	\begin{equation}
	            	A'=Q^{-1}AQ
        	\end{equation}
	\end{result}

        \newdef{Matrix conjugation}{\index{conjugacy class}\index{matrix!conjugation}\label{linalgebra:conjugacy_class}
        	Let $A\in M_n(K)$. The set
        	\begin{equation}
	            	\{Q^{-1}AQ\ |\ Q\in\text{GL}_n(K)\}
        	\end{equation}
	        is called the conjugacy class\footnotemark\ of $A$. Another name for conjugation is \textbf{similarity transformation}.
        	\footnotetext{This is the general definition of conjugacy classes for groups. Furthermore, these classes induce a partitioning of the group.}
	}
        \begin{remark}
        	If $A$ is a matrix representation of a linear operator $f$, then the conjugacy class of $A$ consists out of every possible matrix representation of $f$.
        \end{remark}
        
        \begin{property}\index{trace}
        	From property \ref{linalgebra:trace_commutative} it follows that the trace of a matrix is invariant under similarity transformations:
        	\begin{equation}
            		\label{linalgebra:trace_invariance}
            		\boxed{\text{tr}(Q^{-1}AQ) = \text{tr}(A)}
	        \end{equation}
        \end{property}
        
        \newdef{Matrix congruence}{\index{matrix!congruence}\label{linalgebra:matrix_congruence}
        	Let $A, B\in M_n(K)$. If there exists a matrix $P$ such that
	        \begin{equation}
        	    	A = P^TBP
        	\end{equation}
	        then the matrices are said to be congruent.
        }
        \begin{property}
        	Every matrix congruent to a symmetric matrix is also symmetric.
        \end{property}
        
        \begin{theorem}\label{linalgebra:theorem:orthogonal_transition_matrix}
        	Let $(V, \langle .|. \rangle)$ be an inner-product space defined over $\mathbb{R}$ (or $\mathbb{C}$). Let $\mathcal{B}, \mathcal{B}'$ be two orthonormal bases of $V$ and let $Q$ be the transition matrix. Then $Q$ is orthogonal:
        	\begin{equation}
	        	Q^TQ = \mathbbm{1}_n
	        \end{equation}
	\end{theorem}

\subsection{Determinant}

    	\newdef{Minor}{
        	The $(i, j)$-th minor of $A$ is defined as:\[\det(A_{ij})\] where $A_{ij}\in M_{n-1}(K)$ is the matrix obtained by removing the $i$-th row and the $j$-th column from $A$.
	}
        \newdef{Cofactor}{
        	The cofactor $\alpha_{ij}$ of the matrix element $a_{ij}$ is equal to:\[(-1)^{i+j}\det(A_{ij})\]where $\det(A_{ij})$ is the minor as previously defined.
	}
        \newdef{Adjugate matrix}{\label{linalgebra:adjugate_matrix}
	        The adjugate matrix of $A\in M_n(K)$ is defined as follows:
	       	\begin{equation}
	            	\text{adj}(A) := \left(
	                \begin{array}{cccc}
				\alpha_{11}&\alpha_{21}&\dotsm&\alpha_{n1}\\
		                \alpha_{12}&\alpha_{22}&\dotsm&\alpha_{n2}\\
		                \vdots&\vdots&\vdots&\vdots\\
		                \alpha_{1n}&\alpha_{2n}&\dotsm&\alpha_{nn}\\
			\end{array}
        	        \right)
		\end{equation}
        	or shorter: $\text{adj}(A) = (\alpha_{ij})^T$.
        }
        \begin{remark*}
		It is important to notice that we have to transpose the matrix after the elements have been replaced by their cofactor.
	\end{remark*}
        
        \begin{property}\label{linalgebra:determinant_properties}
	        Let $A,B\in M_n(K)$. Denote the columns of $A$ as $A_1, \dotso, A_n$. We have the following properties of the determinant:
	        \begin{enumerate}
			\item $\det(A^T) = \det(A)$
	                \item $\det(AB) = \det(BA) = \det(A)\det(B)$
	                \item $\det(A_1, \dotso, A_i+\lambda A_i', \dotso, A_n) = \det(A_1, \dotso, A_i, \dotso, A_n) + \lambda\det(A_1, \dotso,A_i', \dotso, A_n)$ for all $A_i,A_i'\in M_{n,1}(K)$.
	                \item If two columns of $A$ are equal then $\det(A) = 0$.
	                \item $\det(A_{\sigma(1)},\dotso,A_{\sigma(n)}) = \text{sgn}(\sigma)\det(A_1,\dotso,A_n)$
	                \item The determinant can be evaluated as follows:
	                	\begin{equation}
					\det(A) = \sum_{i=1}^n(-1)^{i+k}a_{ik}\det(A_{ik})
				\end{equation}
		\end{enumerate}
		\end{property}
        
	\begin{theorem}\label{linalgebra:theorem:rank_det_equivalence}
        	Let $A\in M_n(K)$, the following statements are equivalent:
        	\begin{enumerate}
			\item $\det(A) \neq 0$
        	        \item $\text{rk}(A) = n$
        	        \item $A\in\text{GL}_n(K)$
		\end{enumerate}
	\end{theorem}
        \begin{theorem}\label{linalgebra:theorem:adjugate_matrix}
            For all $A\in M_n(K)$ we find $A\text{adj}(A) = \text{adj}(A)A = \det(A)I_n$.
	\end{theorem}
        \begin{formula}\label{linalgebra:theorem:determinant_inverse}
	        For all $A\in\text{GL}_n(K)$ we find:
		\begin{equation}
            		A^{-1} = \det(A)^{-1}\ \text{adj}(A)
            	\end{equation}
	\end{formula}
        
        An alternative definition of a $k\times k$-minor is: 
        \begin{definition}
		Let $A\in M_{m,n}(K)$ and $k\leq\min(m, n)$. A $k\times k$-minor of $A$ is the determinant of a $k\times k$-partial matrix obtained by removing $m-k$ rows and $n-k$ columns from $A$.
	\end{definition}
        \begin{theorem}
		Let $A\in M_{m,n}(K)$ and $k\leq\min(m, n)$. We find that $\text{rk}(A)\geq k$ if and only if $A$ contains a non-zero $k\times k$-minor.
	\end{theorem}
        
        \begin{theorem}
		Let $f\in\textup{End}_K(V)$. The determinant of the matrix representation of $f$ is invariant under basis transformations.
	\end{theorem}
	
        \newdef{Determinant of a linear operator}{\index{determinant}\label{linalgebra:operator_determinant}
        	The previous theorem allows us to unambiguously define the determinant of $f\in\textup{End}_K(V)$ as follows:
        	\[\det(f) := \det(A)\]
		where $A$ is some matrix representation of $f$.
        }

\subsection{Characteristic polynomial}

    	\begin{definition}[Characteristic polynomial\footnotemark]\index{characteristic!polynomial}\label{linalgebra:characteristic_polynomial}
		Let $V$ be a finite-dimensional K-vector space. Let $f\in \text{End}_K(V)$ be a linear operator with the matrix representation $A$ (with respect to some arbitrary basis). We then find:
		\begin{equation}
                	\chi_f(x) := \det(x\mathbbm{1}_n - A) \in K[x]
		\end{equation}
		is a monic polynomial of degree $n$ in the variable $x$ and the polynomial does not depend on the choice of basis.
		\footnotetext{This polynomial can also be used directly for a matrix $A$ as theorem \ref{linalgebra:theorem:map_matrix_link} matches every matrix $A$ with some linear operator $f$.}
	\end{definition}
        
        \begin{definition}[Characteristic equation\footnotemark]\index{characteristic!equation}
        	\footnotetext{This equation is sometimes called the \textbf{secular equation}.}
		The following equation is called the characteristic equation of $f$:
	        \begin{equation}
            		\label{linalgebra:characteristic_equation}
			\boxed{\chi_f(x) = 0}
		\end{equation}
	\end{definition}
        
        \begin{formula}\label{linalgebra:parts_of_characteristic_polynomial}
        	Let $A=(a_{ij})\in M_n(K)$ with characteristic polynomial: \[\chi_A(x) = x^n + c_{n-1}x^{n-1} + \dotso + c_1x + c_0\] We then have the following result:
        	\begin{equation}
			\begin{cases}
				c_0 = (-1)^n\det(A)\\
				c_{n-1} = -\text{tr}(A)
			\end{cases}
		\end{equation}
	\end{formula}
        
        \begin{theorem}[Cayley-Hamilton]\index{Cayley-Hamilton theorem}\label{linalgebra:cayley_hamilton}\
	        \begin{enumerate}
			\item Let $A\in\text{M}_n(K)$ with characteristic polynomial $\chi_A(x)$. We find the following relation:
		                \begin{equation}
					\chi_A(A) = A^n + \sum_{i=1}^{n-1}c_iA^i= 0
				\end{equation}
	                \item Let $f\in\textup{End}_K(V)$ with characteristic polynomial $\chi_f(x)$. We find that
		                \begin{equation}
					\chi_f(f) = f^n + \sum_{i=1}^{n-1}c_if^i= 0
				\end{equation}
		\end{enumerate}
	\end{theorem}
        \begin{result}
		From theorem \ref{linalgebra:minimal_polynomial_divisor} and the Cayley-Hamilton theorem it follows that the minimal polynomial $\mu_f(x)$ is a divisor of the characteristic polynomial $\chi_f(x)$.
	\end{result}
        
\subsection{Linear groups}\label{linalgebra:section:linear_groups}
        
        \newdef{Elementary matrix}{\label{linalgebra:elementary_matrix}An elementary matrix is a matrix of the following form:
        	\[\left(
                \begin{array}{cccc}
			1&0&\dotsm&0\\
        	        0&1&c_{ij}&0\\
                	\vdots&\vdots&\ddots&\vdots\\
	                0&0&\vdots&1
		\end{array}
		\right),
        	\left(
                \begin{array}{cccc}
			1&0&\dotsm&0\\
                	0&1&\dotsm&0\\
	                \vdots&c_{ij}&\ddots&\vdots\\
        	        0&0&\vdots&1
		\end{array}
		\right),\dotso\]
		i.e. equal to the sum of an identity matrix and a multiple of a matrix unit $U_{ij}, i\neq j$.
        }
        \newnot{Elementary matrix}{$E_{ij}(c)$ is the elementary matrix with element $c$ on the $i,j$-th position.}
        \begin{property}
        	We have the following property:
		\begin{equation}
			\det(E_{ij}(c)) = 1
		\end{equation}
		which implies that $E_{ij}(c)\in\text{GL}_n(K)$.
	\end{property}
        \begin{property}
		We find the following results concerning the multiplication by an elementary matrix:
	        \begin{enumerate}
			\item Left multiplication by an elementary matrix $E_{ij}(c)$ comes down to replacing the $i$-th row of the matrix with the $i$-th row plus $c$ times the $j$-th row.
                	\item Right multiplication by an elementary matrix $E_{ij}(c)$ comes down to replacing the $j$-th column of the matrix with the $j$-th column plus $c$ times the $i$-th column.
		\end{enumerate}
		\end{property}
        
        \begin{theorem}\label{linalgebra:theorem:elementary_matrices}
		Every matrix $A\in\text{GL}_n(K)$ can be written in the following way: \[A = SD\] where $S$ is a product of elementary matrices and $D=\text{diag}(1,\dotso,1,\det(A))$.
	\end{theorem}
        
        \newdef{Special linear group}{\label{linalgebra:special_linear_group}
        	The following subset of GL$_n(K)$ is called the special linear group:
		\begin{equation}
			\text{SL}_n(K) = \{A\in \text{GL}_n(K)\ |\ \det(A) = 1\}
		\end{equation}
        }
        \begin{theorem}
		Every $A\in\text{SL}_n(K)$ can be written as a product of elementary matrices.\footnote{This follows readily from theorem \ref{linalgebra:theorem:elementary_matrices}.}
	\end{theorem}
        
        \newdef{Orthogonal group}{\label{linalgebra:orthogonal_group}The orthogonal and special orthogonal group are defined as follows:
        	\begin{align}
			\text{O}_n(K) &= \{A\in \text{GL}_n(K)\ |\ AA^T = A^TA = I_n\}\nonumber\\
			\text{SO}_n(K) &= \text{O}_n(K)\cap \text{SL}_n(K)\nonumber
		\end{align}
        }
        \begin{property}\label{linalgebra:con_equivalence}
        	For orthogonal matrices, conjugacy \ref{linalgebra:conjugacy_class} and congruency \ref{linalgebra:matrix_congruence} are equivalent.
        \end{property}
        
        \newdef{Unitary group}{\label{linalgebra:unitary_group}\index{involution}
        	The unitary and special unitary group are defined as follows:
        	\begin{align}
			\text{U}_n(K, \sigma) &= \{A\in \text{GL}_n(K)\ |\ A\overline{A}^T = \overline{A}^TA = I_n\}\nonumber\\
			\text{SU}_n(K, \sigma) &= \text{U}_n(K)\cap \text{SL}_n(K)\nonumber
		\end{align}
		where $\sigma$ denotes the \textit{involution}\footnotemark\ $a^\sigma \equiv \overline{a}$.
		\footnotetext{An involution is an operator that is its own inverse: $f(f(x)) = x$.}
        }

	\begin{remark*}
		If $K=\mathbb{C}$ where the involution is taken to be the complex conjugate, the $\sigma$ is often ommited in the definition: U$_n(K)$ and SU$_n(K)$.
	\end{remark*}
	
	\newdef{Unitary equivalence}{
		Let $A, B$ be two matrices in M$_n(K)$. If there is a unitary matrix $U$ such that \[A = U^\dag BU\] then the matrices $A$ and $B$ are said to be \textbf{unitarily equivalent}.
	}
	
\subsection{Matrix decomposition}

	\begin{method}[QR Decomposition]\index{QR!decompositon}
		Every square complex matrix $M$ can be decomposed as:
		\begin{equation}
			M = QR
		\end{equation}
		where $Q$ is unitary and $R$ is upper-triangular. The easiest (but not the most numerically stable) way to do this is by applying the Gram-Schmidt orthonormalisation process:
		
		Let $\{v_i\}_{i\leq n}$ be a basis for the column space of $M$. By applying the Gram-Schmidt process to this basis one obtains a new orthonormal basis $\{e_i\}_{i\leq n}$. The matrix $M$ can then be written as $QR$ where:
		\begin{itemize}
			\item $R$ is an upper-triangular matrix with entries $R_{ij} = \langle e_i|\text{col}_j(M) \rangle$ where col$_j(M)$ denotes the $j^{th}$ column of $M$.
			\item $Q = (a_1\cdots a_n)$ is the unitary matrix constructed by setting the $i^{th}$ column equal to the $i^{th}$ basis vector $a_i$ 
		\end{itemize}
	\end{method}
	\begin{property}
		If $M$ is invertible and if the diagonal elements of $R$ are required to have positive norm then the QR-decomposition is unique.
	\end{property}

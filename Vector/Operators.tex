\chapter{Operator Algebras}

\section{Operators on Banach \& Hilbert spaces}
\subsection{Operator topologies}\index{operator topology}\index{topology!operator|see{operator}}

	In this subsection all spaces will be assumed to be Banach unless stated otherwise.

	\newdef{Weak topology}{
		A sequence of operators $\seq{T}$ on a space $X$ converges to an operator $T$ in the weak (Banach) topology if the sequence $(T_nx)_{n\in\mathbb{N}}$ converges to $Tx$ weakly\footnote{This is with respect to the weak topology \ref{hilbert:weak_topology}.} for all $x$.
	}
	
	\newdef{Strong operator topology}{
		A sequence of operators $\seq{T}$ on a space $X$ converges to an operator $T$ in the strong (operator) topology if the sequence $(T_nx)_{n\in\mathbb{N}}$ converges to $Tx$ for all $x$. Equivalently it is the topology generated by the seminorms $T\rightarrow ||Tx||, \forall x\in X$. 
	}

	\newdef{Operator norm}{\index{operator!norm}
    		The operator norm of $L$ is defined as follows:
    		\begin{gather}
    			||L||_{op} = \inf\{M\in\mathbb{C}|\forall v\in V:||Lv||_W \leq M||v||_V\}
    		\end{gather}
    		Equivalent definitions of the operator norm are:
    		\begin{gather}
    			||L||_{op} = \sup_{||x||\leq1}||L(x)|| = \sup_{||x||=1}||L(x)|| = \sup_{x\neq0}\stylefrac{||L(x)||}{||x||}
    		\end{gather}
	}
	
	\newdef{Norm topology\footnotemark}{
		\footnotetext{Also called the \textbf{uniform (operator) topology}.}
		A sequence of operators $\seq{T}$ on a space $X$ converges to an operator $T$ in the norm topology if the sequence $(||T_n-T||)_{n\in\mathbb{N}}$ converges to 0.
	}

\subsection{Bounded operators}
	
	\newdef{Bounded operator}{\index{bounded!operator}
		Let $L:V\rightarrow W$ be a linear operator between two Banach spaces. The operator is said to be bounded if it satisfies the following condition:
		\begin{gather}
			\label{operator:bounded_operator}
			||L||_{op}<\infty
		\end{gather}
	}
	\begin{notation}
		\nomenclature[S_BVW]{$\mathcal{B}(V, W)$}{Space of bounded linear maps from the space $X$ to the space $Y$.}
		The space of bounded linear operators from $V$ to $W$ is denoted by $\mathcal{B}(V, W)$.
	\end{notation}
	\begin{property}
		If $V$ is a Banach space then $\mathcal{B}(V)$ is also a Banach space.
	\end{property}
	
	Following property reduces the problem of continuity to that of boundedness:
	\begin{property}
		Let $f\in\mathcal{L}(V, W)$. Following statements are equivalent:
		\begin{itemize}
			\item $f$ is bounded.
			\item $f$ is continuous at 0.
			\item $f$ is continuous on $V$.
			\item $f$ is uniformly continuous.
			\item $f$ maps bounded sets to bounded sets.
		\end{itemize}
	\end{property}

	\begin{property}
		Let $A$ be a bounded linear operator with eigenvalue $\lambda$. We then have:
		\begin{gather}
			|\lambda|\leq||A||_{op}
		\end{gather}
	\end{property}
	\begin{property}
		Let $A$ be a bounded linear operator. Let $A^\dag$ denote its adjoint\footnotemark. Then $A^\dag$ is bounded and $||A||_{op} = ||A^\dag||_{op}$.
		\footnotetext{See definition \ref{linalgebra:adjoint_operator}.}
	\end{property}

	\newdef{Schatten class operator}{\index{Schatten class}
		The \textbf{Schatten p-norm} is defined as:
		\begin{gather}
			||T||_p = \text{tr}\left(\sqrt{T^\dag T}^{\ p}\right)^{1/p}
		\end{gather}
		Operators for which this norm is finite are elements of the $p^{th}$ Schatten class $\mathcal{I}_p$.
	}
	\begin{property}
		The Schatten classes are Banach spaces with respect to the associated Schatten norms.
	\end{property}

	We now consider the two most prominent Schatten classes:
	\newdef{Trace class operator}{\index{trace}
		The space of trace class operators on a Hilbert space $\mathcal{H}$ is defined as:
		\begin{gather}
			\mathcal{B}_1(\mathcal{H}) = \{S\in\mathcal{B}(\mathcal{H}):\text{tr}(|S|)<\infty\}
		\end{gather}
	}
	
	The following theorem can be seen as the analogon of Riesz' theorem for trace class operators:
	\begin{theorem}
		For every bounded linear functional $\rho$ on the space of trace class operators $\mathcal{B}_1(\mathcal{H})$ there exists a unique bounded linear operator $T\in\mathcal{B}(\mathcal{H})$ such that:
		\begin{gather}
			\rho(S) = \text{tr}(ST)
		\end{gather}
		for all $S\in\mathcal{B}_1(\mathcal{H})$. Hence $\mathcal{B}_1(\mathcal{H})$ and $\mathcal{B}(\mathcal{H})$ are isometrically equivalent.
	\end{theorem}
	
	\newdef{Hilbert-Schmidt operator}{
		Consider the Hilbert-Schmidt norm $||\cdot||_2$ from equation \ref{linalgebra:hilbert_schmidt_norm}. An operator $T\in\mathcal{B}(\mathcal{H})$ is said to be a Hilbert-Schmidt operator if it satisfies:
		\begin{gather}
			||T||_2<\infty
		\end{gather}
		This space is closed under taking adjoints.
	}
	
	A more general, but still well-behaved class of operators, are the closed operators:
	\newdef{Closed operator}{\index{closed!operator}
		A linear operator $f:X\rightarrow Y$ is said to be closed if for every sequence $\seq{x}$ in $\dom(f)$ converging to a point $x\in X$ such that $f(x_n)$ converges to a point $y\in Y$ one finds that $x\in\dom(f)$ and $f(x)=y$.
		
		Equivalently one can define a closed linear operator as a linear operator for which its graph is a closed subset in the direct sum $X\oplus Y$.
	}
	\newdef{Closure}{\index{closure}\label{operator:closure}
		Let $f$ be a linear operator. Its closure (if it exists) is the closed linear operator $\overline{f}$ such that the graph of $\overline{f}$ is the closure of the graph of $f$ in $X\oplus Y$.
	}

\subsection{Compact operators}

	\newdef{Compact operator}{\index{compact!operator}\label{banach:compact_operator}
		Let $V, W$ be Banach spaces. A linear operator $A:V\rightarrow W$ is compact if the image of any bounded set in $V$ is relatively compact\footnote{See definition \ref{topology:relatively_compact}.}.
	}

	\newadef{Compact operator}{
		Let $V, W$ be Banach spaces. A linear operator $A:V\rightarrow W$ is compact if for every bounded sequence $(x_n)$ in $V$ the sequence $(A[x_n])\subset W$ has a convergent subsequence.
	}
	
	\begin{notation}
		\nomenclature[S_boundcompact]{$\mathcal{B}_0(X, Y)$}{Space of compact bounded oeprators between Banach spaces.}
		The space of compact bounded linear operators between Banach spaces $X, Y$ is denoted by $\mathcal{B}_0(X, Y)$. If $X=Y$ then this is abbreviated to $\mathcal{B}_0(X)$ as usual.
	\end{notation}
	\begin{property}
		$\mathcal{B}_0(X)$ is a two-sided ideal in the (Banach) algebra $\mathcal{B}(X)$.
	\end{property}

\subsection{Fredholm operators}

	\begin{property}
		Every compact operator is bounded and hence continuous.
	\end{property}
	\begin{result}
		Every linear map between finite-dimensional Banach spaces is bounded.
	\end{result}
	
	\newdef{Calkin algebra}{\index{Calkin}
		Consider the algebra $B(V)$ of bounded linear operators on $V$ together with its two-sided ideal $K(V)$ of compact operators. The quotient algebra $\mathcal{Q}(V) = B(V)/K(V)$ is called the Calkin algebra of $V$.
	}

	\newdef{Fredholm operator}{\index{Fredholm!operator}
		Let $V, W$ be Banach spaces. A Fredholm operator $F:V\rightarrow W$ is a bounded linear operator $F\in\mathcal{B}(V, W)$ for which the kernel and cokernel are finite-dimensional.
	}
	
	By a theorem of Atkinson one can characterize Fredholm operators using the Calkin algebra:
	\begin{property}[Atkinson]
		An operator $F:V\rightarrow W$ is a Fredholm operator if and only if it is invertible when projected onto the Calking algebra, i.e. there exists a bounded linear operator $G:W\rightarrow V$ and compact operators $C_1, C_2$ such that $\mathbbm{1}_V - FG = C_1$ and $\mathbbm{1}_W - GF = C_2$.
	\end{property}

\subsection{Spectrum}

	\newdef{Resolvent operator}{\index{resolvent}
		Let $A$ be a bounded linear operator on a normed space $V$. The resolvent operator of $A$ is defined as the operator $(A - \lambda\mathbbm{1}_V)^{-1}$,	where $\lambda\in\mathbb{C}$.
	}

	\newdef{Resolvent set}{
		\nomenclature[S_zsymrho]{$\rho(A)$}{Resolvent set of a bounded linear operator $A$.}
		The resolvent set $\rho(A)$ consists of all scalars $\lambda\in\mathbb{C}$ for which the resolvent operator of A is a bounded linear operator on a dense subset of $V$. These scalars $\lambda$ are called \textbf{regular values} of $A$.
	}
	\newdef{Spectrum}{\index{spectrum}
		The set of scalars $\mu\in\mathbb{C}\setminus \rho(A)$ is called the spectrum $\sigma(A)$.
	}
	
	\begin{remark}
		It is obvious from the definition of an eigenvalue that every eigenvalue of $A$ belongs to the spectrum of $A$. The converse however is not true.
	\end{remark}
	
	\newdef{Point spectrum}{
		The set of scalars $\mu\in\mathbb{C}$ for which $A - \mu\mathbbm{1}_V$ fails to be injective is called the point spectrum $\sigma_p(A)$. This set coincides with the set of eigenvalues of $A$.
	}
	\newdef{Continuous spectrum}{
		The set of scalars $\mu\in\mathbb{C}$ for which $A - \mu\mathbbm{1}_V$ is injective, fails to be surjective and is dense in $V$ is called the continuous spectrum of $A$.
	}
	\newdef{Residual spectrum}{
		The set of scalars $\mu\in\mathbb{C}$ for which $A - \mu\mathbbm{1}_V$ is injective but fails to have a dense range in $V$ is called the residual spectrum $\sigma(A)$. It follows that $\sigma_r(A)\subseteq\sigma(A)$.
	}
	
	\newdef{Essential spectrum}{\index{essential!spectrum}
		The set of scalars $\mu\in\mathbb{C}$ for which $A-\mu\mathbbm{1}_V$ is not a Fredholm operator is called the essential spectrum $\sigma_{\text{ess}}(A)$.
	}
	From Atkinson's theorem\footnote{In fact one could (equivalently) define the essential spectrum in terms of the Calkin algebra using Atkinson's theorem. Then this property would be an obvious consequence.} one can derive the following result:
	\begin{property}
		Let $A$ be a bounded linear operator and let $T$ be a compact operator. The essential spectra of $A$ and $A+T$ coincide.
	\end{property}


\section{Involutive algebras}

	\newdef{Involution}{\index{involution}
		Let $\ast$ be an automorphism of an algebra $A$. If $(a^*)^* = a$ for all $a\in A$ then $\ast$ is called an involution of $A$.
	}
	
	\newdef{Involutive algebra\footnotemark}{\index{involutive algebra}\index{$*$-algebra}
		\footnotetext{Also called a $^*$\textbf{-algebra}.}
		An involutive algebra is an associative algebra $A$ over a commutative involutive ring $(R, \overline{\phantom{z}}\ )$ together with an operator $^*:A\rightarrow A$ such that:
		\begin{itemize}
			\item $(a + b)^* = a^* + b^*$
			\item $(ab)^* = b^*a^*$
			\item $(\lambda a)^* = \overline\lambda a^*$
		\end{itemize}
		where $\lambda\in R$.
	}
	
\section{\texorpdfstring{$C^*$-}{C-star }algebras}

	\newdef{$C^*$-algebra}{\index{C$^*$-algebra}
		A $C^*$-algebra is an involutive Banach algebra\footnote{See definition \ref{linalgebra:banach_space}.} such that the \textbf{$C^*$-identity}
		\begin{gather}
			||a^*a|| = ||a||\ ||a^*||
		\end{gather}
		is satisfied.
	}
	
\subsection{Self-adjoint and positive elements}

	\newdef{Positive element}{\index{positive}
		An element of a $C^*$-algebra is called positive if its spectrum is contained in $[0, +\infty[$.
	}
	\begin{property}
		Every positive element $a$ can be written as $a=b^*b$ for some element $b$. Hence every positive-element is self-adjoint. 
	\end{property}
	
	\newdef{Cuntz algebra}{\index{Cuntz}
		The $n^{th}$ Cuntz algebra $\mathcal{O}_n$ is defined as the (universal) unital $C^*$-algebra generated by $n$ isometric elements $s_i$ under the additional relation
		\begin{gather}
			\sum_{i=1}^ns_i^*s_i = 1
		\end{gather}
		where 1 is the unit element.
	}
	
\subsection{Positive maps}

	\newdef{Positive map}{
		A morphism of $C^*$-algebras is called positive if every positive element is mapped to a positive element.
	}
	\newdef{Completely positive map}{\index{positive!completely}\label{operator:cp_map}
		A morphism of $C^*$-algebras $T:A\rightarrow B$ is called completely positive if for all $k\in\mathbb{N}$ the following map is positive:
		\begin{gather}
			\mathbbm{1}_k\otimes T:\mathbb{C}^{k\times k}\otimes A\rightarrow \mathbb{C}^{k\times k}\otimes B
		\end{gather}
		If $T$ satisfies this condition only up to an integer $n$ then it is called \textbf{$n$-positive}.
	}

	\newdef{State}{\index{state}
		Let $A$ be a $C^*$-algebra. A state $\psi$ on $A$ is a positive linear functional of unit norm.
	}
	
	\newdef{Adjoint map}{\index{adjoint}
		Consider a continuous linear map $\phi$ defined on the Schatten class $\mathcal{I}_p$. Given a trace functional $\text{tr}$ on the $C^*$-algebra one can define the adjoint map $\phi^*$ defined on $\mathcal{I}_q$, where $p, q$ are H\"older conjugate. This adjoint is given by the following equation:
		\begin{gather}
			\text{tr}\Big((\phi^*(A))^*B\Big) = \text{tr}\Big(A^*\phi(B)\Big)
		\end{gather}
		where $A\in\mathcal{I}_q, B\in\mathcal{I}_p$.
	}
	\newdef{Trace preserving map}{
		A map $\phi$ is said to be trace preserving if it satisfies:
		\begin{gather}
			\text{tr}(\phi(A)) = \text{tr}(A)
		\end{gather}
		for all trace class elements $A$. Using the above definition it is easily seen that on a unital $C^*$-algebra this is equivalent to:
		\begin{gather}
			\phi^*(1) = 1
		\end{gather}
	}
	
	\begin{property}
		A completely positive, trace preserving map $\phi$ satisfies:
		\begin{gather}
			||\phi||_1 = 1
		\end{gather}
		where the subscript $1$ denotes the fact that this operator is defined on trace class elements.
	\end{property}
	
	\newdef{Positivity improving}{
		A positive map $\phi$ is said to be positivity improving if it satisfies:
		\begin{gather}
			A\geq0, A\neq0\implies \phi(A)>0
		\end{gather} 
	}
	\newdef{Ergodic map}{\index{ergodic}
		A positive map $\phi$ is said to be ergodic if it satisfies:
		\begin{gather}
			\forall A\geq0, A\neq0:\exists t_A>0:\exp(t_A\phi)A>0 
		\end{gather}
	}

\subsection{Representations}
	
	\newdef{C$^*$-algebra representation}{
		A representation of a C$^*$-algebra $\mathfrak{U}$ is a unital $\ast$-morphism $\mathfrak{U}\rightarrow\mathcal{B}(\mathcal{H})$.
	}
	\newdef{Cyclic vector}{\index{cyclic!vector}
		A cyclic vector for a $C^*$-algebra representation $\rho:\mathcal{C}\rightarrow\mathcal{B}(\mathcal{H})$ is a vector $\xi\in\mathcal{H}$ such that $\{\rho(c)\xi:c\in\mathcal{C}\}$ is norm dense in $\mathcal{H}$.
	}
	
	\begin{construct}[GNS\footnotemark\ construction]\index{GNS}
		\footnotetext{Gel'fand-Naimark-Segal}
		Let $\mathcal{C}$ be a $C^*$-algebra. Given a state $\omega$ on $\mathcal{C}$ there exists a $C^*$-representation $\rho:\mathcal{C}\rightarrow\mathcal{B}(D)$ where $D\subset\mathcal{H}$ is a dense subspace of a Hilbert space $\mathcal{H}$ such that the following conditions are satisfied:
		\begin{itemize}
			\item There exists a distinguished cyclic unit vector $\xi$ such that $D = \{\rho(c)\xi: c\in\mathcal{C}\}$
			\item For all elements $c\in\mathcal{C}$ the following equality holds:
			\begin{gather}
				\omega(c) = \langle \rho(c)\xi,\xi \rangle
			\end{gather}
		\end{itemize}
	\end{construct}
	
	\begin{theorem}[Gel'fand-Naimark]
		Every C$^*$-algebra is isometrically $\ast$-isomorphic to a norm closed ($C^*$-)algebra of bounded operators on a Hilbert space $\mathcal{H}$.
	\end{theorem}

\subsection{Classification}

	\begin{theorem}
		Let $\mathcal{C}$ be a finite-dimensional C$^*$-algebra. Then there exist unique integers $N, d_1, ..., d_N$ such that:
		\begin{gather}
			\mathcal{C}\cong\bigoplus_{i=1}^NM_{d_i}(K)
		\end{gather}
		This implies that every C$^*$-algebra can be represented using block matrices.
	\end{theorem}
	
\subsection{Gel'fand duality}\index{Gel'fand!duality}

	\newdef{Gel'fand spectrum}{
		Consider a unital $C^*$-algebra $A$. Its set of characters, i.e. algebra morphisms $A\rightarrow\mathbb{C}$, can be equipped with a compact\footnote{Locally compact if the algebra is non-unital.} Hausdorff topology (the weak-$\ast$ topology).
	}
	\newdef{Gel'fand representation}{
		Consider a $C^*$-algebra $A$ and let $\Phi_A$ be its Gel'fand spectrum. The Gel'fand transformation of an element $a\in A$ is defined as the morphism $\hat{a}:\Phi_A\rightarrow\mathbb{C}$ given by the following formula:
		\begin{gather}
			\hat{a}(\lambda) = \langle\lambda,a\rangle
		\end{gather}
		where $\langle\cdot,\cdot\rangle$ denotes the pairing between $A$ and $\Phi_A$. By definition of the topology on the Gel'fand spectrum the functional $\hat{a}$ is continuous for all $a\in A$. The mapping $a\mapsto\hat{a}$ is called the Gel'fand representation of $A$.
	}
	
	\begin{theorem}[Gel'fand-Naimark\footnotemark]
		\footnotetext{The restricted version.}
		Let $A$ be a commutative $C^*$-algebra. The Gel'fand representation gives an isometric $\ast$-isomorphism between $A$ and the set of continuous functionals $C_0(\Phi_A)$ which vanish at infinity on its Gel'fand spectrum.
	\end{theorem}
	
\section{von Neumann algebras}

	\newdef{Concrete von Neumann algebra}{\index{Von Neumann!algebra}
		A weakly closed unital $*$-algebra of bounded operators on some Hilbert space.
	}
	\newadef{von Neumann algebra}{
		A $*$-subalgebra of a $C^*$-algebra equal to its double commutant: $M'' = M$
	}
	\begin{theorem}[Double Commutant theorem\footnotemark]
		\footnotetext{Often called \textbf{von Neumann's double commutant theorem}.}
		The above definitions are equivalent.	
	\end{theorem}
	
\subsection{Factors}

	\newdef{Factor}{\index{factor}
		Consider a von Neumann algebra $M$. A $*$-subalgebra $A$ is called a factor of $M$ if its center $Z(A)$ is given by the scalar multiples of the idenity.
	}
	
	\newdef{Powers index}{\index{Powers}
		Consider a Hilbert space $\mathcal{H}$ together with its von Neumann algebra of bounded operators $\mathcal{B}(\mathcal{H})$. A unital $*$-endomorphism $\alpha$ has Powers index $n\in\mathbb{N}$ if the space $\alpha(\mathcal{B}(\mathcal{H}))$ is isomorphic to a type $I_n$ factor.
	}

\chapter{Operator Algebras}

\section{Involutive algebras}

	\newdef{Involution}{\index{involution}
		Let $\ast$ be an automorphism of an algebra $A$. If $\ast(\ast\ a) = a$ for all $a\in A$ then $\ast$ is called an involution of $A$.
	}
	
	\newdef{Involutive algebra\footnotemark}{\index{involutive algebra}\index{$*$-algebra}
		\footnotetext{Also called a $^*$\textbf{-algebra}.}
		An involutive algebra is an associative algebra $A$ over a commutative involutive ring $(R, \overline{\phantom{z}}\ )$ together with an operator $^*:A\rightarrow A$ such that:
		\begin{itemize}
			\item $(a + b)^* = a^* + b^*$
			\item $(ab)^* = b^*a^*$
			\item $(\lambda a)^* = \overline\lambda a^*$
		\end{itemize}
		where $\lambda\in R$.
	}
	
\section{\texorpdfstring{$C^*$-}{C-star }algebras}

	\newdef{C$^*$-algebra}{\index{C$^*$-algebras}
		A C$^*$-algebra is an involutive Banach algebra\footnote{See definition \ref{linalgebra:banach_space}.} $A$ such that the \textbf{C$^*$-identity}
		\begin{equation}
			||a^*a|| = ||a||\ ||a^*||
		\end{equation}
		is satisfied.
	}
	
	\newdef{Positive}{\index{positive}
		An element of a C$^*$-algebra is called positive if it is self-adjoint and if its spectrum is contained in $[0, +\infty[$. A linear functional on a C$^*$-algebra is called positive if every positive element is mapped to a positive number.
	}
	\newdef{State}{\index{state}
		Let $A$ be a C$^*$-algebra. A state $\psi$ on $A$ is a positive linear functional of unit norm.
	}
	
\subsection{Classification}

	\begin{theorem}
		Let $\mathcal{C}$ be a finite-dimensional C$^*$-algebra. Then there exist unique integers $N, d_1, ..., d_N$ such that
		\begin{equation}
			\mathcal{C}\cong\bigoplus_{i=1}^NM_{d_i}(K)
		\end{equation}
		This implies that every C$^*$-algebra can be represented using block matrices.
	\end{theorem}

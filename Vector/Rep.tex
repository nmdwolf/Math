\chapter{Representation Theory}

References for this chapter are \cite{fultonharris, jeevanjee}.

\section{Group representations}
    	\newdef{Representation}{\index{representation}
    		A representation of a group $G$, acting on a vector space $V$, is a homomorphism $\rho: G\rightarrow\text{GL}(V)$ from $G$ itself to the automorphism group\footnote{See definition \ref{linalgebra:automorphism}.}\ of $V$. This is a specific case of a group action\footnote{See definition \ref{group:group_action}.}.
    	}
    	
    	\begin{property}
    		Because every linear map maps the zero vector onto itself, a group representation can never be free\footnote{See definition \ref{group:free_action}.}.
    	\end{property}

	\newdef{Subrepresentation}{
		A subrepresentation of a representation $V$ is a subspace of $V$ invariant under the action of the group $G$.
	}
        
	\begin{example}[Permutation representation]
		Consider a vector space $V$ equipped with a basis $\{e_i\}_{i\in I}$ with $|I| = n$. Let $G = S^n$ be the symmetric group on $n$ symbols. Based on remark \ref{group:permutation_remark} we can consider the action of $G$ on the index set $I$. This representation is given by:
		\begin{equation}
			\rho(g):\sum_{i\in I}v_ie_i\mapsto\sum_{i\in I}v_ie_{g\cdot i}
		\end{equation}
	\end{example}
        
        \begin{example}
        	Consider a representation $\rho$ on $V$. There exists a natural representation on the dual space $V^*$. The homomorphism $\rho^*:G\rightarrow\text{GL}(V^*)$ is given by:
            \begin{equation}
            	\rho^*(g) = \rho^T(g^{-1}): V^*\rightarrow V^*
            \end{equation}
            where $\rho^T$ is the transpose as defined in \ref{linalgebra:transpose}. This map satifies the following defining property:
            \begin{equation}
            	\Big\langle\rho^*(g)(v^*), \rho(g)(v)\Big\rangle = \langle v^*, v\rangle
            \end{equation}
            where $\langle\cdot,\cdot\rangle$ is the natural pairing of $V$ and its dual.
        \end{example}
        
        \begin{example}\index{tensor!product}
        	A representation $\rho$ which acts on spaces $V, W$ can also be extended to the tensor product  $V\otimes W$ in the following way:
            \begin{equation}
            	g(v\otimes w) = g(v)\otimes g(w)
            \end{equation}
        \end{example}
        
        \newdef{Intertwiner}{\index{intertwiner}
        	If we look at the representations of $G$ as $G$-modules then the natural morphisms are the intertwining maps \ref{group:equivariant}.
        }

\section{Irreducible representations}

	\newdef{Irreducibility}{\index{irreducibility}
		A representation is said to be irreducible if there exist no proper non-zero subrepresentation.
	}
	
	\begin{example}[Standard representation]
		Consider the action of $\text{Sym}(n)$ on a vector space $V$. The line generated by $v_1+v_2+...+v_n$ is invariant under the permutation action of $\text{Sym}(n)$. It follows that the permutation representation (on finite-dimensional spaces) is never irreducible.
		
		The $(n-1)$-dimensional complementary subspace
		\begin{equation}
			W = \{a_1v_1 + a_2v_2 + ... + a_nv_n|a_1 + a_2 + ... + a_n = 0\}
		\end{equation}
		does form an irreducible representation when we restrict $\rho$ to $W$. It is called the standard representation of $S^n$.
	\end{example}
        
        \begin{theorem}[Schur's lemma]\index{Schur's lemma}\label{rep:schurs_lemma}
        	Let $V, W$ be two finite-dimensional irreducible representations of a group $G$. Let $\varphi: V\rightarrow W$ be an intertwiner. We then have:
            	\begin{itemize}
	        	\item $\varphi$ is an isomorphism or $\varphi = 0$
	                \item If $V = W$ then $\varphi$ is constant, i.e. $\varphi$ is a scalar multiple of the identity map $\mathbbm{1}_V$.
        	\end{itemize}
        \end{theorem}
        
        \begin{property}
        	If $W$ is a subrepresentation of $V$ then there exists an invariant complementary subspace $W'$ such that $V = W \oplus W'$.
            
            This space can be found as follows: Choose an arbitrary complement $U$ such that $V = W \oplus U$. From this we construct a projection map $\pi_0:V \rightarrow W$. Averaging over $G$ gives
            \begin{equation}
            	\pi(v) = \sum_{g\in G}g\circ\pi_0(g^{-1}v)
            \end{equation}
            which is a $G$-linear map $V\rightarrow W$. On $W$ it is given by the multiplication of $W$ by $|G|$. Its kernel is then an invariant subspace of $V$ complementary to $W$.
        \end{property}
        \begin{theorem}[Maschke]\index{Maschke's theorem}
        	Let $G$ be a finite group. A representation\footnote{The characteristic of the base field should not divide the order of $G$.} $V$ can be uniquely decomposed as
		\begin{equation}
	            	V = V_1^{\oplus a_1}\oplus\cdots\oplus V_k^{\oplus a_k}
		\end{equation}
		where all $V_k$'s are distinct irreducible representations.
        \end{theorem}
        
        The following property is often used in quantum mechanics to quickly find forbidden transitions in atomic or molecular systems:
        \begin{property}[Selection rules]
        	Let $G$ be a semisimple group. Let $W_1, W_2$ be two (finite-dimensional) inequivalent irreducible unitary subrepresentations of $\mathcal{H}$. Let $V$ be a representation equipped with an intertwiner $\rho:V\rightarrow\mathcal{L}(\mathcal{H})$. For all $v\in V, w_i\in W_i$ we have:
        	\begin{equation}
        		\langle w_1|\rho(v)w_2\rangle = 0
        	\end{equation}
        	unless $V\otimes W_2$ contains a subrepresentation equivalent to $W_1$.
        \end{property}
        
\section{Classification by Young tableaux}

	\newdef{Permutation module}{\index{permutation!module}
		Let $\lambda$ be a partition. The permutation module $M^\lambda$ is defined as the vector space generated by the Young tableaux of shape $\lambda$.
	}
	
	\newdef{Specht module}{\index{Specht}
	
	}

\chapter{Representation Theory}

    References for this chapter are \cite{fultonharris, jeevanjee}. Sections \ref{section:groups} and \ref{section:group_actions} can be visited for an introduction to groups and group actions.

\section{Group representations}

    Group actions on vector spaces are so important that they receive their own name:
    \newdef{Representation}{\index{representation}
        A representation of a group $G$ on a vector space $V$ is a group morphism $\rho:G\rightarrow\text{GL}(V)$ from $G$ to the automorphism group \ref{linalgebra:automorphism} of $V$.
    }
    \begin{property}[Never free]
        Because every linear map takes the zero vector to itself, a representation can never be free.
    \end{property}

    \newdef{Subrepresentation}{
        A subrepresentation of a representation on $V$ is a subspace of $V$ invariant under the action of the group $G$ (together with the induced action).
    }

    \begin{example}[Permutation representation]\label{rep:permutation}
        Consider a vector space $V$ with basis $\{e_i\}_{i\leq n}$ and let $G=S_n$ be the symmetric group on $n$ elements. Based on Definition \ref{group:permutation_remark}, one can consider the action of $G$ on the index set $\{1,\ldots,n\}$. This representation is given by
        \begin{gather}
            \rho(g):\sum_{i=1}^nv_ie_i\mapsto\sum_{i=1}^nv_ie_{g\cdot i}.
        \end{gather}
    \end{example}

    \begin{example}
        Consider a representation $\rho$ on $V$. There exists a natural representation on the dual space $V^*$:
        \begin{gather}
            \rho^*(g) := \rho^T(g^{-1}): V^*\rightarrow V^*,
        \end{gather}
        where $\rho^T$ is the transpose as defined in \ref{linalgebra:transpose}. It is implicitly defined by
        \begin{gather}
            \Big\langle\rho^*(g)(v^*),\rho(g)(v)\Big\rangle = \langle v^*,v \rangle,
        \end{gather}
        where $\langle\cdot,\cdot\rangle$ is the natural pairing.
    \end{example}

    \begin{example}\index{tensor!product}
        A group $\rho$ that acts on vector spaces $V,W$ also has a representation on the tensor product $V\otimes W$ in the following way:
        \begin{gather}
            \rho(g)(v\otimes w) := \rho(g)(v)\otimes\rho(g)(w).
        \end{gather}
    \end{example}

    \newdef{Intertwiner}{\index{intertwiner}
        If one views $G$-representations as $G$-modules, the natural morphisms are the intertwiners \ref{group:equivariant}.
    }

\section{Irreducible representations}\label{section:irreducibility}

    \newdef{Irreducibility}{\index{irreducible!representation}
        A representation is said to be irreducible if there exist no proper nonzero subrepresentation.
    }

    \begin{example}[Standard representation]
        Consider the action of $S_n$ on a vector space $V$ with basis $\{e_i\}_{i\leq n}$. The line generated by $e_1+e_2+\ldots+e_n$ is invariant under the permutation action of $S_n$. It follows that the permutation representation (on finite-dimensional spaces) is never irreducible.

        The $(n-1)$-dimensional complementary subspace
        \begin{gather}
            W = \left\{\sum_{i=1}^n\lambda_ie_i\,\middle\vert\,\sum_{i=1}^n\lambda_i=0\right\}
        \end{gather}
        forms an irreducible representation. It is called the standard representation of $S_n$ on $V$.
    \end{example}

    \begin{theorem}[Schur's lemma]\index{Schur's lemma}\label{rep:schurs_lemma}
        Let $V,W$ be two finite-dimensional irreducible representations of a group $G$ and let $\varphi:V\rightarrow W$ be an intertwiner.
        \begin{itemize}
            \item Either $\varphi$ is an isomorphism or $\varphi=0$.
            \item If $V=W$, then $\varphi$ is constant, i.e. $\varphi$ is a scalar multiple of the identity map $\mathbbm{1}_V$.
        \end{itemize}
    \end{theorem}

    \begin{property}[Complementary representation]
        If $W$ is a subrepresentation of $V$, there exists an invariant complementary subspace $W'$. This space can be found as follows. Choose an arbitrary complement $U$ such that $V = W \oplus U$ with associated projection map $\pi_0:V \rightarrow W$. Averaging over $G$ gives the $G$-equivariant map
        \begin{gather}
            \pi(v) := \sum_{g\in G}g\circ\pi_0(g^{-1}v).
        \end{gather}
        On $W$ it is given by multiplication by $|G|$. Its kernel is an invariant subspace of $V$ complementary to $W$.
    \end{property}
    \begin{theorem}[Maschke]\index{Maschke}
        Let $G$ be a finite group with a representation space $V$ such that the characteristic of the underlying field does not divide the order of $G$. The representation space can be uniquely decomposed as
        \begin{gather}
            V = V_1^{\oplus a_1}\oplus\cdots\oplus V_k^{\oplus a_k},
        \end{gather}
        where all $V_k$'s are distinct irreducible representations.
    \end{theorem}

\section{Classification by Young tableaux}

    \newdef{Permutation module}{\index{permutation!module}
        Let $\lambda$ be a partition. The permutation module $M^\lambda$ is defined as the vector space generated by the Young tabloids of shape $\lambda$.
    }

    \newdef{Specht module}{\index{Specht module}\index{polytabloid}
        Consider a permutation module $M^\lambda$ for some $\lambda$ with $|\lambda|=n$. Since the permutation group $S_n$ acts on Young tableaux by permuting the entries, it also has an induced action\footnote{This can be generalized to an action of the group algebra $K[S_n]$.} on $M^\lambda$. In this module one can define a submodule as the span of the following elements (called \textbf{polytabloids}):
        \begin{gather}
            e_t := \sum_{\sigma\in S_n}\sgn(\sigma)\sigma\cdot [t],
        \end{gather}
        where $t$ ranges over the Young tableau $t$ of shape $\lambda$ ($[t]$ denotes the Young tabloid associated to $t$). In fact one can just take the standard Young tableaux as generators. This is sometimes called the \textbf{Specht basis}.
    }

    \begin{property}
        A representation (over $\mathbb{C}$) of $S_n$ is irreducible if and only if it is a Specht module $S^\lambda$ for some partition $\lambda$ of $n$.
    \end{property}

    One can restate the definition of a Specht module using the following operator:
    \newdef{Young symmetrizer}{\index{Young!symmetrizer}
        Given a Young tableau $t$ of shape $\lambda$, one can decompose the permutation group $S_{|\lambda|}$ as the union of two types of permutations. First one has the permutations that preserve the rows, denote these by $P_\lambda$. Then one has also the permutations that preserve the columns, denote these by $Q_\lambda$. These two subgroups induce elements in the group algebra $\mathbb{C}[S_{|\lambda|}]$ as follows:
        \begin{align}
            a_\lambda &:= \sum_{p\in P_\lambda}p\\
            b_\lambda &:= \sum_{q\in Q_\lambda}\sgn(q)q.
        \end{align}
        The product $c_\lambda := a_\lambda b_\lambda$ is called the Young symmetrizer of $\lambda$.
    }
    \newadef{Specht module}{
        The space $\mathbb{C}[S_{|\lambda|}]c_\lambda$ is called the Specht module $S_\lambda$.
    }

    Consider a vector space $V$ together with its general linear group $\text{GL}(V)$. For all $n\in\mathbb{N}$ there is an induced (diagonal) representation on $V^{\otimes n}$ by $\text{GL}(V)$. There is also an action by the permutation group $S_n$ that permutes the elements in a monomial $v_1\otimes\cdots\otimes v_n\in V^{\otimes n}$.
    \begin{theorem}[Schur-Weyl]\index{Schur-Weyl duality}\index{Schur functor}
        The above representation of $\text{\emph{GL}}(V)\times S_n$ can be decomposed as follows:
        \begin{gather}
            V^{\otimes n}\cong\bigoplus_{|\lambda|=n}V_\lambda\otimes S_\lambda V,
        \end{gather}
        where:
        \begin{itemize}
            \item the sum ranges over all partitions of $n$ or, equivalently, all Young diagrams with $n$ boxes,
            \item the $V_\lambda$ are Specht modules (and hence irreducible representations) of $S_n$, and
            \item the $S_\lambda V$ are (possibly zero) irreducible representations of $\text{\emph{GL}}(V)$ of the form $S_\lambda V \equiv\text{\emph{Hom}}_{S_n}(V_\lambda, V^{\otimes n})$.
        \end{itemize}
    \end{theorem}
    The spaces $V_\lambda$ can be interpreted as multiplicity spaces. The spaces $S_\lambda V$ can be rewritten more explicitly as $V_\lambda\otimes_{S_n} V^{\otimes n}$ ($\text{Hom}_{S_n}$ and $\otimes_{S_n}$ are adjoint functors). By using the explicit characterization of Specht modules given above, one can see that the spaces $S_\lambda V$ are of the form $V^{\otimes n}c_\lambda$.

    Because of the functoriality of the involved operations, one can also see that $S_\lambda$ is in fact a functor, the \textbf{Schur functor}.

    \begin{example}[Algebraic curvature tensor]
        The above definition of the spaces $S_\lambda V$ allow for an easy description in terms of Young diagrams (and tableaux). Consider for example the Riemann curvature tensor $R_{ijkl}$ from Chapter \ref{chapter:riemann}. This tensor has the following symmetries:
        \begin{itemize}
            \item $R_{ijkl} = -R_{jikl} = -R_{ijlk}$,
            \item $R_{ijkl} = R_{klij}$, and
            \item $R_{ijkl} + R_{iljk} + R_{iklj} = 0$.
        \end{itemize}
        By looking at the definition of the Young symmetrizer, one can see that these symmetries are exactly those of the irreducible component in $S_\lambda V$ associated to the partition $(2,2)$.

        ?? COMPLETE ??
    \end{example}

\section{Tensor operators}

    \newdef{Representation operators}{\index{representation!operator}\index{tensor!operator|see{representation operator}}
        An intertwiner $\rho:(\mathcal{R},V_0)\rightarrow\text{End}(V)$ from a $G$-representation on an auxiliary vector space $V_0$ to the space of linear maps on a $G$-vector space $V$ (equipped with the adjoint action).

        More explicitly, consider a set of operators $\{\hat{O}_i\}_{i\in I}\subset\text{End}(V)$ acting on a vector space $V$ equipped with a representation $\mathcal{R}$ of the group $G$. This collection defines a representation operator with respect to $G$ if there exists a matrix representation $R$ of $G$ such that the following equation holds:
        \begin{gather}
            \mathcal{R}(g)\hat{O}_i\mathcal{R}(g)^{-1} = \sum_j R(g)_{ij}\hat{O}_j.
        \end{gather}
    }
    \begin{example}[Tensor operators]
        Consider $G=\text{SO}(3)$ and $V_0=\mathcal{T}^r_s(\mathbb{R}^3)$. With this choice a set of operators that transform as tensors under rotations is obtained. By choosing $V_0=\mathbb{R}^3$ or $V_0=\mathcal{H}_l(\mathbb{R}^3)$, the space of spherical harmonics of degree $l$, one obtains the \textbf{vector} and \textbf{spherical operators}.
    \end{example}

    The following property is often used in quantum mechanics to quickly find forbidden transitions in atomic or molecular systems:
    \begin{property}[Selection rules]
        Let $G$ be a semisimple group and let $W_1,W_2$ be two inequivalent (finite-dimensional) irreducible unitary subrepresentations of the Hilbert space $\mathcal{H}$. Let $\rho$ be a representation operator (indexed by a vector space $V$). For all $v\in V,w_i\in W_i$ one has
        \begin{gather}
            \langle w_1|\rho(v)w_2\rangle = 0,
        \end{gather}
        unless $V\otimes W_2$ contains a subrepresentation equivalent to $W_1$.
    \end{property}

    \begin{theorem}[Wigner-Eckart]\index{Wigner-Eckart}
        Consider two irreducible $\mathrm{SU}(2)$-subrepresentations $W_j$ and $W_{j'}$ of some unitary representation $\mathcal{H}$, together with two degree-$q$ spherical tensors $\rho,\tilde{\rho}:V_0\rightarrow\mathrm{End}(\mathcal{H})$. If there exists at least one index $k\leq q$ and one pair of vectors $(v, v')\in W_j\times W_{j'}$ such that \[\langle v'|\rho_k v\rangle\neq 0,\] then for all indices $k\geq 0$ and pairs $(v,v')\in W_j,W_{j'}$ the following equality holds
        \begin{gather}
            \langle v'|\tilde{\rho}_k' v\rangle = C\langle v'|\rho_k v\rangle
        \end{gather}
        for some constant $C$ that only depends on $q, j$ and $j'$.
    \end{theorem}
    By noting that the Clebsch-Gordan coefficients are the components of the projection $W_q\otimes W_j\rightarrow W_{j'}$, which is itself an intertwiner, one can recast the Wigner-Eckart theorem as a statement about matrix elements:
    \begin{result}
        Consider an irreducible tensor operator $T_j^m$ (with respect to the rotation group). The matrix elements of this operator with respect to a symmetry-adapted basis (''angular momentum'' basis) decompose as a product of a Clebsch-Gordan coefficient and a factor that only depends on the eigenvalues of the Casimir operator:
        \begin{gather}
            \langle j',m'|R^{(q)}|j,m \rangle = \langle j'||R_k^{(q)}||j \rangle\langle j',m'|q,j;k,m \rangle.
        \end{gather}
        The factor $\langle j'||R^{(q)}||j \rangle$ is sometimes called the \textbf{reduced matrix element}.
    \end{result}
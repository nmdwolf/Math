\chapter{Representation Theory}

    References for this chapter are \cite{fultonharris, jeevanjee}.

\section{Group representations}

    \newdef{Representation}{\index{representation}
        A representation of a group $G$, acting on a vector space $V$, is a homomorphism $\rho: G\rightarrow\text{GL}(V)$ from $G$ itself to the automorphism group\footnote{See definition \ref{linalgebra:automorphism}.}\ of $V$. This is a specific case of a group action\footnote{See definition \ref{group:group_action}.}.
    }

    \begin{property}
        Because every linear map takes the zero vector to itself, a representation can never be free\footnote{See definition \ref{group:free_action}.}.
    \end{property}

    \newdef{Subrepresentation}{
        A subrepresentation of a representation $V$ is a subspace of $V$ invariant under the action of the group $G$.
    }

    \begin{example}[Permutation representation]
        Consider a vector space $V$ equipped with a basis $\{e_i\}_{i\in I}$ with $|I| = n$. Let $G = S^n$ be the symmetric group on $n$ symbols. Based on remark \ref{group:permutation_remark} we can consider the action of $G$ on the index set $I$. This representation is given by
        \begin{gather}
            \rho(g):\sum_{i\in I}v_ie_i\mapsto\sum_{i\in I}v_ie_{g\cdot i}.
        \end{gather}
    \end{example}

    \begin{example}
        Consider a representation $\rho$ on $V$. There exists a natural representation on the dual space $V^*$:
        \begin{gather}
            \rho^*(g) := \rho^T(g^{-1}): V^*\rightarrow V^*
        \end{gather}
        where $\rho^T$ is the transpose as defined in \ref{linalgebra:transpose}. This map satifies the following defining property:
        \begin{gather}
            \Big\langle\rho^*(g)(v^*), \rho(g)(v)\Big\rangle = \langle v^*, v\rangle
        \end{gather}
        where $\langle\cdot,\cdot\rangle$ is the natural pairing of $V$ and its dual.
    \end{example}

    \begin{example}\index{tensor!product}
        A group $\rho$ that acts on spaces $V, W$ also has a representation on the tensor product  $V\otimes W$ in the following way:
        \begin{gather}
            g(v\otimes w) := g(v)\otimes g(w).
        \end{gather}
    \end{example}

    \newdef{Intertwiner}{\index{intertwiner}
        If we look at the representations of $G$ as $G$-modules then the natural morphisms are the intertwining maps \ref{group:equivariant}.
    }

\section{Irreducible representations}

    \newdef{Irreducibility}{\index{irreducibility}
        A representation is said to be irreducible if there exist no proper non-zero subrepresentation.
    }

    \begin{example}[Standard representation]
        Consider the action of $\text{Sym}(n)$ on a vector space $V$. The line generated by $v_1+v_2+...+v_n$ is invariant under the permutation action of $\text{Sym}(n)$. It follows that the permutation representation (on finite-dimensional spaces) is never irreducible.

        The $(n-1)$-dimensional complementary subspace
        \begin{gather}
            W = \{a_1v_1 + a_2v_2 + \ldots + a_nv_n:a_1 + a_2 + \ldots + a_n = 0\}
        \end{gather}
        does form an irreducible representation when we restrict $\rho$ to $W$. It is called the standard representation of $S^n$.
    \end{example}

    \begin{theorem}[Schur's lemma]\index{Schur's lemma}\label{rep:schurs_lemma}
        Let $V, W$ be two finite-dimensional irreducible representations of a group $G$ and let $\varphi:V\rightarrow W$ be an intertwiner.
        \begin{itemize}
            \item Either $\varphi$ is an isomorphism or $\varphi = 0$.
            \item If $V = W$ then $\varphi$ is constant, i.e. $\varphi$ is a scalar multiple of the identity map $\mathbbm{1}_V$.
        \end{itemize}
    \end{theorem}

    \begin{property}
        If $W$ is a subrepresentation of $V$ then there exists an invariant complementary subspace $W'$ such that $V = W \oplus W'$.

        This space can be found as follows: Choose an arbitrary complement $U$ such that $V = W \oplus U$. From this we construct a projection map $\pi_0:V \rightarrow W$. Averaging over $G$ gives
        \begin{gather}
            \pi(v) := \sum_{g\in G}g\circ\pi_0(g^{-1}v)
        \end{gather}
        which is a $G$-linear map $V\rightarrow W$. On $W$ it is given by the multiplication of $W$ by $|G|$. Its kernel is then an invariant subspace of $V$ complementary to $W$.
    \end{property}
    \begin{theorem}[Maschke]\index{Maschke}
        Let $G$ be a finite group. A representation\footnote{The characteristic of the base field should not divide the order of $G$.} $V$ can be uniquely decomposed as
        \begin{gather}
            V = V_1^{\oplus a_1}\oplus\cdots\oplus V_k^{\oplus a_k}
        \end{gather}
        where all $V_k$'s are distinct irreducible representations.
    \end{theorem}

\section{Classification by Young tableaux}

    \newdef{Permutation module}{\index{permutation!module}
        Let $\lambda$ be a partition. The permutation module $M^\lambda$ is defined as the vector space generated by the Young tableaux of shape $\lambda$.
    }

    \newdef{Specht module}{\index{Specht!module}
        ?? COMPLETE ??
    }

\section{Tensor operators}

    \newdef{Representation operators}{\index{representation!operator}\index{tensor!operator|see{representation operator}}
        An intertwiner $\rho:(\mathcal{R}, V_0)\rightarrow\mathcal{L}(V)$ from a $G$-representation on an auxiliary vector space $V_0$ to the space of linear maps on a vector space $V$ (we equip the operator space with the adjoint action).

        More explicitly, consider a set of operators $\{\hat{O}_i\}_{i\in I}\subset\text{End}(V)$ acting on a vector space $V$ equipped with a representation $\mathcal{R}$ of the group $G$. This collection defines a representation operator with respect to $G$ if there exists a matrix representation $R$ of $G$ such that the following equation holds:
        \begin{gather}
            \mathcal{R}(g)\hat{O}_i\mathcal{R}(g)^{-1} = \sum_j R(g)_{ij}\hat{O}_j.
        \end{gather}
    }
    \begin{example}[Tensor operators]
        Let $G=\text{SO}(3)$ and $V_0=\mathcal{T}^r_s(\mathbb{R}^3)$. With this choice we obtain a set of operators which transform as tensors under rotations. By choosing $V_0=\mathbb{R}^3$ or $V_0=\mathcal{H}_l(\mathbb{R}^3)$ (space of spherical harmonics of degree $l$) one obtains the \textbf{vector} and \textbf{spherical operators}.
    \end{example}

    The following property is often used in quantum mechanics to quickly find forbidden transitions in atomic or molecular systems:
    \begin{property}[Selection rules]
        Let $G$ be a semisimple group. Let $W_1, W_2$ be two (finite-dimensional) inequivalent irreducible unitary subrepresentations of $\mathcal{H}$. Let $\rho$ be a representation operator. For all $v\in V, w_i\in W_i$ we have
        \begin{gather}
            \langle w_1|\rho(v)w_2\rangle = 0
        \end{gather}
        unless $V\otimes W_2$ contains a subrepresentation equivalent to $W_1$.
    \end{property}

    \begin{theorem}[Wigner-Eckart]\index{Wigner-Eckart}
        Consider two irreducible SU$(2)$-subrepresentations $W_j$ and $W_{j'}$ of some unitary representation $\mathcal{H}$ together with two degree $q$ spherical tensors $\rho, \tilde{\rho}:V_0\rightarrow\mathcal{L}(\mathcal{H})$. If there exists at least one index $k\leq q$ and one pair of vectors $(v, v')\in W_j\times W_{j'}$ such that \[\langle v'|\rho_k v\rangle\neq 0\] then for all indices $k\leq 0$ and vectors in $W_j, W_{j'}$ the following equality holds
        \begin{gather}
            \langle v'|\tilde{\rho}_k' v\rangle = C\langle v'|\rho_k v\rangle
        \end{gather}
        for some constant $C$ that only depends on $q, j$ and $j'$.
    \end{theorem}
    By noting that the Clebsch-Gordan coefficients are the components of the projection $W_q\otimes W_j\rightarrow W_{j'}$, which is itself an intertwiner, we can recast the Wigner-Eckart theorem as a statement about matrix elements:
    \begin{result}
        Consider an irreducible tensor operator $T_j^m$ (with respect to the rotation group). The matrix elements of this operator with respect to a symmetry adapted basis (''angular momentum'' basis) decompose as a product of a Clebsch-Gordan coefficient and a factor which does only depend on the eigenvalues of the Casimir element
        \begin{gather}
            \langle j', m'|R^{(q)}|j, m\rangle = \langle j'||R_k^{(q)}||j\rangle\langle j', m'|q,j;k,m\rangle.
        \end{gather}
        The factor $\langle j'||R^{(q)}||j\rangle$ is sometimes called the \textbf{reduced matrix element}.
    \end{result}
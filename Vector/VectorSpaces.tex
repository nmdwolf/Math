\section{Vector spaces}
	In this and coming sections all vector spaces can be finite- or infinite-dimensional. If necessary, the dimension will be specified.

	\newdef{K-vector space}{\label{linalgebra:vector_space}
		Let $K$ be a field. A $K$-vector space $V$ is a set equipped with two operations, vector addition $V\times V\rightarrow V$ and scalar multiplication $K\times V\rightarrow V$, that satisfy the following 8 axioms:
		\begin{enumerate}
			\item $V$ is an Abelian group under vector addition.
			\item $a(b\vec{v}) = (ab)\vec{v}$
			\item $1_K\vec{v} = \vec{v}$ where $1_K$ is the identity element of the field $K$
			\item Distributivity of scalar multiplication with respect to vector addition: $a(\vec{v} + \vec{w}) = a\vec{v} + a\vec{w}$
		\end{enumerate}
	}

\subsection{Linear independence}

	\newdef{Linear combination}{
		The vector $w$ is a linear combination of elements in the set $\{v_n\}$ if it can be written as:
		\begin{equation}
			\label{linalgebra:linear_combination}
			w = \sum_n\lambda_n v_n
		\end{equation}
		for some subset $\{\lambda_n\}$ of the field $K$.
	}
	\newdef{Linear independence}{
		A set finite $\{v_n\}_{n\leq N}$ is said to be linearly independent if the following relation holds:
		\begin{equation}
			\label{linalgebra:linear_independence}
			\sum_{n=0}^N\lambda_n v_n = 0 \iff \forall n:\lambda_n = 0
		\end{equation}
		A general set $\{w_i\}_{i \in I}$ is linearly independent if every finite subset of it is linearly independent.
	}

	\newdef{Span}{\index{span}
		A set of vectors $\{v_n\}$ is said to span $V$ if every vector $v \in V$ can be written as a linear combination of $\{v_n\}$.
	}
	
	\newdef{Frame}{\index{frame}
		A $k$-frame is an ordered set of $k$ linearly independent vectors.
	}
	\newdef{Stiefel manifold}{\index{Stiefel!manifold}
		Let $V$ be an inner product space over a field $K$ (real, complex or quaternionic numbers). The set of orthonormal $k$-frames can be embedded in $K^{n\times k}$. It becomes a compact embedded submanifold, called the Stiefel manifold of $k$-frames over $V$.
	}
	
\subsection{Bases}
	
	\newdef{Basis}{\index{basis}
		A set $\{v_n\}$ is said to be a basis of $V$ if $\{v_n\}$ is linearly independent and if $\{v_n\}$ spans $V$.
	}
	\begin{result}
		Every set $T$ that spans $V$ contains a basis of $V$.
	\end{result}
	
	\begin{remark}
		Here it is time for a little side note. In the previous definition we implicitly used the concept of a \textit{Hamel} basis, which is based on two conditions:
		\begin{itemize}
			\item The basis is linearly independent.
			\item Every element in the vector space can be written as a linear combination of a \underline{finite} subset of the basis.
		\end{itemize}
		Hence for finite-dimensional spaces we do not have to worry. In infinite-dimensional spaces however we have to keep this in mind. An alternative construction, which allows combinations of infinitely many elements is given by the \textit{Schauder basis}.\index{Schauder basis}
	\end{remark}
	
	We now continue by constructing a Hamel basis:
	\begin{construct}[Hamel basis]\index{Hamel basis}\label{linalgebra:hamel_basis}
		Let $V$ be a vector space over a field $K$. Consider the set of all linearly independent subsets of $V$. Under the relation of inclusion this set becomes a partially ordered set\footnote{See definition \ref{set:poset}.}. Zorn's lemma \ref{set:zorns_lemma} tells us that there exists at least one maximal linearly independent set.
		
		Now we will show that this maximal subset $S$ is also a generating set of $V$. Let us choose a vector $v\in V$ that is not already in $S$. From the maximality of $S$ it follows that $S\cup v$ is linearly dependent and hence there exists a finite sequence of numbers $(a^1, ..., a^n, b)$ in $K$ and a finite sequence of elements $(e_1, ..., e_n)$ in $S$ such that:
		\begin{equation}
			\sum_{i=0}^n a^ie_i + bv = 0
		\end{equation}
		where not all scalars are zero. This then implies that $b\neq0$ because else the set $\{e_i\}_{i\leq n}$ and hence $S$ would be linearly dependent. It follows that we can write $v$ as\footnote{It is this step that requires $R$ to be a division ring in property \ref{algebra:module_basis} because else we would not generally be able to divide by $b\in R$.}:
		\begin{equation}
			v = -\frac{1}{b}\sum_{i=0}^na^ie_i
		\end{equation}
		Because $v$ was randomly chosen we conclude that $S$ is a generating set for $V$. It is called a Hamel basis of $V$.
	\end{construct}
	\begin{remark*}
		This construction clearly assumes the ZFC axioms of set theory, only ZF does not suffice. It can even be shown that the existence of a Hamel basis for every vector space\footnotemark\ is equivalent to the axiom of choice (and thus to Zorn's lemma ).
		\footnotetext{This would turn a vector space into a free object int the category of vector spaces.}
	\end{remark*}
    
\subsection{Subspaces}

	\newdef{Subspace}{\label{linalgebra:subspace}
		Let $V$ be a K-vector space. A subset $W$ of $V$ is a subspace if $W$ itself is a K-vector space under the operations of V. Alternatively we can write this as:
		\begin{equation}
			W \leq V\iff \forall w_1, w_2 \in W, \forall \lambda, \mu \in K:\lambda w_1 + \mu w_2 \in W
		\end{equation}
	}

	\newdef{Grassmannian}{\index{Grassmannian}\label{linalgebra:grassmannian}
		Let $V$ be a $K$-vector space. The set of all subspaces of dimension $k$ is denoted by $\text{Gr}(k, V)$.
	}
	\begin{property}\label{linalgebra:grassmannian_construction}
		GL$(V)$ acts transitively\footnote{See definition \ref{group:transitive}} on all $k$-dimensional subspaces of $V$. From property \ref{group:transitive_action_property} it follows that the coset space GL$(V)/H_W$ for any stabilizer $H_W$ of some $W\in \text{Gr}(k, V)$ is isomorphic (as a set) to $\text{Gr}(k, V)$.
	\end{property}
	
	\newdef{Flag}{\index{flag}\index{signature}
		Let $V$ be a finite-dimensional vector space. A sequence of proper subspaces $V_1\leq ... V_n$ is called a flag of $V$. The sequence $(\dim V_1, ..., \dim V_n)$ is called the \textbf{signature} of the flag. If for all $i$, $\dim V_i = i$ then the flag is called \textbf{complete}.
	}
	
	\newdef{Flag variety}{
		The set of all flags of a given signature over a vector space $V$ forms a homogeneous space, called the (generalized) flag variety (of that signature). If the underlying field is the field of real (or complex) numbers then the flag variety is a smooth (or complex) manifold, called the \textbf{flag manifold}.
	}

\subsection{Sum and direct sum}

    	\newdef{Sum}{\label{linalgebra:sum}
    		\nomenclature[O]{$X+Y$}{Sum of the vector spaces $X$ and $Y$.}
    		Let $V$ be a K-vector space. Let $W_1, W_2,..., W_k$ be subspaces of $V$. The 'sum' of the subspaces $W_1,..., W_k$ is defined as follows:
        	\begin{equation}
				W_1+...+W_k:=\left\{\sum_{i=1}^kw_i : w_i\in W_i\right\}
			\end{equation}
        }
        \newdef{Direct sum}{\index{direct!sum}\label{linalgebra:direct_sum}
        	\nomenclature[O]{$X\oplus Y$}{Direct sum of the vector spaces $X$ and $Y$.}
        	If every element $v$ of the sum as defined in definition \ref{linalgebra:sum} can be written as a unique linear combination, then the sum is called a direct sum.
		}
        \newnot{Direct sum}{
        	\label{linalgebra:notation:direct_sum}
            \begin{equation*}
                W_1\oplus...\oplus W_k = \bigoplus_{i=1}^kW_i
            \end{equation*}
		}
        
        \begin{theorem}
			\label{linalgebra:theorem:direct_sum}
            Let $V$ be a K-vector space. Let $W, W_1, W_2$ be three subspaces of $V$ such that $W=W_1\oplus W_2$. We have the following properties:
            \begin{itemize}
				\item If $\mathcal{B}_1$ is a basis of $W_1$ and if $\mathcal{B}_2$ is a basis of $W_2$, $\mathcal{B}_1\cup\mathcal{B}_2$ is a basis of $W$.
                \item $\dim(W) = \dim(W_1) + \dim(W_2)$
			\end{itemize}
		\end{theorem}
        \begin{theorem}
			\label{linalgebra:theorem:sum}
            Let $V$ be a finite-dimensional K-vector space. Let $W_1, W_2$ be two subspaces of $V$. Then the following relation holds:
            \begin{equation}
				\dim(W_1 + W_2) = \dim(W_1) + \dim(W_2) - \dim(W_1\cap W_2)
			\end{equation}
            The second item in previous property is a direct consequence of this property.
		\end{theorem}

        \newdef{Complement}{\index{complement}
        	\label{linalgebra:complement}Let $V$ be a K-vector space. Let $W$ be a subspace of $V$. A subspace $W'$ of $V$ is called a complement of $W$ if $V = W\oplus W'$.
		}
        \begin{theorem}
			\label{linalgebra:theorem:complement}
            Let $V$ be a K-vector space. Let $U,W$ be two subspaces of $V$. If $\ V = U+W$, then there exists a subspace $Y\leq U$ such that $V = W\oplus Y$. Furthermore every subset $W$ of $V$ has a complement in $V$.
		\end{theorem}

    
\subsection{Algebras}

        \newdef{Algebra}{\index{algebra}\label{linalgebra:algebra}
            Let $V$ be a K-vector space. Let V be equipped with the binary operation $\star: V\times V\rightarrow V$. $(V,\star)$ is called an algebra over $K$ if it satisfies the following conditions\footnotemark:
            \begin{enumerate}
                \item Right distributivity: $(\vec{x} + \vec{y})\star\vec{z} = \vec{x}\star\vec{z} + \vec{y}\star\vec{z}$
                \item Left distributivity: $\vec{x}\star(\vec{y} + \vec{z}) = \vec{x}\star\vec{y} + \vec{x}\star\vec{z}$
                \item Compatibility with scalars: $(a\vec{x})\star(b\vec{y}) = (ab)(\vec{x}\star\vec{y}$)
            \end{enumerate}
            These conditions turn the binary operation into a bilinear operation.
            \footnotetext{These conditions imply that the binary operation is a bilinear map.}
        }
        \newdef{Unital algebra}{
        	An algebra $V$ is said to be unital if it contains an identity element with respect to the bilinear map $\star$.
        }

\subsection{Coalgebras}

	\newdef{Coalgebra}{\index{coalgebra}
		A vector space $C$ together with two linear maps $\Delta$ and $\varepsilon$, called the \textbf{comultiplication} and \textbf{counit}, is called a coalgebra if it satisfies following two axioms:
		\begin{enumerate}
			\item $(\mathbbm{1}_C\otimes\Delta)\circ\Delta = (\Delta\otimes\mathbbm{1}_C)\circ\Delta$
			\item $(\mathbbm{1}_C\otimes\varepsilon)\circ\Delta = (\varepsilon\otimes\mathbbm{1}_C)\circ\Delta = \mathbbm{1}_C$
		\end{enumerate}
	}
	\begin{example}
		The simplest example is given by the vector space $V$ with basis $\{e_i\}_{i\in I}$ where the comultiplication and counit are defined as follows:
		\begin{equation}
			\Delta(e_i) = e_i\otimes e_i
		\end{equation}
		and
		\begin{equation}
			\varepsilon(e_i) = 1
		\end{equation}
		By linearity these maps can be extended to all of $V$. Important cases are the tensor algebra and exterior algebra over a vector space. (See definitions \ref{tensor:tensor_algebra} and \ref{tensor:exterior_algebra}.)
	\end{example}
	
	\newdef{Group-like element}{
		An element $c$ in a coalgebra $(C, \Delta, \varepsilon)$ that satisfies $\Delta(c) = c\otimes c$ and $\varepsilon(c) = 1$.
	}

\subsection{Graded vector space}
	Similar to definition \ref{group:graded_ring} we can define the following:
	\newdef{Graded vector space}{\index{degree}\index{graded!vector space}
		Let $V_n$ be a vector space for all $n\in\mathbb{N}$. The vector space
		\begin{equation}
			\label{linalgebra:graded_vector_space}
			V = \bigoplus_{n\in\mathbb{N}} V_n
		\end{equation}
		is called a graded vector space.
	}
	
	\newdef{Graded algebra}{\index{graded!algebra}
		Let $V$ be a graded vector space with the additional structure of an algebra given by the multiplication $\star$. Then $V$ is a graded algebra if $\star$ maps $V^k\times V^l$ to $V^{k+l}$.
	}
	
	\begin{example}[Superalgebra]\index{superalgebra}\label{linalgebra:superalgebra}
		Let $A$ be a $\mathbb{Z}_2$-graded algebra, i.e.:
		\begin{equation}
			A = A_0\oplus A_1
		\end{equation}
		such that for all $i, j \mod 2$:
		\begin{equation}
			A_i\star A_j \subseteq A_{i+j}
		\end{equation}
	\end{example}
        
\section[Linear maps]{Linear maps\footnote{Other names are \textbf{linear mapping} and \textbf{linear transformation}.}}
	
	\newdef{Zero map}{\label{linalgebra:zero_map}Let $f:A\rightarrow B$ be a (linear) map. The map $f$ is called a zero map if:
	    	\begin{equation}
	        	\forall a\in A:f(a) = 0
		\end{equation}
	}

	\newdef{Restriction}{
		Let $f:A\rightarrow B$ be a (linear) map. Let $C\subset A$. The (linear) map $f|_C:C\rightarrow B:c\rightarrow f(c)$ is called the restriction of $f$ to $C$.
	}

	\newdef{Injective}{\label{linalgebra:injective}A map $f:A\rightarrow B$ is called injective if the following condition is satisfied:
	    	\begin{equation}
		        \forall a, a'\in A:f(a)=f(a')\implies a=a'
		\end{equation}
	}
	\newnot{Injective map}{\[f:A\hookrightarrow B\]}
	\newdef{Surjective}{\label{linalgebra:surjective}A map $f:A\rightarrow B$ is called surjective if the following condition is satisfied:
	    	\begin{equation}
		        \forall b\in B, \exists a\in A:f(a) = b
		\end{equation}
	}
	\newnot{Surjective map}{\[f:A\twoheadrightarrow B\]}
	\newdef{Bijective}{A map is called bijective if it is both injective and surjective.}
	\newnot{Bijective map}{\[f:A\xrightarrow{\sim} B\]}
	\newdef{Isomorphism}{\index{isomorphism}A linear bijective map $f$ between two K-vector spaces is called an isomorphism.}
	\newnot{Isomorphic}{If two K-vector spaces $V, W$ are isomorphic we denote it as following:\[V\cong W\]}

	\newdef{Automorphism}{\index{automorphism}\label{linalgebra:automorphism}
		\nomenclature[S]{$\text{Aut}(V)$}{The set of automorphisms (invertible endomorphisms) on a set $V$.}
	    	An isomorphism from V to V is called an automorphism. The set of all automorphisms  on $V$, which is in fact a group, is denoted by $\text{Aut}(V)$.
	}

	\newdef{$C^r$-diffeomorphism}{\index{diffeomorphism}\label{linalgebra:diffeomorphism}
	        An isomorphism of class $C^r(K)$ with an inverse that is also of class $C^r(K)$ is called a $C^r$-diffeomorphism.
	}
    
	\newdef{General linear group\footnotemark}{\index{general linear group}
		\nomenclature[S]{$GL(V)$}{General linear group: group of all automorphisms on a vector space $V$.}
	    	The set of all automorphisms $f:V\rightarrow V$ is called the general linear group $GL_K(V)$ of $GL(V)$.
	    	\footnotetext{This group is isomorphic to the general linear group of invertable matrices, hence the similar name and notation. (See definition \ref{linalgebra:GL_matrices})}
	}

	\newdef{Rank}{\index{rank}\label{linalgebra:image_rank}
		The dimension of the image of a linear map is called the rank.
	}
	\newdef{Kernel}{\index{kernel}\label{linalgebra:kernel}
	        The kernel of a linear map $f: V \rightarrow W$ is the following subset of $V$:
        	\begin{equation}
        		\text{ker}(f) = \{v\in V\ |\ f(v) = 0\}
	        \end{equation}
	}
	\newdef{Nullity}{\label{linalgebra:kernel_nullity}
		The dimension of the kernel is called the nullity.
	}
    
	\begin{theorem}
	    	A linear map $f:V\rightarrow W$ is injective if and only if $\text{ker}(f) = \{0\}$.
	\end{theorem}
	\begin{property}\label{linalgebra:theorem:restriction_kernel_image}
	        Let $f:V\rightarrow W$ be a linear map. Let $U\leq V$. We have the following two properties of the restriction $f|_U$ of $f$ to $U$:
        	\begin{itemize}
			\item $\text{ker}\left(f|_U\right) = \text{ker}(f)\cap U$
        		\item $\text{im}\left(f|_U\right) \leq \text{im}(f)$
		\end{itemize}
	\end{property}
    
\subsection{Linear operator}

    	\begin{definition}[Linear operator]\index{endomorphism}\index{operator}
	    	A linear automorphism $f: V \rightarrow V$ is called a linear operator. It is also more generally known as an \textbf{endomorphism} on $V$.
	\end{definition}
    	\begin{property}
		Let $\lambda, \mu \in K$. An operator $f: V \rightarrow V$ is called linear if it satisfies the following condition:
	        \begin{equation}
			\label{linalgebra:operators:linearity}
        	        f(\lambda v_1 + \mu v_2) = \lambda f(v_1) + \mu f(v_2)
		\end{equation}
	\end{property}
        
        \begin{theorem}
		Let V be finite-dimensional K-vector space. Let $f:V\rightarrow V$ be a linear operator. The following statements are equivalent:
        	\begin{itemize}
            		\item f is injective
			\item f is surjective
                	\item f is bijective
		\end{itemize}
	\end{theorem}
        
\subsection{Dimension}

    	\newdef{Dimension}{\index{dimension}
        	Let $V$ be a finite-dimensional K-vector space. Let $\{v_n\}$ be a basis for $V$ that contains $n$ elements. We then define the dimension of $V$ as following:
        	\begin{equation}
			\label{linalgebra:dimension}
	                \boxed{\dim(V) = n}
		\end{equation}
        }
        \begin{property}
		Let $V$ be a finite-dimensional K-vector space. Every basis of $V$ has the same number of elements.\footnote{This theorem can be generalized to infinite-dimensional spaces by stating that all bases have the same \textit{cardinality}.}
	\end{property}

        \begin{theorem}[Dimension theorem\footnotemark]\index{rank-nullity theorem}
		\footnotetext{Also called the \textbf{rank-nullity theorem}.}
		Let $f: V \rightarrow W$ be a linear map.
	        \begin{equation}
	                \label{linalgebra:dimension_theorem}
	                \dim(\text{\upshape im}(f)) + \dim(\text{\upshape ker}(f)) = \dim(\text{\upshape V})
	        \end{equation}
        \end{theorem}

        \begin{theorem}\label{linalgebra:isomorphic_dimension}
		Two K-vector spaces are isomorphic if and only if they have the same dimension.
	\end{theorem}
